\documentclass[a4paper,12pt,twoside]{book}
\usepackage{fontspec}
\usepackage{xunicode}
\usepackage[german]{babel}


\usepackage{xcolor}

\newcommand{\match}[1]{\textcolor{red}{\textbf{#1}}}





\usepackage{fancyhdr}
\usepackage{etoolbox} % For defining robust commands

\pagestyle{fancy}

\fancyhead[LE]{\thepage}
\fancyhead[RE]{\leftmark}

\fancyhead[LO]{\rightmark}
\fancyhead[RO]{\thepage}

% Define a new command for unnumbered chapters
\newcommand{\unnumberedchapter}[1]{
	\chapter*{#1}
	\addcontentsline{toc}{chapter}{#1}
	\markboth{#1}{#1}
}

% Similarly, for sections
\newcommand{\unnumberedsection}[1]{
	\section*{#1}
	\addcontentsline{toc}{section}{#1}
	\markright{#1}
}

\usepackage[hidelinks]{hyperref} %
\usepackage[numbered]{bookmark}%va avec hyperref; marche mieux pour les signets. l'option numbered: les signets dans le pdf sont numérotés

\author{Eglantine Gaglione - M1 HN}
\title{MS : vocabulaire thématique}

\begin{document}
	
	\maketitle
	
	\tableofcontents
	
	\unnumberedchapter{Agriculture} 
	\unnumberedsection{Anbau (1)} 
	\subsection*{tg477.2.4} 
	\textbf{Source : }Die Metaphysik der Sitten/Zweiter Teil. Metaphysische Anfangsgründe der Tugendlehre/I. Ethische Elementarlehre/I. Teil. Von den Pflichten gegen sich selbst überhaupt/2. Buch: Die Pflichten gegen sich selbst/Erster Abschnitt. Von der Pflicht gegen sich selbst in Entwickelung und Vermehrung seiner Naturvollkommenheit, d.i. in pragmatischer Absicht\\  
	
	\noindent\textbf{Paragraphe : }Der \match{Anbau} (cultura) seiner Naturkräfte (Geistes-, Seelen- und Leibeskräfte), als Mittel zu allerlei möglichen Zwecken ist Pflicht des Menschen gegen sich selbst. – Der Mensch ist es sich selbst (als einem Vernunftwesen) schuldig, die Naturanlage und Vermögen, von denen seine Vernunft dereinst Gebrauch machen kann, nicht unbenutzt und gleichsam rosten zu lassen, sondern, gesetzt daß er auch mit dem angebornen Maß seines Vermögens für die natürlichen Bedürfnisse zufrieden sein könne, so muß ihm doch seine Vernunft dieses Zufriedensein, mit dem geringen Maß seiner Vermögen, erst durch Grundsätze anweisen, weil er, als ein Wesen, das der Zwecke (sich Gegenstände zum Zweck zu machen) fähig ist, den Gebrauch seiner Kräfte nicht bloß dem Instinkt der Natur, sondern der Freiheit, mit der er dieses Maß bestimmt, zu verdanken haben muß. Es ist also nicht Rücksicht auf den Vorteil, den die Kultur seines Vermögens (zu allerlei Zwecken) verschaffen kann; denn dieser würde vielleicht (nach Rousseauschen Grundsätzen) für die Rohigkeit des Naturbedürfnisses vorteilhaft ausfallen: sondern es ist Gebot der moralisch-praktischen Vernunft und Pflicht des Menschen gegen sich selbst, seine  Vermögen (unter denselben eins mehr als das andere, nach Verschiedenheit seiner Zwecke) anzubauen, und in pragmatischer Rücksicht ein dem Zweck seines Daseins angemessener Mensch zu sein. 
	
	\unnumberedsection{Boden (13)} 
	\subsection*{tg433.2.31} 
	\textbf{Source : }Die Metaphysik der Sitten/Erster Teil. Metaphysische Anfangsgründe der Rechtslehre/1. Teil. Das Privatrecht vom äußeren Mein und Dein überhaupt/1. Hauptstück\\  
	
	\noindent\textbf{Paragraphe : }Wenn auch gleich ein \match{Boden} als frei, d.i. zu jedermanns Gebrauch offen angesehen, oder dafür erklärt würde, so kann man doch nicht sagen, daß er es von Natur und ursprünglich, vor allem rechtlichem Akt, frei sei, denn auch das wäre ein Verhältnis zu Sachen, nämlich dem Boden, der jedermann seinen Besitz verweigerte, sondern, weil diese Freiheit des Bodens ein Verbot für jedermann sein würde, sich desselben zu bedienen; wozu ein gemeinsamer Besitz desselben erfordert wird, der ohne Vertrag nicht statt finden kann. Ein Boden aber, der nur durch diesen frei sein kann, muß wirklich im Besitze aller derer (zusammen verbundenen) sein, die sich wechselseitig den Gebrauch desselben untersagen, oder ihn suspendieren. 
	
	\subsection*{tg435.2.26} 
	\textbf{Source : }Die Metaphysik der Sitten/Erster Teil. Metaphysische Anfangsgründe der Rechtslehre/1. Teil. Das Privatrecht vom äußeren Mein und Dein überhaupt/2. Hauptstück. Von der Art, etwas Äußeres zu erwerben/1. Abschnitt. Vom Sachrecht\\  
	
	\noindent\textbf{Paragraphe : }Es ist die Frage: wie weit erstreckt sich die Befugnis der Besitznehmung eines Bodens? So weit, als das Vermögen, ihn in seiner Gewalt zu haben, d.i. als der, so ihn sich zueignen will, ihn verteidigen kann, gleich als ob der \match{Boden} spräche: wenn ihr mich nicht beschützen könnt, so könnt ihr mir auch nicht gebieten. Darnach müßte also auch der Streit über das freie oder verschlossene Meer entschieden werden; z.B. innerhalb der Weite, wohin die Kanonen reichen, darf niemand an der Küste eines Landes, das schon einem gewissen  Staat zugehört, fischen, Bernstein aus dem Grunde der See holen, u. dergl. – Ferner: ist die Bearbeitung des Bodens (Bebauung, Beackerung, Entwässerung u. dergl.) zur Erwerbung desselben notwendig? Nein! denn, da diese Formen (der Spezifizierung) nur Akzidenzen sind, so machen sie kein Objekt eines unmittelbaren Besitzes aus, und können zu dem des Subjekts nur gehören, so fern die Substanz vorher als das Seine desselben anerkannt ist. Die Bearbeitung ist, wenn es auf die Frage von der ersten Erwerbung ankommt, nichts weiter als ein äußeres Zeichen der Besitznehmung, welches man durch viele andere, die weniger Mühe kosten, ersetzen kann. – Ferner: darf man wohl jemanden in dem Akt seiner Besitznehmung hindern, so daß keiner von beiden des Rechts der Priorität teilhaftig werde, und so der Boden immer als keinem angehörig frei bleibe? Gänzlich kann diese Hinderung nicht statt finden, weil der andere, um dieses tun zu können, sich doch auch selbst auf irgend einem benachbarten Boden befinden muß, wo er also selbst behindert werden kann zu sein, mithin eine absolute Verhinderung ein Widerspruch wäre; aber respektiv auf einen gewissen (zwischenliegenden) Boden, diesen, als neutral, zur Scheidung zweier Benachbarten unbenutzt liegen zu lassen, würde doch mit dem Rechte der Bemächtigung zusammen bestehen; aber alsdann gehört wirklich dieser Boden beiden gemeinschaftlich, und ist nicht herrenlos (res nullius), eben darum, weil er von beiden dazu gebraucht wird, um sie von einander zu scheiden. – Ferner: kann man auf einem Boden, davon kein Teil das Seine von jemanden ist, doch eine Sache als die seine haben? Ja, wie in der Mongolei jeder sein Gepäcke, was er hat, liegen lassen, oder sein Pferd, was ihm entlaufen ist, als das Seine in seinen Besitz bringen kann, weil der ganze Boden dem Volk, der Gebrauch desselben also jedem einzelnen zusteht; daß aber jemand eine bewegliche Sache auf dem Boden eines anderen als das Seine haben kann, ist zwar möglich, aber nur durch Vertrag. – Endlich ist die Frage: können zwei benachbarte  Völker (oder Familien) einander widerstehen, eine gewisse Art des Gebrauchs eines Bodens anzunehmen, z.B. die Jagdvölker dem Hirtenvolk, oder den Ackerleuten, oder diese den Pflanzern, u. dergl.? Allerdings; denn die Art, wie sie sich auf dem Erdboden überhaupt ansässig machen wollen, ist, wenn sie sich innerhalb ihrer Grenzen halten, eine Sache des bloßen Beliebens (res merae facultatis). 
	
	\subsection*{tg435.2.31} 
	\textbf{Source : }Die Metaphysik der Sitten/Erster Teil. Metaphysische Anfangsgründe der Rechtslehre/1. Teil. Das Privatrecht vom äußeren Mein und Dein überhaupt/2. Hauptstück. Von der Art, etwas Äußeres zu erwerben/1. Abschnitt. Vom Sachrecht\\  
	
	\noindent\textbf{Paragraphe : }Alle Menschen sind ursprünglich in einem Gesamt-Besitz des Bodens der ganzen Erde (communio fundi originaria), mit dem ihnen von Natur zustehenden Willen (eines jeden), denselben zu gebrauchen (lex iusti), der, wegen der natürlich unvermeidlichen Entgegensetzung der Willkür des einen gegen die des anderen, allen Gebrauch desselben aufheben würde, wenn nicht jener zugleich das Gesetz für diese enthielte, nach welchem einem jeden ein besonderer Besitz auf dem gemeinsamen \match{Boden} bestimmt werden kann (lex iuridica). Aber das austeilende Gesetz des Mein und Dein eines jeden am Boden kann, nach dem Axiom der äußeren Freiheit, nicht anders als aus einem ursprünglich und a priori vereinigten Willen (der zu dieser Vereinigung keinen rechtlichen Akt voraussetzt), mithin nur im bürgerlichen Zustande, hervorgehen (lex iustitiae distributivae), der allein, was recht, was rechtlich und was Rechtens ist, bestimmt. – In diesem Zustand aber, d.i. vor Gründung und doch in Absicht auf denselben, d.i. provisorisch, nach dem Gesetz der äußeren Erwerbung zu verfahren, ist Pflicht, folglich auch rechtliches Vermögen des Willens, jedermann zu verbinden, den Akt der Besitznehmung und Zueignung, ob er gleich nur einseitig ist, als gültig anzuerkennen; mithin ist eine provisorische Erwerbung des Bodens, mit allen ihren rechtlichen Folgen, möglich. 
	
	\subsection*{tg435.2.38} 
	\textbf{Source : }Die Metaphysik der Sitten/Erster Teil. Metaphysische Anfangsgründe der Rechtslehre/1. Teil. Das Privatrecht vom äußeren Mein und Dein überhaupt/2. Hauptstück. Von der Art, etwas Äußeres zu erwerben/1. Abschnitt. Vom Sachrecht\\  
	
	\noindent\textbf{Paragraphe : }Was die Körper auf einem \match{Boden} betritt, der schon der meinige ist, so gehören sie, wenn sie sonst keines anderen sind, mir zu, ohne daß ich zu diesem Zweck eines besonderen rechtlichen Akts bedürfte (nicht facto sondern lege); nämlich, weil sie als der Substanz inhärierende Akzidenzen betrachtet werden können (iure rei  meae), wozu auch alles gehört, was mit meiner Sache so verbunden ist, daß ein anderer sie von dem Meinen nicht trennen kann, ohne dieses selbst zu verändern (z.B. Vergoldung, Mischung eines mir zugehörigen Stoffes mit andern Materien, Anspülung oder auch Veränderung des anstoßenden Strombettes, und dadurch geschehende Erweiterung meines Bodens, u.s.w.). Ob aber der erwerbliche Boden sich noch weiter als das Land, nämlich auch auf eine Strecke des Seegrundes hin aus (das Recht, noch an meinen Ufern zu fischen, oder Bernstein herauszubringen, u. dergl.), sich ausdehnen lasse, muß nach ebendenselben Grundsätzen beurteilt werden. So weit ich aus meinem Sitze mechanisches Vermögen habe, meinen Boden gegen den Eingriff anderer zu sichern (z.B. so weit die Kanonen vom Ufer abreichen), gehört zu meinem Besitz und das Meer ist bis dahin geschlossen (mare clausum). Da aber auf dem weiten Meere selbst kein Sitz möglich ist, so kann der Besitz auch nicht bis dahin ausgedehnt werden und offene See ist frei (mare liberum). Das Stranden aber, es sei der Menschen, oder der ihnen zugehörigen Sachen, kann, als unvorsätzlich, von dem Strandeigentümer nicht zum Erwerbrecht gezählt werden; weil es nicht Läsion (ja überhaupt kein Faktum) ist, und die Sache, die auf einen Boden geraten ist, der doch irgend einem angehört, nicht als res nullius behandelt werden kann. Ein Fluß dagegen kann, so weit der Besitz seines Ufers reicht, so gut wie ein jeder Landboden, unter obbenannten Einschränkungen ursprünglich von dem erworben werden, der im Besitz beider Ufer ist. 
	
	\subsection*{tg435.2.8} 
	\textbf{Source : }Die Metaphysik der Sitten/Erster Teil. Metaphysische Anfangsgründe der Rechtslehre/1. Teil. Das Privatrecht vom äußeren Mein und Dein überhaupt/2. Hauptstück. Von der Art, etwas Äußeres zu erwerben/1. Abschnitt. Vom Sachrecht\\  
	
	\noindent\textbf{Paragraphe : }Der \match{Boden} (unter welchem alles bewohnbare Land verstanden wird) ist, in Ansehung alles Beweglichen auf demselben, als Substanz, die Existenz des letzteren aber nur als Inhärenz zu betrachten und so, wie Im theoretischen Sinne die Akzidenzen nicht außerhalb der Substanz existieren können, so kann im praktischen das Bewegliche auf dem Boden nicht das Seine von jemanden sein, wenn dieser nicht vorher als im rechtlichen Besitz desselben befindlich (als das Seine desselben) angenommen wird. 
	
	\subsection*{tg435.2.9} 
	\textbf{Source : }Die Metaphysik der Sitten/Erster Teil. Metaphysische Anfangsgründe der Rechtslehre/1. Teil. Das Privatrecht vom äußeren Mein und Dein überhaupt/2. Hauptstück. Von der Art, etwas Äußeres zu erwerben/1. Abschnitt. Vom Sachrecht\\  
	
	\noindent\textbf{Paragraphe : }Denn setzet, der \match{Boden} gehöre niemanden an: so werde ich jede bewegliche Sache, die sich auf ihm befindet, aus ihrem Platze stoßen können, um ihn selbst einzunehmen, bis sie sich gänzlich verliert, ohne daß der Freiheit irgend eines anderen, der jetzt gerade nicht Inhaber desselben ist, dadurch Abbruch geschieht; alles aber, was zerstört werden kann, ein Baum, Haus u.s.w. ist (wenigstens der Materie nach) beweglich, und wenn man die Sache, die ohne Zerstörung ihrer Form nicht bewegt werden kann, ein Immobile nennt, so wird das Mein und Dein an jener nicht von der Substanz, sondern dem ihr Anhängenden verstanden, welches nicht die Sache selbst ist. 
	
	\subsection*{tg441.2.49} 
	\textbf{Source : }Die Metaphysik der Sitten/Erster Teil. Metaphysische Anfangsgründe der Rechtslehre/2. Teil. Das öffentliche Recht/1. Abschnitt. Das Staatsrecht\\  
	
	\noindent\textbf{Paragraphe : }Kann der Beherrscher als Obereigentümer (des Bodens), oder muß er nur als Oberbefehlshaber in Ansehung des Volks durch Gesetze betrachtet werden? Da der \match{Boden} die oberste Bedingung ist, unter der allein es möglich ist, äußere Sachen als das Seine zu haben, deren möglicher Besitz und Gebrauch das erste erwerbliche Recht ausmacht, so wird von dem Souverän, als Landesherren, besser als Obereigentümer (dominus territorii) alles solche Recht abgeleitet wer den müssen. Das Volk, als die Menge der Untertanen, gehört ihm auch zu (es ist sein Volk), aber nicht ihm, als Eigentümer (nach dem dinglichen), sondern als Oberbefehlshaber (nach dem persönlichen Recht). – Dieses Obereigentum ist aber nur eine Idee des bürgerlichen Vereins, um die notwendige Vereinigung des Privateigentums aller im Volk unter einem öffentlichen allgemeinen Besitzer, zu Bestimmung des besonderen Eigentums, nicht nach Grundsätzen der Aggregation (die von den Teilen zum Ganzen empirisch fortschreitet), sondern dem notwendigen formalen Prinzip der Einteilung (Division des Bodens) nach Rechtsbegriffen vorstellig zu machen. Nach diesen kann der Obereigentümer kein Privateigentum an irgend einem Boden haben (denn sonst machte er sich zu einer Privatperson), sondern dieses gehört nur dem Volk (und zwar nicht kollektiv- sondern distributiv genommen) zu; wovon doch ein nomadisch-beherrschtes Volk auszunehmen ist, als in welchem gar kein Privateigentum des Bodens statt findet. – Der Oberbefehlshaber kann also keine Domänen, d.i. Ländereien, zu seiner Privatbenutzung (zu Unterhaltung des Hofes) haben. Denn, weil es alsdenn auf sein eigen Gutbefinden ankäme, wie weit sie ausgebreitet sein sollten, so würde der Staat Gefahr laufen, alles Eigentum des Bodens in den Händen der Regierung zu sehen, und alle Untertanen als grunduntertänig (glebae adscripti)und Besitzer von dem, was immer nur Eigentum eines anderen ist, folglich aller Freiheit beraubt (servi) anzusehen. – Von einem Landesherrn kann man sagen: er besitzt nichts (zu eigen), außer sich selbst; denn, wenn er neben einem anderen im Staat etwas zu eigen hätte, so würde mit diesem ein Streit möglich sein, zu dessen Schlichtung kein Richter wäre. Aber man kann auch sagen: er besitzt alles; weil er das Befehlshaberrecht über das Volk hat (jedem das Seine zu Teil kommen zu lassen), dem alle äußere Sachen (divisim) zugehören. 
	
	\subsection*{tg441.2.50} 
	\textbf{Source : }Die Metaphysik der Sitten/Erster Teil. Metaphysische Anfangsgründe der Rechtslehre/2. Teil. Das öffentliche Recht/1. Abschnitt. Das Staatsrecht\\  
	
	\noindent\textbf{Paragraphe : }Hieraus folgt: daß es auch keine Korporation im Staat, keinen Stand und Orden, geben könne, der als Eigentümer den \match{Boden} zur alleinigen Benutzung den folgenden Generationen (ins Unendliche) nach gewissen Statuten überliefern könne. Der Staat kann sie zu aller Zeit aufheben, nur unter der Bedingung, die Überlebenden zu entschädigen. Der Ritterorden (als Korporation, oder auch bloß Rang einzelner, vorzüglich beehrter, Personen); der Orden der Geistlichkeit, die Kirche genannt, können nie durch diese Vorrechte, womit sie begünstigt worden, ein auf Nachfolger übertragbares Eigentum am Boden, sondern nur die einstweilige Benutzung desselben erwerben. Die Komtureien auf einer, die Kirchengüter auf der anderen Seite können, wenn die öffentliche Meinung wegen der Mittel, durch die Kriegsehre
	den Staat wider die Lauigkeit in Verteidigung desselben zu schützen, oder die Menschen in demselben durch Seelmessen, Gebete und eine Menge zu bestellender Seelsorger, um sie vor dem ewigen Feuer zu bewahren, anzutreiben, aufgehört hat, ohne Bedenken (doch unter der vorgenannten Bedingung) aufgehoben werden. Die, so hier in die Reform fallen, können nicht klagen, daß ihnen ihr Eigentum genommen werde; denn der Grund ihres bisherigen Besitzes lag nur in der Volksmeinung, und mußte auch, so lange diese fortwährte, gelten. So bald diese aber erlosch, und zwar auch nur in dem Urteil derjenigen, welche auf Leitung desselben durch ihr Verdienst den größten Anspruch haben, so mußte, gleichsam als durch eine Appellation desselben an den Staat (a rege male informato ad regem melius informandum), das vermeinte Eigentum aufhören. 
	
	\subsection*{tg441.2.86} 
	\textbf{Source : }Die Metaphysik der Sitten/Erster Teil. Metaphysische Anfangsgründe der Rechtslehre/2. Teil. Das öffentliche Recht/1. Abschnitt. Das Staatsrecht\\  
	
	\noindent\textbf{Paragraphe : }1) Der Untertan (auch als Bürger betrachtet) hat das Recht der Auswanderung; denn der Staat könnte ihn nicht als sein Eigentum zurückhalten. Doch kann er nur seine fahrende, nicht die liegende Habe mit herausnehmen, welches alsdann doch geschehen würde, wenn er seinen bisher besessenen \match{Boden} zu verkaufen, und das Geld dafür mit sich zu nehmen, befugt wäre. 
	
	\subsection*{tg441.2.87} 
	\textbf{Source : }Die Metaphysik der Sitten/Erster Teil. Metaphysische Anfangsgründe der Rechtslehre/2. Teil. Das öffentliche Recht/1. Abschnitt. Das Staatsrecht\\  
	
	\noindent\textbf{Paragraphe : }
	2) Der Landesherr hat das Recht der Begünstigung der Einwanderung und Ansiedelung Fremder (Kolonisten), obgleich seine Landeskinder dazu scheel sehen möchten; wenn ihnen nur nicht das Privateigentum derselben am \match{Boden} gekürzt wird. 
	
	\subsection*{tg442.2.32} 
	\textbf{Source : }Die Metaphysik der Sitten/Erster Teil. Metaphysische Anfangsgründe der Rechtslehre/2. Teil. Das öffentliche Recht/2. Abschnitt. Das Völkerrecht\\  
	
	\noindent\textbf{Paragraphe : }Der überwundene Staat, oder dessen Untertanen, verlieren durch die Eroberung des Landes nicht ihre staatsbürgerliche Freiheit, so, daß jener zur Kolonie, diese zu Leibeigenen abgewürdigt würden; denn sonst wäre es ein Strafkrieg gewesen, der an sich selbst widersprechend ist. – Eine Kolonie oder Provinz ist ein Volk, das zwar seine eigene Verfassung, Gesetzgebung, \match{Boden} hat, auf welchem die zu einem anderen Staat Gehörige nur Fremdlinge sind, der dennoch über jenes die oberste ausübende Gewalt hat. Der letztere heißt der Mutterstaat. Der Tochterstaat wird von jenem beherrscht, aber doch von sich selbst (durch sein eigenes Parlament, allenfalls unter dem Vorsitz eines Vizekönigs) regiert (civitas hybrida). Dergleichen war Athen in Beziehung auf verschiedene Inseln, und ist jetzt Großbritannien in Ansehung Irlands. 
	
	\subsection*{tg443.2.5} 
	\textbf{Source : }Die Metaphysik der Sitten/Erster Teil. Metaphysische Anfangsgründe der Rechtslehre/2. Teil. Das öffentliche Recht/3. Abschnitt. Das Weltbürgerrecht\\  
	
	\noindent\textbf{Paragraphe : }Meere können Völker aus aller Gemeinschaft mit einander zu setzen scheinen, und dennoch sind sie, vermittelst der Schiffahrt, gerade die glücklichsten Naturanlagen zu ihrem Verkehr, welches, je mehr es einander nahe Küsten gibt (wie die des mittelländischen), nur desto lebhafter sein kann, deren Besuchung gleichwohl, noch mehr aber die Niederlassung auf denselben, um sie mit dem Mutterlande zu verknüpfen, zugleich die Veranlassung dazu gibt, daß Übel und Gewalttätigkeit, an einem Orte unseres Globs, an allen gefühlt wird. Dieser mögliche Mißbrauch kann aber das Recht des Erdbürgers nicht aufheben, die Gemeinschaft mit allen zu versuchen, und zu diesem Zweck alle Gegenden der Erde zu besuchen, wenn es gleich nicht ein Recht der Ansiedelung auf dem \match{Boden} eines anderen Volks (ius incolatus) ist, als zu welchem ein besonderer Vertrag erfordert wird. 
	
	\subsection*{tg445.2.57} 
	\textbf{Source : }Die Metaphysik der Sitten/Erster Teil. Metaphysische Anfangsgründe der Rechtslehre/Anhang erläutender Bemerkungen zu den metaphysischen Anhangsgründen der Rechtslehre\\  
	
	\noindent\textbf{Paragraphe : }Selbst Stiftungen zu ewigen Zeiten für Arme, oder Schulanstalten, sobald sie einen gewissen, von dem Stifter nach  seiner Idee bestimmten entworfenen Zuschnitt haben, können nicht auf ewige Zeiten fundiert und der \match{Boden} damit belästigt werden; sondern der Staat muß die Freiheit haben, sie nach dem Bedürfnisse der Zeit einzurichten. – Daß es schwerer hält, diese Idee allerwärts auszuführen (z.B. die Pauperbursche die Unzulänglichkeit des wohltätig errichteten Schulfonds durch bettelhaftes Singen ergänzen zu müssen,) darf niemanden wundern; denn der, welcher gutmütiger- aber doch zugleich etwas ehrbegierigerweise eine Stiftung macht, will, daß sie nicht ein anderer nach seinen Begriffen umändere, sondern Er darin unsterblich sei. Das ändert aber nicht die Beschaffenheit der Sache selbst und das Recht des Staats, ja die Pflicht desselben zum Umändern einer jeden Stiftung, wenn sie der Erhaltung und dem Fortschreiten desselben zum Besseren entgegen ist, kann daher niemals als auf ewig begründet betrachtet werden. 
	
	\unnumberedsection{Fallen (10)} 
	\subsection*{tg430.2.23} 
	\textbf{Source : }Die Metaphysik der Sitten/Erster Teil. Metaphysische Anfangsgründe der Rechtslehre/Einleitung in die Metaphysik der Sitten\\  
	
	\noindent\textbf{Paragraphe : }Alle Gesetzgebung also (sie mag auch in Ansehung der Handlung, die sie zur Pflicht macht, mit einer anderen übereinkommen, z.B. die Handlungen mögen in allen \match{Fällen} äußere sein) kann doch in Ansehung der Triebfedern unterschieden sein. Diejenige, welche eine Handlung zur Pflicht, und diese Pflicht zugleich zur Triebfeder macht, ist ethisch. Diejenige aber, welche das letztere nicht im Gesetze mit einschließt, mithin auch eine andere Triebfeder, als die Idee der Pflicht selbst, zuläßt, ist juridisch. Man sieht in Ansehung der letztern leicht ein, daß diese von der Idee der Pflicht unterschiedene Triebfeder von den pathologischen Bestimmungsgründen der Willkür der Neigungen und Abneigungen und unter diesen von denen der letzteren Art hergenommen sein müssen, weil es eine Gesetzgebung, welche nötigend, nicht eine Anlockung, die einladend ist, sein soll. 
	
	\subsection*{tg431.2.41} 
	\textbf{Source : }Die Metaphysik der Sitten/Erster Teil. Metaphysische Anfangsgründe der Rechtslehre/Einleitung in die Rechtslehre\\  
	
	\noindent\textbf{Paragraphe : }Man sieht: daß in beiden Rechtsbeurteilungen (nach dem Billigkeits- und dem Notrechte) die Doppelsinnigkeit (aequivocatio) aus der Verwechselung der objektiven mit den  subjektiven Gründen der Rechtsausübung (vor der Vernunft und vor einem Gericht) entspringt, da dann, was jemand für sich selbst mit gutem Grunde für Recht erkennt, vor einem Gerichtshofe nicht Bestätigung finden, und, was er selbst an sich als unrecht beurteilen muß, von eben demselben Nachsicht erlangen kann; weil der Begriff des Rechts in diesen zwei \match{Fällen} nicht in einerlei Bedeutung ist genommen worden. 
	
	\subsection*{tg439.2.39} 
	\textbf{Source : }Die Metaphysik der Sitten/Erster Teil. Metaphysische Anfangsgründe der Rechtslehre/1. Teil. Das Privatrecht vom äußeren Mein und Dein überhaupt/3. Hauptstück. Von der subjektiv-bedingten Erwerbung durch den Ausspruch einer öffentlichen Gerichtsbarkeit\\  
	
	\noindent\textbf{Paragraphe : }Aber in Beziehung auf einen Gerichtshof, also im bürgerlichen Zustande, wenn man annimmt, daß es kein anderes Mittel gibt, in gewissen \match{Fällen} hinter die Wahrheit zu kommen, als den Eid, muß von der Religion vorausgesetzt werden, daß sie jeder habe, um sie, als ein Notmittel (in casu necessitatis), zum Behuf des rechtlichen Verfahrens vor einem Gerichtshofe zu gebrauchen, welcher diesen Geisteszwang (tortura spiritualis) für ein behenderes und dem abergläubischen Hange der Menschen angemesseneres Mittel der Aufdeckung des Verborgenen, und sich darum für berechtigt hält, es zu gebrauchen. – Die gesetzgebende Gewalt handelt aber im Grunde unrecht, diese Befugnis der richterlichen zu erteilen; weil selbst im bürgerlichen Zustande ein Zwang zu Eidesleistungen der unverlierbaren menschlichen Freiheit zuwider ist. 
	
	\subsection*{tg441.2.73} 
	\textbf{Source : }Die Metaphysik der Sitten/Erster Teil. Metaphysische Anfangsgründe der Rechtslehre/2. Teil. Das öffentliche Recht/1. Abschnitt. Das Staatsrecht\\  
	
	\noindent\textbf{Paragraphe : }So viel also der Mörder sind, die den Mord verübt, oder auch befohlen, oder dazu mitgewirkt haben, so viele müssen auch den Tod leiden; so will es die Gerechtigkeit als Idee der richterlichen Gewalt nach allgemeinen a priori begründeten Gesetzen. – Wenn aber doch die Zahl der Komplizen (correi) zu einer solchen Tat so groß ist, daß der Staat, um keine solche Verbrecher zu haben, bald dahin kommen könnte, keine Untertanen mehr zu haben, und sich doch nicht auflösen, d.i. in den noch viel ärgeren, aller äußeren Gerechtigkeit entbehrenden Naturzustand übergehen (vornehmlich nicht durch das Spektakel einer Schlachtbank das Gefühl des Volks abstumpfen) will, so muß es auch der Souverän in seiner Macht haben, in diesem Notfall (casus necessitatis) selbst den Richter zu machen (vorzustellen) und ein Urteil zu sprechen, welches, statt der Lebensstrafe, eine andere den Verbrechern zuerkennt, bei der die Volksmenge noch erhalten wird; dergleichen die Deportation ist: Dieses selbst aber nicht als nach einem öffentlichen Gesetz, sondern durch einen Machtspruch, d.i. einen Akt des Majestätsrechts, der, als Begnadigung, nur immer in einzelnen \match{Fällen} ausgeübt werden kann. 
	
	\subsection*{tg441.2.76} 
	\textbf{Source : }Die Metaphysik der Sitten/Erster Teil. Metaphysische Anfangsgründe der Rechtslehre/2. Teil. Das öffentliche Recht/1. Abschnitt. Das Staatsrecht\\  
	
	\noindent\textbf{Paragraphe : }Es gibt indessen zwei todeswürdige Verbrechen, in Ansehung deren, ob die Gesetzgebung auch die Befugnis habe, sie mit der Todesstrafe zu belegen, noch zweifelhaft bleibt. Zu beiden verleitet das Ehrgefühl. Das eine ist das der Geschlechtsehre, das andere der Kriegsehre, und zwar der wahren Ehre, welche jeder dieser zwei Menschenklassen als Pflicht obliegt. Das eine Verbrechen ist der mütterliche Kindesmord (infanticidium maternale); das andere der Kriegsgesellenmord (commilitonicidium), der Duell. – Da die Gesetzgebung die Schmach einer unehelichen Geburt nicht wegnehmen, und eben so wenig den Fleck, welcher aus dem Verdacht der Feigheit, der auf einen untergeordneten Kriegsbefehlshaber fällt, welcher einer verächtlichen Begegnung nicht eine über die Todesfurcht erhobene eigene Gewalt entgegensetzt, wegwischen kann: so scheint es, daß Menschen in diesen \match{Fällen} sich im Naturzustande befinden und Tötung (homicidium), die alsdann nicht einmal Mord (homicidium dolosum) heißen müßte, in beiden zwar allerdings strafbar sei, von der obersten Macht aber mit dem Tode nicht könne bestraft werden. Das uneheliche auf die Weit gekommene Kind ist außer dem Gesetz (denn das heißt Ehe), mithin auch außer dem Schutz desselben, geboren. Es ist in das gemeine Wesen gleichsam eingeschlichen (wie verbotene Ware), so daß dieses seine Existenz (weil es billig auf diese Art nicht hätte existieren sollen), mithin auch seine Vernichtung ignorieren kann,  und die Schande der Mutter, wenn ihre uneheliche Niederkunft bekannt, wird, kann keine Verordnung heben. – Der zum Unter-Befehlshaber eingesetzte Kriegesmann, dem ein Schimpf angetan wird, sieht sich eben so wohl durch die öffentliche Meinung der Mitgenossen seines Standes genötigt, sich Genugtuung, und, wie im Naturzustande, Bestrafung des Beleidigers, nicht durchs Gesetz, vor einem Gerichtshofe, sondern durch den Duell, darin er sich selbst der Lebensgefahr aussetzt, zu verschaffen, um seinen Kriegsmut zu beweisen, als worauf die Ehre seines Standes wesentlich beruht, sollte es auch mit der Tötung seines Gegners verbunden sein, die In diesem Kampfe, der öffentlich und mit beiderseitiger Einwilligung, doch auch ungern, geschieht, eigentlich nicht Mord (homicidium dolosum) genannt werden kann. – – Was ist nun in beiden (zur Kriminalgerechtigkeit gehörigen) Fällen Rechtens? – Hier kommt die Strafgerechtigkeit gar sehr ins Gedränge: entweder den Ehrbegriff (der hier kein Wahn ist) durchs Gesetz für nichtig zu erklären, und so mit dem Tode zu strafen, oder von dem Verbrechen die angemessene Todesstrafe wegzunehmen, und so entweder grausam oder nachsichtig zu sein. Die Auflösung dieses Knotens ist: daß der kategorische Imperativ der Strafgerechtigkeit (die gesetzwidrige Tötung eines anderen müsse mit dem Tode bestraft werden) bleibt, die Gesetzgebung selber aber (mithin auch die bürgerliche Verfassung), so lange noch als barbarisch und unausgebildet, daran Schuld ist, daß die Triebfedern der Ehre im Volk (subjektiv) nicht mit den Maßregeln zusammen treffen wollen, die (objektiv) ihrer Absicht gemäß sind, so daß die öffentliche, vom Staat ausgehende, Gerechtigkeit in Ansehung der aus dem Volk eine Ungerechtigkeit wird. 
	
	\subsection*{tg444.2.4} 
	\textbf{Source : }Die Metaphysik der Sitten/Erster Teil. Metaphysische Anfangsgründe der Rechtslehre/Beschluß\\  
	
	\noindent\textbf{Paragraphe : }Man kann sagen, daß diese allgemeine und fortdauernde Friedensstiftung nicht bloß einen Teil, sondern den ganzen Endzweck der Rechtslehre innerhalb den Grenzen der bloßen Vernunft ausmache; denn der Friedenszustand ist allein der unter Gesetzen gesicherte Zustand des Mein und Dein in einer Menge einander benachbarter Menschen, mithin die in einer Verfassung zusammen sind, deren Regel aber nicht von der Erfahrung derjenigen, die sich bisher am besten dabei befunden haben, als einer Norm für andere, sondern die durch die Vernunft a priori von dem Ideal einer rechtlichen Verbindung der Menschen unter öffentlichen Gesetzen überhaupt hergenommen werden muß, weil alle Beispiele (als die nur erläutern, aber nichts beweisen können) trüglich sind, und so allerdings einer Metaphysik bedürfen, deren Notwendigkeit diejenigen, die dieser spotten, doch unvorsichtiger Weise selbst zugestehen, wenn sie z.B., wie sie es oft tun, sagen: »die beste Verfassung ist die, wo nicht die Menschen, sondern die Gesetze machthabend sind«. Denn was kann mehr metaphysisch sublimiert sein, als eben diese Idee, welche gleichwohl, nach jener ihrer eigenen Behauptung, die bewährteste objektive Realität hat, die sich auch in vorkommenden \match{Fällen} leicht darstellen läßt, und welche allein, wenn sie nicht revolutionsmäßig, durch einen Sprung, d.i. durch gewaltsame Umstürzung einer bisher bestandenen fehlerhaften – (denn da würde sich zwischeninne ein Augenblick der Vernichtung alles rechtlichen Zustandes ereignen) sondern durch allmähliche Reform nach festen Grundsätzen versucht und durchgeführt wird, in kontinuierlicher Annäherung zum höchsten politischen Gut, zum ewigen Frieden, hinleiten kann. 
	
	\subsection*{tg465.2.9} 
	\textbf{Source : }Die Metaphysik der Sitten/Zweiter Teil. Metaphysische Anfangsgründe der Tugendlehre/Einleitung/XVII. Vorbegriffe zur Einteilung der Tugendlehre\\  
	
	\noindent\textbf{Paragraphe : }Wie komme ich aber dazu, wird man fragen, die Einteilung der Ethik in Elementarlehre und Methodenlehre einzuführen: da ich ihrer doch in der Rechtslehre überhoben sein konnte? – Die Ursache ist: weil jene es mit weiten, diese aber mit lauter engen Pflichten zu tun hat; weshalb die letztere, welche ihrer Natur nach strenge (präzis) bestimmend sein muß, eben so wenig wie die reine Mathematik einer allgemeinen Vorschrift (Methode), wie im Urteilen verfahren werden soll, bedarf, sondern sie durch die Tatwahr macht. – Die Ethik hingegen führt, wegen des Spielraums, den sie ihren unvollkommenen Pflichten verstattet, unvermeidlich dahin, zu Fragen, welche die Urteilskraft auffordern auszumachen, wie eine Maxime in besonderen \match{Fällen} anzuwenden sei, und zwar so: daß diese wiederum eine (untergeordnete) Maxime an die Hand gebe (wo immer wiederum nach einem Prinzip der Anwendung dieser auf vorkommende Fälle gefragt werden kann); und so gerät sie in eine Kasuistik, von welcher die Rechtslehre nichts weiß. 
	
	\subsection*{tg469.2.8} 
	\textbf{Source : }Die Metaphysik der Sitten/Zweiter Teil. Metaphysische Anfangsgründe der Tugendlehre/I. Ethische Elementarlehre/I. Teil. Von den Pflichten gegen sich selbst überhaupt/Einleitung\\  
	
	\noindent\textbf{Paragraphe : }Denn setzet: es gebe keine solche Pflichten, so würde es überall gar keine, auch keine äußere Pflichten geben. – Denn ich kann mich gegen andere nicht für verbunden erkennen, als nur so fern ich zugleich mich selbst verbinde; weil das Gesetz, kraft dessen ich mich für verbunden achte, in allen  \match{Fällen} aus meiner eigenen praktischen Vernunft hervorgeht, durch welche ich genötigt werde, indem ich zugleich der Nötigende in Ansehung meiner selbst bin.
	
	
	18
	
	
	
	\subsection*{tg481.2.84} 
	\textbf{Source : }Die Metaphysik der Sitten/Zweiter Teil. Metaphysische Anfangsgründe der Tugendlehre/I. Ethische Elementarlehre/II. Teil. Von den Tugendpflichten gegen andere/Erstes Hauptstück. Von den Pflichten gegen andere, bloß als Menschen/Erster Abschnitt. Von der Liebespflicht gegen andere Menschen\\  
	
	\noindent\textbf{Paragraphe : }a) Der Neid (livor), als Hang, das Wohl anderer mit Schmerz, wahrzunehmen, ob zwar dem seinigen dadurch kein Abbruch geschieht, der, wenn er zur Tat (jenes Wohl zu schmälern) ausschlägt, qualifizierter Neid, sonst aber nur Mißgunst (invidentia) heißt, ist doch nur eine indirekt-bösartige Gesinnung, nämlich ein Unwille, unser eigen Wohl durch das Wohl anderer in Schatten gestellt zu sehen, weil wir den Maßstab desselben nicht in dessen innerem Wert, sondern nur in der Vergleichung mit dem Wohl anderer, zu schätzen, und diese Schätzung zu versinnlichen wissen. – Daher spricht man auch wohl von einer beneidungswürdigen Eintracht und Glückseligkeit in einer Ehe, oder Familie u.s.w.; gleich als ob es in manchen \match{Fällen} erlaubt wäre, jemanden zu beneiden. Die Regungen des Neides liegen also in der Natur des Menschen, und nur der Ausbruch derselben macht sie zu dem scheußlichen Laster einer grämischen, sich selbst folternden und auf Zerstörung des Glücks anderer, wenigstens dem Wunsche nach, gerichteten Leidenschaft, ist mithin der Pflicht des Menschen gegen sich selbst so wohl, als gegen andere entgegengesetzt. 
	
	\subsection*{tg487.2.5} 
	\textbf{Source : }Die Metaphysik der Sitten/Zweiter Teil. Metaphysische Anfangsgründe der Tugendlehre/II. Ethische Methodenlehre/2. Abschnitt. Die ethische Asketik\\  
	
	\noindent\textbf{Paragraphe : }Die Kultur der Tugend, d.i. die moralische Asketik, hat, in Ansehung des Prinzips der rüstigen, mutigen und wackeren Tugendübung den Wahlspruch der Stoiker: gewöhne dich, die zufälligen Lebensübel zu ertragen und die eben so überflüssigen Ergötzlichkeiten zu entbehren (assuesce incommodis et desuesce commoditatibus vitae). Es ist eine Art von Diätetik für den Menschen, sich moralisch gesund zu erhalten. Gesundheit ist aber nur ein negatives  Wohlbefinden, sie selber kann nicht gefühlt werden. Es muß etwas dazu kommen, was einen angenehmen Lebensgenuß gewährt und doch bloß moralisch ist. Das ist das jederzeit fröhliche Herz in der Idee des tugendhaften Epikurs. Denn wer sollte wohl mehr Ursache haben, frohen Muts zu sein und nicht darin selbst eine Pflicht finden, sich in eine fröhliche Gemütsstimmung zu versetzen und sie sich habituell zu machen, als der, welcher sich keiner vorsätzlichen Übertretung bewußt und, wegen des Verfalls in eine solche, gesichert ist (hic murus ahenëus esto etc. Horat.). – Die Mönchsasketik hingegen, welche aus abergläubischer Furcht, oder geheucheltem Abscheu an sich selbst, mit Selbstpeinigung und Fleischeskreuzigung zu Werke geht, zweckt auch nicht auf Tugend, sondern auf schwärmerische Entsündigung ab, sich selbst Strafe aufzulegen und, anstatt sie moralisch (d.i. in Absicht auf die Besserung) zu bereuen, sie büßen zu wollen; welches, bei einer selbstgewählten und an sich vollstreckten Strafe (denn die muß immer ein anderer auflegen), ein Widerspruch ist, und kann auch den Frohsinn, der die Tugend begleitet, nicht bewirken, vielmehr nicht ohne geheimen Haß gegen das Tugendgebot statt finden. – Die ethische Gymnastik besteht also nur in der Bekämpfung der Naturtriebe, die das Maß erreicht, über sie bei vorkommenden, der Moralität Gefahr drohenden, \match{Fällen} Meister werden zu können; mithin die wacker und, im Bewußtsein seiner wiedererworbenen Freiheit, fröhlich macht. Etwas bereuen (welches bei der Rückerinnerung ehemaliger Übertretungen unvermeidlich, ja wobei diese Erinnerung nicht schwinden zu lassen es so gar Pflicht ist) und sich eine Pönitenz auferlegen (z.B. das Fasten), nicht in diätetischer, sondern frommer Rücksicht, sind zwei sehr verschiedene, moralisch gemeinte, Vorkehrungen, von denen die letztere, welche freudenlos, finster und mürrisch ist, die Tugend selbst verhaßt macht und ihre Anhänger verjagt. Die Zucht (Disziplin), die der Mensch an sich selbst verübt, kann daher nur durch den Frohsinn, der sie begleitet, verdienstlich und exemplarisch werden. 
	
	\unnumberedsection{Feld (3)} 
	\subsection*{tg430.2.47} 
	\textbf{Source : }Die Metaphysik der Sitten/Erster Teil. Metaphysische Anfangsgründe der Rechtslehre/Einleitung in die Metaphysik der Sitten\\  
	
	\noindent\textbf{Paragraphe : }Die Einfachheit dieses Gesetzes in Vergleichung mit den großen und mannigfaltigen Folgerungen, die daraus gezogen werden können, imgleichen das gebietende Ansehen, ohne daß es doch sichtbar eine Triebfeder bei sich führt, muß freilich anfänglich befremden. Wenn man aber, in dieser Verwunderung über ein Vermögen unserer Vernunft, durch die bloße Idee der Qualifikation einer Maxime zur Allgemeinheit eines praktischen Gesetzes die Willkür zu bestimmen,  belehrt wird: daß eben diese praktischen Gesetze (die moralischen) eine Eigenschaft der Willkür zuerst kund machen, auf die keine spekulative Vernunft, weder aus Gründen a priori, noch durch irgend eine Erfahrung, geraten hätte, und, wenn sie darauf geriet, ihre Möglichkeit theoretisch durch nichts dartun könnte, gleichwohl aber jene praktischen Gesetze diese Eigenschaft, nämlich die Freiheit, unwidersprechlich dartun: so wird es weniger befremden, diese Gesetze, gleich mathematischen Postulaten, unerweislich und doch apodiktisch zu finden, zugleich aber ein ganzes \match{Feld} von praktischen Erkenntnissen vor sich eröffnet zu sehen, wo die Vernunft mit derselben Idee der Freiheit, ja jeder anderer ihrer Ideen des Übersinnlichen im Theoretischen alles schlechterdings vor ihr verschlossen finden muß. Die Übereinstimmung einer Handlung mit dem Pflichtgesetze ist die Gesetzmäßigkeit (legalitas) – die der Maxime der Handlung mit dem Gesetze die Sittlichkeit (moralitas) derselben. Maxime aber ist das subjektive Prinzip zu handeln, was sich das Subjekt selbst zur Regel macht (wie es nämlich handeln will). Dagegen ist der Grundsatz der Pflicht das, was ihm die Vernunft schlechthin, mithin objektiv gebietet (wie es handeln soll). 
	
	\subsection*{tg454.2.2} 
	\textbf{Source : }Die Metaphysik der Sitten/Zweiter Teil. Metaphysische Anfangsgründe der Tugendlehre/Einleitung/VI. Die Ethik gibt nicht Gesetze für die Handlungen [...] sondern nur für die Maximen der Handlungen\\  
	
	\noindent\textbf{Paragraphe : }Der Pflichtbegriff steht unmittelbar in Beziehung auf ein Gesetz (wenn ich gleich noch von allem Zweck, als der Materie desselben, abstrahiere); wie denn das formale Prinzip der Pflicht im kategorischen Imperativ: »handle so, daß die Maxime deiner Handlung ein allgemeines Gesetz werden könne«, es schon anzeigt; nur daß in der Ethik dieses als das Gesetz deines eigenen Willens gedacht wird, nicht des Willens überhaupt, der auch der Wille anderer sein könnte: wo es alsdenn eine Rechtspflicht abgeben würde, die nicht in das \match{Feld} der Ethik gehört. – Die Maximen werden hier als solche subjektive Grundsätze angesehen, die sich zu einer allgemeinen Gesetzgebung bloß qualifizieren; welches nur ein negatives Prinzip (einem Gesetz überhaupt nicht zu widerstreiten) ist. – Wie kann es aber dann noch ein Gesetz für die Maxime der Handlungen geben? 
	
	\subsection*{tg455.2.2} 
	\textbf{Source : }Die Metaphysik der Sitten/Zweiter Teil. Metaphysische Anfangsgründe der Tugendlehre/Einleitung/VII. Die ethischen Pflichten sind von weiter, dagegen die Rechtspflichten von enger Verbindlichkeit\\  
	
	\noindent\textbf{Paragraphe : }Dieser Satz ist eine Folge aus dem vorigen; denn wenn das Gesetz nur die Maxime der Handlungen, nicht die Handlungen selbst, gebieten kann, so ist's ein Zeichen, daß es der Befolgung (Observanz) einen Spielraum (latitudo) für die freie Willkür überlasse, d.i. nicht bestimmt angeben könne, wie und wie viel durch die Handlung zu dem Zweck, der zugleich Pflicht ist, gewirkt werden solle. – Es wird aber unter einer weiten Pflicht nicht eine Erlaubnis zu Ausnahmen von der Maxime der Handlungen, sondern nur die der Einschränkung einer Pflichtmaxime durch die andere (z.B. die allgemeine Nächstenliebe durch die Elternliebe) verstanden, wodurch in der Tat das \match{Feld} für die Tugendpraxis erweitert wird. – Je weiter die Pflicht, je unvollkommener also die Verbindlichkeit des Menschen zur Handlung ist, je näher er gleichwohl die Maxime der Observanz derselben (in seiner Gesinnung) der engen Pflicht (des Rechts) bringt, desto vollkommener ist seine Tugendhandlung. 
	
	\unnumberedsection{Gut (2)} 
	\subsection*{tg433.2.19} 
	\textbf{Source : }Die Metaphysik der Sitten/Erster Teil. Metaphysische Anfangsgründe der Rechtslehre/1. Teil. Das Privatrecht vom äußeren Mein und Dein überhaupt/1. Hauptstück\\  
	
	\noindent\textbf{Paragraphe : }b) Ich kann die Leistung von etwas durch die Willkür des andern nicht mein nennen, wenn ich bloß sagen kann, sie sei mit meinem Versprechen zugleich (pactum re initum) in meinen Besitz gekommen, sondern nur, wenn ich behaupten darf, ich bin im Besitz der Willkür des andern (diesen zur Leistung zu bestimmen), obgleich die Zeit der Leistung noch erst kommen soll; das Versprechen des letzteren gehört demnach zur Habe und \match{Gut} (obligatio activa) und ich kann sie zu dem Meinen rechnen, aber nicht bloß, wenn ich das Versprochene (wie im ersten Falle) schon in meinem Besitz habe, sondern auch, ob ich dieses gleich noch nicht besitze. Also muß ich mich, als von dem auf Zeitbedingung eingeschränkten, mithin vom empirischen Besitze unabhängig, doch im Besitz dieses Gegenstandes zu sein denken können. 
	
	\subsection*{tg445.2.41} 
	\textbf{Source : }Die Metaphysik der Sitten/Erster Teil. Metaphysische Anfangsgründe der Rechtslehre/Anhang erläutender Bemerkungen zu den metaphysischen Anhangsgründen der Rechtslehre\\  
	
	\noindent\textbf{Paragraphe : }Was das Recht der Beerbung anlangt, so hat den Herrn Rezensenten diesesmal sein Scharfblick, den Nerven des Beweises meiner Behauptung zu treffen, verlassen. – Ich sage ja nicht S. 135: »daß ein jeder Mensch notwendigerweise jede ihm angebotene Sache, durch deren Annehmung er nur gewinnen, nichts verlieren kann, annehme« (denn solche Sachen gibt es gar nicht), sondern daß ein jeder das Recht des Angebots in demselben Augenblick unvermeidlich und stillschweigend, dabei aber doch gültig, immer wirklich annehme: wenn es nämlich die Natur der Sache so mit sich bringt, daß der Widerruf schlechterdings unmöglich ist, nämlich im Augenblicke seines Todes; denn da kann der Promittent nicht widerrufen, und der Promissar ist, ohne irgend einen rechtlichen Akt begehen zu dürfen, in demselben Augenblick Akzeptant, nicht der versprochenen Erbschaft, sondern des Rechts, sie anzunehmen oder auszuschlagen. In diesem Augenblicke sieht er sich bei Eröffnung des Testaments, daß er, schon vor der Akzeptation der Erbschaft, vermögender geworden ist, als er war; denn er hat ausschließlich die Befugnis zu akzeptieren erworben, welche schon ein Vermö gensumstand ist. – Daß hiebei ein bürgerlicher Zustand vorausgesetzt wird, um etwas zu dem Seinen eines anderen zu machen, wenn man nicht mehr da ist, dieser Übergang des Besitztums, aus der  Totenhand, ändert in Ansehung der Möglichkeit der Erwerbung nach allgemeinen Prinzipien des Naturrechts nichts, wenn gleich der Anwendung derselben auf den vorkommenden Fall eine bürgerliche Verfassung zum Grunde gelegt werden muß. – Eine Sache nämlich, die ohne Bedingung anzunehmen oder auszuschlagen in meiner freien Wahl gestellt wird, heißt res iacens. Wenn der Eigentümer einer Sache mir etwas, z.B. ein Möbel des Hauses, aus dem ich auszuziehen eben im Begriff bin, umsonst anbietet (verspricht, es soll mein sein), so habe ich, so lange er nicht widerruft (welches, wenn er darüber stirbt, unmöglich ist), ausschließlich ein Recht zur Akzeptation des Angebotenen (ius in re iacente), d.i. ich allein kann es annehmen oder ausschlagen, wie es mir beliebt: und dieses Recht, ausschließlich zu wählen, erlange ich nicht vermittelst eines besonderen rechtlichen Akts meiner Deklaration, ich wolle, dieses Recht solle mir zustehen, sondern ohne denselben (lege). – Ich kann also zwar mich dahin erklären, ich wolle, die Sache solle mir nicht angehören (weil diese Annahme mir Verdrießlichkeiten mit anderen zuziehen dürfte), aber ich kann nicht wollen, ausschließlich die Wahl zu haben, ob sie mir angehören solle oder nicht; denn dieses Recht (des Annehmens oder Ausschlagens) habe ich ohne alle Deklaration meiner Annahme, unmittelbar durchs Angebot: denn wenn ich sogar die Wahl zu haben ausschlagen könnte, so würde ich wählen, nicht zu wählen; welches ein Widerspruch ist. Dieses Recht zu wählen geht nun im Augenblicke des Todes des Erb-Lassers auf mich über, durch dessen Vermächtnis (institutio heredis) ich zwar noch nichts von der Habe und \match{Gut} des Erb-Lassers, aber doch den bloß-rechtlichen (intelligibelen) Besitz dieser Habe oder eines Teils derselben erwerbe: deren Annahme ich mich nun zum Vorteil anderer begeben kann, mithin dieser Besitz keinen Augenblick unterbrochen ist, sondern die Sukzession als eine stetige Reihenfolge, vom Sterbenden zum eingesetzten Erben durch seine Akzeptation übergeht und so der Satz: testamenta sunt iuris naturae, wider alle Zweifel befestigt wird. 
	
	\unnumberedsection{Gutsbesitzer (1)} 
	\subsection*{tg445.2.65} 
	\textbf{Source : }Die Metaphysik der Sitten/Erster Teil. Metaphysische Anfangsgründe der Rechtslehre/Anhang erläutender Bemerkungen zu den metaphysischen Anhangsgründen der Rechtslehre\\  
	
	\noindent\textbf{Paragraphe : }Was endlich die Majoratsstiftung betrifft, da ein \match{Gutsbesitzer} durch Erbeseinsetzung verordnet: daß in der Reihe der auf einander folgenden Erben immer der Nächste von der Familie der Gutsherr sein solle (nach der Analogie mit einer monarchisch-erblichen Verfassung eines Staats, wo der Landesherr es ist), so kann eine solche Stiftung nicht allein mit Beistimmung aller Agnaten jederzeit aufgehoben werden und darf nicht auf ewige Zeiten – gleich als ob das Erbrecht am Boden haftete – immerwährend fortdauern, noch gesagt werden, es sei eine Verletzung der Stiftung und des Willens des Urahnherrn derselben, des Stifters, sie eingehen zu lassen: sondern der Staat hat auch hier ein Recht, ja sogar die Pflicht, bei den allmählich eintretenden Ursachen seiner eigenen Reform ein solches föderatives System seiner Untertanen, gleich als Unterkönige (nach der Analogie von Dynasten und Satrapen), wenn es erloschen ist, nicht weiter aufkommen zu lassen. 
	
	\unnumberedsection{Lage (1)} 
	\subsection*{tg431.2.24} 
	\textbf{Source : }Die Metaphysik der Sitten/Erster Teil. Metaphysische Anfangsgründe der Rechtslehre/Einleitung in die Rechtslehre\\  
	
	\noindent\textbf{Paragraphe : }Das Gesetz eines mit jedermanns Freiheit notwendig zusammenstimmenden wechselseitigen Zwanges, unter dem Prinzip der allgemeinen Freiheit, ist gleichsam die Konstruktion jenes Begriffs, d.i. Darstellung desselben in einer reinen Anschauung a priori, nach der Analogie der Möglichkeit freier Bewegungen der Körper unter dem Gesetze der Gleichheit der Wirkung und Gegenwirkung. So wie wir nun in der reinen Mathematik die Eigenschaften ihres Objekts nicht unmittelbar vom Begriffe ableiten, sondern nur durch die Konstruktion des Begriffs entdecken können, so ist's nicht sowohl der Begriff des Rechts, als vielmehr der, unter allgemeine Gesetze gebrachte, mit ihm zusammenstimmende durchgängig wechselseitige und gleiche Zwang, der die Darstellung jenes Begriffs möglich macht. Dieweil aber diesem dynamischen Begriffe noch ein bloß formaler, in der reinen Mathematik (z.B. der Geometrie) zum Grunde liegt: so hat die Vernunft dafür gesorgt, den Verstand auch mit Anschauungen a priori, zum Behuf der Konstruktion des Rechtsbegriffs, so viel möglich zu versorgen. – Das Rechte (rectum) wird, als das Gerade, teils dem Krummen, teils dem Schiefen entgegen gesetzt. Das erste ist die innere Beschaffenheit einer Linie von der Art, daß es zwischen zweien gegebenen Punkten nur eine einzige, das zweite aber die \match{Lage} zweier einander durchschneidenden oder zusammenstoßenden Linien, von deren Art es auch nur eine einzige (die senkrechte) geben kann, die sich nicht mehr nach einer Seite, als der andern hinneigt, und die den Raum von beiden Seiten gleich abteilt, nach welcher Analogie auch die Rechtslehre das Seine einem jeden (mit mathematischer Genauigkeit) bestimmt wissen will, welches in der Tugendlehre nicht erwartet werden darf, als welche einen gewissen Raum zu Ausnahmen (latitudinem) nicht verweigern kann. – Aber, ohne ins Gebiet der Ethik einzugreifen, gibt es zwei Fälle, die auf Rechtsentscheidung  Anspruch machen, für die aber keiner, der sie entscheide, ausgefunden werden kann, und die gleichsam in Epikurs Intermundia hingehören. – Diese müssen wir zuvörderst aus der eigentlichen Rechtslehre, zu der wir bald schreiten wollen, aussondern, damit ihre schwankenden Prinzipien nicht auf die festen Grundsätze der erstern Einfluß bekommen. 
	
	\unnumberedsection{Land (4)} 
	\subsection*{tg441.2.85} 
	\textbf{Source : }Die Metaphysik der Sitten/Erster Teil. Metaphysische Anfangsgründe der Rechtslehre/2. Teil. Das öffentliche Recht/1. Abschnitt. Das Staatsrecht\\  
	
	\noindent\textbf{Paragraphe : }Das \match{Land} (territorium), dessen Einsassen schon durch die Konstitution, d.i. ohne einen besonderen rechtlichen Akt ausüben zu dürfen (mithin durch die Geburt), Mitbürger eines und desselben gemeinen Wesens sind, heißt das Vaterland; das, worin sie es ohne diese Bedingung nicht sind, das Ausland, und dieses, wenn es einen Teil der Landesherrschaft überhaupt ausmacht, heißt die Provinz (in der Bedeutung, wie die Römer dieses Wort brauchten), welche, weil sie doch keinen koalisierten Teil des Reichs (imperii) als Sitz von Mitbürgern, sondern nur eine Besitzung desselben, als eines Unterhauses ausmacht, den Boden des herrschenden Staats als Mutterland (regio domina) verehren muß. 
	
	\subsection*{tg442.2.14} 
	\textbf{Source : }Die Metaphysik der Sitten/Erster Teil. Metaphysische Anfangsgründe der Rechtslehre/2. Teil. Das öffentliche Recht/2. Abschnitt. Das Völkerrecht\\  
	
	\noindent\textbf{Paragraphe : }Es gibt mancherlei Naturprodukte in einem Lande, die doch, was die Menge derselben von einer gewissen Art betrifft, zugleich als Gemächsel (artefacta) des Staats angesehen werden müssen, weil das \match{Land} sie in solcher Menge nicht liefern würde, wenn es nicht einen Staat und eine ordentliche machthabende Regierung gäbe, sondern die Bewohner im Stande der Natur wären. – Haushühner (die nützlichste Art des Geflügels), Schafe, Schweine, das Rindergeschlecht u.a.m. würden, entweder aus Mangel an Futter, oder der Raubtiere wegen, in dem Lande, wo ich lebe, entweder gar nicht, oder höchst sparsam anzutreffen sein, wenn es darin nicht eine Regierung gäbe, welche den Einwohnern ihren Erwerb und Besitz sicherte. – Eben das gilt auch von der Menschenzahl, die, eben so wie in den amerikanischen Wüsten, ja selbst dann, wenn man diesen den größten Fleiß (den jene nicht haben) beilegte, nur gering sein kann. Die Einwohner würden nur sehr dünn gesäet sein, weil keiner derselben sich, mit samt seinem Gesinde, auf einem Boden weit verbreiten könnte, der immer in Gefahr ist, von Menschen oder Wilden und Raubtieren verwüstet zu werden; mithin sich für eine so große Menge von Menschen, als jetzt auf einem Lande leben, kein hinlänglicher Unterhalt finden würde. – – Sowie man nun von Gewächsen (z.B. den Kartoffeln) und von Haustieren, weil sie, was die Menge betrifft, ein Machwerk der Menschen sind, sagen kann, daß man sie gebrauchen, verbrauchen und verzehren (töten lassen) kann: so, scheint es, könne man auch von der obersten Gewalt im Staat, dem Souverän, sagen, er  habe das Recht, seine Untertanen, die dem größten Teil nach sein eigenes Produkt sind, in den Krieg, wie auf eine Jagd, und zu einer Feldschlacht, wie auf eine Lustpartie zu führen. 
	
	\subsection*{tg442.2.42} 
	\textbf{Source : }Die Metaphysik der Sitten/Erster Teil. Metaphysische Anfangsgründe der Rechtslehre/2. Teil. Das öffentliche Recht/2. Abschnitt. Das Völkerrecht\\  
	
	\noindent\textbf{Paragraphe : }Das Recht eines Staats gegen einen ungerechten Feind hat keine Grenzen (wohl zwar der Qualität, aber nicht der Quantität, d.i. dem Grade nach): d.i. der beeinträchtigte Staat darf sich zwar nicht aller Mittel, aber doch der an sich zulässigen in dem Maße bedienen, um das Seine zu behaupten, als er dazu Kräfte hat. – Was ist aber nun nach Begriffen des Völkerrechts, in welchem, wie überhaupt im Naturzustande, ein jeder Staat in seiner eigenen Sache Richter ist, ein ungerechter Feind? Es ist derjenige, dessen öffentlich (es sei wörtlich oder tätlich) geäußerter Wille eine Maxime verrät, nach welcher, wenn sie zur allgemeinen Regel gemacht würde, kein Friedenszustand unter Völkern möglich, sondern der Naturzustand verewigt werden müßte. Dergleichen ist die Verletzung öffentlicher Verträge, von welcher man voraussetzen kann, daß sie die Sache aller Völker betrifft, deren Freiheit dadurch bedroht wird, und die dadurch aufgefordert werden, sich gegen einen solchen Unfug zu vereinigen und ihm die Macht dazu zu nehmen; – aber doch auch nicht, um sich in sein \match{Land} zu teilen, einen Staat gleichsam auf der Erde verschwinden zu machen, denn das wäre Ungerechtigkeit gegen das Volk, welches sein ursprüngliches Recht, sich in ein gemeines Wesen zu verbinden, nicht verlieren kann, sondern es eine neue Verfassung annehmen zu lassen, die, ihrer Natur nach, der Neigung zum Kriege ungünstig ist. 
	
	\subsection*{tg488.2.16} 
	\textbf{Source : }Die Metaphysik der Sitten/Zweiter Teil. Metaphysische Anfangsgründe der Tugendlehre/Beschluß. Die Religionslehre als Lehre der Pflichten gegen Gott liegt außerhalb den Grenzen der reinen Moralphilosophie\\  
	
	\noindent\textbf{Paragraphe : }Die Strafe läßt (nach dem Horaz) den vor ihr stolz schreitenden Verbrecher nicht aus den Augen, sondern hinkt ihm unablässig nach, bis sie ihn ertappt. – Das unschuldig vergossene Blut schreit um Rache. – Das Verbrechen kann nicht ungerächt bleiben; trifft die Strafe nicht den Verbrecher, so werden es seine Nachkommen entgelten müssen; oder geschieht's nicht bei seinem Leben, so muß es in einem Leben nach dem Tode
	
	
	25
	geschehen, welches ausdrücklich darum auch angenommen und gern geglaubt wird, damit der Anspruch der ewigen Gerechtigkeit ausgeglichen werde. – Ich will keine Blutschuld auf mein \match{Land} kommen lassen, dadurch, daß ich einen boshaft mordenden Duellanten, für den ihr Fürbitte tut, begnadige, sagte einmal ein wohldenkender Landesherr. – Die Sündenschuld muß bezahlt werden, und sollte sich auch ein völlig Unschuldiger zum Sühnopfer  hingeben (wo dann freilich die von ihm übernommene Leiden eigentlich nicht Strafe – denn er hat selbst nichts verbrochen – heißen könnten); aus welchen allen zu ersehen ist, daß es nicht eine die Gerechtigkeit verwaltende Person ist, der man diesen Verurteilungsspruch beilegt (denn die würde nicht so sprechen können, ohne anderen unrecht zu tun), sondern daß die bloße Gerechtigkeit, als überschwengliches, einem übersinnlichen Subjekt angedachtes Prinzip, das Recht dieses Wesens bestimme; welches zwar dem Formalen dieses Prinzips gemäß ist, dem Materialen desselben aber, dem Zweck, welcher immer die Glückseligkeit der Menschen ist, widerstreitet. – Denn, bei der etwanigen großen Menge der Verbrecher, die ihr Schuldenregister immer so fortlaufen lassen, würde die Strafgerechtigkeit den Zweck der Schöpfung nicht in der Liebe des Welturhebers (wie man sich doch denken muß), sondern in der strengen Befolgung des Rechts setzen (das Recht selbst zum Zweck machen, der in der Ehre Gottes gesetzt wird), welches, da das letztere (die Gerechtigkeit) nur die einschränkende Bedingung des ersteren (der Gütigkeit) ist, den Prinzipien der praktischen Vernunft zu widersprechen scheint, nach welchen eine Weltschöpfung hätte unterbleiben müssen, die ein, der Absicht ihres Urhebers, die nur Liebe zum Grunde haben kann, so widerstreitendes Produkt geliefert haben würde. 
	
	\unnumberedsection{Pachter (1)} 
	\subsection*{tg441.2.20} 
	\textbf{Source : }Die Metaphysik der Sitten/Erster Teil. Metaphysische Anfangsgründe der Rechtslehre/2. Teil. Das öffentliche Recht/1. Abschnitt. Das Staatsrecht\\  
	
	\noindent\textbf{Paragraphe : }Nur die Fähigkeit der Stimmgebung macht die Qualifikation zum Staatsbürger aus; jene aber setzt die Selbständigkeit dessen im Volk voraus, der nicht bloß Teil des gemeinen Wesens, sondern auch Glied desselben, d.i. aus eigener Willkür in Gemeinschaft mit anderen handelnder  Teil desselben sein will. Die letztere Qualität macht aber die Unterscheidung des aktiven vom passiven Staatsbürger notwendig: obgleich der Begriff des letzteren mit der Erklärung des Begriffs von einem Staatsbürger überhaupt im Widerspruch zu stehen scheint. – Folgende Beispiele können dazu dienen, diese Schwierigkeit zu heben: Der Geselle bei einem Kaufmann, oder bei einem Handwerker; der Dienstbote (nicht der im Dienste des Staats steht); der Unmündige (naturaliter vel civiliter); alles Frauenzimmer, und überhaupt jedermann, der nicht nach eigenem Betrieb, sondern nach der Verfügung anderer (außer der des Staats), genötigt ist, seine Existenz (Nahrung und Schutz) zu erhalten, entbehrt der bürgerlichen Persönlichkeit, und seine Existenz ist gleichsam nur Inhärenz. – Der Holzhacker, den ich auf meinem Hofe anstelle, der Schmied in Indien, der mit seinem Hammer, Amboß und Blasbalg in die Häuser geht, um da in Eisen zu arbeiten, in Vergleichung mit dem europäischen Tischler oder Schmied, der die Produkte aus dieser Arbeit als Ware öffentlich feil stellen kann, der Hauslehrer in Vergleichung mit dem Schulmann, der Zinsbauer in Vergleichung mit dem \match{Pächter} u. dergl. sind bloß Handlanger des gemeinen Wesens, weil sie von anderen Individuen befehligt oder beschützt werden müssen, mithin keine bürgerliche Selbständigkeit besitzen. 
	
	\unnumberedsection{Richtschnur (2)} 
	\subsection*{tg439.2.33} 
	\textbf{Source : }Die Metaphysik der Sitten/Erster Teil. Metaphysische Anfangsgründe der Rechtslehre/1. Teil. Das Privatrecht vom äußeren Mein und Dein überhaupt/3. Hauptstück. Von der subjektiv-bedingten Erwerbung durch den Ausspruch einer öffentlichen Gerichtsbarkeit\\  
	
	\noindent\textbf{Paragraphe : }Hier tritt nun wiederum die rechtlich-gesetzgebende Vernunft mit dem Grundsatz der distributiven Gerechtigkeit ein, die Rechtmäßigkeit des Besitzes, nicht wie sie an sich in Beziehung auf den Privatwillen eines jeden (im natürlichen Zustande), sondern nur wie sie vor einem Gerichtshofe, in einem durch den allgemein-vereinigten Willen entstandenen Zustande (in einem bürgerlichen) abgeurteilt werden würde, zur \match{Richtschnur} anzunehmen: wo alsdann die Übereinstimmung mit den formalen Bedingungen der Erwerbung, die an sich nur ein persönliches Recht begründen, zu Ersetzung der materialen Gründe (welche die Ableitung von dem Seinen eines vorhergehenden prätendierenden Eigentümers begründen) als hinreichend postuliert wird, und ein an sich persönliches Recht, vor einen Gerichtshof gezogen, als ein Sachenrecht gilt, z.B. daß das Pferd, was, auf öffentlichem, durchs Polizeigesetz geordneten Markt, jedermann feilsteht, wenn alle Regeln des Kaufs und Verkaufs genau beobachtet worden, mein Eigentum werde (so doch, daß dem wahren Eigentümer das Recht bleibt, den Verkäufer, wegen seines altern unverwirkten Besitzes, in Anspruch zu nehmen), und mein sonst persönliches Recht in ein Sachenrecht, nach welchem ich das Meine, wo ich es finde, nehmen (vindizieren) darf, verwandelt wird, ohne mich auf die Art, wie der Verkäufer dazu gekommen, einzulassen. 
	
	\subsection*{tg441.2.14} 
	\textbf{Source : }Die Metaphysik der Sitten/Erster Teil. Metaphysische Anfangsgründe der Rechtslehre/2. Teil. Das öffentliche Recht/1. Abschnitt. Das Staatsrecht\\  
	
	\noindent\textbf{Paragraphe : }Ein Staat (civitas) ist die Vereinigung einer Menge von Menschen unter Rechtsgesetzen. So fern diese als Gesetze a priori notwendig, d.i. aus Begriffen des äußeren Rechts überhaupt von selbst folgend (nicht statutarisch) sind, ist seine Form die Form eines Staats überhaupt, d.i. der Staat in der Idee, wie er nach reinen Rechtsprinzipien sein soll, welche jeder wirklichen Vereinigung zu einem gemeinen Wesen (also im Inneren) zur \match{Richtschnur} (norma) dient. 
	
	\unnumberedsection{Trieb (1)} 
	\subsection*{tg471.2.28} 
	\textbf{Source : }Die Metaphysik der Sitten/Zweiter Teil. Metaphysische Anfangsgründe der Tugendlehre/I. Ethische Elementarlehre/I. Teil. Von den Pflichten gegen sich selbst überhaupt/Erstes Buch. Von den vollkommenen Pflichten gegen sich selbst/Erstes Hauptstück. Die Pflicht des Menschen gegen sich selbst, als einem animalischen Wesen\\  
	
	\noindent\textbf{Paragraphe : }So wie die Liebe zum Leben von der Natur zur Erhaltung der Person, so ist die Liebe zum Geschlecht von ihr zur Erhaltung der Art bestimmt; d.i. eine jede von beiden ist 
	Naturzweck, unter welchem man diejenige Verknüpfung der Ursache mit einer Wirkung versteht, in welcher jene, auch ohne ihr dazu einen Verstand beizulegen, diese doch nach der Analogie mit einem solchen, also gleichsam absichtlich Menschen hervorbringend gedacht wird. Es fragt sich nun, ob der Gebrauch des letzteren Vermögens, in Ansehung der Person selbst, die es ausübt, unter einem einschränkenden Pflichtgesetz stehe, oder ob diese, auch ohne jenen Zweck zu beabsichtigen, den Gebrauch ihrer Geschlechtseigenschaften der bloßen tierischen Lust zu widmen befugt sei, ohne damit einer Pflicht gegen sich selbst zuwider zu handeln. – In der Rechtslehre wird bewiesen, daß der Mensch sich einer anderen Person dieser Lust zu – Gefallen, ohne besondere Einschränkung durch einen rechtlichen Vertrag, nicht bedienen könne; wo dann zwei Personen wechselseitig einander verpflichten. Hier aber ist die Frage: ob in Ansehung dieses Genusses eine Pflicht des Menschen gegen sich selbst obwalte, deren Übertretung eine Schändung (nicht bloß Abwürdigung) der Menschheit in seiner eigenen Person sei. Der \match{Trieb} zu jenem wird Fleischeslust (auch Wohllust schlechthin) genannt. Das Laster, welches dadurch erzeugt wird, heißt Unkeuschheit, die Tugend aber, in Ansehung dieser sinnlichen Antriebe, wird Keuschheit genannt, die nun hier als Pflicht des Menschen gegen sich selbst vorgestellt werden soll. Unnatürlich heißt eine Wohllust, wenn der Mensch dazu, nicht durch den wirklichen Gegenstand) sondern durch die Einbildung von demselben, also zweckwidrig, ihn sich selbst schaffend, gereizt wird. Denn sie bewirkt alsdann eine Begierde wider den Zweck der Natur, und zwar einen noch wichtigem, als selbst der der Liebe zum Leben ist, weil dieser nur auf Erhaltung des Individuum, jener aber auf die der ganzen Spezies abzielt. – 
	
	\unnumberedsection{Veredlung (1)} 
	\subsection*{tg472.2.42} 
	\textbf{Source : }Die Metaphysik der Sitten/Zweiter Teil. Metaphysische Anfangsgründe der Tugendlehre/I. Ethische Elementarlehre/I. Teil. Von den Pflichten gegen sich selbst überhaupt/Erstes Buch. Von den vollkommenen Pflichten gegen sich selbst\\  
	
	\noindent\textbf{Paragraphe : }Werdet nicht der Menschen Knechte. – Laßt euer Recht nicht ungeahndet von anderen mit Füßen treten. – Macht keine Schulden, für die ihr nicht volle Sicherheit leistet. – Nehmt nicht Wohltaten an, die ihr entbehren könnt, und seid nicht Schmarotzer, oder Schmeichler, oder gar (was freilich nur im Grad von dem Vorigen unterschieden ist) Bettler. Daher seid wirtschaftlich, damit ihr nicht bettelarm werdet. – Das Klagen und Winseln, selbst das bloße Schreien bei einem körperlichen Schmerz ist euer schon unwert, am meisten, wenn ihr euch bewußt seid, ihn selbst verschuldet zu haben: Daher die \match{Veredlung} (Abwendung der Schmach) des Todes eines Delinquenten durch die Standhaftigkeit, mit der er stirbt. – Das Hinknien oder Hinwerfen zur Erde, selbst um die Verehrung himmlischer Gegenstände sich dadurch zu versinnlichen, ist der Menschenwürde zuwider, so wie die Anrufung derselben in gegenwärtigen Bildern; denn ihr demütigt euch alsdann nicht unter einem Ideal, das euch eure eigene Vernunft vorstellt, sondern unter einem Idol, was euer eigenes Gemächsel ist. 
	
	\unnumberedsection{Vermehrung (1)} 
	\subsection*{tg461.2.4} 
	\textbf{Source : }Die Metaphysik der Sitten/Zweiter Teil. Metaphysische Anfangsgründe der Tugendlehre/Einleitung/XIII. Allgemeine Grundsätze der Metaphysik der Sitten in Behandlung einer reinen Tugendlehre\\  
	
	\noindent\textbf{Paragraphe : }
	Zweitens. Der Unterschied der Tugend vom Laster kann nie in Graden der Befolgung gewisser Maximen, sondern muß allein in der spezifischen Qualität derselben (dem Verhältnis zum Gesetz) gesucht werden; mit andern Worten, der belobte Grundsatz (des Aristoteles), die Tugend in dem Mittleren zwischen zwei Lastern zu setzen, ist falsch.
	
	
	17
	Es sei z.B. gute Wirtschaft, als das Mittlere
	zwischen zwei Lastern, Verschwendung und Geiz, gegeben: so kann sie als Tugend nicht durch die allmähliche Verminderung des ersten beider genannten Laster (Ersparung), noch durch die \match{Vermehrung} der Ausgaben, des dem letzteren Ergebenen, als entspringend vorgestellt werden: indem sie sich gleichsam nach entgegengesetzten Richtungen in der guten Wirtschaft begegneten: sondern eine jede derselben hat ihre eigene Maxime, die der andern notwendig widerspricht. 
	
	\unnumberedsection{Vermischung (1)} 
	\subsection*{tg442.2.4} 
	\textbf{Source : }Die Metaphysik der Sitten/Erster Teil. Metaphysische Anfangsgründe der Rechtslehre/2. Teil. Das öffentliche Recht/2. Abschnitt. Das Völkerrecht\\  
	
	\noindent\textbf{Paragraphe : }Die Menschen, welche ein Volk ausmachen, können, als Landeseingeborne, nach der Analogie der Erzeugung von einem gemeinschaftlichen Elternstamm (congeniti) vorgestellt werden, ob sie es gleich nicht sind: dennoch aber, in intellektueller und rechtlicher Bedeutung, als von einer gemeinschaftlichen Mutter (der Republik) geboren, gleichsam eine Familie (gens, natio) ausmachen, deren Glieder (Staatsbürger) alle ebenbürtig sind, und mit denen, die neben ihnen im Naturzustande leben möchten, als unedlen keine \match{Vermischung} eingehen, obgleich diese (die Wilden) ihrerseits sich wiederum wegen der gesetzlosen Freiheit, die sie gewählt haben, sich vornehmer dünken, die gleichfalls Völkerschaften, aber nicht Staaten, ausmachen. Das Recht der Staaten in Verhältnis zu einander (welches nicht ganz richtig im Deutschen das Völkerrecht genannt wird, sondern vielmehr das Staatenrecht (ius publicum civitatum) heißen sollte) ist nun dasjenige, was wir unter dem Namen des Völkerrechts zu betrachten haben: wo ein Staat, als eine moralische Person, gegen einen anderen im Zustande der natürlichen Freiheit, folglich auch dem des beständigen Krieges betrachtet, teils das Recht zum Kriege, teils das im Kriege, teils das, einander zu nötigen, aus diesem Kriegszustande herauszugehen, mithin eine den beharrlichen Frieden gründende Verfassung, d.i. das Recht nach dem Kriege zur Aufgabe macht, und führt nur das Unterscheidende von dem des Naturzustandes einzelner Menschen oder Familien (im Verhältnis gegen einander) von dem der Völker bei sich, daß im Völkerrecht nicht bloß ein Verhältnis eines Staats gegen den anderen im ganzen, sondern auch einzelner Personen des einen gegen einzelne des anderen, imgleichen gegen den ganzen anderen Staat selbst in Betrachtung kommt; welcher Unterschied aber vom Recht einzelner im  bloßen Naturzustande nur solcher Bestimmungen bedarf, die sich aus dem Begriffe des letzteren leicht folgern lassen. 
	
	\unnumberedsection{Vieh (1)} 
	\subsection*{tg471.2.29} 
	\textbf{Source : }Die Metaphysik der Sitten/Zweiter Teil. Metaphysische Anfangsgründe der Tugendlehre/I. Ethische Elementarlehre/I. Teil. Von den Pflichten gegen sich selbst überhaupt/Erstes Buch. Von den vollkommenen Pflichten gegen sich selbst/Erstes Hauptstück. Die Pflicht des Menschen gegen sich selbst, als einem animalischen Wesen\\  
	
	\noindent\textbf{Paragraphe : }Daß ein solcher naturwidrige Gebrauch (also Mißbrauch) seiner Geschlechtseigenschaft eine und zwar der Sittlichkeit im höchsten Grad widerstreitende Verletzung der Pflicht wider sich selbst sei, fällt jedem, zugleich mit dem Gedanken von demselben, so fort auf, erregt eine Abkehrung von diesem Gedanken, in der Maße, daß selbst die Nennung  eines solchen Lasters bei seinem eigenen Namen für unsittlich gehalten wird; welches, bei dem des Selbstmords, nicht geschieht, den man, mit allen seinen Greueln (in einer species facti) der Welt vor Augen zu legen im mindesten kein Bedenken trägt; gleich als ob der Mensch überhaupt sich beschämt fühle, einer solchen ihn selbst unter das \match{Vieh} herabwürdigenden Behandlung seiner eigenen Person fähig zu sein: so daß selbst die erlaubte (an sich freilich bloß tierische) körperliche Gemeinschaft beider Geschlechter in der Ehe im gesitteten Umgange viel Feinheit veranlaßt und erfodert, um einen Schleier darüber zu werfen, wenn davon gesprochen werden soll. 
	
	\unnumberedchapter{Alimentation} 
	\unnumberedsection{Alter (1)} 
	\subsection*{tg472.2.29} 
	\textbf{Source : }Die Metaphysik der Sitten/Zweiter Teil. Metaphysische Anfangsgründe der Tugendlehre/I. Ethische Elementarlehre/I. Teil. Von den Pflichten gegen sich selbst überhaupt/Erstes Buch. Von den vollkommenen Pflichten gegen sich selbst\\  
	
	\noindent\textbf{Paragraphe : }Da hier nur von Pflichten gegen sich selbst die Rede ist und Habsucht (Unersättlichkeit im Erwerb), um zu verschwenden, eben so wohl als Knauserei (Peinlichkeit im Vertun), Selbstsucht (solipsismus) zum Grunde haben, und beide, die Verschwendung so wohl als die Kargheit, bloß darum verwerflich zu sein scheinen, weil sie auf Armut hinaus laufen, bei dem einen auf nicht erwartete, bei dem anderen auf willkürliche (armselig leben zu wollen), – so ist die Frage: ob sie, die eine so wohl als die andere, überhaupt Laster und nicht vielmehr beide bloße Unklugheit genannt werden sollen, mithin nicht ganz und gar außerhalb den Grenzen der Pflicht gegen sich selbst liegen mögen. Die Kargheit aber ist nicht bloß mißverstandene Sparsamkeit, sondern sklavische Unterwerfung seiner selbst unter die Glücksgüter, ihrer nicht Herr zu sein, welches Verletzung der Pflicht gegen sich selbst ist. Sie ist der Liberalität (liberalitas moralis) der Denkungsart überhaupt (nicht der Freigebigkeit (liberalitas sumtuosa), welche nur eine Anwendung derselben auf einen besonderen Fall ist), d.i. dem Prinzip der Unabhängigkeit von allem anderen, außer von dem Gesetz, entgegengesetzt, und Defraudation, die das Subjekt an sich selbst begeht. Aber was ist das für ein Gesetz, dessen innerer Gesetzgeber selbst nicht weiß, wo es anzuwenden ist? Soll ich meinem Munde abbrechen, oder nur dem äußeren Aufwande? im \match{Alter} oder schon in der Jugend? oder ist Sparsamkeit überhaupt eine Tugend? 
	
	\unnumberedsection{Bezeichnung (4)} 
	\subsection*{tg434.2.7} 
	\textbf{Source : }Die Metaphysik der Sitten/Erster Teil. Metaphysische Anfangsgründe der Rechtslehre/1. Teil. Das Privatrecht vom äußeren Mein und Dein überhaupt\\  
	
	\noindent\textbf{Paragraphe : }Die Momente (attendenda) der ursprünglichen Erwerbung sind also: 1) die Apprehension eines Gegenstandes, der keinem angehört, widrigenfalls sie der Freiheit anderer nach allgemeinen Gesetzen widerstreiten würde. Diese Apprehension ist die Besitznehmung des Gegenstandes  der Willkür im Raum und der Zeit; der Besitz also, in den ich mich setze, ist (possessio phaenomenon). 2) Die \match{Bezeichnung} (declaratio) des Besitzes dieses Gegenstandes und des Akts meiner Willkür, jeden anderen davon abzuhalten. 3) Die Zueignung (appropriatio) als Akt eines äußerlich allgemein gesetzgebenden Willens (in der Idee), durch welchen jedermann zur Einstimmung mit meiner Willkür verbunden wird. – Die Gültigkeit des letzteren Moments der Erwerbung, als worauf der Schlußsatz: der äußere Gegenstand ist mein, beruht, d.i. daß der Besitz, als ein bloß-rechtlicher, gültig (possessio noumenon) sei, gründet sich darauf: daß, da alle diese Actus rechtlich sind, mithin aus der praktischen Vernunft hervorgehen, und also in der Frage, was Rechtens ist, von den empirischen Bedingungen des Besitzes abstrahiert werden kann, der Schlußsatz: der äußere Gegenstand ist mein, vom sensibelen auf den intelligibelen Besitz richtig geführt wird. 
	
	\subsection*{tg437.2.88} 
	\textbf{Source : }Die Metaphysik der Sitten/Erster Teil. Metaphysische Anfangsgründe der Rechtslehre/1. Teil. Das Privatrecht vom äußeren Mein und Dein überhaupt/2. Hauptstück. Von der Art, etwas Äußeres zu erwerben/3. Abschnitt. Von dem auf dingliche Art persönlichen Recht\\  
	
	\noindent\textbf{Paragraphe : }
	Schrift ist nicht unmittelbar \match{Bezeichnung} eines Begriffs (wie etwa ein Kupferstich, der als Porträt, oder ein Gipsabguß, der als die Büste eine bestimmte Person vorstellt) sondern eine Rede ans Publikum, d.i. der Schriftsteller spricht durch den Verleger öffentlich. – Dieser aber, nämlich der Verleger, spricht (durch seinen Werkmeister, operarius, den Drucker) nicht in seinem eigenen Namen (denn sonst würde er sich für den Autor ausgeben), sondern im Namen des Schriftstellers, wozu er also nur durch eine ihm von dem letzteren erteilte Vollmacht (mandatum) berechtigt ist. – Nun spricht der Nachdrucker durch seinen eigenmächtigen Verlag zwar auch im Namen des Schriftstellers, aber ohne dazu Vollmacht von demselben zu haben (gerit se mandatarium absque mandato); folglich begeht er an dem von dem Autor bestellten (mithin einzig rechtmäßigen)  Verleger ein Verbrechen der Entwendung des Vorteils, den der letztere aus dem Gebrauch seines Rechtsziehen konnte und wollte (furtum usus); also ist der Büchernachdruck von rechtswegen verboten. 
	
	\subsection*{tg484.2.4} 
	\textbf{Source : }Die Metaphysik der Sitten/Zweiter Teil. Metaphysische Anfangsgründe der Tugendlehre/I. Ethische Elementarlehre/II. Teil. Von den Tugendpflichten gegen andere/Beschluß der Elementarlehre. Von der innigsten Vereinigung der Liebe mit der Achtung in der Freundschaft\\  
	
	\noindent\textbf{Paragraphe : }
	Freundschaft (in ihrer Vollkommenheit betrachtet) ist die Vereinigung zweier Personen durch gleiche wechselseitige Liebe und Achtung. – Man sieht leicht, daß sie ein Ideal der Teilnehmung und Mitteilung an dem Wohl eines jeden dieser durch den moralisch guten Willen Vereinigten sei, und, wenn es auch nicht das ganze Glück des Lebens bewirkt, die Aufnahme desselben in ihre beiderseitige Gesinnung die Würdigkeit enthalte, glücklich zu sein, mithin daß Freundschaft unter Menschen Pflicht derselben ist. – Daß aber Freundschaft eine bloße (aber doch praktisch-notwendige) Idee, in der Ausübung zwar unerreichbar, aber doch darnach (als einem Maximum der guten Gesinnung  gegen einander) zu streben von der Vernunft aufgegebene, nicht etwa gemeine, sondern ehrenvolle Pflicht sei, ist leicht zu ersehen. Denn, wie ist es für den Menschen in Verhältnis zu seinem Nächsten möglich, die Gleichheit eines der dazu erforderlichen Stücke eben derselben Pflicht (z.B. des wechselseitigen Wohlwollens) in dem einen, mit eben derselben Gesinnung im anderen auszumitteln, noch mehr aber, welches Verhältnis das Gefühl aus der einen Pflicht zu dem aus der andern (z.B. das aus dem Wohlwollen, zu dem aus der Achtung) in derselben Person habe, und ob, wenn die eine in der Liebe inbrünstiger ist, sie nicht eben dadurch in der Achtung des anderen etwas einbüße, so daß beiderseitig Liebe und Hochschätzung subjektiv schwerlich in das Ebenmaß des Gleichgewichts gebracht werden wird; welches doch zur Freundschaft erforderlich ist? – Denn man kann jene als Anziehung, diese als Abstoßung betrachten, und wenn das Prinzip der ersteren Annäherung gebietet, das der zweiten sich einander in geziemendem Abstande zu halten fordert; welche Einschränkung der Vertraulichkeit, durch die Regel: daß auch die besten Freunde sich unter einander nicht gemein machen sollen, ausgedrückt, eine Maxime enthält, die nicht bloß dem Höheren gegen den Niedrigen, sondern auch umgekehrt gilt. Denn der Höhere fühlt, ehe man es sich versieht, seinen Stolz gekränkt und will die Achtung des Niedrigen, etwa für einen Augenblick aufgeschoben, nicht aber aufgehoben wissen, welche aber einmal verletzt, innerlich unwiderbringlich verloren ist; wenn gleich die äußere \match{Bezeichnung} derselben (das Zeremoniell) wieder in den alten Gang gebracht wird. 
	
	\subsection*{tg484.2.7} 
	\textbf{Source : }Die Metaphysik der Sitten/Zweiter Teil. Metaphysische Anfangsgründe der Tugendlehre/I. Ethische Elementarlehre/II. Teil. Von den Tugendpflichten gegen andere/Beschluß der Elementarlehre. Von der innigsten Vereinigung der Liebe mit der Achtung in der Freundschaft\\  
	
	\noindent\textbf{Paragraphe : }Ein Freund in der Not, wie erwünscht ist er nicht (wohl zu verstehen, wenn er ein tätiger, mit eigenem Aufwande hülfreicher Freund ist)? Aber es ist doch auch eine große Last, sich an anderer ihrem Schicksal angekettet und mit fremden Bedürfnis beladen zu fühlen. – Die Freundschaft kann also nicht eine auf wechselseitigen Vorteil abgezweckte Verbindung, sondern diese muß rein moralisch sein, und der Beistand, auf den jeder von beiden von dem anderen im Falle der Not rechnen darf, muß nicht als Zweck und Bestimmungsgrund zu derselben – dadurch würde er die Achtung des andern Teils verlieren – sondern kann nur als äußere \match{Bezeichnung} des inneren herzlich gemeinten Wohlwollens, ohne es doch auf die Probe, als die immer gefährlich ist, ankommen zu lassen, gemeint sein, indem ein jeder großmütig den anderen dieser Last zu überheben, sie für sich allein zu tragen, ja ihm sie gänzlich zu verhehlen bedacht ist, sich aber immer doch damit schmeicheln kann, daß im Falle der Not er auf den Beistand des andern sicher würde rechnen können. Wenn aber einer von dem andern eine Wohltat annimmt, so kann er wohl vielleicht auf Gleichheit in der Liebe, aber nicht in der Achtung rechnen, denn er sieht sich offenbar eine Stufe niedriger, verbindlich zu sein und nicht gegenseitig verbinden zu können. – Freundschaft ist, bei der Süßigkeit der Empfindung des bis zum Zusammenschmelzen in eine Person sich annähernden wechselseitigen Besitzes, doch zugleich etwas so Zartes (teneritas amicitiae), daß, wenn man sie auf Gefühle beruhen läßt, und dieser wechselseitigen Mitteilung und Ergebung nicht Grundsätze oder das Gemeinmachen verhütende, und die Wechselliebe durch Foderungen der Achtung einschränkende  Regeln unterlegt, sie keinen Augenblick vor Unterbrechungen sicher ist; dergleichen unter unkultivierten Personen gewöhnlich sind, ob sie zwar darum eben nicht immer Trennung bewirken (denn Pöbel schlägt sich und Pöbel verträgt sich); sie können von einander nicht lassen, aber sich auch nicht unter einander einigen, weil das Zanken selbst ihnen Bedürfnis ist, um die Süßigkeit der Eintracht in der Versöhnung zu schmecken. – Auf alle Fälle aber kann die Liebe in der Freundschaft nicht Affekt sein; weil dieser in der Wahl blind und in der Fortsetzung verrauchend ist. 
	
	\unnumberedsection{Ernahrung (1)} 
	\subsection*{tg437.2.32} 
	\textbf{Source : }Die Metaphysik der Sitten/Erster Teil. Metaphysische Anfangsgründe der Rechtslehre/1. Teil. Das Privatrecht vom äußeren Mein und Dein überhaupt/2. Hauptstück. Von der Art, etwas Äußeres zu erwerben/3. Abschnitt. Von dem auf dingliche Art persönlichen Recht\\  
	
	\noindent\textbf{Paragraphe : }Aus dieser Pflicht entspringt auch notwendig das Recht der Eltern zur Handhabung und Bildung des Kindes, so lange es des eigenen Gebrauchs seiner Gliedmaßen, imgleichen des Verstandesgebrauchs, noch nicht mächtig ist, außer der \match{Ernährung} und Pflege es zu erziehen, und sowohl pragmatisch, damit es künftig sich selbst erhalten und fortbringen  könne, als auch moralisch, weil sonst die Schuld ihrer Verwahrlosung auf die Eltern fallen würde, – es zu bilden; alles bis zur Zeit der Entlassung (emancipatio), da diese, sowohl ihrem väterlichen Recht zu befehlen, als auch allem Anspruch auf Kostenerstattung für ihre bisherige Verpflegung und Mühe entsagen, wofür, und nach vollendeter Erziehung, sie der Kinder ihre Verbindlichkeit (gegen die Eltern) nur als bloße Tugendpflicht, nämlich als Dankbarkeit, in Anschlag bringen können. 
	
	\unnumberedsection{Fleisch (1)} 
	\subsection*{tg464.2.2} 
	\textbf{Source : }Die Metaphysik der Sitten/Zweiter Teil. Metaphysische Anfangsgründe der Tugendlehre/Einleitung/XVI. Zur Tugend wird Apathie (als Stärke betrachtet) notwendig vorausgesetzt\\  
	
	\noindent\textbf{Paragraphe : }Dieses Wort ist, gleich als ob es Fühllosigkeit, mithin subjektive Gleichgültigkeit in Ansehung der Gegenstände der Willkür, bedeutete, in übelen Ruf gekommen; man nahm es für Schwäche. Dieser Mißdeutung kann dadurch vorgebeugt werden, daß man diejenige Affektlosigkeit, welche von der Indifferenz zu unterscheiden ist, die moralische Apathie nennt: da die Gefühle aus sinnlichen Eindrücken ihren Einfluß auf das moralische nur dadurch verlieren, daß die Achtung fürs Gesetz über sie insgesamt mächtiger wird. – Es ist nur die scheinbare Stärke eines Fieberkranken, die den lebhaften Anteil selbst am Guten bis zum Affekt steigen, oder vielmehr darin ausarten läßt. Man nennt den Affekt dieser Art Enthusiasm, und dahin ist auch die Mäßigung zu deuten, die man selbst für Tugendausübungen zu empfehlen pflegt (insani sapiens nomen habeat aequus iniqui – ultra, quam satis est virtutem si petat ipsam. Horat.). Denn sonst ist es ungereimt zu wähnen, man könne auch wohl allzuweise, allzutugendhaft sein. Der Affekt gehört immer zur Sinnlichkeit; er mag durch einen Gegenstand erregt werden, welcher es wolle. Die wahre Stärke der Tugend ist das Gemüt in Ruhe, mit einer überlegten und festen Entschließung, ihr Gesetz in Ausübung zu bringen. Das ist der Zustand der Gesundheit im moralischen Leben; dagegen der Affekt, selbst wenn er durch die Vorstellung des Guten aufgeregt wird, eine augenblicklich glänzende Erscheinung ist, welche Mattigkeit hinterläßt. – Phantastisch-tugendhaft aber kann doch der genannt werden, der keine in Ansehung der Moralität gleichgültige Dinge (adiaphora) einräumt und sich alle seine Schritte und Tritte mit Pflichten als mit Fußangeln bestreut und es nicht gleichgültig findet, ob ich mich mit \match{Fleisch} oder Fisch, mit Bier oder Wein, wenn mir beides bekömmt, nähre; eine Mikrologie, welche, wenn man sie in die Lehre der Tugend aufnähme, die Herrschaft derselben zur Tyrannei machen würde. 
	
	\unnumberedsection{Fuß (2)} 
	\subsection*{tg441.2.59} 
	\textbf{Source : }Die Metaphysik der Sitten/Erster Teil. Metaphysische Anfangsgründe der Rechtslehre/2. Teil. Das öffentliche Recht/1. Abschnitt. Das Staatsrecht\\  
	
	\noindent\textbf{Paragraphe : }Da auch das Kirchenwesen, welches von der Religion, als innerer Gesinnung, die ganz außer dem Wirkungskreise der bürgerlichen Macht ist, sorgfältig unterschieden werden muß (als Anstalt zum öffentlichen Gottesdienst für das Volk, aus welchem dieser auch seinen Ursprung hat, es sei Meinung oder Überzeugung), ein wahres Staatsbedürfnis  wird, sich auch als Untertanen einer höchsten unsichtbaren Macht, der sie huldigen müssen, und die mit der bürgerlichen oft in einen sehr ungleichen Streit kommen kann, zu betrachten: so hat der Staat das Recht, nicht etwa der inneren Konstitutionalgesetzgebung, das Kirchenwesen nach seinem Sinne, wie es ihm vorteilhaft dünkt, einzurichten, den Glauben und gottesdienstliche Formen (ritus) dem Volk vorzuschreiben, oder zu befehlen (denn dieses muß gänzlich den Lehrern und Vorstehern, die es sich selbst gewählt hat, überlassen bleiben), sondern nur das negative Recht, den Einfluß der öffentlichen Lehrer auf das sichtbare, politische gemeine Wesen, der der öffentlichen Ruhe nachteilig sein möchte, abzuhalten, mithin bei dem inneren Streit, oder dem der verschiedenen Kirchen unter einander, die bürgerliche Eintracht nicht in Gefahr kommen zu lassen, welches also ein Recht der Polizei ist. Daß eine Kirche einen gewissen Glauben, und welchen sie haben, oder daß sie ihn unabänderlich erhalten müsse, und sich nicht selbst reformieren dürfe, sind Einmischungen der obrigkeitlichen Gewalt, die unter ihrer Würde sind; weil sie sich dabei, als einem Schulgezänke, auf den \match{Fuß} der Gleichheit mit ihren Untertanen einläßt (der Monarch sich zum Priester macht), die ihr geradezu sagen können, daß sie hievon nichts verstehe; vornehmlich was des letztere, nämlich das Verbot innerer Reformen, betrifft; – denn, was das gesamte Volk nicht über sich selbst beschließen kann, das kann auch der Gesetzgeber nicht über das Volk beschließen. Nun kann aber kein Volk beschließen, in seinen den Glauben betreffenden Einsichten (der Aufklärung) niemals weiter fortzuschreiten, mithin auch sich in Ansehung des Kirchenwesens nie zu reformieren; weil dies der Menschheit in seiner eigenen Person, mithin dem höchsten Rechte desselben entgegen sein würde. Also kann es auch keine obrigkeitliche Gewalt über das Volk beschließen. – – Was aber die Kosten der Erhaltung des Kirchenwesens betrifft, so können diese, aus ebenderselben Ursache, nicht dem Staat, sondern müssen dem Teil des Volks, der sich zu einem oder dem anderen Glauben bekennt, d.i. nur der Gemeine zu Lasten kommen. 
	
	\subsection*{tg472.2.34} 
	\textbf{Source : }Die Metaphysik der Sitten/Zweiter Teil. Metaphysische Anfangsgründe der Tugendlehre/I. Ethische Elementarlehre/I. Teil. Von den Pflichten gegen sich selbst überhaupt/Erstes Buch. Von den vollkommenen Pflichten gegen sich selbst\\  
	
	\noindent\textbf{Paragraphe : }Allein der Mensch als Person betrachtet, d.i. als Subjekt einer moralisch-praktischen Vernunft, ist über allen Preis erhaben; denn als ein solcher (homo noumenon) ist er nicht bloß als Mittel zu anderer ihren, ja selbst seinen eigenen Zwecken, sondern als Zweck an sich seihst zu schätzen, d.i. er besitzt eine Würde (einen absoluten innern Wert), wodurch er allen andern vernünftigen Weltwesen Achtung für ihn abnötigt, sich mit jedem anderen dieser Art messen und auf den \match{Fuß} der Gleichheit schätzen kann. 
	
	\unnumberedsection{Gericht (10)} 
	\subsection*{tg431.2.39} 
	\textbf{Source : }Die Metaphysik der Sitten/Erster Teil. Metaphysische Anfangsgründe der Rechtslehre/Einleitung in die Rechtslehre\\  
	
	\noindent\textbf{Paragraphe : }Es ist klar: daß diese Behauptung nicht objektiv, nach dem, was ein Gesetz vorschreiben, sondern bloß subjektiv, wie vor \match{Gericht} die Sentenz gefället werden würde, zu verstehen sei. Es kann nämlich kein Strafgesetz geben, welches demjenigen den Tod zuerkennete, der im Schiffbruche, mit einem andern in gleicher Lebensgefahr schwebend, diesen von dem Brette, worauf er sich gerettet hat, wegstieße, um sich selbst zu retten. Denn die durchs Gesetz angedrohete Strafe könnte doch nicht größer sein, als die des Verlusts des Lebens des ersteren. Nun kann ein solches Strafgesetz die beabsichtigte Wirkung gar nicht haben; denn die Bedrohung mit einem Übel, was noch ungewiß ist (dem Tode durch den richterlichen Ausspruch), kann die Furcht vor dem Übel, was gewiß ist (nämlich dem Ersaufen), nicht überwiegen. Also ist die Tat der gewalttätigen Selbsterhaltung nicht etwa als unsträflich (inculpabile), sondern nur als unstrafbar (inpunibile) zu beurteilen und diese subjektive Straflosigkeit wird, durch eine wunderliche Verwechselung, von den Rechtslehrern für eine objektive (Gesetzmäßigkeit) gehalten. 
	
	\subsection*{tg439.2.12} 
	\textbf{Source : }Die Metaphysik der Sitten/Erster Teil. Metaphysische Anfangsgründe der Rechtslehre/1. Teil. Das Privatrecht vom äußeren Mein und Dein überhaupt/3. Hauptstück. Von der subjektiv-bedingten Erwerbung durch den Ausspruch einer öffentlichen Gerichtsbarkeit\\  
	
	\noindent\textbf{Paragraphe : }Dieser Vertrag (donatio), wodurch ich das Mein, meine Sache (oder mein Recht) unvergolten (gratis) veräußere, enthält ein Verhältnis von mir, dem Schenkenden (donans), zu einem anderen, dem Beschenkten (donatarius), nach dem Privatrecht, wodurch das Meine auf diesen durch Annehmung des letzteren (donum) übergeht. – Es ist aber nicht zu präsumieren, daß ich hiebei gemeinet sei, zu der Haltung meines Versprechens gezwungen zu werden, und also auch meine Freiheit umsonst wegzugeben, und gleichsam mich selbst wegzuwerfen (nemo suum iactare praesumitur), welches doch nach dem Recht im bürgerlichen Zustande geschehen würde; denn da kann der Zubeschenkende mich zu Leistung des Versprechens zwingen. Es müßte also, wenn die Sache vor \match{Gericht} käme, d.i. nach einem öffentlichen Recht, entweder präsumiert werden, der  Verschenkende willigte zu diesem Zwange ein, welches ungereimt ist, oder der Gerichtshof sehe in seinem Spruch (Sentenz) gar nicht darauf, ob jener die Freiheit, von seinem Versprechen abzugehen, hat vorbehalten wollen, oder nicht, sondern auf das, was gewiß ist, nämlich das Versprechen und die Akzeptation des Promissars. Wenn also gleich der Promittent, wie wohl vermutet werden kann, gedacht hat, daß, wenn es ihn noch vor der Erfüllung gereuet, das Versprechen getan zu haben, man ihn daran nicht binden könne: so nimmt doch das Gericht an, daß er sich dieses ausdrücklich hätte vorbehalten müssen, und, wenn er es nicht getan hat, zu Erfüllung des Versprechens könne gezwungen werden, und dieses Prinzip nimmt der Gerichtshof darum an, weil ihm sonst das Rechtsprechen unendlich erschwert, oder gar unmöglich gemacht werden würde. 
	
	\subsection*{tg439.2.20} 
	\textbf{Source : }Die Metaphysik der Sitten/Erster Teil. Metaphysische Anfangsgründe der Rechtslehre/1. Teil. Das Privatrecht vom äußeren Mein und Dein überhaupt/3. Hauptstück. Von der subjektiv-bedingten Erwerbung durch den Ausspruch einer öffentlichen Gerichtsbarkeit\\  
	
	\noindent\textbf{Paragraphe : }Da nun über das Mein und Dein aus dem Leihvertrage, wenn (wie es die Natur dieses Vertrages so mit sich bringt) über die mögliche Verunglückung (casus), die die Sache treffen möchte, nichts verabredet worden, er also, weil die  Einwilligung nur präsumiert worden, ein ungewisser Vertrag (pactum incertum) ist, das Urteil darüber, d.i. die Entscheidung, wen das Unglück treffen müsse, nicht aus den Bedingungen des Vertrages an sich selbst, sondern wie sie allein vor einem Gerichtshofe, der immer nur auf das Gewisse in jenem sieht (welches hier der Besitz der Sache als Eigentum ist), entschieden werden kann, so wird das Urteil im Naturzustande, d.i. nach der Sache innerer Beschaffenheit, so lauten: der Schade aus der Verunglückung einer geliehenen Sache fällt auf den Beliehenen (casum sentit commodatarius), dagegen im bürgerlichen, also vor einem Gerichtshofe, wird die Sentenz so ausfallen: der Schade fällt auf den Anleiher (casum sentit dominus), und zwar aus dem Grunde verschieden von dem Ausspruche der bloßen gesunden Vernunft, weil ein öffentlicher Richter sich nicht auf Präsumtionen von dem, was der eine oder andere Teil gedacht haben mag, einlassen kann, sondern der, welcher sich nicht die Freiheit von allem Schaden an der geliehenen Sache durch einen besonderen angehängten Vertrag ausbedungen hat, diesen selbst tragen muß. – Also ist der Unterschied zwischen dem Urteile, wie es ein \match{Gericht} fällen müßte, und dem, was die Privatvernunft eines jeden für sich zu fällen berechtigt ist, ein durchaus nicht zu übersehender Punkt in Berichtigung der Rechtsurteile. 
	
	\subsection*{tg439.2.37} 
	\textbf{Source : }Die Metaphysik der Sitten/Erster Teil. Metaphysische Anfangsgründe der Rechtslehre/1. Teil. Das Privatrecht vom äußeren Mein und Dein überhaupt/3. Hauptstück. Von der subjektiv-bedingten Erwerbung durch den Ausspruch einer öffentlichen Gerichtsbarkeit\\  
	
	\noindent\textbf{Paragraphe : }Man kann keinen anderen Grund angeben, der rechtlich Menschen verbinden könnte, zu glauben und zu bekennen, daß es Götter gebe, als den, damit sie einen Eid schwören, und durch die Furcht vor einer allsehenden obersten Macht, deren Rache sie feierlich gegen sich aufrufen mußten, im Fall, daß ihre Aussage falsch wäre, genötigt werden könnten, wahrhaft im Aussagen und treu im Versprechen zu sein. Daß man hiebei nicht auf die Moralität dieser beiden Stücke, sondern bloß auf einen blinden Aberglauben derselben rechnete, ist daraus zu ersehen, daß man sich von ihrer bloßen feierlichen Aussage vor \match{Gericht} in Rechtssachen keine Sicherheit versprach, ob gleich die Pflicht der Wahrhaftigkeit in einem Fall, wo es auf das Heiligste, was unter Menschen nur sein kann (aufs Recht der Menschen), an kommt, jedermann so klar einleuchtet, mithin bloße Märchen den Bewegungsgrund ausmachen: wie z.B. das unter den Rejangs, einem heidnischen Volk auf Sumatra, welche, nach Marsdens Zeugnis, bei den Knochen ihrer verstorbenen Anverwandten schwören, ob sie gleich gar nicht glauben, daß es noch ein Leben nach dem Tode gebe, oder der Eid der Guineaschwarzen bei ihrem Fetisch, etwa einer Vogelfeder, auf die sie sich vermessen, daß sie ihnen den Hals brechen solle u. dergl. Sie glauben, daß eine unsichtbare  Macht, sie mag nun Verstand haben oder nicht, schon ihrer Natur nach, diese Zauberkraft habe, die durch einen solchen Aufruf in Tat versetzt wird. – Ein solcher Glaube, dessen Name Religion ist, eigentlich aber Superstition heißen sollte, ist aber für die Rechtsverwaltung unentbehrlich, weil, ohne auf ihn zu rechnen, der Gerichtshof nicht genugsam im Stande wäre, geheim gehaltene Facta auszumitteln, und Recht zu sprechen. Ein Gesetz, das hiezu verbindet, ist also offenbar nur zum Behuf der richtenden Gewalt gegeben. 
	
	\subsection*{tg439.2.38} 
	\textbf{Source : }Die Metaphysik der Sitten/Erster Teil. Metaphysische Anfangsgründe der Rechtslehre/1. Teil. Das Privatrecht vom äußeren Mein und Dein überhaupt/3. Hauptstück. Von der subjektiv-bedingten Erwerbung durch den Ausspruch einer öffentlichen Gerichtsbarkeit\\  
	
	\noindent\textbf{Paragraphe : }Aber nun ist die Frage: worauf gründet man die Verbindlichkeit, die jemand vor \match{Gericht} haben soll, eines anderen Eid als zu Recht gültigen Beweisgrund der Wahrheit seines Vorgebens anzunehmen, der allem Hader ein Ende mache, d.i. was verbindet mich rechtlich, zu glauben, daß ein anderer (der Schwörende) überhaupt Religion habe, um mein Recht auf seinen Eid ankommen zu lassen? Imgleichen umgekehrt: kann ich überhaupt verbunden werden, zu schwören? Beides ist an sich unrecht. 
	
	\subsection*{tg439.2.40} 
	\textbf{Source : }Die Metaphysik der Sitten/Erster Teil. Metaphysische Anfangsgründe der Rechtslehre/1. Teil. Das Privatrecht vom äußeren Mein und Dein überhaupt/3. Hauptstück. Von der subjektiv-bedingten Erwerbung durch den Ausspruch einer öffentlichen Gerichtsbarkeit\\  
	
	\noindent\textbf{Paragraphe : }Wenn die Amtseide, welche gewöhnlich promissorisch sind, daß man nämlich den ernstlichen Vorsatz habe, sein Amt pflichtmäßig zu verwalten, in assertorische verwandelt würden, daß nämlich der Beamte etwa zu Ende eines Jahres (oder mehrerer) verbunden  wäre, die Treue seiner Amtsführung während desselben zu beschwören; so würde dieses teils das Gewissen mehr in Bewegung bringen, als der Versprechungseid, welcher hinterher noch immer den inneren Vorwand übrig läßt, man habe, bei dem besten Vorsatz, die Beschwerden nicht voraus gesehen, die man nur nachher während der Amtsverwaltung erfahren habe, und die Pflichtübertretungen würden auch, wenn ihre Summierung durch Aufmerker bevorstände, mehr Besorgnis der Anklage wegen erregen, als wenn sie bloß eine nach der anderen (über welche die vorigen vergessen sind) gerügt würden. – Was aber das Beschwören des Glaubens (de credulitate) betrifft, so kann dieses gar nicht von einem \match{Gericht} verlangt werden. Denn erstlich enthält es in sich selbst einen Widerspruch: dieses Mittelding zwischen Meinen und Wissen, weil es so etwas ist, worauf man wohl zu wetten, keinesweges aber darauf zu schwören sich getrauen kann. Zweitens begeht der Richter, der solchen Glaubenseid dem Parten ansinnete, um etwas zu seiner Absicht Gehöriges, gesetzt es sei auch das gemeine Beste, auszumitteln, einen großen Verstoß an der Gewissenhaftigkeit des Eidleistenden, teils durch den Leichtsinn, zu dem er verleitet und wodurch der Richter seine eigene Absicht vereitelt, teils durch Gewissensbisse, die ein Mensch fühlen muß, der heute eine Sache, aus einem gewissen Gesichtspunkt betrachtet, sehr wahrscheinlich, morgen aber, aus einem anderen, ganz unwahrscheinlich finden kann, und lädiert also denjenigen, den er zu einer solchen Eidesleistung nötigt. 
	
	\subsection*{tg439.2.5} 
	\textbf{Source : }Die Metaphysik der Sitten/Erster Teil. Metaphysische Anfangsgründe der Rechtslehre/1. Teil. Das Privatrecht vom äußeren Mein und Dein überhaupt/3. Hauptstück. Von der subjektiv-bedingten Erwerbung durch den Ausspruch einer öffentlichen Gerichtsbarkeit\\  
	
	\noindent\textbf{Paragraphe : }Die moralische Person, welche der Gerechtigkeit vorsteht; ist der Gerichtshof (forum), und im Zustande ihrer Amtsführung, das \match{Gericht} (iudicium): alles nur nach Rechtsbedingungen a priori gedacht, ohne, wie eine solche Verfassung wirklich einzurichten und zu organisieren sei (wozu Statute, also empirische Prinzipien gehören), in Betrachtung zu ziehen. 
	
	\subsection*{tg441.2.72} 
	\textbf{Source : }Die Metaphysik der Sitten/Erster Teil. Metaphysische Anfangsgründe der Rechtslehre/2. Teil. Das öffentliche Recht/1. Abschnitt. Das Staatsrecht\\  
	
	\noindent\textbf{Paragraphe : }Diese Gleichheit der Strafen, die allein durch die Erkenntnis des Richters auf den Tod, nach dem strengen Wiedervergeltungsrechte, möglich ist, offenbaret sich daran, daß dadurch allein proportionierlich mit der inneren Bösartigkeit der Verbrecher das Todesurteil über alle (selbst wenn es nicht einen Mord, sondern ein anderes nur mit dem Tode zu tilgendes Staatsverbrechen beträfe) ausgesprochen wird. – Setzet: daß, wie in der letzten schottischen Rebellion, da verschiedene Teilnehmer an derselben (wie Balmerino und andere) durch ihre Empörung nichts als eine dem Hause Stuart schuldige Pflicht auszuüben glaubten, andere dagegen Privatabsichten hegten, von dem höchsten \match{Gericht} das Urteil so gesprochen worden wäre: ein jeder solle die Freiheit der Wahl zwischen dem Tode und der Karrenstrafe haben: so sage ich, der ehrliche Mann wählt den Tod, der Schelm aber die Karre; so bringt es die Natur des menschlichen Gemüts mit sich. Denn der erstere kennt etwas, was er noch höher schätzt, als selbst das Leben: nämlich die
	Ehre; der andere hält ein mit Schande bedecktes Leben doch immer noch für besser, als gar nicht zu sein (animam praeferre pudori. Juven.). Der erstere ist nun ohne Widerrede weniger strafbar als der andere, und so werden sie durch den über alle gleich verhängten Tod ganz proportionierlich bestraft, jener gelinde, nach seiner Empfindungsart, und dieser hart, nach der seinigen; da hingegen, wenn durchgängig auf die Karrenstrafe erkannt würde, der erstere zu hart, der andere, für seine Niederträchtigkeit, gar zu gelinde bestraft wäre; und so ist auch hier im Ausspruche über eine im Komplott vereinigte Zahl von Verbrechern der beste Ausgleicher, vor der öffentlichen Gerechtigkeit, der Tod. – Überdem hat man nie gehört, daß ein wegen Mordes zum Tode Verurteilter sich beschwert hätte, daß ihm damit zu viel und also unrecht geschehe, jeder würde ihm ins Gesicht lachen, wenn er sich dessen äußerte. – Man müßte sonst annehmen, daß, wenn dem Verbrecher gleich nach dem Gesetz nicht unrecht geschieht, doch die gesetzgebende Gewalt im Staat diese Art von Strafe zu verhängen nicht befugt, und, wenn sie es tut, mit sich selbst im Widerspruch sei. 
	
	\subsection*{tg441.2.75} 
	\textbf{Source : }Die Metaphysik der Sitten/Erster Teil. Metaphysische Anfangsgründe der Rechtslehre/2. Teil. Das öffentliche Recht/1. Abschnitt. Das Staatsrecht\\  
	
	\noindent\textbf{Paragraphe : }Strafe erleidet jemand nicht, weil er sie, sondern weil er eine strafbare Handlung gewollt hat; denn es ist keine Strafe, wenn einem geschieht, was er will, und es ist unmöglich, gestraft werden zu wollen. – Sagen: ich will gestraft werden, wenn ich jemand ermorde, heißt nichts mehr, als: ich unterwerfe mich samt allen übrigen den Gesetzen, welche natürlicherweise, wenn es Verbrecher im Volk gibt, auch Strafgesetze sein werden. Ich, als Mitgesetzgeber, der das Strafgesetz diktiert, kann unmöglich dieselbe Person sein, die, als Untertan, nach dem Gesetz bestraft wird; denn als ein solcher, nämlich als Verbrecher, kann ich unmöglich eine Stimme in der Gesetzgebung haben (der Gesetzgeber ist heilig). Wenn ich also ein Strafgesetz gegen mich, als einen Verbrecher, abfasse, so ist es in mir die reine rechtlich-gesetzgebende Vernunft (homo noumenon), die mich als einen des Verbrechens Fähigen, folglich als eine andere Person (homo phaenomenon), samt allen übrigen in einem Bürgerverein dem Strafgesetze unterwirft. Mit andern Worten: nicht das Volk (jeder einzelne in demselben), sondern das \match{Gericht} (die öffentliche Gerechtigkeit), mithin ein anderer als der Verbrecher, diktiert die Todesstrafe, und im Sozialkontrakt ist gar nicht das Versprechen enthalten, sich strafen zu lassen, und so über sich selbst und sein Leben zu disponieren. Denn, wenn der Befugnis zu strafen ein Versprechen des Missetäters zum Grunde liegen müßte, sich  strafen lassen zu wollen, so müßte es diesem auch überlassen werden, sich straffällig zu finden, und der Verbrecher würde sein eigener Richter sein. – Der Hauptpunkt des Irrtums (prôton pseudos) dieses Sophisms besteht darin: daß das eigene Urteil des Verbrechers (das man seiner Vernunft notwendig zutrauen muß), des Lebens verlustig werden zu müssen, für einen Beschluß des Willens ansieht, es sich selbst zu nehmen, und so sich die Rechtsvollziehung mit der Rechtsbeurteilung in einer und derselben Person vereinigt vorstellt. 
	
	\subsection*{tg473.2.4} 
	\textbf{Source : }Die Metaphysik der Sitten/Zweiter Teil. Metaphysische Anfangsgründe der Tugendlehre/I. Ethische Elementarlehre/I. Teil. Von den Pflichten gegen sich selbst überhaupt/Erstes Buch. Von den vollkommenen Pflichten gegen sich selbst/Zweites Hauptstück. Die Pflicht des Menschen gegen sich selbst, bloß als einem moralischen Wesen/1. Abschnitt. Von der Pflicht des Menschen gegen sich selbst, als dem angebornen Richter über sich selbst\\  
	
	\noindent\textbf{Paragraphe : }Ein jeder Pflichtbegriff enthält objektive Nötigung durchs Gesetz (als moralischen unsere Freiheit einschränkenden Imperativ) und gehört dem praktischen Verstande zu, der die Regel gibt; die innere Zurechnung aber einer Tat, als eines unter dem Gesetz stehenden Falles (in meritum aut demeritum) gehört zur Urteilskraft (iudicium), welche, als das subjektive Prinzip der Zurechnung der Handlung, ob sie als Tat (unter einem Gesetz stehende Handlung) geschehen sei oder nicht, rechtskräftig urteilt; worauf denn der Schluß der Vernunft (die Sentenz), d.i. die Verknüpfung  der rechtlichen Wirkung mit der Handlung (die Verurteilung oder Lossprechung) folgt: welches alles vor \match{Gericht} (coram iudicio), als einer dem Gesetz Effekt verschaffenden moralischen Person, Gerichtshof (forum) genannt, geschiehet. – Das Bewußtsein eines inneren Gerichtshofes im Menschen (»vor welchem sich seine Gedanken einander verklagen oder entschuldigen«) ist das Gewissen. 
	
	\unnumberedsection{Geschmack (3)} 
	\subsection*{tg471.2.37} 
	\textbf{Source : }Die Metaphysik der Sitten/Zweiter Teil. Metaphysische Anfangsgründe der Tugendlehre/I. Ethische Elementarlehre/I. Teil. Von den Pflichten gegen sich selbst überhaupt/Erstes Buch. Von den vollkommenen Pflichten gegen sich selbst/Erstes Hauptstück. Die Pflicht des Menschen gegen sich selbst, als einem animalischen Wesen\\  
	
	\noindent\textbf{Paragraphe : }Die Geschlechtsneigung wird auch Liebe (in der engsten Bedeutung des Worts) genannt und ist in der Tat die größte Sinnenlust, die an einem Gegenstande möglich ist; – nicht bloß sinnliche Lust, wie an Gegenständen, die in der bloßen Reflexion über sie gefallen (da die Empfänglichkeit für sie \match{Geschmack} heißt), sondern die Lust aus dem Genusse einer anderen Person, die also zum Begehrungsvermögen und zwar der höchsten Stufe desselben, der Leidenschaft, gehört. Sie kann aber weder zur Liebe des Wohlgefallens, noch der des Wohlwollens gezählt werden (denn beide halten eher vom fleischlichen Genuß ab), sondern ist eine Lust von besonderer Art (sui generis) und das Brünstigsein hat mit der moralischen Liebe eigentlich nichts gemein, wiewohl sie mit der letzteren, wenn die praktische Vernunft mit ihren einschränkenden Bedingungen hinzu kommt, in enge Verbindung treten kann. 
	
	\subsection*{tg477.2.6} 
	\textbf{Source : }Die Metaphysik der Sitten/Zweiter Teil. Metaphysische Anfangsgründe der Tugendlehre/I. Ethische Elementarlehre/I. Teil. Von den Pflichten gegen sich selbst überhaupt/2. Buch: Die Pflichten gegen sich selbst/Erster Abschnitt. Von der Pflicht gegen sich selbst in Entwickelung und Vermehrung seiner Naturvollkommenheit, d.i. in pragmatischer Absicht\\  
	
	\noindent\textbf{Paragraphe : }
	Seelenkräfte sind diejenige, welche dem Verstande und der Regel, die er zu Befriedigung beliebiger Absichten braucht, zu Gebote stehen, und so fern an dem Leitfaden der Erfahrung geführt werden. Dergleichen ist das Gedächtnis, die Einbildungskraft u. dgl., worauf Gelahrtheit, \match{Geschmack} (innere und äußere Verschönerung) etc. gegründet werden können, welche zu mannigfaltiger Absicht die Werkzeuge darbieten. 
	
	\subsection*{tg483.2.4} 
	\textbf{Source : }Die Metaphysik der Sitten/Zweiter Teil. Metaphysische Anfangsgründe der Tugendlehre/I. Ethische Elementarlehre/II. Teil. Von den Tugendpflichten gegen andere/Zweites Hauptstück. Von den ethischen Pflichten der Menschen gegen einander in Ansehung ihres Zustandes\\  
	
	\noindent\textbf{Paragraphe : }Diese (Tugendpflichten) können zwar in der reinen Ethik keinen Anlaß zu einem besondern Hauptstück im System derselben geben, denn sie enthalten nicht Prinzipien der Verpflichtung der Menschen als solcher gegen einander, und können also von den metaphysischen Anfangsgründen der Tugendlehre eigentlich nicht einen Teil abgeben, sondern sind nur, nach Verschiedenheit der Subjekte der Anwendung des Tugendprinzips (dem Formale nach) auf in der Erfahrung vorkommende Fälle (das Materiale) modifizierte, Regeln, weshalb sie auch, wie alle empirische Einteilungen, keine gesichert-vollständige Klassifikation zulassen. Indessen, gleichwie von der Metaphysik der Natur zur Physik ein Überschritt, der seine besondern Regeln hat, verlangt wird: so wird der Metaphysik der Sitten ein Ähnliches mit Recht angesonnen: nämlich durch Anwendung reiner Pflichtprinzipien auf Fälle der Erfahrung jene gleichsam  zu schematisieren und zum moralisch-praktischen Gebrauch fertig darzulegen, – Welches Verhalten also gegen Menschen, z.B. in der moralischen Reinigkeit ihres Zustandes, oder in ihrer Verdorbenheit; welches im kultivierten, oder rohen Zustände; was den Gelehrten oder Ungelehrten, und jenen im Gebrauch ihrer Wissenschaft als umgänglichen (geschliffenen), oder in ihrem Fach unumgänglichen Gelehrten (Pedanten), pragmatischen oder mehr auf Geist und \match{Geschmack} ausgehenden; welches nach Verschiedenheit der Stände, des Alters, des Geschlechts, des Gesundheitszustandes, des der Wohlhabenheit oder Armut u.s.w. zukomme: das gibt nicht so vielerlei Arten der ethischen Verpflichtung (denn es ist nur eine, nämlich die der Tugend überhaupt), sondern nur Arten der Anwendung (Porismen) ab; die also nicht, als Abschnitte der Ethik und Glieder der Einteilung eines Systems (das a priori aus einem Vernunftbegriffe hervorgehen muß), aufgeführt, sondern nur angehängt werden können. – Aber eben diese Anwendung gehört zur Vollständigkeit der Darstellung desselben. 
	
	\unnumberedsection{Hals (1)} 
	\subsection*{tg445.2.33} 
	\textbf{Source : }Die Metaphysik der Sitten/Erster Teil. Metaphysische Anfangsgründe der Rechtslehre/Anhang erläutender Bemerkungen zu den metaphysischen Anhangsgründen der Rechtslehre\\  
	
	\noindent\textbf{Paragraphe : }Die bloße Idee einer Staatsverfassung unter Menschen führt schon den Begriff einer Strafgerechtigkeit bei sich, welche der obersten Gewalt zusteht. Es fragt sich nur, ob die Strafarten dem Gesetzgeber gleichgültig sind, wenn sie nur als Mittel dazu taugen, das Verbrechen (als Verletzung der Staatssicherheit im Besitz des Seinen eines jeden) zu entfernen, oder ob auch noch auf Achtung für die Menschheit, in der Person des Missetäters ( d.i. für die Gattung), Rücksicht genommen werden müsse, und zwar aus bloßen Rechtsgründen, indem ich das ius talionis, der Form nach, noch immer für die einzige a priori bestimmende (nicht aus der Erfahrung, welche Heilmittel zu dieser Absicht die kräftigsten wären, hergenommen) Idee als Prinzip des Strafrechts halte.
	
	
	12
	– Wie wird es aber mit den Strafen gehalten werden, die keine Erwiderung zulassen;  weil diese entweder an sich unmöglich, oder selbst ein strafbares Verbrechen an der Menschheit überhaupt sein würden, wie z.B. das der Notzüchtigung; imgleichen das der Päderastie, oder Bestialität. Die beiden ersteren durch Kastration (entweder wie eines weißen oder schwärzen Verschnittenen im Serail), das letztere durch Ausstoßung aus der bürgerlichen Gesellschaft auf immer, weil er sich selbst der menschlichen unwürdig gemacht hat. – Per quod quis peccat, per idem punitur et idem. – Die gedachten Verbrechen heißen darum unnatürlich, weil sie an der Menschheit selbst ausgeübt werden. – Willkürlich Strafen für sie zu verhängen ist dem Begriff einer Strafgerechtigkeit buchstäblich zuwider. Nur dann kann der Verbrecher nicht klagen, daß ihm unrecht geschehe, wenn er seine Übeltat sich selbst über den \match{Hals} zieht, und ihm, wenn gleich nicht dem Buchstaben, doch dem Geiste des Strafgesetzes gemäß, das widerfährt, was er an anderen verbrochen hat. 
	
	\unnumberedsection{Herz (4)} 
	\subsection*{tg474.2.4} 
	\textbf{Source : }Die Metaphysik der Sitten/Zweiter Teil. Metaphysische Anfangsgründe der Tugendlehre/I. Ethische Elementarlehre/I. Teil. Von den Pflichten gegen sich selbst überhaupt/Erstes Buch. Von den vollkommenen Pflichten gegen sich selbst/Zweites Hauptstück. Die Pflicht des Menschen gegen sich selbst, bloß als einem moralischen Wesen/2. Abschnitt. Von dem ersten Gebot aller Pflichten gegen sich selbst\\  
	
	\noindent\textbf{Paragraphe : }Dieses ist: Erkenne (erforsche, ergründe) dich selbst nicht nach deiner physischen Vollkommenheit (der Tauglichkeit oder Untauglichkeit zu allerlei dir beliebigen oder auch gebotenen Zwecke), sondern nach der moralischen, in Beziehung auf deine Pflicht – dein \match{Herz} – ob es gut oder böse sei, ob die Quelle deiner Handlungen lauter oder unlauter, und was, entweder als ursprünglich zur Substanz des Menschen gehörend, oder, als abgeleitet (erworben oder zugezogen) ihm selbst zugerechnet werden kann und zum moralischen Zustande gehören mag. 
	
	\subsection*{tg486.2.28} 
	\textbf{Source : }Die Metaphysik der Sitten/Zweiter Teil. Metaphysische Anfangsgründe der Tugendlehre/II. Ethische Methodenlehre/1. Abschnitt. Die ethische Didaktik\\  
	
	\noindent\textbf{Paragraphe : }4. L. Das beweist nun wohl, daß du noch so ziemlich ein gutes \match{Herz} hast; laß aber sehen, ob du dabei auch guten Verstand zeigest. – Würdest du wohl dem Faulenzer weiche Polster verschaffen, damit er im süßen Nichtstun sein Leben dahin bringe, oder dem Trunkenbolde es an Wein, und was sonst zur Berauschung gehört, nicht ermangeln lassen, dem Betrüger eine einnehmende Gestalt und Manieren geben, um andere zu überlisten, oder dem Gewalttätigen Kühnheit und starke Faust, um andere überwältigen zu können? Das sind ja so viel Mittel, die ein jeder sich wünscht, um nach seiner Art glücklich zu sein. S. Nein, das nicht. 
	
	\subsection*{tg486.2.35} 
	\textbf{Source : }Die Metaphysik der Sitten/Zweiter Teil. Metaphysische Anfangsgründe der Tugendlehre/II. Ethische Methodenlehre/1. Abschnitt. Die ethische Didaktik\\  
	
	\noindent\textbf{Paragraphe : }Wenn dieses nun weislich und pünktlich nach Verschiedenheit der Stufen des Alters, des Geschlechts und des Standes, die der Mensch nach und nach betritt, aus der eigenen Vernunft des Menschen entwickelt worden,  so ist noch etwas, was den Beschluß machen muß, was die Seele inniglich bewegt und den Menschen auf eine Stelle setzt, wo er sich selbst nicht anders als mit der größten Bewunderung der ihm beiwohnenden ursprünglichen Anlagen betrachten kann, und wovon der Eindruck nie erlischt. – Wenn ihm nämlich beim Schlüsse seiner Unterweisung seine Pflichten in ihrer Ordnung noch einmal summarisch vorerzählt (rekapituliert), wenn er, bei jeder derselben, darauf aufmerksam gemacht wird, daß alle Übel, Drangsale und Leiden des Lebens, selbst Bedrohung mit dem Tode, die ihn darüber, daß er seiner Pflicht treu gehorcht, treffen mögen, ihm doch das Bewußtsein, über sie alle erhoben und Meister zu sein, nicht rauben können, so liegt ihm nun die Frage ganz nahe: was ist das in dir, was sich getrauen darf, mit allen Kräften der Natur in dir und um dich in Kampf zu treten und sie, wenn sie mit deinen sittlichen Grundsätzen in Streit kommen, zu besiegen? Wenn diese Frage, deren Auflösung das Vermögen der spekulativen Vernunft gänzlich übersteigt und die sich dennoch von selbst einstellt, ans \match{Herz} gelegt wird, so muß selbst die Unbegreiflichkeit in diesem Selbsterkenntnisse der Seele eine Erhebung geben, die sie zum Heilighalten ihrer Pflicht nur desto stärker belebt, je mehr sie angefochten wird. 
	
	\subsection*{tg487.2.5} 
	\textbf{Source : }Die Metaphysik der Sitten/Zweiter Teil. Metaphysische Anfangsgründe der Tugendlehre/II. Ethische Methodenlehre/2. Abschnitt. Die ethische Asketik\\  
	
	\noindent\textbf{Paragraphe : }Die Kultur der Tugend, d.i. die moralische Asketik, hat, in Ansehung des Prinzips der rüstigen, mutigen und wackeren Tugendübung den Wahlspruch der Stoiker: gewöhne dich, die zufälligen Lebensübel zu ertragen und die eben so überflüssigen Ergötzlichkeiten zu entbehren (assuesce incommodis et desuesce commoditatibus vitae). Es ist eine Art von Diätetik für den Menschen, sich moralisch gesund zu erhalten. Gesundheit ist aber nur ein negatives  Wohlbefinden, sie selber kann nicht gefühlt werden. Es muß etwas dazu kommen, was einen angenehmen Lebensgenuß gewährt und doch bloß moralisch ist. Das ist das jederzeit fröhliche \match{Herz} in der Idee des tugendhaften Epikurs. Denn wer sollte wohl mehr Ursache haben, frohen Muts zu sein und nicht darin selbst eine Pflicht finden, sich in eine fröhliche Gemütsstimmung zu versetzen und sie sich habituell zu machen, als der, welcher sich keiner vorsätzlichen Übertretung bewußt und, wegen des Verfalls in eine solche, gesichert ist (hic murus ahenëus esto etc. Horat.). – Die Mönchsasketik hingegen, welche aus abergläubischer Furcht, oder geheucheltem Abscheu an sich selbst, mit Selbstpeinigung und Fleischeskreuzigung zu Werke geht, zweckt auch nicht auf Tugend, sondern auf schwärmerische Entsündigung ab, sich selbst Strafe aufzulegen und, anstatt sie moralisch (d.i. in Absicht auf die Besserung) zu bereuen, sie büßen zu wollen; welches, bei einer selbstgewählten und an sich vollstreckten Strafe (denn die muß immer ein anderer auflegen), ein Widerspruch ist, und kann auch den Frohsinn, der die Tugend begleitet, nicht bewirken, vielmehr nicht ohne geheimen Haß gegen das Tugendgebot statt finden. – Die ethische Gymnastik besteht also nur in der Bekämpfung der Naturtriebe, die das Maß erreicht, über sie bei vorkommenden, der Moralität Gefahr drohenden, Fällen Meister werden zu können; mithin die wacker und, im Bewußtsein seiner wiedererworbenen Freiheit, fröhlich macht. Etwas bereuen (welches bei der Rückerinnerung ehemaliger Übertretungen unvermeidlich, ja wobei diese Erinnerung nicht schwinden zu lassen es so gar Pflicht ist) und sich eine Pönitenz auferlegen (z.B. das Fasten), nicht in diätetischer, sondern frommer Rücksicht, sind zwei sehr verschiedene, moralisch gemeinte, Vorkehrungen, von denen die letztere, welche freudenlos, finster und mürrisch ist, die Tugend selbst verhaßt macht und ihre Anhänger verjagt. Die Zucht (Disziplin), die der Mensch an sich selbst verübt, kann daher nur durch den Frohsinn, der sie begleitet, verdienstlich und exemplarisch werden. 
	
	\unnumberedsection{Kopf (2)} 
	\subsection*{tg431.2.8} 
	\textbf{Source : }Die Metaphysik der Sitten/Erster Teil. Metaphysische Anfangsgründe der Rechtslehre/Einleitung in die Rechtslehre\\  
	
	\noindent\textbf{Paragraphe : }Diese Frage möchte wohl den Rechtsgelehrten, wenn er nicht in Tautologie verfallen, oder, statt einer allgemeinen Auflösung, auf das, was in irgend einem Lande die Gesetze zu irgend einer Zeit wollen, verweisen will, eben so in Verlegenheit setzen, als die berufene Aufforderung: Was ist Wahrheit? den Logiker, Was Rechtens sei (quid sit iuris), d.i. was die Gesetze an einem gewissen Ort und zu einer gewissen Zeit sagen oder gesagt haben, kann er noch wohl angeben; aber, ob das, was sie wollten, auch recht sei, und das allgemeine Kriterium, woran man überhaupt Recht sowohl als Unrecht (iustum et iniustum) erkennen könne, bleibt ihm wohl verborgen, wenn er nicht eine Zeitlang jene empirischen Prinzipien verläßt, die Quellen jener Urteile in der bloßen Vernunft sucht (wiewohl ihm dazu jene Gesetze vortrefflich zum Leitfaden dienen können), um zu einer möglichen positiven Gesetzgebung die Grundlage zu errichten. Eine bloß empirische Rechtslehre ist (wie der hölzerne \match{Kopf} in Phädrus' Fabel) ein Kopf, der schön sein mag, nur schade! daß er kein Gehirn hat.  Der Begriff des Rechts, sofern er sich auf eine ihm korrespondierende Verbindlichkeit bezieht (d.i. der moralische Begriff derselben), betrifft erstlich nur das äußere und zwar praktische Verhältnis einer Person gegen eine andere, sofern ihre Handlungen als Facta aufeinander (unmittelbar, oder mittelbar) Einfluß haben können. Aber zweitens bedeutet er nicht das Verhältnis der Willkür auf den Wunsch (folglich auch auf das bloße Bedürfnis) des anderen, wie etwa in den Handlungen der Wohltätigkeit oder Hartherzigkeit, sondern lediglich auf die Willkür des anderen. Drittens in diesem wechselseitigen Verhältnis der Willkür kommt auch gar nicht die Materie der Willkür, d.i. der Zweck, den ein jeder mit dem Objekt, was er will, zur Absicht hat, in Betrachtung, z.B. es wird nicht gefragt, ob jemand bei der Ware, die er zu seinem eigenen Handel von mir kauft, auch seinen Vorteil finden möge, oder nicht, sondern nur nach der Form im Verhältnis der beiderseitigen Willkür, sofern sie bloß als frei betrachtet wird, und ob durch die Handlung eines von beiden sich mit der Freiheit des andern nach einem allgemeinen Gesetze zusammen vereinigen lasse. 
	
	\subsection*{tg481.2.85} 
	\textbf{Source : }Die Metaphysik der Sitten/Zweiter Teil. Metaphysische Anfangsgründe der Tugendlehre/I. Ethische Elementarlehre/II. Teil. Von den Tugendpflichten gegen andere/Erstes Hauptstück. Von den Pflichten gegen andere, bloß als Menschen/Erster Abschnitt. Von der Liebespflicht gegen andere Menschen\\  
	
	\noindent\textbf{Paragraphe : }b) Undankbarkeit gegen seinen Wohltäter, welche, wenn sie gar so weit geht, seinen Wohltäter zu hassen, 
	qualifizierte Undankbarkeit, sonst aber bloß Unerkenntlichkeit heißt, ist ein zwar im öffentlichen Urteile höchst verabscheutes Laster, gleichwohl ist der Mensch desselben wegen so berüchtigt, daß man es nicht für unwahrscheinlich hält, man könne sich durch erzeigte Wohltaten wohl gar einen Feind machen. – Der Grund der Möglichkeit eines solchen Lasters liegt in der mißverstandenen Pflicht gegen sich selbst, die Wohltätigkeit anderer, weil sie uns Verbindlichkeit gegen sie auferlegt, nicht zu bedürfen und aufzufordern, sondern lieber die Beschwerden des Lebens selbst zu ertragen, als andere damit zu belästigen, mithin dadurch bei ihnen in Schulden (Verpflichtung) zu kommen; weil wir dadurch auf die niedere Stufe des Beschützten gegen seinen Beschützer zu geraten fürchten; welches der echten Selbstschätzung (auf die Würde der Menschheit in seiner eigenen Person stolz zu sein) zuwider ist. Daher Dankbarkeit gegen die, die uns im Wohltun unvermeidlich zuvor kommen mußten (gegen Vorfahren im Angedenken, oder gegen Eltern), freigebig, die aber gegen Zeitgenossen nur kärglich, ja, um dieses Verhältnis der Ungleichheit unsichtbar zu machen, wohl gar das Gegenteil derselben bewiesen wird. – Dieses ist aber alsdann ein die Menschheit empörendes Laster, nicht bloß des Schadens wegen, den ein solches Beispiel Menschen überhaupt zuziehen muß, von fernerer Wohltätigkeit abzuschrecken (denn diese können mit echtmoralischer Gesinnung, eben in der Verschmähung alles solchen Lohns ihrem Wohltun nur einen desto größeren inneren moralischen Wert setzen): sondern weil die Menschenliebe hier gleichsam auf den \match{Kopf} gestellt, und der Mangel der Liebe gar in die Befugnis, den Liebenden zu hassen, verunedelt wird. 
	
	\unnumberedsection{Korper (4)} 
	\subsection*{tg430.2.13} 
	\textbf{Source : }Die Metaphysik der Sitten/Erster Teil. Metaphysische Anfangsgründe der Rechtslehre/Einleitung in die Metaphysik der Sitten\\  
	
	\noindent\textbf{Paragraphe : }Daß man für die Naturwissenschaft, welche es mit den Gegenständen äußerer Sinne zu tun hat, Prinzipien a priori haben müsse, und daß es möglich, ja notwendig sei, ein System dieser Prinzipien, unter dem Namen einer metaphysischen Naturwissenschaft, vor der auf besondere Erfahrungen angewandten, d.i. der Physik, voranzuschicken, ist an einem andern Orte bewiesen worden. Allein die letztere kann (wenigstens wenn es ihr darum zu tun ist, von ihren Sätzen den Irrtum abzuhalten) manches Prinzip auf das Zeugnis der Erfahrung als allgemein annehmen, obgleich das letztere, wenn es in strenger Bedeutung allgemein gelten soll, aus Gründen a priori abgeleitet werden müßte, wie Newton das Prinzip der Gleichheit der Wirkung und Gegenwirkung im Einflusse der \match{Körper} auf einander als auf Erfahrung gegründet annahm, und es gleichwohl über die ganze materielle Natur ausdehnte. Die Chymiker gehen noch weiter und gründen ihre allgemeinste Gesetze der Vereinigung und Trennung der Materien durch ihre eigene Kräfte gänzlich auf Erfahrung, und vertrauen gleichwohl auf ihre Allgemeinheit und Notwendigkeit so, daß sie in den mit ihnen angestellten Versuchen keine Entdeckung eines Irrtums besorgen. 
	
	\subsection*{tg431.2.24} 
	\textbf{Source : }Die Metaphysik der Sitten/Erster Teil. Metaphysische Anfangsgründe der Rechtslehre/Einleitung in die Rechtslehre\\  
	
	\noindent\textbf{Paragraphe : }Das Gesetz eines mit jedermanns Freiheit notwendig zusammenstimmenden wechselseitigen Zwanges, unter dem Prinzip der allgemeinen Freiheit, ist gleichsam die Konstruktion jenes Begriffs, d.i. Darstellung desselben in einer reinen Anschauung a priori, nach der Analogie der Möglichkeit freier Bewegungen der \match{Körper} unter dem Gesetze der Gleichheit der Wirkung und Gegenwirkung. So wie wir nun in der reinen Mathematik die Eigenschaften ihres Objekts nicht unmittelbar vom Begriffe ableiten, sondern nur durch die Konstruktion des Begriffs entdecken können, so ist's nicht sowohl der Begriff des Rechts, als vielmehr der, unter allgemeine Gesetze gebrachte, mit ihm zusammenstimmende durchgängig wechselseitige und gleiche Zwang, der die Darstellung jenes Begriffs möglich macht. Dieweil aber diesem dynamischen Begriffe noch ein bloß formaler, in der reinen Mathematik (z.B. der Geometrie) zum Grunde liegt: so hat die Vernunft dafür gesorgt, den Verstand auch mit Anschauungen a priori, zum Behuf der Konstruktion des Rechtsbegriffs, so viel möglich zu versorgen. – Das Rechte (rectum) wird, als das Gerade, teils dem Krummen, teils dem Schiefen entgegen gesetzt. Das erste ist die innere Beschaffenheit einer Linie von der Art, daß es zwischen zweien gegebenen Punkten nur eine einzige, das zweite aber die Lage zweier einander durchschneidenden oder zusammenstoßenden Linien, von deren Art es auch nur eine einzige (die senkrechte) geben kann, die sich nicht mehr nach einer Seite, als der andern hinneigt, und die den Raum von beiden Seiten gleich abteilt, nach welcher Analogie auch die Rechtslehre das Seine einem jeden (mit mathematischer Genauigkeit) bestimmt wissen will, welches in der Tugendlehre nicht erwartet werden darf, als welche einen gewissen Raum zu Ausnahmen (latitudinem) nicht verweigern kann. – Aber, ohne ins Gebiet der Ethik einzugreifen, gibt es zwei Fälle, die auf Rechtsentscheidung  Anspruch machen, für die aber keiner, der sie entscheide, ausgefunden werden kann, und die gleichsam in Epikurs Intermundia hingehören. – Diese müssen wir zuvörderst aus der eigentlichen Rechtslehre, zu der wir bald schreiten wollen, aussondern, damit ihre schwankenden Prinzipien nicht auf die festen Grundsätze der erstern Einfluß bekommen. 
	
	\subsection*{tg435.2.38} 
	\textbf{Source : }Die Metaphysik der Sitten/Erster Teil. Metaphysische Anfangsgründe der Rechtslehre/1. Teil. Das Privatrecht vom äußeren Mein und Dein überhaupt/2. Hauptstück. Von der Art, etwas Äußeres zu erwerben/1. Abschnitt. Vom Sachrecht\\  
	
	\noindent\textbf{Paragraphe : }Was die \match{Körper} auf einem Boden betritt, der schon der meinige ist, so gehören sie, wenn sie sonst keines anderen sind, mir zu, ohne daß ich zu diesem Zweck eines besonderen rechtlichen Akts bedürfte (nicht facto sondern lege); nämlich, weil sie als der Substanz inhärierende Akzidenzen betrachtet werden können (iure rei  meae), wozu auch alles gehört, was mit meiner Sache so verbunden ist, daß ein anderer sie von dem Meinen nicht trennen kann, ohne dieses selbst zu verändern (z.B. Vergoldung, Mischung eines mir zugehörigen Stoffes mit andern Materien, Anspülung oder auch Veränderung des anstoßenden Strombettes, und dadurch geschehende Erweiterung meines Bodens, u.s.w.). Ob aber der erwerbliche Boden sich noch weiter als das Land, nämlich auch auf eine Strecke des Seegrundes hin aus (das Recht, noch an meinen Ufern zu fischen, oder Bernstein herauszubringen, u. dergl.), sich ausdehnen lasse, muß nach ebendenselben Grundsätzen beurteilt werden. So weit ich aus meinem Sitze mechanisches Vermögen habe, meinen Boden gegen den Eingriff anderer zu sichern (z.B. so weit die Kanonen vom Ufer abreichen), gehört zu meinem Besitz und das Meer ist bis dahin geschlossen (mare clausum). Da aber auf dem weiten Meere selbst kein Sitz möglich ist, so kann der Besitz auch nicht bis dahin ausgedehnt werden und offene See ist frei (mare liberum). Das Stranden aber, es sei der Menschen, oder der ihnen zugehörigen Sachen, kann, als unvorsätzlich, von dem Strandeigentümer nicht zum Erwerbrecht gezählt werden; weil es nicht Läsion (ja überhaupt kein Faktum) ist, und die Sache, die auf einen Boden geraten ist, der doch irgend einem angehört, nicht als res nullius behandelt werden kann. Ein Fluß dagegen kann, so weit der Besitz seines Ufers reicht, so gut wie ein jeder Landboden, unter obbenannten Einschränkungen ursprünglich von dem erworben werden, der im Besitz beider Ufer ist. 
	
	\subsection*{tg469.2.17} 
	\textbf{Source : }Die Metaphysik der Sitten/Zweiter Teil. Metaphysische Anfangsgründe der Tugendlehre/I. Ethische Elementarlehre/I. Teil. Von den Pflichten gegen sich selbst überhaupt/Einleitung\\  
	
	\noindent\textbf{Paragraphe : }Die Einteilung kann nur in Ansehung des Objekts der Pflicht, nicht in Ansehung des sich verpflichtenden Subjekts, gemacht werden. Das verpflichtete so wohl als das verpflichtende Subjekt ist immer nur der Mensch, und wenn es uns, in theoretischer Rücksicht, gleich erlaubt ist, im Menschen Seele und \match{Körper} als Naturbeschaffenheiten des Menschen von einander zu unterscheiden, so ist es doch nicht erlaubt, sie als verschiedene den Menschen verpflichtende Substanzen zu denken, um zur Einteilung in Pflichten gegen den Körper und gegen die Seele berechtigt zu sein. – Wir sind, weder durch Erfahrung, noch durch Schlüsse der Vernunft, hinreichend darüber belehrt, ob der Mensch eine Seele (als in ihm wohnende, vom Körper unterschiedene und von diesem unabhängig zu denken vermögende, d.i. geistige Substanz) enthalte, oder ob nicht vielmehr das Leben eine Eigenschaft der Materie sein möge, und wenn es sich auch auf die erstere Art verhielte, so würde doch keine Pflicht des Men schen gegen einen Körper (als verpflichtendes Subjekt), ob er gleich der menschliche ist, denkbar sein. 
	
	\unnumberedsection{Nahrung (1)} 
	\subsection*{tg441.2.20} 
	\textbf{Source : }Die Metaphysik der Sitten/Erster Teil. Metaphysische Anfangsgründe der Rechtslehre/2. Teil. Das öffentliche Recht/1. Abschnitt. Das Staatsrecht\\  
	
	\noindent\textbf{Paragraphe : }Nur die Fähigkeit der Stimmgebung macht die Qualifikation zum Staatsbürger aus; jene aber setzt die Selbständigkeit dessen im Volk voraus, der nicht bloß Teil des gemeinen Wesens, sondern auch Glied desselben, d.i. aus eigener Willkür in Gemeinschaft mit anderen handelnder  Teil desselben sein will. Die letztere Qualität macht aber die Unterscheidung des aktiven vom passiven Staatsbürger notwendig: obgleich der Begriff des letzteren mit der Erklärung des Begriffs von einem Staatsbürger überhaupt im Widerspruch zu stehen scheint. – Folgende Beispiele können dazu dienen, diese Schwierigkeit zu heben: Der Geselle bei einem Kaufmann, oder bei einem Handwerker; der Dienstbote (nicht der im Dienste des Staats steht); der Unmündige (naturaliter vel civiliter); alles Frauenzimmer, und überhaupt jedermann, der nicht nach eigenem Betrieb, sondern nach der Verfügung anderer (außer der des Staats), genötigt ist, seine Existenz (\match{Nahrung} und Schutz) zu erhalten, entbehrt der bürgerlichen Persönlichkeit, und seine Existenz ist gleichsam nur Inhärenz. – Der Holzhacker, den ich auf meinem Hofe anstelle, der Schmied in Indien, der mit seinem Hammer, Amboß und Blasbalg in die Häuser geht, um da in Eisen zu arbeiten, in Vergleichung mit dem europäischen Tischler oder Schmied, der die Produkte aus dieser Arbeit als Ware öffentlich feil stellen kann, der Hauslehrer in Vergleichung mit dem Schulmann, der Zinsbauer in Vergleichung mit dem Pächter u. dergl. sind bloß Handlanger des gemeinen Wesens, weil sie von anderen Individuen befehligt oder beschützt werden müssen, mithin keine bürgerliche Selbständigkeit besitzen. 
	
	\unnumberedsection{Speisen (1)} 
	\subsection*{tg471.2.45} 
	\textbf{Source : }Die Metaphysik der Sitten/Zweiter Teil. Metaphysische Anfangsgründe der Tugendlehre/I. Ethische Elementarlehre/I. Teil. Von den Pflichten gegen sich selbst überhaupt/Erstes Buch. Von den vollkommenen Pflichten gegen sich selbst/Erstes Hauptstück. Die Pflicht des Menschen gegen sich selbst, als einem animalischen Wesen\\  
	
	\noindent\textbf{Paragraphe : }Die tierische Unmäßigkeit, im Genuß der Nahrung, ist der Mißbrauch der Genießmittel, wodurch das Vermögen des intellektuellen Gebrauchs derselben gehemmt oder erschöpft wird. Versoffenheit und Gefräßigkeit sind die Laster, die unter diese Rubrik gehören. Im Zustande der Betrunkenheit ist der Mensch nur wie ein Tier, nicht als Mensch, zu behandeln; durch die Überladung mit \match{Speisen} und in einem solchen Zustande ist er für Handlungen, wozu Gewandtheit und Überlegung im Gebrauch seiner Kräfte erfordert wird, auf eine gewisse Zeit gelähmt. – Daß sich in einen solchen Zustand zu versetzen Verletzung einer Pflicht wider sich selbst sei, fällt von selbst in die Augen. Die erste dieser Erniedrigungen, selbst unter die tierische Natur, wird gewöhnlich durch gegorene Getränke, aber auch durch andere betäubende Mittel, als den Mohnsaft und andere Produkte des Gewächsreichs, bewirkt, und wird dadurch verführerisch, daß dadurch auf eine Weile geträumte Glückseligkeit und Sorgenfreiheit, ja wohl auch eingebildete Stärke hervorgebracht, Niedergeschlagenheit aber und Schwäche, und, was das Schlimmste ist, Notwendigkeit, dieses Betäubungsmittel zu wiederholen, ja wohl gar damit zu steigern, eingeführt wird. Die Gefräßigkeit ist sofern noch unter jener tierischen Sinnenbelustigung, daß sie bloß den Sinn  als passive Beschaffenheit und nicht einmal die Einbildungskraft, welche doch noch ein tätiges Spiel der Vorstellungen, wie im vorerwähnten Genuß der Fall ist, beschäftigt; mithin sich dem des Viehes noch mehr nähert. 
	
	\unnumberedsection{Waßer (1)} 
	\subsection*{tg481.2.7} 
	\textbf{Source : }Die Metaphysik der Sitten/Zweiter Teil. Metaphysische Anfangsgründe der Tugendlehre/I. Ethische Elementarlehre/II. Teil. Von den Tugendpflichten gegen andere/Erstes Hauptstück. Von den Pflichten gegen andere, bloß als Menschen/Erster Abschnitt. Von der Liebespflicht gegen andere Menschen\\  
	
	\noindent\textbf{Paragraphe : }Wenn von Pflichtgesetzen (nicht von Naturgesetzen) die Rede ist und zwar im äußeren Verhältnis der Menschen gegen einander, so betrachten wir uns in einer moralischen (intelligibelen) Welt, in welcher, nach der Analogie mit der physischen, die Verbindung vernünftiger Wesen (auf Erden) durch Anziehung und Abstoßung bewirkt wird. Vermöge des Prinzips der Wechselliebe sind sie angewiesen, sich einander beständig zu nähern, durch das der Achtung, die sie einander schuldig sind, sich im Abstande von einander zu erhalten, und, sollte eine dieser großen sittlichen Kräfte sinken: »so würde dann das Nichts (der Immoralität) mit aufgesperrtem Schlund der (moralischen) Wesen ganzes Reich, wie einen Tropfen \match{Wasser} trinken« (wenn ich mich hier der Worte Hallers, nur in einer andern Beziehung, bedienen darf). 
	
	\unnumberedsection{Wein (1)} 
	\subsection*{tg471.2.50} 
	\textbf{Source : }Die Metaphysik der Sitten/Zweiter Teil. Metaphysische Anfangsgründe der Tugendlehre/I. Ethische Elementarlehre/I. Teil. Von den Pflichten gegen sich selbst überhaupt/Erstes Buch. Von den vollkommenen Pflichten gegen sich selbst/Erstes Hauptstück. Die Pflicht des Menschen gegen sich selbst, als einem animalischen Wesen\\  
	
	\noindent\textbf{Paragraphe : }Kann man dem Wein, wenn gleich nicht als Panegyrist, doch wenigstens als Apologet, einen Gebrauch verstatten, der bis nahe an die Berauschung reicht; weil er doch die Gesellschaft zur Gesprächigkeit belebt, und damit Offenherzigkeit verbindet? – Oder kann man ihm wohl gar das Verdienst zugestehen, das zu befördern, was Seneca vom Cato rühmt: virtus eius incaluit mero? – Der Gebrauch des Opium und Branntweins sind, als Genießmittel, der Niederträchtigkeit näher, weil sie, bei dem geträumten Wohlbefinden, stumm, zurückhaltend und unmitteilbar machen, daher auch nur als Arzneimittel erlaubt sind. – Wer kann aber das Maß für einen bestimmen, der in den Zustand, wo er zum Messen keine klare Augen mehr hat, überzugehen eben in Bereitschaft ist? Der Mohammedanism, welcher den \match{Wein} ganz verbietet, hat also sehr schlecht gewählt, dafür das Opium zu erlauben. 
	
	\unnumberedsection{Wirtschaft (2)} 
	\subsection*{tg461.2.4} 
	\textbf{Source : }Die Metaphysik der Sitten/Zweiter Teil. Metaphysische Anfangsgründe der Tugendlehre/Einleitung/XIII. Allgemeine Grundsätze der Metaphysik der Sitten in Behandlung einer reinen Tugendlehre\\  
	
	\noindent\textbf{Paragraphe : }
	Zweitens. Der Unterschied der Tugend vom Laster kann nie in Graden der Befolgung gewisser Maximen, sondern muß allein in der spezifischen Qualität derselben (dem Verhältnis zum Gesetz) gesucht werden; mit andern Worten, der belobte Grundsatz (des Aristoteles), die Tugend in dem Mittleren zwischen zwei Lastern zu setzen, ist falsch.
	
	
	17
	Es sei z.B. gute Wirtschaft, als das Mittlere
	zwischen zwei Lastern, Verschwendung und Geiz, gegeben: so kann sie als Tugend nicht durch die allmähliche Verminderung des ersten beider genannten Laster (Ersparung), noch durch die Vermehrung der Ausgaben, des dem letzteren Ergebenen, als entspringend vorgestellt werden: indem sie sich gleichsam nach entgegengesetzten Richtungen in der guten \match{Wirtschaft} begegneten: sondern eine jede derselben hat ihre eigene Maxime, die der andern notwendig widerspricht. 
	
	\subsection*{tg472.2.24} 
	\textbf{Source : }Die Metaphysik der Sitten/Zweiter Teil. Metaphysische Anfangsgründe der Tugendlehre/I. Ethische Elementarlehre/I. Teil. Von den Pflichten gegen sich selbst überhaupt/Erstes Buch. Von den vollkommenen Pflichten gegen sich selbst\\  
	
	\noindent\textbf{Paragraphe : }Wenn ich nämlich zwischen Verschwendung und Geiz die gute \match{Wirtschaft} als das Mittlere ansehe, und dieses das Mittlere des Grades sein soll: so würde ein Laster in das (contrarie) entgegengesetzte Laster nicht anders übergehen, als durch die Tugend, und so würde diese nichts anders, als ein vermindertes, oder vielmehr verschwindendes Laster sein, und die Folge wäre in dem gegenwärtigen Fall: daß von den Mitteln des Wohllebens gar keinen Gebrauch zu machen die echte Tugendpflicht sei. 
	
	\unnumberedchapter{Botanique} 
	\unnumberedsection{Apfel (2)} 
	\subsection*{tg433.2.18} 
	\textbf{Source : }Die Metaphysik der Sitten/Erster Teil. Metaphysische Anfangsgründe der Rechtslehre/1. Teil. Das Privatrecht vom äußeren Mein und Dein überhaupt/1. Hauptstück\\  
	
	\noindent\textbf{Paragraphe : }a) Ich kann einen Gegenstand im Raume (eine körperliche Sache) nicht mein nennen, außer wenn, obgleich ich nicht im physischen Besitz desselben bin, ich dennoch  in einem anderen wirklichen (also nichtphysischen) Besitz desselben zu sein behaupten darf. – So werde ich einen \match{Apfel} nicht darum mein nennen, weil ich ihn in meiner Hand habe (physisch besitze), sondern nur, wenn ich sagen kann: ich besitze ihn, ob ich ihn gleich aus meiner Hand, wohin es auch sei, gelegt habe; imgleichen werde ich von dem Boden, auf den ich mich gelagert habe, nicht sagen können, er sei darum mein; sondern nur, wenn ich behaupten darf, er sei immer noch in meinem Besitz, ob ich gleich diesen Platz verlassen habe. Denn der, welcher mir im erstern Falle (des empirischen Besitzes) den Apfel aus der Hand winden, oder mich von meiner Lagerstätte wegschleppen wollte, würde mich zwar freilich in Ansehung des inneren Meinen (der Freiheit), aber nicht des äußeren Meinen lädieren, wenn ich nicht, auch ohne Inhabung, mich im Besitz des Gegenstandes zu sein behaupten könnte; ich könnte also diese Gegenstände (den Apfel und das Lager) auch nicht mein nennen. 
	
	\subsection*{tg433.2.28} 
	\textbf{Source : }Die Metaphysik der Sitten/Erster Teil. Metaphysische Anfangsgründe der Rechtslehre/1. Teil. Das Privatrecht vom äußeren Mein und Dein überhaupt/1. Hauptstück\\  
	
	\noindent\textbf{Paragraphe : }Alle Rechtssätze sind Sätze a priori, denn sie sind Vernunftgesetze (dictamina rationis). Der Rechtssatz a priori in Ansehung des empirischen Besitzes ist analytisch; denn er sagt nichts mehr, als was nach dem Satz des Widerspruchs aus dem letzteren folgt, daß nämlich, wenn ich Inhaber einer Sache (mit ihr also physisch verbunden) bin, derjenige, der sie wider meine Einwilligung affiziert (z.B. mir den \match{Apfel} aus der Hand reißt), das innere Meine (meine Freiheit) affiziere und schmälere, mithin in seiner Maxime mit dem Axiom des Rechts im geraden Widerspruch stehe. Der Satz von einem empirischen rechtmäßigen Besitz geht also nicht über das Recht einer Person in Ansehung ihrer selbst hinaus. 
	
	\unnumberedsection{Familie (9)} 
	\subsection*{tg437.2.38} 
	\textbf{Source : }Die Metaphysik der Sitten/Erster Teil. Metaphysische Anfangsgründe der Rechtslehre/1. Teil. Das Privatrecht vom äußeren Mein und Dein überhaupt/2. Hauptstück. Von der Art, etwas Äußeres zu erwerben/3. Abschnitt. Von dem auf dingliche Art persönlichen Recht\\  
	
	\noindent\textbf{Paragraphe : }Die Kinder des Hauses, die mit den Eltern zusammen eine \match{Familie} ausmachten, werden, auch ohne allen Vertrag  der Aufkündigung ihrer bisherigen Abhängigkeit, durch die bloße Gelangung zu dem Vermögen ihrer Selbsterhaltung (so wie es, teils als natürliche Volljährigkeit) dem allgemeinen Laufe der Natur überhaupt, teils ihrer besonderen Naturbeschaffenheit gemäß, eintritt, mündig (maiorennes). d.i. ihre eigene Herren (sui iuris), und erwerben dieses Recht ohne besonderen rechtlichen Akt, mithin bloß durchs Gesetz (lege) – sind den Eltern für ihre Erziehung nichts schuldig, so wie gegenseitig die letzteren ihrer Verbindlichkeit gegen diese auf ebendieselbe Art loswerden, hiemit beide ihre natürliche Freiheit gewinnen oder wieder gewinnen – die häusliche Gesellschaft aber, welche nach dem Gesetz notwendig war, nunmehr aufgelöset wird. 
	
	\subsection*{tg437.2.39} 
	\textbf{Source : }Die Metaphysik der Sitten/Erster Teil. Metaphysische Anfangsgründe der Rechtslehre/1. Teil. Das Privatrecht vom äußeren Mein und Dein überhaupt/2. Hauptstück. Von der Art, etwas Äußeres zu erwerben/3. Abschnitt. Von dem auf dingliche Art persönlichen Recht\\  
	
	\noindent\textbf{Paragraphe : }Beide Teile können nun wirklich ebendasselbe Hauswesen, aber in einer anderen Form der Verpflichtung, nämlich als Verknüpfung des Hausherren mit dem Gesinde (den Dienern oder Dienerinnen des Hauses), mithin eben diese häusliche Gesellschaft, aber jetzt als hausherrliche (societas herilis) erhalten, durch einen Vertrag, den der erstere mit den mündig gewordenen Kindern, oder, wenn die \match{Familie} keine Kinder hat, mit anderen freien Personen (der Hausgenossenschaft) eine häusliche Gesellschaft stiften, welche eine ungleiche Gesellschaft (des Gebietenden, oder der Herrschaft und der Gehorchenden, d.i. der Dienerschaft (imperantis et subiecti domestici)) sein würde. 
	
	\subsection*{tg437.2.7} 
	\textbf{Source : }Die Metaphysik der Sitten/Erster Teil. Metaphysische Anfangsgründe der Rechtslehre/1. Teil. Das Privatrecht vom äußeren Mein und Dein überhaupt/2. Hauptstück. Von der Art, etwas Äußeres zu erwerben/3. Abschnitt. Von dem auf dingliche Art persönlichen Recht\\  
	
	\noindent\textbf{Paragraphe : }Die Erwerbung nach diesem Gesetz ist dem Gegenstande nach dreierlei: Der Mann erwirbt ein Weib, das Paar erwirbt Kinder und die \match{Familie} Gesinde. – Alles dieses Erwerbliche ist zugleich unveräußerlich und das Recht des Besitzers dieser Gegenstände das allerpersönlichste. 
	
	\subsection*{tg442.2.4} 
	\textbf{Source : }Die Metaphysik der Sitten/Erster Teil. Metaphysische Anfangsgründe der Rechtslehre/2. Teil. Das öffentliche Recht/2. Abschnitt. Das Völkerrecht\\  
	
	\noindent\textbf{Paragraphe : }Die Menschen, welche ein Volk ausmachen, können, als Landeseingeborne, nach der Analogie der Erzeugung von einem gemeinschaftlichen Elternstamm (congeniti) vorgestellt werden, ob sie es gleich nicht sind: dennoch aber, in intellektueller und rechtlicher Bedeutung, als von einer gemeinschaftlichen Mutter (der Republik) geboren, gleichsam eine \match{Familie} (gens, natio) ausmachen, deren Glieder (Staatsbürger) alle ebenbürtig sind, und mit denen, die neben ihnen im Naturzustande leben möchten, als unedlen keine Vermischung eingehen, obgleich diese (die Wilden) ihrerseits sich wiederum wegen der gesetzlosen Freiheit, die sie gewählt haben, sich vornehmer dünken, die gleichfalls Völkerschaften, aber nicht Staaten, ausmachen. Das Recht der Staaten in Verhältnis zu einander (welches nicht ganz richtig im Deutschen das Völkerrecht genannt wird, sondern vielmehr das Staatenrecht (ius publicum civitatum) heißen sollte) ist nun dasjenige, was wir unter dem Namen des Völkerrechts zu betrachten haben: wo ein Staat, als eine moralische Person, gegen einen anderen im Zustande der natürlichen Freiheit, folglich auch dem des beständigen Krieges betrachtet, teils das Recht zum Kriege, teils das im Kriege, teils das, einander zu nötigen, aus diesem Kriegszustande herauszugehen, mithin eine den beharrlichen Frieden gründende Verfassung, d.i. das Recht nach dem Kriege zur Aufgabe macht, und führt nur das Unterscheidende von dem des Naturzustandes einzelner Menschen oder Familien (im Verhältnis gegen einander) von dem der Völker bei sich, daß im Völkerrecht nicht bloß ein Verhältnis eines Staats gegen den anderen im ganzen, sondern auch einzelner Personen des einen gegen einzelne des anderen, imgleichen gegen den ganzen anderen Staat selbst in Betrachtung kommt; welcher Unterschied aber vom Recht einzelner im  bloßen Naturzustande nur solcher Bestimmungen bedarf, die sich aus dem Begriffe des letzteren leicht folgern lassen. 
	
	\subsection*{tg445.2.24} 
	\textbf{Source : }Die Metaphysik der Sitten/Erster Teil. Metaphysische Anfangsgründe der Rechtslehre/Anhang erläutender Bemerkungen zu den metaphysischen Anhangsgründen der Rechtslehre\\  
	
	\noindent\textbf{Paragraphe : }Endlich, wenn bei eintretender Volljährigkeit die Pflicht der Eltern zur Erhaltung ihrer Kinder aufhört, so haben jene noch das Recht, diese als ihren Befehlen unterworfene Hausgenossen zu Erhaltung des Hauswesens zu brauchen, bis zur Entlassung derselben; welches eine Pflicht der Eltern gegen diese ist, die aus der natürlichen Beschränkung des Rechts der ersteren folgt. Bis dahin sind sie zwar Hausgenossen und gehören zur Familie, aber von nun an gehören sie zur Dienerschaft (famulatus) in derselben, die folglich nicht anders als durch Vertrag zu dem Seinen des Hausherrn (als seine Domestiken) hinzu kommen können. – Eben so kann auch eine Dienerschaft 
	außer der \match{Familie} zu dem Seinen des Hausherren nach einem auf dingliche Art persönlichen Rechte gemacht und als Gesinde (famulatus domesticus) durch Vertrag erworben werden. Ein solcher Vertrag ist nicht der einer bloßen Verdingung (locatio conductio operae) sondern der Hingebung seiner Person in den Besitz des Hausherrn, Vermietung (locatio conductio personae), welche darin von jener Verdingung unterschieden ist, daß das Gesinde sich zu allem Erlaubten versteht, was das Wohl des Hauswesens betrifft und ihm nicht, als bestellte und spezifisch bestimmte Arbeit, aufgetragen wird: Anstatt daß der zur bestimmten Arbeit Gedungene (Handwerker oder Tagelöhner) sich nicht zu dem Seinen des anderen hingibt und so auch kein Hausgenosse ist. – Des letzteren, weil er nicht im rechtlichen Besitz des anderen ist, der ihn zu gewissen Leistungen verpflichtet, kann der Hausherr, wenn jener auch sein häuslicher Einwohner (inquilinus) wäre, sich nicht (via facti) als einer Sache bemächtigen, sondern muß nach dem persönlichen Recht, auf die Leistung des Versprochenen dringen, welche ihm durch Rechtsmittel (via iuris) zu Gebote stehen. – – So viel zur Erläuterung und Verteidigung eines befremdlichen, neu hinzukommenden, Rechtstitels in der natürlichen Gesetzlehre, der doch, stillschweigend immer in Gebrauch gewesen ist. 
	
	\subsection*{tg445.2.46} 
	\textbf{Source : }Die Metaphysik der Sitten/Erster Teil. Metaphysische Anfangsgründe der Rechtslehre/Anhang erläutender Bemerkungen zu den metaphysischen Anhangsgründen der Rechtslehre\\  
	
	\noindent\textbf{Paragraphe : }
	Stiftung (sanctio testamentaria beneficii perpetui) ist die freiwillige, durch den Staat bestätigte, für gewisse auf einander folgende Glieder desselben, bis zu ihrem gänzlichen Aussterben, errichtete wohltätige Anstalt. – Sie heißt ewig, wenn die Verordnung zu Erhaltung derselben mit der Konstitution des Staats selbst vereinigt ist (denn der Staat muß für ewig angesehen werden); ihre Wohltätigkeit aber ist entweder für das Volk überhaupt oder für einen nach gewissen besonderen Grundsätzen vereinigten Teil desselben, einen Stand oder für eine \match{Familie} und die ewige Fortdauer ihrer Deszendenten abgezweckt. Ein Beispiel vom ersteren sind die Hospitäler, vom zweiten die Kirchen, vom dritten die Orden (geistliche und weltliche), vom vierten die Majorate. 
	
	\subsection*{tg445.2.65} 
	\textbf{Source : }Die Metaphysik der Sitten/Erster Teil. Metaphysische Anfangsgründe der Rechtslehre/Anhang erläutender Bemerkungen zu den metaphysischen Anhangsgründen der Rechtslehre\\  
	
	\noindent\textbf{Paragraphe : }Was endlich die Majoratsstiftung betrifft, da ein Gutsbesitzer durch Erbeseinsetzung verordnet: daß in der Reihe der auf einander folgenden Erben immer der Nächste von der \match{Familie} der Gutsherr sein solle (nach der Analogie mit einer monarchisch-erblichen Verfassung eines Staats, wo der Landesherr es ist), so kann eine solche Stiftung nicht allein mit Beistimmung aller Agnaten jederzeit aufgehoben werden und darf nicht auf ewige Zeiten – gleich als ob das Erbrecht am Boden haftete – immerwährend fortdauern, noch gesagt werden, es sei eine Verletzung der Stiftung und des Willens des Urahnherrn derselben, des Stifters, sie eingehen zu lassen: sondern der Staat hat auch hier ein Recht, ja sogar die Pflicht, bei den allmählich eintretenden Ursachen seiner eigenen Reform ein solches föderatives System seiner Untertanen, gleich als Unterkönige (nach der Analogie von Dynasten und Satrapen), wenn es erloschen ist, nicht weiter aufkommen zu lassen. 
	
	\subsection*{tg481.2.83} 
	\textbf{Source : }Die Metaphysik der Sitten/Zweiter Teil. Metaphysische Anfangsgründe der Tugendlehre/I. Ethische Elementarlehre/II. Teil. Von den Tugendpflichten gegen andere/Erstes Hauptstück. Von den Pflichten gegen andere, bloß als Menschen/Erster Abschnitt. Von der Liebespflicht gegen andere Menschen\\  
	
	\noindent\textbf{Paragraphe : }Sie machen die abscheuliche \match{Familie} des Neides, der Undankbarkeit und der Schadenfreude aus. – Der Haß ist aber hier nicht offen und gewalttätig, sondern geheim und verschleiert, welches zu der Pflichtvergessenheit gegen seinen Nächsten noch Niederträchtigkeit hinzutut, und so zugleich die Pflicht gegen sich selbst verletzt. 
	
	\subsection*{tg481.2.84} 
	\textbf{Source : }Die Metaphysik der Sitten/Zweiter Teil. Metaphysische Anfangsgründe der Tugendlehre/I. Ethische Elementarlehre/II. Teil. Von den Tugendpflichten gegen andere/Erstes Hauptstück. Von den Pflichten gegen andere, bloß als Menschen/Erster Abschnitt. Von der Liebespflicht gegen andere Menschen\\  
	
	\noindent\textbf{Paragraphe : }a) Der Neid (livor), als Hang, das Wohl anderer mit Schmerz, wahrzunehmen, ob zwar dem seinigen dadurch kein Abbruch geschieht, der, wenn er zur Tat (jenes Wohl zu schmälern) ausschlägt, qualifizierter Neid, sonst aber nur Mißgunst (invidentia) heißt, ist doch nur eine indirekt-bösartige Gesinnung, nämlich ein Unwille, unser eigen Wohl durch das Wohl anderer in Schatten gestellt zu sehen, weil wir den Maßstab desselben nicht in dessen innerem Wert, sondern nur in der Vergleichung mit dem Wohl anderer, zu schätzen, und diese Schätzung zu versinnlichen wissen. – Daher spricht man auch wohl von einer beneidungswürdigen Eintracht und Glückseligkeit in einer Ehe, oder \match{Familie} u.s.w.; gleich als ob es in manchen Fällen erlaubt wäre, jemanden zu beneiden. Die Regungen des Neides liegen also in der Natur des Menschen, und nur der Ausbruch derselben macht sie zu dem scheußlichen Laster einer grämischen, sich selbst folternden und auf Zerstörung des Glücks anderer, wenigstens dem Wunsche nach, gerichteten Leidenschaft, ist mithin der Pflicht des Menschen gegen sich selbst so wohl, als gegen andere entgegengesetzt. 
	
	\unnumberedsection{Fortpflanzung (1)} 
	\subsection*{tg475.2.4} 
	\textbf{Source : }Die Metaphysik der Sitten/Zweiter Teil. Metaphysische Anfangsgründe der Tugendlehre/I. Ethische Elementarlehre/I. Teil. Von den Pflichten gegen sich selbst überhaupt/Erstes Buch. Von den vollkommenen Pflichten gegen sich selbst/Zweites Hauptstück. Die Pflicht des Menschen gegen sich selbst, bloß als einem moralischen Wesen/Episodischer Abschnitt. Von der Amphibolie der moralischen Reflexionsbegriffe\\  
	
	\noindent\textbf{Paragraphe : }Diese vermeinte Pflicht kann nun auf unpersönliche, oder zwar persönliche aber schlechterdings unsichtbare (den äußeren Sinnen nicht darzustellende) Gegenstände bezogen werden. – Die erstere (außermenschliche) können der bloße Naturstoff, oder der zur \match{Fortpflanzung} organisierte, aber empfindungslose, oder der mit Empfindung und Willkür begabte Teil der Natur (Mineralien, Pflanzen, Tiere) sein; die zweite (übermenschliche) können als geistige Wesen (Engel, Gott) gedacht werden. – Ob zwischen Wesen beider Art und den Menschen ein Pflichtverhältnis, und welches dazwischen statt finde, wird nun gefragt. 
	
	\unnumberedsection{Gemeine (1)} 
	\subsection*{tg441.2.59} 
	\textbf{Source : }Die Metaphysik der Sitten/Erster Teil. Metaphysische Anfangsgründe der Rechtslehre/2. Teil. Das öffentliche Recht/1. Abschnitt. Das Staatsrecht\\  
	
	\noindent\textbf{Paragraphe : }Da auch das Kirchenwesen, welches von der Religion, als innerer Gesinnung, die ganz außer dem Wirkungskreise der bürgerlichen Macht ist, sorgfältig unterschieden werden muß (als Anstalt zum öffentlichen Gottesdienst für das Volk, aus welchem dieser auch seinen Ursprung hat, es sei Meinung oder Überzeugung), ein wahres Staatsbedürfnis  wird, sich auch als Untertanen einer höchsten unsichtbaren Macht, der sie huldigen müssen, und die mit der bürgerlichen oft in einen sehr ungleichen Streit kommen kann, zu betrachten: so hat der Staat das Recht, nicht etwa der inneren Konstitutionalgesetzgebung, das Kirchenwesen nach seinem Sinne, wie es ihm vorteilhaft dünkt, einzurichten, den Glauben und gottesdienstliche Formen (ritus) dem Volk vorzuschreiben, oder zu befehlen (denn dieses muß gänzlich den Lehrern und Vorstehern, die es sich selbst gewählt hat, überlassen bleiben), sondern nur das negative Recht, den Einfluß der öffentlichen Lehrer auf das sichtbare, politische gemeine Wesen, der der öffentlichen Ruhe nachteilig sein möchte, abzuhalten, mithin bei dem inneren Streit, oder dem der verschiedenen Kirchen unter einander, die bürgerliche Eintracht nicht in Gefahr kommen zu lassen, welches also ein Recht der Polizei ist. Daß eine Kirche einen gewissen Glauben, und welchen sie haben, oder daß sie ihn unabänderlich erhalten müsse, und sich nicht selbst reformieren dürfe, sind Einmischungen der obrigkeitlichen Gewalt, die unter ihrer Würde sind; weil sie sich dabei, als einem Schulgezänke, auf den Fuß der Gleichheit mit ihren Untertanen einläßt (der Monarch sich zum Priester macht), die ihr geradezu sagen können, daß sie hievon nichts verstehe; vornehmlich was des letztere, nämlich das Verbot innerer Reformen, betrifft; – denn, was das gesamte Volk nicht über sich selbst beschließen kann, das kann auch der Gesetzgeber nicht über das Volk beschließen. Nun kann aber kein Volk beschließen, in seinen den Glauben betreffenden Einsichten (der Aufklärung) niemals weiter fortzuschreiten, mithin auch sich in Ansehung des Kirchenwesens nie zu reformieren; weil dies der Menschheit in seiner eigenen Person, mithin dem höchsten Rechte desselben entgegen sein würde. Also kann es auch keine obrigkeitliche Gewalt über das Volk beschließen. – – Was aber die Kosten der Erhaltung des Kirchenwesens betrifft, so können diese, aus ebenderselben Ursache, nicht dem Staat, sondern müssen dem Teil des Volks, der sich zu einem oder dem anderen Glauben bekennt, d.i. nur der \match{Gemeine} zu Lasten kommen. 
	
	\unnumberedsection{Getreide (2)} 
	\subsection*{tg437.2.62} 
	\textbf{Source : }Die Metaphysik der Sitten/Erster Teil. Metaphysische Anfangsgründe der Rechtslehre/1. Teil. Das Privatrecht vom äußeren Mein und Dein überhaupt/2. Hauptstück. Von der Art, etwas Äußeres zu erwerben/3. Abschnitt. Von dem auf dingliche Art persönlichen Recht\\  
	
	\noindent\textbf{Paragraphe : }c) Die Anleihe (mutuum). Veräußerung einer Sache, unter der Bedingung, sie nur der Spezies nach wieder zu erhalten (z.B. \match{Getreide} gegen Getreide, oder Geld gegen Geld). 
	
	\subsection*{tg437.2.77} 
	\textbf{Source : }Die Metaphysik der Sitten/Erster Teil. Metaphysische Anfangsgründe der Rechtslehre/1. Teil. Das Privatrecht vom äußeren Mein und Dein überhaupt/2. Hauptstück. Von der Art, etwas Äußeres zu erwerben/3. Abschnitt. Von dem auf dingliche Art persönlichen Recht\\  
	
	\noindent\textbf{Paragraphe : }Ein Scheffel \match{Getreide} hat den größten direkten Wert als Mittel zu menschlichen Bedürfnissen. Man kann damit Tiere füttern, die uns zur Nahrung, zur Bewegung und zur Arbeit an unserer statt, und dann auch vermittelst desselben also Menschen vermehren und erhalten, welche nicht allein jene Naturprodukte immer wieder erzeugen, sondern auch durch Kunstprodukte allen unseren Bedürfnissen zu Hülfe kommen können; zur Verfertigung unserer Wohnung, Kleidung, ausgesuchtem Genusse und aller Gemächlichkeit überhaupt, welche die Güter der Industrie ausmachen. Der Wert des Geldes ist dagegen nur indirekt. Man kann es selbst nicht genießen, oder als ein solches irgend wozu unmittelbar gebrauchen; aber doch ist es ein Mittel, was unter allen Sachen von der höchsten Brauchbarkeit ist. 
	
	\unnumberedsection{Große (2)} 
	\subsection*{tg450.2.10} 
	\textbf{Source : }Die Metaphysik der Sitten/Zweiter Teil. Metaphysische Anfangsgründe der Tugendlehre/Einleitung/II. Erörterung des Begriffs von einem Zwecke, der zugleich Pflicht ist\\  
	
	\noindent\textbf{Paragraphe : }Der Tugend = + a ist die negative Untugend (moralische Schwäche) = 0 als logisches Gegenteil (contradictorie oppositum), das Laster aber = – a als Widerspiel (contrarie s. realiter oppositum) entgegen gesetzt und es ist eine, nicht bloß unnötige, sondern auch anstößige Frage: ob zu großen Verbrechen nicht etwa mehr Stärke der Seele als selbst zu großen Tugenden gehöre. Denn unter Stärke der Seele verstehen wir die Stärke des Vorsatzes eines Menschen, als mit Freiheit begabten Wesens, mithin so fern er seiner selbst mächtig (bei Sinnen) ist, also im gesunden Zustande des Menschen. \match{Große} Verbrechen aber sind Paroxysmen, deren Anblick den an Seele gesunden Menschen schaudern macht. Die Frage würde also etwa dahin auslaufen: ob ein Mensch im Anfall einer Krankheit mehr physische Stärke haben könne, als wenn er bei Sinnen ist; welches  man einräumen kann, ohne ihm darum mehr Seelenstärke beizulegen, wenn man unter Seele das Lebensprinzip des Menschen im freien Gebrauch seiner Kräfte versteht. Denn, weil jene bloß in der Macht der die Vernunft schwächenden Neigungen ihren Grund haben, welches keine Seelenstärke beweiset, so würde diese Frage mit der ziemlich auf einerlei hinauslaufen: ob ein Mensch im Anfall einer Krankheit mehr Stärke als im gesunden Zustande beweisen könne, welche geradezu verneinend beantwortet werden kann, weil der Mangel der Gesundheit, die im Gleichgewicht aller körperlichen Kräfte des Menschen besteht, eine Schwächung im System dieser Kräfte ist, nach welchem man allein die absolute Gesundheit beurteilen kann. 
	
	\subsection*{tg461.2.12} 
	\textbf{Source : }Die Metaphysik der Sitten/Zweiter Teil. Metaphysische Anfangsgründe der Tugendlehre/Einleitung/XIII. Allgemeine Grundsätze der Metaphysik der Sitten in Behandlung einer reinen Tugendlehre\\  
	
	\noindent\textbf{Paragraphe : }Tugend bedeutet eine moralische Stärke des Willens. Aber dies erschöpft noch nicht den Begriff; denn eine solche Stärke könnte auch einem heiligen (übermenschlichen) Wesen zukommen, in welchem kein hindernder Antrieb dem Gesetze seines Willens entgegen wirkt; das also alles dem Gesetz gemäß gerne tut. Tugend ist also die moralische Stärke des Willens eines Menschen in Befolgung seiner Pflicht: welche eine moralische Nötigung durch seine eigene gesetzgebende Vernunft ist, insofern diese sich zu einer das Gesetz ausführenden Gewalt selbst konstituiert. – Sie ist nicht selbst, oder sie zu besitzen ist nicht Pflicht (denn sonst würde es eine Verpflichtung zur Pflicht geben müssen), sondern sie gebietet und begleitet ihr Gebot durch einen sittlichen (nach Gesetzen der inneren Freiheit möglichen) Zwang; wozu aber, weil er unwiderstehlich sein soll, Stärke erforderlich ist, deren Grad wir nur durch die \match{Große} der Hindernisse, die der Mensch durch seine Neigungen sich selber schafft, schätzen können. Die Laster, als die Brut gesetzwidriger Gesinnungen, sind die Ungeheuer, die er nun zu bekämpfen hat: weshalb diese sittliche Stärke auch, als Tapferkeit (fortitudo moralis), die größte und einzige wahre Kriegsehre des Menschen ausmacht; auch wird sie die eigentliche, nämlich praktische Weisheit genannt: weil sie den Endzweck des Daseins der Menschen auf Erden zu dem ihrigen macht. – In ihrem Besitz ist der Mensch allein frei, gesund, reich, ein König u.s.w. und kann, weder durch Zufall, noch Schicksal einbüßen; weil er sich selbst besitzt und der Tugendhafte seine Tugend nicht verlieren kann. 
	
	\unnumberedsection{Krone (2)} 
	\subsection*{tg431.2.32} 
	\textbf{Source : }Die Metaphysik der Sitten/Erster Teil. Metaphysische Anfangsgründe der Rechtslehre/Einleitung in die Rechtslehre\\  
	
	\noindent\textbf{Paragraphe : }Hieraus folgt auch, daß ein Gerichtshof der Billigkeit (in einem Streit anderer über ihre Rechte) einen Widerspruch in sich schließe. Nur da, wo es die eigenen Rechte des Richters betrifft, und in dem, worüber er für seine Person disponieren kann, darf und soll er der Billigkeit Gehör geben; z.B. wenn die \match{Krone} den Schaden, den andre in ihrem Dienste erlitten haben, und den sie zu vergüten angeflehet wird, selber trägt, ob sie gleich, nach dem strengen Rechte, diesen Ausspruch unter der Vorschützung, daß sie solche auf ihre eigene Gefahr übernommen haben, abweisen könnte. 
	
	\subsection*{tg489.2.17} 
	\textbf{Source : }Die Metaphysik der Sitten/Fußnoten\\  
	
	\noindent\textbf{Paragraphe : }
	
	8 Weil die Entthronung eines Monarchen doch auch als freiwillige Ablegung der \match{Krone} und Niederlegung seiner Gewalt, mit Zurückgebung derselben an das Volk, gedacht werden kann, oder auch als eine, ohne Vergreifung an der höchsten Person, vorgenommene Verlassung derselben, wodurch sie in den Privatstand versetzt werden würde, so hat das Verbrechen des Volks, welches sie erzwang, doch noch wenigstens den Vorwand des Notrechts (casus necessitatis) für sich, niemals aber das mindeste Recht, ihn, das Oberhaupt, wegen der vorigen Verwaltung zu strafen; weil alles, was er vorher in der Qualität eines Oberhaupts tat, als äußerlich rechtmäßig geschehen angesehen werden muß, und er selbst, als Quell der Gesetze betrachtet, nicht unrecht tun kann. Unter allen Greueln einer Staatsumwälzung durch Aufruhr ist selbst die Ermordung des Monarchen noch nicht das Ärgste: denn noch kann man sich vorstellen, sie geschehe vom Volk aus Furcht, er könne, wenn er am Leben bleibt, sich wieder ermannen, und jenes die verdiente Strafe fühlen lassen, und solle also nicht eine Verfügung der Strafgerechtigkeit, sondern bloß der Selbsterhaltung sein. Die formale Hinrichtung ist es, was die mit Ideen des Menschenrechts erfüllete Seele mit einem Schaudern ergreift, das man wiederholentlich fühlt, so bald und so oft man sich diesen Auftritt denkt, wie das Schicksal Karls I. oder Ludwigs XVI. Wie erklärt man sich aber dieses Gefühl, was hier nicht ästhetisch (ein Mitgefühl, Wirkung der Einbildungskraft, die sich in die Stelle des Leidenden versetzt), sondern moralisch, der gänzlichen Umkehrung aller Rechtsbegriffe ist? Es wird als Verbrechen, was ewig bleibt, und nie ausgetilgt werden kann (crimen immortale, inexpiabile), angesehen, und scheint demjenigen ähnlich zu sein, was die Theologen diejenige Sünde nennen, welche weder in dieser noch in jener Welt vergeben werden kann. Die Erklärung dieses Phänomens im menschlichen Gemüte scheint aus folgenden Reflexionen über sich selbst, die selbst auf die staatsrechtlichen Prinzipien ein Licht werfen, hervorzugehen. 
	
	\unnumberedsection{Ordnung (5)} 
	\subsection*{tg441.2.41} 
	\textbf{Source : }Die Metaphysik der Sitten/Erster Teil. Metaphysische Anfangsgründe der Rechtslehre/2. Teil. Das öffentliche Recht/1. Abschnitt. Das Staatsrecht\\  
	
	\noindent\textbf{Paragraphe : }Der Ursprung der obersten Gewalt ist für das Volk, das unter derselben steht, in praktischer Absicht unerforschlich: d.i. der Untertan soll nicht über diesen Ursprung, als ein noch in Ansehung des ihr schuldigen Gehorsams zu bezweifelndes Recht (ius controversum), werktätig vernünfteln. Denn, da das Volk, um rechtskräftig über die oberste Staatsgewalt (summum imperium) zu urteilen, schon als unter einem allgemein gesetzgebenden Willen vereint angesehen werden muß, so kann und darf es nicht anders urteilen, als das gegenwärtige Staatsoberhaupt (summus imperans) es will. – Ob ursprünglich ein wirklicher Vertrag der Unterwerfung unter denselben (pactum subiectionis  civilis) als ein Faktum vorher gegangen, oder ob die Gewalt vorherging, und das Gesetz nur hintennach gekommen sei, oder auch in dieser \match{Ordnung} sich habe folgen sollen: das sind für das Volk, das nun schon unter dem bürgerlichen Gesetze steht, ganz zweckleere, und doch den Staat mit Gefahr bedrohende Vernünfteleien; denn, wollte der Untertan, der den letzteren Ursprung nun ergrübelt hätte, sich jener jetzt herrschenden Autorität widersetzen, so würde er nach den Gesetzen derselben, d.i. mit allem Recht, bestraft, vertilgt, oder (als vogelfrei, exlex) ausgestoßen werden. – Ein Gesetz, das so heilig (unverletzlich) ist, daß es, praktisch, auch nur in Zweifel zu ziehen, mithin seinen Effekt einen Augenblick zu suspendieren, schon ein Verbrechen ist, wird so vorgestellt, als ob es nicht von Menschen, aber doch von irgend einem höchsten tadelfreien Gesetzgeber herkommen müsse, und das ist die Bedeutung des Satzes: »alle Obrigkeit ist von Gott«, welcher nicht einen Geschichtsgrund der bürgerlichen Verfassung, sondern eine Idee, als praktisches Vernunftprinzip, aussagt: der jetzt bestehenden gesetzgebenden Gewalt gehorchen zu sollen; ihr Ursprung mag sein, welcher er wolle. 
	
	\subsection*{tg441.2.46} 
	\textbf{Source : }Die Metaphysik der Sitten/Erster Teil. Metaphysische Anfangsgründe der Rechtslehre/2. Teil. Das öffentliche Recht/1. Abschnitt. Das Staatsrecht\\  
	
	\noindent\textbf{Paragraphe : }Übrigens, wenn eine Revolution einmal gelungen, und eine neue Verfassung gegründet ist, so kann die Unrechtmäßigkeit des Beginnens und der Vollführung derselben die Untertanen von der Verbindlichkeit, der neuen \match{Ordnung} der Dinge sich, als gute Staatsbürger, zu fügen, nicht befreien, und sie können sich nicht weigern, derjenigen Obrigkeit ehrlich zu gehorchen, die jetzt die Gewalt hat. Der entthronte Monarch (der jene Umwälzung überlebt) kann wegen seiner vorigen Geschäftsführung nicht in Anspruch genommen, noch weniger aber gestraft werden, wenn er, in  den Stand eines Staatsbürgers zurückgetreten, seine und des Staats Ruhe dem Wagstück vorzieht, sich von diesem zu entfernen, um als Prätendent das Abenteuer der Wiedererlangung desselben, es sei durch ingeheim angestiftete Gegenrevolution, oder durch Beistand anderer Mächte, zu bestehen. Wenn er aber das letztere vorzieht, so bleibt ihm, weil der Aufruhr, der ihn aus seinem Besitz vertrieb, ungerecht war, sein Recht an demselben unbenommen. Ob aber andere Mächte das Recht haben, sich, diesem verunglückten Oberhaupt zum Besten, in ein Staatenbündnis zu vereinigen, bloß um jenes vom Volk begangene Verbrechen nicht ungeahndet, noch als Skandal für alle Staaten bestehen zu lassen, mithin eine in jedem anderen Staat durch Revolution zu Stande gekommene Verfassung in ihre alte mit Gewalt zurückzubringen berechtigt und berufen sein, das gehört zum Völkerrecht. 
	
	\subsection*{tg481.2.91} 
	\textbf{Source : }Die Metaphysik der Sitten/Zweiter Teil. Metaphysische Anfangsgründe der Tugendlehre/I. Ethische Elementarlehre/II. Teil. Von den Tugendpflichten gegen andere/Erstes Hauptstück. Von den Pflichten gegen andere, bloß als Menschen/Erster Abschnitt. Von der Liebespflicht gegen andere Menschen\\  
	
	\noindent\textbf{Paragraphe : }Alle Laster, welche selbst die menschliche Natur hassenswert machen würden, wenn man sie (als qualifiziert) in der Bedeutung von Grundsätzen nehmen wollte, sind inhuman, objektiv betrachtet, aber doch menschlich, subjektiv erwogen: d.i. wie die Erfahrung uns unsere Gattung kennen lehrt. Ob man also zwar einige derselben in der Heftigkeit des Abscheues teuflisch nennen möchte, so wie ihr Gegenstück Engelstugend genannt werden könnte: so sind beide Begriffe doch nur Ideen von einem Maximum, als Maßstab zum Behuf der Vergleichung des Grades der Moralität gedacht, indem man dem Menschen seinen Platz im Himmel oder der Hölle anweiset, ohne aus ihm ein Mittelwesen, was weder den einen dieser Plätze, noch den anderen einnimmt, zu machen. Ob es Haller, mit seinem »zweideutig Mittelding von Engeln und von Vieh«, besser getroffen habe, mag hier unausgemacht bleiben. Aber das Halbieren. in einer Zusammenstellung heterogener Dinge führt auf gar keinen bestimmten Begriff, und zu diesem kann uns in der \match{Ordnung} der Wesen nach ihrem uns unbekannten Klassenunterschiede nichts hinleiten. Die erstere Gegeneinanderstellung (von Engelstugend und teuflischem  Laster) ist Übertreibung. Die zweite, ob zwar Menschen leider! auch in viehische Laster fallen, berechtigt doch nicht, eine zu ihrer Spezies gehörige Anlage dazu ihnen beizulegen, so wenig, als die Verkrüppelung einiger Bäume im Walde ein Grund ist, sie zu einer besondern Art von Gewächsen zu machen. 
	
	\subsection*{tg486.2.32} 
	\textbf{Source : }Die Metaphysik der Sitten/Zweiter Teil. Metaphysische Anfangsgründe der Tugendlehre/II. Ethische Methodenlehre/1. Abschnitt. Die ethische Didaktik\\  
	
	\noindent\textbf{Paragraphe : }8. L. Hat die Vernunft wohl Gründe für sich, eine solche, die Glückseligkeit nach Verdienst und Schuld der  Menschen austeilende, über die ganze Natur gebietende und die Welt mit höchster Weisheit regierende Macht als wirklich anzunehmen, d.i. an Gott zu glauben? S. Ja; denn wir sehen an den Werken der Natur, die wir beurteilen können, so ausgebreitete und tiefe Weisheit, die wir uns nicht anders als durch eine unaussprechlich große Kunst eines Weltschöpfers erklären können, von welchem wir uns denn auch, was die sittliche \match{Ordnung} betrifft, in der doch die höchste Zierde der Welt besteht, eine nicht minder weise Regierung zu versprechen Ursache haben: nämlich, daß, wenn wir uns nicht selbst der Glückseligkeit unwürdig machen, welches durch Übertretung unserer Pflicht geschieht, wir auch hoffen können, ihrer teilhaftig zu werden. 
	
	\subsection*{tg486.2.35} 
	\textbf{Source : }Die Metaphysik der Sitten/Zweiter Teil. Metaphysische Anfangsgründe der Tugendlehre/II. Ethische Methodenlehre/1. Abschnitt. Die ethische Didaktik\\  
	
	\noindent\textbf{Paragraphe : }Wenn dieses nun weislich und pünktlich nach Verschiedenheit der Stufen des Alters, des Geschlechts und des Standes, die der Mensch nach und nach betritt, aus der eigenen Vernunft des Menschen entwickelt worden,  so ist noch etwas, was den Beschluß machen muß, was die Seele inniglich bewegt und den Menschen auf eine Stelle setzt, wo er sich selbst nicht anders als mit der größten Bewunderung der ihm beiwohnenden ursprünglichen Anlagen betrachten kann, und wovon der Eindruck nie erlischt. – Wenn ihm nämlich beim Schlüsse seiner Unterweisung seine Pflichten in ihrer \match{Ordnung} noch einmal summarisch vorerzählt (rekapituliert), wenn er, bei jeder derselben, darauf aufmerksam gemacht wird, daß alle Übel, Drangsale und Leiden des Lebens, selbst Bedrohung mit dem Tode, die ihn darüber, daß er seiner Pflicht treu gehorcht, treffen mögen, ihm doch das Bewußtsein, über sie alle erhoben und Meister zu sein, nicht rauben können, so liegt ihm nun die Frage ganz nahe: was ist das in dir, was sich getrauen darf, mit allen Kräften der Natur in dir und um dich in Kampf zu treten und sie, wenn sie mit deinen sittlichen Grundsätzen in Streit kommen, zu besiegen? Wenn diese Frage, deren Auflösung das Vermögen der spekulativen Vernunft gänzlich übersteigt und die sich dennoch von selbst einstellt, ans Herz gelegt wird, so muß selbst die Unbegreiflichkeit in diesem Selbsterkenntnisse der Seele eine Erhebung geben, die sie zum Heilighalten ihrer Pflicht nur desto stärker belebt, je mehr sie angefochten wird. 
	
	\unnumberedsection{Wurzel (1)} 
	\subsection*{tg450.2.3} 
	\textbf{Source : }Die Metaphysik der Sitten/Zweiter Teil. Metaphysische Anfangsgründe der Tugendlehre/Einleitung/II. Erörterung des Begriffs von einem Zwecke, der zugleich Pflicht ist\\  
	
	\noindent\textbf{Paragraphe : }Die Ethik aber nimmt einen entgegengesetzten Weg. Sie kann nicht von den Zwecken ausgehen, die der Mensch sich setzen mag und darnach über seine zu nehmende Maximen,  d.i. über seine Pflicht, verfügen; denn das wären empirische Gründe der Maximen, die keinen Pflichtbegriff abgeben, als welcher (das kategorische Sollen) in der reinen Vernunft allein seine \match{Wurzel} hat; wie denn auch, wenn die Maximen nach jenen Zwecken (welche alle selbstsüchtig sind) genommen werden sollten, vom Pflichtbegriff eigentlich gar nicht die Rede sein könnte. – Also wird in der Ethik der Pflichtbegriff auf Zwecke leiten und die Maximen, in Ansehung der Zwecke, die wir uns setzen sollen, nach moralischen Grundsätzen begründen müssen. 
	
	\unnumberedchapter{Monde} 
	\unnumberedsection{All (1)} 
	\subsection*{tg453.2.3} 
	\textbf{Source : }Die Metaphysik der Sitten/Zweiter Teil. Metaphysische Anfangsgründe der Tugendlehre/Einleitung/V. Erläuterung dieser zwei Begriffe\\  
	
	\noindent\textbf{Paragraphe : }Das Wort Vollkommenheit ist mancher Mißdeutung ausgesetzt. Es wird bisweilen als ein zur Transzendentalphilosophie gehörender Begriff der Allheit des Mannigfaltigen, was zusammengenommen ein Ding ausmacht, – dann aber auch, als zur Teleologie gehörend, so verstanden, daß es die Zusammenstimmung der Beschaffenheiten eines Dinges zu einem Zwecke bedeutet. Man könnte die Vollkommenheit in der ersteren Bedeutung die quantitative (materiale), in der zweiten die qualitative (formale) Vollkommenheit nennen. Jene kann nur eine sein (denn das \match{All} des einem Dinge Zugehörigen ist eins). Von dieser aber kann es in einem Dinge mehrere geben; und von der letzteren wird hier auch eigentlich gehandelt. 
	
	\unnumberedsection{Bekampfung (2)} 
	\subsection*{tg486.2.7} 
	\textbf{Source : }Die Metaphysik der Sitten/Zweiter Teil. Metaphysische Anfangsgründe der Tugendlehre/II. Ethische Methodenlehre/1. Abschnitt. Die ethische Didaktik\\  
	
	\noindent\textbf{Paragraphe : }Daß sie könne und müsse gelehrt werden, folgt schon daraus, daß sie nicht angeboren ist; die Tugendlehre ist also eine Doktrin. Weil aber durch die bloße Lehre, wie man sich verhalten solle, um dem Tugendbegriffe angemessen zu sein, die Kraft zur Ausübung der Regeln noch nicht erworben wird, so meinten die Stoiker hiemit nur, die Tugend könne nicht durch bloße Vorstellungen der Pflicht, durch Ermahnungen (paränetisch) gelehrt, sondern sie müsse durch Versuche der \match{Bekämpfung} des inneren Feindes im Menschen (asketisch) kultiviert, geübt werden; denn man kann nicht alles so fort was man will, wenn man nicht vorher seine Kräfte versucht, und geübt hat, wozu aber freilich die Entschließung auf einmal vollständig genommen werden muß; weil die Gesinnung (animus) sonst, bei einer Kapitulation mit dem Laster, um es allmählich zu verlassen, an sich unlauter und selbst lasterhaft sein, mithin auch keine Tugend (als die auf einem einzigen Prinzip beruhet) hervorbringen könnte. 
	
	\subsection*{tg487.2.5} 
	\textbf{Source : }Die Metaphysik der Sitten/Zweiter Teil. Metaphysische Anfangsgründe der Tugendlehre/II. Ethische Methodenlehre/2. Abschnitt. Die ethische Asketik\\  
	
	\noindent\textbf{Paragraphe : }Die Kultur der Tugend, d.i. die moralische Asketik, hat, in Ansehung des Prinzips der rüstigen, mutigen und wackeren Tugendübung den Wahlspruch der Stoiker: gewöhne dich, die zufälligen Lebensübel zu ertragen und die eben so überflüssigen Ergötzlichkeiten zu entbehren (assuesce incommodis et desuesce commoditatibus vitae). Es ist eine Art von Diätetik für den Menschen, sich moralisch gesund zu erhalten. Gesundheit ist aber nur ein negatives  Wohlbefinden, sie selber kann nicht gefühlt werden. Es muß etwas dazu kommen, was einen angenehmen Lebensgenuß gewährt und doch bloß moralisch ist. Das ist das jederzeit fröhliche Herz in der Idee des tugendhaften Epikurs. Denn wer sollte wohl mehr Ursache haben, frohen Muts zu sein und nicht darin selbst eine Pflicht finden, sich in eine fröhliche Gemütsstimmung zu versetzen und sie sich habituell zu machen, als der, welcher sich keiner vorsätzlichen Übertretung bewußt und, wegen des Verfalls in eine solche, gesichert ist (hic murus ahenëus esto etc. Horat.). – Die Mönchsasketik hingegen, welche aus abergläubischer Furcht, oder geheucheltem Abscheu an sich selbst, mit Selbstpeinigung und Fleischeskreuzigung zu Werke geht, zweckt auch nicht auf Tugend, sondern auf schwärmerische Entsündigung ab, sich selbst Strafe aufzulegen und, anstatt sie moralisch (d.i. in Absicht auf die Besserung) zu bereuen, sie büßen zu wollen; welches, bei einer selbstgewählten und an sich vollstreckten Strafe (denn die muß immer ein anderer auflegen), ein Widerspruch ist, und kann auch den Frohsinn, der die Tugend begleitet, nicht bewirken, vielmehr nicht ohne geheimen Haß gegen das Tugendgebot statt finden. – Die ethische Gymnastik besteht also nur in der \match{Bekämpfung} der Naturtriebe, die das Maß erreicht, über sie bei vorkommenden, der Moralität Gefahr drohenden, Fällen Meister werden zu können; mithin die wacker und, im Bewußtsein seiner wiedererworbenen Freiheit, fröhlich macht. Etwas bereuen (welches bei der Rückerinnerung ehemaliger Übertretungen unvermeidlich, ja wobei diese Erinnerung nicht schwinden zu lassen es so gar Pflicht ist) und sich eine Pönitenz auferlegen (z.B. das Fasten), nicht in diätetischer, sondern frommer Rücksicht, sind zwei sehr verschiedene, moralisch gemeinte, Vorkehrungen, von denen die letztere, welche freudenlos, finster und mürrisch ist, die Tugend selbst verhaßt macht und ihre Anhänger verjagt. Die Zucht (Disziplin), die der Mensch an sich selbst verübt, kann daher nur durch den Frohsinn, der sie begleitet, verdienstlich und exemplarisch werden. 
	
	\unnumberedsection{Boden (12)} 
	\subsection*{tg433.2.31} 
	\textbf{Source : }Die Metaphysik der Sitten/Erster Teil. Metaphysische Anfangsgründe der Rechtslehre/1. Teil. Das Privatrecht vom äußeren Mein und Dein überhaupt/1. Hauptstück\\  
	
	\noindent\textbf{Paragraphe : }Wenn auch gleich ein \match{Boden} als frei, d.i. zu jedermanns Gebrauch offen angesehen, oder dafür erklärt würde, so kann man doch nicht sagen, daß er es von Natur und ursprünglich, vor allem rechtlichem Akt, frei sei, denn auch das wäre ein Verhältnis zu Sachen, nämlich dem Boden, der jedermann seinen Besitz verweigerte, sondern, weil diese Freiheit des Bodens ein Verbot für jedermann sein würde, sich desselben zu bedienen; wozu ein gemeinsamer Besitz desselben erfordert wird, der ohne Vertrag nicht statt finden kann. Ein Boden aber, der nur durch diesen frei sein kann, muß wirklich im Besitze aller derer (zusammen verbundenen) sein, die sich wechselseitig den Gebrauch desselben untersagen, oder ihn suspendieren. 
	
	\subsection*{tg435.2.26} 
	\textbf{Source : }Die Metaphysik der Sitten/Erster Teil. Metaphysische Anfangsgründe der Rechtslehre/1. Teil. Das Privatrecht vom äußeren Mein und Dein überhaupt/2. Hauptstück. Von der Art, etwas Äußeres zu erwerben/1. Abschnitt. Vom Sachrecht\\  
	
	\noindent\textbf{Paragraphe : }Es ist die Frage: wie weit erstreckt sich die Befugnis der Besitznehmung eines Bodens? So weit, als das Vermögen, ihn in seiner Gewalt zu haben, d.i. als der, so ihn sich zueignen will, ihn verteidigen kann, gleich als ob der \match{Boden} spräche: wenn ihr mich nicht beschützen könnt, so könnt ihr mir auch nicht gebieten. Darnach müßte also auch der Streit über das freie oder verschlossene Meer entschieden werden; z.B. innerhalb der Weite, wohin die Kanonen reichen, darf niemand an der Küste eines Landes, das schon einem gewissen  Staat zugehört, fischen, Bernstein aus dem Grunde der See holen, u. dergl. – Ferner: ist die Bearbeitung des Bodens (Bebauung, Beackerung, Entwässerung u. dergl.) zur Erwerbung desselben notwendig? Nein! denn, da diese Formen (der Spezifizierung) nur Akzidenzen sind, so machen sie kein Objekt eines unmittelbaren Besitzes aus, und können zu dem des Subjekts nur gehören, so fern die Substanz vorher als das Seine desselben anerkannt ist. Die Bearbeitung ist, wenn es auf die Frage von der ersten Erwerbung ankommt, nichts weiter als ein äußeres Zeichen der Besitznehmung, welches man durch viele andere, die weniger Mühe kosten, ersetzen kann. – Ferner: darf man wohl jemanden in dem Akt seiner Besitznehmung hindern, so daß keiner von beiden des Rechts der Priorität teilhaftig werde, und so der Boden immer als keinem angehörig frei bleibe? Gänzlich kann diese Hinderung nicht statt finden, weil der andere, um dieses tun zu können, sich doch auch selbst auf irgend einem benachbarten Boden befinden muß, wo er also selbst behindert werden kann zu sein, mithin eine absolute Verhinderung ein Widerspruch wäre; aber respektiv auf einen gewissen (zwischenliegenden) Boden, diesen, als neutral, zur Scheidung zweier Benachbarten unbenutzt liegen zu lassen, würde doch mit dem Rechte der Bemächtigung zusammen bestehen; aber alsdann gehört wirklich dieser Boden beiden gemeinschaftlich, und ist nicht herrenlos (res nullius), eben darum, weil er von beiden dazu gebraucht wird, um sie von einander zu scheiden. – Ferner: kann man auf einem Boden, davon kein Teil das Seine von jemanden ist, doch eine Sache als die seine haben? Ja, wie in der Mongolei jeder sein Gepäcke, was er hat, liegen lassen, oder sein Pferd, was ihm entlaufen ist, als das Seine in seinen Besitz bringen kann, weil der ganze Boden dem Volk, der Gebrauch desselben also jedem einzelnen zusteht; daß aber jemand eine bewegliche Sache auf dem Boden eines anderen als das Seine haben kann, ist zwar möglich, aber nur durch Vertrag. – Endlich ist die Frage: können zwei benachbarte  Völker (oder Familien) einander widerstehen, eine gewisse Art des Gebrauchs eines Bodens anzunehmen, z.B. die Jagdvölker dem Hirtenvolk, oder den Ackerleuten, oder diese den Pflanzern, u. dergl.? Allerdings; denn die Art, wie sie sich auf dem Erdboden überhaupt ansässig machen wollen, ist, wenn sie sich innerhalb ihrer Grenzen halten, eine Sache des bloßen Beliebens (res merae facultatis). 
	
	\subsection*{tg435.2.38} 
	\textbf{Source : }Die Metaphysik der Sitten/Erster Teil. Metaphysische Anfangsgründe der Rechtslehre/1. Teil. Das Privatrecht vom äußeren Mein und Dein überhaupt/2. Hauptstück. Von der Art, etwas Äußeres zu erwerben/1. Abschnitt. Vom Sachrecht\\  
	
	\noindent\textbf{Paragraphe : }Was die Körper auf einem \match{Boden} betritt, der schon der meinige ist, so gehören sie, wenn sie sonst keines anderen sind, mir zu, ohne daß ich zu diesem Zweck eines besonderen rechtlichen Akts bedürfte (nicht facto sondern lege); nämlich, weil sie als der Substanz inhärierende Akzidenzen betrachtet werden können (iure rei  meae), wozu auch alles gehört, was mit meiner Sache so verbunden ist, daß ein anderer sie von dem Meinen nicht trennen kann, ohne dieses selbst zu verändern (z.B. Vergoldung, Mischung eines mir zugehörigen Stoffes mit andern Materien, Anspülung oder auch Veränderung des anstoßenden Strombettes, und dadurch geschehende Erweiterung meines Bodens, u.s.w.). Ob aber der erwerbliche Boden sich noch weiter als das Land, nämlich auch auf eine Strecke des Seegrundes hin aus (das Recht, noch an meinen Ufern zu fischen, oder Bernstein herauszubringen, u. dergl.), sich ausdehnen lasse, muß nach ebendenselben Grundsätzen beurteilt werden. So weit ich aus meinem Sitze mechanisches Vermögen habe, meinen Boden gegen den Eingriff anderer zu sichern (z.B. so weit die Kanonen vom Ufer abreichen), gehört zu meinem Besitz und das Meer ist bis dahin geschlossen (mare clausum). Da aber auf dem weiten Meere selbst kein Sitz möglich ist, so kann der Besitz auch nicht bis dahin ausgedehnt werden und offene See ist frei (mare liberum). Das Stranden aber, es sei der Menschen, oder der ihnen zugehörigen Sachen, kann, als unvorsätzlich, von dem Strandeigentümer nicht zum Erwerbrecht gezählt werden; weil es nicht Läsion (ja überhaupt kein Faktum) ist, und die Sache, die auf einen Boden geraten ist, der doch irgend einem angehört, nicht als res nullius behandelt werden kann. Ein Fluß dagegen kann, so weit der Besitz seines Ufers reicht, so gut wie ein jeder Landboden, unter obbenannten Einschränkungen ursprünglich von dem erworben werden, der im Besitz beider Ufer ist. 
	
	\subsection*{tg435.2.8} 
	\textbf{Source : }Die Metaphysik der Sitten/Erster Teil. Metaphysische Anfangsgründe der Rechtslehre/1. Teil. Das Privatrecht vom äußeren Mein und Dein überhaupt/2. Hauptstück. Von der Art, etwas Äußeres zu erwerben/1. Abschnitt. Vom Sachrecht\\  
	
	\noindent\textbf{Paragraphe : }Der \match{Boden} (unter welchem alles bewohnbare Land verstanden wird) ist, in Ansehung alles Beweglichen auf demselben, als Substanz, die Existenz des letzteren aber nur als Inhärenz zu betrachten und so, wie Im theoretischen Sinne die Akzidenzen nicht außerhalb der Substanz existieren können, so kann im praktischen das Bewegliche auf dem Boden nicht das Seine von jemanden sein, wenn dieser nicht vorher als im rechtlichen Besitz desselben befindlich (als das Seine desselben) angenommen wird. 
	
	\subsection*{tg435.2.9} 
	\textbf{Source : }Die Metaphysik der Sitten/Erster Teil. Metaphysische Anfangsgründe der Rechtslehre/1. Teil. Das Privatrecht vom äußeren Mein und Dein überhaupt/2. Hauptstück. Von der Art, etwas Äußeres zu erwerben/1. Abschnitt. Vom Sachrecht\\  
	
	\noindent\textbf{Paragraphe : }Denn setzet, der \match{Boden} gehöre niemanden an: so werde ich jede bewegliche Sache, die sich auf ihm befindet, aus ihrem Platze stoßen können, um ihn selbst einzunehmen, bis sie sich gänzlich verliert, ohne daß der Freiheit irgend eines anderen, der jetzt gerade nicht Inhaber desselben ist, dadurch Abbruch geschieht; alles aber, was zerstört werden kann, ein Baum, Haus u.s.w. ist (wenigstens der Materie nach) beweglich, und wenn man die Sache, die ohne Zerstörung ihrer Form nicht bewegt werden kann, ein Immobile nennt, so wird das Mein und Dein an jener nicht von der Substanz, sondern dem ihr Anhängenden verstanden, welches nicht die Sache selbst ist. 
	
	\subsection*{tg441.2.49} 
	\textbf{Source : }Die Metaphysik der Sitten/Erster Teil. Metaphysische Anfangsgründe der Rechtslehre/2. Teil. Das öffentliche Recht/1. Abschnitt. Das Staatsrecht\\  
	
	\noindent\textbf{Paragraphe : }Kann der Beherrscher als Obereigentümer (des Bodens), oder muß er nur als Oberbefehlshaber in Ansehung des Volks durch Gesetze betrachtet werden? Da der \match{Boden} die oberste Bedingung ist, unter der allein es möglich ist, äußere Sachen als das Seine zu haben, deren möglicher Besitz und Gebrauch das erste erwerbliche Recht ausmacht, so wird von dem Souverän, als Landesherren, besser als Obereigentümer (dominus territorii) alles solche Recht abgeleitet wer den müssen. Das Volk, als die Menge der Untertanen, gehört ihm auch zu (es ist sein Volk), aber nicht ihm, als Eigentümer (nach dem dinglichen), sondern als Oberbefehlshaber (nach dem persönlichen Recht). – Dieses Obereigentum ist aber nur eine Idee des bürgerlichen Vereins, um die notwendige Vereinigung des Privateigentums aller im Volk unter einem öffentlichen allgemeinen Besitzer, zu Bestimmung des besonderen Eigentums, nicht nach Grundsätzen der Aggregation (die von den Teilen zum Ganzen empirisch fortschreitet), sondern dem notwendigen formalen Prinzip der Einteilung (Division des Bodens) nach Rechtsbegriffen vorstellig zu machen. Nach diesen kann der Obereigentümer kein Privateigentum an irgend einem Boden haben (denn sonst machte er sich zu einer Privatperson), sondern dieses gehört nur dem Volk (und zwar nicht kollektiv- sondern distributiv genommen) zu; wovon doch ein nomadisch-beherrschtes Volk auszunehmen ist, als in welchem gar kein Privateigentum des Bodens statt findet. – Der Oberbefehlshaber kann also keine Domänen, d.i. Ländereien, zu seiner Privatbenutzung (zu Unterhaltung des Hofes) haben. Denn, weil es alsdenn auf sein eigen Gutbefinden ankäme, wie weit sie ausgebreitet sein sollten, so würde der Staat Gefahr laufen, alles Eigentum des Bodens in den Händen der Regierung zu sehen, und alle Untertanen als grunduntertänig (glebae adscripti)und Besitzer von dem, was immer nur Eigentum eines anderen ist, folglich aller Freiheit beraubt (servi) anzusehen. – Von einem Landesherrn kann man sagen: er besitzt nichts (zu eigen), außer sich selbst; denn, wenn er neben einem anderen im Staat etwas zu eigen hätte, so würde mit diesem ein Streit möglich sein, zu dessen Schlichtung kein Richter wäre. Aber man kann auch sagen: er besitzt alles; weil er das Befehlshaberrecht über das Volk hat (jedem das Seine zu Teil kommen zu lassen), dem alle äußere Sachen (divisim) zugehören. 
	
	\subsection*{tg441.2.50} 
	\textbf{Source : }Die Metaphysik der Sitten/Erster Teil. Metaphysische Anfangsgründe der Rechtslehre/2. Teil. Das öffentliche Recht/1. Abschnitt. Das Staatsrecht\\  
	
	\noindent\textbf{Paragraphe : }Hieraus folgt: daß es auch keine Korporation im Staat, keinen Stand und Orden, geben könne, der als Eigentümer den \match{Boden} zur alleinigen Benutzung den folgenden Generationen (ins Unendliche) nach gewissen Statuten überliefern könne. Der Staat kann sie zu aller Zeit aufheben, nur unter der Bedingung, die Überlebenden zu entschädigen. Der Ritterorden (als Korporation, oder auch bloß Rang einzelner, vorzüglich beehrter, Personen); der Orden der Geistlichkeit, die Kirche genannt, können nie durch diese Vorrechte, womit sie begünstigt worden, ein auf Nachfolger übertragbares Eigentum am Boden, sondern nur die einstweilige Benutzung desselben erwerben. Die Komtureien auf einer, die Kirchengüter auf der anderen Seite können, wenn die öffentliche Meinung wegen der Mittel, durch die Kriegsehre
	den Staat wider die Lauigkeit in Verteidigung desselben zu schützen, oder die Menschen in demselben durch Seelmessen, Gebete und eine Menge zu bestellender Seelsorger, um sie vor dem ewigen Feuer zu bewahren, anzutreiben, aufgehört hat, ohne Bedenken (doch unter der vorgenannten Bedingung) aufgehoben werden. Die, so hier in die Reform fallen, können nicht klagen, daß ihnen ihr Eigentum genommen werde; denn der Grund ihres bisherigen Besitzes lag nur in der Volksmeinung, und mußte auch, so lange diese fortwährte, gelten. So bald diese aber erlosch, und zwar auch nur in dem Urteil derjenigen, welche auf Leitung desselben durch ihr Verdienst den größten Anspruch haben, so mußte, gleichsam als durch eine Appellation desselben an den Staat (a rege male informato ad regem melius informandum), das vermeinte Eigentum aufhören. 
	
	\subsection*{tg441.2.86} 
	\textbf{Source : }Die Metaphysik der Sitten/Erster Teil. Metaphysische Anfangsgründe der Rechtslehre/2. Teil. Das öffentliche Recht/1. Abschnitt. Das Staatsrecht\\  
	
	\noindent\textbf{Paragraphe : }1) Der Untertan (auch als Bürger betrachtet) hat das Recht der Auswanderung; denn der Staat könnte ihn nicht als sein Eigentum zurückhalten. Doch kann er nur seine fahrende, nicht die liegende Habe mit herausnehmen, welches alsdann doch geschehen würde, wenn er seinen bisher besessenen \match{Boden} zu verkaufen, und das Geld dafür mit sich zu nehmen, befugt wäre. 
	
	\subsection*{tg441.2.87} 
	\textbf{Source : }Die Metaphysik der Sitten/Erster Teil. Metaphysische Anfangsgründe der Rechtslehre/2. Teil. Das öffentliche Recht/1. Abschnitt. Das Staatsrecht\\  
	
	\noindent\textbf{Paragraphe : }
	2) Der Landesherr hat das Recht der Begünstigung der Einwanderung und Ansiedelung Fremder (Kolonisten), obgleich seine Landeskinder dazu scheel sehen möchten; wenn ihnen nur nicht das Privateigentum derselben am \match{Boden} gekürzt wird. 
	
	\subsection*{tg442.2.32} 
	\textbf{Source : }Die Metaphysik der Sitten/Erster Teil. Metaphysische Anfangsgründe der Rechtslehre/2. Teil. Das öffentliche Recht/2. Abschnitt. Das Völkerrecht\\  
	
	\noindent\textbf{Paragraphe : }Der überwundene Staat, oder dessen Untertanen, verlieren durch die Eroberung des Landes nicht ihre staatsbürgerliche Freiheit, so, daß jener zur Kolonie, diese zu Leibeigenen abgewürdigt würden; denn sonst wäre es ein Strafkrieg gewesen, der an sich selbst widersprechend ist. – Eine Kolonie oder Provinz ist ein Volk, das zwar seine eigene Verfassung, Gesetzgebung, \match{Boden} hat, auf welchem die zu einem anderen Staat Gehörige nur Fremdlinge sind, der dennoch über jenes die oberste ausübende Gewalt hat. Der letztere heißt der Mutterstaat. Der Tochterstaat wird von jenem beherrscht, aber doch von sich selbst (durch sein eigenes Parlament, allenfalls unter dem Vorsitz eines Vizekönigs) regiert (civitas hybrida). Dergleichen war Athen in Beziehung auf verschiedene Inseln, und ist jetzt Großbritannien in Ansehung Irlands. 
	
	\subsection*{tg445.2.57} 
	\textbf{Source : }Die Metaphysik der Sitten/Erster Teil. Metaphysische Anfangsgründe der Rechtslehre/Anhang erläutender Bemerkungen zu den metaphysischen Anhangsgründen der Rechtslehre\\  
	
	\noindent\textbf{Paragraphe : }Selbst Stiftungen zu ewigen Zeiten für Arme, oder Schulanstalten, sobald sie einen gewissen, von dem Stifter nach  seiner Idee bestimmten entworfenen Zuschnitt haben, können nicht auf ewige Zeiten fundiert und der \match{Boden} damit belästigt werden; sondern der Staat muß die Freiheit haben, sie nach dem Bedürfnisse der Zeit einzurichten. – Daß es schwerer hält, diese Idee allerwärts auszuführen (z.B. die Pauperbursche die Unzulänglichkeit des wohltätig errichteten Schulfonds durch bettelhaftes Singen ergänzen zu müssen,) darf niemanden wundern; denn der, welcher gutmütiger- aber doch zugleich etwas ehrbegierigerweise eine Stiftung macht, will, daß sie nicht ein anderer nach seinen Begriffen umändere, sondern Er darin unsterblich sei. Das ändert aber nicht die Beschaffenheit der Sache selbst und das Recht des Staats, ja die Pflicht desselben zum Umändern einer jeden Stiftung, wenn sie der Erhaltung und dem Fortschreiten desselben zum Besseren entgegen ist, kann daher niemals als auf ewig begründet betrachtet werden. 
	
	\subsection*{tg445.2.65} 
	\textbf{Source : }Die Metaphysik der Sitten/Erster Teil. Metaphysische Anfangsgründe der Rechtslehre/Anhang erläutender Bemerkungen zu den metaphysischen Anhangsgründen der Rechtslehre\\  
	
	\noindent\textbf{Paragraphe : }Was endlich die Majoratsstiftung betrifft, da ein Gutsbesitzer durch Erbeseinsetzung verordnet: daß in der Reihe der auf einander folgenden Erben immer der Nächste von der Familie der Gutsherr sein solle (nach der Analogie mit einer monarchisch-erblichen Verfassung eines Staats, wo der Landesherr es ist), so kann eine solche Stiftung nicht allein mit Beistimmung aller Agnaten jederzeit aufgehoben werden und darf nicht auf ewige Zeiten – gleich als ob das Erbrecht am \match{Boden} haftete – immerwährend fortdauern, noch gesagt werden, es sei eine Verletzung der Stiftung und des Willens des Urahnherrn derselben, des Stifters, sie eingehen zu lassen: sondern der Staat hat auch hier ein Recht, ja sogar die Pflicht, bei den allmählich eintretenden Ursachen seiner eigenen Reform ein solches föderatives System seiner Untertanen, gleich als Unterkönige (nach der Analogie von Dynasten und Satrapen), wenn es erloschen ist, nicht weiter aufkommen zu lassen. 
	
	\unnumberedsection{Dunkelheit (3)} 
	\subsection*{tg429.2.5} 
	\textbf{Source : }Die Metaphysik der Sitten/Erster Teil. Metaphysische Anfangsgründe der Rechtslehre/Vorrede\\  
	
	\noindent\textbf{Paragraphe : }Der weise Mann fordert (in seinem Werk, Vermischte Aufsätze betitelt, S. 352 u. f.) mit Recht, eine jede philosophische Lehre müsse, wenn der Lehrer nicht selbst in den Verdacht der \match{Dunkelheit} seiner Begriffe kommen soll – zur Popularität (einer zur allgemeinen Mitteilung hinreichenden Versinnlichung) gebracht werden können. Ich räume das gern ein, nur mit Ausnahme des Systems einer Kritik des Vernunftvermögens selbst und alles dessen, was nur durch dieser ihre Bestimmung beurkundet werden kann; weil es zur Unterscheidung des Sinnlichen in unserem Erkenntnis vom Übersinnlichen, dennoch aber der Vernunft Zustehenden, gehört. Dieses kann nie populär werden, so wie überhaupt keine formelle Metaphysik; obgleich ihre Resultate für die gesunde Vernunft (eines Metaphysikers, ohne es zu wissen) ganz einleuchtend gemacht werden können. Hier ist an keine Popularität (Volkssprache) zu denken, sondern es muß auf scholastische Pünktlichkeit, wenn sie auch Peinlichkeit gescholten würde, gedrungen werden (denn es ist Schulsprache); weil dadurch allein die voreilige Vernunft dahin gebracht werden kann, vor ihren dogmatischen Behauptungen sich erst selbst zu verstehen. 
	
	\subsection*{tg442.2.47} 
	\textbf{Source : }Die Metaphysik der Sitten/Erster Teil. Metaphysische Anfangsgründe der Rechtslehre/2. Teil. Das öffentliche Recht/2. Abschnitt. Das Völkerrecht\\  
	
	\noindent\textbf{Paragraphe : }Man kann einen solchen Verein einiger Staaten, um den Frieden zu erhalten, den permanenten Staatenkongreß nennen, zu welchem sich zu gesellen jedem benachbarten unbenommen bleibt; dergleichen (wenigstens was die Förmlichkeiten des Völkerrechts, in Absicht auf Erhaltung des Friedens, betrifft) in der ersten Hälfte dieses Jahrhunderts in der Versammlung der Generalstaaten im Haag noch statt fand; wo die Minister der meisten europäischen Höfe, und selbst der kleinsten Republiken, ihre Beschwerden  über die Befehdungen, die einem von dem anderen widerfahren waren, anbrachten, und so sich ganz Europa als einen einzigen föderierten Staat dachten, den sie in jener ihren öffentlichen Streitigkeiten gleichsam als Schiedsrichter annahmen, statt dessen späterhin das Völkerrecht bloß in Büchern übrig geblieben, aus Kabinetten aber verschwunden, oder, nach schon verübter Gewalt, in Form der Deduktionen, der \match{Dunkelheit} der Archive anvertrauet worden ist. 
	
	\subsection*{tg489.2.11} 
	\textbf{Source : }Die Metaphysik der Sitten/Fußnoten\\  
	
	\noindent\textbf{Paragraphe : }Der philosophische Rechtslehrer wird diese Nachforschung bis zu den ersten Elementen der Transzendentalphilosophie in einer Metaphysik der Sitten nicht für unnötige Grübelei erklären, die sich in zwecklose \match{Dunkelheit} verliert, wenn er die Schwierigkeit der zu lösenden Aufgabe und doch auch die Notwendigkeit, hierin den Rechtsprinzipien genug zu tun, in Überlegung zieht. 
	
	\unnumberedsection{Eisen (1)} 
	\subsection*{tg441.2.20} 
	\textbf{Source : }Die Metaphysik der Sitten/Erster Teil. Metaphysische Anfangsgründe der Rechtslehre/2. Teil. Das öffentliche Recht/1. Abschnitt. Das Staatsrecht\\  
	
	\noindent\textbf{Paragraphe : }Nur die Fähigkeit der Stimmgebung macht die Qualifikation zum Staatsbürger aus; jene aber setzt die Selbständigkeit dessen im Volk voraus, der nicht bloß Teil des gemeinen Wesens, sondern auch Glied desselben, d.i. aus eigener Willkür in Gemeinschaft mit anderen handelnder  Teil desselben sein will. Die letztere Qualität macht aber die Unterscheidung des aktiven vom passiven Staatsbürger notwendig: obgleich der Begriff des letzteren mit der Erklärung des Begriffs von einem Staatsbürger überhaupt im Widerspruch zu stehen scheint. – Folgende Beispiele können dazu dienen, diese Schwierigkeit zu heben: Der Geselle bei einem Kaufmann, oder bei einem Handwerker; der Dienstbote (nicht der im Dienste des Staats steht); der Unmündige (naturaliter vel civiliter); alles Frauenzimmer, und überhaupt jedermann, der nicht nach eigenem Betrieb, sondern nach der Verfügung anderer (außer der des Staats), genötigt ist, seine Existenz (Nahrung und Schutz) zu erhalten, entbehrt der bürgerlichen Persönlichkeit, und seine Existenz ist gleichsam nur Inhärenz. – Der Holzhacker, den ich auf meinem Hofe anstelle, der Schmied in Indien, der mit seinem Hammer, Amboß und Blasbalg in die Häuser geht, um da in \match{Eisen} zu arbeiten, in Vergleichung mit dem europäischen Tischler oder Schmied, der die Produkte aus dieser Arbeit als Ware öffentlich feil stellen kann, der Hauslehrer in Vergleichung mit dem Schulmann, der Zinsbauer in Vergleichung mit dem Pächter u. dergl. sind bloß Handlanger des gemeinen Wesens, weil sie von anderen Individuen befehligt oder beschützt werden müssen, mithin keine bürgerliche Selbständigkeit besitzen. 
	
	\unnumberedsection{Erdboden (1)} 
	\subsection*{tg441.2.6} 
	\textbf{Source : }Die Metaphysik der Sitten/Erster Teil. Metaphysische Anfangsgründe der Rechtslehre/2. Teil. Das öffentliche Recht/1. Abschnitt. Das Staatsrecht\\  
	
	\noindent\textbf{Paragraphe : }Der Inbegriff der Gesetze, die einer allgemeinen Bekanntmachung bedürfen, um einen rechtlichen Zustand hervorzubringen, ist das öffentliche Recht. – Dieses ist also ein System von Gesetzen für ein Volk, d.i. eine Menge von Menschen, oder für eine Menge von Völkern, die, im wechselseitigen Einflusse gegen einander stehend, des rechtlichen Zustandes unter einem sie vereinigenden Willen, einer Verfassung (constitutio) bedürfen, um dessen, was Rechtens ist, teilhaftig zu werden. – Dieser Zustand der einzelnen im Volke, in Verhältnis untereinander, heißt der bürgerliche (status civilis), und das Ganze derselben, in Beziehung auf seine eigene Glieder, der Staat (civitas), welcher, seiner Form wegen, als verbunden durch das gemeinsame Interesse aller, im rechtlichen Zustande zu sein, das gemeine Wesen (res publica latius sic dicta) genannt wird, in Verhältnis aber auf an dere Völker eine Macht (potentia) schlechthin heißt (daher das Wort Potentaten), was sich auch wegen (anmaßlich) angeerbter Vereinigung ein Stammvolk (gens) nennt, und so, unter dem allgemeinen Begriffe des öffentlichen Rechts, nicht bloß das Staats- sondern auch ein Völkerrecht (ius gentium) zu denken Anlaß gibt: welches dann, weil der \match{Erdboden} eine nicht grenzenlose, sondern sich selbst schließende Fläche ist, beides zusammen zu der Idee eines Völkerstaatsrechts (ius gentium) oder des Weltbürgerrechts (ius cosmopoliticum) unumgänglich hinleitet: so, daß, wenn unter diesen drei möglichen Formen des rechtlichen Zustandes es nur einer an dem die äußere Freiheit durch Gesetze einschränkenden Prinzip fehlt, das Gebäude aller übrigen unvermeidlich untergraben werden, und endlich einstürzen muß. 
	
	\unnumberedsection{Erde (5)} 
	\subsection*{tg433.2.42} 
	\textbf{Source : }Die Metaphysik der Sitten/Erster Teil. Metaphysische Anfangsgründe der Rechtslehre/1. Teil. Das Privatrecht vom äußeren Mein und Dein überhaupt/1. Hauptstück\\  
	
	\noindent\textbf{Paragraphe : }Die Art also, etwas außer mir als das Meine zu haben, ist die bloß-rechtliche Verbindung des Willens des Subjekts mit jenem Gegenstande, unabhängig von dem Verhältnisse zu demselben im Raum und in der Zeit, nach dem Begriff eines intelligibelen Besitzes. – Ein Platz auf der \match{Erde} ist nicht darum ein äußeres Meine, weil ich ihn mit meinem Leibe einnehme (denn es betrifft hier nur meine äußere Freiheit, mithin nur den Besitz meiner selbst, kein Ding außer mir, und ist also nur ein inneres Recht); sondern, wenn ich ihn noch besitze, ob ich mich gleich von ihm weg und an einen andern Ort begeben habe, nur alsdenn betrifft es mein äußeres Recht, und derjenige, der die fortwährende Besetzung dieses Platzes durch meine Person zur Bedingung machen wollte, ihn als das Meine zu haben, muß entweder behaupten, es sei gar nicht möglich, etwas Äußeres als das Seine zu haben (welches dem Postulat § 2 widerstreitet), oder er verlangt, daß, um dieses zu können, ich in zwei Orten zugleich sei; welches denn aber so viel sagt, als: ich solle an einem Orte sein und auch nicht sein, wodurch er sich selbst widerspricht. 
	
	\subsection*{tg435.2.31} 
	\textbf{Source : }Die Metaphysik der Sitten/Erster Teil. Metaphysische Anfangsgründe der Rechtslehre/1. Teil. Das Privatrecht vom äußeren Mein und Dein überhaupt/2. Hauptstück. Von der Art, etwas Äußeres zu erwerben/1. Abschnitt. Vom Sachrecht\\  
	
	\noindent\textbf{Paragraphe : }Alle Menschen sind ursprünglich in einem Gesamt-Besitz des Bodens der ganzen \match{Erde} (communio fundi originaria), mit dem ihnen von Natur zustehenden Willen (eines jeden), denselben zu gebrauchen (lex iusti), der, wegen der natürlich unvermeidlichen Entgegensetzung der Willkür des einen gegen die des anderen, allen Gebrauch desselben aufheben würde, wenn nicht jener zugleich das Gesetz für diese enthielte, nach welchem einem jeden ein besonderer Besitz auf dem gemeinsamen Boden bestimmt werden kann (lex iuridica). Aber das austeilende Gesetz des Mein und Dein eines jeden am Boden kann, nach dem Axiom der äußeren Freiheit, nicht anders als aus einem ursprünglich und a priori vereinigten Willen (der zu dieser Vereinigung keinen rechtlichen Akt voraussetzt), mithin nur im bürgerlichen Zustande, hervorgehen (lex iustitiae distributivae), der allein, was recht, was rechtlich und was Rechtens ist, bestimmt. – In diesem Zustand aber, d.i. vor Gründung und doch in Absicht auf denselben, d.i. provisorisch, nach dem Gesetz der äußeren Erwerbung zu verfahren, ist Pflicht, folglich auch rechtliches Vermögen des Willens, jedermann zu verbinden, den Akt der Besitznehmung und Zueignung, ob er gleich nur einseitig ist, als gültig anzuerkennen; mithin ist eine provisorische Erwerbung des Bodens, mit allen ihren rechtlichen Folgen, möglich. 
	
	\subsection*{tg443.2.5} 
	\textbf{Source : }Die Metaphysik der Sitten/Erster Teil. Metaphysische Anfangsgründe der Rechtslehre/2. Teil. Das öffentliche Recht/3. Abschnitt. Das Weltbürgerrecht\\  
	
	\noindent\textbf{Paragraphe : }Meere können Völker aus aller Gemeinschaft mit einander zu setzen scheinen, und dennoch sind sie, vermittelst der Schiffahrt, gerade die glücklichsten Naturanlagen zu ihrem Verkehr, welches, je mehr es einander nahe Küsten gibt (wie die des mittelländischen), nur desto lebhafter sein kann, deren Besuchung gleichwohl, noch mehr aber die Niederlassung auf denselben, um sie mit dem Mutterlande zu verknüpfen, zugleich die Veranlassung dazu gibt, daß Übel und Gewalttätigkeit, an einem Orte unseres Globs, an allen gefühlt wird. Dieser mögliche Mißbrauch kann aber das Recht des Erdbürgers nicht aufheben, die Gemeinschaft mit allen zu versuchen, und zu diesem Zweck alle Gegenden der \match{Erde} zu besuchen, wenn es gleich nicht ein Recht der Ansiedelung auf dem Boden eines anderen Volks (ius incolatus) ist, als zu welchem ein besonderer Vertrag erfordert wird. 
	
	\subsection*{tg443.2.7} 
	\textbf{Source : }Die Metaphysik der Sitten/Erster Teil. Metaphysische Anfangsgründe der Rechtslehre/2. Teil. Das öffentliche Recht/3. Abschnitt. Das Weltbürgerrecht\\  
	
	\noindent\textbf{Paragraphe : }Wenn Anbauung in solcher Entlegenheit vom Sitz des ersteren geschieht, daß keines derselben im Gebrauch seines Bodens dem anderen Eintrag tut, so ist das Recht dazu nicht zu bezweifeln; wenn es aber Hirten- oder Jagdvölker  sind (wie die Hottentotten, Tungusen und die meisten amerikanischen Nationen), deren Unterhalt von großen öden Landstrecken abhängt, so würde dies nicht mit Gewalt, sondern nur durch Vertrag, und selbst dieser nicht mit Benutzung der Unwissenheit jener Einwohner in Ansehung der Abtretung solcher Ländereien, geschehen können; obzwar die Rechtfertigungsgründe scheinbar genug sind, daß eine solche Gewalttätigkeit zum Weltbesten gereiche; teils durch Kultur roher Völker (wie der Vorwand, durch den selbst Büsching die blutige Einführung der christlichen Religion in Deutschland entschuldigen will), teils zur Reinigung seines eigenen Landes von verderbten Menschen und gehoffter Besserung derselben, oder ihrer Nachkommenschaft, in einem anderen Weltteile (wie in Neuholland); denn alle diese vermeintlich gute Absichten können doch den Flecken der Ungerechtigkeit in den dazu gebrauchten Mitteln nicht abwaschen. – Wendet man hiegegen ein: daß, bei solcher Bedenklichkeit, mit der Gewalt den Anfang zu Gründung eines gesetzlichen Zustandes zu machen, vielleicht die ganze \match{Erde} noch in gesetzlosem Zustande sein würde: so kann das eben so wenig jene Rechtsbedingung aufheben, als der Vorwand der Staatrevolutionisten, daß es auch, wenn Verfassungen verunartet sind, dem Volk zustehe, sie mit Gewalt umzuformen, und überhaupt einmal für allemal ungerecht zu sein, um nachher die Gerechtigkeit desto sicherer zu gründen und aufblühen zu machen. 
	
	\subsection*{tg472.2.48} 
	\textbf{Source : }Die Metaphysik der Sitten/Zweiter Teil. Metaphysische Anfangsgründe der Tugendlehre/I. Ethische Elementarlehre/I. Teil. Von den Pflichten gegen sich selbst überhaupt/Erstes Buch. Von den vollkommenen Pflichten gegen sich selbst\\  
	
	\noindent\textbf{Paragraphe : }Die vorzügliche Achtungsbezeigung in Worten und Manieren, selbst gegen einen nicht Gebietenden in der bürgerlichen Verfassung – die Reverenzen, Verbeugungen (Komplimente), höfische – den Unterschied der Stände mit sorgfältiger Pünktlichkeit bezeichnende Phrasen, – welche von der Höflichkeit (die auch sich gleich Achtenden notwendig ist) ganz unterschieden sind, – das Du, Er, Ihr und Sie, oder Ew. Wohledlen, Hochedeln, Hochedelgebornen, Wohlgebornen (ohe, iam satis est!) in der Anrede – als in welcher Pedanterei die Deutschen unter allen Völkern der \match{Erde} (die indischen Kasten vielleicht ausgenommen) es am weitesten gebracht haben, sind das nicht Beweise eines ausgebreiteten Hanges zur Kriecherei unter Menschen? (Hae nugae in seria ducunt.) Wer sich aber zum Wurm macht, kann nachher nicht klagen, daß er mit Füßen getreten wird. 
	
	\unnumberedsection{Erhebung (1)} 
	\subsection*{tg472.2.38} 
	\textbf{Source : }Die Metaphysik der Sitten/Zweiter Teil. Metaphysische Anfangsgründe der Tugendlehre/I. Ethische Elementarlehre/I. Teil. Von den Pflichten gegen sich selbst überhaupt/Erstes Buch. Von den vollkommenen Pflichten gegen sich selbst\\  
	
	\noindent\textbf{Paragraphe : }Aus unserer aufrichtigen und genauen Vergleichung mit dem moralischen Gesetz (dessen Heiligkeit und Strenge) muß unvermeidlich wahre Demut folgen: aber daraus, daß wir einer solchen inneren Gesetzgebung fähig sind, daß der (physische) Mensch den (moralischen) Menschen in seiner eigenen Person zu verehren sich gedrungen fühlt, zugleich \match{Erhebung} und die höchste Selbstschätzung, als Gefühl seines inneren Werts (valor), nach welchem er für keinen Preis (pretium) feil ist, und eine unverlierbare Würde (dignitas interna) besitzt, die ihm Achtung (reverentia) gegen sich selbst einflößt. 
	
	\unnumberedsection{Fehler (9)} 
	\subsection*{tg430.2.16} 
	\textbf{Source : }Die Metaphysik der Sitten/Erster Teil. Metaphysische Anfangsgründe der Rechtslehre/Einleitung in die Metaphysik der Sitten\\  
	
	\noindent\textbf{Paragraphe : }Allein mit den Lehren der Sittlichkeit ist es anders bewandt. Sie gebieten für jedermann, ohne Rücksicht auf seine Neigungen zu nehmen; bloß weil und sofern er frei ist und praktische Vernunft hat. Die Belehrung in ihren Gesetzen ist nicht aus der Beobachtung seiner selbst und der Tierheit in ihm, nicht aus der Wahrnehmung des Weltlaufs geschöpft,  von dem, was geschieht und wie gehandelt wird (obgleich das deutsche Wort Sitten, eben so wie das lateinische mores, nur Manieren und Lebensart bedeutet), sondern die Vernunft gebietet, wie gehandelt werden soll, wenn gleich noch kein Beispiel davon angetroffen würde, auch nimmt sie keine Rücksicht auf den Vorteil, der uns dadurch erwachsen kann, und den freilich nur die Erfahrung lehren könnte. Denn, ob sie zwar erlaubt, unsern Vorteil, auf alle uns mögliche Art, zu suchen, überdem auch sich, auf Erfahrungszeugnisse fußend, von der Befolgung ihrer Gebote, vornehmlich wenn Klugheit dazu kommt, im Durchschnitte größere Vorteile, als von ihrer Übertretung wahrscheinlich versprechen kann, so beruht darauf doch nicht die Autorität ihrer Vorschriften als Gebote, sondern sie bedient sich derselben (als Ratschläge) nur als eines Gegengewichts wider die Verleitungen zum Gegenteil, um den \match{Fehler} einer parteiischen Wage in der praktischen Beurteilung vorher auszugleichen und alsdenn allererst dieser, nach dem Gewicht der Gründe a priori einer reinen praktischen Vernunft, den Ausschlag zu sichern. 
	
	\subsection*{tg439.2.7} 
	\textbf{Source : }Die Metaphysik der Sitten/Erster Teil. Metaphysische Anfangsgründe der Rechtslehre/1. Teil. Das Privatrecht vom äußeren Mein und Dein überhaupt/3. Hauptstück. Von der subjektiv-bedingten Erwerbung durch den Ausspruch einer öffentlichen Gerichtsbarkeit\\  
	
	\noindent\textbf{Paragraphe : }Es ist ein gewöhnlicher \match{Fehler} der Erschleichung (vitium subreptionis) der Rechtslehrer, dasjenige rechtliche Prinzip, was ein Gerichtshof, zu seinem eigenen Behuf (also in subjektiver Absicht), anzunehmen befugt, ja sogar verbunden ist, um über jedes einem zustehende Recht zu sprechen und zu richten, auch objektiv, für das, was an sich selbst recht ist, zu halten: da das erstere doch von dem letzteren sehr unterschieden ist. – Es ist daher von nicht geringer Wichtigkeit, diese spezifische Verschiedenheit kennbar und darauf aufmerksam zu machen. 
	
	\subsection*{tg441.2.64} 
	\textbf{Source : }Die Metaphysik der Sitten/Erster Teil. Metaphysische Anfangsgründe der Rechtslehre/2. Teil. Das öffentliche Recht/1. Abschnitt. Das Staatsrecht\\  
	
	\noindent\textbf{Paragraphe : }Die Würde betreffend, nicht bloß die, welche ein Amt bei sich führen mag, sondern auch die, welche den Besitzer auch ohne besondere Bedienungen zum Gliede eines höheren Standes macht, ist der Adel, der, vom bürgerlichen Stande, in welchem das Volk ist, unterschieden, den männlichen  Nachkommen anerbt, durch diese auch wohl den weiblichen unadliger Geburt, nur so, daß die Adlig-Geborne ihrem unadligen Ehemann nicht um gekehrt diesen Rang mitteilt, sondern selbst in den bloß bürgerlichen (des Volks) zurückfällt. – Die Frage ist nun: ob der Souverän einen Adelstand, als einen erblichen Mittelstand zwischen ihm und den übrigen Staatsbürgern, zu gründen berechtigt sei. In dieser Frage kommt es nicht darauf an: ob es der Klugheit des Souveräns, wegen seines oder des Volks Vorteils, sondern nur, ob es dem Rechte des Volks gemäß sei, einen Stand von Personen über sich zu haben, die zwar selbst Untertanen, aber doch in Ansehung des Volks geborne Befehlshaber (wenigstens privilegierte) sind. – – Die Beantwortung derselben geht nun hier, eben so wie vorher, aus dem Prinzip hervor: »was das Volk (die ganze Masse der Untertanen) nicht über sich selbst und seine Genossen beschließen kann, das kann auch der Souverän nicht über das Volk beschließen«. Nun ist ein angeerbter Adel ein Rang, der vor dem Verdienst vorher geht, und dieses auch mit keinem Grunde hoffen läßt, ein Gedankending, ohne alle Realität. Denn, wenn der Vorfahr Verdienst hatte, so konnte er dieses doch nicht auf seine Nachkommen vererben, sondern diese mußten es sich immer selbst erwerben; da die Natur es nicht so fügt, daß das Talent und der Wille, welche Verdienste um den Staat möglich machen, auch anarten. Weil nun von keinem Menschen angenommen werden kann, er werde seine Freiheit wegwerfen, so ist es unmöglich, daß der allgemeine Volkswille zu einem solchen grundlosen Prärogativ zusammen stimme, mithin kann der Souverän es auch nicht geltend machen. – – Wenn indessen gleich eine solche Anomalie in das Maschinenwesen einer Regierung von alten Zeiten (des Lehnswesens, das fast gänzlich auf den Krieg angelegt war) eingeschlichen, von Untertanen, die mehr als Staatsbürger, nämlich geborne Beamte (wie etwa ein Erbprofessor) sein wollen, so kann der Staat diesen von ihm begangenen \match{Fehler} eines widerrechtlich erteilten erblichen Vorzugs nicht anders, als durch Eingehen und Nichtbesetzung der Stellen allmählich wiederum gut machen, und  so hat er provisorisch ein Recht, diese Würde dem Titel nach fortdauern zu lassen, bis selbst in der öffentlichen Meinung die Einteilung in Souverän, Adel und Volk der einzigen natürlichen in Souverän und Volk Platz gemacht haben wird. 
	
	\subsection*{tg472.2.11} 
	\textbf{Source : }Die Metaphysik der Sitten/Zweiter Teil. Metaphysische Anfangsgründe der Tugendlehre/I. Ethische Elementarlehre/I. Teil. Von den Pflichten gegen sich selbst überhaupt/Erstes Buch. Von den vollkommenen Pflichten gegen sich selbst\\  
	
	\noindent\textbf{Paragraphe : }Unredlichkeit ist bloß Ermangelung an Gewissenhaftigkeit, d.i. an Lauterkeit des Bekenntnisses vor seinem inneren Richter, der als eine andere Person gedacht wird, wenn diese in ihrer höchsten Strenge betrachtet wird, wo ein Wunsch (aus Selbstliebe) für die Tat genommen wird, weil er einen an sich guten Zweck vor sich hat, und die innere Lüge, ob sie zwar der Pflicht des Menschen gegen sich selbst zuwider ist, erhält hier den Namen einer Schwachheit, so wie der Wunsch eines Liebhabers, lauter gute. Eigenschaften an seiner Geliebten zu finden, ihm ihre augenscheinliche \match{Fehler} unsichtbar macht. – Indessen verdient diese Unlauterkeit in Erklärungen, die man gegen sich selbst verübt, doch die ernstlichste Rüge: weil, von einer solchen faulen Stelle (der Falschheit, welche in der menschlichen Natur gewurzelt zu sein scheint) aus, das Übel der Unwahrhaftigkeit sich auch in Beziehung auf andere Menschen verbreitet, nachdem einmal der oberste Grundsatz der Wahrhaftigkeit verletzt worden. – 
	
	\subsection*{tg482.2.20} 
	\textbf{Source : }Die Metaphysik der Sitten/Zweiter Teil. Metaphysische Anfangsgründe der Tugendlehre/I. Ethische Elementarlehre/II. Teil. Von den Tugendpflichten gegen andere/Erstes Hauptstück. Von den Pflichten gegen andere, bloß als Menschen/Zweiter Abschnitt. Von den Tugendpflichten gegen andere Menschen aus der ihnen gebührenden Achtung\\  
	
	\noindent\textbf{Paragraphe : }Die Achtung vor dem Gesetze, welche subjektiv als moralisches Gefühl bezeichnet wird, ist mit dem Bewußtsein seiner Pflicht einerlei. Eben darum ist auch die Bezeigung der Achtung vor dem Menschen als moralischen (seine Pflicht höchstschätzenden) Wesen selbst eine Pflicht, die andere gegen ihn haben, und ein Recht, worauf er den Anspruch nicht aufgeben kann. – Man nennt diesen Anspruch Ehrliebe, deren Phänomen im äußeren Betragen Ehrbarkeit (honestas externa), der Verstoß dawider aber Skandal heißt: ein Beispiel der Nichtachtung derselben, das Nachfolge bewirken dürfte; welches zu geben zwar höchstpflichtwidrig, aber am bloß Widersinnischen (paradoxon),  sonst an sich Guten, zu nehmen, ein Wahn (da man das Nichtgebräuchliche auch für nicht erlaubt hält), ein der Tugend gefährlicher und verderblicher \match{Fehler} ist. – Denn die schuldige Achtung, für andere ein Beispiel gebende Menschen, kann nicht bis zur blinden Nachahmung (da der Gebrauch, mos, zur Würde eines Gesetzes erhoben wird) ausarten; als welche Tyrannei der Volkssitte der Pflicht des Menschen gegen sich selbst zuwider sein würde. 
	
	\subsection*{tg482.2.33} 
	\textbf{Source : }Die Metaphysik der Sitten/Zweiter Teil. Metaphysische Anfangsgründe der Tugendlehre/I. Ethische Elementarlehre/II. Teil. Von den Tugendpflichten gegen andere/Erstes Hauptstück. Von den Pflichten gegen andere, bloß als Menschen/Zweiter Abschnitt. Von den Tugendpflichten gegen andere Menschen aus der ihnen gebührenden Achtung\\  
	
	\noindent\textbf{Paragraphe : }Er ist vom Stolz (animus elatus), als Ehrliebe, d.i. Sorgfalt, seiner Menschenwürde in Vergleichung mit anderen nichts zu vergeben (der daher auch mit dem Beiwort des edlen belegt zu werden pflegt), unterschieden; denn der Hochmut verlangt von anderen eine Achtung, die er ihnen doch verweigert. – Aber dieser Stolz selbst wird doch zum \match{Fehler} und Beleidigung, wenn er auch bloß ein Ansinnen an andere ist, sich mit seiner Wichtigkeit zu beschäftigen. 
	
	\subsection*{tg482.2.47} 
	\textbf{Source : }Die Metaphysik der Sitten/Zweiter Teil. Metaphysische Anfangsgründe der Tugendlehre/I. Ethische Elementarlehre/II. Teil. Von den Tugendpflichten gegen andere/Erstes Hauptstück. Von den Pflichten gegen andere, bloß als Menschen/Zweiter Abschnitt. Von den Tugendpflichten gegen andere Menschen aus der ihnen gebührenden Achtung\\  
	
	\noindent\textbf{Paragraphe : }Die leichtfertige Tadelsucht und der Hang, andere zum Gelächter bloß zu stellen, die Spottsucht, um die \match{Fehler} eines anderen zum unmittelbaren Gegenstande seiner Belustigung zu machen, ist Bosheit, und von dem 
	Scherz, der Vertraulichkeit unter Freunden, sie nur zum Schein als Fehler, in der Tat aber als Vorzüge des Muts, bisweilen auch außer der Regel der Mode zu sein, zu belachen (welches dann kein Hohnlachen ist), gänzlich unterschieden. Wirkliche Fehler aber, oder, gleich als ob sie wirklich wären, angedichtete, welche die Person ihrer verdienten Achtung zu berauben abgezweckt sind, dem Gelächter bloß zu stellen, und der Hang dazu, die bittere Spottsucht (spiritus causticus), hat etwas von teuflischer Freude an sich und ist darum eben eine desto härtere Verletzung der Pflicht der Achtung gegen andere Menschen. 
	
	\subsection*{tg484.2.11} 
	\textbf{Source : }Die Metaphysik der Sitten/Zweiter Teil. Metaphysische Anfangsgründe der Tugendlehre/I. Ethische Elementarlehre/II. Teil. Von den Tugendpflichten gegen andere/Beschluß der Elementarlehre. Von der innigsten Vereinigung der Liebe mit der Achtung in der Freundschaft\\  
	
	\noindent\textbf{Paragraphe : }Der Mensch ist ein für die Gesellschaft bestimmtes (obzwar doch auch ungeselliges) Wesen, und in der Kultur des gesellschaftlichen Zustandes fühlt er mächtig das Bedürfnis, sich anderen zu eröffnen (selbst ohne etwas dabei zu beabsichtigen); andererseits aber auch durch die Furcht vor dem Mißbrauch, den andere von dieser Aufdeckung seiner Gedanken machen dürften, beengt und gewarnt sieht er sich genötigt, einen guten Teil seiner Urteile (vornehmlich über andere Menschen) in sich selbst zu verschließen. Er möchte sich gern darüber mit irgend jemand unterhalten, wie er über die Menschen, mit denen er umgeht, wie er über die Regierung, Religion u.s.w. denkt; aber er darf es nicht wagen: teils weil der andere, der sein Urteil behutsam zurückhält, davon zu seinem Schaden Gebrauch machen, teils, was die Eröffnung seiner eigenen \match{Fehler} betrifft, der andere die seinigen verhehlen, und er so in der Achtung dieselbe einbüßen würde, wenn er sich ganz offenherzig gegen ihn darstellete. 
	
	\subsection*{tg484.2.6} 
	\textbf{Source : }Die Metaphysik der Sitten/Zweiter Teil. Metaphysische Anfangsgründe der Tugendlehre/I. Ethische Elementarlehre/II. Teil. Von den Tugendpflichten gegen andere/Beschluß der Elementarlehre. Von der innigsten Vereinigung der Liebe mit der Achtung in der Freundschaft\\  
	
	\noindent\textbf{Paragraphe : }Moralisch erwogen ist es freilich Pflicht, daß ein Freund dem anderen seine \match{Fehler} bemerklich mache; denn das geschieht  ja zu seinem Besten und es ist also Liebespflicht. Seine andere Hälfte aber sieht hierin einen Mangel der Achtung, die er von jenem erwartete, und zwar, daß er entweder darin schon gefallen sei, oder, da er von dem anderen beobachtet und ingeheim kritisiert wird, beständig Gefahr läuft, in den Verlust seiner Achtung zu fallen; wie dann selbst, daß er beobachtet und gemeistert werden solle, ihm schon für sich selbst beleidigend zu sein dünken wird. 
	
	\unnumberedsection{Fuß (2)} 
	\subsection*{tg441.2.59} 
	\textbf{Source : }Die Metaphysik der Sitten/Erster Teil. Metaphysische Anfangsgründe der Rechtslehre/2. Teil. Das öffentliche Recht/1. Abschnitt. Das Staatsrecht\\  
	
	\noindent\textbf{Paragraphe : }Da auch das Kirchenwesen, welches von der Religion, als innerer Gesinnung, die ganz außer dem Wirkungskreise der bürgerlichen Macht ist, sorgfältig unterschieden werden muß (als Anstalt zum öffentlichen Gottesdienst für das Volk, aus welchem dieser auch seinen Ursprung hat, es sei Meinung oder Überzeugung), ein wahres Staatsbedürfnis  wird, sich auch als Untertanen einer höchsten unsichtbaren Macht, der sie huldigen müssen, und die mit der bürgerlichen oft in einen sehr ungleichen Streit kommen kann, zu betrachten: so hat der Staat das Recht, nicht etwa der inneren Konstitutionalgesetzgebung, das Kirchenwesen nach seinem Sinne, wie es ihm vorteilhaft dünkt, einzurichten, den Glauben und gottesdienstliche Formen (ritus) dem Volk vorzuschreiben, oder zu befehlen (denn dieses muß gänzlich den Lehrern und Vorstehern, die es sich selbst gewählt hat, überlassen bleiben), sondern nur das negative Recht, den Einfluß der öffentlichen Lehrer auf das sichtbare, politische gemeine Wesen, der der öffentlichen Ruhe nachteilig sein möchte, abzuhalten, mithin bei dem inneren Streit, oder dem der verschiedenen Kirchen unter einander, die bürgerliche Eintracht nicht in Gefahr kommen zu lassen, welches also ein Recht der Polizei ist. Daß eine Kirche einen gewissen Glauben, und welchen sie haben, oder daß sie ihn unabänderlich erhalten müsse, und sich nicht selbst reformieren dürfe, sind Einmischungen der obrigkeitlichen Gewalt, die unter ihrer Würde sind; weil sie sich dabei, als einem Schulgezänke, auf den \match{Fuß} der Gleichheit mit ihren Untertanen einläßt (der Monarch sich zum Priester macht), die ihr geradezu sagen können, daß sie hievon nichts verstehe; vornehmlich was des letztere, nämlich das Verbot innerer Reformen, betrifft; – denn, was das gesamte Volk nicht über sich selbst beschließen kann, das kann auch der Gesetzgeber nicht über das Volk beschließen. Nun kann aber kein Volk beschließen, in seinen den Glauben betreffenden Einsichten (der Aufklärung) niemals weiter fortzuschreiten, mithin auch sich in Ansehung des Kirchenwesens nie zu reformieren; weil dies der Menschheit in seiner eigenen Person, mithin dem höchsten Rechte desselben entgegen sein würde. Also kann es auch keine obrigkeitliche Gewalt über das Volk beschließen. – – Was aber die Kosten der Erhaltung des Kirchenwesens betrifft, so können diese, aus ebenderselben Ursache, nicht dem Staat, sondern müssen dem Teil des Volks, der sich zu einem oder dem anderen Glauben bekennt, d.i. nur der Gemeine zu Lasten kommen. 
	
	\subsection*{tg472.2.34} 
	\textbf{Source : }Die Metaphysik der Sitten/Zweiter Teil. Metaphysische Anfangsgründe der Tugendlehre/I. Ethische Elementarlehre/I. Teil. Von den Pflichten gegen sich selbst überhaupt/Erstes Buch. Von den vollkommenen Pflichten gegen sich selbst\\  
	
	\noindent\textbf{Paragraphe : }Allein der Mensch als Person betrachtet, d.i. als Subjekt einer moralisch-praktischen Vernunft, ist über allen Preis erhaben; denn als ein solcher (homo noumenon) ist er nicht bloß als Mittel zu anderer ihren, ja selbst seinen eigenen Zwecken, sondern als Zweck an sich seihst zu schätzen, d.i. er besitzt eine Würde (einen absoluten innern Wert), wodurch er allen andern vernünftigen Weltwesen Achtung für ihn abnötigt, sich mit jedem anderen dieser Art messen und auf den \match{Fuß} der Gleichheit schätzen kann. 
	
	\unnumberedsection{Gipfel (1)} 
	\subsection*{tg436.2.17} 
	\textbf{Source : }Die Metaphysik der Sitten/Erster Teil. Metaphysische Anfangsgründe der Rechtslehre/1. Teil. Das Privatrecht vom äußeren Mein und Dein überhaupt/2. Hauptstück. Von der Art, etwas Äußeres zu erwerben/2. Abschnitt. Vom persönlichen Recht\\  
	
	\noindent\textbf{Paragraphe : }Die Übertragung des Meinen durch Vertrag geschieht nach dem Gesetz der Stetigkeit (lex continui), d.i. der Besitz des Gegenstandes ist während diesem Akt keinen Augenblick unterbrochen, denn sonst würde ich in diesem Zustande einen Gegenstand als etwas, das keinen Besitzer hat (res vacua), folglich ursprünglich erwerben; welches dem Begriff des Vertrages widerspricht. – Diese Stetigkeit aber bringt es mit sich, daß nicht eines von beiden (promittentis et acceptantis) besonderer, sondern ihr vereinigter Wille derjenige ist, welcher das Meine auf den anderen überträgt; also nicht auf die Art: daß der Versprechende zuerst seinen Besitz zum Vorteil des anderen verläßt (derelinquit), oder seinem Recht entsagt (renunciat) und der andere sogleich darin eintritt, oder umgekehrt. Die Translation ist also ein Akt, in welchem der Gegenstand einen Augenblick beiden zusammen angehört, so wie in der parabolischen Bahn eines geworfenen Steins dieser im \match{Gipfel} derselben einen Augenblick als im Steigen und Fallen zugleich begriffen betrachtet werden kann, und so allererst von der steigenden Bewegung zum Fallen übergeht. 
	
	\unnumberedsection{Grenze (1)} 
	\subsection*{tg438.2.24} 
	\textbf{Source : }Die Metaphysik der Sitten/Erster Teil. Metaphysische Anfangsgründe der Rechtslehre/1. Teil. Das Privatrecht vom äußeren Mein und Dein überhaupt/2. Hauptstück. Von der Art, etwas Äußeres zu erwerben/Episodischer Abschnitt. Von der idealen Erwerbung eines äußeren Gegenstandes der Willkür\\  
	
	\noindent\textbf{Paragraphe : }Daß durch ein tadelloses Leben und einen dasselbe beschließenden Tod der Mensch einen (negativ-) guten Namen  als das Seine, welches ihm übrig bleibt, erwerbe, wenn er als homo phaenomenon nicht mehr existiert, und daß die Überlebenden (angehörige, oder fremde) ihn auch vor Recht zu verteidigen befugt sind (weil unerwiesene Anklage sie insgesamt wegen ähnlicher Begegnung auf ihren Sterbefall in Gefahr bringt), daß er, sage ich, ein solches Recht erwerben könne, ist eine sonderbare, nichtsdestoweniger unleugbare Erscheinung der a priori gesetzgebenden Vernunft, die ihr Gebot und Verbot auch über die \match{Grenze} des Lebens hinaus erstreckt. – Wenn jemand von einem Verstorbenen ein Verbrechen verbreitet, das diesen im Leben ehrlos, oder nur verächtlich gemacht haben würde: so kann ein jeder, welcher einen Beweis führen kann, daß diese Beschuldigung vorsätzlich unwahr und gelogen sei, den, welcher jenen in böse Nachrede bringt, für einen Kalumnianten öffentlich erklären, mithin ihn selbst ehrlos machen; welches er nicht tun dürfte, wenn er nicht mit Recht voraussetzte, daß der Verstorbene dadurch beleidigt wäre, ob er gleich tot ist, und daß diesem durch jene Apologie Genugtuung widerfahre, ob er gleich nicht mehr existiert.
	
	
	6
	Die Befugnis, die Rolle  des Apologeten für den Verstorbenen zu spielen, darf dieser auch nicht beweisen; denn jeder Mensch maßt sie sich unvermeidlich an, als nicht bloß zur Tugendpflicht (ethisch betrachtet), sondern so gar zum Recht der Menschheit überhaupt gehörig: und es bedarf hiezu keiner besonderen persönlichen Nachteile, die etwa Freunden und Anverwandten aus einem solchen Schandfleck am Verstorbenen erwachsen dürften, um jenen zu einer solchen Rüge zu berechtigen. – Daß also eine solche ideale Erwerbung und ein Recht des Menschen nach seinem Tode gegen die Überlebenden gegründet sei, ist nicht zu streiten, ob schon die Möglichkeit desselben keiner Deduktion fähig ist. 
	
	\unnumberedsection{Grund (20)} 
	\subsection*{tg430.2.31} 
	\textbf{Source : }Die Metaphysik der Sitten/Erster Teil. Metaphysische Anfangsgründe der Rechtslehre/Einleitung in die Metaphysik der Sitten\\  
	
	\noindent\textbf{Paragraphe : }Auf diesem (in praktischer Rücksicht) positiven Begriffe der Freiheit gründen sich unbedingte praktische Gesetze, welche moralisch heißen, die in Ansehung unser, deren Willkür sinnlich affiziert und so dem reinen Willen nicht von selbst angemessen, sondern oft widerstrebend ist, Imperativen (Gebote oder Verbote) und zwar kategorische (unbedingte) Imperativen sind, wodurch sie sich von den technischen (den Kunst-Vorschriften), als die jederzeit nur bedingt gebieten, unterscheiden, nach denen gewisse Handlungen erlaubt oder unerlaubt, d.i. moralisch möglich oder unmöglich, einige derselben aber, oder ihr Gegenteil moralisch notwendig, d.i. verbindlich sind, woraus dann für jene der Begriff einer Pflicht entspringt, deren Befolgung oder Übertretung zwar auch mit einer Lust oder Unlust von besonderer Art (der eines moralischen Gefühls) verbunden ist, auf welche wir aber (weil sie nicht den \match{Grund} der praktischen Gesetze, sondern nur die subjektive Wirkung im Gemüt bei der Bestimmung unserer Willkür durch jene betreffen und (ohne jener ihrer Gültigkeit oder Einflusse objektiv, d.i. im Urteil der Vernunft, etwas hinzuzutun oder zu benehmen) nach Verschiedenheit der Subjekte verschieden sein kann) in praktischen Gesetzen der Vernunft gar nicht Rücksicht nehmen. 
	
	\subsection*{tg430.2.34} 
	\textbf{Source : }Die Metaphysik der Sitten/Erster Teil. Metaphysische Anfangsgründe der Rechtslehre/Einleitung in die Metaphysik der Sitten\\  
	
	\noindent\textbf{Paragraphe : }
	Der Imperativ ist eine praktische Regel, wodurch die an sich zufällige Handlung notwendig gemacht wird. Er unterscheidet sich darin von einem praktischen Gesetze, daß dieses zwar die Notwendigkeit einer Handlung vorstellig macht, aber ohne Rücksicht darauf zu nehmen, ob diese an sich schon dem handelnden Subjekte (etwa einem heiligen Wesen) innerlich notwendig beiwohne, oder (wie dem Menschen) zufällig sei; denn, wo das erstere ist, da findet kein Imperativ statt. Also ist der Imperativ eine Regel, deren Vorstellung die subjektiv-zufällige Handlung notwendig macht, mithin das Subjekt, als ein solches, was zur Übereinstimmung mit dieser Regel genötigt (nezessitiert) werden muß, vorstellt. – Der kategorische (unbedingte) Imperativ ist derjenige, welcher nicht etwa mittelbar, durch die Vorstellung eines Zwecks, der durch die Handlung erreicht werden könne, sondern der sie durch die bloße Vorstellung dieser Handlung selbst (ihrer Form), also unmittelbar als objektiv-notwendig denkt und notwendig macht; dergleichen Imperativen keine andere praktische Lehre, als allein die, welche Verbindlichkeit vorschreibt (die der Sitten), zum Beispiele aufstellen kann. Alle andere Imperativen sind technisch und insgesamt bedingt. Der \match{Grund} der Möglichkeit kategorischer Imperativen liegt aber darin: daß sie sich auf keine andere Bestimmung der Willkür (wodurch ihr eine Absicht untergelegt werden kann), als lediglich auf die Freiheit derselben beziehen. 
	
	\subsection*{tg431.2.31} 
	\textbf{Source : }Die Metaphysik der Sitten/Erster Teil. Metaphysische Anfangsgründe der Rechtslehre/Einleitung in die Rechtslehre\\  
	
	\noindent\textbf{Paragraphe : }Die Billigkeit (objektiv betrachtet) ist keinesweges ein \match{Grund} zur Aufforderung bloß an die ethische Pflicht anderer (ihr Wohlwollen und Gütigkeit), sondern der, welcher aus diesem Grunde etwas fordert, fußt sich auf sein Recht, nur daß ihm die für den Richter erforderlichen Bedingungen mangeln, nach welchen dieser bestimmen könnte, wie viel, oder auf welche Art dem Anspruche desselben genug getan werden könne. Der in einer auf gleiche Vorteile eingegangenen Maskopei dennoch mehr getan, dabei aber wohl gar durch Unglücksfälle mehr verloren hat, als die übrigen Glieder, kann nach der Billigkeit von der Gesellschaft  mehr fordern, als bloß zu gleichen Teilen mit ihnen zu gehen. Allein nach dem eigentlichen (strikten) Recht, weil, wenn man sich in seinem Fall einen Richter denkt, dieser keine bestimmte Angaben (data) hat, um, wie viel nach dem Kontrakt ihm zukomme, auszumachen, würde er mit seiner Forderung abzuweisen sein. Der Hausdiener, dem sein bis zu Ende des Jahres laufender Lohn in einer binnen der Zeit verschlechterten Münzsorte bezahlt wird, womit er das nicht ausrichten kann, was er bei Schließung des Kontrakts sich dafür anschaffen konnte, kann, bei gleichem Zahlwert, aber ungleichem Geldwert, sich nicht auf sein Recht berufen, deshalb schadlos gehalten zu werden, sondern nur die Billigkeit zum Grunde aufrufen (eine stumme Gottheit, die nicht gehöret werden kann); weil nichts hierüber im Kontrakt bestimmt war, ein Richter aber nach unbestimmten Bedingungen nicht sprechen kann. 
	
	\subsection*{tg431.2.53} 
	\textbf{Source : }Die Metaphysik der Sitten/Erster Teil. Metaphysische Anfangsgründe der Rechtslehre/Einleitung in die Rechtslehre\\  
	
	\noindent\textbf{Paragraphe : }2) Der Rechte, als (moralischer) Vermögen, andere zu verpflichten, d.i. als einen gesetzlichen \match{Grund} zu den letzteren (titulum), von denen die Obereinteilung die in das angeborne und erworbene Recht ist, deren ersteres dasjenige Recht ist, welches, unabhängig von allem rechtlichen Akt, jedermann von Natur zukommt; das zweite das, wozu ein solcher Akt erfordert wird. 
	
	\subsection*{tg431.2.66} 
	\textbf{Source : }Die Metaphysik der Sitten/Erster Teil. Metaphysische Anfangsgründe der Rechtslehre/Einleitung in die Rechtslehre\\  
	
	\noindent\textbf{Paragraphe : }Warum wird aber die Sittenlehre (Moral) gewöhnlich (namentlich vom Cicero) die Lehre von den Pflichten und nicht auch von den Rechten betitelt? da doch die einen sich auf die andern beziehen. – Der \match{Grund} ist dieser: Wir kennen unsere eigene Freiheit (von der alle moralische Gesetze, mithin auch alle Rechte sowohl als Pflichten ausgehen) nur durch den moralischen Imperativ, welcher ein pflichtgebieten der Satz ist, aus welchem nachher das Vermögen, andere zu verpflichten, d.i. der Begriff des Rechts, entwickelt werden kann. 
	
	\subsection*{tg433.2.34} 
	\textbf{Source : }Die Metaphysik der Sitten/Erster Teil. Metaphysische Anfangsgründe der Rechtslehre/1. Teil. Das Privatrecht vom äußeren Mein und Dein überhaupt/1. Hauptstück\\  
	
	\noindent\textbf{Paragraphe : }Der bloße physische Besitz (die Inhabung) des Bodens ist schon ein Recht in einer Sache, obzwar freilich noch nicht hinreichend, ihn als das Meine anzusehen. Beziehungsweise auf andere ist er, als (so viel man weiß) erster Besitz, mit dem Gesetz der äußern Freiheit einstimmig, und zugleich in dem ursprünglichen Gesamtbesitz enthalten, der a priori den \match{Grund} der Möglichkeit eines Privatbesitzes enthält; mithin den ersten Inhaber eines Bodens in seinem Gebrauch desselben zu stören eine Läsion. Die erste Besitznehmung hat also einen Rechtsgrund (titulus possessionis) für sich, welcher der ursprünglich gemeinsame Besitz ist, und der Satz: wohl dem, der im Besitz ist (beati possidentes)! weil niemand verbunden ist, seinen Besitz zu beurkunden, ist ein Grundsatz des natürlichen Rechts, der die erste Besitznehmung als einen rechtlichen Grund zur Erwerbung aufstellt, auf den sich jeder erste Besitzer fußen kann. 
	
	\subsection*{tg433.2.41} 
	\textbf{Source : }Die Metaphysik der Sitten/Erster Teil. Metaphysische Anfangsgründe der Rechtslehre/1. Teil. Das Privatrecht vom äußeren Mein und Dein überhaupt/1. Hauptstück\\  
	
	\noindent\textbf{Paragraphe : }Der Begriff eines bloß-rechtlichen Besitzes ist kein empirischer (von Raum und Zeitbedingungen abhängiger) Begriff, und gleichwohl hat er praktische Realität, d.i. er muß auf Gegenstände der Erfahrung, deren Erkenntnis von jenen Bedingungen abhängig ist, anwendbar sein. – Das Verfahren mit dem Rechtsbegriffe in Ansehung der letzteren, als des möglichen äußeren Mein und Dein, ist folgendes: Der Rechtsbegriff, der bloß in der Vernunft liegt, kann nicht unmittelbar auf Erfahrungsobjekte, und auf den Begriff eines empirischen Besitzes, sondern muß zunächst auf den reinen Verstandesbegriff eines Besitzes überhaupt angewandt werden, so daß, statt der Inhabung (detentio), als einer empirischen Vorstellung des Besitzes, der von allen Raumes- und Zeitbedingungen abstrahierende Begriff des Habens und nur, daß der Gegenstand als in meiner Gewalt (in potestate mea positum esse) sei, gedacht werde; da dann der Ausdruck des Äußeren nicht das Dasein in einem anderen Orte, als wo ich bin, oder meiner Willensentschließung und Annahme als in einer anderen Zeit, wie der des Angebots, sondern nur einen von mir unterschiedenen Gegenstand bedeutet. Nun will die praktische Vernunft durch ihr Rechtsgesetz, daß ich das Mein und Dein in der Anwendung auf Gegenstände nicht nach sinnlichen Bedingungen, sondern abgesehen von denselben, weil es eine Bestimmung der Willkür nach Freiheitsgesetzen betrifft, auch den Besitz desselben denke, indem nur ein Verstandesbegriff unter Rechtsbegriffe subsumiert werden kann. Also werde ich sagen: ich besitze einen Acker, ob er zwar ein ganz anderer Platz ist, als worauf ich mich wirklich befinde. Denn die Rede ist hier nur von einem intellektuellen Verhältnis zum Gegenstande, so fern ich ihn in meiner Gewalt habe (ein von Raumesbestimmungen unabhängiger Verstandesbegriff des Besitzes), und er ist mein,  weil mein, zu desselben beliebigem Gebrauch sich bestimmender, Wille dem Gesetz der äußeren Freiheit nicht widerstreitet. Gerade darin: daß, abgesehen vom Besitz in der Erscheinung (der Inhabung) dieses Gegenstandes meiner Willkür, die praktische Vernunft den Besitz nach Verstandesbegriffen, nicht nach empirischen, sondern solchen, die a priori die Bedingungen desselben enthalten können, gedacht wissen will, liegt der \match{Grund} der Gültigkeit eines solchen Begriffs vom Besitze (possessio noumenon) als einer allgemeingeltenden Gesetzgebung; denn eine solche ist in dem Ausdrucke enthalten: »dieser äußere Gegenstand ist mein«; weil allen andern dadurch eine Verbindlichkeit auferlegt wird, die sie sonst nicht hätten, sich des Gebrauchs desselben zu enthalten. 
	
	\subsection*{tg435.2.37} 
	\textbf{Source : }Die Metaphysik der Sitten/Erster Teil. Metaphysische Anfangsgründe der Rechtslehre/1. Teil. Das Privatrecht vom äußeren Mein und Dein überhaupt/2. Hauptstück. Von der Art, etwas Äußeres zu erwerben/1. Abschnitt. Vom Sachrecht\\  
	
	\noindent\textbf{Paragraphe : }Daß die erste Bearbeitung, Begrenzung, oder überhaupt Formgebung eines Bodens keinen Titel der Erwerbung desselben, d.i. der Besitz des Akzidens nicht ein \match{Grund} des rechtlichen Besitzes der Substanz abgeben könne, sondern vielmehr umgekehrt das Mein und Dein nach der Regel (accessorium sequitur suum principale) aus dem Eigentum der Substanz gefolgert werden müsse, und daß der, welcher an einen Boden, der nicht schon vorher der seine war, Fleiß verwendet, seine Mühe und Arbeit gegen den ersteren verloren hat, ist für sich selbst so klar, daß man jene so alte und noch weit und breit herrschende Meinung schwerlich einer anderen Ursache zuschreiben kann, als der in geheim obwaltenden Täuschung, Sachen zu personifizieren und, gleich als ob jemand sie sich durch an sie verwandte Arbeit verbindlich machen könne, keinem anderen als ihm zu Diensten zu stehen, unmittelbar gegen sie sich ein Recht zu denken; denn wahrscheinlicherweise würde man auch nicht so leichten Fußes über die natürliche Frage (von der oben schon Erwähnung geschehen) weggeglitten sein: »wie ist ein Recht in einer Sache möglich?« Denn das Recht gegen einen jeden Besitzer einer Sache bedeutet nur die Befugnis der besonderen Willkür zum Gebrauch eines, Objekts, so fern sie als im synthetisch-allgemeinen Willen enthalten, und mit dem Gesetz desselben zusammenstimmend gedacht werden kann. 
	
	\subsection*{tg439.2.37} 
	\textbf{Source : }Die Metaphysik der Sitten/Erster Teil. Metaphysische Anfangsgründe der Rechtslehre/1. Teil. Das Privatrecht vom äußeren Mein und Dein überhaupt/3. Hauptstück. Von der subjektiv-bedingten Erwerbung durch den Ausspruch einer öffentlichen Gerichtsbarkeit\\  
	
	\noindent\textbf{Paragraphe : }Man kann keinen anderen \match{Grund} angeben, der rechtlich Menschen verbinden könnte, zu glauben und zu bekennen, daß es Götter gebe, als den, damit sie einen Eid schwören, und durch die Furcht vor einer allsehenden obersten Macht, deren Rache sie feierlich gegen sich aufrufen mußten, im Fall, daß ihre Aussage falsch wäre, genötigt werden könnten, wahrhaft im Aussagen und treu im Versprechen zu sein. Daß man hiebei nicht auf die Moralität dieser beiden Stücke, sondern bloß auf einen blinden Aberglauben derselben rechnete, ist daraus zu ersehen, daß man sich von ihrer bloßen feierlichen Aussage vor Gericht in Rechtssachen keine Sicherheit versprach, ob gleich die Pflicht der Wahrhaftigkeit in einem Fall, wo es auf das Heiligste, was unter Menschen nur sein kann (aufs Recht der Menschen), an kommt, jedermann so klar einleuchtet, mithin bloße Märchen den Bewegungsgrund ausmachen: wie z.B. das unter den Rejangs, einem heidnischen Volk auf Sumatra, welche, nach Marsdens Zeugnis, bei den Knochen ihrer verstorbenen Anverwandten schwören, ob sie gleich gar nicht glauben, daß es noch ein Leben nach dem Tode gebe, oder der Eid der Guineaschwarzen bei ihrem Fetisch, etwa einer Vogelfeder, auf die sie sich vermessen, daß sie ihnen den Hals brechen solle u. dergl. Sie glauben, daß eine unsichtbare  Macht, sie mag nun Verstand haben oder nicht, schon ihrer Natur nach, diese Zauberkraft habe, die durch einen solchen Aufruf in Tat versetzt wird. – Ein solcher Glaube, dessen Name Religion ist, eigentlich aber Superstition heißen sollte, ist aber für die Rechtsverwaltung unentbehrlich, weil, ohne auf ihn zu rechnen, der Gerichtshof nicht genugsam im Stande wäre, geheim gehaltene Facta auszumitteln, und Recht zu sprechen. Ein Gesetz, das hiezu verbindet, ist also offenbar nur zum Behuf der richtenden Gewalt gegeben. 
	
	\subsection*{tg439.2.50} 
	\textbf{Source : }Die Metaphysik der Sitten/Erster Teil. Metaphysische Anfangsgründe der Rechtslehre/1. Teil. Das Privatrecht vom äußeren Mein und Dein überhaupt/3. Hauptstück. Von der subjektiv-bedingten Erwerbung durch den Ausspruch einer öffentlichen Gerichtsbarkeit\\  
	
	\noindent\textbf{Paragraphe : }Aus dem Privatrecht im natürlichen Zustande geht nun das Postulat des öffentlichen Rechts hervor: du sollst, im Verhältnisse eines unvermeidlichen Nebeneinanderseins, mit allen anderen, aus jenem heraus, in einen rechtlichen Zustand, d.i. den einer austeilenden Gerechtigkeit, übergehen. – Der \match{Grund} davon läßt sich analytisch aus dem Begriffe des Rechts, im äußeren Verhältnis, im Gegensatz der Gewalt (violentia) entwickeln. 
	
	\subsection*{tg441.2.44} 
	\textbf{Source : }Die Metaphysik der Sitten/Erster Teil. Metaphysische Anfangsgründe der Rechtslehre/2. Teil. Das öffentliche Recht/1. Abschnitt. Das Staatsrecht\\  
	
	\noindent\textbf{Paragraphe : }Wider das gesetzgebende Oberhaupt des Staats gibt es also keinen rechtmäßigen Widerstand des Volks; denn nur durch Unterwerfung unter seinen allgemein-gesetzgebenden Willen ist ein rechtlicher Zustand möglich; also kein Recht des Aufstandes (seditio), noch weniger des Aufruhrs (rebellio), am allerwenigsten gegen ihn, als einzelne Person (Monarch), unter dem Verwände des Mißbrauchs seiner Gewalt (tyrannis), Vergreifung an seiner Person, ja an seinem Leben (monarchomachismus sub specie tyrannicidii). Der geringste Versuch hiezu ist Hochverrat (proditio eminens), und der Verräter dieser Art kann als einer, der sein 
	Vaterland umzubringen versucht (parricida), nicht minder als mit dem Tode bestraft werden. – – Der \match{Grund} der Pflicht des Volks, einen, selbst den für unerträglich ausgegebenen Mißbrauch der obersten Gewalt dennoch zu ertragen, liegt darin: daß sein Widerstand wider die höchste Gesetzgebung selbst niemals anders, als gesetzwidrig, ja als die ganze gesetzliche Verfassung zernichtend gedacht werden muß. Denn, um zu demselben befugt zu sein, müßte ein öffentliches Gesetz vorhanden sein, welches diesen Widerstand des Volks erlaubte, d.i. die oberste Gesetzgebung enthielte eine Bestimmung in sich, nicht die oberste zu sein, und das Volk, als Untertan, in einem und demselben Urteile zum Souverän über den zu machen, dem es untertänig ist; welches sich widerspricht und wovon der Widerspruch durch die Frage alsbald in die Augen fällt: wer denn in diesem Streit zwischen Volk und Souverän Richter sein sollte (denn es sind rechtlich betrachtet doch immer zwei verschiedene moralische Personen); wo sich dann zeigt, daß das erstere es in seiner eigenen Sache sein will.
	
	
	8
	
	
	
	\subsection*{tg456.2.5} 
	\textbf{Source : }Die Metaphysik der Sitten/Zweiter Teil. Metaphysische Anfangsgründe der Tugendlehre/Einleitung/VIII. Exposition der Tugendpflichten als weiter Pflichten\\  
	
	\noindent\textbf{Paragraphe : }b) Kultur der Moralität in uns. Die größte moralische Vollkommenheit des Menschen ist: seine Pflicht zu tun und zwar aus Pflicht (daß das Gesetz nicht bloß die Regel sondern auch die Triebfeder der Handlungen sei). – Nun scheint dieses zwar beim ersten Anblick eine enge Verbindlichkeit zu sein und das Pflichtprinzip zu jeder Handlung nicht bloß die Legalität, sondern auch die Moralität, d.i. Gesinnung, mit der Pünktlichkeit und Strenge eines Gesetzes zu gebieten; aber in der Tat gebietet das Gesetz auch hier nur, die Maxime der Handlung, nämlich den \match{Grund} der Verpflichtung nicht in den sinnlichen Antrieben (Vorteil oder Nachteil), sondern ganz und gar im Gesetz zu suchen – mithin nicht die Handlung selbst. – – Denn es ist dem Menschen nicht möglich, so in die Tiefe seines eigenen Herzens einzuschauen, daß er jemals von der Reinigkeit seiner moralischen Absicht und der Lauterkeit seiner Gesinnung auch nur in einer Handlung völlig gewiß sein könnte; wenn er gleich über die Legalität derselben gar nicht zweifelhaft ist. Vielmals wird Schwäche, welche das Wagstück eines Verbrechens abrät, von demselben Menschen für Tugend (die den Begriff von Stärke gibt) gehalten, und wie viele mögen ein langes schuldloses Leben geführt haben, die nur Glückliche sind, so vielen Versuchungen entgangen zu sein; wie viel reiner moralischer Gehalt bei jeder Tat in der Gesinnung gelegen habe, das bleibt ihnen selbst verborgen. 
	
	\subsection*{tg460.2.27} 
	\textbf{Source : }Die Metaphysik der Sitten/Zweiter Teil. Metaphysische Anfangsgründe der Tugendlehre/Einleitung/XII. Ästhetische Vorbegriffe der Empfänglichkeit des Gemüts für Pflichtbegriffe überhaupt\\  
	
	\noindent\textbf{Paragraphe : }Achtung (reverentia) ist eben sowohl etwas bloß Subjektives; ein Gefühl eigener Art, nicht ein Urteil über einen Gegenstand, den zu bewirken, oder zu befördern, es eine Pflicht gäbe. Denn sie könnte, als Pflicht betrachtet, nur durch die Achtung, die wir vor ihr haben, vorgestellt werden. Zu dieser also eine Pflicht zu haben würde so viel sagen, als zur Pflicht verpflichtet werden. – Wenn es demnach heißt: Der Mensch hat eine Pflicht der Selbstschätzung, so ist das unrichtig gesagt und es müßte vielmehr heißen: das Gesetz in ihm zwingt ihm unvermeidlich Achtung für sein eigenes Wesen ab und dieses Gefühl (welches von eigner Art ist) ist ein \match{Grund} gewisser Pflichten, d.i. gewisser Handlungen, die mit der Pflicht gegen sich selbst zusammen bestehen können, nicht: er habe eine Pflicht der Achtung gegen sich; denn er muß Achtung vor dem Gesetz in sich selbst haben, um sich nur eine Pflicht überhaupt denken zu können. 
	
	\subsection*{tg461.2.2} 
	\textbf{Source : }Die Metaphysik der Sitten/Zweiter Teil. Metaphysische Anfangsgründe der Tugendlehre/Einleitung/XIII. Allgemeine Grundsätze der Metaphysik der Sitten in Behandlung einer reinen Tugendlehre\\  
	
	\noindent\textbf{Paragraphe : }
	Erstlich: Für Eine Pflicht kann auch nur ein einziger \match{Grund} der Verpflichtung gefunden werden, und, werden zwei oder mehrere Beweise darüber geführt, so ist es ein sicheres Kennzeichen, daß man entweder noch gar keinen gültigen Beweis habe, oder es auch mehrere und verschiedne Pflichten sind, die man für Eine gehalten hat. 
	
	\subsection*{tg461.2.3} 
	\textbf{Source : }Die Metaphysik der Sitten/Zweiter Teil. Metaphysische Anfangsgründe der Tugendlehre/Einleitung/XIII. Allgemeine Grundsätze der Metaphysik der Sitten in Behandlung einer reinen Tugendlehre\\  
	
	\noindent\textbf{Paragraphe : }Denn alle moralische Beweise können, als philosophische, nur vermittelst einer Vernunfterkenntnis aus Begriffen, nicht, wie die Mathematik sie gibt, durch die Konstruktion der Begriffe geführt werden; die letztern verstatten Mehrheit  der Beweise eines und desselben Satzes; weil in der Anschauung a priori es mehrere Bestimmungen der Beschaffenheit eines Objekts geben kann, die alle auf eben denselben \match{Grund} zurück führen. – Wenn z.B. für die Pflicht der Wahrhaftigkeit ein Beweis, erstlich aus dem Schaden, den die Lüge andern Menschen verursacht, dann aber auch aus der Nichtswürdigkeit eines Lügners und der Verletzung der Achtung gegen sich selbst, geführt werden will, so ist im ersteren eine Pflicht des Wohlwollens, nicht eine der Wahrhaftigkeit, mithin nicht diese, von der man den Beweis verlangte, sondern eine andere Pflicht bewiesen worden. – Was aber die Mehrheit der Beweise für einen und denselben Satz betrifft, womit man sich tröstet, daß die Menge der Gründe den Mangel am Gewicht eines jeden einzeln genommen ergänzen werde, so ist dieses ein sehr unphilosophischer Behelf: weil er Hinterlist und Unredlichkeit verrät; – denn verschiedene unzureichende Gründe, neben einander gestellt, ergänzen nicht der eine den Mangel des anderen zur Gewißheit, ja nicht einmal zur Wahrscheinlichkeit. Sie müssen als Grund und Folge in einer Reihe, bis zum zureichenden Grunde, fortschreiten und können auch nur auf solche Art beweisend sein. – Und gleichwohl ist dies der gewöhnliche Handgriff der Überredungskunst. 
	
	\subsection*{tg481.2.60} 
	\textbf{Source : }Die Metaphysik der Sitten/Zweiter Teil. Metaphysische Anfangsgründe der Tugendlehre/I. Ethische Elementarlehre/II. Teil. Von den Tugendpflichten gegen andere/Erstes Hauptstück. Von den Pflichten gegen andere, bloß als Menschen/Erster Abschnitt. Von der Liebespflicht gegen andere Menschen\\  
	
	\noindent\textbf{Paragraphe : }Dankbarkeit aber muß auch noch besonders als heilige Pflicht, d.i. als eine solche, deren Verletzung die moralische Triebfeder zum Wohltun in dem Grundsatze selbst vernichten kann (als skandalöses Beispiel), angesehen werden. Denn heilig ist derjenige moralische Gegenstand, in Ansehung dessen die Verbindlichkeit durch keinen ihr gemäßen Akt völlig getilgt werden kann (wobei der Verpflichtete immer noch verpflichtet bleibt). Alle andere ist gemeine Pflicht. – Man kann aber durch keine Vergeltung einer empfangenen Wohltat über dieselbe quittieren; weil der Empfänger den Vorzug des Verdienstes, den der Geber hat, nämlich der erste im Wohlwollen gewesen zu sein, diesem nie abgewinnen kann. – Aber, auch ohne einen solchen Akt (des Wohltuns) ist selbst das bloße herzliche Wohlwollen schon \match{Grund} der Verpflichtung zur Dankbarkeit. Eine dankbare Gesinnung dieser Art wird Erkenntlichkeit genannt. 
	
	\subsection*{tg481.2.85} 
	\textbf{Source : }Die Metaphysik der Sitten/Zweiter Teil. Metaphysische Anfangsgründe der Tugendlehre/I. Ethische Elementarlehre/II. Teil. Von den Tugendpflichten gegen andere/Erstes Hauptstück. Von den Pflichten gegen andere, bloß als Menschen/Erster Abschnitt. Von der Liebespflicht gegen andere Menschen\\  
	
	\noindent\textbf{Paragraphe : }b) Undankbarkeit gegen seinen Wohltäter, welche, wenn sie gar so weit geht, seinen Wohltäter zu hassen, 
	qualifizierte Undankbarkeit, sonst aber bloß Unerkenntlichkeit heißt, ist ein zwar im öffentlichen Urteile höchst verabscheutes Laster, gleichwohl ist der Mensch desselben wegen so berüchtigt, daß man es nicht für unwahrscheinlich hält, man könne sich durch erzeigte Wohltaten wohl gar einen Feind machen. – Der \match{Grund} der Möglichkeit eines solchen Lasters liegt in der mißverstandenen Pflicht gegen sich selbst, die Wohltätigkeit anderer, weil sie uns Verbindlichkeit gegen sie auferlegt, nicht zu bedürfen und aufzufordern, sondern lieber die Beschwerden des Lebens selbst zu ertragen, als andere damit zu belästigen, mithin dadurch bei ihnen in Schulden (Verpflichtung) zu kommen; weil wir dadurch auf die niedere Stufe des Beschützten gegen seinen Beschützer zu geraten fürchten; welches der echten Selbstschätzung (auf die Würde der Menschheit in seiner eigenen Person stolz zu sein) zuwider ist. Daher Dankbarkeit gegen die, die uns im Wohltun unvermeidlich zuvor kommen mußten (gegen Vorfahren im Angedenken, oder gegen Eltern), freigebig, die aber gegen Zeitgenossen nur kärglich, ja, um dieses Verhältnis der Ungleichheit unsichtbar zu machen, wohl gar das Gegenteil derselben bewiesen wird. – Dieses ist aber alsdann ein die Menschheit empörendes Laster, nicht bloß des Schadens wegen, den ein solches Beispiel Menschen überhaupt zuziehen muß, von fernerer Wohltätigkeit abzuschrecken (denn diese können mit echtmoralischer Gesinnung, eben in der Verschmähung alles solchen Lohns ihrem Wohltun nur einen desto größeren inneren moralischen Wert setzen): sondern weil die Menschenliebe hier gleichsam auf den Kopf gestellt, und der Mangel der Liebe gar in die Befugnis, den Liebenden zu hassen, verunedelt wird. 
	
	\subsection*{tg488.2.5} 
	\textbf{Source : }Die Metaphysik der Sitten/Zweiter Teil. Metaphysische Anfangsgründe der Tugendlehre/Beschluß. Die Religionslehre als Lehre der Pflichten gegen Gott liegt außerhalb den Grenzen der reinen Moralphilosophie\\  
	
	\noindent\textbf{Paragraphe : }Das Formale aller Religion, wenn man sie so erklärt: sie sei »der Inbegriff aller Pflichten als (instar) göttlicher Gebote«, gehört zur philosophischen Moral, indem dadurch nur die Beziehung der Vernunft auf die Idee von Gott, welche sie sich selber macht, ausgedrückt wird, und eine Religionspflicht wird alsdann noch nicht zur Pflicht gegen (erga) Gott, als ein außer unserer Idee existierendes Wesen gemacht, indem wir hiebei von der Existenz desselben noch abstrahieren. – Daß alle Menschenpflichten diesem Formalen (der Beziehung derselben auf einen göttlichen, a priori gegebenen, Willen) gemäß gedacht werden sollen, davon ist der \match{Grund} nur subjektiv-logisch. Wir können uns nämlich Verpflichtung (moralische Nötigung) nicht wohl anschaulich machen, ohne einen anderen und dessen Willen (von dem die allgemein gesetzgebende Vernunft nur der Sprecher ist), nämlich Gott, dabei zu denken. – – Allein diese Pflicht in Ansehung Gottes (eigentlich der Idee, welche wir uns von einem solchen Wesen machen) ist Pflicht des Menschen gegen sich selbst, d.i. nicht objektive die Verbindlichkeit zur Leistung gewisser Dienste an einen anderen, sondern nur subjektive zur Stärkung der moralischen Triebfeder in unserer eigenen gesetzgebenden Vernunft. 
	
	\subsection*{tg489.2.10} 
	\textbf{Source : }Die Metaphysik der Sitten/Fußnoten\\  
	
	\noindent\textbf{Paragraphe : }
	
	5 Selbst nicht, wie es möglich ist, daß Gott freie Wesen erschaffe; denn da wären, wie es scheint, alle künftige Handlungen derselben, durch jenen ersten Akt vorherbestimmt, in der Kette der Naturnotwendigkeit enthalten, mithin nicht frei. Daß sie aber (wir Menschen) doch frei sind, beweiset der kategorische Imperativ in moralisch-praktischer Absicht, wie durch einen Machtspruch der Vernunft, ohne daß diese doch die Möglichkeit dieses Verhältnisses einer Ursache zur Wirkung in theoretischer begreiflich machen kann, weil beide übersinnlich sind. – Was man ihr hiebei allein zumuten kann, wäre bloß: daß sie beweist, es sei in dem Begriffe von einer Schöpfung freier Wesen kein Widerspruch; und dieses kann dadurch gar wohl geschehen, daß gezeigt wird: der Widerspruch eräugne sich nur dann, wenn mit der Kategorie der Kausalität zugleich die Zeitbedingung, die im Verhältnis zu Sinnenobjekten nicht vermieden werden kann (daß nämlich der \match{Grund} einer Wirkung vor dieser vorhergehe), auch in das Verhältnis des Übersinnlichen zu einander hinüber gezogen wird (welches auch wirklich, wenn jener Kausalbegriff in theoretischer Absicht objektive Realität bekommen soll, geschehen müßte), er – der Widerspruch – aber verschwinde, wenn, in moralisch-praktischer, mithin nicht-sinnlicher Absicht, die reine Kategorie (ohne ein ihr untergelegtes Schema) im Schöpfungsbegriffe gebraucht wird. 
	
	\subsection*{tg489.2.19} 
	\textbf{Source : }Die Metaphysik der Sitten/Fußnoten\\  
	
	\noindent\textbf{Paragraphe : }Der \match{Grund} des Schauderhaften, bei dem Gedanken von der förmlichen Hinrichtung eines Monarchen durch sein Volk, ist also der, daß der Mord nur als Ausnahme von der Regel, welche dieses sich zur Maxime machte, die Hinrichtung aber als eine völlige Umkehrung der Prinzipien des Verhältnisses zwischen Souverän und Volk (dieses, was sein Dasein nur der Gesetzgebung des ersteren zu verdanken hat, zum Herrscher über jenen zu machen) gedacht werden muß, und so die Gewalttätigkeit mit dreuster Stirn und nach Grundsätzen über das heiligste Recht erhoben wird; welches, wie ein alles ohne Wiederkehr verschlingender Abgrund, als ein vom Staate an ihm verübter Selbstmord, ein keiner Entsündigung fähiges Verbrechen zu sein scheint. Man hat also Ursache anzunehmen, daß die Zustimmung zu solchen Hinrichtungen wirklich nicht aus einem vermeint-rechtlichen Prinzip, sondern aus Furcht vor Rache des vielleicht dereinst wiederauflebenden Staats am Volk herrührte, und jene Förmlichkeit nur vorgenommen worden, um jener Tat den Anstrich von Bestrafung, mithin eines rechtlichen Verfahrens (dergleichen der Mord nicht sein würde) zu geben, welche Bemäntelung aber verunglückt, weil eine solche Anmaßung des Volks noch ärger ist, als selbst der Mord, da diese einen Grundsatz enthält, der selbst die Wiedererzeugung eines umgestürzten Staats unmöglich machen müßte. 
	
	\unnumberedsection{Hang (2)} 
	\subsection*{tg475.2.9} 
	\textbf{Source : }Die Metaphysik der Sitten/Zweiter Teil. Metaphysische Anfangsgründe der Tugendlehre/I. Ethische Elementarlehre/I. Teil. Von den Pflichten gegen sich selbst überhaupt/Erstes Buch. Von den vollkommenen Pflichten gegen sich selbst/Zweites Hauptstück. Die Pflicht des Menschen gegen sich selbst, bloß als einem moralischen Wesen/Episodischer Abschnitt. Von der Amphibolie der moralischen Reflexionsbegriffe\\  
	
	\noindent\textbf{Paragraphe : }In Ansehung des Schönen obgleich Leblosen in der Natur ist ein \match{Hang} zum bloßen Zerstören (spiritus destructionis) der Pflicht des Menschen gegen sich selbst zuwider; weil es dasjenige Gefühl im Menschen schwächt oder vertilgt, was zwar nicht für sich allein schon moralisch ist, aber doch diejenige Stimmung der Sinnlichkeit, welche die Moralität sehr befördert, wenigstens dazu vorbereitet, nämlich etwas auch ohne Absicht auf Nutzen zu lieben (z.B. die schöne Kristallisationen, das unbeschreiblich Schöne des Gewächsreichs). 
	
	\subsection*{tg489.2.31} 
	\textbf{Source : }Die Metaphysik der Sitten/Fußnoten\\  
	
	\noindent\textbf{Paragraphe : }
	
	14 Der Mensch aber findet sich doch als moralisches Wesen zugleich, wenn er sich objektiv, wozu er durch seine reine praktische Vernunft bestimmt ist, (nach der Menschheit in seiner eigenen Person) betrachtet, heilig genug, um das innere Gesetz ungern zu übertreten; denn es gibt keinen so verruchten Menschen, der bei dieser Übertretung in sich nicht einen Widerstand fühlete und eine Verabscheuung seiner selbst, bei der er sich selbst Zwang antun muß. – Das Phänomen nun: daß der Mensch auf diesem Scheidewege (wo die schöne Fabel den Herkules zwischen Tugend und Wohllust hinstellt) mehr \match{Hang} zeigt, der Neigung als dem Gesetz Gehör zu geben, zu erklären ist unmöglich: weil wir, was geschieht, nur erklären können, indem wir es von einer Ursache nach Gesetzen der Natur ableiten; wobei wir aber die Willkür nicht als frei denken würden. – Dieser wechselseitig entgegengesetzte Selbstzwang aber und die Unvermeidlichkeit desselben gibt doch die unbegreifliche Eigenschaft der Freiheit selbst zu erkennen. 
	
	\unnumberedsection{Himmel (3)} 
	\subsection*{tg445.2.54} 
	\textbf{Source : }Die Metaphysik der Sitten/Erster Teil. Metaphysische Anfangsgründe der Rechtslehre/Anhang erläutender Bemerkungen zu den metaphysischen Anhangsgründen der Rechtslehre\\  
	
	\noindent\textbf{Paragraphe : }Die Frage ist hier: ob die Kirche dem Staat oder der Staat der Kirche als das Seine angehören könne; denn zwei oberste Gewalten können einander ohne Widerspruch nicht untergeordnet sein. – Daß nur die erstere Verfassung (politico-hierarchica) Bestand an sich haben könne, ist an sich klar: denn alle bürgerliche Verfassung ist von dieser Welt, weil sie eine irdische Gewalt (der Menschen) ist, die sich samt ihren Folgen in der Erfahrung dokumentieren läßt. Die Gläubigen, deren Reich im \match{Himmel} und in jener Welt ist, müssen, in so fern man ihnen eine sich auf dieses beziehende Verfassung (hierarchico-politica) zugesteht, sich den Leiden dieser Zeit unter der Obergewalt der Weltmenschen unterwerfen. – Also findet nur die erstere Verfassung statt. 
	
	\subsection*{tg473.2.7} 
	\textbf{Source : }Die Metaphysik der Sitten/Zweiter Teil. Metaphysische Anfangsgründe der Tugendlehre/I. Ethische Elementarlehre/I. Teil. Von den Pflichten gegen sich selbst überhaupt/Erstes Buch. Von den vollkommenen Pflichten gegen sich selbst/Zweites Hauptstück. Die Pflicht des Menschen gegen sich selbst, bloß als einem moralischen Wesen/1. Abschnitt. Von der Pflicht des Menschen gegen sich selbst, als dem angebornen Richter über sich selbst\\  
	
	\noindent\textbf{Paragraphe : }Eine solche idealische Person (der autorisierte Gewissensrichter) muß ein Herzenskündiger sein; denn der Gerichtshof ist im Inneren des Menschen aufgeschlagen – zugleich muß er aber auch allverpflichtend, d.i. eine solche Person sein, oder als eine solche gedacht werden, in Verhältnis auf welche alle Pflichten überhaupt auch als ihre Gebote anzusehen sind; weil das Gewissen über alle freie Handlungen der innere Richter ist. – – Da nun ein solches moralisches Wesen zugleich alle Gewalt (im \match{Himmel} und auf Erden) haben muß, weil es sonst nicht (was doch zum Richteramt notwendig gehört) seinen Gesetzen den ihnen angemessenen Effekt verschaffen könnte, ein solches über alles machthabende moralische Wesen aber Gott heißt: so wird das Gewissen als subjektives Prinzip einer vor Gott seiner Taten wegen zu leistenden Verantwortung gedacht werden müssen; ja es wird der letztere Begriff (wenn gleich nur auf dunkele Art) in jenem moralischen Selbstbewußtsein jederzeit enthalten sein. 
	
	\subsection*{tg481.2.91} 
	\textbf{Source : }Die Metaphysik der Sitten/Zweiter Teil. Metaphysische Anfangsgründe der Tugendlehre/I. Ethische Elementarlehre/II. Teil. Von den Tugendpflichten gegen andere/Erstes Hauptstück. Von den Pflichten gegen andere, bloß als Menschen/Erster Abschnitt. Von der Liebespflicht gegen andere Menschen\\  
	
	\noindent\textbf{Paragraphe : }Alle Laster, welche selbst die menschliche Natur hassenswert machen würden, wenn man sie (als qualifiziert) in der Bedeutung von Grundsätzen nehmen wollte, sind inhuman, objektiv betrachtet, aber doch menschlich, subjektiv erwogen: d.i. wie die Erfahrung uns unsere Gattung kennen lehrt. Ob man also zwar einige derselben in der Heftigkeit des Abscheues teuflisch nennen möchte, so wie ihr Gegenstück Engelstugend genannt werden könnte: so sind beide Begriffe doch nur Ideen von einem Maximum, als Maßstab zum Behuf der Vergleichung des Grades der Moralität gedacht, indem man dem Menschen seinen Platz im \match{Himmel} oder der Hölle anweiset, ohne aus ihm ein Mittelwesen, was weder den einen dieser Plätze, noch den anderen einnimmt, zu machen. Ob es Haller, mit seinem »zweideutig Mittelding von Engeln und von Vieh«, besser getroffen habe, mag hier unausgemacht bleiben. Aber das Halbieren. in einer Zusammenstellung heterogener Dinge führt auf gar keinen bestimmten Begriff, und zu diesem kann uns in der Ordnung der Wesen nach ihrem uns unbekannten Klassenunterschiede nichts hinleiten. Die erstere Gegeneinanderstellung (von Engelstugend und teuflischem  Laster) ist Übertreibung. Die zweite, ob zwar Menschen leider! auch in viehische Laster fallen, berechtigt doch nicht, eine zu ihrer Spezies gehörige Anlage dazu ihnen beizulegen, so wenig, als die Verkrüppelung einiger Bäume im Walde ein Grund ist, sie zu einer besondern Art von Gewächsen zu machen. 
	
	\unnumberedsection{Lage (1)} 
	\subsection*{tg431.2.24} 
	\textbf{Source : }Die Metaphysik der Sitten/Erster Teil. Metaphysische Anfangsgründe der Rechtslehre/Einleitung in die Rechtslehre\\  
	
	\noindent\textbf{Paragraphe : }Das Gesetz eines mit jedermanns Freiheit notwendig zusammenstimmenden wechselseitigen Zwanges, unter dem Prinzip der allgemeinen Freiheit, ist gleichsam die Konstruktion jenes Begriffs, d.i. Darstellung desselben in einer reinen Anschauung a priori, nach der Analogie der Möglichkeit freier Bewegungen der Körper unter dem Gesetze der Gleichheit der Wirkung und Gegenwirkung. So wie wir nun in der reinen Mathematik die Eigenschaften ihres Objekts nicht unmittelbar vom Begriffe ableiten, sondern nur durch die Konstruktion des Begriffs entdecken können, so ist's nicht sowohl der Begriff des Rechts, als vielmehr der, unter allgemeine Gesetze gebrachte, mit ihm zusammenstimmende durchgängig wechselseitige und gleiche Zwang, der die Darstellung jenes Begriffs möglich macht. Dieweil aber diesem dynamischen Begriffe noch ein bloß formaler, in der reinen Mathematik (z.B. der Geometrie) zum Grunde liegt: so hat die Vernunft dafür gesorgt, den Verstand auch mit Anschauungen a priori, zum Behuf der Konstruktion des Rechtsbegriffs, so viel möglich zu versorgen. – Das Rechte (rectum) wird, als das Gerade, teils dem Krummen, teils dem Schiefen entgegen gesetzt. Das erste ist die innere Beschaffenheit einer Linie von der Art, daß es zwischen zweien gegebenen Punkten nur eine einzige, das zweite aber die \match{Lage} zweier einander durchschneidenden oder zusammenstoßenden Linien, von deren Art es auch nur eine einzige (die senkrechte) geben kann, die sich nicht mehr nach einer Seite, als der andern hinneigt, und die den Raum von beiden Seiten gleich abteilt, nach welcher Analogie auch die Rechtslehre das Seine einem jeden (mit mathematischer Genauigkeit) bestimmt wissen will, welches in der Tugendlehre nicht erwartet werden darf, als welche einen gewissen Raum zu Ausnahmen (latitudinem) nicht verweigern kann. – Aber, ohne ins Gebiet der Ethik einzugreifen, gibt es zwei Fälle, die auf Rechtsentscheidung  Anspruch machen, für die aber keiner, der sie entscheide, ausgefunden werden kann, und die gleichsam in Epikurs Intermundia hingehören. – Diese müssen wir zuvörderst aus der eigentlichen Rechtslehre, zu der wir bald schreiten wollen, aussondern, damit ihre schwankenden Prinzipien nicht auf die festen Grundsätze der erstern Einfluß bekommen. 
	
	\unnumberedsection{Land (4)} 
	\subsection*{tg441.2.85} 
	\textbf{Source : }Die Metaphysik der Sitten/Erster Teil. Metaphysische Anfangsgründe der Rechtslehre/2. Teil. Das öffentliche Recht/1. Abschnitt. Das Staatsrecht\\  
	
	\noindent\textbf{Paragraphe : }Das \match{Land} (territorium), dessen Einsassen schon durch die Konstitution, d.i. ohne einen besonderen rechtlichen Akt ausüben zu dürfen (mithin durch die Geburt), Mitbürger eines und desselben gemeinen Wesens sind, heißt das Vaterland; das, worin sie es ohne diese Bedingung nicht sind, das Ausland, und dieses, wenn es einen Teil der Landesherrschaft überhaupt ausmacht, heißt die Provinz (in der Bedeutung, wie die Römer dieses Wort brauchten), welche, weil sie doch keinen koalisierten Teil des Reichs (imperii) als Sitz von Mitbürgern, sondern nur eine Besitzung desselben, als eines Unterhauses ausmacht, den Boden des herrschenden Staats als Mutterland (regio domina) verehren muß. 
	
	\subsection*{tg442.2.14} 
	\textbf{Source : }Die Metaphysik der Sitten/Erster Teil. Metaphysische Anfangsgründe der Rechtslehre/2. Teil. Das öffentliche Recht/2. Abschnitt. Das Völkerrecht\\  
	
	\noindent\textbf{Paragraphe : }Es gibt mancherlei Naturprodukte in einem Lande, die doch, was die Menge derselben von einer gewissen Art betrifft, zugleich als Gemächsel (artefacta) des Staats angesehen werden müssen, weil das \match{Land} sie in solcher Menge nicht liefern würde, wenn es nicht einen Staat und eine ordentliche machthabende Regierung gäbe, sondern die Bewohner im Stande der Natur wären. – Haushühner (die nützlichste Art des Geflügels), Schafe, Schweine, das Rindergeschlecht u.a.m. würden, entweder aus Mangel an Futter, oder der Raubtiere wegen, in dem Lande, wo ich lebe, entweder gar nicht, oder höchst sparsam anzutreffen sein, wenn es darin nicht eine Regierung gäbe, welche den Einwohnern ihren Erwerb und Besitz sicherte. – Eben das gilt auch von der Menschenzahl, die, eben so wie in den amerikanischen Wüsten, ja selbst dann, wenn man diesen den größten Fleiß (den jene nicht haben) beilegte, nur gering sein kann. Die Einwohner würden nur sehr dünn gesäet sein, weil keiner derselben sich, mit samt seinem Gesinde, auf einem Boden weit verbreiten könnte, der immer in Gefahr ist, von Menschen oder Wilden und Raubtieren verwüstet zu werden; mithin sich für eine so große Menge von Menschen, als jetzt auf einem Lande leben, kein hinlänglicher Unterhalt finden würde. – – Sowie man nun von Gewächsen (z.B. den Kartoffeln) und von Haustieren, weil sie, was die Menge betrifft, ein Machwerk der Menschen sind, sagen kann, daß man sie gebrauchen, verbrauchen und verzehren (töten lassen) kann: so, scheint es, könne man auch von der obersten Gewalt im Staat, dem Souverän, sagen, er  habe das Recht, seine Untertanen, die dem größten Teil nach sein eigenes Produkt sind, in den Krieg, wie auf eine Jagd, und zu einer Feldschlacht, wie auf eine Lustpartie zu führen. 
	
	\subsection*{tg442.2.42} 
	\textbf{Source : }Die Metaphysik der Sitten/Erster Teil. Metaphysische Anfangsgründe der Rechtslehre/2. Teil. Das öffentliche Recht/2. Abschnitt. Das Völkerrecht\\  
	
	\noindent\textbf{Paragraphe : }Das Recht eines Staats gegen einen ungerechten Feind hat keine Grenzen (wohl zwar der Qualität, aber nicht der Quantität, d.i. dem Grade nach): d.i. der beeinträchtigte Staat darf sich zwar nicht aller Mittel, aber doch der an sich zulässigen in dem Maße bedienen, um das Seine zu behaupten, als er dazu Kräfte hat. – Was ist aber nun nach Begriffen des Völkerrechts, in welchem, wie überhaupt im Naturzustande, ein jeder Staat in seiner eigenen Sache Richter ist, ein ungerechter Feind? Es ist derjenige, dessen öffentlich (es sei wörtlich oder tätlich) geäußerter Wille eine Maxime verrät, nach welcher, wenn sie zur allgemeinen Regel gemacht würde, kein Friedenszustand unter Völkern möglich, sondern der Naturzustand verewigt werden müßte. Dergleichen ist die Verletzung öffentlicher Verträge, von welcher man voraussetzen kann, daß sie die Sache aller Völker betrifft, deren Freiheit dadurch bedroht wird, und die dadurch aufgefordert werden, sich gegen einen solchen Unfug zu vereinigen und ihm die Macht dazu zu nehmen; – aber doch auch nicht, um sich in sein \match{Land} zu teilen, einen Staat gleichsam auf der Erde verschwinden zu machen, denn das wäre Ungerechtigkeit gegen das Volk, welches sein ursprüngliches Recht, sich in ein gemeines Wesen zu verbinden, nicht verlieren kann, sondern es eine neue Verfassung annehmen zu lassen, die, ihrer Natur nach, der Neigung zum Kriege ungünstig ist. 
	
	\subsection*{tg488.2.16} 
	\textbf{Source : }Die Metaphysik der Sitten/Zweiter Teil. Metaphysische Anfangsgründe der Tugendlehre/Beschluß. Die Religionslehre als Lehre der Pflichten gegen Gott liegt außerhalb den Grenzen der reinen Moralphilosophie\\  
	
	\noindent\textbf{Paragraphe : }Die Strafe läßt (nach dem Horaz) den vor ihr stolz schreitenden Verbrecher nicht aus den Augen, sondern hinkt ihm unablässig nach, bis sie ihn ertappt. – Das unschuldig vergossene Blut schreit um Rache. – Das Verbrechen kann nicht ungerächt bleiben; trifft die Strafe nicht den Verbrecher, so werden es seine Nachkommen entgelten müssen; oder geschieht's nicht bei seinem Leben, so muß es in einem Leben nach dem Tode
	
	
	25
	geschehen, welches ausdrücklich darum auch angenommen und gern geglaubt wird, damit der Anspruch der ewigen Gerechtigkeit ausgeglichen werde. – Ich will keine Blutschuld auf mein \match{Land} kommen lassen, dadurch, daß ich einen boshaft mordenden Duellanten, für den ihr Fürbitte tut, begnadige, sagte einmal ein wohldenkender Landesherr. – Die Sündenschuld muß bezahlt werden, und sollte sich auch ein völlig Unschuldiger zum Sühnopfer  hingeben (wo dann freilich die von ihm übernommene Leiden eigentlich nicht Strafe – denn er hat selbst nichts verbrochen – heißen könnten); aus welchen allen zu ersehen ist, daß es nicht eine die Gerechtigkeit verwaltende Person ist, der man diesen Verurteilungsspruch beilegt (denn die würde nicht so sprechen können, ohne anderen unrecht zu tun), sondern daß die bloße Gerechtigkeit, als überschwengliches, einem übersinnlichen Subjekt angedachtes Prinzip, das Recht dieses Wesens bestimme; welches zwar dem Formalen dieses Prinzips gemäß ist, dem Materialen desselben aber, dem Zweck, welcher immer die Glückseligkeit der Menschen ist, widerstreitet. – Denn, bei der etwanigen großen Menge der Verbrecher, die ihr Schuldenregister immer so fortlaufen lassen, würde die Strafgerechtigkeit den Zweck der Schöpfung nicht in der Liebe des Welturhebers (wie man sich doch denken muß), sondern in der strengen Befolgung des Rechts setzen (das Recht selbst zum Zweck machen, der in der Ehre Gottes gesetzt wird), welches, da das letztere (die Gerechtigkeit) nur die einschränkende Bedingung des ersteren (der Gütigkeit) ist, den Prinzipien der praktischen Vernunft zu widersprechen scheint, nach welchen eine Weltschöpfung hätte unterbleiben müssen, die ein, der Absicht ihres Urhebers, die nur Liebe zum Grunde haben kann, so widerstreitendes Produkt geliefert haben würde. 
	
	\unnumberedsection{Lange (2)} 
	\subsection*{tg438.2.10} 
	\textbf{Source : }Die Metaphysik der Sitten/Erster Teil. Metaphysische Anfangsgründe der Rechtslehre/1. Teil. Das Privatrecht vom äußeren Mein und Dein überhaupt/2. Hauptstück. Von der Art, etwas Äußeres zu erwerben/Episodischer Abschnitt. Von der idealen Erwerbung eines äußeren Gegenstandes der Willkür\\  
	
	\noindent\textbf{Paragraphe : }Denn setzet: die Versäumung dieses Besitzakts hätte nicht die Folge, daß ein anderer auf seinen gesetzmäßigen und ehrlichen Besitz (possessio bonae fidei) einen zu Recht beständigen (possessio irrefragabilis) gründe, und die Sache, die in seinem Besitz ist, als von ihm erworben ansehe, so würde gar keine Erwerbung peremtorisch (gesichert), sondern alle nur provisorisch (einstweilig) sein; weil die Geschichtskunde ihre Nachforschung bis zum ersten Besitzer und dessen Erwerbakt hinauf zurückzuführen nicht vermögend  ist. – Die Präsumtion, auf welcher sich die Ersitzung (usucapio) gründet, ist also nicht bloß rechtmäßig (erlaubt, iusta) als Vermutung, sondern auch rechtlich (praesumtio iuris et de iure) als Voraussetzung nach Zwangsgesetzen (suppositio legalis): wer seinen Besitzakt zu dokumentieren verabsäumt, hat seinen Anspruch auf den dermaligen Besitzer verloren, wobei die \match{Länge} der Zeit der Verabsäumung (die gar nicht bestimmt werden kann und darf) nur zum Behuf der Gewißheit dieser Unterlassung angeführt wird. Daß aber ein bisher unbekannter Besitzer, wenn jener Besitzakt (es sei auch ohne seine Schuld) unterbrochen worden, die Sache immer wiedererlangen (vindizieren) könne (dominia rerum incerta facere), widerspricht dem obigen Postulat der rechtlich-praktischen Vernunft. 
	
	\subsection*{tg445.2.37} 
	\textbf{Source : }Die Metaphysik der Sitten/Erster Teil. Metaphysische Anfangsgründe der Rechtslehre/Anhang erläutender Bemerkungen zu den metaphysischen Anhangsgründen der Rechtslehre\\  
	
	\noindent\textbf{Paragraphe : }Denn die Frage ist hier: wer soll seine rechtmäßige Erwerbung beweisen? Dem Besitzer kann diese Verbindlichkeit (onus probandi) nicht aufgebürdet werden; denn er ist, so weit wie seine konstatierte Geschichte reicht, im Besitz derselben. Der frühere angebliche Eigentümer der Sache ist durch eine Zwischenzeit, innerhalb deren er keine bürgerlich gültige Zeichen seines Eigentums gab, von der Reihe der auf einander folgenden Besitzer nach Rechtsprinzipien ganz abgeschnitten. Diese Unterlassung irgend eines öffentlichen Besitzakts macht ihn zu einem unbetitelten Prätendenten. (Dagegen heißt es hier, wie bei der Theologie, conservatio est continua creatio.) Wenn sich auch ein bisher nicht manifestierter, obzwar hinten nach mit aufgefundenen Dokumenten versehener Prätendent vorfände, so würde doch wiederum auch bei diesem der Zweifel vorwalten, ob nicht ein noch älterer Prätendent dereinst auftreten, und seine Ansprüche auf den früheren Besitz gründen könnte. – Auf die \match{Länge} der Zeit des Besitzes kommt es hiebei gar nicht an, um die Sache endlich zu ersitzen (acquirere per usucapionem). Denn es ist ungereimt, anzunehmen, daß ein Unrecht dadurch, daß es lange gewährt hat, nach gerade ein Recht werde. Der (noch so lange) Gebrauch setzt das Recht in der Sache voraus: weit gefehlt, daß dieses sich auf jenen gründen sollte. Also ist die Ersitzung (usucapio) als Erwerbung durch den langen Gebrauch einer Sache ein sich selbst widersprechender Begriff. Die Verjährung der Ansprüche als Erhaltungsart (conservatio possessionis meae per praescriptionem) ist es nicht weniger: indessen doch ein von dem vorigen unterschiedener Begriff, was das Argument der Zueignung betrifft. Es ist nämlich ein negativer Grund, d.i. der gänzliche Nichtgebrauch seines Rechts, selbst nicht einmal der, welcher nötig ist, um sich als Besitzer zu manifestieren, für eine Verzichttuung auf dieselbe (derelictio), welche ein rechtlicher Akt, d.i. Gebrauch seines Rechts gegen einen  anderen ist, um durch Ausschließung desselben vom Anspruche (per praescriptionem) das Objekt desselben zu erwerben, welches einen Widerspruch enthält. 
	
	\unnumberedsection{Lauf (1)} 
	\subsection*{tg486.2.31} 
	\textbf{Source : }Die Metaphysik der Sitten/Zweiter Teil. Metaphysische Anfangsgründe der Tugendlehre/II. Ethische Methodenlehre/1. Abschnitt. Die ethische Didaktik\\  
	
	\noindent\textbf{Paragraphe : }7. L. Wenn wir uns aber auch eines solchen guten und tätigen Willens, durch den wir uns würdig (wenigstens nicht unwürdig) halten, glücklich zu sein, auch bewußt sind, können wir darauf auch die sichere Hoffnung gründen, dieser Glückseligkeit teilhaftig zu werden? S. Nein! darauf allein nicht; denn es steht nicht immer in unserem Vermögen, sie uns zu verschaffen, und der \match{Lauf} der Natur richtet sich auch nicht so von selbst nach dem Verdienst, sondern das Glück des Lebens (unsere Wohlfahrt überhaupt) hängt von Umständen ab, die bei weitem nicht alle in des Menschen Gewalt sind. Also bleibt unsere Glückseligkeit immer nur ein Wunsch, ohne daß, wenn nicht irgend eine andere Macht hinzukommt, dieser jemals Hoffnung werden kann. 
	
	\unnumberedsection{Licht (2)} 
	\subsection*{tg469.2.5} 
	\textbf{Source : }Die Metaphysik der Sitten/Zweiter Teil. Metaphysische Anfangsgründe der Tugendlehre/I. Ethische Elementarlehre/I. Teil. Von den Pflichten gegen sich selbst überhaupt/Einleitung\\  
	
	\noindent\textbf{Paragraphe : }Wenn das verpflichtende Ich mit dem verpflichteten in einerlei Sinn genommen wird, so ist Pflicht gegen sich selbst ein sich widersprechender Begriff. Denn in dem Begriffe der Pflicht ist der einer passiven Nötigung enthalten (ich werde verbunden). Darin aber, daß es eine Pflicht gegen mich selbst ist, stelle ich mich als verbindend, mithin in einer aktiven Nötigung vor (Ich, eben dasselbe Subjekt, bin der Verbindende); und der Satz, der eine Pflicht gegen sich selbst ausspricht (ich soll mich selbst verbinden), würde eine Verbindlichkeit verbunden zu sein (passive Obligation, die doch zugleich, in demselben Sinne des Verhältnisses, eine aktive wäre), mithin einen Widerspruch enthalten. – Man kann diesen Widerspruch auch dadurch ins \match{Licht} stellen: daß man zeigt, der Verbindende (auctor obligationis) könne den Verbundenen (subiectum obligationis) jederzeit von der Verbindlichkeit (terminus obligationis) lossprechen; mithin (wenn beide ein und dasselbe Subjekt sind), er sei an eine Pflicht, die er sich auferlegt, gar nicht gebunden: welches einen Widerspruch enthält. 
	
	\subsection*{tg481.2.86} 
	\textbf{Source : }Die Metaphysik der Sitten/Zweiter Teil. Metaphysische Anfangsgründe der Tugendlehre/I. Ethische Elementarlehre/II. Teil. Von den Tugendpflichten gegen andere/Erstes Hauptstück. Von den Pflichten gegen andere, bloß als Menschen/Erster Abschnitt. Von der Liebespflicht gegen andere Menschen\\  
	
	\noindent\textbf{Paragraphe : }c) Die Schadenfreude, welche das gerade Umgekehrte der Teilnehmung ist, ist der menschlichen Natur auch nicht fremd; wiewohl, wenn sie so weit geht, das Übel oder Böses selbst bewirken zu helfen, sie als qualifizierte Schadenfreude den Menschenhaß sichtbar macht und in ihrer Gräßlichkeit erscheint. Sein Wohlsein und selbst sein Wohlverhalten stärker zu fühlen, wenn Unglück, oder Verfall  anderer in Skandale, gleichsam als die Folie unserem eigenen Wohlstande untergelegt wird, um diesen in ein desto helleres \match{Licht} zu stellen, ist freilich nach Gesetzen der Einbildungskraft, nämlich des Kontrastes, in der Natur gegründet. Aber über die Existenz solcher das allgemeine Weltbeste zerstörenden Enormitäten unmittelbar sich zu freuen, mithin dergleichen Eräugnisse auch wohl zu wünschen, ist ein geheimer Menschenhaß und das gerade Widerspiel der Nächstenliebe, die uns als Pflicht obliegt. – Der Übermut anderer bei ununterbrochenem Wohlergehn und der Eigendünkel im Wohlverhalten (eigentlich aber nur im Glück, der Verleitung zum öffentlichen Laster noch immer entwischt zu sein), welches beides der eigenliebige Mensch sich zum Verdienst anrechnet, bringen diese feindselige Freude hervor, die der Pflicht nach dem Prinzip der Teilnehmung (des ehrlichen Chremes beim Terenz) »ich bin ein Mensch; alles, was Men schen widerfährt, das trifft auch mich« gerade entgegengesetzt ist. 
	
	\unnumberedsection{Luft (1)} 
	\subsection*{tg484.2.12} 
	\textbf{Source : }Die Metaphysik der Sitten/Zweiter Teil. Metaphysische Anfangsgründe der Tugendlehre/I. Ethische Elementarlehre/II. Teil. Von den Tugendpflichten gegen andere/Beschluß der Elementarlehre. Von der innigsten Vereinigung der Liebe mit der Achtung in der Freundschaft\\  
	
	\noindent\textbf{Paragraphe : }Findet er also einen, der Verstand hat, bei dem er in Ansehung jener Gefahr gar nicht besorgt sein darf, sondern  dem er sich mit völligem Vertrauen eröffnen kann, der überdem auch eine mit der seinigen übereinstimmende Art, die Dinge zu beurteilen, an sich hat, so kann er seinen Gedanken \match{Luft} machen; er ist mit seinen Gedanken nicht völlig allein, wie im Gefängnis, und genießt eine Freiheit, der er in dem großen Haufen entbehrt, wo er sich in sich selbst verschließen muß. Ein jeder Mensch hat Geheimnisse und darf sich nicht blindlings anderen anvertrauen; teils wegen der unedlen Denkungsart der meisten, davon einen ihm nachteiligen Gebrauch zu machen, teils wegen des Unverstandes mancher in der Beurteilung und Unterscheidung dessen, was sich nachsagen läßt, oder nicht (der Indiskretion), welche Eigenschaften zusammen in einem Subjekt anzutreffen selten ist (rara avis in terris, et nigro simillima cygno); zumal da die engeste Freundschaft es verlangt, daß dieser verständige und vertraute Freund zugleich verbunden ist, ebendasselbe ihm anvertraute Geheimnis einem anderen, für eben so zuverlässig gehaltenen, ohne des ersteren ausdrückliche Erlaubnis nicht mitzuteilen. 
	
	\unnumberedsection{Mantel (1)} 
	\subsection*{tg439.2.16} 
	\textbf{Source : }Die Metaphysik der Sitten/Erster Teil. Metaphysische Anfangsgründe der Rechtslehre/1. Teil. Das Privatrecht vom äußeren Mein und Dein überhaupt/3. Hauptstück. Von der subjektiv-bedingten Erwerbung durch den Ausspruch einer öffentlichen Gerichtsbarkeit\\  
	
	\noindent\textbf{Paragraphe : }Wenn ich, z.B. bei einfallendem Regen, in ein Haus eintrete, und erbitte mir einen \match{Mantel} zu leihen, der aber, etwa durch unvorsichtige Ausgießung abfärbender Materien aus dem Fenster, auf immer verdorben, oder, wenn er, indem ich ihn in einem anderen Hause, wo ich eintrete, ablege, mir gestohlen wird, so muß doch die Behauptung jedem Menschen als ungereimt auffallen, ich hätte nichts weiter zu tun, als jenen, so wie er ist, zurückzuschicken, oder den geschehenen Diebstahl nur zu melden; allenfalls sei es noch eine Höflichkeit, den Eigentümer dieses Verlustes wegen zu beklagen, da er aus seinem Recht nichts fordern könne. – Ganz anders lautet es, wenn ich bei der Erbittung dieses Gebrauchs zugleich auf den Fall, daß die Sache unter meinen Händen verunglückte, mir zum voraus erbäte, auch diese Gefahr zu übernehmen, weil ich arm und den Verlust zu ersetzen unvermögend wäre. Niemand wird das letztere überflüssig und lächerlich finden, außer etwa, wenn der Anleihende ein bekanntlich vermögender und wohldenkender Mann wäre, weil es alsdann beinahe Beleidigung sein würde, die großmütige Erfassung meiner Schuld in diesem Falle nicht zu präsumieren. 
	
	\unnumberedsection{Materie (19)} 
	\subsection*{tg430.2.36} 
	\textbf{Source : }Die Metaphysik der Sitten/Erster Teil. Metaphysische Anfangsgründe der Rechtslehre/Einleitung in die Metaphysik der Sitten\\  
	
	\noindent\textbf{Paragraphe : }
	Pflicht ist diejenige Handlung, zu welcher jemand verbunden ist. Sie ist also die \match{Materie} der Verbindlichkeit, und es kann einerlei Pflicht (der Handlung nach) sein, ob wir zwar auf verschiedene Art dazu verbunden werden können. 
	
	\subsection*{tg430.2.9} 
	\textbf{Source : }Die Metaphysik der Sitten/Erster Teil. Metaphysische Anfangsgründe der Rechtslehre/Einleitung in die Metaphysik der Sitten\\  
	
	\noindent\textbf{Paragraphe : }Unter dem Willen kann die Willkür, aber auch der bloße Wunsch enthalten sein, sofern die Vernunft das Begehrungsvermögen überhaupt bestimmen kann; die Willkür, die durch reine Vernunft bestimmt werden kann, heißt  die freie Willkür. Die, welche nur durch Neigung (sinnlichen Antrieb, stimulus) bestimmbar ist, würde tierische Willkür (arbitrium brutum) sein. Die menschliche Willkür ist dagegen eine solche, welche durch Antriebe zwar affiziert, aber nicht bestimmt wird, und ist also für sich (ohne erworbene Fertigkeit der Vernunft) nicht rein, kann aber doch zu Handlungen aus reinem Willen bestimmt werden. Die Freiheit der Willkür ist jene Unabhängigkeit ihrer Bestimmung durch sinnliche Antriebe; dies ist der negative Begriff derselben. Der positive ist: das Vermögen der reinen Vernunft, für sich selbst praktisch zu sein. Dieses ist aber nicht anders möglich, als durch die Unterwerfung der Maxime einer jeden Handlung unter die Bedingung der Tauglichkeit der erstern zum allgemeinen Gesetze. Denn, als reine Vernunft, auf die Willkür, unangesehen dieser ihres Objekts, angewandt, kann sie, als Vermögen der Prinzipien (und hier praktischer Prinzipien, mithin als gesetzgebendes Vermögen), da ihr die \match{Materie} des Gesetzes abgeht, nichts mehr, als die Form der Tauglichkeit der Maxime der Willkür zum allgemeinen Gesetze selbst zum obersten Gesetze und Bestimmungsgrunde der Willkür machen, und, da die Maximen des Menschen aus subjektiven Ursachen mit jenen objektiven nicht von selbst übereinstimmen, dieses Gesetz nur schlechthin, als Imperativ des Verbots oder Gebots, vorschreiben. 
	
	\subsection*{tg433.2.10} 
	\textbf{Source : }Die Metaphysik der Sitten/Erster Teil. Metaphysische Anfangsgründe der Rechtslehre/1. Teil. Das Privatrecht vom äußeren Mein und Dein überhaupt/1. Hauptstück\\  
	
	\noindent\textbf{Paragraphe : }Denn ein Gegenstand meiner Willkür ist etwas, was zu gebrauchen ich physisch in meiner Macht habe. Sollte es nun doch rechtlich schlechterdings nicht in meiner Macht stehen, d.i. mit der Freiheit von jedermann nach einem allgemeinen Gesetz nicht zusammen bestehen können (unrecht sein), Gebrauch von demselben zu machen: so würde die Freiheit sich selbst des Gebrauchs ihrer Willkür in Ansehung eines Gegenstandes derselben berauben, dadurch, daß sie brauchbare Gegenstände außer aller Möglichkeit des Gebrauchs setzte: d.i. diese in praktischer Rücksicht vernichtete, und zur res nullius machte; obgleich die Willkür, formaliter, im Gebrauch der Sachen mit jedermanns äußeren Freiheit nach allgemeinen Gesetzen zusammenstimmete. – Da nun die reine praktische Vernunft keine andere als formale Gesetze des Gebrauchs der Willkür zum Gründe legt, und also von der \match{Materie} der Willkür, d.i. der übrigen Beschaffenheit des Objekts, wenn es nur ein Gegenstand der Willkür ist, abstrahiert, so kann sie in Ansehung eines solchen Gegenstandes kein absolutes Verbot seines Gebrauchs enthalten, weil dieses ein Widerspruch der äußeren Freiheit mit sich selbst sein würde. – Ein Gegenstand meiner Willkür aber ist das, wovon beliebigen Gebrauch zu machen ich das physische Vermögen habe, dessen Gebrauch in meiner Macht (potentia) steht: wovon noch unterschieden werden muß, denselben Gegenstand in meiner Gewalt (in potestatem meam redactum) zu haben, welches nicht bloß ein Vermögen, sondern auch einen Akt der Willkür voraus setzt. Um aber etwas bloß als Gegenstand meiner Willkür zu denken, ist hinreichend, mir bewußt zu sein, daß ich ihn in meiner Macht habe. – Also ist es eine Voraussetzung a priori der praktischen Vernunft, einen jeden  Gegenstand meiner Willkür als objektiv-mögliches Mein oder Dein anzusehen und zu behandeln. 
	
	\subsection*{tg434.2.12} 
	\textbf{Source : }Die Metaphysik der Sitten/Erster Teil. Metaphysische Anfangsgründe der Rechtslehre/1. Teil. Das Privatrecht vom äußeren Mein und Dein überhaupt\\  
	
	\noindent\textbf{Paragraphe : }1) Der \match{Materie} (dem Objekte) nach erwerbe ich entweder eine körperliche Sache (Substanz) oder die Leistung (Kausalität) eines anderen oder diese andere Person selbst, d.i. den Zustand derselben, so fern ich ein Recht erlange, über denselben zu verfügen (das Commercium mit derselben). 
	
	\subsection*{tg437.2.40} 
	\textbf{Source : }Die Metaphysik der Sitten/Erster Teil. Metaphysische Anfangsgründe der Rechtslehre/1. Teil. Das Privatrecht vom äußeren Mein und Dein überhaupt/2. Hauptstück. Von der Art, etwas Äußeres zu erwerben/3. Abschnitt. Von dem auf dingliche Art persönlichen Recht\\  
	
	\noindent\textbf{Paragraphe : }Das Gesinde gehört nun zu dem Seinen des Hausherrn, und zwar, was die Form (den Besitzstand) betrifft, gleich als nach einem Sachenrecht; denn der Hausherr kann, wenn es ihm entläuft, es durch einseitige Willkür in seine Gewalt bringen; was aber die \match{Materie} betrifft, d.i. welchen Gebrauch er von diesen seinen Hausgenossen machen kann, so kann er sich nie als Eigentümer desselben (dominus servi) betragen: weil er nur durch Vertrag unter seine Gewalt gebracht ist, ein Vertrag aber, durch den ein Teil zum Vorteil des anderen auf seine ganze Freiheit Verzicht tut, mithin aufhört, eine Person zu sein, folglich auch keine Pflicht hat, einen Vertrag zu halten, sondern nur Gewalt  anerkennt, in sich selbst widersprechend, d.i. null und nichtig ist. (Von dem Eigentumsrecht gegen den, der sich durch ein Verbrechen seiner Persönlichkeit verlustig gemacht hat, ist hier nicht die Rede.) 
	
	\subsection*{tg437.2.73} 
	\textbf{Source : }Die Metaphysik der Sitten/Erster Teil. Metaphysische Anfangsgründe der Rechtslehre/1. Teil. Das Privatrecht vom äußeren Mein und Dein überhaupt/2. Hauptstück. Von der Art, etwas Äußeres zu erwerben/3. Abschnitt. Von dem auf dingliche Art persönlichen Recht\\  
	
	\noindent\textbf{Paragraphe : }In dieser Tafel aller Arten der Übertragung (translatio) des Seinen auf einen anderen finden sich Begriffe von Objekten, oder Werkzeugen dieser Übertragung vor, welche ganz empirisch zu sein, und selbst ihrer Möglichkeit nach, in einer  metaphysischen Rechtslehre, eigentlich nicht Platz haben, in der die Einteilungen nach Prinzipien a priori gemacht werden müssen, mithin von der \match{Materie} des Verkehrs (welche konventionell sein könnte) abstrahiert, und bloß auf die Form gesehen werden muß, dergleichen der Begriff des Geldes, im Gegensatz mit aller anderen veräußerlichen Sache, nämlich der Ware, im Titel des Kaufs und Verkaufs, oder der eines Buchs ist. – Allein es wird sich zeigen, daß jener Begriff des größten und brauchbarsten aller Mittel des Verkehrs der Menschen mit Sachen, Kauf und Verkauf (Handel) genannt, imgleichen der eines Buchs, als das des größten Verkehrs der Gedanken, sich doch in lauter intellektuelle Verhältnisse auflösen lasse, und so die Tafel der reinen Verträge nicht durch empirische Beimischung verunreinigen dürfe. 
	
	\subsection*{tg437.2.80} 
	\textbf{Source : }Die Metaphysik der Sitten/Erster Teil. Metaphysische Anfangsgründe der Rechtslehre/1. Teil. Das Privatrecht vom äußeren Mein und Dein überhaupt/2. Hauptstück. Von der Art, etwas Äußeres zu erwerben/3. Abschnitt. Von dem auf dingliche Art persönlichen Recht\\  
	
	\noindent\textbf{Paragraphe : }Wie ist es aber möglich, daß das, was anfänglich Ware war, endlich Geld ward? Wenn ein großer und machthabender Vertuer einer Materie, die er anfangs bloß zum Schmuck und Glanz seiner Diener (des Hofes) brauchte (z.B. Gold, Silber, Kupfer, oder eine Art schöner Muschelschalen, Kauris, oder auch, wie in Kongo, eine Art Matten, Makuten genannt, oder, wie am Senegal, Eisenstangen, und auf der Guineaküste selbst Negersklaven), d.i. wenn ein Landesherr die Abgaben von seinen Untertanen in dieser \match{Materie} (als Ware) einfordert, und die, deren Fleiß in Anschaffung derselben  dadurch bewegt werden soll, mit eben denselben, nach Verordnungen des Verkehrs unter und mit ihnen überhaupt (auf einem Markt, oder einer Börse), wieder lohnt. – Dadurch allein hat (meinem Bedünken nach) eine Ware ein gesetzliches Mittel des Verkehrs des Fleißes der Untertanen unter einander und hiemit auch des Staatsreichtums, d.i. Geld, werden können. 
	
	\subsection*{tg437.2.82} 
	\textbf{Source : }Die Metaphysik der Sitten/Erster Teil. Metaphysische Anfangsgründe der Rechtslehre/1. Teil. Das Privatrecht vom äußeren Mein und Dein überhaupt/2. Hauptstück. Von der Art, etwas Äußeres zu erwerben/3. Abschnitt. Von dem auf dingliche Art persönlichen Recht\\  
	
	\noindent\textbf{Paragraphe : }»Geld ist also (nach Adam Smith) derjenige Körper, dessen Veräußerung das Mittel und zugleich der Maßstab des Fleißes ist, mit welchem Menschen und Völker unter einander Verkehr treiben.« – Diese Erklärung führt den empirischen Begriff des Geldes dadurch auf den intellektuellen hinaus, daß sie nur auf die Form der wechselseitigen Leistungen  im belästigten Vertrage sieht (und von dieser ihrer \match{Materie} abstrahiert), und so auf Rechtsbegriff in der Umsetzung des Mein und Dein (commutatio late sic dicta) überhaupt, um die obige Tafel einer dogmatischen Einteilung a priori, mithin der Metaphysik des Rechts, als eines Systems, angemessen vorzustellen. 
	
	\subsection*{tg439.2.44} 
	\textbf{Source : }Die Metaphysik der Sitten/Erster Teil. Metaphysische Anfangsgründe der Rechtslehre/1. Teil. Das Privatrecht vom äußeren Mein und Dein überhaupt/3. Hauptstück. Von der subjektiv-bedingten Erwerbung durch den Ausspruch einer öffentlichen Gerichtsbarkeit\\  
	
	\noindent\textbf{Paragraphe : }Der rechtliche Zustand ist dasjenige Verhältnis der Menschen unter einander, welches die Bedingungen enthält,  unter denen allein jeder seines Rechts teilhaftig werden kann, und das formale Prinzip der Möglichkeit desselben, nach der Idee eines allgemein gesetzgebenden Willens betrachtet, heißt die öffentliche Gerechtigkeit, welche in Beziehung, entweder auf die Möglichkeit, oder Wirklichkeit, oder Notwendigkeit des Besitzes der Gegenstände (als der \match{Materie} der Willkür) nach Gesetzen in die beschützende (iustitia tutatrix), die wechselseitig erwerbende (iustitia commutativa) und die austeilende Gerechtigkeit (iustitia distributiva) eingeteilt werden kann. – Das Gesetz sagt hiebei erstens bloß, welches Verhalten innerlich der Form nach recht ist (lex iusti); zweitens, was als Materie noch auch äußerlich gesetzfähig, d.i. dessen Besitzstand rechtlich ist (lex iuridica); drittens, was und wovon der Ausspruch vor einem Gerichtshofe in einem besonderen Falle unter dem gegebenen Gesetze diesem gemäß, d.i. Rechtens ist (lex iustitiae), wo man denn auch jenen Gerichtshof selbst die Gerechtigkeit eines Landes nennt, und, ob eine solche sei oder nicht sei, als die wichtigste unter allen rechtlichen Angelegenheiten gefragt werden kann. 
	
	\subsection*{tg439.2.46} 
	\textbf{Source : }Die Metaphysik der Sitten/Erster Teil. Metaphysische Anfangsgründe der Rechtslehre/1. Teil. Das Privatrecht vom äußeren Mein und Dein überhaupt/3. Hauptstück. Von der subjektiv-bedingten Erwerbung durch den Ausspruch einer öffentlichen Gerichtsbarkeit\\  
	
	\noindent\textbf{Paragraphe : }Man kann den ersteren und zweiten Zustand den des Privatrechts, den letzteren und dritten aber den des 
	öffentlichen Rechts nennen. Dieses enthält nicht mehr, oder andere Pflichten der Menschen unter sich, als in jenem gedacht werden können; die \match{Materie} des Privatrechts ist eben dieselbe in beiden. Die Gesetze des letzteren betreffen also nur die rechtliche Form ihres Beisammenseins (Verfassung), in Ansehung deren diese Gesetze notwendig als öffentliche gedacht werden müssen. 
	
	\subsection*{tg447.2.4} 
	\textbf{Source : }Die Metaphysik der Sitten/Zweiter Teil. Metaphysische Anfangsgründe der Tugendlehre/Vorrede\\  
	
	\noindent\textbf{Paragraphe : }Wenn es über irgend einen Gegenstand eine Philosophie (System der Vernunfterkenntnis aus Begriffen) gibt, so muß es für diese Philosophie auch ein System reiner, von aller Anschauungsbedingung unabhängiger Vernunftbegriffe, d.i. eine Metaphysik geben. – Es fragt sich nur: ob es für jede praktische Philosophie, als Pflichtenlehre, mithin auch für die Tugendlehre (Ethik), auch metaphysischer Anfangsgründe bedürfe, um sie, als wahre Wissenschaft (systematisch), nicht bloß als Aggregat einzeln aufgesuchter Lehren (fragmentarisch) aufstellen zu können. – Von der reinen Rechtslehre wird niemand dies Bedürfnis bezweifeln; denn sie betrifft nur das Förmliche der nach Freiheitsgesetzen im äußeren Verhältnis einzuschränkenden Willkür; abgesehen von allem Zweck (als der \match{Materie} derselben). Die Pflichtenlehre ist also hier eine bloße Wissenslehre (doctrina scientiae).
	
	
	13
	
	
	
	\subsection*{tg447.2.7} 
	\textbf{Source : }Die Metaphysik der Sitten/Zweiter Teil. Metaphysische Anfangsgründe der Tugendlehre/Vorrede\\  
	
	\noindent\textbf{Paragraphe : }
	Geht man von diesem Grundsatze ab und fängt vom pathologischen, oder dem reinästhetischen, oder auch dem moralischen Gefühl (dem subjektivpraktischen statt des objektiven), d.i. von der \match{Materie} des Willens, dem Zweck, nicht von der Form desselben, d.i. dem Gesetz an, um von da aus die Pflichten zu bestimmen: so finden freilich keine metaphysischen Anfangsgründe der Tugendlehre statt – denn Gefühl, wodurch es auch immer erregt werden mag, ist jederzeit physisch. – Aber die Tugendlehre wird alsdenn auch in ihrer Quelle, einerlei ob in Schulen, oder Hörsälen, u.s.w., verderbt. Denn es ist nicht gleichviel, durch welche Triebfedern als Mittel man zu einer guten Absicht (der Befolgung aller Pflicht) hingeleitet werde. – – Es mag also den orakel– oder auch geniemäßig über Pflichtenlehre absprechenden vermeinten Weisheitslehrern Metaphysik noch so sehr anekeln: so ist es doch für die, welche sich dazu aufwerfen, unerläßliche Pflicht, selbst in der Tugendlehre zu jener ihren Grundsätzen zurückzugehen und auf ihren Bänken vorerst selbst die Schule zu machen. 
	
	\subsection*{tg449.2.6} 
	\textbf{Source : }Die Metaphysik der Sitten/Zweiter Teil. Metaphysische Anfangsgründe der Tugendlehre/Einleitung/I. Erörterung des Begriffs einer Tugendlehre\\  
	
	\noindent\textbf{Paragraphe : }Die Rechtslehre hatte es bloß mit der formalen Bedingung der äußeren Freiheit (durch die Zusammenstimmung mit sich selbst, wenn ihre Maxime zum allgemeinen Gesetz gemacht wurde), d.i. mit dem Recht zu tun. Die Ethik dagegen gibt noch eine \match{Materie} (einen Gegenstand der freien Willkür), einen Zweck der reinen Vernunft, der zugleich als objektiv-notwendiger Zweck, d.i. für den Menschen als Pflicht vorgestellt wird, an die Hand. – Denn, da die sinnlichen Neigungen zu Zwecken (als der Materie der Willkür) verleiten, die der Pflicht zuwider sein können, so  kann die gesetzgebende Vernunft ihrem Einfluß nicht anders wehren, als wiederum durch einen entgegengesetzten moralischen Zweck, der also von der Neigung unabhängig a priori gegeben sein muß. 
	
	\subsection*{tg454.2.2} 
	\textbf{Source : }Die Metaphysik der Sitten/Zweiter Teil. Metaphysische Anfangsgründe der Tugendlehre/Einleitung/VI. Die Ethik gibt nicht Gesetze für die Handlungen [...] sondern nur für die Maximen der Handlungen\\  
	
	\noindent\textbf{Paragraphe : }Der Pflichtbegriff steht unmittelbar in Beziehung auf ein Gesetz (wenn ich gleich noch von allem Zweck, als der \match{Materie} desselben, abstrahiere); wie denn das formale Prinzip der Pflicht im kategorischen Imperativ: »handle so, daß die Maxime deiner Handlung ein allgemeines Gesetz werden könne«, es schon anzeigt; nur daß in der Ethik dieses als das Gesetz deines eigenen Willens gedacht wird, nicht des Willens überhaupt, der auch der Wille anderer sein könnte: wo es alsdenn eine Rechtspflicht abgeben würde, die nicht in das Feld der Ethik gehört. – Die Maximen werden hier als solche subjektive Grundsätze angesehen, die sich zu einer allgemeinen Gesetzgebung bloß qualifizieren; welches nur ein negatives Prinzip (einem Gesetz überhaupt nicht zu widerstreiten) ist. – Wie kann es aber dann noch ein Gesetz für die Maxime der Handlungen geben? 
	
	\subsection*{tg454.2.3} 
	\textbf{Source : }Die Metaphysik der Sitten/Zweiter Teil. Metaphysische Anfangsgründe der Tugendlehre/Einleitung/VI. Die Ethik gibt nicht Gesetze für die Handlungen [...] sondern nur für die Maximen der Handlungen\\  
	
	\noindent\textbf{Paragraphe : }Der Begriff eines Zwecks, der zugleich Pflicht ist, welcher der Ethik eigentümlich zugehört, ist es allein, der ein Gesetz für die Maximen der Handlungen begründet, indem der subjektive Zweck (den jedermann hat) dem objektiven (den sich jedermann dazu machen soll) untergeordnet wird. Der Imperativ: »du sollst dir dieses oder jenes (z.B. die Glückseligkeit anderer) zum Zweck machen«, geht auf die \match{Materie} der Willkür (ein Objekt). Da nun keine freie Handlung möglich ist, ohne daß der Handelnde hiebei zugleich einen Zweck (als Materie der Willkür) beabsichtigte, so muß, wenn es einen Zweck gibt, der zugleich Pflicht ist, die Maxime der Handlungen, als Mittel zu Zwecken, nur die Bedingung der Qualifikation zu einer möglichen allgemeinen Gesetzgebung enthalten; wogegen der Zweck, der zugleich Pflicht ist, es zu einem Gesetz machen kann, eine solche Maxime zu haben, indessen daß für die Maxime selbst die bloße Möglichkeit, zu einer allgemeinen Gesetzgebung zusammen zu stimmen, schon genug ist. 
	
	\subsection*{tg457.2.4} 
	\textbf{Source : }Die Metaphysik der Sitten/Zweiter Teil. Metaphysische Anfangsgründe der Tugendlehre/Einleitung/IX. Was ist Tugendpflicht\\  
	
	\noindent\textbf{Paragraphe : }Aber, was zu tun Tugend ist, das ist darum noch nicht so fort eigentliche Tugendpflicht. Jenes kann bloß das Formale der Maximen betreffen, diese aber geht auf die  \match{Materie} derselben, nämlich auf einen Zweck, der zugleich als Pflicht gedacht wird. – Da aber die ethische Verbindlichkeit zu Zwecken, deren es mehrere geben kann, nur eine weite ist, weil sie da bloß ein Gesetz für die Maxime der Handlungen enthält und der Zweck die Materie (Objekt) der Willkür ist, so gibt es viele, nach Verschiedenheit des gesetzlichen Zwecks verschiedene, Pflichten, welche Tugendpflichten (officia honestatis) genannt werden; eben darum, weil sie bloß dem freien Selbstzwange, nicht dem anderer Menschen, unterworfen sind und die den Zweck bestimmen, der zugleich Pflicht ist. 
	
	\subsection*{tg466.2.2} 
	\textbf{Source : }Die Metaphysik der Sitten/Zweiter Teil. Metaphysische Anfangsgründe der Tugendlehre/Einleitung/XVIII.\\  
	
	\noindent\textbf{Paragraphe : }Die Einteilung, welche die praktische Vernunft zu Gründung eines Systems ihrer Begriffe in einer Ethik entwirft (die architektonische), kann nun nach zweierlei Prinzipien, einzeln oder zusammen verbunden, gemacht werden: das eine, welches das subjektive Verhältnis der Verpflichteten zu dem Verpflichtenden, der \match{Materie} nach, das andere, welches das objektive Verhältnis der ethischen Gesetze zu den Pflichten überhaupt in einem System der Form nach vorstellt. – Die erste Einteilung ist die der Wesen, in Beziehung auf welche eine ethische Verbindlichkeit gedacht werden kann; die zweite wäre die der Begriffe der reinen ethisch-praktischen Vernunft; welche zu jener ihren Pflichten gehören, die also zur Ethik, nur so fern sie Wissenschaft sein soll, also zu der methodischen Zusammensetzung aller Sätze, welche nach der ersteren aufgefunden worden, erforderlich sind. 
	
	\subsection*{tg469.2.17} 
	\textbf{Source : }Die Metaphysik der Sitten/Zweiter Teil. Metaphysische Anfangsgründe der Tugendlehre/I. Ethische Elementarlehre/I. Teil. Von den Pflichten gegen sich selbst überhaupt/Einleitung\\  
	
	\noindent\textbf{Paragraphe : }Die Einteilung kann nur in Ansehung des Objekts der Pflicht, nicht in Ansehung des sich verpflichtenden Subjekts, gemacht werden. Das verpflichtete so wohl als das verpflichtende Subjekt ist immer nur der Mensch, und wenn es uns, in theoretischer Rücksicht, gleich erlaubt ist, im Menschen Seele und Körper als Naturbeschaffenheiten des Menschen von einander zu unterscheiden, so ist es doch nicht erlaubt, sie als verschiedene den Menschen verpflichtende Substanzen zu denken, um zur Einteilung in Pflichten gegen den Körper und gegen die Seele berechtigt zu sein. – Wir sind, weder durch Erfahrung, noch durch Schlüsse der Vernunft, hinreichend darüber belehrt, ob der Mensch eine Seele (als in ihm wohnende, vom Körper unterschiedene und von diesem unabhängig zu denken vermögende, d.i. geistige Substanz) enthalte, oder ob nicht vielmehr das Leben eine Eigenschaft der \match{Materie} sein möge, und wenn es sich auch auf die erstere Art verhielte, so würde doch keine Pflicht des Men schen gegen einen Körper (als verpflichtendes Subjekt), ob er gleich der menschliche ist, denkbar sein. 
	
	\subsection*{tg489.2.4} 
	\textbf{Source : }Die Metaphysik der Sitten/Fußnoten\\  
	
	\noindent\textbf{Paragraphe : }
	
	2 Man kann Sinnlichkeit durch das Subjektive unserer Vorstellungen überhaupt erklären; denn der Verstand bezieht allererst die Vorstellungen auf ein Objekt, d.i. er allein denkt sich etwas vermittelst derselben. Nun kann das Subjektive unserer Vorstellung entweder von der Art sein, daß es auch auf ein Objekt zum Erkenntnis desselben (der Form oder \match{Materie} nach, da es im ersteren Falle reine Anschauung, im zweiten Empfindung heißt) bezogen werden kann. In diesem Fall ist die Sinnlichkeit, als Empfänglichkeit der gedachten Vorstellung, der Sinn: aber das Subjektive der Vorstellung kann gar kein Erkenntnisstück werden; weil es bloß die Beziehung derselben aufs Subjekt und nichts zur Erkenntnis des Objekts Brauchbares enthält, und alsdann heißt diese Empfänglichkeit der Vorstellung Gefühl; welches die Wirkung der Vorstellung (diese mag sinnlich oder intellektuell 
	sein) aufs Subjekt enthält und zur Sinnlichkeit gehört, obgleich die Vorstellung selbst zum Verstande oder der Vernunft gehören mag. 
	
	\unnumberedsection{Nachahmung (1)} 
	\subsection*{tg486.2.19} 
	\textbf{Source : }Die Metaphysik der Sitten/Zweiter Teil. Metaphysische Anfangsgründe der Tugendlehre/II. Ethische Methodenlehre/1. Abschnitt. Die ethische Didaktik\\  
	
	\noindent\textbf{Paragraphe : }Das experimentale (technische) Mittel der Bildung zur Tugend ist das gute Beispiel an dem Lehrer selbst (von exemplarischer Führung zu sein) und das warnende an andern; denn \match{Nachahmung} ist dem noch ungebildeten Menschen die erste Willensbestimmung zu Annehmung von Maximen, die er sich in der Folge macht. – Die Angewöhnung oder Abgewöhnung ist die Begründung einer beharrlichen Neigung ohne alle Maximen, durch die öftere Befriedigung derselben; und ist ein Mechanism der Sinnesart, statt eines Prinzips der Denkungsart (wobei das Verlernen in der Folge schwerer wird als das Erlernen). – Was aber die Kraft des Exempels (es sei zum Guten oder Bösen) betrifft, was sich dem Hange zur Nachahmung oder Warnung darbietet,
	
	
	22
	so kann das, was uns andere geben, keine Tugendmaxime begründen. Denn diese besteht gerade in der subjektiven Autonomie der praktischen Vernunft eines jeden Menschen, mithin, daß nicht anderer Menschenverhalten, sondern das Gesetz, uns zur Triebfeder dienen müsse. Daher wird der Erzieher seinem verunarteten Lehrling nicht sagen: Nimm ein Exempel an jenem guten (ordentlichen, fleißigen) Knaben! denn das wird jenem nur zur Ursache dienen, diesen zu hassen, weil er durch ihn in ein nachteiliges Licht gestellt wird. Das gute Exempel (der exemplarische Wandel) soll nicht als Muster, sondern nur zum Beweise der Tunlichkeit des Pflichtmäßigen dienen. Also nicht die Vergleichung mit irgend einem andern Menschen (wie er ist), sondern mit der Idee (der Menschheit) wie er sein soll, also mit dem Gesetz, muß dem Lehrer das nie fehlende Richtmaß seiner Erziehung an die Hand geben. 
	
	\unnumberedsection{Ort (2)} 
	\subsection*{tg431.2.8} 
	\textbf{Source : }Die Metaphysik der Sitten/Erster Teil. Metaphysische Anfangsgründe der Rechtslehre/Einleitung in die Rechtslehre\\  
	
	\noindent\textbf{Paragraphe : }Diese Frage möchte wohl den Rechtsgelehrten, wenn er nicht in Tautologie verfallen, oder, statt einer allgemeinen Auflösung, auf das, was in irgend einem Lande die Gesetze zu irgend einer Zeit wollen, verweisen will, eben so in Verlegenheit setzen, als die berufene Aufforderung: Was ist Wahrheit? den Logiker, Was Rechtens sei (quid sit iuris), d.i. was die Gesetze an einem gewissen \match{Ort} und zu einer gewissen Zeit sagen oder gesagt haben, kann er noch wohl angeben; aber, ob das, was sie wollten, auch recht sei, und das allgemeine Kriterium, woran man überhaupt Recht sowohl als Unrecht (iustum et iniustum) erkennen könne, bleibt ihm wohl verborgen, wenn er nicht eine Zeitlang jene empirischen Prinzipien verläßt, die Quellen jener Urteile in der bloßen Vernunft sucht (wiewohl ihm dazu jene Gesetze vortrefflich zum Leitfaden dienen können), um zu einer möglichen positiven Gesetzgebung die Grundlage zu errichten. Eine bloß empirische Rechtslehre ist (wie der hölzerne Kopf in Phädrus' Fabel) ein Kopf, der schön sein mag, nur schade! daß er kein Gehirn hat.  Der Begriff des Rechts, sofern er sich auf eine ihm korrespondierende Verbindlichkeit bezieht (d.i. der moralische Begriff derselben), betrifft erstlich nur das äußere und zwar praktische Verhältnis einer Person gegen eine andere, sofern ihre Handlungen als Facta aufeinander (unmittelbar, oder mittelbar) Einfluß haben können. Aber zweitens bedeutet er nicht das Verhältnis der Willkür auf den Wunsch (folglich auch auf das bloße Bedürfnis) des anderen, wie etwa in den Handlungen der Wohltätigkeit oder Hartherzigkeit, sondern lediglich auf die Willkür des anderen. Drittens in diesem wechselseitigen Verhältnis der Willkür kommt auch gar nicht die Materie der Willkür, d.i. der Zweck, den ein jeder mit dem Objekt, was er will, zur Absicht hat, in Betrachtung, z.B. es wird nicht gefragt, ob jemand bei der Ware, die er zu seinem eigenen Handel von mir kauft, auch seinen Vorteil finden möge, oder nicht, sondern nur nach der Form im Verhältnis der beiderseitigen Willkür, sofern sie bloß als frei betrachtet wird, und ob durch die Handlung eines von beiden sich mit der Freiheit des andern nach einem allgemeinen Gesetze zusammen vereinigen lasse. 
	
	\subsection*{tg489.2.27} 
	\textbf{Source : }Die Metaphysik der Sitten/Fußnoten\\  
	
	\noindent\textbf{Paragraphe : }
	
	12 In jeder Bestrafung liegt etwas das Ehrgefühl des Angeklagten (mit Recht) Kränkendes; weil sie einen bloßen einseitigen Zwang enthält und so an ihm die Würde eines Staatsbürgers, als eines solchen, in einem besonderen Fall wenigstens suspendiert ist: Da er einer äußeren Pflicht unterworfen wird, der er seiner seits keinen Widerstand entgegen setzen darf. Der Vornehme und Reiche, der auf den Beutel geklopft wird, fühlt mehr seine Erniedrigung, sich unter den Willen des geringeren Mannes beugen zu müssen, als den Geldverlust. Die Strafgerechtigkeit (iustitia punitiva), da nämlich das Argument der Strafbarkeit moralisch ist (quia peccatum est), muß hier von der Strafklugheit, da es bloß pragmatisch ist (ne peccetur) und sich auf Erfahrung von dem gründet, was am stärksten wirkt, Verbrechen abzuhalten, unterschieden werden, und hat in der Topik der Rechtsbegriffe einen ganz anderen Ort, locus iusti, nicht des Conducibilis, oder des Zuträglichen in gewisser Absicht noch auch den des bloßen Honesti, dessen \match{Ort} in der Ethik aufgesucht werden muß. 
	
	\unnumberedsection{Qülle (3)} 
	\subsection*{tg430.2.14} 
	\textbf{Source : }Die Metaphysik der Sitten/Erster Teil. Metaphysische Anfangsgründe der Rechtslehre/Einleitung in die Metaphysik der Sitten\\  
	
	\noindent\textbf{Paragraphe : }Allein mit den Sittengesetzen ist es anders bewandt. Nur sofern sie als a priori gegründet und notwendig eingesehen
	werden können, gelten sie als Gesetze, ja die Begriffe und Urteile über uns selbst und unser Tun und Lassen bedeuten gar nichts Sittliches, wenn sie das, was sich bloß von der Erfahrung lernen läßt, enthalten, und, wenn man sich etwa verleiten läßt, etwas aus der letztern \match{Quelle} zum moralischen Grundsatze zu machen, so gerät man in Gefahr der gröbsten und verderblichsten Irrtümer. 
	
	\subsection*{tg474.2.4} 
	\textbf{Source : }Die Metaphysik der Sitten/Zweiter Teil. Metaphysische Anfangsgründe der Tugendlehre/I. Ethische Elementarlehre/I. Teil. Von den Pflichten gegen sich selbst überhaupt/Erstes Buch. Von den vollkommenen Pflichten gegen sich selbst/Zweites Hauptstück. Die Pflicht des Menschen gegen sich selbst, bloß als einem moralischen Wesen/2. Abschnitt. Von dem ersten Gebot aller Pflichten gegen sich selbst\\  
	
	\noindent\textbf{Paragraphe : }Dieses ist: Erkenne (erforsche, ergründe) dich selbst nicht nach deiner physischen Vollkommenheit (der Tauglichkeit oder Untauglichkeit zu allerlei dir beliebigen oder auch gebotenen Zwecke), sondern nach der moralischen, in Beziehung auf deine Pflicht – dein Herz – ob es gut oder böse sei, ob die \match{Quelle} deiner Handlungen lauter oder unlauter, und was, entweder als ursprünglich zur Substanz des Menschen gehörend, oder, als abgeleitet (erworben oder zugezogen) ihm selbst zugerechnet werden kann und zum moralischen Zustande gehören mag. 
	
	\subsection*{tg489.2.42} 
	\textbf{Source : }Die Metaphysik der Sitten/Fußnoten\\  
	
	\noindent\textbf{Paragraphe : }
	
	19 Der Satz: man soll keiner Sache zu viel oder zu wenig tun, sagt so viel als nichts; denn er ist tautologisch. Was heißt zu viel tun? Antw. Mehr als gut ist; was heißt zu wenig tun? Antw. Weniger tun als gut ist. Was heißt: ich soll (etwas tun oder unterlassen)? Antw. Es ist nicht gut (wider die Pflicht), mehr oder auch weniger zu tun, als gut ist. Wenn das die Weisheit ist, die zu erforschen wir zu den Alten (dem Aristoteles), gleich als solchen, die der \match{Quelle} näher waren, zurückkehren sollen; virtus consistit in medio, medium tenuere beati, est modus in rebus, sunt certi denique fines, quos ultra citraque nequit consistere rectum, so haben wir schlecht gewählt, uns an ihr Orakel zu wenden. Es gibt zwischen Wahrhaftigkeit und Lüge (als contradictorie oppositis) kein Mittleres: aber wohl zwischen Offenherzigkeit und Zurückhaltung (als contrarie oppositis), da an dem, welcher seine Meinung erklärt, alles, was er sagt, wahr ist, er aber nicht die ganze Wahrheit sagt. Nun ist doch ganz natürlich von dem Tugendlehrer zu fordern, daß er mir dieses Mittlere anweise. Das kann er aber nicht; denn beide Tugendpflichten haben einen Spielraum der Anwendung (latitudinem) und, was zu tun sei, kann nur von der Urteilskraft, nach Regeln der Klugheit (den pragmatischen), nicht denen der Sittlichkeit (den moralischen), d.i. nicht als enge (officium strictum), sondern nur als weite Pflicht (officium latum) entschieden werden. Daher der, welcher die Grundsätze der Tugend befolgt, zwar in der Ausübung im Mehr oder Weniger, als die Klugheit vorschreibt, einen Fehler (peccatum) begehn, aber nicht darin, daß er diesen Grundsätzen mit Strenge anhänglich ist, ein Laster (vitium) ausüben, und Horazens Vers: insani sapiens nomen habeat aequus iniqui, ultra quam satis est virtutem si petat ipsam, ist, nach dem Buchstaben genommen, grundfalsch. Sapiens bedeutet hier wohl nur einen gescheuten Mann (prudens), der sich nicht phantastisch Tugendvollkommenheit denkt, die, als Ideal, zwar die Annäherung zu diesem Zwecke, aber nicht die Vollendung fordert, als welche Federung die menschlichen Kräfte übersteigt, und Unsinn (Phantasterei) in ihr Prinzip hinein bringt. Denn gar zu tugendhaft, d.i. seiner Pflicht gar zu anhänglich, zu sein, würde ohngefähr so viel sagen: als einen Zirkel gar zu rund, oder eine gerade Linie gar zu gerade machen. 
	
	\unnumberedsection{Schatten (2)} 
	\subsection*{tg481.2.84} 
	\textbf{Source : }Die Metaphysik der Sitten/Zweiter Teil. Metaphysische Anfangsgründe der Tugendlehre/I. Ethische Elementarlehre/II. Teil. Von den Tugendpflichten gegen andere/Erstes Hauptstück. Von den Pflichten gegen andere, bloß als Menschen/Erster Abschnitt. Von der Liebespflicht gegen andere Menschen\\  
	
	\noindent\textbf{Paragraphe : }a) Der Neid (livor), als Hang, das Wohl anderer mit Schmerz, wahrzunehmen, ob zwar dem seinigen dadurch kein Abbruch geschieht, der, wenn er zur Tat (jenes Wohl zu schmälern) ausschlägt, qualifizierter Neid, sonst aber nur Mißgunst (invidentia) heißt, ist doch nur eine indirekt-bösartige Gesinnung, nämlich ein Unwille, unser eigen Wohl durch das Wohl anderer in \match{Schatten} gestellt zu sehen, weil wir den Maßstab desselben nicht in dessen innerem Wert, sondern nur in der Vergleichung mit dem Wohl anderer, zu schätzen, und diese Schätzung zu versinnlichen wissen. – Daher spricht man auch wohl von einer beneidungswürdigen Eintracht und Glückseligkeit in einer Ehe, oder Familie u.s.w.; gleich als ob es in manchen Fällen erlaubt wäre, jemanden zu beneiden. Die Regungen des Neides liegen also in der Natur des Menschen, und nur der Ausbruch derselben macht sie zu dem scheußlichen Laster einer grämischen, sich selbst folternden und auf Zerstörung des Glücks anderer, wenigstens dem Wunsche nach, gerichteten Leidenschaft, ist mithin der Pflicht des Menschen gegen sich selbst so wohl, als gegen andere entgegengesetzt. 
	
	\subsection*{tg482.2.42} 
	\textbf{Source : }Die Metaphysik der Sitten/Zweiter Teil. Metaphysische Anfangsgründe der Tugendlehre/I. Ethische Elementarlehre/II. Teil. Von den Tugendpflichten gegen andere/Erstes Hauptstück. Von den Pflichten gegen andere, bloß als Menschen/Zweiter Abschnitt. Von den Tugendpflichten gegen andere Menschen aus der ihnen gebührenden Achtung\\  
	
	\noindent\textbf{Paragraphe : }Die geflissentliche Verbreitung (propalatio) desjenigen, die Ehre eines andern Schmälernden, was auch nicht zur öffentlichen Gerichtsbarkeit gehört, es mag übrigens auch wahr sein, ist Verringerung der Achtung für die Menschheit überhaupt, um endlich auf unsere Gattung selbst den \match{Schatten} der Nichtswürdigkeit zu werfen, und Misanthropie (Menschenscheu) oder Verachtung zur herrschenden Denkungsart zu machen, oder sein moralisches Gefühl durch den öfteren Anblick derselben abzustumpfen und sich daran zu gewöhnen. Es ist also Tugendpflicht, statt einer hämischen Lust an der Bloßstellung der Fehler anderer, um sich dadurch die Meinung, gut, wenigstens nicht schlechter als alle andere Menschen zu sein, zu sichern, den Schleier der Menschenliebe, nicht bloß durch Milderung unserer Urteile, sondern auch durch Verschweigung derselben, über die Fehler anderer zu werfen; weil Beispiele der Achtung, welche uns andere geben, auch die Bestrebung rege machen können, sie gleichmäßig zu verdienen. – Um deswillen ist die Ausspähungssucht der Sitten anderer (allotrio-episcopia) auch für sich selbst schon ein beleidigender Vorwitz der Menschenkunde, welchem jedermann sich mit Recht als Verletzung der ihm schuldigen Achtung widersetzen kann. 
	
	\unnumberedsection{Seite (2)} 
	\subsection*{tg435.2.3} 
	\textbf{Source : }Die Metaphysik der Sitten/Erster Teil. Metaphysische Anfangsgründe der Rechtslehre/1. Teil. Das Privatrecht vom äußeren Mein und Dein überhaupt/2. Hauptstück. Von der Art, etwas Äußeres zu erwerben/1. Abschnitt. Vom Sachrecht\\  
	
	\noindent\textbf{Paragraphe : }Die gewöhnliche Erklärung des Rechts in einer Sache (ius reale, ius in re): »es sei das Recht gegen jeden Besitzer derselben«, ist eine richtige Nominaldefinition. – Aber, was ist das, was da macht, daß ich mich wegen eines äußeren Gegenstandes an jeden Inhaber desselben halten, und ihn (per vindicationem) nötigen kann, mich wieder in Besitz desselben zu setzen? Ist dieses äußere rechtliche Verhältnis meiner Willkür etwa ein unmittelbares Verhältnis zu einem körperlichen Dinge? So müßte derjenige, welcher sein Recht nicht unmittelbar auf Personen, sondern auf Sachen bezogen denkt, es sich freilich (obzwar nur auf dunkele Art) vorstellen: nämlich, weil dem Recht auf einer  \match{Seite} eine Pflicht auf der andern korrespondiert, daß die äußere Sache, ob sie zwar dem ersten Besitzer abhanden gekommen, diesem doch immer verpflichtet bleibe, d.i. sich jedem anmaßlichen anderen Besitzer weigere, weil sie jenem schon verbindlich ist, und so mein Recht, gleich einem die Sache begleitenden und vor allem fremden Angriffe bewahrenden Genius, den fremden Besitzer immer an mich weise. Es ist also ungereimt, sich Verbindlichkeit einer Person gegen Sachen und umgekehrt zu denken, wenn es gleich allenfalls erlaubt werden mag, das rechtliche Verhältnis durch ein solches Bild zu versinnlichen, und sich so auszudrücken. 
	
	\subsection*{tg441.2.71} 
	\textbf{Source : }Die Metaphysik der Sitten/Erster Teil. Metaphysische Anfangsgründe der Rechtslehre/2. Teil. Das öffentliche Recht/1. Abschnitt. Das Staatsrecht\\  
	
	\noindent\textbf{Paragraphe : }Welche Art aber und welcher Grad der Bestrafung ist es, welche die öffentliche Gerechtigkeit sich zum Prinzip und Richtmaße macht? Kein anderes, als das Prinzip der Gleichheit (im Stande des Züngleins an der Wage der Gerechtigkeit), sich nicht mehr auf die eine, als auf die andere \match{Seite} hinzuneigen. Also: was für unverschuldetes Übel du einem  anderen im Volk zufügst, das tust du dir selbst an. Beschimpfst du ihn, so beschimpfst du dich selbst; bestiehlst du ihn, so bestiehlst du dich selbst; schlägst du ihn, so schlägst du dich selbst; tötest du ihn, so tötest du dich selbst. Nur das Wiedervergeltungsrecht (ius talionis), aber, wohl zu verstehen, vor den Schranken des Gerichts (nicht in deinem Privaturteil), kann die Qualität und Quantität der Strafe bestimmt angeben; alle andere sind hin und her schwankend, und können, anderer sich einmischenden Rücksichten wegen, keine Angemessenheit mit dem Spruch der reinen und strengen Gerechtigkeit enthalten. – Nun scheint es zwar, daß der Unterschied der Stände das Prinzip der Wiedervergeltung Gleiches mit Gleichem nicht verstatte; aber, wenn es gleich nicht nach dem Buchstaben möglich sein kann, so kann es doch der Wirkung nach, respektive auf die Empfindungsart der Vornehmeren, immer geltend bleiben. – So hat z.B. Geldstrafe wegen einer Verbalinjurie gar kein Verhältnis zur Beleidigung, denn, der des Geldes viel hat, kann diese sich wohl einmal zur Lust erlauben; aber die Kränkung der Ehrliebe des einen kann doch dem Wehtun des Hochmuts des anderen sehr gleich kommen: wenn dieser nicht allein öffentlich abzubitten, sondern jenem, ob er zwar niedriger ist, etwa zugleich die Hand zu küssen, durch Urteil und Recht genötigt würde. Eben so, wenn der gewalttätige Vornehme für die Schläge, die er dem niederen aber schuldlosen Staatsbürger zumißt, außer der Abbitte noch zu einem einsamen und beschwerlichen Arrest verurteilt würde, weil hiemit, außer der Ungemächlichkeit, noch die Eitelkeit des Täters schmerzhaft angegriffen, und so durch Beschämung Gleiches mit Gleichem gehörig vergolten würde. – Was heißt das aber: »bestiehlst du ihn, so bestiehlst du dich selbst «? Wer da stiehlt, macht aller anderer Eigentum unsicher; er beraubt sich also (nach dem Recht der Wiedervergeltung) der Sicherheit alles möglichen Eigentums; er hat nichts und kann auch nichts erwerben, will aber doch leben; welches nun nicht anders möglich ist, als daß ihn andere ernähren. Weil dieses aber der Staat nicht umsonst tun wird, so muß er diesem seine  Kräfte zu ihm beliebigen Arbeiten (Karren- oder Zuchthausarbeit) überlassen, und kommt auf gewisse Zeit, oder, nach Befinden, auch auf immer, in den Sklavenstand. – Hat er aber gemordet, so muß er sterben. Es gibt hier Kein Surrogat zur Befriedigung der Gerechtigkeit. Es ist keine Gleichartigkeit zwischen einem noch so kummervollen Leben und dem Tode, also auch keine Gleichheit des Verbrechens und der Wiedervergeltung, als durch den am Täter gerichtlich vollzogenen, doch von aller Mißhandlung, welche die Menschheit in der leidenden Person zum Scheusal machen könnte, befreieten Tod. – Selbst, wenn sich die bürgerliche Gesellschaft mit aller Glieder Einstimmung auflösete (z.B. das eine Insel bewohnende Volk beschlösse, auseinander zu gehen, und sich in alle Welt zu zerstreuen), müßte der letzte im Gefängnis befindliche Mörder vorher hingerichtet werden, damit jedermann das widerfahre, was seine Taten wert sind, und die Blutschuld nicht auf dem Volke hafte, das auf diese Bestrafung nicht gedrungen hat; weil es als Teilnehmer an dieser öffentlichen Verletzung der Gerechtigkeit betrachtet werden kann. 
	
	\unnumberedsection{Starke (13)} 
	\subsection*{tg430.2.18} 
	\textbf{Source : }Die Metaphysik der Sitten/Erster Teil. Metaphysische Anfangsgründe der Rechtslehre/Einleitung in die Metaphysik der Sitten\\  
	
	\noindent\textbf{Paragraphe : }Das Gegenstück einer Metaphysik der Sitten, als das andere Glied der Einteilung der praktischen Philosophie überhaupt, würde die moralische Anthropologie sein, welche, aber nur die subjektive, hindernde sowohl, als begünstigende, Bedingungen der Ausführung der Gesetze der ersteren in der menschlichen Natur, die Erzeugung, Ausbreitung und Stärkung moralischer Grundsätze (in der Erziehung der Schul- und Volksbelehrung) und dergleichen andere sich auf Erfahrung gründende Lehren und Vorschriften enthalten würde, und die nicht entbehrt werden kann, aber durchaus nicht vor jener vorausgeschickt, oder mit ihr vermischt werden muß; weil man alsdenn Gefahr läuft, falsche, oder wenigstens nachsichtliche moralische Gesetze herauszubringen, welche das für unerreichbar vorspiegeln, was nur eben darum nicht er reicht wird, weil das Gesetz nicht in seiner Reinigkeit (als worin auch seine \match{Stärke} besteht) eingesehen und vorgetragen worden, oder gar unechte, oder unlautere Triebfedern zu dem, was an sich pflichtmäßig und gut ist, gebraucht werden, welche keine sichere moralische Grundsätze übrig lassen; weder zum Leitfaden der Beurteilung, noch zur Disziplin des Gemüts in der Befolgung der Pflicht, deren Vorschrift schlechterdings nur durch reine Vernunft a priori gegeben werden muß. 
	
	\subsection*{tg447.2.5} 
	\textbf{Source : }Die Metaphysik der Sitten/Zweiter Teil. Metaphysische Anfangsgründe der Tugendlehre/Vorrede\\  
	
	\noindent\textbf{Paragraphe : }In dieser Philosophie (der Tugendlehre) scheint es nun der Idee derselben gerade zuwider zu sein, bis zu metaphysischen Anfangsgründen zurückzugehen, um den Pflichtbegriff, von allem Empirischen (jedem Gefühl) gereinigt, doch zur Triebfeder zu machen. Denn was kann man sich für einen Begriff von einer Kraft und herkulischer \match{Stärke}  machen, um die lastergebärende Neigungen zu überwältigen, wenn die Tugend ihre Waffen aus der Rüstkammer der Metaphysik entlehnen soll? welche eine Sache der Spekulation ist, die nur wenig Menschen zu handhaben wissen. Daher fallen auch alle Tugendlehren, in Hörsälen, von Kanzeln und in Volksbüchern, wenn sie mit metaphysischen Brocken ausgeschmückt werden, ins Lächerliche. – Aber darum ist es doch nicht unnütz, vielweniger lächerlich, den ersten Gründen der Tugendlehre in einer Metaphysik nachzuspüren; denn irgend einer muß doch als Philosoph auf die ersten Gründe dieses Pflichtbegriffs hinausgehen: weil sonst weder Sicherheit noch Lauterkeit für die Tugendlehre überhaupt zu erwarten wäre. Sich desfalls auf ein gewisses Gefühl, welches man, seiner davon erwarteten Wirkung halber, moralisch nennt, zu verlassen, kann auch wohl dem Volkslehrer gnügen: indem dieser zum Probierstein einer Tugendpflicht, ob sie es sei oder nicht, die Aufgabe zu beherzigen verlangt: »wie, wenn nun ein jeder in jedem Fall deine Maxime zum allgemeinen Gesetz machte, würde eine solche wohl mit sich selbst zusammenstimmen können?« Aber, wenn es bloß Gefühl wäre, was auch diesen Satz zum Probierstein zu nehmen uns zur Pflicht machte, so wäre diese doch alsdann nicht durch die Vernunft diktiert, sondern nur instinktmäßig, mithin blindlings dafür angenommen. 
	
	\subsection*{tg450.2.10} 
	\textbf{Source : }Die Metaphysik der Sitten/Zweiter Teil. Metaphysische Anfangsgründe der Tugendlehre/Einleitung/II. Erörterung des Begriffs von einem Zwecke, der zugleich Pflicht ist\\  
	
	\noindent\textbf{Paragraphe : }Der Tugend = + a ist die negative Untugend (moralische Schwäche) = 0 als logisches Gegenteil (contradictorie oppositum), das Laster aber = – a als Widerspiel (contrarie s. realiter oppositum) entgegen gesetzt und es ist eine, nicht bloß unnötige, sondern auch anstößige Frage: ob zu großen Verbrechen nicht etwa mehr \match{Stärke} der Seele als selbst zu großen Tugenden gehöre. Denn unter Stärke der Seele verstehen wir die Stärke des Vorsatzes eines Menschen, als mit Freiheit begabten Wesens, mithin so fern er seiner selbst mächtig (bei Sinnen) ist, also im gesunden Zustande des Menschen. Große Verbrechen aber sind Paroxysmen, deren Anblick den an Seele gesunden Menschen schaudern macht. Die Frage würde also etwa dahin auslaufen: ob ein Mensch im Anfall einer Krankheit mehr physische Stärke haben könne, als wenn er bei Sinnen ist; welches  man einräumen kann, ohne ihm darum mehr Seelenstärke beizulegen, wenn man unter Seele das Lebensprinzip des Menschen im freien Gebrauch seiner Kräfte versteht. Denn, weil jene bloß in der Macht der die Vernunft schwächenden Neigungen ihren Grund haben, welches keine Seelenstärke beweiset, so würde diese Frage mit der ziemlich auf einerlei hinauslaufen: ob ein Mensch im Anfall einer Krankheit mehr Stärke als im gesunden Zustande beweisen könne, welche geradezu verneinend beantwortet werden kann, weil der Mangel der Gesundheit, die im Gleichgewicht aller körperlichen Kräfte des Menschen besteht, eine Schwächung im System dieser Kräfte ist, nach welchem man allein die absolute Gesundheit beurteilen kann. 
	
	\subsection*{tg455.2.3} 
	\textbf{Source : }Die Metaphysik der Sitten/Zweiter Teil. Metaphysische Anfangsgründe der Tugendlehre/Einleitung/VII. Die ethischen Pflichten sind von weiter, dagegen die Rechtspflichten von enger Verbindlichkeit\\  
	
	\noindent\textbf{Paragraphe : }Die unvollkommenen Pflichten sind also allein Tugendpflichten. Die Erfüllung derselben ist Verdienst (meritum) = + a; ihre Übertretung aber ist nicht so fort Verschuldung (demeritum) = – a, sondern bloß moralischer Unwert = 0, außer, wenn es dem Subjekt Grundsatz wäre, sich jenen Pflichten nicht zu fügen. Die \match{Stärke} des Vorsatzes im ersteren heißt eigentlich allein Tugend (virtus), die Schwäche in der zweiten nicht sowohl Laster (vitium) als vielmehr bloß Untugend, Mangel an moralischer Stärke  (defectus moralis). (Wie das Wort Tugend von taugen, so stammt Untugend von zu nichts taugen.) Eine jede pflichtwidrige Handlung heißt Übertretung (peccatum). Die vorsätzliche aber, die zum Grundsatz geworden ist, macht eigentlich das aus, was man Laster (vitium) nennt. 
	
	\subsection*{tg457.2.2} 
	\textbf{Source : }Die Metaphysik der Sitten/Zweiter Teil. Metaphysische Anfangsgründe der Tugendlehre/Einleitung/IX. Was ist Tugendpflicht\\  
	
	\noindent\textbf{Paragraphe : }
	Tugend ist die \match{Stärke} der Maxime des Menschen in Befolgung seiner Pflicht. – Alle Stärke wird nur durch Hindernisse erkannt, die sie überwältigen kann; bei der Tugend aber sind diese die Naturneigungen, welche mit dem sittlichen Vorsatz in Streit kommen können, und, da der Mensch es selbst ist, der seinen Maximen diese Hindernisse in den Weg legt, so ist die Tugend nicht bloß ein Selbstzwang (denn da könnte eine Naturneigung die andere zu bezwingen trachten), sondern auch ein Zwang nach einem Prinzip der innern Freiheit, mithin durch die bloße Vorstellung seiner Pflicht, nach dem formalen Gesetz derselben. 
	
	\subsection*{tg458.2.5} 
	\textbf{Source : }Die Metaphysik der Sitten/Zweiter Teil. Metaphysische Anfangsgründe der Tugendlehre/Einleitung/X. Das oberste Prinzip der Rechtslehre war analytisch; das der Tugendlehre ist synthetisch\\  
	
	\noindent\textbf{Paragraphe : }Man kann auch gar wohl sagen: der Mensch sei zur Tugend (als einer moralischen Stärke) verbunden. Denn obgleich das Vermögen (facultas) der Überwindung aller sinnlich entgegenwirkenden Antriebe, seiner Freiheit halber, schlechthin vorausgesetzt werden kann und muß: so ist doch dieses Vermögen als \match{Stärke} (robur) etwas, was erworben werden muß, dadurch, daß die moralische Triebfeder (die Vorstellung des Gesetzes) durch Betrachtung (contemplatione) der Würde des reinen Vernunftgesetzes in uns, zugleich aber auch durch Übung (exercitio) erhoben wird. 
	
	\subsection*{tg461.2.12} 
	\textbf{Source : }Die Metaphysik der Sitten/Zweiter Teil. Metaphysische Anfangsgründe der Tugendlehre/Einleitung/XIII. Allgemeine Grundsätze der Metaphysik der Sitten in Behandlung einer reinen Tugendlehre\\  
	
	\noindent\textbf{Paragraphe : }Tugend bedeutet eine moralische \match{Stärke} des Willens. Aber dies erschöpft noch nicht den Begriff; denn eine solche Stärke könnte auch einem heiligen (übermenschlichen) Wesen zukommen, in welchem kein hindernder Antrieb dem Gesetze seines Willens entgegen wirkt; das also alles dem Gesetz gemäß gerne tut. Tugend ist also die moralische Stärke des Willens eines Menschen in Befolgung seiner Pflicht: welche eine moralische Nötigung durch seine eigene gesetzgebende Vernunft ist, insofern diese sich zu einer das Gesetz ausführenden Gewalt selbst konstituiert. – Sie ist nicht selbst, oder sie zu besitzen ist nicht Pflicht (denn sonst würde es eine Verpflichtung zur Pflicht geben müssen), sondern sie gebietet und begleitet ihr Gebot durch einen sittlichen (nach Gesetzen der inneren Freiheit möglichen) Zwang; wozu aber, weil er unwiderstehlich sein soll, Stärke erforderlich ist, deren Grad wir nur durch die Große der Hindernisse, die der Mensch durch seine Neigungen sich selber schafft, schätzen können. Die Laster, als die Brut gesetzwidriger Gesinnungen, sind die Ungeheuer, die er nun zu bekämpfen hat: weshalb diese sittliche Stärke auch, als Tapferkeit (fortitudo moralis), die größte und einzige wahre Kriegsehre des Menschen ausmacht; auch wird sie die eigentliche, nämlich praktische Weisheit genannt: weil sie den Endzweck des Daseins der Menschen auf Erden zu dem ihrigen macht. – In ihrem Besitz ist der Mensch allein frei, gesund, reich, ein König u.s.w. und kann, weder durch Zufall, noch Schicksal einbüßen; weil er sich selbst besitzt und der Tugendhafte seine Tugend nicht verlieren kann. 
	
	\subsection*{tg463.2.2} 
	\textbf{Source : }Die Metaphysik der Sitten/Zweiter Teil. Metaphysische Anfangsgründe der Tugendlehre/Einleitung/XV. Zur Tugend wird zuerst erfordert die Herrschaft über sich selbst\\  
	
	\noindent\textbf{Paragraphe : }
	Affekten und Leidenschaften sind wesentlich von einander unterschieden; die erstern gehören zum Gefühl, so fern es, vor der Überlegung vorhergehend, diese selbst unmöglich oder schwerer macht. Daher heißt der Affekt jäh, oder jach (animus praeceps) und die Vernunft sagt durch den Tugendbegriff, man solle sich fassen; doch ist diese Schwäche  im Gebrauch seines Verstandes, verbunden mit der \match{Stärke} der Gemütsbewegung, nur eine Untugend und gleichsam etwas Kindisches und Schwaches, was mit dem besten Willen gar wohl zusammen bestehen kann, und das einzige Gute noch an sich hat, daß dieser Sturm bald aufhört. Ein Hang zum Affekt (z.B. Zorn) verschwistert sich daher nicht so sehr mit dem Laster, als die Leidenschaft. Leidenschaft dagegen ist die zur bleibenden Neigung gewordene sinnliche Begierde (z.B. der Haß im Gegensatz des Zorns). Die Ruhe, mit der ihr nachgehangen wird, läßt Überlegung zu und verstattet dem Gemüt, sich darüber Grundsätze zu machen und so, wenn die Neigung auf das Gesetzwidrige fällt, über sie zu brüten, sie tief zu wurzeln und das Böse dadurch (als vorsätzlich) in seine Maxime aufzunehmen; welches alsdann ein qualifiziertes Böse, d.i. ein wahres Laster ist. 
	
	\subsection*{tg464.2.2} 
	\textbf{Source : }Die Metaphysik der Sitten/Zweiter Teil. Metaphysische Anfangsgründe der Tugendlehre/Einleitung/XVI. Zur Tugend wird Apathie (als Stärke betrachtet) notwendig vorausgesetzt\\  
	
	\noindent\textbf{Paragraphe : }Dieses Wort ist, gleich als ob es Fühllosigkeit, mithin subjektive Gleichgültigkeit in Ansehung der Gegenstände der Willkür, bedeutete, in übelen Ruf gekommen; man nahm es für Schwäche. Dieser Mißdeutung kann dadurch vorgebeugt werden, daß man diejenige Affektlosigkeit, welche von der Indifferenz zu unterscheiden ist, die moralische Apathie nennt: da die Gefühle aus sinnlichen Eindrücken ihren Einfluß auf das moralische nur dadurch verlieren, daß die Achtung fürs Gesetz über sie insgesamt mächtiger wird. – Es ist nur die scheinbare \match{Stärke} eines Fieberkranken, die den lebhaften Anteil selbst am Guten bis zum Affekt steigen, oder vielmehr darin ausarten läßt. Man nennt den Affekt dieser Art Enthusiasm, und dahin ist auch die Mäßigung zu deuten, die man selbst für Tugendausübungen zu empfehlen pflegt (insani sapiens nomen habeat aequus iniqui – ultra, quam satis est virtutem si petat ipsam. Horat.). Denn sonst ist es ungereimt zu wähnen, man könne auch wohl allzuweise, allzutugendhaft sein. Der Affekt gehört immer zur Sinnlichkeit; er mag durch einen Gegenstand erregt werden, welcher es wolle. Die wahre Stärke der Tugend ist das Gemüt in Ruhe, mit einer überlegten und festen Entschließung, ihr Gesetz in Ausübung zu bringen. Das ist der Zustand der Gesundheit im moralischen Leben; dagegen der Affekt, selbst wenn er durch die Vorstellung des Guten aufgeregt wird, eine augenblicklich glänzende Erscheinung ist, welche Mattigkeit hinterläßt. – Phantastisch-tugendhaft aber kann doch der genannt werden, der keine in Ansehung der Moralität gleichgültige Dinge (adiaphora) einräumt und sich alle seine Schritte und Tritte mit Pflichten als mit Fußangeln bestreut und es nicht gleichgültig findet, ob ich mich mit Fleisch oder Fisch, mit Bier oder Wein, wenn mir beides bekömmt, nähre; eine Mikrologie, welche, wenn man sie in die Lehre der Tugend aufnähme, die Herrschaft derselben zur Tyrannei machen würde. 
	
	\subsection*{tg471.2.45} 
	\textbf{Source : }Die Metaphysik der Sitten/Zweiter Teil. Metaphysische Anfangsgründe der Tugendlehre/I. Ethische Elementarlehre/I. Teil. Von den Pflichten gegen sich selbst überhaupt/Erstes Buch. Von den vollkommenen Pflichten gegen sich selbst/Erstes Hauptstück. Die Pflicht des Menschen gegen sich selbst, als einem animalischen Wesen\\  
	
	\noindent\textbf{Paragraphe : }Die tierische Unmäßigkeit, im Genuß der Nahrung, ist der Mißbrauch der Genießmittel, wodurch das Vermögen des intellektuellen Gebrauchs derselben gehemmt oder erschöpft wird. Versoffenheit und Gefräßigkeit sind die Laster, die unter diese Rubrik gehören. Im Zustande der Betrunkenheit ist der Mensch nur wie ein Tier, nicht als Mensch, zu behandeln; durch die Überladung mit Speisen und in einem solchen Zustande ist er für Handlungen, wozu Gewandtheit und Überlegung im Gebrauch seiner Kräfte erfordert wird, auf eine gewisse Zeit gelähmt. – Daß sich in einen solchen Zustand zu versetzen Verletzung einer Pflicht wider sich selbst sei, fällt von selbst in die Augen. Die erste dieser Erniedrigungen, selbst unter die tierische Natur, wird gewöhnlich durch gegorene Getränke, aber auch durch andere betäubende Mittel, als den Mohnsaft und andere Produkte des Gewächsreichs, bewirkt, und wird dadurch verführerisch, daß dadurch auf eine Weile geträumte Glückseligkeit und Sorgenfreiheit, ja wohl auch eingebildete \match{Stärke} hervorgebracht, Niedergeschlagenheit aber und Schwäche, und, was das Schlimmste ist, Notwendigkeit, dieses Betäubungsmittel zu wiederholen, ja wohl gar damit zu steigern, eingeführt wird. Die Gefräßigkeit ist sofern noch unter jener tierischen Sinnenbelustigung, daß sie bloß den Sinn  als passive Beschaffenheit und nicht einmal die Einbildungskraft, welche doch noch ein tätiges Spiel der Vorstellungen, wie im vorerwähnten Genuß der Fall ist, beschäftigt; mithin sich dem des Viehes noch mehr nähert. 
	
	\subsection*{tg478.2.10} 
	\textbf{Source : }Die Metaphysik der Sitten/Zweiter Teil. Metaphysische Anfangsgründe der Tugendlehre/I. Ethische Elementarlehre/I. Teil. Von den Pflichten gegen sich selbst überhaupt/2. Buch: Die Pflichten gegen sich selbst/Zweiter Abschnitt. Von der Pflicht gegen sich selbst in Erhöhung seiner moralischen Vollkommenheit, d.i. in bloß sittlicher Absicht\\  
	
	\noindent\textbf{Paragraphe : }Die Tiefen des menschlichen Herzens sind unergründlich. Wer kennt sich gnugsam, wenn die Triebfeder zur Pflichtbeobachtung von ihm gefühlt wird, ob sie gänzlich aus der Vorstellung des Gesetzes hervorgehe, oder ob nicht manche andere, sinnliche Antriebe mitwirken, die auf den Vorteil (oder zur Verhütung eines Nachteils) angelegt sind und bei anderer Gelegenheit auch wohl dem Laster zu Diensten stehen könnten. – Was aber die Vollkommenheit als moralischen Zweck betrifft, so gibt's zwar in der Idee (objektiv) nur eine Tugend (als sittliche \match{Stärke} der Maximen), in der Tat (subjektiv) aber eine Menge derselben von heterogener Beschaffenheit, worunter es unmöglich sein dürfte, nicht irgend eine Untugend (ob sie gleich eben jener wegen den Namen des Lasters nicht zu führen pflegen) aufzufinden, wenn man sie suchen wollte. Eine Summe von Tugenden aber, deren Vollständigkeit oder Mängel das Selbsterkenntnis uns nie hinreichend einschauen läßt, kann keine andere als unvollkommene Pflicht vollkommen zu sein begründen. 
	
	\subsection*{tg482.2.52} 
	\textbf{Source : }Die Metaphysik der Sitten/Zweiter Teil. Metaphysische Anfangsgründe der Tugendlehre/I. Ethische Elementarlehre/II. Teil. Von den Tugendpflichten gegen andere/Erstes Hauptstück. Von den Pflichten gegen andere, bloß als Menschen/Zweiter Abschnitt. Von den Tugendpflichten gegen andere Menschen aus der ihnen gebührenden Achtung\\  
	
	\noindent\textbf{Paragraphe : }Die verschiedene andern zu beweisende Achtung nach Verschiedenheit der Beschaffenheit der Menschen, oder ihrer zufälligen Verhältnisse, nämlich der des Alters, des Geschlechts, der Abstammung, der \match{Stärke} oder Schwäche, oder gar des Standes und der Würde, welche zum Teil auf beliebigen Anordnungen beruhen, darf in metaphysischen Anfangsgründen der Tugendlehre nicht ausführlich dargestellt und klassifiziert werden, da es hier nur um die reinen Vernunftprinzipien derselben zu tun ist. 
	
	\subsection*{tg486.2.6} 
	\textbf{Source : }Die Metaphysik der Sitten/Zweiter Teil. Metaphysische Anfangsgründe der Tugendlehre/II. Ethische Methodenlehre/1. Abschnitt. Die ethische Didaktik\\  
	
	\noindent\textbf{Paragraphe : }Daß Tugend erworben werden müsse (nicht angeboren sei), liegt, ohne sich deshalb auf anthropologische Kenntnisse aus der Erfahrung berufen zu dürfen, schon in dem Begriffe derselben. Denn das sittliche Vermögen des Menschen wäre nicht Tugend, wenn es nicht durch die \match{Stärke} des Vorsatzes, in dem Streit mit so mächtigen entgegenstehenden Neigungen, hervorgebracht wäre. Sie ist das Produkt aus der reinen praktischen Vernunft, so fern diese im Bewußtsein ihrer Überlegenheit (aus Freiheit) über jene die Obermacht gewinnt. 
	
	\unnumberedsection{Stelle (7)} 
	\subsection*{tg433.2.6} 
	\textbf{Source : }Die Metaphysik der Sitten/Erster Teil. Metaphysische Anfangsgründe der Rechtslehre/1. Teil. Das Privatrecht vom äußeren Mein und Dein überhaupt/1. Hauptstück\\  
	
	\noindent\textbf{Paragraphe : }Der Ausdruck: ein Gegenstand ist außer mir, kann aber entweder so viel bedeuten, als: er ist ein nur von mir (dem Subjekt) unterschiedener, oder auch ein in einer anderen \match{Stelle} (positus), im Raum oder in der Zeit, befindlicher Gegenstand. Nur in der ersteren Bedeutung genommen kann der Besitz als Vernunftbesitz gedacht werden; in der zweiten aber würde er ein empirischer heißen müssen. – Ein intelligibler Besitz (wenn ein solcher möglich ist) ist ein Besitz ohne Inhabung (detentio). 
	
	\subsection*{tg437.2.41} 
	\textbf{Source : }Die Metaphysik der Sitten/Erster Teil. Metaphysische Anfangsgründe der Rechtslehre/1. Teil. Das Privatrecht vom äußeren Mein und Dein überhaupt/2. Hauptstück. Von der Art, etwas Äußeres zu erwerben/3. Abschnitt. Von dem auf dingliche Art persönlichen Recht\\  
	
	\noindent\textbf{Paragraphe : }Dieser Vertrag also der Hausherrschaft mit dem Gesinde kann nicht von solcher Beschaffenheit sein, daß der Gebrauch desselben ein Verbrauch sein würde, worüber das Urteil aber nicht bloß dem Hausherrn, sondern auch der Dienerschaft (die also nie Leibeigenschaft sein kann) zukommt; kann also nicht auf lebenslängliche, sondern allenfalls nur auf unbestimmte Zeit, binnen der ein Teil dem anderen die Verbindung aufkündigen darf, geschlossen werden. Die Kinder aber (selbst die eines durch sein Verbrechen zum Sklaven Gewordenen) sind jederzeit frei. Denn frei geboren ist jeder Mensch, weil er noch nichts verbrochen hat, und die Kosten der Erziehung bis zu seiner Volljährigkeit können ihm auch nicht als eine Schuld angerechnet werden, die er zu tilgen habe. Denn der Sklave müßte, wenn er könnte, seine Kinder auch erziehen, ohne ihnen dafür Kosten zu verrechnen; der Besitzer des Sklaven tritt also, bei dieses seinem Unvermögen, in die \match{Stelle} seiner Verbindlichkeit. 
	
	\subsection*{tg437.2.66} 
	\textbf{Source : }Die Metaphysik der Sitten/Erster Teil. Metaphysische Anfangsgründe der Rechtslehre/1. Teil. Das Privatrecht vom äußeren Mein und Dein überhaupt/2. Hauptstück. Von der Art, etwas Äußeres zu erwerben/3. Abschnitt. Von dem auf dingliche Art persönlichen Recht\\  
	
	\noindent\textbf{Paragraphe : }γ) Der Bevollmächtigungsvertrag (mandatum): Die Geschäftsführung an der \match{Stelle} und im Namen eines anderen, welche, wenn sie bloß an des anderen Stelle, nicht zugleich in seinem (des Vertretenen) Namen, geführt wird, Geschäftsführung ohne Auftrag (gestio negotii), wird sie aber im Namen des anderen verrichtet, Mandat heißt, das hier, als Verdingungsvertrag, ein belästigter Vertrag (mandatum onerosum) ist. 
	
	\subsection*{tg437.2.79} 
	\textbf{Source : }Die Metaphysik der Sitten/Erster Teil. Metaphysische Anfangsgründe der Rechtslehre/1. Teil. Das Privatrecht vom äußeren Mein und Dein überhaupt/2. Hauptstück. Von der Art, etwas Äußeres zu erwerben/3. Abschnitt. Von dem auf dingliche Art persönlichen Recht\\  
	
	\noindent\textbf{Paragraphe : }Die Sache nun, welche Geld heißen soll, muß also selbst so viel Fleiß gekostet haben, um sie hervorzubringen, oder auch anderen Menschen in die Hände zu schaffen, daß dieser demjenigen Fleiß, durch welchen die Ware (in Natur- oder Kunstprodukten) hat erworben werden müssen, und gegen welchen jener ausgetauscht wird, gleich komme. Denn wäre es leichter, den Stoff, der Geld heißt, als die Ware anzuschaffen, so käme mehr Geld zu Markte, als Ware  feilsteht, und weil der Käufer mehr Fleiß auf seine Ware verwenden müßte, als der Käufer, dem das Geld schneller zuströmt: so würde der Fleiß in Verfertigung der Ware und so das Gewerbe überhaupt mit dem Erwerbfleiß, der den öffentlichen Reichtum zu Folge hat, zugleich schwinden und verkürzt werden. – Daher können Banknoten und Assignaten nicht für Geld angesehen werden, ob sie gleich eine Zeit hindurch die \match{Stelle} desselben vertreten; weil es beinahe gar keine Arbeit kostet, sie zu verfertigen, und ihr Wert sich bloß auf die Meinung der ferneren Fortdauer der bisher gelungenen Umsetzung derselben in Barschaft gründet, welche, bei einer etwanigen Entdeckung, daß die letztere nicht in einer zum leichten und sicheren Verkehr hinreichenden Menge da sei, plötzlich verschwindet, und den Ausfall der Zahlung unvermeidlich macht. – So ist der Erwerbfleiß derer, welche die Gold- und Silberbergwerke in Peru, oder Neumexiko anbauen, vornehmlich bei den so vielfältig mißlingenden Versuchen eines vergeblich angewandten Fleißes, im Aufsuchen der Erzgange, wahrscheinlich noch größer, als der auf Verfertigung der Waren in Europa verwendete, und würde, als unvergolten, mithin von selbst nachlassend, jene Länder bald in Armut sinken lassen, wenn nicht der Fleiß Europens dagegen, eben durch diese Materialien gereizt, sich proportionierlich zugleich erweiterte, um bei jenen die Lust zum Bergbau, durch ihnen angebotene Sachen des Luxus, beständig rege zu erhalten; so daß immer Fleiß gegen Fleiß in Konkurrenz kommen. 
	
	\subsection*{tg445.2.73} 
	\textbf{Source : }Die Metaphysik der Sitten/Erster Teil. Metaphysische Anfangsgründe der Rechtslehre/Anhang erläutender Bemerkungen zu den metaphysischen Anhangsgründen der Rechtslehre\\  
	
	\noindent\textbf{Paragraphe : }Wenn dann nun ein Volk, durch Gesetze unter einer Obrigkeit vereinigt, da ist, so ist der Idee der Einheit desselben überhaupt unter einem machthabenden obersten Willen gemäß  als Gegenstand der Erfahrung gegeben; aber freilich nur in der Erscheinung; d.i. eine rechtliche Verfassung, im allgemeinen Sinne des Worts, ist da; und, obgleich sie mit großen Mängeln und groben Fehlern behaftet sein und nach und nach wichtiger Verbesserungen bedürfen mag, so ist es doch schlechterdings unerlaubt und sträflich, ihr zu widerstehen; weil, wenn das Volk dieser, obgleich noch fehlerhaften Verfassung und der obersten Autorität Gewalt entgegen setzen zu dürfen sich berechtigt hielte, es sich dünken würde, ein Recht zu haben: Gewalt an die \match{Stelle} der alle Rechte zu oberst vorschreibenden Gesetzgebung zu setzen; welches einen sich selbstzerstörenden obersten Willen abgeben würde. 
	
	\subsection*{tg486.2.35} 
	\textbf{Source : }Die Metaphysik der Sitten/Zweiter Teil. Metaphysische Anfangsgründe der Tugendlehre/II. Ethische Methodenlehre/1. Abschnitt. Die ethische Didaktik\\  
	
	\noindent\textbf{Paragraphe : }Wenn dieses nun weislich und pünktlich nach Verschiedenheit der Stufen des Alters, des Geschlechts und des Standes, die der Mensch nach und nach betritt, aus der eigenen Vernunft des Menschen entwickelt worden,  so ist noch etwas, was den Beschluß machen muß, was die Seele inniglich bewegt und den Menschen auf eine \match{Stelle} setzt, wo er sich selbst nicht anders als mit der größten Bewunderung der ihm beiwohnenden ursprünglichen Anlagen betrachten kann, und wovon der Eindruck nie erlischt. – Wenn ihm nämlich beim Schlüsse seiner Unterweisung seine Pflichten in ihrer Ordnung noch einmal summarisch vorerzählt (rekapituliert), wenn er, bei jeder derselben, darauf aufmerksam gemacht wird, daß alle Übel, Drangsale und Leiden des Lebens, selbst Bedrohung mit dem Tode, die ihn darüber, daß er seiner Pflicht treu gehorcht, treffen mögen, ihm doch das Bewußtsein, über sie alle erhoben und Meister zu sein, nicht rauben können, so liegt ihm nun die Frage ganz nahe: was ist das in dir, was sich getrauen darf, mit allen Kräften der Natur in dir und um dich in Kampf zu treten und sie, wenn sie mit deinen sittlichen Grundsätzen in Streit kommen, zu besiegen? Wenn diese Frage, deren Auflösung das Vermögen der spekulativen Vernunft gänzlich übersteigt und die sich dennoch von selbst einstellt, ans Herz gelegt wird, so muß selbst die Unbegreiflichkeit in diesem Selbsterkenntnisse der Seele eine Erhebung geben, die sie zum Heilighalten ihrer Pflicht nur desto stärker belebt, je mehr sie angefochten wird. 
	
	\subsection*{tg489.2.17} 
	\textbf{Source : }Die Metaphysik der Sitten/Fußnoten\\  
	
	\noindent\textbf{Paragraphe : }
	
	8 Weil die Entthronung eines Monarchen doch auch als freiwillige Ablegung der Krone und Niederlegung seiner Gewalt, mit Zurückgebung derselben an das Volk, gedacht werden kann, oder auch als eine, ohne Vergreifung an der höchsten Person, vorgenommene Verlassung derselben, wodurch sie in den Privatstand versetzt werden würde, so hat das Verbrechen des Volks, welches sie erzwang, doch noch wenigstens den Vorwand des Notrechts (casus necessitatis) für sich, niemals aber das mindeste Recht, ihn, das Oberhaupt, wegen der vorigen Verwaltung zu strafen; weil alles, was er vorher in der Qualität eines Oberhaupts tat, als äußerlich rechtmäßig geschehen angesehen werden muß, und er selbst, als Quell der Gesetze betrachtet, nicht unrecht tun kann. Unter allen Greueln einer Staatsumwälzung durch Aufruhr ist selbst die Ermordung des Monarchen noch nicht das Ärgste: denn noch kann man sich vorstellen, sie geschehe vom Volk aus Furcht, er könne, wenn er am Leben bleibt, sich wieder ermannen, und jenes die verdiente Strafe fühlen lassen, und solle also nicht eine Verfügung der Strafgerechtigkeit, sondern bloß der Selbsterhaltung sein. Die formale Hinrichtung ist es, was die mit Ideen des Menschenrechts erfüllete Seele mit einem Schaudern ergreift, das man wiederholentlich fühlt, so bald und so oft man sich diesen Auftritt denkt, wie das Schicksal Karls I. oder Ludwigs XVI. Wie erklärt man sich aber dieses Gefühl, was hier nicht ästhetisch (ein Mitgefühl, Wirkung der Einbildungskraft, die sich in die \match{Stelle} des Leidenden versetzt), sondern moralisch, der gänzlichen Umkehrung aller Rechtsbegriffe ist? Es wird als Verbrechen, was ewig bleibt, und nie ausgetilgt werden kann (crimen immortale, inexpiabile), angesehen, und scheint demjenigen ähnlich zu sein, was die Theologen diejenige Sünde nennen, welche weder in dieser noch in jener Welt vergeben werden kann. Die Erklärung dieses Phänomens im menschlichen Gemüte scheint aus folgenden Reflexionen über sich selbst, die selbst auf die staatsrechtlichen Prinzipien ein Licht werfen, hervorzugehen. 
	
	\unnumberedsection{Stern (1)} 
	\subsection*{tg445.2.14} 
	\textbf{Source : }Die Metaphysik der Sitten/Erster Teil. Metaphysische Anfangsgründe der Rechtslehre/Anhang erläutender Bemerkungen zu den metaphysischen Anhangsgründen der Rechtslehre\\  
	
	\noindent\textbf{Paragraphe : }Ob nun jener Begriff »als neues Phänomen am juristischen Himmel« eine stella mirabilis (eine bis zum \match{Stern} erster Größe wachsende, vorher nie gesehene, allmählich aber wieder verschwindende,  vielleicht einmal wiederkehrende Erscheinung), oder bloß eine Sternschnuppe sei? das soll jetzt untersucht werden. 
	
	\unnumberedsection{Stufe (5)} 
	\subsection*{tg450.2.6} 
	\textbf{Source : }Die Metaphysik der Sitten/Zweiter Teil. Metaphysische Anfangsgründe der Tugendlehre/Einleitung/II. Erörterung des Begriffs von einem Zwecke, der zugleich Pflicht ist\\  
	
	\noindent\textbf{Paragraphe : }Die Tugendpflicht ist von der Rechtspflicht wesentlich darin unterschieden: daß zu dieser ein äußerer Zwang moralisch-möglich ist, jene aber auf dem freien Selbstzwange allein beruht. – Für endliche, heilige, Wesen (die zur Verletzung der Pflicht gar nicht einmal versucht werden können) gibt es keine Tugendlehre, sondern bloß Sittenlehre, welche letztere eine Autonomie der praktischen Vernunft ist, indessen daß die erstere zugleich eine Autokratie derselben, d.i. ein, wenn gleich nicht unmittelbar wahrgenommenes, doch aus dem sittlichen kategorischen Imperativ richtig geschlossenes Bewußtsein des Vermögens enthält, über seine dem Gesetz widerspenstigen Neigungen Meister zu werden: so daß die menschliche Moralität in ihrer höchsten \match{Stufe} doch nichts mehr als Tugend sein kann; selbst wenn sie ganz rein (vom Einflusse aller fremdartigen Triebfeder als der der Pflicht völlig frei) wäre, da sie dann gemeiniglich als ein Ideal (dem man stets sich annähern müsse) unter dem Namen des Weisen dichterisch personifiziert wird. 
	
	\subsection*{tg471.2.37} 
	\textbf{Source : }Die Metaphysik der Sitten/Zweiter Teil. Metaphysische Anfangsgründe der Tugendlehre/I. Ethische Elementarlehre/I. Teil. Von den Pflichten gegen sich selbst überhaupt/Erstes Buch. Von den vollkommenen Pflichten gegen sich selbst/Erstes Hauptstück. Die Pflicht des Menschen gegen sich selbst, als einem animalischen Wesen\\  
	
	\noindent\textbf{Paragraphe : }Die Geschlechtsneigung wird auch Liebe (in der engsten Bedeutung des Worts) genannt und ist in der Tat die größte Sinnenlust, die an einem Gegenstande möglich ist; – nicht bloß sinnliche Lust, wie an Gegenständen, die in der bloßen Reflexion über sie gefallen (da die Empfänglichkeit für sie Geschmack heißt), sondern die Lust aus dem Genusse einer anderen Person, die also zum Begehrungsvermögen und zwar der höchsten \match{Stufe} desselben, der Leidenschaft, gehört. Sie kann aber weder zur Liebe des Wohlgefallens, noch der des Wohlwollens gezählt werden (denn beide halten eher vom fleischlichen Genuß ab), sondern ist eine Lust von besonderer Art (sui generis) und das Brünstigsein hat mit der moralischen Liebe eigentlich nichts gemein, wiewohl sie mit der letzteren, wenn die praktische Vernunft mit ihren einschränkenden Bedingungen hinzu kommt, in enge Verbindung treten kann. 
	
	\subsection*{tg481.2.64} 
	\textbf{Source : }Die Metaphysik der Sitten/Zweiter Teil. Metaphysische Anfangsgründe der Tugendlehre/I. Ethische Elementarlehre/II. Teil. Von den Tugendpflichten gegen andere/Erstes Hauptstück. Von den Pflichten gegen andere, bloß als Menschen/Erster Abschnitt. Von der Liebespflicht gegen andere Menschen\\  
	
	\noindent\textbf{Paragraphe : }Was aber die Intension, d.i. den Grad der Verbindlichkeit zu dieser Tugend betrifft, so ist er nach dem Nutzen, den der Verpflichtete aus der Wohltat gezogen hat, und der Uneigennützigkeit, mit der ihm diese erteilt worden, zu schätzen. Der mindeste Grad ist, gleiche Dienstleistungen dem Wohltäter, der dieser empfänglich (noch lebend) ist, und, wenn er es nicht ist, anderen zu erweisen: eine empfangene Wohltat nicht wie eine Last, deren man gern überhoben sein möchte (weil der so Begünstigte gegen seinen Gönner eine \match{Stufe} niedriger steht und dies dessen Stolz kränkt), anzusehen: sondern selbst die Veranlassung dazu als moralische Wohltat aufzunehmen, d.i. als gegebene Gelegenheit, diese Tugend der Menschenliebe, welche, mit der Innigkeit der wohlwollenden Gesinnung zugleich, Zärtlichkeit des Wohlwollens (Aufmerksamkeit auf den kleinsten Grad derselben in der Pflichtvorstellung) ist, zu verbinden und so die Menschenliebe zu kultivieren. 
	
	\subsection*{tg481.2.78} 
	\textbf{Source : }Die Metaphysik der Sitten/Zweiter Teil. Metaphysische Anfangsgründe der Tugendlehre/I. Ethische Elementarlehre/II. Teil. Von den Tugendpflichten gegen andere/Erstes Hauptstück. Von den Pflichten gegen andere, bloß als Menschen/Erster Abschnitt. Von der Liebespflicht gegen andere Menschen\\  
	
	\noindent\textbf{Paragraphe : }Dankbarkeit ist eigentlich nicht Gegenliebe des Verpflichteten gegen den Wohltäter, sondern Achtung vor demselben. Denn der allgemeinen Nächstenliebe kann und muß Gleichheit der Pflichten zum Grunde gelegt werden; in der Dankbarkeit aber steht der Verpflichtete um eine \match{Stufe} niedriger als sein Wohltäter. Sollte das nicht die Ursache so mancher Undankbarkeit sein, nämlich der Stolz,  einen über sich zu sehen; der Widerwille, sich nicht in völlige Gleichheit (was die Pflichtverhältnisse betrifft) mit ihm setzen zu können? 
	
	\subsection*{tg484.2.7} 
	\textbf{Source : }Die Metaphysik der Sitten/Zweiter Teil. Metaphysische Anfangsgründe der Tugendlehre/I. Ethische Elementarlehre/II. Teil. Von den Tugendpflichten gegen andere/Beschluß der Elementarlehre. Von der innigsten Vereinigung der Liebe mit der Achtung in der Freundschaft\\  
	
	\noindent\textbf{Paragraphe : }Ein Freund in der Not, wie erwünscht ist er nicht (wohl zu verstehen, wenn er ein tätiger, mit eigenem Aufwande hülfreicher Freund ist)? Aber es ist doch auch eine große Last, sich an anderer ihrem Schicksal angekettet und mit fremden Bedürfnis beladen zu fühlen. – Die Freundschaft kann also nicht eine auf wechselseitigen Vorteil abgezweckte Verbindung, sondern diese muß rein moralisch sein, und der Beistand, auf den jeder von beiden von dem anderen im Falle der Not rechnen darf, muß nicht als Zweck und Bestimmungsgrund zu derselben – dadurch würde er die Achtung des andern Teils verlieren – sondern kann nur als äußere Bezeichnung des inneren herzlich gemeinten Wohlwollens, ohne es doch auf die Probe, als die immer gefährlich ist, ankommen zu lassen, gemeint sein, indem ein jeder großmütig den anderen dieser Last zu überheben, sie für sich allein zu tragen, ja ihm sie gänzlich zu verhehlen bedacht ist, sich aber immer doch damit schmeicheln kann, daß im Falle der Not er auf den Beistand des andern sicher würde rechnen können. Wenn aber einer von dem andern eine Wohltat annimmt, so kann er wohl vielleicht auf Gleichheit in der Liebe, aber nicht in der Achtung rechnen, denn er sieht sich offenbar eine \match{Stufe} niedriger, verbindlich zu sein und nicht gegenseitig verbinden zu können. – Freundschaft ist, bei der Süßigkeit der Empfindung des bis zum Zusammenschmelzen in eine Person sich annähernden wechselseitigen Besitzes, doch zugleich etwas so Zartes (teneritas amicitiae), daß, wenn man sie auf Gefühle beruhen läßt, und dieser wechselseitigen Mitteilung und Ergebung nicht Grundsätze oder das Gemeinmachen verhütende, und die Wechselliebe durch Foderungen der Achtung einschränkende  Regeln unterlegt, sie keinen Augenblick vor Unterbrechungen sicher ist; dergleichen unter unkultivierten Personen gewöhnlich sind, ob sie zwar darum eben nicht immer Trennung bewirken (denn Pöbel schlägt sich und Pöbel verträgt sich); sie können von einander nicht lassen, aber sich auch nicht unter einander einigen, weil das Zanken selbst ihnen Bedürfnis ist, um die Süßigkeit der Eintracht in der Versöhnung zu schmecken. – Auf alle Fälle aber kann die Liebe in der Freundschaft nicht Affekt sein; weil dieser in der Wahl blind und in der Fortsetzung verrauchend ist. 
	
	\unnumberedsection{Sturm (1)} 
	\subsection*{tg471.2.23} 
	\textbf{Source : }Die Metaphysik der Sitten/Zweiter Teil. Metaphysische Anfangsgründe der Tugendlehre/I. Ethische Elementarlehre/I. Teil. Von den Pflichten gegen sich selbst überhaupt/Erstes Buch. Von den vollkommenen Pflichten gegen sich selbst/Erstes Hauptstück. Die Pflicht des Menschen gegen sich selbst, als einem animalischen Wesen\\  
	
	\noindent\textbf{Paragraphe : }Wer sich die Pocken einimpfen zu lassen beschließt, wagt sein Leben aufs Ungewisse: ob er es zwar tut, um sein Leben zu erhalten, und ist so fern in einem weit bedenklicheren Fall des Pflichtgesetzes, als der Seefahrer, welcher doch wenigstens den \match{Sturm} nicht macht, dem er sich anvertraut, statt dessen jener die Krankheit, die ihn in Todesgefahr bringt, sich selbst zuzieht. Ist also die Pockeninokulation erlaubt? 
	
	\unnumberedsection{Umlauf (1)} 
	\subsection*{tg437.2.81} 
	\textbf{Source : }Die Metaphysik der Sitten/Erster Teil. Metaphysische Anfangsgründe der Rechtslehre/1. Teil. Das Privatrecht vom äußeren Mein und Dein überhaupt/2. Hauptstück. Von der Art, etwas Äußeres zu erwerben/3. Abschnitt. Von dem auf dingliche Art persönlichen Recht\\  
	
	\noindent\textbf{Paragraphe : }Der intellektuelle Begriff, dem der empirische vom Gelde untergelegt ist, ist also der von einer Sache, die, im \match{Umlauf} des Besitzes begriffen (permutatio publica), den Preis aller anderen Dinge (Waren) bestimmt, unter welche letztere so gar Wissenschaften, so fern sie anderen nicht umsonst gelehrt werden, gehören: dessen Menge also in einem Volk die Begüterung (opulentia) desselben ausmacht. Denn Preis (pretium) ist das öffentliche Urteil über den Wert (valor) einer Sache, in Verhältnis auf die proportionierte Menge desjenigen, was das allgemeine stellvertretende Mittel der gegenseitigen Vertauschung des Fleißes (des Umlaufs) ist. – Daher werden, wo der Verkehr groß ist, weder Gold noch Kupfer für eigentliches Geld, sondern nur für Ware gehalten; weil von dem ersteren zu wenig, vom anderen zu viel da ist, um es leicht in Umlauf zu bringen, und dennoch in so kleinen Teilen zu haben, als zum Umsatz gegen Ware, oder eine Menge derselben im kleinsten Erwerb nötig ist. Silber (weniger oder mehr mit Kupfer versetzt) wird daher im großen Verkehr der Welt für das eigentliche Material des Geldes und den Maßstab der Berechnung aller Preise genommen; die übrigen Metalle (noch viel mehr also die unmetallischen Materien) können nur in einem Volk von kleinem Verkehr statt finden. – Die erstern beiden, wenn sie nicht bloß gewogen, sondern auch gestempelt, d.i. mit einem Zeichen, für wie viel sie gelten sollen, versehen worden, sind gesetzliches Geld, d.i. Münze. 
	
	\unnumberedsection{Verwerfung (2)} 
	\subsection*{tg437.2.11} 
	\textbf{Source : }Die Metaphysik der Sitten/Erster Teil. Metaphysische Anfangsgründe der Rechtslehre/1. Teil. Das Privatrecht vom äußeren Mein und Dein überhaupt/2. Hauptstück. Von der Art, etwas Äußeres zu erwerben/3. Abschnitt. Von dem auf dingliche Art persönlichen Recht\\  
	
	\noindent\textbf{Paragraphe : }
	Geschlechtsgemeinschaft (commercium sexuale) ist der wechselseitige Gebrauch, den ein Mensch von eines anderen Geschlechtsorganen und Vermögen macht (usus membrorum et facultatum sexualium alterius), und entweder  ein natürlicher (wodurch seines Gleichen erzeugt werden kann), oder unnatürlicher Gebrauch, und dieser entweder an einer Person ebendesselben Geschlechts, oder einem Tiere von einer anderen als der Menschen-Gattung: welche Übertretungen der Gesetze, unnatürliche Laster (crimina carnis contra naturam), die auch unnennbar heißen, als Läsion der Menschheit in unserer eigenen Person, durch gar keine Einschränkungen und Ausnahmen wider die gänzliche \match{Verwerfung} gerettet werden können. 
	
	\subsection*{tg438.2.17} 
	\textbf{Source : }Die Metaphysik der Sitten/Erster Teil. Metaphysische Anfangsgründe der Rechtslehre/1. Teil. Das Privatrecht vom äußeren Mein und Dein überhaupt/2. Hauptstück. Von der Art, etwas Äußeres zu erwerben/Episodischer Abschnitt. Von der idealen Erwerbung eines äußeren Gegenstandes der Willkür\\  
	
	\noindent\textbf{Paragraphe : }Also sind die Testamente auch nach dem bloßen Naturrecht gültig (sunt iuris naturae); welche Behauptung aber so zu verstehen ist, daß sie fähig und würdig sein, im bürgerlichen Zustande (wenn dieser dereinst eintritt) eingeführt und sanktioniert zu werden. Denn nur dieser (der allgemeine Wille in demselben) bewahrt den Besitz der Verlassenschaft während dessen, daß diese zwischen der Annahme und der \match{Verwerfung} schwebt, und eigentlich keinem angehört. 
	
	\unnumberedsection{Warme (1)} 
	\subsection*{tg481.2.69} 
	\textbf{Source : }Die Metaphysik der Sitten/Zweiter Teil. Metaphysische Anfangsgründe der Tugendlehre/I. Ethische Elementarlehre/II. Teil. Von den Tugendpflichten gegen andere/Erstes Hauptstück. Von den Pflichten gegen andere, bloß als Menschen/Erster Abschnitt. Von der Liebespflicht gegen andere Menschen\\  
	
	\noindent\textbf{Paragraphe : }
	Mitfreude und Mitleid (sympathia moralis) sind zwar sinnliche Gefühle einer (darum ästhetisch zu nennenden) Lust oder Unlust an dem Zustande des Vergnügens so wohl als Schmerzens anderer (Mitgefühl, teilnehmende Empfindung), wozu schon die Natur in den Menschen die Empfänglichkeit gelegt hat. Aber diese als Mittel zu Beförderung des tätigen und vernünftigen Wohlwollens zu gebrauchen, ist noch eine besondere, obzwar nur bedingte, Pflicht, unter dem Namen der Menschlichkeit (humanitas); weil hier der Mensch nicht bloß als vernünftiges Wesen, sondern  auch als mit Vernunft begabtes Tier betrachtet wird. Diese kann nun in dem Vermögen und Willen, sich einander in Ansehung seiner Gefühle mitzuteilen (humanitas practica), oder bloß in der Empfänglichkeit für das gemeinsame Gefühl des Vergnügens oder Schmerzens (humanitas aesthetica), was die Natur selbst gibt, gesetzt werden. Das erstere ist frei und wird daher teilnehmend genannt (communio sentiendi liberalis) und gründet sich auf praktische Vernunft; das zweite ist unfrei (communio sentiendi illiberalis, servilis) und kann mitteilend (wie die der \match{Wärme} oder ansteckender Krankheiten), auch Mitleidenschaft heißen; weil sie sich unter nebeneinander lebenden Menschen natürlicher Weise verbreitet. Nur zu dem ersten gibt's Verbindlichkeit. 
	
	\unnumberedsection{Waßer (1)} 
	\subsection*{tg481.2.7} 
	\textbf{Source : }Die Metaphysik der Sitten/Zweiter Teil. Metaphysische Anfangsgründe der Tugendlehre/I. Ethische Elementarlehre/II. Teil. Von den Tugendpflichten gegen andere/Erstes Hauptstück. Von den Pflichten gegen andere, bloß als Menschen/Erster Abschnitt. Von der Liebespflicht gegen andere Menschen\\  
	
	\noindent\textbf{Paragraphe : }Wenn von Pflichtgesetzen (nicht von Naturgesetzen) die Rede ist und zwar im äußeren Verhältnis der Menschen gegen einander, so betrachten wir uns in einer moralischen (intelligibelen) Welt, in welcher, nach der Analogie mit der physischen, die Verbindung vernünftiger Wesen (auf Erden) durch Anziehung und Abstoßung bewirkt wird. Vermöge des Prinzips der Wechselliebe sind sie angewiesen, sich einander beständig zu nähern, durch das der Achtung, die sie einander schuldig sind, sich im Abstande von einander zu erhalten, und, sollte eine dieser großen sittlichen Kräfte sinken: »so würde dann das Nichts (der Immoralität) mit aufgesperrtem Schlund der (moralischen) Wesen ganzes Reich, wie einen Tropfen \match{Wasser} trinken« (wenn ich mich hier der Worte Hallers, nur in einer andern Beziehung, bedienen darf). 
	
	\unnumberedsection{Welt (15)} 
	\subsection*{tg437.2.29} 
	\textbf{Source : }Die Metaphysik der Sitten/Erster Teil. Metaphysische Anfangsgründe der Rechtslehre/1. Teil. Das Privatrecht vom äußeren Mein und Dein überhaupt/2. Hauptstück. Von der Art, etwas Äußeres zu erwerben/3. Abschnitt. Von dem auf dingliche Art persönlichen Recht\\  
	
	\noindent\textbf{Paragraphe : }Denn da das Erzeugte eine Person ist, und es unmöglich ist, sich von der Erzeugung eines mit Freiheit begabten Wesens durch eine physische Operation einen Begriff zu machen
	
	
	5
	: so ist es eine in praktischer Hinsicht ganz  richtige und auch notwendige Idee, den Akt der Zeugung als einen solchen anzusehen, wodurch wir eine Person ohne ihre Einwilligung auf die \match{Welt} gesetzt, und eigenmächtig in sie herüber gebracht haben; für welche Tat auf den Eltern nun auch eine Verbindlichkeit haftet, sie, so viel in ihren Kräften ist, mit diesem ihrem Zustande zufrieden zu machen. – Sie können ihr Kind nicht gleichsam als ihr Gemächsel (denn ein solches kann kein mit Freiheit begabtes Wesen sein) und als ihr Eigentum zerstören oder es auch nur dem Zufall überlassen, weil an ihm nicht bloß ein Weltwesen, sondern auch ein Weltbürger in einen Zustand herüber zogen, der ihnen nun auch nach Rechtsbegriffen nicht gleichgültig sein kann. 
	
	\subsection*{tg445.2.56} 
	\textbf{Source : }Die Metaphysik der Sitten/Erster Teil. Metaphysische Anfangsgründe der Rechtslehre/Anhang erläutender Bemerkungen zu den metaphysischen Anhangsgründen der Rechtslehre\\  
	
	\noindent\textbf{Paragraphe : }Wenn nun gewisse andächtige und gläubige Seelen, um der Gnade teilhaftig zu werden, welche die Kirche den Gläubigen auch nach dieser ihrem Tode zu erzeigen verspricht, eine Stiftung auf ewige Zeiten errichten, durch welche gewisse Ländereien derselben nach ihrem Tode ein Eigentum der Kirche werden sollen, und der Staat an diesem oder jenem Teil, oder gar ganz, sich der Kirche lehnspflichtig macht, um durch Gebete, Ablässe und Büßungen, durch welche die dazu bestellten Diener derselben (die Geistlichen) das Los in der anderen \match{Welt} ihnen vorteilhaft zu machen verheißen: so ist eine solche vermeintlich auf ewige Zeiten gemachte Stiftung keineswegs auf ewig begründet, sondern der Staat kann diese Last, die ihm von der Kirche aufgelegt worden, abwerfen, wenn er will. – Denn die Kirche selbst ist als ein bloß auf Glauben errichtetes Institut, und, wenn die Täuschung aus dieser Meinung durch Volksaufklärung verschwunden ist, so fällt auch die darauf gegründete furchtbare Gewalt des Klerus weg, und der Staat bemächtigt sich mit vollem Rechte des angemaßten Eigentums der Kirche: nämlich des durch Vermächtnisse an sie verschenkten Bodens; wiewohl die Lehnsträger des bis dahin bestandenen Instituts für ihre Lebenszeit schadenfrei gehalten zu werden aus ihrem Rechte fordern können. 
	
	\subsection*{tg471.2.11} 
	\textbf{Source : }Die Metaphysik der Sitten/Zweiter Teil. Metaphysische Anfangsgründe der Tugendlehre/I. Ethische Elementarlehre/I. Teil. Von den Pflichten gegen sich selbst überhaupt/Erstes Buch. Von den vollkommenen Pflichten gegen sich selbst/Erstes Hauptstück. Die Pflicht des Menschen gegen sich selbst, als einem animalischen Wesen\\  
	
	\noindent\textbf{Paragraphe : }a) Die Selbstentleibung ist ein Verbrechen (Mord). Dieses kann nun zwar auch als Übertretung seiner Pflicht gegen andere Menschen (Eheleute, Eltern gegen Kinder, des Untertans gegen seine Obrigkeit, oder seine Mitbürger, endlich auch gegen Gott betrachtet werden, dessen uns anvertrauten Posten in der \match{Welt} der Mensch verläßt, ohne davon abgerufen zu sein) betrachtet werden; – aber hier ist nur die Rede von Verletzung einer Pflicht gegen sich selbst, ob nämlich, wenn ich auch alle jene Rücksichten bei Seite setzte, der Mensch doch zur Erhaltung seines Lebens, bloß durch seine Qualität als Person verbunden sei, und hierin eine (und zwar strenge) Pflicht gegen sich selbst anerkennen müsse. 
	
	\subsection*{tg471.2.13} 
	\textbf{Source : }Die Metaphysik der Sitten/Zweiter Teil. Metaphysische Anfangsgründe der Tugendlehre/I. Ethische Elementarlehre/I. Teil. Von den Pflichten gegen sich selbst überhaupt/Erstes Buch. Von den vollkommenen Pflichten gegen sich selbst/Erstes Hauptstück. Die Pflicht des Menschen gegen sich selbst, als einem animalischen Wesen\\  
	
	\noindent\textbf{Paragraphe : }Der Persönlichkeit kann der Mensch sich nicht entäußern, so lange von Pflichten die Rede ist, folglich so lange er lebt, und es ist ein Widerspruch, die Befugnis zu haben, sich aller Verbindlichkeit zu entziehen, d.i. frei so zu handeln, als ob es zu dieser Handlung gar keiner Befugnis bedürfte. Das Subjekt der Sittlichkeit in seiner eigenen Person zernichten, ist eben so viel, als die Sittlichkeit selbst ihrer Existenz nach, so viel an ihm ist, aus der \match{Welt} vertilgen, wel che doch Zweck an sich selbst ist; mithin über sich als bloßes Mittel zu ihm beliebigen Zweck zu disponieren, heißt die Menschheit in seiner Person (homo noumenon) abwürdigen, der doch der Mensch (homo phaenomenon) zur Erhaltung anvertrauet war. 
	
	\subsection*{tg471.2.29} 
	\textbf{Source : }Die Metaphysik der Sitten/Zweiter Teil. Metaphysische Anfangsgründe der Tugendlehre/I. Ethische Elementarlehre/I. Teil. Von den Pflichten gegen sich selbst überhaupt/Erstes Buch. Von den vollkommenen Pflichten gegen sich selbst/Erstes Hauptstück. Die Pflicht des Menschen gegen sich selbst, als einem animalischen Wesen\\  
	
	\noindent\textbf{Paragraphe : }Daß ein solcher naturwidrige Gebrauch (also Mißbrauch) seiner Geschlechtseigenschaft eine und zwar der Sittlichkeit im höchsten Grad widerstreitende Verletzung der Pflicht wider sich selbst sei, fällt jedem, zugleich mit dem Gedanken von demselben, so fort auf, erregt eine Abkehrung von diesem Gedanken, in der Maße, daß selbst die Nennung  eines solchen Lasters bei seinem eigenen Namen für unsittlich gehalten wird; welches, bei dem des Selbstmords, nicht geschieht, den man, mit allen seinen Greueln (in einer species facti) der \match{Welt} vor Augen zu legen im mindesten kein Bedenken trägt; gleich als ob der Mensch überhaupt sich beschämt fühle, einer solchen ihn selbst unter das Vieh herabwürdigenden Behandlung seiner eigenen Person fähig zu sein: so daß selbst die erlaubte (an sich freilich bloß tierische) körperliche Gemeinschaft beider Geschlechter in der Ehe im gesitteten Umgange viel Feinheit veranlaßt und erfodert, um einen Schleier darüber zu werfen, wenn davon gesprochen werden soll. 
	
	\subsection*{tg472.2.14} 
	\textbf{Source : }Die Metaphysik der Sitten/Zweiter Teil. Metaphysische Anfangsgründe der Tugendlehre/I. Ethische Elementarlehre/I. Teil. Von den Pflichten gegen sich selbst überhaupt/Erstes Buch. Von den vollkommenen Pflichten gegen sich selbst\\  
	
	\noindent\textbf{Paragraphe : }Es ist merkwürdig, daß die Bibel das erste Verbrechen, wodurch das Böse in die \match{Welt} gekommen ist, nicht vom Brudermorde (Kains), sondern von der ersten Lüge datiert (weil gegen jenen sich doch die Natur empört), und als den Urheber alles Bösen den Lügner von Anfang und den Vater der Lügen nennt; wiewohl die Vernunft von diesem Hange der Menschen zur Gleisnerei (esprit fourbe), der  doch vorher gegangen sein muß, keinen Grund weiter angeben kann; weil ein Akt der Freiheit nicht (gleich einer physischen Wirkung) nach dem Naturgesetz des Zusammenhanges der Wirkung und ihrer Ursache, welche insgesamt Erscheinungen sind, deduziert und erklärt werden kann. 
	
	\subsection*{tg477.2.10} 
	\textbf{Source : }Die Metaphysik der Sitten/Zweiter Teil. Metaphysische Anfangsgründe der Tugendlehre/I. Ethische Elementarlehre/I. Teil. Von den Pflichten gegen sich selbst überhaupt/2. Buch: Die Pflichten gegen sich selbst/Erster Abschnitt. Von der Pflicht gegen sich selbst in Entwickelung und Vermehrung seiner Naturvollkommenheit, d.i. in pragmatischer Absicht\\  
	
	\noindent\textbf{Paragraphe : }Auf welche von diesen physischen Vollkommenheiten vorzüglich, und in welcher Proportion, in Vergleichung gegen einander, sie sich zum Zweck zu machen es Pflicht des Menschen gegen sich selbst sei, bleibt ihrer eigenen vernünftigen Überlegung, in Ansehung der Lust zu einer gewissen Lebensart und zugleich der Schätzung seiner dazu erforderlichen Kräfte, überlassen, um darunter zu wählen (z.B. ob es ein Handwerk, oder der Kaufhandel, oder die  Gelehrsamkeit sein sollte). Denn, abgesehen von dem Bedürfnis der Selbsterhaltung, welches an sich keine Pflicht begründen kann, ist es Pflicht des Menschen gegen sich selbst, ein der \match{Welt} nützliches Glied zu sein, weil dieses auch zum Wert der Menschheit in seiner eigenen Person gehört, die er also nicht abwürdigen soll. 
	
	\subsection*{tg481.2.51} 
	\textbf{Source : }Die Metaphysik der Sitten/Zweiter Teil. Metaphysische Anfangsgründe der Tugendlehre/I. Ethische Elementarlehre/II. Teil. Von den Tugendpflichten gegen andere/Erstes Hauptstück. Von den Pflichten gegen andere, bloß als Menschen/Erster Abschnitt. Von der Liebespflicht gegen andere Menschen\\  
	
	\noindent\textbf{Paragraphe : }Wie weit soll man den Aufwand seines Vermögens im Wohltun treiben? Doch wohl nicht bis dahin, daß man zuletzt selbst anderer Wohltätigkeit bedürftig würde. Wie viel ist die Wohltat wert, die man mit kalter Hand (im Abscheiden aus der \match{Welt} durch ein Testament) beweiset? – Kann derjenige, welcher eine ihm durchs Landesgesetz erlaubte Obergewalt über einen übt, dem er die Freiheit raubt, nach seiner eigenen Wahl glücklich zu sein (seinem Erbuntertan eines Guts), kann, sage ich, dieser sich als Wohltäter ansehen, wenn er nach seinen eigenen Begriffen  von Glückseligkeit für ihn gleichsam väterlich sorgt? Oder ist nicht vielmehr die Ungerechtigkeit, einen seiner Freiheit zu berauben, etwas der Rechtspflicht überhaupt so Widerstreitendes, daß, unter dieser Bedingung auf die Wohltätigkeit der Herrschaft rechnend, sich hinzugeben, die größte Wegwerfung der Menschheit für den sein würde, der sich dazu freiwillig verstände, und die größte Vorsorge der Herrschaft für den letzteren gar keine Wohltätigkeit sein würde? Oder kann etwa das Verdienst mit der letzteren so groß sein, daß es gegen das Menschenrecht aufgewogen werden könnte? – Ich kann niemand nach meinen Begriffen von Glückseligkeit wohltun (außer unmündigen Kindern oder Gestörten), sondern nach jenes seinen Begriffen, dem ich eine Wohltat zu erweisen denke, indem ich ihm ein Geschenk aufdringe. 
	
	\subsection*{tg481.2.63} 
	\textbf{Source : }Die Metaphysik der Sitten/Zweiter Teil. Metaphysische Anfangsgründe der Tugendlehre/I. Ethische Elementarlehre/II. Teil. Von den Tugendpflichten gegen andere/Erstes Hauptstück. Von den Pflichten gegen andere, bloß als Menschen/Erster Abschnitt. Von der Liebespflicht gegen andere Menschen\\  
	
	\noindent\textbf{Paragraphe : }Was die Extension dieser Dankbarkeit betrifft, so geht sie nicht allein auf Zeitgenossen, sondern auch auf die Vorfahren, selbst diejenige, die man nicht mit Gewißheit namhaft machen kann. Das ist auch die Ursache, weswegen es für unanständig gehalten wird, die Alten, die als unsere Lehrer angesehen werden können, nicht nach Möglichkeit wider alle Angriffe, Beschuldigungen und Geringschätzung zu verteidigen; wobei es aber ein törichter Wahn ist, ihnen um des Altertums willen einen Vorzug in Talenten und gutem Willen  vor den Neueren, gleich als ob die \match{Welt} in kontinuierlicher Abnahme ihrer ursprünglichen Vollkommenheit nach Naturgesetzen wäre, anzudichten und alles Neue in Vergleichung damit zu verachten. 
	
	\subsection*{tg481.2.71} 
	\textbf{Source : }Die Metaphysik der Sitten/Zweiter Teil. Metaphysische Anfangsgründe der Tugendlehre/I. Ethische Elementarlehre/II. Teil. Von den Tugendpflichten gegen andere/Erstes Hauptstück. Von den Pflichten gegen andere, bloß als Menschen/Erster Abschnitt. Von der Liebespflicht gegen andere Menschen\\  
	
	\noindent\textbf{Paragraphe : }In der Tat, wenn ein anderer leidet und ich mich durch seinen Schmerz, dem ich doch nicht abhelfen kann, auch (vermittelst der Einbildungskraft) anstecken lasse, so leiden ihrer zwei; ob zwar das Übel eigentlich (in der Natur) nur Einen trifft. Es kann aber unmöglich Pflicht sein, die Übel in der \match{Welt} zu vermehren, mithin auch nicht, aus Mitleid wohl zu tun; wie dann dieses auch eine beleidigende Art des Wohltuns sein würde, indem es ein Wohlwollen ausdrückt, was sich auf den Unwürdigen bezieht und Barmherzigkeit genannt wird, unter Menschen, welche mit ihrer Würdigkeit, glücklich zu sein, eben nicht prahlen dürfen, und respektiv gegen einander gar nicht vorkommen sollte. 
	
	\subsection*{tg481.2.77} 
	\textbf{Source : }Die Metaphysik der Sitten/Zweiter Teil. Metaphysische Anfangsgründe der Tugendlehre/I. Ethische Elementarlehre/II. Teil. Von den Tugendpflichten gegen andere/Erstes Hauptstück. Von den Pflichten gegen andere, bloß als Menschen/Erster Abschnitt. Von der Liebespflicht gegen andere Menschen\\  
	
	\noindent\textbf{Paragraphe : }Würde es mit dem Wohl der \match{Welt} überhaupt nicht besser stehen, wenn alle Moralität der Menschen nur auf Rechtspflichten, doch mit der größten Gewissenhaftigkeit, eingeschränkt, das Wohlwollen aber unter die Adiaphora gezählt würde? Es ist nicht so leicht zu übersehen, welche Folge es auf die Glückseligkeit der Menschen haben dürfte. Aber in diesem Fall würde es doch wenigstens an einer großen moralischen Zierde der Welt, nämlich der Menschenliebe fehlen, welche also für sich, auch ohne die Vorteile (der Glückseligkeit) zu berechnen, die Welt als ein schönes moralisches Ganze in ihrer ganzen Vollkommenheit darzustellen erfordert wird. 
	
	\subsection*{tg484.2.18} 
	\textbf{Source : }Die Metaphysik der Sitten/Zweiter Teil. Metaphysische Anfangsgründe der Tugendlehre/I. Ethische Elementarlehre/II. Teil. Von den Tugendpflichten gegen andere/Beschluß der Elementarlehre. Von der innigsten Vereinigung der Liebe mit der Achtung in der Freundschaft\\  
	
	\noindent\textbf{Paragraphe : }Es ist Pflicht, so wohl gegen sich selbst, als auch gegen andere, mit seinen sittlichen Vollkommenheiten unter einander Verkehr zu treiben (officium commercii, sociabilitas); sich nicht zu isolieren (separatistam agere); zwar sich einen unbeweglichen Mittelpunkt seiner Grundsätze zu machen, aber diesen um sich gezogenen Kreis doch auch als einen, der den Teil von einem allbefassenden, der weltbürgerlichen Gesinnung, ausmacht, anzusehen; nicht eben um das \match{Welt} beste als Zweck zu befördern, sondern nur die wechselseitige, die indirekt dahin führt, die Annehmlichkeit in derselben, die Verträglichkeit, die wechselseitige Liebe und Achtung (Leutseligkeit und Wohlanständigkeit, humanitas aesthetica, et decorum) zu kultivieren, und so der Tugend die Grazien beizugesellen; welches zu bewerkstelligen selbst Tugendpflicht ist. 
	
	\subsection*{tg484.2.20} 
	\textbf{Source : }Die Metaphysik der Sitten/Zweiter Teil. Metaphysische Anfangsgründe der Tugendlehre/I. Ethische Elementarlehre/II. Teil. Von den Tugendpflichten gegen andere/Beschluß der Elementarlehre. Von der innigsten Vereinigung der Liebe mit der Achtung in der Freundschaft\\  
	
	\noindent\textbf{Paragraphe : }Es frägt sich aber hiebei: ob man auch mit Lasterhaften Umgang pflegen dürfe? Die Zusammenkunft mit ihnen kann man nicht vermeiden; man müßte denn sonst aus der \match{Welt} gehen, und selbst unser Urteil über sie ist nicht kompetent. – Wo aber das Laster ein Skandal, d.i. ein öffentlich gegebenes Beispiel der Verachtung strenger Pflichtgesetze ist, mithin Ehrlosigkeit bei sich führt: da muß, wenn gleich das Landesgesetz es nicht bestraft, der Umgang, der bis dahin statt fand, abgebrochen, oder so viel möglich gemieden werden; weil die fernere Fortsetzung desselben die Tugend um alle Ehre bringt, und sie für jeden zu Kauf stellt, der reich genug ist, um den Schmarotzer durch die Vergnügungen der Üppigkeit zu bestechen. 
	
	\subsection*{tg486.2.27} 
	\textbf{Source : }Die Metaphysik der Sitten/Zweiter Teil. Metaphysische Anfangsgründe der Tugendlehre/II. Ethische Methodenlehre/1. Abschnitt. Die ethische Didaktik\\  
	
	\noindent\textbf{Paragraphe : }3. L. Wenn du nun alle Glückseligkeit (die in der \match{Welt} möglich ist) in deiner Hand hättest, würdest du sie alle für dich behalten, oder sie auch deinen Nebenmenschen mitteilen? S. Ich würde sie mitteilen; andere auch glücklich und zufrieden machen. 
	
	\subsection*{tg486.2.32} 
	\textbf{Source : }Die Metaphysik der Sitten/Zweiter Teil. Metaphysische Anfangsgründe der Tugendlehre/II. Ethische Methodenlehre/1. Abschnitt. Die ethische Didaktik\\  
	
	\noindent\textbf{Paragraphe : }8. L. Hat die Vernunft wohl Gründe für sich, eine solche, die Glückseligkeit nach Verdienst und Schuld der  Menschen austeilende, über die ganze Natur gebietende und die \match{Welt} mit höchster Weisheit regierende Macht als wirklich anzunehmen, d.i. an Gott zu glauben? S. Ja; denn wir sehen an den Werken der Natur, die wir beurteilen können, so ausgebreitete und tiefe Weisheit, die wir uns nicht anders als durch eine unaussprechlich große Kunst eines Weltschöpfers erklären können, von welchem wir uns denn auch, was die sittliche Ordnung betrifft, in der doch die höchste Zierde der Welt besteht, eine nicht minder weise Regierung zu versprechen Ursache haben: nämlich, daß, wenn wir uns nicht selbst der Glückseligkeit unwürdig machen, welches durch Übertretung unserer Pflicht geschieht, wir auch hoffen können, ihrer teilhaftig zu werden. 
	
	\unnumberedchapter{Sciences exactes} 
	\unnumberedsection{Ablauf (3)} 
	\subsection*{tg437.2.93} 
	\textbf{Source : }Die Metaphysik der Sitten/Erster Teil. Metaphysische Anfangsgründe der Rechtslehre/1. Teil. Das Privatrecht vom äußeren Mein und Dein überhaupt/2. Hauptstück. Von der Art, etwas Äußeres zu erwerben/3. Abschnitt. Von dem auf dingliche Art persönlichen Recht\\  
	
	\noindent\textbf{Paragraphe : }Die Verwechselung des persönlichen Rechts mit dem Sachenrecht ist noch in einem anderen, unter den Verdingungsvertrag gehörigen, Falle (B, II, α), nämlich dem der Einmietung (ius incolatus), ein Stoff zu Streitigkeiten. – Es fragt sich nämlich: ist der Eigentümer, wenn er sein an jemanden vermietetes Haus (oder seinen Grund) vor \match{Ablauf} der Mietszeit an einen anderen verkauft, verbunden, die Bedingung der fortdauernden Miete dem Kaufkontrakte beizufügen, oder kann man sagen: Kauf bricht Miete (doch in einer durch den Gebrauch bestimmten Zeit der Aufkündigung)? – Im ersteren Fall hätte das Haus wirklich eine Belästigung (onus) auf sich liegend, ein Recht in dieser Sache, das der Mieter sich an derselben (dem Hause) erworben hätte; welches auch wohl geschehen kann (durch Ingrossation des Mietskontrakts auf das Haus), aber alsdenn kein bloßer Mietskontrakt sein würde, sondern wozu noch ein anderer Vertrag (dazu sich nicht viel Vermieter verstehen würden) hinzukommen müßte. Also gilt der Satz: »Kauf bricht Miete «, d.i. das volle Recht in einer Sache (das Eigentum) überwiegt alles persönliche Recht, was mit ihm  nicht zusammen bestehen kann; wobei doch die Klage aus dem Grunde des letzteren dem Mieter offen bleibt, ihn wegen des aus der Zerreißung des Kontrakts entspringenden Nachteils schadenfrei zu halten. 
	
	\subsection*{tg445.2.28} 
	\textbf{Source : }Die Metaphysik der Sitten/Erster Teil. Metaphysische Anfangsgründe der Rechtslehre/Anhang erläutender Bemerkungen zu den metaphysischen Anhangsgründen der Rechtslehre\\  
	
	\noindent\textbf{Paragraphe : }Daß jemand die Miete seines Hauses, vor \match{Ablauf} der bedungenen Zeit der Einwohnung, dem Mieter aufkündigen, und also gegen diesen, wie es scheint, sein Versprechen brechen könne, wenn er es nur zur gewöhnlichen Zeit des Verziehens, in der dazu gewohnten bürgerlich-gesetzlichen Frist, tut, scheint freilich beim ersten Anblick allen Rechten aus einem Vertrage zu widerstreiten. – Wenn aber bewiesen werden kann, daß der Mieter, da er seinen Mietskontrakt machte, wußte oder wissen  mußte: daß das ihm getane Versprechen des Vermieters, als Eigentümers, natürlicherweise (ohne daß es im Kontrakt ausdrücklich gesagt werden durfte), also stillschweigend, an die Bedingung geknüpft war: wofern dieser sein Haus binnen dieser Zeit nicht verkaufen sollte (oder es bei einem, etwa über ihn eintretenden Konkurs seinen Gläubigern überlassen müßte): so hat dieser sein schon an sich der Vernunft nach bedingtes Versprechen nicht gebrochen, und der Mieter ist, durch die ihm vor der Mietszeit geschehene Aufkündigung, an seinem Rechte nicht verkürzt worden. 
	
	\subsection*{tg445.2.30} 
	\textbf{Source : }Die Metaphysik der Sitten/Erster Teil. Metaphysische Anfangsgründe der Rechtslehre/Anhang erläutender Bemerkungen zu den metaphysischen Anhangsgründen der Rechtslehre\\  
	
	\noindent\textbf{Paragraphe : }Nun konnte der Mieter sich wohl in seinem Mietskontrakte sichern und sich ein dingliches Recht am Hause verschaffen: er durfte nämlich diesen nur auf das Haus des Vermieters, als am Grunde haftend, einschreiben (ingrossieren) lassen: alsdann konnte er durch keine Aufkündigung des Eigentümers, selbst nicht durch dessen Tod (den natürlichen oder auch den bürgerlichen, den Bankrott), vor \match{Ablauf} der abgemachten Zeit aus der Miete gesetzt werden. Wenn er es nicht tat, weil er etwa frei sein wollte, anderweitig eine Miete auf bessere Bedingungen zu schließen, oder der Eigentümer sein Haus nicht mit einem solchen Onus belegt wissen wollte, so ist daraus zu schließen: daß ein jeder von beiden in Ansehung der Zeit der Aufkündigung (die bürgerlich bestimmte Frist zu derselben ausgenommen) einen stillschweigend-bedingten Kontrakt gemacht zu haben sich bewußt war, ihn ihrer Konvenienz nach wieder aufzulösen. Die Bestätigung der Befugnis, durch den Kauf Miete zu brechen, zeigt sich auch an gewissen rechtlichen Folgerungen aus einem solchen nackten Mietskontrakte: Denn den Erben des Mieters, wenn dieser verstorben ist, wird doch nicht die Verbindlichkeit zugemutet, die Miete fortzusetzen; weil diese nur die Verbindlichkeit gegen eine gewisse Person ist, die mit dieser ihrem Tode aufhört (wobei  doch die gesetzliche Zeit der Aufkündigung immer mit in Anschlag gebracht werden muß). Eben so wenig kann auch das Recht des Mieters, als eines solchen, auch auf seine Erben, ohne einen besonderen Vertrag übergehen; so wie er auch beim Leben beider Teile, ohne ausdrückliche Übereinkunft, keinen Aftermieter zu setzen befugt ist. 
	
	\unnumberedsection{Ableitung (2)} 
	\subsection*{tg431.2.49} 
	\textbf{Source : }Die Metaphysik der Sitten/Erster Teil. Metaphysische Anfangsgründe der Rechtslehre/Einleitung in die Rechtslehre\\  
	
	\noindent\textbf{Paragraphe : }
	Also sind obstehende drei klassische Formeln zugleich Einteilungsprinzipien des Systems der Rechtspflichten in innere, äußere und in diejenigen, welche die \match{Ableitung} der letzteren vom Prinzip der ersteren durch Subsumtion enthalten. 
	
	\subsection*{tg439.2.29} 
	\textbf{Source : }Die Metaphysik der Sitten/Erster Teil. Metaphysische Anfangsgründe der Rechtslehre/1. Teil. Das Privatrecht vom äußeren Mein und Dein überhaupt/3. Hauptstück. Von der subjektiv-bedingten Erwerbung durch den Ausspruch einer öffentlichen Gerichtsbarkeit\\  
	
	\noindent\textbf{Paragraphe : }Wenn gefragt wird, was (im Naturzustande) unter Menschen, nach Prinzipien der Gerechtigkeit im Verkehr derselben untereinander (iustitia commutativa) in Erwerbung äußerer Sachen an sich Rechtens sei, so muß man eingestehen: daß, wer dieses zur Absicht hat, durchaus nötig habe, noch nachzuforschen, ob die Sache, die er erwerben will, nicht schon einem anderen angehöre; nämlich, wenn er gleich die formalen Bedingungen der \match{Ableitung} der Sache von dem Seinen des anderen genau beobachtet (das Pferd auf dem Markte ordentlich erhandelt) hat, er dennoch höchstens nur ein persönliches Recht in Ansehung einer Sache (ius ad rem) habe erwerben können, so lange es ihm noch unbekannt ist, ob nicht ein anderer (als der Verkäufer) der wahre Eigentümer derselben sei; so daß, wenn sich einer vorfindet, der sein vorhergehendes Eigentum daran dokumentieren könnte, dem vermeinten neuen Eigentümer nichts übrig bliebe, als den Nutzen, so er, als ehrlicher Besitzer, bisher daraus gezogen hat, bis auf diesen Augenblick rechtmäßig genossen zu haben. – Da nun in der Reihe der von einander ihr Recht ableitenden sich dünkenden Eigentümer den schlechthin ersten (Stammeigentümer) auszufinden mehrenteils unmöglich ist: so kann kein Verkehr mit äußeren Sachen, so gut er auch mit den formalen Bedingungen dieser Art von Gerechtigkeit (iustitia  commutativa) übereinstimmen möchte, einen sicheren Erwerb gewähren. 
	
	\unnumberedsection{Anfang (3)} 
	\subsection*{tg435.2.18} 
	\textbf{Source : }Die Metaphysik der Sitten/Erster Teil. Metaphysische Anfangsgründe der Rechtslehre/1. Teil. Das Privatrecht vom äußeren Mein und Dein überhaupt/2. Hauptstück. Von der Art, etwas Äußeres zu erwerben/1. Abschnitt. Vom Sachrecht\\  
	
	\noindent\textbf{Paragraphe : }Die Besitznehmung (apprehensio), als der \match{Anfang} der Inhabung einer körperlichen Sache im Raume (possessionis physicae), stimmt unter keiner, anderen Bedingung mit dem Gesetz der äußeren Freiheit von jedermann (mithin a priori) zusammen, als unter der der Priorität in Ansehung der Zeit, d.i. nur als erste Besitznehmung (prior apprehensio), welche ein Akt der Willkür ist. Der Wille aber, die Sache (mithin auch ein bestimmter abgeteilter Platz auf Erden) solle mein sein, d.i. die Zueignung (appropriatio) kann in einer ursprünglichen Erwerbung nicht anders als einseitig (voluntas unilateralis s. propria) sein. Die Erwerbung eines äußeren Gegenstandes der Willkür durch einseitigen  Willen ist die Bemächtigung. Also kann die ursprüngliche Erwerbung desselben, mithin auch eines abgemessenen Bodens nur durch Bemächtigung (occupatio) geschehen. – 
	
	\subsection*{tg443.2.7} 
	\textbf{Source : }Die Metaphysik der Sitten/Erster Teil. Metaphysische Anfangsgründe der Rechtslehre/2. Teil. Das öffentliche Recht/3. Abschnitt. Das Weltbürgerrecht\\  
	
	\noindent\textbf{Paragraphe : }Wenn Anbauung in solcher Entlegenheit vom Sitz des ersteren geschieht, daß keines derselben im Gebrauch seines Bodens dem anderen Eintrag tut, so ist das Recht dazu nicht zu bezweifeln; wenn es aber Hirten- oder Jagdvölker  sind (wie die Hottentotten, Tungusen und die meisten amerikanischen Nationen), deren Unterhalt von großen öden Landstrecken abhängt, so würde dies nicht mit Gewalt, sondern nur durch Vertrag, und selbst dieser nicht mit Benutzung der Unwissenheit jener Einwohner in Ansehung der Abtretung solcher Ländereien, geschehen können; obzwar die Rechtfertigungsgründe scheinbar genug sind, daß eine solche Gewalttätigkeit zum Weltbesten gereiche; teils durch Kultur roher Völker (wie der Vorwand, durch den selbst Büsching die blutige Einführung der christlichen Religion in Deutschland entschuldigen will), teils zur Reinigung seines eigenen Landes von verderbten Menschen und gehoffter Besserung derselben, oder ihrer Nachkommenschaft, in einem anderen Weltteile (wie in Neuholland); denn alle diese vermeintlich gute Absichten können doch den Flecken der Ungerechtigkeit in den dazu gebrauchten Mitteln nicht abwaschen. – Wendet man hiegegen ein: daß, bei solcher Bedenklichkeit, mit der Gewalt den \match{Anfang} zu Gründung eines gesetzlichen Zustandes zu machen, vielleicht die ganze Erde noch in gesetzlosem Zustande sein würde: so kann das eben so wenig jene Rechtsbedingung aufheben, als der Vorwand der Staatrevolutionisten, daß es auch, wenn Verfassungen verunartet sind, dem Volk zustehe, sie mit Gewalt umzuformen, und überhaupt einmal für allemal ungerecht zu sein, um nachher die Gerechtigkeit desto sicherer zu gründen und aufblühen zu machen. 
	
	\subsection*{tg472.2.14} 
	\textbf{Source : }Die Metaphysik der Sitten/Zweiter Teil. Metaphysische Anfangsgründe der Tugendlehre/I. Ethische Elementarlehre/I. Teil. Von den Pflichten gegen sich selbst überhaupt/Erstes Buch. Von den vollkommenen Pflichten gegen sich selbst\\  
	
	\noindent\textbf{Paragraphe : }Es ist merkwürdig, daß die Bibel das erste Verbrechen, wodurch das Böse in die Welt gekommen ist, nicht vom Brudermorde (Kains), sondern von der ersten Lüge datiert (weil gegen jenen sich doch die Natur empört), und als den Urheber alles Bösen den Lügner von \match{Anfang} und den Vater der Lügen nennt; wiewohl die Vernunft von diesem Hange der Menschen zur Gleisnerei (esprit fourbe), der  doch vorher gegangen sein muß, keinen Grund weiter angeben kann; weil ein Akt der Freiheit nicht (gleich einer physischen Wirkung) nach dem Naturgesetz des Zusammenhanges der Wirkung und ihrer Ursache, welche insgesamt Erscheinungen sind, deduziert und erklärt werden kann. 
	
	\unnumberedsection{Arbeit (2)} 
	\subsection*{tg445.2.24} 
	\textbf{Source : }Die Metaphysik der Sitten/Erster Teil. Metaphysische Anfangsgründe der Rechtslehre/Anhang erläutender Bemerkungen zu den metaphysischen Anhangsgründen der Rechtslehre\\  
	
	\noindent\textbf{Paragraphe : }Endlich, wenn bei eintretender Volljährigkeit die Pflicht der Eltern zur Erhaltung ihrer Kinder aufhört, so haben jene noch das Recht, diese als ihren Befehlen unterworfene Hausgenossen zu Erhaltung des Hauswesens zu brauchen, bis zur Entlassung derselben; welches eine Pflicht der Eltern gegen diese ist, die aus der natürlichen Beschränkung des Rechts der ersteren folgt. Bis dahin sind sie zwar Hausgenossen und gehören zur Familie, aber von nun an gehören sie zur Dienerschaft (famulatus) in derselben, die folglich nicht anders als durch Vertrag zu dem Seinen des Hausherrn (als seine Domestiken) hinzu kommen können. – Eben so kann auch eine Dienerschaft 
	außer der Familie zu dem Seinen des Hausherren nach einem auf dingliche Art persönlichen Rechte gemacht und als Gesinde (famulatus domesticus) durch Vertrag erworben werden. Ein solcher Vertrag ist nicht der einer bloßen Verdingung (locatio conductio operae) sondern der Hingebung seiner Person in den Besitz des Hausherrn, Vermietung (locatio conductio personae), welche darin von jener Verdingung unterschieden ist, daß das Gesinde sich zu allem Erlaubten versteht, was das Wohl des Hauswesens betrifft und ihm nicht, als bestellte und spezifisch bestimmte Arbeit, aufgetragen wird: Anstatt daß der zur bestimmten \match{Arbeit} Gedungene (Handwerker oder Tagelöhner) sich nicht zu dem Seinen des anderen hingibt und so auch kein Hausgenosse ist. – Des letzteren, weil er nicht im rechtlichen Besitz des anderen ist, der ihn zu gewissen Leistungen verpflichtet, kann der Hausherr, wenn jener auch sein häuslicher Einwohner (inquilinus) wäre, sich nicht (via facti) als einer Sache bemächtigen, sondern muß nach dem persönlichen Recht, auf die Leistung des Versprochenen dringen, welche ihm durch Rechtsmittel (via iuris) zu Gebote stehen. – – So viel zur Erläuterung und Verteidigung eines befremdlichen, neu hinzukommenden, Rechtstitels in der natürlichen Gesetzlehre, der doch, stillschweigend immer in Gebrauch gewesen ist. 
	
	\subsection*{tg475.2.10} 
	\textbf{Source : }Die Metaphysik der Sitten/Zweiter Teil. Metaphysische Anfangsgründe der Tugendlehre/I. Ethische Elementarlehre/I. Teil. Von den Pflichten gegen sich selbst überhaupt/Erstes Buch. Von den vollkommenen Pflichten gegen sich selbst/Zweites Hauptstück. Die Pflicht des Menschen gegen sich selbst, bloß als einem moralischen Wesen/Episodischer Abschnitt. Von der Amphibolie der moralischen Reflexionsbegriffe\\  
	
	\noindent\textbf{Paragraphe : }In Ansehung des lebenden, obgleich vernunftlosen Teils der Geschöpfe ist die Pflicht der Enthaltung von gewaltsamer  und zugleich grausamer Behandlung der Tiere der Pflicht des Menschen gegen sich selbst weit inniglicher entgegengesetzt, weil dadurch das Mitgefühl an ihrem Leiden im Menschen abgestumpft und dadurch eine der Moralität, im Verhältnisse zu anderen Menschen, sehr diensame natürliche Anlage geschwächt und nach und nach ausgetilgt wird; obgleich ihre behende (ohne Qual verrichtete) Tötung, oder auch ihre, nur nicht bis über Vermögen angestrengte, \match{Arbeit} (dergleichen auch wohl Menschen sich gefallen lassen müssen) unter die Befugnisse des Menschen gehören; da hingegen die martervolle physische Versuche, zum bloßen Behuf der Spekulation, wenn auch ohne sie der Zweck erreicht werden könnte, zu verabscheuen sind. – Selbst Dankbarkeit für lang geleistete Dienste eines alten Pferdes oder Hundes (gleich als ob sie Hausgenossen wären) gehört indirekt zur Pflicht des Menschen, nämlich in Ansehung dieser Tiere, direkt aber betrachtet ist sie immer nur Pflicht des Menschen gegen sich selbst. 
	
	\unnumberedsection{Auflosung (4)} 
	\subsection*{tg441.2.51} 
	\textbf{Source : }Die Metaphysik der Sitten/Erster Teil. Metaphysische Anfangsgründe der Rechtslehre/2. Teil. Das öffentliche Recht/1. Abschnitt. Das Staatsrecht\\  
	
	\noindent\textbf{Paragraphe : }Auf diesem ursprünglich erworbenen Grundeigentum beruht das Recht des Oberbefehlshabers, als Obereigentümers (des Landesherrn), die Privateigentümer des Bodens zu beschatzen, d.i. Abgaben durch die Landtaxe, Akzise und Zölle, oder Dienstleistung (dergleichen die Stellung der Mannschaft zum Kriegsdienst ist) zu fordern: so doch, daß das Volk sich selber beschatzt, weil dieses die einzige Art ist, hiebei nach Rechtsgesetzen zu verfahren, wenn es durch das Korps der Deputierten desselben geschieht, auch als gezwungene (von dem bisher bestandenen Gesetz abweichende) Anleihe, nach dem Majestätsrechte, als in einem Falle, da der Staat in Gefahr seiner \match{Auflösung} kommt, erlaubt ist. 
	
	\subsection*{tg441.2.76} 
	\textbf{Source : }Die Metaphysik der Sitten/Erster Teil. Metaphysische Anfangsgründe der Rechtslehre/2. Teil. Das öffentliche Recht/1. Abschnitt. Das Staatsrecht\\  
	
	\noindent\textbf{Paragraphe : }Es gibt indessen zwei todeswürdige Verbrechen, in Ansehung deren, ob die Gesetzgebung auch die Befugnis habe, sie mit der Todesstrafe zu belegen, noch zweifelhaft bleibt. Zu beiden verleitet das Ehrgefühl. Das eine ist das der Geschlechtsehre, das andere der Kriegsehre, und zwar der wahren Ehre, welche jeder dieser zwei Menschenklassen als Pflicht obliegt. Das eine Verbrechen ist der mütterliche Kindesmord (infanticidium maternale); das andere der Kriegsgesellenmord (commilitonicidium), der Duell. – Da die Gesetzgebung die Schmach einer unehelichen Geburt nicht wegnehmen, und eben so wenig den Fleck, welcher aus dem Verdacht der Feigheit, der auf einen untergeordneten Kriegsbefehlshaber fällt, welcher einer verächtlichen Begegnung nicht eine über die Todesfurcht erhobene eigene Gewalt entgegensetzt, wegwischen kann: so scheint es, daß Menschen in diesen Fällen sich im Naturzustande befinden und Tötung (homicidium), die alsdann nicht einmal Mord (homicidium dolosum) heißen müßte, in beiden zwar allerdings strafbar sei, von der obersten Macht aber mit dem Tode nicht könne bestraft werden. Das uneheliche auf die Weit gekommene Kind ist außer dem Gesetz (denn das heißt Ehe), mithin auch außer dem Schutz desselben, geboren. Es ist in das gemeine Wesen gleichsam eingeschlichen (wie verbotene Ware), so daß dieses seine Existenz (weil es billig auf diese Art nicht hätte existieren sollen), mithin auch seine Vernichtung ignorieren kann,  und die Schande der Mutter, wenn ihre uneheliche Niederkunft bekannt, wird, kann keine Verordnung heben. – Der zum Unter-Befehlshaber eingesetzte Kriegesmann, dem ein Schimpf angetan wird, sieht sich eben so wohl durch die öffentliche Meinung der Mitgenossen seines Standes genötigt, sich Genugtuung, und, wie im Naturzustande, Bestrafung des Beleidigers, nicht durchs Gesetz, vor einem Gerichtshofe, sondern durch den Duell, darin er sich selbst der Lebensgefahr aussetzt, zu verschaffen, um seinen Kriegsmut zu beweisen, als worauf die Ehre seines Standes wesentlich beruht, sollte es auch mit der Tötung seines Gegners verbunden sein, die In diesem Kampfe, der öffentlich und mit beiderseitiger Einwilligung, doch auch ungern, geschieht, eigentlich nicht Mord (homicidium dolosum) genannt werden kann. – – Was ist nun in beiden (zur Kriminalgerechtigkeit gehörigen) Fällen Rechtens? – Hier kommt die Strafgerechtigkeit gar sehr ins Gedränge: entweder den Ehrbegriff (der hier kein Wahn ist) durchs Gesetz für nichtig zu erklären, und so mit dem Tode zu strafen, oder von dem Verbrechen die angemessene Todesstrafe wegzunehmen, und so entweder grausam oder nachsichtig zu sein. Die \match{Auflösung} dieses Knotens ist: daß der kategorische Imperativ der Strafgerechtigkeit (die gesetzwidrige Tötung eines anderen müsse mit dem Tode bestraft werden) bleibt, die Gesetzgebung selber aber (mithin auch die bürgerliche Verfassung), so lange noch als barbarisch und unausgebildet, daran Schuld ist, daß die Triebfedern der Ehre im Volk (subjektiv) nicht mit den Maßregeln zusammen treffen wollen, die (objektiv) ihrer Absicht gemäß sind, so daß die öffentliche, vom Staat ausgehende, Gerechtigkeit in Ansehung der aus dem Volk eine Ungerechtigkeit wird. 
	
	\subsection*{tg441.2.97} 
	\textbf{Source : }Die Metaphysik der Sitten/Erster Teil. Metaphysische Anfangsgründe der Rechtslehre/2. Teil. Das öffentliche Recht/1. Abschnitt. Das Staatsrecht\\  
	
	\noindent\textbf{Paragraphe : }Der Geschichtsurkunde dieses Mechanismus nachzuspüren, ist vergeblich, d.i. man kann zum Zeitpunkt  des Anfangs der bürgerlichen Gesellschaft nicht herauslangen (denn die Wilden errichten kein Instrument ihrer Unterwerfung unter das Gesetz, und es ist auch schon aus der Natur roher Menschen abzunehmen, daß sie es mit der Gewalt angefangen haben werden). Diese Nachforschung aber in der Absicht anzustellen, um allenfalls die jetzt bestehende Verfassung mit Gewalt abzuändern, ist sträflich. Denn diese Umänderung müßte durchs Volk, welches sich dazu rottierte, also nicht durch die Gesetzgebung geschehen; Meuterei aber, in einer schon bestehenden Verfassung, ist ein Umsturz aller bürgerlich-rechtlichen Verhältnisse, mithin alles Rechts, d.i. nicht Veränderung der bürgerlichen Verfassung, sondern \match{Auflösung} derselben, und dann der Übergang in die bessere nicht Metamorphose sondern Palingenesie, welche einen neuen gesellschaftlichen Vertrag erfordert, auf den der vorige (nun aufgehobene) keinen Einfluß hat. – Es muß aber dem Souverän doch möglich sein, die bestehende Staatsverfassung zu andern, wenn sie mit der Idee des ursprünglichen Vertrags nicht wohl vereinbar ist, und hiebei doch diejenige Form bestehen zu lassen, die dazu, daß das Volk einen Staat ausmache, wesentlich gehöret. Diese Veränderung kann nun nicht darin bestehen, daß der Staat sich von einer dieser drei Formen zu einer der beiden anderen selbst konstituiert, z.B. daß die Aristokraten einig werden, sich einer Autokratie zu unterwerfen, oder in eine Demokratie verschmelzen zu wollen, und so umgekehrt; gleich als ob es auf der freien Wahl und dem Belieben des Souveräns beruhe, welcher Verfassung er das Volk unterwerfen wolle. Denn selbst dann, wenn er sich zu einer Demokratie umzuändern beschlösse, würde er doch dem Volk unrecht tun können, weil es selbst diese Verfassung verabscheuen könnte, und eine der zwei übrigen für sich zuträglicher fände. 
	
	\subsection*{tg486.2.35} 
	\textbf{Source : }Die Metaphysik der Sitten/Zweiter Teil. Metaphysische Anfangsgründe der Tugendlehre/II. Ethische Methodenlehre/1. Abschnitt. Die ethische Didaktik\\  
	
	\noindent\textbf{Paragraphe : }Wenn dieses nun weislich und pünktlich nach Verschiedenheit der Stufen des Alters, des Geschlechts und des Standes, die der Mensch nach und nach betritt, aus der eigenen Vernunft des Menschen entwickelt worden,  so ist noch etwas, was den Beschluß machen muß, was die Seele inniglich bewegt und den Menschen auf eine Stelle setzt, wo er sich selbst nicht anders als mit der größten Bewunderung der ihm beiwohnenden ursprünglichen Anlagen betrachten kann, und wovon der Eindruck nie erlischt. – Wenn ihm nämlich beim Schlüsse seiner Unterweisung seine Pflichten in ihrer Ordnung noch einmal summarisch vorerzählt (rekapituliert), wenn er, bei jeder derselben, darauf aufmerksam gemacht wird, daß alle Übel, Drangsale und Leiden des Lebens, selbst Bedrohung mit dem Tode, die ihn darüber, daß er seiner Pflicht treu gehorcht, treffen mögen, ihm doch das Bewußtsein, über sie alle erhoben und Meister zu sein, nicht rauben können, so liegt ihm nun die Frage ganz nahe: was ist das in dir, was sich getrauen darf, mit allen Kräften der Natur in dir und um dich in Kampf zu treten und sie, wenn sie mit deinen sittlichen Grundsätzen in Streit kommen, zu besiegen? Wenn diese Frage, deren \match{Auflösung} das Vermögen der spekulativen Vernunft gänzlich übersteigt und die sich dennoch von selbst einstellt, ans Herz gelegt wird, so muß selbst die Unbegreiflichkeit in diesem Selbsterkenntnisse der Seele eine Erhebung geben, die sie zum Heilighalten ihrer Pflicht nur desto stärker belebt, je mehr sie angefochten wird. 
	
	\unnumberedsection{Augenblick (4)} 
	\subsection*{tg436.2.17} 
	\textbf{Source : }Die Metaphysik der Sitten/Erster Teil. Metaphysische Anfangsgründe der Rechtslehre/1. Teil. Das Privatrecht vom äußeren Mein und Dein überhaupt/2. Hauptstück. Von der Art, etwas Äußeres zu erwerben/2. Abschnitt. Vom persönlichen Recht\\  
	
	\noindent\textbf{Paragraphe : }Die Übertragung des Meinen durch Vertrag geschieht nach dem Gesetz der Stetigkeit (lex continui), d.i. der Besitz des Gegenstandes ist während diesem Akt keinen \match{Augenblick} unterbrochen, denn sonst würde ich in diesem Zustande einen Gegenstand als etwas, das keinen Besitzer hat (res vacua), folglich ursprünglich erwerben; welches dem Begriff des Vertrages widerspricht. – Diese Stetigkeit aber bringt es mit sich, daß nicht eines von beiden (promittentis et acceptantis) besonderer, sondern ihr vereinigter Wille derjenige ist, welcher das Meine auf den anderen überträgt; also nicht auf die Art: daß der Versprechende zuerst seinen Besitz zum Vorteil des anderen verläßt (derelinquit), oder seinem Recht entsagt (renunciat) und der andere sogleich darin eintritt, oder umgekehrt. Die Translation ist also ein Akt, in welchem der Gegenstand einen Augenblick beiden zusammen angehört, so wie in der parabolischen Bahn eines geworfenen Steins dieser im Gipfel derselben einen Augenblick als im Steigen und Fallen zugleich begriffen betrachtet werden kann, und so allererst von der steigenden Bewegung zum Fallen übergeht. 
	
	\subsection*{tg438.2.15} 
	\textbf{Source : }Die Metaphysik der Sitten/Erster Teil. Metaphysische Anfangsgründe der Rechtslehre/1. Teil. Das Privatrecht vom äußeren Mein und Dein überhaupt/2. Hauptstück. Von der Art, etwas Äußeres zu erwerben/Episodischer Abschnitt. Von der idealen Erwerbung eines äußeren Gegenstandes der Willkür\\  
	
	\noindent\textbf{Paragraphe : }Die Beerbung ist die Übertragung (translatio) der Habe und des Guts eines Sterbenden auf den Überlebenden durch Zusammenstimmung des Willens beider. – Die Erwerbung des Erbnehmers (heredis instituti) und die Verfassung des Erblassers (testatoris), d.i. dieser Wechsel des Mein und Dein geschieht in einem \match{Augenblick} (articulo mortis), nämlich, da der letztere eben aufhört zu sein, und ist also eigentlich keine Übertragung (translatio) im empirischen Sinn, welche zwei Actus nach einander, nämlich, wo der eine zuerst seinen Besitz verläßt, und darauf der andere darin eintritt, voraussetzt, sondern eine ideale Erwerbung. – Da die Beerbung ohne Vermächtnis (dispositio ultimae voluntatis) im Naturzustande nicht gedacht werden kann, und, ob es ein Erbvertrag (pactum successorium), oder einseitige Erbeseinsetzung (testamentum) sei, es bei der Frage, ob und wie gerade in demselben Augenblick, da das Subjekt aufhört zu sein, ein Übergang des Mein und Dein möglich sei, ankommt, so muß die Frage: wie ist die Erwerbart durch Beerbung möglich? von den mancherlei möglichen Formen ihrer Ausführung (die nur in einem gemeinen Wesen statt finden) unabhängig untersucht werden. 
	
	\subsection*{tg438.2.16} 
	\textbf{Source : }Die Metaphysik der Sitten/Erster Teil. Metaphysische Anfangsgründe der Rechtslehre/1. Teil. Das Privatrecht vom äußeren Mein und Dein überhaupt/2. Hauptstück. Von der Art, etwas Äußeres zu erwerben/Episodischer Abschnitt. Von der idealen Erwerbung eines äußeren Gegenstandes der Willkür\\  
	
	\noindent\textbf{Paragraphe : }»Es ist möglich, durch Erbeseinsetzung zu erwerben.« – Denn der Erblasser Caius verspricht und erklärt in seinem letzten Willen dem Titus, der nichts von jenem Versprechen weiß, seine Habe solle im Sterbefall auf diesen übergehen, und bleibt also, so lange er lebt, alleiniger Eigentümer derselben. Nun kann zwar durch den bloßen einseitigen Willen nichts auf den anderen übergehen: sondern es wird über dem Versprechen noch Annehmung (acceptatio) des anderen Teils dazu erfordert und ein gleichzeitiger Wille (voluntas simultanea), welcher jedoch hier mangelt; denn so lange Caius lebt, kann Titus nicht ausdrücklich akzeptieren, um dadurch zu erwerben; weil jener nur auf den Fall des Todes versprochen hat (denn sonst wäre das Eigentum einen \match{Augenblick} gemeinschaftlich, welches nicht der Wille des Erblassers ist). – Dieser aber erwirbt doch stillschweigend ein eigentümliches Recht an der Verlassenschaft als ein Sachenrecht, nämlich ausschlüßlich sie zu akzeptieren (ius in re iacente), daher diese in dem gedachten Zeitpunkt hereditas iacens heißt. Da nun jeder Mensch notwendigerweise (weil er dadurch wohl gewinnen, nie aber verlieren kann) ein solches Recht, mithin auch stillschweigend akzeptiert und Titus nach dem Tode des Caius in diesem Falle ist, so kann er die Erbschaft durch Annahme des Versprechens erwerben, und sie ist nicht etwa mittlerweile ganz herrenlos (res nullius),  sondern nur erledigt (res vacua) gewesen; weil er ausschlüßlich das Recht der Wahl hatte, ob er die hinterlassene Habe zu der seinigen machen wollte, oder nicht. 
	
	\subsection*{tg445.2.41} 
	\textbf{Source : }Die Metaphysik der Sitten/Erster Teil. Metaphysische Anfangsgründe der Rechtslehre/Anhang erläutender Bemerkungen zu den metaphysischen Anhangsgründen der Rechtslehre\\  
	
	\noindent\textbf{Paragraphe : }Was das Recht der Beerbung anlangt, so hat den Herrn Rezensenten diesesmal sein Scharfblick, den Nerven des Beweises meiner Behauptung zu treffen, verlassen. – Ich sage ja nicht S. 135: »daß ein jeder Mensch notwendigerweise jede ihm angebotene Sache, durch deren Annehmung er nur gewinnen, nichts verlieren kann, annehme« (denn solche Sachen gibt es gar nicht), sondern daß ein jeder das Recht des Angebots in demselben \match{Augenblick} unvermeidlich und stillschweigend, dabei aber doch gültig, immer wirklich annehme: wenn es nämlich die Natur der Sache so mit sich bringt, daß der Widerruf schlechterdings unmöglich ist, nämlich im Augenblicke seines Todes; denn da kann der Promittent nicht widerrufen, und der Promissar ist, ohne irgend einen rechtlichen Akt begehen zu dürfen, in demselben Augenblick Akzeptant, nicht der versprochenen Erbschaft, sondern des Rechts, sie anzunehmen oder auszuschlagen. In diesem Augenblicke sieht er sich bei Eröffnung des Testaments, daß er, schon vor der Akzeptation der Erbschaft, vermögender geworden ist, als er war; denn er hat ausschließlich die Befugnis zu akzeptieren erworben, welche schon ein Vermö gensumstand ist. – Daß hiebei ein bürgerlicher Zustand vorausgesetzt wird, um etwas zu dem Seinen eines anderen zu machen, wenn man nicht mehr da ist, dieser Übergang des Besitztums, aus der  Totenhand, ändert in Ansehung der Möglichkeit der Erwerbung nach allgemeinen Prinzipien des Naturrechts nichts, wenn gleich der Anwendung derselben auf den vorkommenden Fall eine bürgerliche Verfassung zum Grunde gelegt werden muß. – Eine Sache nämlich, die ohne Bedingung anzunehmen oder auszuschlagen in meiner freien Wahl gestellt wird, heißt res iacens. Wenn der Eigentümer einer Sache mir etwas, z.B. ein Möbel des Hauses, aus dem ich auszuziehen eben im Begriff bin, umsonst anbietet (verspricht, es soll mein sein), so habe ich, so lange er nicht widerruft (welches, wenn er darüber stirbt, unmöglich ist), ausschließlich ein Recht zur Akzeptation des Angebotenen (ius in re iacente), d.i. ich allein kann es annehmen oder ausschlagen, wie es mir beliebt: und dieses Recht, ausschließlich zu wählen, erlange ich nicht vermittelst eines besonderen rechtlichen Akts meiner Deklaration, ich wolle, dieses Recht solle mir zustehen, sondern ohne denselben (lege). – Ich kann also zwar mich dahin erklären, ich wolle, die Sache solle mir nicht angehören (weil diese Annahme mir Verdrießlichkeiten mit anderen zuziehen dürfte), aber ich kann nicht wollen, ausschließlich die Wahl zu haben, ob sie mir angehören solle oder nicht; denn dieses Recht (des Annehmens oder Ausschlagens) habe ich ohne alle Deklaration meiner Annahme, unmittelbar durchs Angebot: denn wenn ich sogar die Wahl zu haben ausschlagen könnte, so würde ich wählen, nicht zu wählen; welches ein Widerspruch ist. Dieses Recht zu wählen geht nun im Augenblicke des Todes des Erb-Lassers auf mich über, durch dessen Vermächtnis (institutio heredis) ich zwar noch nichts von der Habe und Gut des Erb-Lassers, aber doch den bloß-rechtlichen (intelligibelen) Besitz dieser Habe oder eines Teils derselben erwerbe: deren Annahme ich mich nun zum Vorteil anderer begeben kann, mithin dieser Besitz keinen Augenblick unterbrochen ist, sondern die Sukzession als eine stetige Reihenfolge, vom Sterbenden zum eingesetzten Erben durch seine Akzeptation übergeht und so der Satz: testamenta sunt iuris naturae, wider alle Zweifel befestigt wird. 
	
	\unnumberedsection{Ausdehnung (1)} 
	\subsection*{tg442.2.46} 
	\textbf{Source : }Die Metaphysik der Sitten/Erster Teil. Metaphysische Anfangsgründe der Rechtslehre/2. Teil. Das öffentliche Recht/2. Abschnitt. Das Völkerrecht\\  
	
	\noindent\textbf{Paragraphe : }Da der Naturzustand der Völker, eben so wohl als einzelner Menschen, ein Zustand ist, aus dem man herausgehen soll, um in einen gesetzlichen zu treten: so ist, vor dieser Ereignis, alles Recht der Völker und alles durch den Krieg erwerbliche oder erhaltbare äußere Mein und Dein der Staaten bloß provisorisch, und kann nur in einem allgemeinen Staatenverein (analogisch mit dem, wodurch ein Volk Staat wird) peremtorisch geltend und ein wahrer Friedenszustand werden. Weil aber, bei gar zu großer \match{Ausdehnung} eines solchen Völkerstaats über weite Landstriche, die Regierung desselben, mithin auch die Beschützung eines jeden Gliedes endlich unmöglich werden muß, eine Menge solcher Korporationen aber wiederum einen Kriegszustand herbeiführt: so ist der ewige Friede (das letzte Ziel des ganzen Völkerrechts) freilich eine unausführbare Idee. Die politische Grundsätze aber, die darauf abzwecken, nämlich solche Verbindungen der Staaten einzugehen, als zur kontinuierlichen Annäherung zu demselben dienen, sind es nicht, sondern, so wie diese eine auf der Pflicht, mithin auch auf dem Recht der Menschen und Staaten gegründete Aufgabe ist, allerdings ausführbar. 
	
	\unnumberedsection{Außage (1)} 
	\subsection*{tg439.2.37} 
	\textbf{Source : }Die Metaphysik der Sitten/Erster Teil. Metaphysische Anfangsgründe der Rechtslehre/1. Teil. Das Privatrecht vom äußeren Mein und Dein überhaupt/3. Hauptstück. Von der subjektiv-bedingten Erwerbung durch den Ausspruch einer öffentlichen Gerichtsbarkeit\\  
	
	\noindent\textbf{Paragraphe : }Man kann keinen anderen Grund angeben, der rechtlich Menschen verbinden könnte, zu glauben und zu bekennen, daß es Götter gebe, als den, damit sie einen Eid schwören, und durch die Furcht vor einer allsehenden obersten Macht, deren Rache sie feierlich gegen sich aufrufen mußten, im Fall, daß ihre \match{Aussage} falsch wäre, genötigt werden könnten, wahrhaft im Aussagen und treu im Versprechen zu sein. Daß man hiebei nicht auf die Moralität dieser beiden Stücke, sondern bloß auf einen blinden Aberglauben derselben rechnete, ist daraus zu ersehen, daß man sich von ihrer bloßen feierlichen Aussage vor Gericht in Rechtssachen keine Sicherheit versprach, ob gleich die Pflicht der Wahrhaftigkeit in einem Fall, wo es auf das Heiligste, was unter Menschen nur sein kann (aufs Recht der Menschen), an kommt, jedermann so klar einleuchtet, mithin bloße Märchen den Bewegungsgrund ausmachen: wie z.B. das unter den Rejangs, einem heidnischen Volk auf Sumatra, welche, nach Marsdens Zeugnis, bei den Knochen ihrer verstorbenen Anverwandten schwören, ob sie gleich gar nicht glauben, daß es noch ein Leben nach dem Tode gebe, oder der Eid der Guineaschwarzen bei ihrem Fetisch, etwa einer Vogelfeder, auf die sie sich vermessen, daß sie ihnen den Hals brechen solle u. dergl. Sie glauben, daß eine unsichtbare  Macht, sie mag nun Verstand haben oder nicht, schon ihrer Natur nach, diese Zauberkraft habe, die durch einen solchen Aufruf in Tat versetzt wird. – Ein solcher Glaube, dessen Name Religion ist, eigentlich aber Superstition heißen sollte, ist aber für die Rechtsverwaltung unentbehrlich, weil, ohne auf ihn zu rechnen, der Gerichtshof nicht genugsam im Stande wäre, geheim gehaltene Facta auszumitteln, und Recht zu sprechen. Ein Gesetz, das hiezu verbindet, ist also offenbar nur zum Behuf der richtenden Gewalt gegeben. 
	
	\unnumberedsection{Axiom (1)} 
	\subsection*{tg435.2.31} 
	\textbf{Source : }Die Metaphysik der Sitten/Erster Teil. Metaphysische Anfangsgründe der Rechtslehre/1. Teil. Das Privatrecht vom äußeren Mein und Dein überhaupt/2. Hauptstück. Von der Art, etwas Äußeres zu erwerben/1. Abschnitt. Vom Sachrecht\\  
	
	\noindent\textbf{Paragraphe : }Alle Menschen sind ursprünglich in einem Gesamt-Besitz des Bodens der ganzen Erde (communio fundi originaria), mit dem ihnen von Natur zustehenden Willen (eines jeden), denselben zu gebrauchen (lex iusti), der, wegen der natürlich unvermeidlichen Entgegensetzung der Willkür des einen gegen die des anderen, allen Gebrauch desselben aufheben würde, wenn nicht jener zugleich das Gesetz für diese enthielte, nach welchem einem jeden ein besonderer Besitz auf dem gemeinsamen Boden bestimmt werden kann (lex iuridica). Aber das austeilende Gesetz des Mein und Dein eines jeden am Boden kann, nach dem \match{Axiom} der äußeren Freiheit, nicht anders als aus einem ursprünglich und a priori vereinigten Willen (der zu dieser Vereinigung keinen rechtlichen Akt voraussetzt), mithin nur im bürgerlichen Zustande, hervorgehen (lex iustitiae distributivae), der allein, was recht, was rechtlich und was Rechtens ist, bestimmt. – In diesem Zustand aber, d.i. vor Gründung und doch in Absicht auf denselben, d.i. provisorisch, nach dem Gesetz der äußeren Erwerbung zu verfahren, ist Pflicht, folglich auch rechtliches Vermögen des Willens, jedermann zu verbinden, den Akt der Besitznehmung und Zueignung, ob er gleich nur einseitig ist, als gültig anzuerkennen; mithin ist eine provisorische Erwerbung des Bodens, mit allen ihren rechtlichen Folgen, möglich. 
	
	\unnumberedsection{Beweis (3)} 
	\subsection*{tg433.2.58} 
	\textbf{Source : }Die Metaphysik der Sitten/Erster Teil. Metaphysische Anfangsgründe der Rechtslehre/1. Teil. Das Privatrecht vom äußeren Mein und Dein überhaupt/1. Hauptstück\\  
	
	\noindent\textbf{Paragraphe : }Dieses Prärogativ des Rechts aus dem empirischen Besitzstande nach der Formel: wohl dem der im Besitz ist (beati possidentes), besteht nicht darin: daß, weil er die Präsumtion eines rechtlichen Mannes hat, er nicht nötig habe, den \match{Beweis} zu führen, er besitze etwas rechtmäßig (denn das gilt nur im streitigen Rechte), sondern weil, nach dem Postulat der praktischen Vernunft, jedermann das Vermögen zukommt, einen äußeren Gegenstand seiner Willkür als das Seine zu haben, mithin jede Inhabung ein Zustand ist, dessen Rechtmäßigkeit ich auf jenem Postulat durch einen Akt des vorhergehenden Willens gründet, und der, wenn nicht ein älterer Besitz eines anderen von ebendemselben Gegenstande dawider ist, also vorläufig, nach dem Gesetz der äußeren Freiheit, jedermann, der mit mir nicht in den Zustand einer öffentlich gesetzlichen Freiheit treten will, von aller Anmaßung des Gebrauchs eines solchen Gegenstandes abzuhalten berechtigt, um, dem Postulat der Vernunft gemäß, eine Sache, die sonst praktisch vernichtet sein würde, seinem Gebrauch zu unterwerfen. 
	
	\subsection*{tg438.2.24} 
	\textbf{Source : }Die Metaphysik der Sitten/Erster Teil. Metaphysische Anfangsgründe der Rechtslehre/1. Teil. Das Privatrecht vom äußeren Mein und Dein überhaupt/2. Hauptstück. Von der Art, etwas Äußeres zu erwerben/Episodischer Abschnitt. Von der idealen Erwerbung eines äußeren Gegenstandes der Willkür\\  
	
	\noindent\textbf{Paragraphe : }Daß durch ein tadelloses Leben und einen dasselbe beschließenden Tod der Mensch einen (negativ-) guten Namen  als das Seine, welches ihm übrig bleibt, erwerbe, wenn er als homo phaenomenon nicht mehr existiert, und daß die Überlebenden (angehörige, oder fremde) ihn auch vor Recht zu verteidigen befugt sind (weil unerwiesene Anklage sie insgesamt wegen ähnlicher Begegnung auf ihren Sterbefall in Gefahr bringt), daß er, sage ich, ein solches Recht erwerben könne, ist eine sonderbare, nichtsdestoweniger unleugbare Erscheinung der a priori gesetzgebenden Vernunft, die ihr Gebot und Verbot auch über die Grenze des Lebens hinaus erstreckt. – Wenn jemand von einem Verstorbenen ein Verbrechen verbreitet, das diesen im Leben ehrlos, oder nur verächtlich gemacht haben würde: so kann ein jeder, welcher einen \match{Beweis} führen kann, daß diese Beschuldigung vorsätzlich unwahr und gelogen sei, den, welcher jenen in böse Nachrede bringt, für einen Kalumnianten öffentlich erklären, mithin ihn selbst ehrlos machen; welches er nicht tun dürfte, wenn er nicht mit Recht voraussetzte, daß der Verstorbene dadurch beleidigt wäre, ob er gleich tot ist, und daß diesem durch jene Apologie Genugtuung widerfahre, ob er gleich nicht mehr existiert.
	
	
	6
	Die Befugnis, die Rolle  des Apologeten für den Verstorbenen zu spielen, darf dieser auch nicht beweisen; denn jeder Mensch maßt sie sich unvermeidlich an, als nicht bloß zur Tugendpflicht (ethisch betrachtet), sondern so gar zum Recht der Menschheit überhaupt gehörig: und es bedarf hiezu keiner besonderen persönlichen Nachteile, die etwa Freunden und Anverwandten aus einem solchen Schandfleck am Verstorbenen erwachsen dürften, um jenen zu einer solchen Rüge zu berechtigen. – Daß also eine solche ideale Erwerbung und ein Recht des Menschen nach seinem Tode gegen die Überlebenden gegründet sei, ist nicht zu streiten, ob schon die Möglichkeit desselben keiner Deduktion fähig ist. 
	
	\subsection*{tg461.2.2} 
	\textbf{Source : }Die Metaphysik der Sitten/Zweiter Teil. Metaphysische Anfangsgründe der Tugendlehre/Einleitung/XIII. Allgemeine Grundsätze der Metaphysik der Sitten in Behandlung einer reinen Tugendlehre\\  
	
	\noindent\textbf{Paragraphe : }
	Erstlich: Für Eine Pflicht kann auch nur ein einziger Grund der Verpflichtung gefunden werden, und, werden zwei oder mehrere Beweise darüber geführt, so ist es ein sicheres Kennzeichen, daß man entweder noch gar keinen gültigen \match{Beweis} habe, oder es auch mehrere und verschiedne Pflichten sind, die man für Eine gehalten hat. 
	
	\unnumberedsection{Beweisfuhrung (2)} 
	\subsection*{tg431.2.58} 
	\textbf{Source : }Die Metaphysik der Sitten/Erster Teil. Metaphysische Anfangsgründe der Rechtslehre/Einleitung in die Rechtslehre\\  
	
	\noindent\textbf{Paragraphe : }Die Absicht, weswegen man eine solche Einteilung in das System des Naturrechts (sofern es das angeborne angeht) eingeführt hat, geht darauf hinaus, damit, wenn über ein erworbenes Recht ein Streit entsteht und die Frage eintritt, wem die \match{Beweisführung} (onus probandi) obliege, entweder von einer bezweifelten Tat, oder, wenn diese ausgemittelt ist, von einem bezweifelten Recht, derjenige, welcher diese Verbindlichkeit von sich ablehnt, sich auf sein angebornes Recht der Freiheit (welches nun nach seinen verschiedenen Verhältnissen spezifiziert wird) methodisch und gleich als nach verschiedenen Rechtstiteln berufen könne. 
	
	\subsection*{tg445.2.38} 
	\textbf{Source : }Die Metaphysik der Sitten/Erster Teil. Metaphysische Anfangsgründe der Rechtslehre/Anhang erläutender Bemerkungen zu den metaphysischen Anhangsgründen der Rechtslehre\\  
	
	\noindent\textbf{Paragraphe : }Ich erwerbe also ohne \match{Beweisführung} und ohne allen rechtlichen Akt: Ich brauche nicht zu beweisen, sondern durchs Gesetz (lege); und was dann? Die öffentliche Befreiung von Ansprüchen, d.i. die gesetzliche Sicherheit meines Besitzes, dadurch, daß ich nicht den Beweis führen darf, und mich auf einen ununterbrochenen Besitz gründe. Daß aber alle Erwerbung im Naturstande bloß provisorisch ist, das hat keinen Einfluß auf die Frage von der Sicherheit des Besitzes des Erworbenen, welche vor jener vorhergehen muß. 
	
	\unnumberedsection{Deduktion (6)} 
	\subsection*{tg433.2.23} 
	\textbf{Source : }Die Metaphysik der Sitten/Erster Teil. Metaphysische Anfangsgründe der Rechtslehre/1. Teil. Das Privatrecht vom äußeren Mein und Dein überhaupt/1. Hauptstück\\  
	
	\noindent\textbf{Paragraphe : }Die Namenerklärung, d.i. diejenige, welche bloß zur Unterscheidung des Objekts von allen andern zureicht und aus einer vollständigen und bestimmten Exposition des Begriffs hervorgeht, würde sein: Das äußere Meine ist dasjenige außer mir, an dessen mir beliebigen Gebrauch mich zu hindern Läsion (Unrecht) sein würde. – Die Sacherklärung dieses Begriffs aber, d.i. die, welche auch zur \match{Deduktion} desselben (der Erkenntnis der Möglichkeit des Gegenstandes) zureicht, lautet nun so: Das äußere Meine ist dasjenige, in dessen Gebrauch mich zu stören Läsion sein würde, ob ich gleich nicht im Besitz desselben (nicht Inhaber des Gegenstandes) bin. – In irgend einem Besitz des äußeren Gegenstandes muß ich sein, wenn der Gegenstand mein heißen soll; denn sonst würde der, welcher diesen Gegenstand wider meinen Willen affizierte, mich nicht zugleich affizieren, mithin auch nicht lädieren. Also muß, zu Folge des § 4, ein intelligibler Besitz (possessio noumenon) als möglich vorausgesetzt werden, wenn es ein äußeres Mein oder Dein geben soll; der empirische Besitz (Inhabung) ist alsdenn nur Besitz in der Erscheinung (possessio phaenomenon), obgleich der Gegenstand, den ich besitze, hier nicht so, wie es in der transzendentalen Analytik geschieht, selbst als Erscheinung, sondern als Sache an sich selbst betrachtet wird; denn dort war es der Vernunft um das theoretische Erkenntnis der  Natur der Dinge und, wie weit sie reichen könne, hier aber ist es ihr um praktische Bestimmung der Willkür nach Gesetzen der Freiheit zu tun, der Gegenstand mag nun durch Sinne, oder auch bloß den reinen Verstand erkennbar sein, und das Recht ist ein solcher reiner praktischer Vernunftbegriff der Willkür unter Freiheitsgesetzen. 
	
	\subsection*{tg433.2.36} 
	\textbf{Source : }Die Metaphysik der Sitten/Erster Teil. Metaphysische Anfangsgründe der Rechtslehre/1. Teil. Das Privatrecht vom äußeren Mein und Dein überhaupt/1. Hauptstück\\  
	
	\noindent\textbf{Paragraphe : }Die Möglichkeit eines solchen Besitzes, mithin die \match{Deduktion} des Begriffs eines nicht-empirischen Besitzes, gründet sich auf dem rechtlichen Postulat der praktischen Vernunft: »daß es Rechtspflicht sei, gegen andere so zu handeln, daß das Äußere (Brauchbare) auch das Seine von irgend jemanden werden könne«, zugleich mit der Exposition des letzteren Begriffs, welcher das äußere Seine nur auf einen nicht-physischen Besitz gründet, verbunden. Die Möglichkeit des letzteren aber kann keinesweges für sich selbst bewiesen, oder eingesehen werden (eben weil es ein Vernunftbegriff ist, dem keine Anschauung korrespondierend gegeben werden kann), sondern ist eine unmittelbare Folge aus dem gedachten Postulat, Denn, wenn es notwendig ist, nach jenem Rechtsgrundsatz zu handeln, so muß auch die intelligibele Bedingung (eines bloß-rechtlichen Besitzes) möglich sein. – Es darf auch niemand befremden, daß die theoretischen Prinzipien des äußeren Mein und Dein sich im Intelligibelen verlieren und kein erweitertes Erkenntnis vorstellen; weil der Begriff der Freiheit, auf dem sie beruhen, keiner theoretischen Deduktion seiner Möglichkeit fähig ist, und nur aus dem praktischen Gesetze der Vernunft (dem kategorischen Imperativ), als einem Faktum derselben, geschlossen werden kann. 
	
	\subsection*{tg436.2.12} 
	\textbf{Source : }Die Metaphysik der Sitten/Erster Teil. Metaphysische Anfangsgründe der Rechtslehre/1. Teil. Das Privatrecht vom äußeren Mein und Dein überhaupt/2. Hauptstück. Von der Art, etwas Äußeres zu erwerben/2. Abschnitt. Vom persönlichen Recht\\  
	
	\noindent\textbf{Paragraphe : }Aber die transzendentale \match{Deduktion} des Begriffs der Erwerbung durch Vertrag kann allein alle diese Schwierigkeiten heben. In einem rechtlichen äußeren Verhältnisse wird meine Besitznehmung der Willkür eines anderen (und so wechselseitig), als Bestimmungsgrund desselben zu einer Tat, zwar erst empirisch durch Erklärung und Gegenerklärung der Willkür eines jeden von beiden in der Zeit, als sinnlicher Bedingung der Apprehension, gedacht, wo beide rechtliche Akte immer nur auf einander folgen; weil jenes Verhältnis (als ein rechtliches) rein intellektuell ist, durch den Willen als ein gesetzgebendes Vernunftvermögen jener Besitz als ein intelligibeler (possessio noumenon) nach Freiheitsbegriffen mit Abstraktion von jenen empirischen Bedingungen  als das Mein oder Dein vorgestellt; wo beide Akte, des Versprechens und der Annehmung, nicht als aufeinander folgend, sondern (gleich als pactum re initum) aus einem einzigen gemeinsamen Willen hervorgehend (welches durch das Wort zugleich ausgedruckt wird) und der Gegenstand (promissum) durch Weglassung der empirischen Bedingungen nach dem Gesetz der reinen praktischen Vernunft als erworben vorgestellt wird. 
	
	\subsection*{tg436.2.13} 
	\textbf{Source : }Die Metaphysik der Sitten/Erster Teil. Metaphysische Anfangsgründe der Rechtslehre/1. Teil. Das Privatrecht vom äußeren Mein und Dein überhaupt/2. Hauptstück. Von der Art, etwas Äußeres zu erwerben/2. Abschnitt. Vom persönlichen Recht\\  
	
	\noindent\textbf{Paragraphe : }Daß dieses die wahre und einzig mögliche \match{Deduktion} des Begriffs der Erwerbung durch Vertrag sei, wird durch die mühselige und doch immer vergebliche Bestrebung der Rechtsforscher (z.B. Moses Mendelssohns in seinem Jerusalem) zur Beweisführung jener Möglichkeit hinreichend bestätigt. – Die Frage war: warum soll ich mein Versprechen halten? Denn daß ich es soll, begreift ein jeder von selbst. Es ist aber schlechterdings unmöglich, von diesem kategorischen Imperativ noch einen Beweis zu führen; eben so, wie es für den Geometer unmöglich ist, durch Vernunftschlüsse zu beweisen, daß ich, um ein Dreieck zu machen, drei Linien nehmen müsse (ein analytischer Satz), deren zwei aber zusammengenommen größer sein müssen, als die dritte (ein synthetischer; beide aber a priori). Es ist ein Postulat der reinen (von allen sinnlichen Bedingungen des Raumes und der Zeit, was den Rechtsbegriff betrifft, abstrahierenden) Vernunft, und die Lehre der Möglichkeit der Abstraktion von jenen Bedingungen, ohne daß dadurch der Besitz desselben aufgehoben wird, ist selbst die Deduktion des Begriffs der Erwerbung durch Vertrag; so wie es in dem vorigen Titel die Lehre von der Erwerbung durch Bemächtigung der äußeren Sache war. 
	
	\subsection*{tg458.2.3} 
	\textbf{Source : }Die Metaphysik der Sitten/Zweiter Teil. Metaphysische Anfangsgründe der Tugendlehre/Einleitung/X. Das oberste Prinzip der Rechtslehre war analytisch; das der Tugendlehre ist synthetisch\\  
	
	\noindent\textbf{Paragraphe : }Dagegen geht das Prinzip der Tugendlehre über den Begriff der äußern Freiheit hinaus und verknüpft nach allgemeinen Gesetzen mit demselben noch einen Zweck, den es zur Pflicht macht. Dieses Prinzip ist also synthetisch. – Die Möglichkeit desselben ist in der \match{Deduktion} (§ IX) enthalten. 
	
	\subsection*{tg489.2.6} 
	\textbf{Source : }Die Metaphysik der Sitten/Fußnoten\\  
	
	\noindent\textbf{Paragraphe : }
	
	3 Die \match{Deduktion} der Einteilung eines Systems: d.i. der Beweis ihrer Vollständigkeit sowohl, als auch der Stetigkeit, daß nämlich der Übergang vom eingeteilten Begriffe zum Gliede der Einteilung in der ganzen Reihe der Untereinteilungen durch keinen Sprung (divisio per saltum) geschehe, ist eine der am schwersten zu erfüllenden Bedingungen für den Baumeister eines Systems. Auch was der oberste eingeteilte Begriff zu der Einteilung Recht oder Unrecht (aut fas aut nefas) sei, hat seine Bedenklichkeit. Es ist der Akt der freien Willkür überhaupt. So wie die Lehrer der Ontologie vom Etwas und Nichts zu oberst anfangen, ohne inne zu werden, daß dieses schon Glieder einer Einteilung sind, dazu noch der eingeteilte Begriff fehlt, der kein anderer, als der Begriff von einem Gegenstande überhaupt sein kann. 
	
	\unnumberedsection{Definition (2)} 
	\subsection*{tg429.2.8} 
	\textbf{Source : }Die Metaphysik der Sitten/Erster Teil. Metaphysische Anfangsgründe der Rechtslehre/Vorrede\\  
	
	\noindent\textbf{Paragraphe : }
	Von minderer Bedeutung, jedoch nicht ganz ohne alle Wichtigkeit, wäre der Vorwurf: daß ein diese Philosophie wesentlich unterscheidendes Stück doch nicht ihr eigenes Gewächs, sondern etwa einer anderen Philosophie (oder Mathematik) abgeborgt sei; dergleichen ist der Fund, den ein Tübingscher Rezensent gemacht haben will, und der die \match{Definition} der Philosophie überhaupt angeht, welche der Verfasser der Kritik d. r. V. für sein eigenes, nicht unerhebliches, Produkt ausgibt, und die doch schon vor vielen Jahren von einem anderen fast mit denselben Ausdrücken gegeben worden sei.
	
	
	1
	Ich überlasse es einem jeden, zu beurteilen, ob die Worte: intellectualis quaedam constructio, den Gedanken der Darstellung eines gegebenen Begriffs in einer Anschauung a priori hätten hervorbringen können, wodurch auf einmal die Philosophie von der Mathematik ganz bestimmt geschieden wird. Ich bin gewiß: Hausen selbst würde sich geweigert haben, diese Erklärung seines Ausdrucks anzuerkennen; denn die Möglichkeit einer Anschauung a priori, und, daß der Raum eine solche und nicht ein bloß der empirischen Anschauung (Wahrnehmung) gegebenes Nebeneinandersein des Mannigfaltigen außer einander sei (wie Wolff ihn erklärt), würde ihn schon aus dem Grunde abgeschreckt haben, weil er sich hiemit in weithinaussehende philosophische Untersuchungen verwickelt gefühlt hätte. Die gleichsam durch den Verstand gemachte Darstellung bedeutete dem scharfsinnigen Mathematiker nichts weiter, als die einem Begriffe korrespondierende (empirische) Verzeichnung einer Linie, bei der bloß auf die Regel Acht gegeben, von den in der Ausführung unvermeidlichen Abweichungen  aber abstrahiert wird; wie man es in der Geometrie auch an der Konstruktion der Gleichungen wahrnehmen kann. 
	
	\subsection*{tg445.2.13} 
	\textbf{Source : }Die Metaphysik der Sitten/Erster Teil. Metaphysische Anfangsgründe der Rechtslehre/Anhang erläutender Bemerkungen zu den metaphysischen Anhangsgründen der Rechtslehre\\  
	
	\noindent\textbf{Paragraphe : }Die \match{Definition} des auf dingliche Art persönlichen Rechts ist nun kurz und gut diese: »es ist das Recht des Menschen, eine Person außer sich als das Seine
	
	
	
	10
	zu haben«. Ich sage mit Fleiß: eine Person; denn einen anderen Menschen, der durch Verbrechen seine Persönlichkeit eingebüßt hat (zum Leibeigenen geworden ist), könnte man wohl als das Seine haben; von diesem Sachenrecht ist aber hier nicht die Rede. 
	
	\unnumberedsection{Ebene (1)} 
	\subsection*{tg435.2.13} 
	\textbf{Source : }Die Metaphysik der Sitten/Erster Teil. Metaphysische Anfangsgründe der Rechtslehre/1. Teil. Das Privatrecht vom äußeren Mein und Dein überhaupt/2. Hauptstück. Von der Art, etwas Äußeres zu erwerben/1. Abschnitt. Vom Sachrecht\\  
	
	\noindent\textbf{Paragraphe : }
	Alle Menschen sind ursprünglich (d.i. vor allem rechtlichem Akt der Willkür) im rechtmäßigen Besitz des Bodens, d.i. sie haben ein Recht, da zu sein, wohin sie die Natur, oder der Zufall (ohne ihren Willen) gesetzt hat. Dieser Besitz (possessio), der vom Sitz (sedes), als einem willkürlichen, mithin erworbenen, dauernden Besitzunterschieden ist, ist ein gemeinsamer Besitz, wegen der Einheit aller Plätze auf der Erdfläche, als Kugelfläche; weil, wenn sie eine unendliche \match{Ebene} wäre, die Menschen sich darauf so zerstreuen könnten, daß sie in gar keine Gemeinschaft mit einander kämen, diese also nicht eine notwendige Folge von ihrem Dasein auf Erden wäre. – Der Besitz aller Menschen auf Erden, der vor allem rechtlichem Akt derselben vorhergeht (von der Natur selbst konstituiert ist), ist ein ursprünglicher Gesamtbesitz (communio possessionis originaria), dessen Begriff nicht empirisch und von Zeitbedingungen abhängig ist, wie etwa der gedichtete aber nie erweisliche eines uranfänglichen Gesamtbesitzes (communio primaeva), sondern ein praktischer Vernunftbegriff, der a priori das Prinzip enthält, nach welchem allein die Menschen den Platz auf Erden nach Rechtsgesetzen gebrauchen können. 
	
	\unnumberedsection{Effekt (4)} 
	\subsection*{tg430.2.53} 
	\textbf{Source : }Die Metaphysik der Sitten/Erster Teil. Metaphysische Anfangsgründe der Rechtslehre/Einleitung in die Metaphysik der Sitten\\  
	
	\noindent\textbf{Paragraphe : }Was jemand pflichtmäßig mehr tut, als wozu er nach dem Gesetze gezwungen werden kann, ist verdienstlich (meritum); was er nur gerade dem letzteren angemessen tut, ist Schuldigkeit (debitum); was er endlich weniger tut, als die letztere fordert, ist moralische Verschuldung (demeritum). Der rechtliche \match{Effekt} einer Verschuldung ist die Strafe (poena); der einer verdienstlichen Tat Belohnung (praemium) (vorausgesetzt, daß sie, im Gesetz verheißen, die Bewegursache war); die Angemessenheit des Verfahrens zur Schuldigkeit hat gar keinen rechtlichen Effekt. – Die gütige Vergeltung (remuneratio s. repensio benefica) steht zur Tat in gar keinem Rechtsverhältnis. 
	
	\subsection*{tg435.2.32} 
	\textbf{Source : }Die Metaphysik der Sitten/Erster Teil. Metaphysische Anfangsgründe der Rechtslehre/1. Teil. Das Privatrecht vom äußeren Mein und Dein überhaupt/2. Hauptstück. Von der Art, etwas Äußeres zu erwerben/1. Abschnitt. Vom Sachrecht\\  
	
	\noindent\textbf{Paragraphe : }Eine solche Erwerbung aber bedarf doch und hat auch eine Gunst des Gesetzes (lex permissiva), in Ansehung der Bestimmung der Grenzen des rechtlich-möglichen Besitzes, für sich, weil sie vor dem rechtlichen Zustande vorhergeht und, als bloß dazu einleitend, noch nicht peremtorisch ist, welche Gunst sich aber nicht weiter erstreckt, als bis zur  Einwilligung anderer (teilnehmender) zu Errichtung des letzteren, bei dem Widerstande derselben aber, in diesen (den bürgerlichen) zu treten, und so lange derselbe währt, allen \match{Effekt} einer rechtmäßigen Erwerbung bei sich führt, weil dieser Ausgang auf Pflicht gegründet ist. 
	
	\subsection*{tg441.2.41} 
	\textbf{Source : }Die Metaphysik der Sitten/Erster Teil. Metaphysische Anfangsgründe der Rechtslehre/2. Teil. Das öffentliche Recht/1. Abschnitt. Das Staatsrecht\\  
	
	\noindent\textbf{Paragraphe : }Der Ursprung der obersten Gewalt ist für das Volk, das unter derselben steht, in praktischer Absicht unerforschlich: d.i. der Untertan soll nicht über diesen Ursprung, als ein noch in Ansehung des ihr schuldigen Gehorsams zu bezweifelndes Recht (ius controversum), werktätig vernünfteln. Denn, da das Volk, um rechtskräftig über die oberste Staatsgewalt (summum imperium) zu urteilen, schon als unter einem allgemein gesetzgebenden Willen vereint angesehen werden muß, so kann und darf es nicht anders urteilen, als das gegenwärtige Staatsoberhaupt (summus imperans) es will. – Ob ursprünglich ein wirklicher Vertrag der Unterwerfung unter denselben (pactum subiectionis  civilis) als ein Faktum vorher gegangen, oder ob die Gewalt vorherging, und das Gesetz nur hintennach gekommen sei, oder auch in dieser Ordnung sich habe folgen sollen: das sind für das Volk, das nun schon unter dem bürgerlichen Gesetze steht, ganz zweckleere, und doch den Staat mit Gefahr bedrohende Vernünfteleien; denn, wollte der Untertan, der den letzteren Ursprung nun ergrübelt hätte, sich jener jetzt herrschenden Autorität widersetzen, so würde er nach den Gesetzen derselben, d.i. mit allem Recht, bestraft, vertilgt, oder (als vogelfrei, exlex) ausgestoßen werden. – Ein Gesetz, das so heilig (unverletzlich) ist, daß es, praktisch, auch nur in Zweifel zu ziehen, mithin seinen \match{Effekt} einen Augenblick zu suspendieren, schon ein Verbrechen ist, wird so vorgestellt, als ob es nicht von Menschen, aber doch von irgend einem höchsten tadelfreien Gesetzgeber herkommen müsse, und das ist die Bedeutung des Satzes: »alle Obrigkeit ist von Gott«, welcher nicht einen Geschichtsgrund der bürgerlichen Verfassung, sondern eine Idee, als praktisches Vernunftprinzip, aussagt: der jetzt bestehenden gesetzgebenden Gewalt gehorchen zu sollen; ihr Ursprung mag sein, welcher er wolle. 
	
	\subsection*{tg481.2.88} 
	\textbf{Source : }Die Metaphysik der Sitten/Zweiter Teil. Metaphysische Anfangsgründe der Tugendlehre/I. Ethische Elementarlehre/II. Teil. Von den Tugendpflichten gegen andere/Erstes Hauptstück. Von den Pflichten gegen andere, bloß als Menschen/Erster Abschnitt. Von der Liebespflicht gegen andere Menschen\\  
	
	\noindent\textbf{Paragraphe : }Eine jede das Recht eines Menschen kränkende Tat verdient Strafe; wodurch das Verbrechen an dem Täter gerächet (nicht bloß der zugefügte Schade ersetzt) wird. Nun ist aber Strafe nicht ein Akt der Privatautorität des Beleidigten, sondern eines von ihm unterschiedenen Gerichtshofes, der den Gesetzen eines Oberen über alle, die demselben unterworfen sind) \match{Effekt} gibt, und, wenn wir die Menschen (wie es in der Ethik notwendig ist) in einem rechtlichen Zustande, aber nach bloßen Vernunftgesetzen (nicht nach bürgerlichen) betrachten, so hat niemand die Befugnis, Strafen zu verhängen und von Menschen erlittene Beleidigung zu rächen, als der, welcher auch der oberste moralische Gesetzgeber ist, und dieser allein (nämlich Gott) kann sagen: »Die Rache ist mein; ich will vergelten«. Es ist also Tugendpflicht, nicht allein selbst, bloß aus Rache, die  Feindseligkeit anderer nicht mit Haß zu erwidern, sondern selbst nicht einmal den Weltrichter zur Rache aufzufordern; teils weil der Mensch von eigener Schuld genug auf sich sitzen hat, um der Verzeihung selbst sehr zu bedürfen, teils, und zwar vornehmlich, weil keine Strafe, von wem es auch sei, aus Haß verhängt werden darf. – Daher ist Versöhnlichkeit (placabilitas) Menschenpflicht; womit doch die sanfte Duldsamkeit der Beleidigungen (mitis iniuriarum patientia) nicht verwechselt werden muß, als Entsagung auf harte (rigorosa) Mittel, um der fortgesetzten Beleidigung anderer vorzubeugen; denn das wäre Wegwerfung seiner Rechte unter die Füße anderer, und Verletzung der Pflicht des Menschen gegen sich selbst. 
	
	\unnumberedsection{Ende (8)} 
	\subsection*{tg429.2.10} 
	\textbf{Source : }Die Metaphysik der Sitten/Erster Teil. Metaphysische Anfangsgründe der Rechtslehre/Vorrede\\  
	
	\noindent\textbf{Paragraphe : }Gegen das \match{Ende} des Buchs habe ich einige Abschnitte mit minderer Ausführlichkeit bearbeitet, als in Vergleichung mit den vorhergehenden erwartet werden konnte: teils, weil sie mir aus diesen leicht gefolgert werden zu können schienen, teils auch, weil die letzte (das öffentliche Recht betreffende) eben jetzt so vielen Diskussionen unterworfen und dennoch so wichtig sind, daß sie den Aufschub des entscheidenden Urteils auf einige Zeit wohl rechtfertigen können. 
	
	\subsection*{tg431.2.31} 
	\textbf{Source : }Die Metaphysik der Sitten/Erster Teil. Metaphysische Anfangsgründe der Rechtslehre/Einleitung in die Rechtslehre\\  
	
	\noindent\textbf{Paragraphe : }Die Billigkeit (objektiv betrachtet) ist keinesweges ein Grund zur Aufforderung bloß an die ethische Pflicht anderer (ihr Wohlwollen und Gütigkeit), sondern der, welcher aus diesem Grunde etwas fordert, fußt sich auf sein Recht, nur daß ihm die für den Richter erforderlichen Bedingungen mangeln, nach welchen dieser bestimmen könnte, wie viel, oder auf welche Art dem Anspruche desselben genug getan werden könne. Der in einer auf gleiche Vorteile eingegangenen Maskopei dennoch mehr getan, dabei aber wohl gar durch Unglücksfälle mehr verloren hat, als die übrigen Glieder, kann nach der Billigkeit von der Gesellschaft  mehr fordern, als bloß zu gleichen Teilen mit ihnen zu gehen. Allein nach dem eigentlichen (strikten) Recht, weil, wenn man sich in seinem Fall einen Richter denkt, dieser keine bestimmte Angaben (data) hat, um, wie viel nach dem Kontrakt ihm zukomme, auszumachen, würde er mit seiner Forderung abzuweisen sein. Der Hausdiener, dem sein bis zu \match{Ende} des Jahres laufender Lohn in einer binnen der Zeit verschlechterten Münzsorte bezahlt wird, womit er das nicht ausrichten kann, was er bei Schließung des Kontrakts sich dafür anschaffen konnte, kann, bei gleichem Zahlwert, aber ungleichem Geldwert, sich nicht auf sein Recht berufen, deshalb schadlos gehalten zu werden, sondern nur die Billigkeit zum Grunde aufrufen (eine stumme Gottheit, die nicht gehöret werden kann); weil nichts hierüber im Kontrakt bestimmt war, ein Richter aber nach unbestimmten Bedingungen nicht sprechen kann. 
	
	\subsection*{tg439.2.38} 
	\textbf{Source : }Die Metaphysik der Sitten/Erster Teil. Metaphysische Anfangsgründe der Rechtslehre/1. Teil. Das Privatrecht vom äußeren Mein und Dein überhaupt/3. Hauptstück. Von der subjektiv-bedingten Erwerbung durch den Ausspruch einer öffentlichen Gerichtsbarkeit\\  
	
	\noindent\textbf{Paragraphe : }Aber nun ist die Frage: worauf gründet man die Verbindlichkeit, die jemand vor Gericht haben soll, eines anderen Eid als zu Recht gültigen Beweisgrund der Wahrheit seines Vorgebens anzunehmen, der allem Hader ein \match{Ende} mache, d.i. was verbindet mich rechtlich, zu glauben, daß ein anderer (der Schwörende) überhaupt Religion habe, um mein Recht auf seinen Eid ankommen zu lassen? Imgleichen umgekehrt: kann ich überhaupt verbunden werden, zu schwören? Beides ist an sich unrecht. 
	
	\subsection*{tg439.2.40} 
	\textbf{Source : }Die Metaphysik der Sitten/Erster Teil. Metaphysische Anfangsgründe der Rechtslehre/1. Teil. Das Privatrecht vom äußeren Mein und Dein überhaupt/3. Hauptstück. Von der subjektiv-bedingten Erwerbung durch den Ausspruch einer öffentlichen Gerichtsbarkeit\\  
	
	\noindent\textbf{Paragraphe : }Wenn die Amtseide, welche gewöhnlich promissorisch sind, daß man nämlich den ernstlichen Vorsatz habe, sein Amt pflichtmäßig zu verwalten, in assertorische verwandelt würden, daß nämlich der Beamte etwa zu \match{Ende} eines Jahres (oder mehrerer) verbunden  wäre, die Treue seiner Amtsführung während desselben zu beschwören; so würde dieses teils das Gewissen mehr in Bewegung bringen, als der Versprechungseid, welcher hinterher noch immer den inneren Vorwand übrig läßt, man habe, bei dem besten Vorsatz, die Beschwerden nicht voraus gesehen, die man nur nachher während der Amtsverwaltung erfahren habe, und die Pflichtübertretungen würden auch, wenn ihre Summierung durch Aufmerker bevorstände, mehr Besorgnis der Anklage wegen erregen, als wenn sie bloß eine nach der anderen (über welche die vorigen vergessen sind) gerügt würden. – Was aber das Beschwören des Glaubens (de credulitate) betrifft, so kann dieses gar nicht von einem Gericht verlangt werden. Denn erstlich enthält es in sich selbst einen Widerspruch: dieses Mittelding zwischen Meinen und Wissen, weil es so etwas ist, worauf man wohl zu wetten, keinesweges aber darauf zu schwören sich getrauen kann. Zweitens begeht der Richter, der solchen Glaubenseid dem Parten ansinnete, um etwas zu seiner Absicht Gehöriges, gesetzt es sei auch das gemeine Beste, auszumitteln, einen großen Verstoß an der Gewissenhaftigkeit des Eidleistenden, teils durch den Leichtsinn, zu dem er verleitet und wodurch der Richter seine eigene Absicht vereitelt, teils durch Gewissensbisse, die ein Mensch fühlen muß, der heute eine Sache, aus einem gewissen Gesichtspunkt betrachtet, sehr wahrscheinlich, morgen aber, aus einem anderen, ganz unwahrscheinlich finden kann, und lädiert also denjenigen, den er zu einer solchen Eidesleistung nötigt. 
	
	\subsection*{tg441.2.57} 
	\textbf{Source : }Die Metaphysik der Sitten/Erster Teil. Metaphysische Anfangsgründe der Rechtslehre/2. Teil. Das öffentliche Recht/1. Abschnitt. Das Staatsrecht\\  
	
	\noindent\textbf{Paragraphe : }Der allgemeine Volkswille hat sich nämlich zu einer Gesellschaft vereinigt, welche sich immerwährend erhalten soll, und zu dem \match{Ende} sich der inneren Staatsgewalt unterworfen, um die Glieder dieser Gesellschaft, die es selbst nicht vermögen, zu erhalten. Von Staatswegen ist also die Regierung berechtigt, die Vermögenden zu nötigen, die Mittel der Erhaltung derjenigen, die es, selbst den notwendigsten Naturbedürfnissen nach, nicht sind, herbei zu schaffen; weil ihre Existenz zugleich als Akt der Unterwerfung unter den Schutz und die zu ihrem Dasein nötige Vorsorge des gemeinen Wesens ist, wozu sie sich verbindlich gemacht haben, auf welche der Staat nun sein Recht gründet, zur Erhaltung ihrer Mitbürger das Ihrige beizutragen. Das kann nun geschehen: durch Belastung des Eigentums der Staatsbürger, oder ihres Handelsverkehrs, oder durch errichtete Fonds und deren Zinsen: nicht zu Staats- (denn der ist reich), sondern zu Volksbedürfnissen, aber nicht bloß durch freiwillige Beiträge (weil hier nur vom Rechte
	des Staats gegen das Volk die Rede ist), worunter einige gewinnsüchtige sind (als Lotterien, die mehr Arme, und dem öffentlichen Eigentum Gefährliche machen, als sonst sein würden, und die also nicht erlaubt sein sollten), sondern zwangsmäßig, als Staatslasten. Hier fragt sich nun: ob die Versorgung der Armen durch laufende Beiträge, so daß jedes Zeitalter die Seinigen ernährt, oder durch nach und nachgesammelte Bestände und überhaupt fromme Stiftungen (dergleichen Witwenhäuser, Hospitäler, u. dergl. sind) und zwar jenes nicht durch Bettelei, welche mit der Räuberei nahe verwandt ist, sondern durch gesetzliche Auflage ausgerichtet werden soll. – Die erstere Anordnung muß für die einzige dem Rechte des Staats angemessene, der sich niemand entziehen kann, der zu leben hat, gehalten werden; weil sie nicht (wie von frommen Stiftungen zu besorgen ist), wenn sie mit der Zahl der Armen anwachsen, das Armsein zum Erwerbmittel für faule Menschen machen, und so eine ungerechte Belästigung des Volks durch die Regierung sein würden. 
	
	\subsection*{tg447.2.11} 
	\textbf{Source : }Die Metaphysik der Sitten/Zweiter Teil. Metaphysische Anfangsgründe der Tugendlehre/Vorrede\\  
	
	\noindent\textbf{Paragraphe : }Man muß sich hiebei billig wundern: wie es, nach allen bisherigen Läuterungen des Pflichtprinzips, so fern es aus reiner Vernunft abgeleitet wird, noch möglich war, es wiederum auf Glückseligkeitslehre zurück zu führen: doch so, daß eine gewisse moralische Glückseligkeit, die nicht auf empirischen Ursachen beruhete, zu dem \match{Ende} angedacht worden, welche ein sich selbst widersprechendes Unding ist. – Der denkende Mensch nämlich, wenn er über die Anreize zum Laster gesiegt hat und seine, oft sauere, Pflicht getan zu haben sich bewußt ist, findet sich in einem Zustande der Seelenruhe und Zufriedenheit, den man gar wohl Glückseligkeit nennen kann; in welchem die Tugend ihr eigener Lohn ist. – Nun sagt der Eudämonist: diese Wonne, diese Glückseligkeit ist der eigentliche Bewegungsgrund, warum er tugendhaft handelt. Nicht der Begriff der  Pflicht bestimme unmittelbar seinen Willen, sondern nur vermittelst der im Prospekt gesehnen Glückseligkeit werde er bewogen, seine Pflicht zu tun. – Nun ist aber klar, daß, weil er sich diesen Tugendlohn nur von dem Bewußtsein, seine Pflicht getan zu haben, versprechen kann, das letztgenannte doch vorangehen müsse; d.i. er muß sich verbunden finden, seine Pflicht zu tun, ehe er noch, und ohne daß er daran denkt, daß Glückseligkeit die Folge der Pflichtbeobachtung sein werde. Er dreht sich mit seiner Ätiologie im Zirkel herum. Er kann nämlich nur hoffen, glücklich (oder innerlich selig) zu sein, wenn er sich seiner Pflichtbeobachtung bewußt ist; er kann aber zur Beobachtung seiner Pflicht nur bewogen werden, wenn er voraussieht, daß er sich dadurch glücklich machen werde. – Aber es ist in dieser Vernünftelei auch ein Widerspruch. Denn einerseits soll er seine Pflicht beobachten, ohne erst zu fragen, welche Wirkung dieses auf seine Glückseligkeit haben werde, mithin aus einem moralischen Grunde; andrerseits aber kann er doch nur etwas für seine Pflicht anerkennen, wenn er auf Glückseligkeit rechnen kann, die ihm dadurch erwachsen wird, mithin nach pathologischem Prinzip, welches gerade das Gegenteil des vorigen ist. 
	
	\subsection*{tg472.2.17} 
	\textbf{Source : }Die Metaphysik der Sitten/Zweiter Teil. Metaphysische Anfangsgründe der Tugendlehre/I. Ethische Elementarlehre/I. Teil. Von den Pflichten gegen sich selbst überhaupt/Erstes Buch. Von den vollkommenen Pflichten gegen sich selbst\\  
	
	\noindent\textbf{Paragraphe : }Kann eine Unwahrheit aus bloßer Höflichkeit (z.B. das ganz gehorsamster Diener am \match{Ende} eines Briefes) für Lüge gehalten werden? Niemand wird ja dadurch betrogen. – Ein Autor fragt einen seiner Leser: wie gefällt Ihnen mein Werk? Die Antwort könnte nun zwar illusorisch gegeben werden, da man über die Verfänglichkeit einer solchen Frage spöttelte; aber wer hat den Witz immer bei der Hand? Das geringste Zögern, mit der Antwort, ist schon Kränkung des Verfassers; darf er diesem also zum Munde reden? 
	
	\subsection*{tg481.2.74} 
	\textbf{Source : }Die Metaphysik der Sitten/Zweiter Teil. Metaphysische Anfangsgründe der Tugendlehre/I. Ethische Elementarlehre/II. Teil. Von den Tugendpflichten gegen andere/Erstes Hauptstück. Von den Pflichten gegen andere, bloß als Menschen/Erster Abschnitt. Von der Liebespflicht gegen andere Menschen\\  
	
	\noindent\textbf{Paragraphe : }Ob zwar aber Mitleid (und so auch Mitfreude) mit anderen zu haben an sich selbst nicht Pflicht ist, so ist es doch tätige Teilnehmung an ihrem Schicksale und zu dem \match{Ende} also indirekte Pflicht, die mitleidige natürliche (ästhetische) Gefühle in uns zu kultivieren, und sie, als so viele Mittel zur Teilnehmung aus moralischen Grundsätzen und dem ihnen gemäßen Gefühl zu benutzen. – So ist es Pflicht: nicht die Stellen, wo sich Arme befinden, denen das Notwendigste abgeht, umzugehen, sondern sie aufzusuchen, die Krankenstuben, oder die Gefängnisse der Schuldener u. dergl. zu fliehen, um dem schmerzhaften Mitgefühl, dessen man sich nicht erwehren könne, auszuweichen; weil dieses doch einer der in uns von der Natur gelegten Antriebe ist, dasjenige zu tun, was die Pflichtvorstellung für sich allein nicht ausrichten würde. 
	
	\unnumberedsection{Erhebung (1)} 
	\subsection*{tg472.2.38} 
	\textbf{Source : }Die Metaphysik der Sitten/Zweiter Teil. Metaphysische Anfangsgründe der Tugendlehre/I. Ethische Elementarlehre/I. Teil. Von den Pflichten gegen sich selbst überhaupt/Erstes Buch. Von den vollkommenen Pflichten gegen sich selbst\\  
	
	\noindent\textbf{Paragraphe : }Aus unserer aufrichtigen und genauen Vergleichung mit dem moralischen Gesetz (dessen Heiligkeit und Strenge) muß unvermeidlich wahre Demut folgen: aber daraus, daß wir einer solchen inneren Gesetzgebung fähig sind, daß der (physische) Mensch den (moralischen) Menschen in seiner eigenen Person zu verehren sich gedrungen fühlt, zugleich \match{Erhebung} und die höchste Selbstschätzung, als Gefühl seines inneren Werts (valor), nach welchem er für keinen Preis (pretium) feil ist, und eine unverlierbare Würde (dignitas interna) besitzt, die ihm Achtung (reverentia) gegen sich selbst einflößt. 
	
	\unnumberedsection{Februar (1)} 
	\subsection*{tg445.2.2} 
	\textbf{Source : }Die Metaphysik der Sitten/Erster Teil. Metaphysische Anfangsgründe der Rechtslehre/Anhang erläutender Bemerkungen zu den metaphysischen Anhangsgründen der Rechtslehre\\  
	
	\noindent\textbf{Paragraphe : }Die Veranlassung zu denselben nehme ich größtenteils von der Rezension dieses Buchs in den Götting. Anz. 28stes Stück, den 18ten \match{Februar} 1797; welche, mit Einsicht und Schärfe der Prüfung, dabei aber doch auch mit Teilnahme und »der Hoffnung, daß jene Anfangsgründe Gewinn für die Wissenschaft bleiben werden,« abgefaßt, ich hier zum Leitfaden der Beurteilung, überdem auch einiger Erweiterung dieses Systems gebrauchen will. 
	
	\unnumberedsection{Feld (1)} 
	\subsection*{tg430.2.47} 
	\textbf{Source : }Die Metaphysik der Sitten/Erster Teil. Metaphysische Anfangsgründe der Rechtslehre/Einleitung in die Metaphysik der Sitten\\  
	
	\noindent\textbf{Paragraphe : }Die Einfachheit dieses Gesetzes in Vergleichung mit den großen und mannigfaltigen Folgerungen, die daraus gezogen werden können, imgleichen das gebietende Ansehen, ohne daß es doch sichtbar eine Triebfeder bei sich führt, muß freilich anfänglich befremden. Wenn man aber, in dieser Verwunderung über ein Vermögen unserer Vernunft, durch die bloße Idee der Qualifikation einer Maxime zur Allgemeinheit eines praktischen Gesetzes die Willkür zu bestimmen,  belehrt wird: daß eben diese praktischen Gesetze (die moralischen) eine Eigenschaft der Willkür zuerst kund machen, auf die keine spekulative Vernunft, weder aus Gründen a priori, noch durch irgend eine Erfahrung, geraten hätte, und, wenn sie darauf geriet, ihre Möglichkeit theoretisch durch nichts dartun könnte, gleichwohl aber jene praktischen Gesetze diese Eigenschaft, nämlich die Freiheit, unwidersprechlich dartun: so wird es weniger befremden, diese Gesetze, gleich mathematischen Postulaten, unerweislich und doch apodiktisch zu finden, zugleich aber ein ganzes \match{Feld} von praktischen Erkenntnissen vor sich eröffnet zu sehen, wo die Vernunft mit derselben Idee der Freiheit, ja jeder anderer ihrer Ideen des Übersinnlichen im Theoretischen alles schlechterdings vor ihr verschlossen finden muß. Die Übereinstimmung einer Handlung mit dem Pflichtgesetze ist die Gesetzmäßigkeit (legalitas) – die der Maxime der Handlung mit dem Gesetze die Sittlichkeit (moralitas) derselben. Maxime aber ist das subjektive Prinzip zu handeln, was sich das Subjekt selbst zur Regel macht (wie es nämlich handeln will). Dagegen ist der Grundsatz der Pflicht das, was ihm die Vernunft schlechthin, mithin objektiv gebietet (wie es handeln soll). 
	
	\unnumberedsection{Folge (15)} 
	\subsection*{tg429.2.4} 
	\textbf{Source : }Die Metaphysik der Sitten/Erster Teil. Metaphysische Anfangsgründe der Rechtslehre/Vorrede\\  
	
	\noindent\textbf{Paragraphe : }
	Ich kann dem so oft gemachten Vorwurf der Dunkelheit, ja wohl gar einer geflissenen, den Schein tiefer Einsicht affektierenden, Undeutlichkeit im philosophischen Vortrage nicht besser zuvorkommen, oder abhelfen, als daß ich, was Herr Garve, ein Philosoph in der echten Bedeutung des Worts, jedem, vornehmlich dem philosophierenden Schriftsteller zur Pflicht macht, bereitwillig annehme, und meinerseits diesen Anspruch bloß auf die Bedingung einschränke, ihm nur so weit \match{Folge} zu leisten, als es die Natur der Wissenschaft erlaubt, die zu berichtigen und zu erweitern ist. 
	
	\subsection*{tg433.2.35} 
	\textbf{Source : }Die Metaphysik der Sitten/Erster Teil. Metaphysische Anfangsgründe der Rechtslehre/1. Teil. Das Privatrecht vom äußeren Mein und Dein überhaupt/1. Hauptstück\\  
	
	\noindent\textbf{Paragraphe : }
	Einem theoretischen Grundsatze a priori müßte nämlich (zu \match{Folge} der Krit. der r. V.) dem gegebenen Begriff eine Anschauung a priori untergelegt, mithin etwas zu dem Begriffe vom Besitz des Gegenstandes hinzugetan werden; allein in diesem praktischen wird umgekehrt verfahren und alle Bedingungen der Anschauung, welche den empirischen Besitz begründen, müssen weggeschafft (von ihnen abgesehen) werden, um den Begriff des Besitzes über den empirischen hinaus zu erweitern und sagen zu können: ein jeder äußere Gegenstand der Willkür kann zu dem rechtlich-Meinen gezählt werden, den ich (und auch nur so fern ich ihn) in meiner Gewalt habe, ohne im Besitz desselben zu sein. 
	
	\subsection*{tg435.2.19} 
	\textbf{Source : }Die Metaphysik der Sitten/Erster Teil. Metaphysische Anfangsgründe der Rechtslehre/1. Teil. Das Privatrecht vom äußeren Mein und Dein überhaupt/2. Hauptstück. Von der Art, etwas Äußeres zu erwerben/1. Abschnitt. Vom Sachrecht\\  
	
	\noindent\textbf{Paragraphe : }Die Möglichkeit, auf solche Art zu erwerben, läßt sich auf keine Weise einsehen, noch durch Gründe dartun, sondern ist die unmittelbare \match{Folge} aus dem Postulat der praktischen Vernunft. Derselbe Wille aber kann doch eine äußere Erwerbung nicht anders berechtigen, als nur so fern er in einem a priori vereinigten (d.i. durch die Vereinigung der Willkür aller, die in ein praktisches Verhältnis gegen einander kommen können) absolut gebietenden Willen enthalten ist; denn der einseitige Wille (wozu auch der doppelseitige, aber doch besondere Wille gehört) kann nicht jedermann eine Verbindlichkeit auflegen, die an sich zufällig ist, sondern dazu wird ein allseitiger nicht zufällig, sondern a priori, mithin notwendig vereinigter und darum allein gesetzgebender Wille erfordert; denn nur nach dieses seinem Prinzip ist Übereinstimmung der freien Willkür eines jeden mit der Freiheit von jedermann, mithin ein Recht überhaupt, und also auch ein äußeres Mein und Dein möglich. 
	
	\subsection*{tg437.2.24} 
	\textbf{Source : }Die Metaphysik der Sitten/Erster Teil. Metaphysische Anfangsgründe der Rechtslehre/1. Teil. Das Privatrecht vom äußeren Mein und Dein überhaupt/2. Hauptstück. Von der Art, etwas Äußeres zu erwerben/3. Abschnitt. Von dem auf dingliche Art persönlichen Recht\\  
	
	\noindent\textbf{Paragraphe : }Die Erwerbung einer Gattin oder eines Gatten geschieht also nicht facto (durch die Beiwohnung) ohne vorhergehenden  Vertrag, auch nicht pacto (durch den bloßen ehelichen Vertrag, ohne nachfolgende Beiwohnung), sondern nur lege: d.i. als rechtliche \match{Folge} aus der Verbindlichkeit, in eine Geschlechtsverbindung nicht anders, als vermittelst des wechselseitigen Besitzes der Personen, als welcher nur durch den gleichfalls wechselseitigen Gebrauch ihrer Geschlechtseigentümlichkeiten seine Wirklichkeit erhält, zu treten. 
	
	\subsection*{tg437.2.50} 
	\textbf{Source : }Die Metaphysik der Sitten/Erster Teil. Metaphysische Anfangsgründe der Rechtslehre/1. Teil. Das Privatrecht vom äußeren Mein und Dein überhaupt/2. Hauptstück. Von der Art, etwas Äußeres zu erwerben/3. Abschnitt. Von dem auf dingliche Art persönlichen Recht\\  
	
	\noindent\textbf{Paragraphe : }Aller Vertrag besteht an sich, d.i. objektiv betrachtet, aus zwei rechtlichen Akten: dem Versprechen und der Annehmung desselben; die Erwerbung durch die letztere (wenn es nicht ein pactum re initum ist, welches Übergabe erfordert) ist nicht ein Teil, sondern die rechtlich notwendige \match{Folge} desselben. – Subjektiv aber erwogen, d.i. als Antwort auf die Frage: ob jene nach der Vernunft notwendige Folge (welche die Erwerbung sein sollte) auch wirklich erfolgen (physische Folge sein) werde, dafür habe ich durch die Annehmung des Versprechens noch keine Sicherheit. Diese ist also, als äußerlich zur Modalität des Vertrages, nämlich der Gewißheit der Erwerbung durch denselben, gehörend, ein Ergänzungsstück zur Vollständigkeit der Mittel zur Erreichung der Absicht des Vertrags, nämlich der Erwerbung. – Es treten zu diesem Behuf drei Personen auf: der Promittent, der Akzeptant und der Kavent; durch welchen letzteren, und seinen besonderen Vertrag mit dem Promittenten, der Akzeptant zwar nichts mehr in Ansehung des Objekts, aber doch der Zwangsmittel gewinnt, zu dem Seinen zu gelangen. 
	
	\subsection*{tg437.2.79} 
	\textbf{Source : }Die Metaphysik der Sitten/Erster Teil. Metaphysische Anfangsgründe der Rechtslehre/1. Teil. Das Privatrecht vom äußeren Mein und Dein überhaupt/2. Hauptstück. Von der Art, etwas Äußeres zu erwerben/3. Abschnitt. Von dem auf dingliche Art persönlichen Recht\\  
	
	\noindent\textbf{Paragraphe : }Die Sache nun, welche Geld heißen soll, muß also selbst so viel Fleiß gekostet haben, um sie hervorzubringen, oder auch anderen Menschen in die Hände zu schaffen, daß dieser demjenigen Fleiß, durch welchen die Ware (in Natur- oder Kunstprodukten) hat erworben werden müssen, und gegen welchen jener ausgetauscht wird, gleich komme. Denn wäre es leichter, den Stoff, der Geld heißt, als die Ware anzuschaffen, so käme mehr Geld zu Markte, als Ware  feilsteht, und weil der Käufer mehr Fleiß auf seine Ware verwenden müßte, als der Käufer, dem das Geld schneller zuströmt: so würde der Fleiß in Verfertigung der Ware und so das Gewerbe überhaupt mit dem Erwerbfleiß, der den öffentlichen Reichtum zu \match{Folge} hat, zugleich schwinden und verkürzt werden. – Daher können Banknoten und Assignaten nicht für Geld angesehen werden, ob sie gleich eine Zeit hindurch die Stelle desselben vertreten; weil es beinahe gar keine Arbeit kostet, sie zu verfertigen, und ihr Wert sich bloß auf die Meinung der ferneren Fortdauer der bisher gelungenen Umsetzung derselben in Barschaft gründet, welche, bei einer etwanigen Entdeckung, daß die letztere nicht in einer zum leichten und sicheren Verkehr hinreichenden Menge da sei, plötzlich verschwindet, und den Ausfall der Zahlung unvermeidlich macht. – So ist der Erwerbfleiß derer, welche die Gold- und Silberbergwerke in Peru, oder Neumexiko anbauen, vornehmlich bei den so vielfältig mißlingenden Versuchen eines vergeblich angewandten Fleißes, im Aufsuchen der Erzgange, wahrscheinlich noch größer, als der auf Verfertigung der Waren in Europa verwendete, und würde, als unvergolten, mithin von selbst nachlassend, jene Länder bald in Armut sinken lassen, wenn nicht der Fleiß Europens dagegen, eben durch diese Materialien gereizt, sich proportionierlich zugleich erweiterte, um bei jenen die Lust zum Bergbau, durch ihnen angebotene Sachen des Luxus, beständig rege zu erhalten; so daß immer Fleiß gegen Fleiß in Konkurrenz kommen. 
	
	\subsection*{tg441.2.15} 
	\textbf{Source : }Die Metaphysik der Sitten/Erster Teil. Metaphysische Anfangsgründe der Rechtslehre/2. Teil. Das öffentliche Recht/1. Abschnitt. Das Staatsrecht\\  
	
	\noindent\textbf{Paragraphe : }Ein jeder Staat enthält drei Gewalten in sich, d.i. den allgemein vereinigten Willen in dreifacher Person (trias politica): die Herrschergewalt (Souveränität), in der des Gesetzgebers, die vollziehende Gewalt, in der des Regierers (zu \match{Folge} dem Gesetz) und die rechtsprechende Gewalt (als Zuerkennung des Seinen eines jeden nach dem Gesetz), in der Person des Richters (potestas legislatoria, rectoria et iudiciaria), gleich den drei Sätzen in einem praktischen Vernunftschluß: dem Obersatz, der das Gesetz jenes  Willens, dem Untersatz, der das Gebot des Verfahrens nach dem Gesetz, d.i. das Prinzip der Subsumtion unter denselben, und dem Schlußsatz, der den Rechtsspruch (die Sentenz) enthält, was im vorkommenden Falle Rechtens ist. 
	
	\subsection*{tg445.2.4} 
	\textbf{Source : }Die Metaphysik der Sitten/Erster Teil. Metaphysische Anfangsgründe der Rechtslehre/Anhang erläutender Bemerkungen zu den metaphysischen Anhangsgründen der Rechtslehre\\  
	
	\noindent\textbf{Paragraphe : }Gleich beim Anfange der Einleitung in die Rechtslehre stößt sich mein scharfprüfender Rezensent an einer Definition. – Was heißt Begehrungsvermögen? Sie ist, sagt der Text, das Vermögen, durch seine Vorstellungen Ursache der Gegenstände dieser Vorstellungen zu sein. – Dieser Erklärung wird entgegengesetzt: »daß sie nichts wird, sobald man von äußeren Bedingungen der \match{Folge} des Begehrens abstrahiert. – Das Begehrungsvermögen ist aber auch dem Idealisten etwas; obgleich diesem die Außenwelt nichts ist.« Antwort: Gibt es aber nicht auch eine heftige, und doch zugleich mit Bewußtsein vergebliche, Sehnsucht (z.B.: wollte Gott, jener Mann lebte noch!), die zwar tatleer, aber doch nicht folgeleer ist, und, zwar nicht an Außendingen, aber doch im Innern des Subjekts selbst mächtig wirkt (krank macht). Eine Begierde als Bestreben (nisus), vermittelst seiner Vorstellungen Ursache zu sein, ist, wenn das Subjekt gleich die Unzulänglichkeit der letzteren zur beabsichtigten Wirkung einsieht, doch immer Kausalität, wenigstens im Innern desselben. – Was hier den Mißverstand ausmacht, ist: daß, da das Bewußtsein seines Vermögens überhaupt (in dem genannten Falle) zugleich das Bewußtsein seines Unvermögens in Ansehung der Außenwelt ist, die Definition auf den Idealisten nicht anwendbar ist; indessen daß doch, da hier bloß von dem Verhältnisse einer Ursache (der  Vorstellung) zur Wirkung (dem Gefühl) überhaupt die Rede ist, die Kausalität der Vorstellung (jene mag äußerlich oder innerlich sein) in Ansehung ihres Gegenstandes im Begriff des Begehrungsvermögens unvermeidlich gedacht werden muß. 
	
	\subsection*{tg448.2.2} 
	\textbf{Source : }Die Metaphysik der Sitten/Zweiter Teil. Metaphysische Anfangsgründe der Tugendlehre\\  
	
	\noindent\textbf{Paragraphe : }
	Ethik bedeutete in den alten Zeiten die Sittenlehre (philosophia moralis) überhaupt, welche man auch die Lehre von den Pflichten benannte. In der \match{Folge} hat man es ratsam gefunden, diesen Namen auf einen Teil der Sittenlehre, nämlich auf die Lehre von den Pflichten, die nicht unter äußeren Gesetzen stehen, allein zu übertragen (dem man im Deutschen den Namen Tugendlehre angemessen gefunden hat): so, daß jetzt das System der allgemeinen Pflichtenlehre in das der Rechtslehre (ius), welche äußerer Gesetze fähig ist, und der Tugendlehre (ethica) eingeteilt wird, die deren nicht fähig ist; wobei es denn auch sein Bewenden haben mag. 
	
	\subsection*{tg449.2.8} 
	\textbf{Source : }Die Metaphysik der Sitten/Zweiter Teil. Metaphysische Anfangsgründe der Tugendlehre/Einleitung/I. Erörterung des Begriffs einer Tugendlehre\\  
	
	\noindent\textbf{Paragraphe : }Aus diesem Grunde kann die Ethik auch als das System der Zwecke der reinen praktischen Vernunft definiert werden, – Zweck und Pflicht unterscheiden die zwei Abteilungen der allgemeinen Sittenlehre. Daß die Ethik Pflichten enthalte, zu deren Beobachtung man von andern nicht (physisch) gezwungen werden kann, ist bloß die \match{Folge} daraus, daß sie eine Lehre der Zwecke ist, weil dazu (sie zu haben) ein Zwang sich selbst widerspricht. 
	
	\subsection*{tg469.2.21} 
	\textbf{Source : }Die Metaphysik der Sitten/Zweiter Teil. Metaphysische Anfangsgründe der Tugendlehre/I. Ethische Elementarlehre/I. Teil. Von den Pflichten gegen sich selbst überhaupt/Einleitung\\  
	
	\noindent\textbf{Paragraphe : }Was aber die Pflicht des Menschen gegen sich selbst, bloß als moralisches Wesen (ohne auf seine Tierheit zu sehen) betrifft, so besteht sie im Formalen, der Übereinstimmung der Maximen seines Willens mit der Würde der Menschheit in seiner Person; also im Verbot, daß er sich selbst des Vorzugs eines moralischen Wesens, nämlich nach Prinzipien zu handeln, d.i. der inneren Freiheit nicht beraube und dadurch zum Spiel bloßer Neigungen, also zur Sache, mache. – Die Laster, welche dieser Pflicht entgegen stehen, sind: die Lüge, der Geiz, und die falsche Demut (Kriecherei). Diese nehmen sich Grundsätze, welche ihrem  Charakter, als moralischer Wesen, d.i. der inneren Freiheit, der angebornen Würde des Menschen geradezu (schon der Form nach) widersprechen, welches so viel sagt: sie machen sich es zum Grundsatz, keinen Grundsatz, und so auch keinen Charakter, zu haben, d.i. sich wegzuwerfen und sich zum Gegenstande der Verachtung zu machen. – Die Tugend, welche allen diesen Lastern entgegen steht, könnte die Ehrliebe (honestas interna, iustum sui aestimium), eine von der Ehrbegierde (ambitio) (welche auch sehr niederträchtig sein kann) himmelweit unterschiedene Denkungsart, genannt werden, wird aber unter dieser Betitelung in der \match{Folge} besonders vorkommen. 
	
	\subsection*{tg472.2.24} 
	\textbf{Source : }Die Metaphysik der Sitten/Zweiter Teil. Metaphysische Anfangsgründe der Tugendlehre/I. Ethische Elementarlehre/I. Teil. Von den Pflichten gegen sich selbst überhaupt/Erstes Buch. Von den vollkommenen Pflichten gegen sich selbst\\  
	
	\noindent\textbf{Paragraphe : }Wenn ich nämlich zwischen Verschwendung und Geiz die gute Wirtschaft als das Mittlere ansehe, und dieses das Mittlere des Grades sein soll: so würde ein Laster in das (contrarie) entgegengesetzte Laster nicht anders übergehen, als durch die Tugend, und so würde diese nichts anders, als ein vermindertes, oder vielmehr verschwindendes Laster sein, und die \match{Folge} wäre in dem gegenwärtigen Fall: daß von den Mitteln des Wohllebens gar keinen Gebrauch zu machen die echte Tugendpflicht sei. 
	
	\subsection*{tg481.2.10} 
	\textbf{Source : }Die Metaphysik der Sitten/Zweiter Teil. Metaphysische Anfangsgründe der Tugendlehre/I. Ethische Elementarlehre/II. Teil. Von den Tugendpflichten gegen andere/Erstes Hauptstück. Von den Pflichten gegen andere, bloß als Menschen/Erster Abschnitt. Von der Liebespflicht gegen andere Menschen\\  
	
	\noindent\textbf{Paragraphe : }Die Liebe wird hier aber nicht als Gefühl (ästhetisch), d.i. als Lust an der Vollkommenheit anderer Menschen, nicht als Liebe des Wohlgefallens, verstanden (denn Gefühle zu haben, dazu kann es keine Verpflichtung durch andere geben), sondern muß als Maxime des Wohlwollens (als praktisch) gedacht werden, welche das Wohltun zur \match{Folge} hat. 
	
	\subsection*{tg481.2.77} 
	\textbf{Source : }Die Metaphysik der Sitten/Zweiter Teil. Metaphysische Anfangsgründe der Tugendlehre/I. Ethische Elementarlehre/II. Teil. Von den Tugendpflichten gegen andere/Erstes Hauptstück. Von den Pflichten gegen andere, bloß als Menschen/Erster Abschnitt. Von der Liebespflicht gegen andere Menschen\\  
	
	\noindent\textbf{Paragraphe : }Würde es mit dem Wohl der Welt überhaupt nicht besser stehen, wenn alle Moralität der Menschen nur auf Rechtspflichten, doch mit der größten Gewissenhaftigkeit, eingeschränkt, das Wohlwollen aber unter die Adiaphora gezählt würde? Es ist nicht so leicht zu übersehen, welche \match{Folge} es auf die Glückseligkeit der Menschen haben dürfte. Aber in diesem Fall würde es doch wenigstens an einer großen moralischen Zierde der Welt, nämlich der Menschenliebe fehlen, welche also für sich, auch ohne die Vorteile (der Glückseligkeit) zu berechnen, die Welt als ein schönes moralisches Ganze in ihrer ganzen Vollkommenheit darzustellen erfordert wird. 
	
	\subsection*{tg486.2.19} 
	\textbf{Source : }Die Metaphysik der Sitten/Zweiter Teil. Metaphysische Anfangsgründe der Tugendlehre/II. Ethische Methodenlehre/1. Abschnitt. Die ethische Didaktik\\  
	
	\noindent\textbf{Paragraphe : }Das experimentale (technische) Mittel der Bildung zur Tugend ist das gute Beispiel an dem Lehrer selbst (von exemplarischer Führung zu sein) und das warnende an andern; denn Nachahmung ist dem noch ungebildeten Menschen die erste Willensbestimmung zu Annehmung von Maximen, die er sich in der \match{Folge} macht. – Die Angewöhnung oder Abgewöhnung ist die Begründung einer beharrlichen Neigung ohne alle Maximen, durch die öftere Befriedigung derselben; und ist ein Mechanism der Sinnesart, statt eines Prinzips der Denkungsart (wobei das Verlernen in der Folge schwerer wird als das Erlernen). – Was aber die Kraft des Exempels (es sei zum Guten oder Bösen) betrifft, was sich dem Hange zur Nachahmung oder Warnung darbietet,
	
	
	22
	so kann das, was uns andere geben, keine Tugendmaxime begründen. Denn diese besteht gerade in der subjektiven Autonomie der praktischen Vernunft eines jeden Menschen, mithin, daß nicht anderer Menschenverhalten, sondern das Gesetz, uns zur Triebfeder dienen müsse. Daher wird der Erzieher seinem verunarteten Lehrling nicht sagen: Nimm ein Exempel an jenem guten (ordentlichen, fleißigen) Knaben! denn das wird jenem nur zur Ursache dienen, diesen zu hassen, weil er durch ihn in ein nachteiliges Licht gestellt wird. Das gute Exempel (der exemplarische Wandel) soll nicht als Muster, sondern nur zum Beweise der Tunlichkeit des Pflichtmäßigen dienen. Also nicht die Vergleichung mit irgend einem andern Menschen (wie er ist), sondern mit der Idee (der Menschheit) wie er sein soll, also mit dem Gesetz, muß dem Lehrer das nie fehlende Richtmaß seiner Erziehung an die Hand geben. 
	
	\unnumberedsection{Fortdaür (3)} 
	\subsection*{tg442.2.37} 
	\textbf{Source : }Die Metaphysik der Sitten/Erster Teil. Metaphysische Anfangsgründe der Rechtslehre/2. Teil. Das öffentliche Recht/2. Abschnitt. Das Völkerrecht\\  
	
	\noindent\textbf{Paragraphe : }Das Recht des Friedens ist 1) das, im Frieden zu sein, wenn in der Nachbarschaft Krieg ist, oder das der Neutralität; 2) sich die \match{Fortdauer} des geschlossenen Friedens zusichern zu lassen, d.i. das der Garantie; 3) zu wechselseitiger Verbindung (Bundsgenossenschaft) mehrerer Staaten, sich gegen alle äußere oder innere etwanige Angriffe gemeinschaftlich zu verteidigen; nicht ein Bund zum Angreifen und innerer Vergrößerung. 
	
	\subsection*{tg445.2.46} 
	\textbf{Source : }Die Metaphysik der Sitten/Erster Teil. Metaphysische Anfangsgründe der Rechtslehre/Anhang erläutender Bemerkungen zu den metaphysischen Anhangsgründen der Rechtslehre\\  
	
	\noindent\textbf{Paragraphe : }
	Stiftung (sanctio testamentaria beneficii perpetui) ist die freiwillige, durch den Staat bestätigte, für gewisse auf einander folgende Glieder desselben, bis zu ihrem gänzlichen Aussterben, errichtete wohltätige Anstalt. – Sie heißt ewig, wenn die Verordnung zu Erhaltung derselben mit der Konstitution des Staats selbst vereinigt ist (denn der Staat muß für ewig angesehen werden); ihre Wohltätigkeit aber ist entweder für das Volk überhaupt oder für einen nach gewissen besonderen Grundsätzen vereinigten Teil desselben, einen Stand oder für eine Familie und die ewige \match{Fortdauer} ihrer Deszendenten abgezweckt. Ein Beispiel vom ersteren sind die Hospitäler, vom zweiten die Kirchen, vom dritten die Orden (geistliche und weltliche), vom vierten die Majorate. 
	
	\subsection*{tg453.2.7} 
	\textbf{Source : }Die Metaphysik der Sitten/Zweiter Teil. Metaphysische Anfangsgründe der Tugendlehre/Einleitung/V. Erläuterung dieser zwei Begriffe\\  
	
	\noindent\textbf{Paragraphe : }Glückseligkeit, d.i. Zufriedenheit mit seinem Zustande, sofern man der \match{Fortdauer} derselben gewiß ist, sich zu wünschen und zu suchen ist der menschlichen Natur unvermeidlich; eben darum aber auch nicht ein Zweck, der zugleich Pflicht ist. – Da einige noch einen Unterschied zwischen einer moralischen und physischen Glückseligkeit machen (deren erstere in der Zufriedenheit mit seiner Person und ihrem eigenen sittlichen Verhalten, also mit dem was man tut, die andere mit dem was die Natur beschert, mithin was man als fremde Gabe genießt, bestehe): so muß man bemerken, daß, ohne den Mißbrauch des Worts hier zu rügen (das schon einen Widerspruch in sich enthält), die erstere Art zu empfinden allein zum vorigen Titel, nämlich dem der  Vollkommenheit, gehöre. – Denn der, welcher sich im bloßen Bewußtsein seiner Rechtschaffenheit glücklich fühlen soll, besitzt schon diejenige Vollkommenheit, die im vorigen Titel für denjenigen Zweck erklärt war, der zugleich Pflicht ist. 
	
	\unnumberedsection{Funktion (1)} 
	\subsection*{tg441.2.27} 
	\textbf{Source : }Die Metaphysik der Sitten/Erster Teil. Metaphysische Anfangsgründe der Rechtslehre/2. Teil. Das öffentliche Recht/1. Abschnitt. Das Staatsrecht\\  
	
	\noindent\textbf{Paragraphe : }Die drei Gewalten im Staate sind also erstlich einander, als so viel moralische Personen, beigeordnet (potestates coordinatae),  d.i. die eine ist das Ergänzungsstück der anderen zur Vollständigkeit (complementum ad sufficientiam) der Staatsverfassung; aber, zweitens, auch einander untergeordnet (subordinatae), so, daß eine nicht zugleich die \match{Funktion} der anderen, der sie zur Hand geht, usurpieren kann, sondern ihr eigenes Prinzip hat, d.i. zwar in der Qualität einer besonderen Person, aber doch unter der Bedingung des Willens einer oberen gebietet; drittens, durch Vereinigung beider jedem Untertanen sein Recht erteilend sein. 
	
	\unnumberedsection{Fuß (2)} 
	\subsection*{tg441.2.59} 
	\textbf{Source : }Die Metaphysik der Sitten/Erster Teil. Metaphysische Anfangsgründe der Rechtslehre/2. Teil. Das öffentliche Recht/1. Abschnitt. Das Staatsrecht\\  
	
	\noindent\textbf{Paragraphe : }Da auch das Kirchenwesen, welches von der Religion, als innerer Gesinnung, die ganz außer dem Wirkungskreise der bürgerlichen Macht ist, sorgfältig unterschieden werden muß (als Anstalt zum öffentlichen Gottesdienst für das Volk, aus welchem dieser auch seinen Ursprung hat, es sei Meinung oder Überzeugung), ein wahres Staatsbedürfnis  wird, sich auch als Untertanen einer höchsten unsichtbaren Macht, der sie huldigen müssen, und die mit der bürgerlichen oft in einen sehr ungleichen Streit kommen kann, zu betrachten: so hat der Staat das Recht, nicht etwa der inneren Konstitutionalgesetzgebung, das Kirchenwesen nach seinem Sinne, wie es ihm vorteilhaft dünkt, einzurichten, den Glauben und gottesdienstliche Formen (ritus) dem Volk vorzuschreiben, oder zu befehlen (denn dieses muß gänzlich den Lehrern und Vorstehern, die es sich selbst gewählt hat, überlassen bleiben), sondern nur das negative Recht, den Einfluß der öffentlichen Lehrer auf das sichtbare, politische gemeine Wesen, der der öffentlichen Ruhe nachteilig sein möchte, abzuhalten, mithin bei dem inneren Streit, oder dem der verschiedenen Kirchen unter einander, die bürgerliche Eintracht nicht in Gefahr kommen zu lassen, welches also ein Recht der Polizei ist. Daß eine Kirche einen gewissen Glauben, und welchen sie haben, oder daß sie ihn unabänderlich erhalten müsse, und sich nicht selbst reformieren dürfe, sind Einmischungen der obrigkeitlichen Gewalt, die unter ihrer Würde sind; weil sie sich dabei, als einem Schulgezänke, auf den \match{Fuß} der Gleichheit mit ihren Untertanen einläßt (der Monarch sich zum Priester macht), die ihr geradezu sagen können, daß sie hievon nichts verstehe; vornehmlich was des letztere, nämlich das Verbot innerer Reformen, betrifft; – denn, was das gesamte Volk nicht über sich selbst beschließen kann, das kann auch der Gesetzgeber nicht über das Volk beschließen. Nun kann aber kein Volk beschließen, in seinen den Glauben betreffenden Einsichten (der Aufklärung) niemals weiter fortzuschreiten, mithin auch sich in Ansehung des Kirchenwesens nie zu reformieren; weil dies der Menschheit in seiner eigenen Person, mithin dem höchsten Rechte desselben entgegen sein würde. Also kann es auch keine obrigkeitliche Gewalt über das Volk beschließen. – – Was aber die Kosten der Erhaltung des Kirchenwesens betrifft, so können diese, aus ebenderselben Ursache, nicht dem Staat, sondern müssen dem Teil des Volks, der sich zu einem oder dem anderen Glauben bekennt, d.i. nur der Gemeine zu Lasten kommen. 
	
	\subsection*{tg472.2.34} 
	\textbf{Source : }Die Metaphysik der Sitten/Zweiter Teil. Metaphysische Anfangsgründe der Tugendlehre/I. Ethische Elementarlehre/I. Teil. Von den Pflichten gegen sich selbst überhaupt/Erstes Buch. Von den vollkommenen Pflichten gegen sich selbst\\  
	
	\noindent\textbf{Paragraphe : }Allein der Mensch als Person betrachtet, d.i. als Subjekt einer moralisch-praktischen Vernunft, ist über allen Preis erhaben; denn als ein solcher (homo noumenon) ist er nicht bloß als Mittel zu anderer ihren, ja selbst seinen eigenen Zwecken, sondern als Zweck an sich seihst zu schätzen, d.i. er besitzt eine Würde (einen absoluten innern Wert), wodurch er allen andern vernünftigen Weltwesen Achtung für ihn abnötigt, sich mit jedem anderen dieser Art messen und auf den \match{Fuß} der Gleichheit schätzen kann. 
	
	\unnumberedsection{Gegenwart (1)} 
	\subsection*{tg433.2.33} 
	\textbf{Source : }Die Metaphysik der Sitten/Erster Teil. Metaphysische Anfangsgründe der Rechtslehre/1. Teil. Das Privatrecht vom äußeren Mein und Dein überhaupt/1. Hauptstück\\  
	
	\noindent\textbf{Paragraphe : }Von dem Besitz (possessio) ist noch der Sitz (sedes) und von der Besitznehmung des Bodens, in der Absicht, ihn dereinst zu erwerben, ist noch die Niederlassung, Ansiedelung (incolatus) unterschieden, welche ein fortdauernder Privatbesitz eines Platzes ist, der von der \match{Gegenwart} des Subjekts auf demselben abhängt. Von einer Niederlassung als einem zweiten rechtlichen Akt, der auf die Besitznehmung folgen, oder auch ganz unterbleiben kann, ist hier nicht die Rede; weil sie kein ursprünglicher, sondern von der Beistimmung anderer abgeleiteter Besitz sein würde. 
	
	\unnumberedsection{Gewicht (1)} 
	\subsection*{tg430.2.16} 
	\textbf{Source : }Die Metaphysik der Sitten/Erster Teil. Metaphysische Anfangsgründe der Rechtslehre/Einleitung in die Metaphysik der Sitten\\  
	
	\noindent\textbf{Paragraphe : }Allein mit den Lehren der Sittlichkeit ist es anders bewandt. Sie gebieten für jedermann, ohne Rücksicht auf seine Neigungen zu nehmen; bloß weil und sofern er frei ist und praktische Vernunft hat. Die Belehrung in ihren Gesetzen ist nicht aus der Beobachtung seiner selbst und der Tierheit in ihm, nicht aus der Wahrnehmung des Weltlaufs geschöpft,  von dem, was geschieht und wie gehandelt wird (obgleich das deutsche Wort Sitten, eben so wie das lateinische mores, nur Manieren und Lebensart bedeutet), sondern die Vernunft gebietet, wie gehandelt werden soll, wenn gleich noch kein Beispiel davon angetroffen würde, auch nimmt sie keine Rücksicht auf den Vorteil, der uns dadurch erwachsen kann, und den freilich nur die Erfahrung lehren könnte. Denn, ob sie zwar erlaubt, unsern Vorteil, auf alle uns mögliche Art, zu suchen, überdem auch sich, auf Erfahrungszeugnisse fußend, von der Befolgung ihrer Gebote, vornehmlich wenn Klugheit dazu kommt, im Durchschnitte größere Vorteile, als von ihrer Übertretung wahrscheinlich versprechen kann, so beruht darauf doch nicht die Autorität ihrer Vorschriften als Gebote, sondern sie bedient sich derselben (als Ratschläge) nur als eines Gegengewichts wider die Verleitungen zum Gegenteil, um den Fehler einer parteiischen Wage in der praktischen Beurteilung vorher auszugleichen und alsdenn allererst dieser, nach dem \match{Gewicht} der Gründe a priori einer reinen praktischen Vernunft, den Ausschlag zu sichern. 
	
	\unnumberedsection{Gleichheit (9)} 
	\subsection*{tg430.2.13} 
	\textbf{Source : }Die Metaphysik der Sitten/Erster Teil. Metaphysische Anfangsgründe der Rechtslehre/Einleitung in die Metaphysik der Sitten\\  
	
	\noindent\textbf{Paragraphe : }Daß man für die Naturwissenschaft, welche es mit den Gegenständen äußerer Sinne zu tun hat, Prinzipien a priori haben müsse, und daß es möglich, ja notwendig sei, ein System dieser Prinzipien, unter dem Namen einer metaphysischen Naturwissenschaft, vor der auf besondere Erfahrungen angewandten, d.i. der Physik, voranzuschicken, ist an einem andern Orte bewiesen worden. Allein die letztere kann (wenigstens wenn es ihr darum zu tun ist, von ihren Sätzen den Irrtum abzuhalten) manches Prinzip auf das Zeugnis der Erfahrung als allgemein annehmen, obgleich das letztere, wenn es in strenger Bedeutung allgemein gelten soll, aus Gründen a priori abgeleitet werden müßte, wie Newton das Prinzip der \match{Gleichheit} der Wirkung und Gegenwirkung im Einflusse der Körper auf einander als auf Erfahrung gegründet annahm, und es gleichwohl über die ganze materielle Natur ausdehnte. Die Chymiker gehen noch weiter und gründen ihre allgemeinste Gesetze der Vereinigung und Trennung der Materien durch ihre eigene Kräfte gänzlich auf Erfahrung, und vertrauen gleichwohl auf ihre Allgemeinheit und Notwendigkeit so, daß sie in den mit ihnen angestellten Versuchen keine Entdeckung eines Irrtums besorgen. 
	
	\subsection*{tg437.2.20} 
	\textbf{Source : }Die Metaphysik der Sitten/Erster Teil. Metaphysische Anfangsgründe der Rechtslehre/1. Teil. Das Privatrecht vom äußeren Mein und Dein überhaupt/2. Hauptstück. Von der Art, etwas Äußeres zu erwerben/3. Abschnitt. Von dem auf dingliche Art persönlichen Recht\\  
	
	\noindent\textbf{Paragraphe : }Daß der Konkubinat keines zu Recht beständigen Kontrakts fähig sei, so wenig als die Verdingung einer Person zum einmaligen Genuß (pactum fornicationis), folgt aus dem obigen Grunde. Denn, was den letzteren Vertrag betrifft: so wird jedermann gestehen, daß die Person, welche ihn geschlossen hat, zur Erfüllung ihres Versprechen rechtlich nicht angehalten werden könnte, wenn es ihr gereuete; und so fällt auch der erstere, nämlich der des Konkubinats (als pactum turpe) weg, weil dieser ein Kontrakt der Verdingung (locatio-conductio) sein würde, und zwar eines Gliedmaßes zum Gebrauch eines anderen, mithin wegen der unzertrennlichen Einheit der  Glieder an einer Person diese sich selbst als Sache der Willkür des anderen hingeben würde; daher jeder Teil den eingegangenen Vertrag mit dem anderen aufheben kann, so bald es ihm beliebt, ohne daß der andere über Läsion seines Rechts gegründete Beschwerde führen kann. – Eben dasselbe gilt auch von der Ehe an der linken Hand, um die Ungleichheit des Standes beider Teile zur größeren Herrschaft des einen Teils über den anderen zu benutzen; denn in der Tat ist sie nach dem bloßen Naturrecht vom Konkubinat nicht unterschieden, und keine wahre Ehe. – Wenn daher die Frage ist: ob es auch der \match{Gleichheit} der Verehlichten, als solcher widerstreite, wenn das Gesetz von dem Manne in Verhältnis auf das Weib sagt: er soll dein Herr (er der befehlende, sie der gehorchende Teil) sein: so kann dieses nicht als der natürlichen Gleichheit eines Menschenpaares widerstreitend angesehen werden, wenn dieser Herrschaft nur die natürliche Überlegenheit des Vermögens des Mannes über das weibliche, in Bewirkung des gemeinschaftlichen Interesse des Hauswesens und des darauf gegründeten Rechts zum Befehl zum Grunde liegt, welches daher selbst aus der Pflicht der Einheit und Gleichheit in Ansehung des Zwecks abgeleitet werden kann. 
	
	\subsection*{tg441.2.21} 
	\textbf{Source : }Die Metaphysik der Sitten/Erster Teil. Metaphysische Anfangsgründe der Rechtslehre/2. Teil. Das öffentliche Recht/1. Abschnitt. Das Staatsrecht\\  
	
	\noindent\textbf{Paragraphe : }Diese Abhängigkeit von dem Willen anderer, und Ungleichheit, ist gleichwohl keinesweges der Freiheit und \match{Gleichheit} derselben als Menschen, die zusammen ein Volk ausmachen, entgegen: vielmehr kann, bloß den Bedingungen derselben gemäß, dieses Volk ein Staat werden, und in eine bürgerliche Verfassung eintreten. In dieser Verfassung aber das Recht der Stimmgebung zu haben, d.i. Staatsbürger, nicht bloß Staatsgenosse zu sein, dazu qualifizieren sich nicht alle mit gleichem Recht. Denn daraus, daß sie fordern können, von allen anderen nach Gesetzen der natürlichen Freiheit und Gleichheit als passive Teile des Staats behandelt zu werden, folgt nicht das Recht, auch als aktive Glieder den Staat selbst  zu behandeln, zu organisieren oder zu Einführung gewisser Gesetze mitzuwirken: sondern nur, daß, welcherlei Art die positiven Gesetze, wozu sie stimmen, auch sein möchten, sie doch den natürlichen der Freiheit und der dieser angemessenen Gleichheit aller im Volk, sich nämlich aus diesem passiven Zustande zu dem aktiven empor arbeiten zu können, nicht zuwider sein müssen. 
	
	\subsection*{tg441.2.42} 
	\textbf{Source : }Die Metaphysik der Sitten/Erster Teil. Metaphysische Anfangsgründe der Rechtslehre/2. Teil. Das öffentliche Recht/1. Abschnitt. Das Staatsrecht\\  
	
	\noindent\textbf{Paragraphe : }Hieraus folgt nun der Satz: der Herrscher im Staat hat gegen den Untertan lauter Rechte und keine (Zwangs-) Pflichten. – Ferner, wenn das Organ des Herrschers, der Regent, auch den Gesetzen zuwider verführe, z.B. mit Auflagen, Rekrutierungen, u. dergl., wider das Gesetz der \match{Gleichheit} in Verteilung der Staatslasten, so darf der Untertan dieser Ungerechtigkeit zwar Beschwerden (gravamina), aber keinen Widerstand entgegensetzen. 
	
	\subsection*{tg441.2.72} 
	\textbf{Source : }Die Metaphysik der Sitten/Erster Teil. Metaphysische Anfangsgründe der Rechtslehre/2. Teil. Das öffentliche Recht/1. Abschnitt. Das Staatsrecht\\  
	
	\noindent\textbf{Paragraphe : }Diese \match{Gleichheit} der Strafen, die allein durch die Erkenntnis des Richters auf den Tod, nach dem strengen Wiedervergeltungsrechte, möglich ist, offenbaret sich daran, daß dadurch allein proportionierlich mit der inneren Bösartigkeit der Verbrecher das Todesurteil über alle (selbst wenn es nicht einen Mord, sondern ein anderes nur mit dem Tode zu tilgendes Staatsverbrechen beträfe) ausgesprochen wird. – Setzet: daß, wie in der letzten schottischen Rebellion, da verschiedene Teilnehmer an derselben (wie Balmerino und andere) durch ihre Empörung nichts als eine dem Hause Stuart schuldige Pflicht auszuüben glaubten, andere dagegen Privatabsichten hegten, von dem höchsten Gericht das Urteil so gesprochen worden wäre: ein jeder solle die Freiheit der Wahl zwischen dem Tode und der Karrenstrafe haben: so sage ich, der ehrliche Mann wählt den Tod, der Schelm aber die Karre; so bringt es die Natur des menschlichen Gemüts mit sich. Denn der erstere kennt etwas, was er noch höher schätzt, als selbst das Leben: nämlich die
	Ehre; der andere hält ein mit Schande bedecktes Leben doch immer noch für besser, als gar nicht zu sein (animam praeferre pudori. Juven.). Der erstere ist nun ohne Widerrede weniger strafbar als der andere, und so werden sie durch den über alle gleich verhängten Tod ganz proportionierlich bestraft, jener gelinde, nach seiner Empfindungsart, und dieser hart, nach der seinigen; da hingegen, wenn durchgängig auf die Karrenstrafe erkannt würde, der erstere zu hart, der andere, für seine Niederträchtigkeit, gar zu gelinde bestraft wäre; und so ist auch hier im Ausspruche über eine im Komplott vereinigte Zahl von Verbrechern der beste Ausgleicher, vor der öffentlichen Gerechtigkeit, der Tod. – Überdem hat man nie gehört, daß ein wegen Mordes zum Tode Verurteilter sich beschwert hätte, daß ihm damit zu viel und also unrecht geschehe, jeder würde ihm ins Gesicht lachen, wenn er sich dessen äußerte. – Man müßte sonst annehmen, daß, wenn dem Verbrecher gleich nach dem Gesetz nicht unrecht geschieht, doch die gesetzgebende Gewalt im Staat diese Art von Strafe zu verhängen nicht befugt, und, wenn sie es tut, mit sich selbst im Widerspruch sei. 
	
	\subsection*{tg442.2.31} 
	\textbf{Source : }Die Metaphysik der Sitten/Erster Teil. Metaphysische Anfangsgründe der Rechtslehre/2. Teil. Das öffentliche Recht/2. Abschnitt. Das Völkerrecht\\  
	
	\noindent\textbf{Paragraphe : }Das Recht nach dem Kriege, d.i. im Zeitpunkte des Friedensvertrags und in Hinsicht auf die Folgen desselben,  besteht darin: der Sieger macht die Bedingungen, über die mit dem Besiegten übereinzukommen und zum Friedensschluß zu gelangen Traktaten gepflogen werden, und zwar nicht gemäß irgend einem vorzuschützenden Recht, was ihm wegen der vorgeblichen Läsion seines Gegners zustehe, sondern, indem er diese Frage auf sich beruhen läßt, sich stützend auf seine Gewalt. Daher kann der Überwinder nicht auf Erstattung der Kriegskosten antragen; weil er den Krieg seines Gegners alsdann für ungerecht ausgeben müßte: sondern, ob er sich gleich dieses Argument denken mag, so darf er es doch nicht anführen, weil er ihn sonst für einen Bestrafungskrieg erklären, und so wiederum eine Beleidigung ausüben würde. Hiezu gehört auch die (auf keinen Loskauf zu stellende) Auswechselung der Gefangenen, ohne auf \match{Gleichheit} der Zahl zu sehen. 
	
	\subsection*{tg481.2.78} 
	\textbf{Source : }Die Metaphysik der Sitten/Zweiter Teil. Metaphysische Anfangsgründe der Tugendlehre/I. Ethische Elementarlehre/II. Teil. Von den Tugendpflichten gegen andere/Erstes Hauptstück. Von den Pflichten gegen andere, bloß als Menschen/Erster Abschnitt. Von der Liebespflicht gegen andere Menschen\\  
	
	\noindent\textbf{Paragraphe : }Dankbarkeit ist eigentlich nicht Gegenliebe des Verpflichteten gegen den Wohltäter, sondern Achtung vor demselben. Denn der allgemeinen Nächstenliebe kann und muß \match{Gleichheit} der Pflichten zum Grunde gelegt werden; in der Dankbarkeit aber steht der Verpflichtete um eine Stufe niedriger als sein Wohltäter. Sollte das nicht die Ursache so mancher Undankbarkeit sein, nämlich der Stolz,  einen über sich zu sehen; der Widerwille, sich nicht in völlige Gleichheit (was die Pflichtverhältnisse betrifft) mit ihm setzen zu können? 
	
	\subsection*{tg484.2.14} 
	\textbf{Source : }Die Metaphysik der Sitten/Zweiter Teil. Metaphysische Anfangsgründe der Tugendlehre/I. Ethische Elementarlehre/II. Teil. Von den Tugendpflichten gegen andere/Beschluß der Elementarlehre. Von der innigsten Vereinigung der Liebe mit der Achtung in der Freundschaft\\  
	
	\noindent\textbf{Paragraphe : }Ein Menschenfreund überhaupt aber (d.i. der ganzen Gattung) ist der, Welcher an dem Wohl aller Menschen ästhetischen Anteil (der Mitfreude) nimmt, und es nie ohne inneres Bedauren stören wird. Doch ist der Ausdruck eines Freundes der Menschen noch von etwas engerer Bedeutung, als der des bloß Menschenliebenden (Philanthrop). Denn in jenem ist auch die Vorstellung und Beherzigung der \match{Gleichheit} unter Menschen, mithin die Idee, dadurch  selbst verpflichtet zu werden, indem man andere durch Wohltun verpflichtet, enthalten; gleichsam als Brüder unter einem allgemeinen Vater, der aller Glückseligkeit will. – Denn das Verhältnis des Beschützers, als Wohltäters, zu dem Beschützten, als Dankpflichtigen, ist zwar ein Verhältnis der Wechselliebe, aber nicht der Freundschaft: weil die schuldige Achtung beider gegen einander nicht gleich ist. Die Pflicht, als Freund den Menschen wohl zu wollen (eine notwendige Herablassung), und die Beherzigung derselben, dient dazu, vor dem Stolz zu verwahren, der die Glücklichen anzuwandeln pflegt, welche das Vermögen wohl zu tun besitzen. 
	
	\subsection*{tg484.2.7} 
	\textbf{Source : }Die Metaphysik der Sitten/Zweiter Teil. Metaphysische Anfangsgründe der Tugendlehre/I. Ethische Elementarlehre/II. Teil. Von den Tugendpflichten gegen andere/Beschluß der Elementarlehre. Von der innigsten Vereinigung der Liebe mit der Achtung in der Freundschaft\\  
	
	\noindent\textbf{Paragraphe : }Ein Freund in der Not, wie erwünscht ist er nicht (wohl zu verstehen, wenn er ein tätiger, mit eigenem Aufwande hülfreicher Freund ist)? Aber es ist doch auch eine große Last, sich an anderer ihrem Schicksal angekettet und mit fremden Bedürfnis beladen zu fühlen. – Die Freundschaft kann also nicht eine auf wechselseitigen Vorteil abgezweckte Verbindung, sondern diese muß rein moralisch sein, und der Beistand, auf den jeder von beiden von dem anderen im Falle der Not rechnen darf, muß nicht als Zweck und Bestimmungsgrund zu derselben – dadurch würde er die Achtung des andern Teils verlieren – sondern kann nur als äußere Bezeichnung des inneren herzlich gemeinten Wohlwollens, ohne es doch auf die Probe, als die immer gefährlich ist, ankommen zu lassen, gemeint sein, indem ein jeder großmütig den anderen dieser Last zu überheben, sie für sich allein zu tragen, ja ihm sie gänzlich zu verhehlen bedacht ist, sich aber immer doch damit schmeicheln kann, daß im Falle der Not er auf den Beistand des andern sicher würde rechnen können. Wenn aber einer von dem andern eine Wohltat annimmt, so kann er wohl vielleicht auf \match{Gleichheit} in der Liebe, aber nicht in der Achtung rechnen, denn er sieht sich offenbar eine Stufe niedriger, verbindlich zu sein und nicht gegenseitig verbinden zu können. – Freundschaft ist, bei der Süßigkeit der Empfindung des bis zum Zusammenschmelzen in eine Person sich annähernden wechselseitigen Besitzes, doch zugleich etwas so Zartes (teneritas amicitiae), daß, wenn man sie auf Gefühle beruhen läßt, und dieser wechselseitigen Mitteilung und Ergebung nicht Grundsätze oder das Gemeinmachen verhütende, und die Wechselliebe durch Foderungen der Achtung einschränkende  Regeln unterlegt, sie keinen Augenblick vor Unterbrechungen sicher ist; dergleichen unter unkultivierten Personen gewöhnlich sind, ob sie zwar darum eben nicht immer Trennung bewirken (denn Pöbel schlägt sich und Pöbel verträgt sich); sie können von einander nicht lassen, aber sich auch nicht unter einander einigen, weil das Zanken selbst ihnen Bedürfnis ist, um die Süßigkeit der Eintracht in der Versöhnung zu schmecken. – Auf alle Fälle aber kann die Liebe in der Freundschaft nicht Affekt sein; weil dieser in der Wahl blind und in der Fortsetzung verrauchend ist. 
	
	\unnumberedsection{Glied (6)} 
	\subsection*{tg430.2.18} 
	\textbf{Source : }Die Metaphysik der Sitten/Erster Teil. Metaphysische Anfangsgründe der Rechtslehre/Einleitung in die Metaphysik der Sitten\\  
	
	\noindent\textbf{Paragraphe : }Das Gegenstück einer Metaphysik der Sitten, als das andere \match{Glied} der Einteilung der praktischen Philosophie überhaupt, würde die moralische Anthropologie sein, welche, aber nur die subjektive, hindernde sowohl, als begünstigende, Bedingungen der Ausführung der Gesetze der ersteren in der menschlichen Natur, die Erzeugung, Ausbreitung und Stärkung moralischer Grundsätze (in der Erziehung der Schul- und Volksbelehrung) und dergleichen andere sich auf Erfahrung gründende Lehren und Vorschriften enthalten würde, und die nicht entbehrt werden kann, aber durchaus nicht vor jener vorausgeschickt, oder mit ihr vermischt werden muß; weil man alsdenn Gefahr läuft, falsche, oder wenigstens nachsichtliche moralische Gesetze herauszubringen, welche das für unerreichbar vorspiegeln, was nur eben darum nicht er reicht wird, weil das Gesetz nicht in seiner Reinigkeit (als worin auch seine Stärke besteht) eingesehen und vorgetragen worden, oder gar unechte, oder unlautere Triebfedern zu dem, was an sich pflichtmäßig und gut ist, gebraucht werden, welche keine sichere moralische Grundsätze übrig lassen; weder zum Leitfaden der Beurteilung, noch zur Disziplin des Gemüts in der Befolgung der Pflicht, deren Vorschrift schlechterdings nur durch reine Vernunft a priori gegeben werden muß. 
	
	\subsection*{tg434.2.14} 
	\textbf{Source : }Die Metaphysik der Sitten/Erster Teil. Metaphysische Anfangsgründe der Rechtslehre/1. Teil. Das Privatrecht vom äußeren Mein und Dein überhaupt\\  
	
	\noindent\textbf{Paragraphe : }3) Nachdem Rechtsgrunde (titulus) der Erwerbung; welches eigentlich kein besonderes \match{Glied} der Einteilung der Rechte, aber doch ein Moment der Art ihrer Ausübung ist: entweder durch den Akt einer einseitigen, oder doppelseitigen, oder allseitigen Willkür, wodurch etwas Äußeres (facto, pacto, lege) er worben wird. 
	
	\subsection*{tg438.2.11} 
	\textbf{Source : }Die Metaphysik der Sitten/Erster Teil. Metaphysische Anfangsgründe der Rechtslehre/1. Teil. Das Privatrecht vom äußeren Mein und Dein überhaupt/2. Hauptstück. Von der Art, etwas Äußeres zu erwerben/Episodischer Abschnitt. Von der idealen Erwerbung eines äußeren Gegenstandes der Willkür\\  
	
	\noindent\textbf{Paragraphe : }Nun kann ihm aber, wenn er ein \match{Glied} des gemeinen Wesens ist, d.i. im bürgerlichen Zustande, der Staat wohl seinen Besitz (stellvertretend) erhalten, ob dieser gleich als Privatbesitz unterbrochen war, und der jetzige Besitzer darf seinen Titel der Erwerbung bis zur ersten nicht beweisen, noch auch sich auf den der Ersitzung gründen. Aber im Naturzustande ist der letztere rechtmäßig, nicht eigentlich eine Sache dadurch zu erwerben, sondern ohne einen rechtlichen Akt sich im Besitz derselben zu erhalten: welche Befreiung von Ansprüchen dann auch Erwerbung genannt zu werden pflegt. – Die Präskription des älteren Besitzers gehört also zum Naturrecht (est iuris naturae). 
	
	\subsection*{tg441.2.19} 
	\textbf{Source : }Die Metaphysik der Sitten/Erster Teil. Metaphysische Anfangsgründe der Rechtslehre/2. Teil. Das öffentliche Recht/1. Abschnitt. Das Staatsrecht\\  
	
	\noindent\textbf{Paragraphe : }Die zur Gesetzgebung vereinigten Glieder einer solchen Gesellschaft (societas civilis), d.i. eines Staats, heißen Staatsbürger (cives), und die rechtlichen, von ihrem Wesen (als solchem) unabtrennlichen Attribute derselben sind gesetzliche Freiheit, keinem anderen Gesetz zu gehorchen, als zu welchem er seine Beistimmung gegeben hat – bürgerliche Gleichheit, keinen Oberen im Volk, in Ansehung seiner zu erkennen, als nur einen solchen, den er eben so rechtlich zu verbinden das moralische Vermögen hat, als dieser ihn verbinden kann, – drittens, das Attribut der bürgerlichen Selbständigkeit, seine Existenz und Erhaltung nicht der Willkür eines anderen im Volke, sondern seinen eigenen Rechten und Kräften, als \match{Glied} des gemeinen Wesens verdanken zu können, folglich die bürgerliche Persönlichkeit, in Rechtsangelegenheiten durch keinen anderen vorgestellt werden zu dürfen. 
	
	\subsection*{tg441.2.20} 
	\textbf{Source : }Die Metaphysik der Sitten/Erster Teil. Metaphysische Anfangsgründe der Rechtslehre/2. Teil. Das öffentliche Recht/1. Abschnitt. Das Staatsrecht\\  
	
	\noindent\textbf{Paragraphe : }Nur die Fähigkeit der Stimmgebung macht die Qualifikation zum Staatsbürger aus; jene aber setzt die Selbständigkeit dessen im Volk voraus, der nicht bloß Teil des gemeinen Wesens, sondern auch \match{Glied} desselben, d.i. aus eigener Willkür in Gemeinschaft mit anderen handelnder  Teil desselben sein will. Die letztere Qualität macht aber die Unterscheidung des aktiven vom passiven Staatsbürger notwendig: obgleich der Begriff des letzteren mit der Erklärung des Begriffs von einem Staatsbürger überhaupt im Widerspruch zu stehen scheint. – Folgende Beispiele können dazu dienen, diese Schwierigkeit zu heben: Der Geselle bei einem Kaufmann, oder bei einem Handwerker; der Dienstbote (nicht der im Dienste des Staats steht); der Unmündige (naturaliter vel civiliter); alles Frauenzimmer, und überhaupt jedermann, der nicht nach eigenem Betrieb, sondern nach der Verfügung anderer (außer der des Staats), genötigt ist, seine Existenz (Nahrung und Schutz) zu erhalten, entbehrt der bürgerlichen Persönlichkeit, und seine Existenz ist gleichsam nur Inhärenz. – Der Holzhacker, den ich auf meinem Hofe anstelle, der Schmied in Indien, der mit seinem Hammer, Amboß und Blasbalg in die Häuser geht, um da in Eisen zu arbeiten, in Vergleichung mit dem europäischen Tischler oder Schmied, der die Produkte aus dieser Arbeit als Ware öffentlich feil stellen kann, der Hauslehrer in Vergleichung mit dem Schulmann, der Zinsbauer in Vergleichung mit dem Pächter u. dergl. sind bloß Handlanger des gemeinen Wesens, weil sie von anderen Individuen befehligt oder beschützt werden müssen, mithin keine bürgerliche Selbständigkeit besitzen. 
	
	\subsection*{tg442.2.15} 
	\textbf{Source : }Die Metaphysik der Sitten/Erster Teil. Metaphysische Anfangsgründe der Rechtslehre/2. Teil. Das öffentliche Recht/2. Abschnitt. Das Völkerrecht\\  
	
	\noindent\textbf{Paragraphe : }Dieser Rechtsgrund aber (der vermutlich den Monarchen auch dunkel vorschweben mag) gilt zwar freilich in Ansehung der Tiere, die ein Eigentum des Menschen sein können, will sich aber doch schlechterdings nicht auf den Menschen, vornehmlich als Staatsbürger, anwenden lassen, der im Staat immer als mit gesetzgebendes \match{Glied} betrachtet werden muß (nicht bloß als Mittel, sondern auch zugleich als Zweck an sich selbst), und der also zum Kriegführen nicht allein überhaupt, sondern auch zu jeder besondern Kriegserklärung, vermittelst seiner Repräsentanten, seine freie Beistimmung geben muß, unter welcher einschränkenden Bedingung allein der Staat über seinen gefahrvollen Dienst disponieren kann. 
	
	\unnumberedsection{Grad (8)} 
	\subsection*{tg430.2.56} 
	\textbf{Source : }Die Metaphysik der Sitten/Erster Teil. Metaphysische Anfangsgründe der Rechtslehre/Einleitung in die Metaphysik der Sitten\\  
	
	\noindent\textbf{Paragraphe : }
	Subjektiv ist der \match{Grad} der Zurechnungsfähigkeit (imputabilitas) der Handlungen nach der Größe der Hindernisse zu schätzen, die dabei haben überwunden werden müssen. – Je größer die Naturhindernisse (der Sinnlichkeit), je kleiner das moralische Hindernis (der Pflicht), desto mehr wird die gute Tat zum Verdienst angerechnet. Z.B. wenn ich einen mir ganz fremden Menschen mit meiner beträchtlichen Aufopferung aus großer Not rette. 
	
	\subsection*{tg441.2.65} 
	\textbf{Source : }Die Metaphysik der Sitten/Erster Teil. Metaphysische Anfangsgründe der Rechtslehre/2. Teil. Das öffentliche Recht/1. Abschnitt. Das Staatsrecht\\  
	
	\noindent\textbf{Paragraphe : }Ohne alle Würde kann nun wohl kein Mensch im Staate sein, denn er hat wenigstens die des Staatsbürgers; außer, wenn er sich durch sein eigenes Verbrechen darum gebracht hat, da er dann zwar im Leben erhalten, aber zum bloßen Werkzeuge der Willkür eines anderen (entweder des Staats, oder eines anderen Staatsbürgers) gemacht wird. Wer nun das letztere ist (was er aber nur durch Urteil und Recht werden kann), ist ein Leibeigener (servus in sensu stricto) und gehört zum Eigentum (dominium) eines anderen, der daher nicht bloß sein Herr (herus), sondern auch sein Eigentümer (dominus) ist, der ihn als eine Sache veräußern und nach Belieben (nur nicht zu schandbaren Zwecken) brauchen, und über seine Kräfte, wenn gleich nicht über sein Leben und Gliedmaßen verfügen (disponieren) kann. Durch einen Vertrag kann sich niemand zu einer solchen Abhängigkeit verbinden, dadurch er aufhört, eine Person zu sein; denn nur als Person kann er einen Vertrag machen. Nun scheint es zwar, ein Mensch könne sich zu gewissen, der Qualität nach erlaubten, dem \match{Grad} nach aber unbestimmten Diensten gegen einen andern (für Lohn, Kost, oder Schutz) verpflichten, durch einen Verdingungsvertrag (locatio conductio), und er werde dadurch bloß Untertan (subiectus), nicht Leibeigener (servus); allein das ist nur ein falscher Schein. Denn, wenn sein Herr befugt ist, die Kräfte seines Untertans nach Belieben zu benutzen, so kann er sie auch (wie es mit den Negern auf den Zuckerinseln der Fall ist) erschöpfen, bis zum Tode oder der Verzweiflung, und jener hat sich seinem Herrn wirklich als Eigentum weggegeben; welches unmöglich ist. – Er kann sich also nur zu, der Qualität und dem Grade nach bestimmten, Arbeiten verdingen: entweder als Tagelöhner, oder ansässiger Untertan; im letzteren Fall, daß er teils, für den Gebrauch des Bodens seines Herrn, statt des Tagelohns. Dienste auf demselben Boden, teils für die eigene Benutzung  desselben bestimmte Abgaben (einen Zins) nach einem Pachtvertrage leistet, ohne sich dabei zum Gutsuntertan (glebae adscriptus) zu machen, als wodurch er seine Persönlichkeit einbüßen würde, mithin eine Zeit- oder Erbpacht gründen kann. Er mag nun aber auch durch sein Verbrechen ein persönlicher Untertan geworden sein, so kann diese Untertänigkeit ihm doch nicht anerben, weil er sie sich nur durch seine eigene Schuld zugezogen hat, und eben so wenig kann der von einem Leibeigenen Erzeugte, wegen der Erziehungskosten, die er gemacht hat, in Anspruch genommen werden, weil Erziehung eine absolute Naturpflicht der Eltern, und, im Falle daß diese Leibeigene waren, der Herren ist, welche mit dem Besitz ihrer Untertanen auch die Pflichten derselben übernommen haben. 
	
	\subsection*{tg441.2.71} 
	\textbf{Source : }Die Metaphysik der Sitten/Erster Teil. Metaphysische Anfangsgründe der Rechtslehre/2. Teil. Das öffentliche Recht/1. Abschnitt. Das Staatsrecht\\  
	
	\noindent\textbf{Paragraphe : }Welche Art aber und welcher \match{Grad} der Bestrafung ist es, welche die öffentliche Gerechtigkeit sich zum Prinzip und Richtmaße macht? Kein anderes, als das Prinzip der Gleichheit (im Stande des Züngleins an der Wage der Gerechtigkeit), sich nicht mehr auf die eine, als auf die andere Seite hinzuneigen. Also: was für unverschuldetes Übel du einem  anderen im Volk zufügst, das tust du dir selbst an. Beschimpfst du ihn, so beschimpfst du dich selbst; bestiehlst du ihn, so bestiehlst du dich selbst; schlägst du ihn, so schlägst du dich selbst; tötest du ihn, so tötest du dich selbst. Nur das Wiedervergeltungsrecht (ius talionis), aber, wohl zu verstehen, vor den Schranken des Gerichts (nicht in deinem Privaturteil), kann die Qualität und Quantität der Strafe bestimmt angeben; alle andere sind hin und her schwankend, und können, anderer sich einmischenden Rücksichten wegen, keine Angemessenheit mit dem Spruch der reinen und strengen Gerechtigkeit enthalten. – Nun scheint es zwar, daß der Unterschied der Stände das Prinzip der Wiedervergeltung Gleiches mit Gleichem nicht verstatte; aber, wenn es gleich nicht nach dem Buchstaben möglich sein kann, so kann es doch der Wirkung nach, respektive auf die Empfindungsart der Vornehmeren, immer geltend bleiben. – So hat z.B. Geldstrafe wegen einer Verbalinjurie gar kein Verhältnis zur Beleidigung, denn, der des Geldes viel hat, kann diese sich wohl einmal zur Lust erlauben; aber die Kränkung der Ehrliebe des einen kann doch dem Wehtun des Hochmuts des anderen sehr gleich kommen: wenn dieser nicht allein öffentlich abzubitten, sondern jenem, ob er zwar niedriger ist, etwa zugleich die Hand zu küssen, durch Urteil und Recht genötigt würde. Eben so, wenn der gewalttätige Vornehme für die Schläge, die er dem niederen aber schuldlosen Staatsbürger zumißt, außer der Abbitte noch zu einem einsamen und beschwerlichen Arrest verurteilt würde, weil hiemit, außer der Ungemächlichkeit, noch die Eitelkeit des Täters schmerzhaft angegriffen, und so durch Beschämung Gleiches mit Gleichem gehörig vergolten würde. – Was heißt das aber: »bestiehlst du ihn, so bestiehlst du dich selbst «? Wer da stiehlt, macht aller anderer Eigentum unsicher; er beraubt sich also (nach dem Recht der Wiedervergeltung) der Sicherheit alles möglichen Eigentums; er hat nichts und kann auch nichts erwerben, will aber doch leben; welches nun nicht anders möglich ist, als daß ihn andere ernähren. Weil dieses aber der Staat nicht umsonst tun wird, so muß er diesem seine  Kräfte zu ihm beliebigen Arbeiten (Karren- oder Zuchthausarbeit) überlassen, und kommt auf gewisse Zeit, oder, nach Befinden, auch auf immer, in den Sklavenstand. – Hat er aber gemordet, so muß er sterben. Es gibt hier Kein Surrogat zur Befriedigung der Gerechtigkeit. Es ist keine Gleichartigkeit zwischen einem noch so kummervollen Leben und dem Tode, also auch keine Gleichheit des Verbrechens und der Wiedervergeltung, als durch den am Täter gerichtlich vollzogenen, doch von aller Mißhandlung, welche die Menschheit in der leidenden Person zum Scheusal machen könnte, befreieten Tod. – Selbst, wenn sich die bürgerliche Gesellschaft mit aller Glieder Einstimmung auflösete (z.B. das eine Insel bewohnende Volk beschlösse, auseinander zu gehen, und sich in alle Welt zu zerstreuen), müßte der letzte im Gefängnis befindliche Mörder vorher hingerichtet werden, damit jedermann das widerfahre, was seine Taten wert sind, und die Blutschuld nicht auf dem Volke hafte, das auf diese Bestrafung nicht gedrungen hat; weil es als Teilnehmer an dieser öffentlichen Verletzung der Gerechtigkeit betrachtet werden kann. 
	
	\subsection*{tg461.2.5} 
	\textbf{Source : }Die Metaphysik der Sitten/Zweiter Teil. Metaphysische Anfangsgründe der Tugendlehre/Einleitung/XIII. Allgemeine Grundsätze der Metaphysik der Sitten in Behandlung einer reinen Tugendlehre\\  
	
	\noindent\textbf{Paragraphe : }Eben so wenig und aus demselben Grunde kann kein Laster überhaupt durch eine größere Ausübung gewisser Absichten als es zweckmäßig ist (e. g. prodigalitas est excessus in consumendis opibus) oder durch die kleinere Bewirkung derselben, als sich schickt (e. g. avaritia est defectus etc.) erklärt werden. Denn, da hiedurch der \match{Grad} gar nicht bestimmt wird, auf diesen aber, ob das Betragen pflichtmäßig sei oder nicht, alles ankommt: so kann es nicht zur Erklärung dienen. 
	
	\subsection*{tg471.2.29} 
	\textbf{Source : }Die Metaphysik der Sitten/Zweiter Teil. Metaphysische Anfangsgründe der Tugendlehre/I. Ethische Elementarlehre/I. Teil. Von den Pflichten gegen sich selbst überhaupt/Erstes Buch. Von den vollkommenen Pflichten gegen sich selbst/Erstes Hauptstück. Die Pflicht des Menschen gegen sich selbst, als einem animalischen Wesen\\  
	
	\noindent\textbf{Paragraphe : }Daß ein solcher naturwidrige Gebrauch (also Mißbrauch) seiner Geschlechtseigenschaft eine und zwar der Sittlichkeit im höchsten \match{Grad} widerstreitende Verletzung der Pflicht wider sich selbst sei, fällt jedem, zugleich mit dem Gedanken von demselben, so fort auf, erregt eine Abkehrung von diesem Gedanken, in der Maße, daß selbst die Nennung  eines solchen Lasters bei seinem eigenen Namen für unsittlich gehalten wird; welches, bei dem des Selbstmords, nicht geschieht, den man, mit allen seinen Greueln (in einer species facti) der Welt vor Augen zu legen im mindesten kein Bedenken trägt; gleich als ob der Mensch überhaupt sich beschämt fühle, einer solchen ihn selbst unter das Vieh herabwürdigenden Behandlung seiner eigenen Person fähig zu sein: so daß selbst die erlaubte (an sich freilich bloß tierische) körperliche Gemeinschaft beider Geschlechter in der Ehe im gesitteten Umgange viel Feinheit veranlaßt und erfodert, um einen Schleier darüber zu werfen, wenn davon gesprochen werden soll. 
	
	\subsection*{tg471.2.30} 
	\textbf{Source : }Die Metaphysik der Sitten/Zweiter Teil. Metaphysische Anfangsgründe der Tugendlehre/I. Ethische Elementarlehre/I. Teil. Von den Pflichten gegen sich selbst überhaupt/Erstes Buch. Von den vollkommenen Pflichten gegen sich selbst/Erstes Hauptstück. Die Pflicht des Menschen gegen sich selbst, als einem animalischen Wesen\\  
	
	\noindent\textbf{Paragraphe : }Der Vernunftbeweis aber der Unzulässigkeit jenes unnatürlichen, und selbst auch des bloß unzweckmäßigen Gebrauchs seiner Geschlechtseigenschaften, als Verletzung (und zwar, was den ersteren betrifft, im höchsten Grade) der Pflicht gegen sich selbst, ist nicht so leicht geführt. – Der Beweisgrund liegt freilich darin, daß der Mensch seine Persönlichkeit dadurch (wegwerfend) aufgibt, indem er sich bloß zum Mittel der Befriedigung tierischer Triebe braucht. Aber der hohe \match{Grad} der Verletzung der Menschheit in seiner eigenen Person durch ein solches Laster in seiner Unnatürlichkeit, da es, der Form (der Gesinnung) nach, selbst das des Selbstmordes noch zu übergehen scheint, ist dabei nicht erklärt. Es sei denn, daß, da die trotzige Wegwerfung seiner selbst im letzteren, als einer Lebenslast, wenigstens nicht eine weichliche Hingebung an tierische Reize ist, sondern Mut erfordert, wo immer noch Achtung für die Menschheit in seiner eigenen Person Platz findet, jene, welche sich gänzlich der tierischen Neigung überläßt, den Menschen zur genießbaren, aber hierin doch zugleich naturwidrigen Sache, d.i. zum ekelhaften Gegenstande macht, und so aller Achtung für sich selbst beraubt. 
	
	\subsection*{tg472.2.42} 
	\textbf{Source : }Die Metaphysik der Sitten/Zweiter Teil. Metaphysische Anfangsgründe der Tugendlehre/I. Ethische Elementarlehre/I. Teil. Von den Pflichten gegen sich selbst überhaupt/Erstes Buch. Von den vollkommenen Pflichten gegen sich selbst\\  
	
	\noindent\textbf{Paragraphe : }Werdet nicht der Menschen Knechte. – Laßt euer Recht nicht ungeahndet von anderen mit Füßen treten. – Macht keine Schulden, für die ihr nicht volle Sicherheit leistet. – Nehmt nicht Wohltaten an, die ihr entbehren könnt, und seid nicht Schmarotzer, oder Schmeichler, oder gar (was freilich nur im \match{Grad} von dem Vorigen unterschieden ist) Bettler. Daher seid wirtschaftlich, damit ihr nicht bettelarm werdet. – Das Klagen und Winseln, selbst das bloße Schreien bei einem körperlichen Schmerz ist euer schon unwert, am meisten, wenn ihr euch bewußt seid, ihn selbst verschuldet zu haben: Daher die Veredlung (Abwendung der Schmach) des Todes eines Delinquenten durch die Standhaftigkeit, mit der er stirbt. – Das Hinknien oder Hinwerfen zur Erde, selbst um die Verehrung himmlischer Gegenstände sich dadurch zu versinnlichen, ist der Menschenwürde zuwider, so wie die Anrufung derselben in gegenwärtigen Bildern; denn ihr demütigt euch alsdann nicht unter einem Ideal, das euch eure eigene Vernunft vorstellt, sondern unter einem Idol, was euer eigenes Gemächsel ist. 
	
	\subsection*{tg481.2.64} 
	\textbf{Source : }Die Metaphysik der Sitten/Zweiter Teil. Metaphysische Anfangsgründe der Tugendlehre/I. Ethische Elementarlehre/II. Teil. Von den Tugendpflichten gegen andere/Erstes Hauptstück. Von den Pflichten gegen andere, bloß als Menschen/Erster Abschnitt. Von der Liebespflicht gegen andere Menschen\\  
	
	\noindent\textbf{Paragraphe : }Was aber die Intension, d.i. den \match{Grad} der Verbindlichkeit zu dieser Tugend betrifft, so ist er nach dem Nutzen, den der Verpflichtete aus der Wohltat gezogen hat, und der Uneigennützigkeit, mit der ihm diese erteilt worden, zu schätzen. Der mindeste Grad ist, gleiche Dienstleistungen dem Wohltäter, der dieser empfänglich (noch lebend) ist, und, wenn er es nicht ist, anderen zu erweisen: eine empfangene Wohltat nicht wie eine Last, deren man gern überhoben sein möchte (weil der so Begünstigte gegen seinen Gönner eine Stufe niedriger steht und dies dessen Stolz kränkt), anzusehen: sondern selbst die Veranlassung dazu als moralische Wohltat aufzunehmen, d.i. als gegebene Gelegenheit, diese Tugend der Menschenliebe, welche, mit der Innigkeit der wohlwollenden Gesinnung zugleich, Zärtlichkeit des Wohlwollens (Aufmerksamkeit auf den kleinsten Grad derselben in der Pflichtvorstellung) ist, zu verbinden und so die Menschenliebe zu kultivieren. 
	
	\unnumberedsection{Große (2)} 
	\subsection*{tg445.2.14} 
	\textbf{Source : }Die Metaphysik der Sitten/Erster Teil. Metaphysische Anfangsgründe der Rechtslehre/Anhang erläutender Bemerkungen zu den metaphysischen Anhangsgründen der Rechtslehre\\  
	
	\noindent\textbf{Paragraphe : }Ob nun jener Begriff »als neues Phänomen am juristischen Himmel« eine stella mirabilis (eine bis zum Stern erster \match{Größe} wachsende, vorher nie gesehene, allmählich aber wieder verschwindende,  vielleicht einmal wiederkehrende Erscheinung), oder bloß eine Sternschnuppe sei? das soll jetzt untersucht werden. 
	
	\subsection*{tg472.2.36} 
	\textbf{Source : }Die Metaphysik der Sitten/Zweiter Teil. Metaphysische Anfangsgründe der Tugendlehre/I. Ethische Elementarlehre/I. Teil. Von den Pflichten gegen sich selbst überhaupt/Erstes Buch. Von den vollkommenen Pflichten gegen sich selbst\\  
	
	\noindent\textbf{Paragraphe : }Das Bewußtsein und Gefühl der Geringfähigkeit seines moralischen Werts in Vergleichung mit dem Gesetz
	ist die Demut (humilitas moralis). Die Überredung von einer \match{Größe} dieses seinen Werts, aber nur aus Mangel der Vergleichung mit dem Gesetz, kann der Tugendstolz (arrogantia moralis) genannt werden. – Die Entsagung alles Anspruchs auf irgend einen moralischen Wert seiner selbst, in der Überredung, sich eben dadurch einen geborgten zu erwerben, ist die sittlich-falsche Kriecherei (humilitas spuria). 
	
	\unnumberedsection{Grundsatz (14)} 
	\subsection*{tg430.2.27} 
	\textbf{Source : }Die Metaphysik der Sitten/Erster Teil. Metaphysische Anfangsgründe der Rechtslehre/Einleitung in die Metaphysik der Sitten\\  
	
	\noindent\textbf{Paragraphe : }
	Die ethische Gesetzgebung (die Pflichten mögen allenfalls auch äußere sein) ist diejenige, welche nicht äußerlich sein kann; die juridische ist, welche auch äußerlich sein kann. So ist es eine äußerliche Pflicht, sein vertragsmäßiges Versprechen zu halten; aber das Gebot, dieses bloß darum zu tun, weil es Pflicht ist, ohne auf eine andere Triebfeder Rücksicht zu nehmen, ist bloß zur innern Gesetzgebung gehörig. Also nicht als besondere Art von Pflicht (eine besondere Art Handlungen, zu denen man verbunden ist) – denn es ist in der Ethik sowohl als im Rechte eine äußere Pflicht –, sondern weil die Gesetzgebung, im angeführten Falle, eine innere ist und keinen äußeren Gesetzgeber haben kann, wird die Verbindlichkeit zur Ethik gezählt. Aus eben dem Grunde werden die Pflichten des Wohlwollens, ob sie gleich äußere Pflichten (Verbindlichkeiten zu äußeren Handlungen) sind, doch zur Ethik gezählt, weil ihre Gesetzgebung nur innerlich sein kann. – Die Ethik hat freilich auch ihre besondern Pflichten (z.B. die gegen sich selbst), aber hat doch auch mit dem Rechte Pflichten, aber nur nicht die Art der Verpflichtung gemein. Denn Handlungen bloß darum, weil es Pflichten sind, ausüben, und den \match{Grundsatz} der Pflicht selbst, woher sie auch komme, zur hinreichenden Triebfeder der Willkür zu machen, ist das Eigentümliche der ethischen Gesetzgebung. So gibt es also zwar viele direkt-ethische Pflichten, aber die innere Gesetzgebung macht auch die übrigen, alle und insgesamt, zu indirekt-ethischen. 
	
	\subsection*{tg430.2.46} 
	\textbf{Source : }Die Metaphysik der Sitten/Erster Teil. Metaphysische Anfangsgründe der Rechtslehre/Einleitung in die Metaphysik der Sitten\\  
	
	\noindent\textbf{Paragraphe : }Der kategorische Imperativ, der überhaupt nur aussagt, was Verbindlichkeit sei, ist: handle nach einer Maxime, welche zugleich als ein allgemeines Gesetz gelten kann. – Deine Handlungen mußt du also zuerst nach ihrem subjektiven Grundsatze betrachten: ob aber dieser \match{Grundsatz} auch objektiv gültig sei, kannst du nur daran erkennen, daß, weil deine Vernunft ihn der Probe unterwirft, durch denselben dich zugleich als allgemein gesetzgebend zu denken, er sich zu einer solchen allgemeinen Gesetzgebung qualifiziere. 
	
	\subsection*{tg430.2.48} 
	\textbf{Source : }Die Metaphysik der Sitten/Erster Teil. Metaphysische Anfangsgründe der Rechtslehre/Einleitung in die Metaphysik der Sitten\\  
	
	\noindent\textbf{Paragraphe : }Der oberste \match{Grundsatz} der Sittenlehre ist also: handle nach einer Maxime, die zugleich als allgemeines Gesetz gelten kann. – Jede Maxime, die sich hiezu nicht qualifiziert, ist der Moral zuwider. 
	
	\subsection*{tg433.2.34} 
	\textbf{Source : }Die Metaphysik der Sitten/Erster Teil. Metaphysische Anfangsgründe der Rechtslehre/1. Teil. Das Privatrecht vom äußeren Mein und Dein überhaupt/1. Hauptstück\\  
	
	\noindent\textbf{Paragraphe : }Der bloße physische Besitz (die Inhabung) des Bodens ist schon ein Recht in einer Sache, obzwar freilich noch nicht hinreichend, ihn als das Meine anzusehen. Beziehungsweise auf andere ist er, als (so viel man weiß) erster Besitz, mit dem Gesetz der äußern Freiheit einstimmig, und zugleich in dem ursprünglichen Gesamtbesitz enthalten, der a priori den Grund der Möglichkeit eines Privatbesitzes enthält; mithin den ersten Inhaber eines Bodens in seinem Gebrauch desselben zu stören eine Läsion. Die erste Besitznehmung hat also einen Rechtsgrund (titulus possessionis) für sich, welcher der ursprünglich gemeinsame Besitz ist, und der Satz: wohl dem, der im Besitz ist (beati possidentes)! weil niemand verbunden ist, seinen Besitz zu beurkunden, ist ein \match{Grundsatz} des natürlichen Rechts, der die erste Besitznehmung als einen rechtlichen Grund zur Erwerbung aufstellt, auf den sich jeder erste Besitzer fußen kann. 
	
	\subsection*{tg439.2.33} 
	\textbf{Source : }Die Metaphysik der Sitten/Erster Teil. Metaphysische Anfangsgründe der Rechtslehre/1. Teil. Das Privatrecht vom äußeren Mein und Dein überhaupt/3. Hauptstück. Von der subjektiv-bedingten Erwerbung durch den Ausspruch einer öffentlichen Gerichtsbarkeit\\  
	
	\noindent\textbf{Paragraphe : }Hier tritt nun wiederum die rechtlich-gesetzgebende Vernunft mit dem \match{Grundsatz} der distributiven Gerechtigkeit ein, die Rechtmäßigkeit des Besitzes, nicht wie sie an sich in Beziehung auf den Privatwillen eines jeden (im natürlichen Zustande), sondern nur wie sie vor einem Gerichtshofe, in einem durch den allgemein-vereinigten Willen entstandenen Zustande (in einem bürgerlichen) abgeurteilt werden würde, zur Richtschnur anzunehmen: wo alsdann die Übereinstimmung mit den formalen Bedingungen der Erwerbung, die an sich nur ein persönliches Recht begründen, zu Ersetzung der materialen Gründe (welche die Ableitung von dem Seinen eines vorhergehenden prätendierenden Eigentümers begründen) als hinreichend postuliert wird, und ein an sich persönliches Recht, vor einen Gerichtshof gezogen, als ein Sachenrecht gilt, z.B. daß das Pferd, was, auf öffentlichem, durchs Polizeigesetz geordneten Markt, jedermann feilsteht, wenn alle Regeln des Kaufs und Verkaufs genau beobachtet worden, mein Eigentum werde (so doch, daß dem wahren Eigentümer das Recht bleibt, den Verkäufer, wegen seines altern unverwirkten Besitzes, in Anspruch zu nehmen), und mein sonst persönliches Recht in ein Sachenrecht, nach welchem ich das Meine, wo ich es finde, nehmen (vindizieren) darf, verwandelt wird, ohne mich auf die Art, wie der Verkäufer dazu gekommen, einzulassen. 
	
	\subsection*{tg441.2.9} 
	\textbf{Source : }Die Metaphysik der Sitten/Erster Teil. Metaphysische Anfangsgründe der Rechtslehre/2. Teil. Das öffentliche Recht/1. Abschnitt. Das Staatsrecht\\  
	
	\noindent\textbf{Paragraphe : }Es ist nicht etwa die Erfahrung, durch die wir von der Maxime der Gewalttätigkeit der Menschen belehrt werden, und ihrer Bösartigkeit, sich, ehe eine äußere machthabende Gesetzgebung erscheint, einander zu befehden, also nicht etwa ein Faktum, welches den öffentlich gesetzlichen Zwang notwendig macht, sondern, sie mögen auch so gutartig und rechtliebend gedacht werden, wie man will, so liegt es doch a priori in der Vernunftidee eines solchen (nicht-rechtlichen) Zustandes, daß, bevor ein öffentlich gesetzlicher Zustand errichtet worden, vereinzelte Menschen, Völker und Staaten niemals vor Gewalttätigkeit gegen einander sicher sein können, und zwar aus jedes seinem eigenen Recht, zu tun, was ihm recht und gut dünkt, und hierin von der Meinung des anderen nicht abzuhängen; mithin das erste, was ihm zu beschließen obliegt, wenn er nicht allen Rechtsbegriffen entsagen will, der \match{Grundsatz} sei: man müsse aus dem Naturzustande, in welchem jeder seinem eigenen Kopfe folgt, herausgehen und sich mit allen anderen (mit denen in Wechselwirkung zu geraten er nicht vermeiden kann) dahin vereinigen, sich einem öffentlich gesetzlichen äußeren Zwange zu unterwerfen, also in einen Zustand treten, darin jedem das, was für das Seine anerkannt werden soll, gesetzlich bestimmt, und durch hinreichende Macht (die nicht die seinige, sondern eine äußere ist) zu Teil wird, d.i. er solle vor allen Dingen in einen bürgerlichen Zustand treten. 
	
	\subsection*{tg445.2.22} 
	\textbf{Source : }Die Metaphysik der Sitten/Erster Teil. Metaphysische Anfangsgründe der Rechtslehre/Anhang erläutender Bemerkungen zu den metaphysischen Anhangsgründen der Rechtslehre\\  
	
	\noindent\textbf{Paragraphe : }Ohne diese Bedingung ist der fleischliche Genuß dem \match{Grundsatz} (wenn gleich nicht immer der Wirkung nach) kannibalisch.  Ob, mit Maul und Zähnen, der weibliche Teil durch Schwängerung, und daraus vielleicht erfolgende, für ihn tödliche, Niederkunft, der männliche aber durch, von öfteren Ansprüchen des Weibes an das Geschlechtsvermögen des Mannes herrührende Erschöpfungen aufgezehrt wird, ist bloß in der Manier zu genießen unterschieden, und ein Teil ist in Ansehung des anderen, bei diesem wechselseitigen Gebrauche der Geschlechtsorganen, wirklich eine verbrauchbare Sache (res fungibilis); zu welcher also sich vermittelst eines Vertrags zu machen es ein gesetzwidriger Vertrag (pactum turpe) sein würde. 
	
	\subsection*{tg455.2.3} 
	\textbf{Source : }Die Metaphysik der Sitten/Zweiter Teil. Metaphysische Anfangsgründe der Tugendlehre/Einleitung/VII. Die ethischen Pflichten sind von weiter, dagegen die Rechtspflichten von enger Verbindlichkeit\\  
	
	\noindent\textbf{Paragraphe : }Die unvollkommenen Pflichten sind also allein Tugendpflichten. Die Erfüllung derselben ist Verdienst (meritum) = + a; ihre Übertretung aber ist nicht so fort Verschuldung (demeritum) = – a, sondern bloß moralischer Unwert = 0, außer, wenn es dem Subjekt \match{Grundsatz} wäre, sich jenen Pflichten nicht zu fügen. Die Stärke des Vorsatzes im ersteren heißt eigentlich allein Tugend (virtus), die Schwäche in der zweiten nicht sowohl Laster (vitium) als vielmehr bloß Untugend, Mangel an moralischer Stärke  (defectus moralis). (Wie das Wort Tugend von taugen, so stammt Untugend von zu nichts taugen.) Eine jede pflichtwidrige Handlung heißt Übertretung (peccatum). Die vorsätzliche aber, die zum Grundsatz geworden ist, macht eigentlich das aus, was man Laster (vitium) nennt. 
	
	\subsection*{tg457.2.7} 
	\textbf{Source : }Die Metaphysik der Sitten/Zweiter Teil. Metaphysische Anfangsgründe der Tugendlehre/Einleitung/IX. Was ist Tugendpflicht\\  
	
	\noindent\textbf{Paragraphe : }Dieser \match{Grundsatz} der Tugendlehre verstattet, als ein kategorischer Imperativ, keinen Beweis, aber wohl eine Deduktion aus der reinen praktischen Vernunft. – Was im Verhältnis der Menschen, zu sich selbst und anderen, Zweck sein kann, das ist Zweck vor der reinen praktischen Vernunft, denn sie ist ein Vermögen der Zwecke überhaupt; in Ansehung derselben indifferent sein, d.i. kein Interesse daran zu nehmen, ist also ein Widerspruch; weil sie alsdann auch nicht die Maximen zu Handlungen (als welche letztere jederzeit einen Zweck enthalten) bestimmen, mithin keine praktische Vernunft sein würde. Die reine Vernunft aber  kann a priori keine Zwecke gebieten, als nur so fern sie solche zugleich als Pflicht ankündigt; welche Pflicht alsdann Tugendpflicht heißt. 
	
	\subsection*{tg469.2.18} 
	\textbf{Source : }Die Metaphysik der Sitten/Zweiter Teil. Metaphysische Anfangsgründe der Tugendlehre/I. Ethische Elementarlehre/I. Teil. Von den Pflichten gegen sich selbst überhaupt/Einleitung\\  
	
	\noindent\textbf{Paragraphe : }1) Es wird daher nur eine objektive Einteilung der Pflichten gegen sich selbst in das Formale und Materiale derselben statt finden; wovon die eine einschränkend (negative Pflichten), die andere erweiternd (positive Pflichten gegen sich selbst) sind: jene, welche dem Menschen in Ansehung des Zwecks seiner Natur verbieten, demselben zuwider zu handeln, mithin bloß auf die moralische Selbsterhaltung, diese, welche gebieten, sich einen gewissen Gegenstand der Willkür zum Zweck zu machen, und auf die Vervollkommnung seiner selbst gehen: von welchen beide zur Tugend entweder als Unterlassungspflichten (sustine et abstine) oder als Begehungspflichten (viribus concessis utere), beide aber als Tugendpflichten gehören. Die erstere gehört zur moralischen Gesundheit (ad esse) des Menschen, so wohl als Gegenstandes seiner äußeren, als seines inneren Sinnes zu Erhaltung seiner Natur in ihrer  Vollkommenheit (als Rezeptivität). Die andere zur moralischen Wohlhabenheit (ad melius esse; opulentia moralis), welche in dem Besitz eines zu allen Zwecken hinreichenden Vermögens besteht, so fern dieses erwerblich ist, und zur Kultur (als tätiger Vollkommenheit) seiner selbst gehört. – Der erstere \match{Grundsatz} der Pflicht gegen sich selbst liegt in dem Spruch: lebe der Natur gemäß (naturae convenienter vive), d.i. erhalte dich in der Vollkommenheit deiner Natur, der zweite in dem Satz: mache dich vollkommner, als die bloße Natur dich schuf (perfice te ut finem; perfice te ut medium). 
	
	\subsection*{tg472.2.11} 
	\textbf{Source : }Die Metaphysik der Sitten/Zweiter Teil. Metaphysische Anfangsgründe der Tugendlehre/I. Ethische Elementarlehre/I. Teil. Von den Pflichten gegen sich selbst überhaupt/Erstes Buch. Von den vollkommenen Pflichten gegen sich selbst\\  
	
	\noindent\textbf{Paragraphe : }Unredlichkeit ist bloß Ermangelung an Gewissenhaftigkeit, d.i. an Lauterkeit des Bekenntnisses vor seinem inneren Richter, der als eine andere Person gedacht wird, wenn diese in ihrer höchsten Strenge betrachtet wird, wo ein Wunsch (aus Selbstliebe) für die Tat genommen wird, weil er einen an sich guten Zweck vor sich hat, und die innere Lüge, ob sie zwar der Pflicht des Menschen gegen sich selbst zuwider ist, erhält hier den Namen einer Schwachheit, so wie der Wunsch eines Liebhabers, lauter gute. Eigenschaften an seiner Geliebten zu finden, ihm ihre augenscheinliche Fehler unsichtbar macht. – Indessen verdient diese Unlauterkeit in Erklärungen, die man gegen sich selbst verübt, doch die ernstlichste Rüge: weil, von einer solchen faulen Stelle (der Falschheit, welche in der menschlichen Natur gewurzelt zu sein scheint) aus, das Übel der Unwahrhaftigkeit sich auch in Beziehung auf andere Menschen verbreitet, nachdem einmal der oberste \match{Grundsatz} der Wahrhaftigkeit verletzt worden. – 
	
	\subsection*{tg472.2.26} 
	\textbf{Source : }Die Metaphysik der Sitten/Zweiter Teil. Metaphysische Anfangsgründe der Tugendlehre/I. Ethische Elementarlehre/I. Teil. Von den Pflichten gegen sich selbst überhaupt/Erstes Buch. Von den vollkommenen Pflichten gegen sich selbst\\  
	
	\noindent\textbf{Paragraphe : }Also ist das eigentümliche Merkmal des letzteren Lasters der \match{Grundsatz} des Besitzes der Mittel zu allerlei Zwecken, doch mit dem Vorbehalt, keines derselben für sich brauchen zu wollen und sich so des angenehmen Lebensgenusses zu berauben: welches der Pflicht gegen sich selbst in Ansehung  des Zwecks gerade entgegengesetzt ist.
	
	
	19
	Verschwendung und Kargheit sind also nicht durch den Grad, sondern spezifisch durch die entgegengesetzte Maximen von einander unterschieden. 
	
	\subsection*{tg489.2.19} 
	\textbf{Source : }Die Metaphysik der Sitten/Fußnoten\\  
	
	\noindent\textbf{Paragraphe : }Der Grund des Schauderhaften, bei dem Gedanken von der förmlichen Hinrichtung eines Monarchen durch sein Volk, ist also der, daß der Mord nur als Ausnahme von der Regel, welche dieses sich zur Maxime machte, die Hinrichtung aber als eine völlige Umkehrung der Prinzipien des Verhältnisses zwischen Souverän und Volk (dieses, was sein Dasein nur der Gesetzgebung des ersteren zu verdanken hat, zum Herrscher über jenen zu machen) gedacht werden muß, und so die Gewalttätigkeit mit dreuster Stirn und nach Grundsätzen über das heiligste Recht erhoben wird; welches, wie ein alles ohne Wiederkehr verschlingender Abgrund, als ein vom Staate an ihm verübter Selbstmord, ein keiner Entsündigung fähiges Verbrechen zu sein scheint. Man hat also Ursache anzunehmen, daß die Zustimmung zu solchen Hinrichtungen wirklich nicht aus einem vermeint-rechtlichen Prinzip, sondern aus Furcht vor Rache des vielleicht dereinst wiederauflebenden Staats am Volk herrührte, und jene Förmlichkeit nur vorgenommen worden, um jener Tat den Anstrich von Bestrafung, mithin eines rechtlichen Verfahrens (dergleichen der Mord nicht sein würde) zu geben, welche Bemäntelung aber verunglückt, weil eine solche Anmaßung des Volks noch ärger ist, als selbst der Mord, da diese einen \match{Grundsatz} enthält, der selbst die Wiedererzeugung eines umgestürzten Staats unmöglich machen müßte. 
	
	\subsection*{tg489.2.29} 
	\textbf{Source : }Die Metaphysik der Sitten/Fußnoten\\  
	
	\noindent\textbf{Paragraphe : }
	
	13 Ein der praktischen Philosophie Kundiger ist darum eben nicht ein praktischer Philosoph. Der letztere ist derjenige, welcher sich den Vernunftendzweck zum \match{Grundsatz} seiner Handlungen macht, indem er damit zugleich das dazu nötige Wissen verbindet; welches, da es aufs Tun abgezweckt ist, nicht eben bis zu den subtilsten faden der Metaphysik ausgesponnen werden darf, wenn es nicht etwan eine Rechtspflicht betrifft – als bei welcher auf der Wage der Gerechtigkeit das Mein und Dein, nach dem Prinzip der Gleichheit der Wirkung und Gegenwirkung, genau bestimmt werden und darum der mathematischen Abgemessenheit analog sein muß; – sondern eine bloße Tugendpflicht angeht. Denn da kommt es nicht bloß darauf an, zu wissen, was zu tun Pflicht ist (welches, wegen der Zwecke, die natürlicherweise alle Menschen haben, leicht angegeben werden kann): sondern vornehmlich auf dem inneren Prinzip des Willens, nämlich daß das Bewußtsein dieser Pflicht zugleich Triebfeder der Handlungen sei, um von dem, der mit seinem Wissen dieses Weisheitsprinzip verknüpft, zu sagen: daß er ein praktischer Philosoph sei. 
	
	\unnumberedsection{Identitat (1)} 
	\subsection*{tg430.2.39} 
	\textbf{Source : }Die Metaphysik der Sitten/Erster Teil. Metaphysische Anfangsgründe der Rechtslehre/Einleitung in die Metaphysik der Sitten\\  
	
	\noindent\textbf{Paragraphe : }
	Person ist dasjenige Subjekt, dessen Handlungen einer Zurechnung fähig sind. Die moralische Persönlichkeit ist also nichts anders, als die Freiheit eines vernünftigen Wesens unter moralischen Gesetzen (die psychologische aber bloß das Vermögen, sich seiner selbst in den verschiedenen Zuständen, der \match{Identität} seines Daseins bewußt zu werden), woraus dann folgt, daß eine Person keinen anderen  Gesetzen, als denen, die sie (entweder allein, oder wenigstens zugleich mit anderen) sich selbst gibt, unterworfen ist. 
	
	\unnumberedsection{Induktion (1)} 
	\subsection*{tg430.2.15} 
	\textbf{Source : }Die Metaphysik der Sitten/Erster Teil. Metaphysische Anfangsgründe der Rechtslehre/Einleitung in die Metaphysik der Sitten\\  
	
	\noindent\textbf{Paragraphe : }Wenn die Sittenlehre nichts als Glückseligkeitslehre wäre, so würde es ungereimt sein, zum Behuf derselben sich nach Prinzipien a priori umzusehen. Denn so scheinbar es immer auch lauten mag: daß die Vernunft noch vor der Erfahrung einsehen könne, durch welche Mittel man zum dauerhaften Genuß wahrer Freuden des Lebens gelangen könne, so ist doch alles, was man darüber a priori lehrt, entweder tautologisch, oder ganz grundlos angenommen. Nur die Erfahrung kann lehren, was uns Freude bringe. Die natürlichen Triebe zur Nahrung, zum Geschlecht, zur Ruhe, zur Bewegung, und (bei der Entwickelung unserer Naturanlagen) die Triebe zur Ehre, zur Erweiterung unserer Erkenntnis u. d. gl., können allein und einem jeden nur auf seine besondere Art zu erkennen geben, worin er jene Freuden zu setzen, ebendieselbe kann ihm auch die Mittel lehren, wodurch er sie zu suchen habe. Alles scheinbare Vernünfteln a priori ist hier im Grunde nichts, als durch \match{Induktion} zur Allgemeinheit erhobene Erfahrung, welche Allgemeinheit (secundum principia generalia non universalia) noch dazu so kümmerlich ist, daß man einem jeden unendlich viel Ausnahmen erlauben muß, um jene Wahl seiner Lebensweise seiner besondern Neigung und seiner Empfänglichkeit für die Vergnügen anzupassen, und am Ende doch nur durch seinen, oder anderer ihren Schaden klug zu werden. 
	
	\unnumberedsection{Konstruktion (1)} 
	\subsection*{tg431.2.24} 
	\textbf{Source : }Die Metaphysik der Sitten/Erster Teil. Metaphysische Anfangsgründe der Rechtslehre/Einleitung in die Rechtslehre\\  
	
	\noindent\textbf{Paragraphe : }Das Gesetz eines mit jedermanns Freiheit notwendig zusammenstimmenden wechselseitigen Zwanges, unter dem Prinzip der allgemeinen Freiheit, ist gleichsam die \match{Konstruktion} jenes Begriffs, d.i. Darstellung desselben in einer reinen Anschauung a priori, nach der Analogie der Möglichkeit freier Bewegungen der Körper unter dem Gesetze der Gleichheit der Wirkung und Gegenwirkung. So wie wir nun in der reinen Mathematik die Eigenschaften ihres Objekts nicht unmittelbar vom Begriffe ableiten, sondern nur durch die Konstruktion des Begriffs entdecken können, so ist's nicht sowohl der Begriff des Rechts, als vielmehr der, unter allgemeine Gesetze gebrachte, mit ihm zusammenstimmende durchgängig wechselseitige und gleiche Zwang, der die Darstellung jenes Begriffs möglich macht. Dieweil aber diesem dynamischen Begriffe noch ein bloß formaler, in der reinen Mathematik (z.B. der Geometrie) zum Grunde liegt: so hat die Vernunft dafür gesorgt, den Verstand auch mit Anschauungen a priori, zum Behuf der Konstruktion des Rechtsbegriffs, so viel möglich zu versorgen. – Das Rechte (rectum) wird, als das Gerade, teils dem Krummen, teils dem Schiefen entgegen gesetzt. Das erste ist die innere Beschaffenheit einer Linie von der Art, daß es zwischen zweien gegebenen Punkten nur eine einzige, das zweite aber die Lage zweier einander durchschneidenden oder zusammenstoßenden Linien, von deren Art es auch nur eine einzige (die senkrechte) geben kann, die sich nicht mehr nach einer Seite, als der andern hinneigt, und die den Raum von beiden Seiten gleich abteilt, nach welcher Analogie auch die Rechtslehre das Seine einem jeden (mit mathematischer Genauigkeit) bestimmt wissen will, welches in der Tugendlehre nicht erwartet werden darf, als welche einen gewissen Raum zu Ausnahmen (latitudinem) nicht verweigern kann. – Aber, ohne ins Gebiet der Ethik einzugreifen, gibt es zwei Fälle, die auf Rechtsentscheidung  Anspruch machen, für die aber keiner, der sie entscheide, ausgefunden werden kann, und die gleichsam in Epikurs Intermundia hingehören. – Diese müssen wir zuvörderst aus der eigentlichen Rechtslehre, zu der wir bald schreiten wollen, aussondern, damit ihre schwankenden Prinzipien nicht auf die festen Grundsätze der erstern Einfluß bekommen. 
	
	\unnumberedsection{Korper (2)} 
	\subsection*{tg435.2.38} 
	\textbf{Source : }Die Metaphysik der Sitten/Erster Teil. Metaphysische Anfangsgründe der Rechtslehre/1. Teil. Das Privatrecht vom äußeren Mein und Dein überhaupt/2. Hauptstück. Von der Art, etwas Äußeres zu erwerben/1. Abschnitt. Vom Sachrecht\\  
	
	\noindent\textbf{Paragraphe : }Was die \match{Körper} auf einem Boden betritt, der schon der meinige ist, so gehören sie, wenn sie sonst keines anderen sind, mir zu, ohne daß ich zu diesem Zweck eines besonderen rechtlichen Akts bedürfte (nicht facto sondern lege); nämlich, weil sie als der Substanz inhärierende Akzidenzen betrachtet werden können (iure rei  meae), wozu auch alles gehört, was mit meiner Sache so verbunden ist, daß ein anderer sie von dem Meinen nicht trennen kann, ohne dieses selbst zu verändern (z.B. Vergoldung, Mischung eines mir zugehörigen Stoffes mit andern Materien, Anspülung oder auch Veränderung des anstoßenden Strombettes, und dadurch geschehende Erweiterung meines Bodens, u.s.w.). Ob aber der erwerbliche Boden sich noch weiter als das Land, nämlich auch auf eine Strecke des Seegrundes hin aus (das Recht, noch an meinen Ufern zu fischen, oder Bernstein herauszubringen, u. dergl.), sich ausdehnen lasse, muß nach ebendenselben Grundsätzen beurteilt werden. So weit ich aus meinem Sitze mechanisches Vermögen habe, meinen Boden gegen den Eingriff anderer zu sichern (z.B. so weit die Kanonen vom Ufer abreichen), gehört zu meinem Besitz und das Meer ist bis dahin geschlossen (mare clausum). Da aber auf dem weiten Meere selbst kein Sitz möglich ist, so kann der Besitz auch nicht bis dahin ausgedehnt werden und offene See ist frei (mare liberum). Das Stranden aber, es sei der Menschen, oder der ihnen zugehörigen Sachen, kann, als unvorsätzlich, von dem Strandeigentümer nicht zum Erwerbrecht gezählt werden; weil es nicht Läsion (ja überhaupt kein Faktum) ist, und die Sache, die auf einen Boden geraten ist, der doch irgend einem angehört, nicht als res nullius behandelt werden kann. Ein Fluß dagegen kann, so weit der Besitz seines Ufers reicht, so gut wie ein jeder Landboden, unter obbenannten Einschränkungen ursprünglich von dem erworben werden, der im Besitz beider Ufer ist. 
	
	\subsection*{tg469.2.17} 
	\textbf{Source : }Die Metaphysik der Sitten/Zweiter Teil. Metaphysische Anfangsgründe der Tugendlehre/I. Ethische Elementarlehre/I. Teil. Von den Pflichten gegen sich selbst überhaupt/Einleitung\\  
	
	\noindent\textbf{Paragraphe : }Die Einteilung kann nur in Ansehung des Objekts der Pflicht, nicht in Ansehung des sich verpflichtenden Subjekts, gemacht werden. Das verpflichtete so wohl als das verpflichtende Subjekt ist immer nur der Mensch, und wenn es uns, in theoretischer Rücksicht, gleich erlaubt ist, im Menschen Seele und \match{Körper} als Naturbeschaffenheiten des Menschen von einander zu unterscheiden, so ist es doch nicht erlaubt, sie als verschiedene den Menschen verpflichtende Substanzen zu denken, um zur Einteilung in Pflichten gegen den Körper und gegen die Seele berechtigt zu sein. – Wir sind, weder durch Erfahrung, noch durch Schlüsse der Vernunft, hinreichend darüber belehrt, ob der Mensch eine Seele (als in ihm wohnende, vom Körper unterschiedene und von diesem unabhängig zu denken vermögende, d.i. geistige Substanz) enthalte, oder ob nicht vielmehr das Leben eine Eigenschaft der Materie sein möge, und wenn es sich auch auf die erstere Art verhielte, so würde doch keine Pflicht des Men schen gegen einen Körper (als verpflichtendes Subjekt), ob er gleich der menschliche ist, denkbar sein. 
	
	\unnumberedsection{Kraft (5)} 
	\subsection*{tg433.2.57} 
	\textbf{Source : }Die Metaphysik der Sitten/Erster Teil. Metaphysische Anfangsgründe der Rechtslehre/1. Teil. Das Privatrecht vom äußeren Mein und Dein überhaupt/1. Hauptstück\\  
	
	\noindent\textbf{Paragraphe : }Das Naturrecht im Zustande einer bürgerlichen Verfassung (d.i. dasjenige, was für die letztere aus Prinzipien a priori abgeleitet werden kann) kann durch die statutarischen Gesetze der letzteren nicht Abbruch leiden, und so bleibt das rechtliche Prinzip in Kraft: »der, welcher nach einer Maxime verfährt, nach der es unmöglich wird, einen Gegenstand meiner Willkür als das Meine zu haben, lädiert mich«; denn bürgerliche Verfassung ist allein der rechtliche Zustand, durch welchen jedem das Seine nur gesichert, eigentlich aber nicht ausgemacht und bestimmt wird. – Alle Garantie setzt also das Seine von jemanden (dem es gesichert wird) schon voraus. Mithin muß vor der bürgerlichen Verfassung (oder von ihr abgesehen) ein äußeres Mein und Dein als möglich angenommen werden, und zugleich ein Recht, jedermann, mit dem wir irgend auf eine Art in Verkehr kommen könnten, zu nötigen, mit uns in eine Verfassung zusammen zu treten, worin jenes gesichert werden kann. – Ein Besitz in Erwartung und Vorbereitung eines solchen Zustandes, der allein auf einem Gesetz des gemeinsamen Willens gegründet werden kann, der also zu der Möglichkeit des letzteren zusammenstimmt, ist ein provisorisch-rechtlicher Besitz, wogegen derjenige, der in  einem solchen wirklichen Zustande angetroffen wird, ein peremtorischer Besitz sein würde. – Vor dem Eintritt in diesen Zustand, zu dem das Subjekt bereit ist, widersteht er denen mit Recht, die dazu sich nicht bequemen und ihn in seinem einstweiligen Besitz stören wollen; weil der Wille aller anderen, außer ihm selbst, der ihm eine Verbindlichkeit aufzulegen denkt, von einem gewissen Besitz abzustehen, bloß einseitig ist, mithin eben so wenig gesetzliche \match{Kraft} (als die nur im allgemeinen Willen angetroffen wird) zum Widersprechen hat, als jener zum Behaupten, indessen daß der letztere doch dies voraus hat, zur Einführung und Errichtung eines bürgerlichen Zustandes zusammenzustimmen. – Mit einem Worte: die Art, etwas Äußeres als das Seine im Naturzustande zu haben, ist ein physischer Besitz, der die rechtliche Präsumtion für sich hat, ihn, durch Vereinigung mit dem Willen aller in einer öffentlichen Gesetzgebung, zu einem rechtlichen zu machen, und gilt in der Erwartung komparativ für einen rechtlichen. 
	
	\subsection*{tg447.2.5} 
	\textbf{Source : }Die Metaphysik der Sitten/Zweiter Teil. Metaphysische Anfangsgründe der Tugendlehre/Vorrede\\  
	
	\noindent\textbf{Paragraphe : }In dieser Philosophie (der Tugendlehre) scheint es nun der Idee derselben gerade zuwider zu sein, bis zu metaphysischen Anfangsgründen zurückzugehen, um den Pflichtbegriff, von allem Empirischen (jedem Gefühl) gereinigt, doch zur Triebfeder zu machen. Denn was kann man sich für einen Begriff von einer \match{Kraft} und herkulischer Stärke  machen, um die lastergebärende Neigungen zu überwältigen, wenn die Tugend ihre Waffen aus der Rüstkammer der Metaphysik entlehnen soll? welche eine Sache der Spekulation ist, die nur wenig Menschen zu handhaben wissen. Daher fallen auch alle Tugendlehren, in Hörsälen, von Kanzeln und in Volksbüchern, wenn sie mit metaphysischen Brocken ausgeschmückt werden, ins Lächerliche. – Aber darum ist es doch nicht unnütz, vielweniger lächerlich, den ersten Gründen der Tugendlehre in einer Metaphysik nachzuspüren; denn irgend einer muß doch als Philosoph auf die ersten Gründe dieses Pflichtbegriffs hinausgehen: weil sonst weder Sicherheit noch Lauterkeit für die Tugendlehre überhaupt zu erwarten wäre. Sich desfalls auf ein gewisses Gefühl, welches man, seiner davon erwarteten Wirkung halber, moralisch nennt, zu verlassen, kann auch wohl dem Volkslehrer gnügen: indem dieser zum Probierstein einer Tugendpflicht, ob sie es sei oder nicht, die Aufgabe zu beherzigen verlangt: »wie, wenn nun ein jeder in jedem Fall deine Maxime zum allgemeinen Gesetz machte, würde eine solche wohl mit sich selbst zusammenstimmen können?« Aber, wenn es bloß Gefühl wäre, was auch diesen Satz zum Probierstein zu nehmen uns zur Pflicht machte, so wäre diese doch alsdann nicht durch die Vernunft diktiert, sondern nur instinktmäßig, mithin blindlings dafür angenommen. 
	
	\subsection*{tg447.2.6} 
	\textbf{Source : }Die Metaphysik der Sitten/Zweiter Teil. Metaphysische Anfangsgründe der Tugendlehre/Vorrede\\  
	
	\noindent\textbf{Paragraphe : }Allein kein moralisches Prinzip gründet sich in der Tat, wie man wohl wähnt, auf irgend ein Gefühl, sondern ist wirklich nichts anders, als dunkel gedachte Metaphysik, die jedem Menschen in seiner Vernunftanlage beiwohnt; wie der Lehrer es leicht gewahr wird, der seinen Lehrling über den Pflichtimperativ, und dessen Anwendung auf moralische Beurteilung seiner Handlungen, sokratisch zu katechisieren versucht. – Der Vortrag desselben (die Technik) darf eben nicht allemal metaphysisch und die Sprache scholastisch sein, wenn jener den Lehrling nicht etwa zum Philosophen bilden will. Aber der Gedanke muß bis auf die Elemente der Metaphysik zurück gehen, ohne die keine Sicherheit und Reinigkeit, ja selbst nicht einmal bewegende \match{Kraft} in der Tugendlehre zu erwarten ist. 
	
	\subsection*{tg461.2.14} 
	\textbf{Source : }Die Metaphysik der Sitten/Zweiter Teil. Metaphysische Anfangsgründe der Tugendlehre/Einleitung/XIII. Allgemeine Grundsätze der Metaphysik der Sitten in Behandlung einer reinen Tugendlehre\\  
	
	\noindent\textbf{Paragraphe : }Die Tugend, in ihrer ganzen Vollkommenheit betrachtet, wird also vorgestellt, nicht wie der Mensch die Tugend, sondern als ob die Tugend den Menschen besitze; weil es im ersteren Falle so aussehen würde, als ob er noch die Wahl gehabt hätte (wozu er als dann noch einer andern Tugend bedürfen würde, um die Tugend vor jeder anderer angebotenen Ware zu erlesen). – Eine Mehrheit der Tugenden sich zu denken (wie es denn unvermeidlich ist) ist nichts anderes, als sich verschiedne moralische Gegenstände denken, auf die der Wille aus dem einigen Prinzip der Tugend geleitet wird; eben so ist es mit den entgegenstehenden Lastern bewandt. Der Ausdruck, der beide verpersönlicht, ist eine ästhetische Maschinerie, die aber doch auf einen moralischen Sinn hinweiset. – Daher ist eine Ästhetik der Sitten zwar nicht ein Teil, aber doch eine subjektive Darstellung der Metaphysik derselben: wo die Gefühle, welche die nötigende \match{Kraft} des moralischen Gesetzes begleiten, jener ihre Wirksamkeit empfindbar machen (z.B. Ekel, Grauen etc., welche den moralischen Widerwillen versinnlichen), um der bloß-sinnlichen Anreizung den Vorrang abzugewinnen. 
	
	\subsection*{tg486.2.7} 
	\textbf{Source : }Die Metaphysik der Sitten/Zweiter Teil. Metaphysische Anfangsgründe der Tugendlehre/II. Ethische Methodenlehre/1. Abschnitt. Die ethische Didaktik\\  
	
	\noindent\textbf{Paragraphe : }Daß sie könne und müsse gelehrt werden, folgt schon daraus, daß sie nicht angeboren ist; die Tugendlehre ist also eine Doktrin. Weil aber durch die bloße Lehre, wie man sich verhalten solle, um dem Tugendbegriffe angemessen zu sein, die \match{Kraft} zur Ausübung der Regeln noch nicht erworben wird, so meinten die Stoiker hiemit nur, die Tugend könne nicht durch bloße Vorstellungen der Pflicht, durch Ermahnungen (paränetisch) gelehrt, sondern sie müsse durch Versuche der Bekämpfung des inneren Feindes im Menschen (asketisch) kultiviert, geübt werden; denn man kann nicht alles so fort was man will, wenn man nicht vorher seine Kräfte versucht, und geübt hat, wozu aber freilich die Entschließung auf einmal vollständig genommen werden muß; weil die Gesinnung (animus) sonst, bei einer Kapitulation mit dem Laster, um es allmählich zu verlassen, an sich unlauter und selbst lasterhaft sein, mithin auch keine Tugend (als die auf einem einzigen Prinzip beruhet) hervorbringen könnte. 
	
	\unnumberedsection{Kreis (2)} 
	\subsection*{tg447.2.13} 
	\textbf{Source : }Die Metaphysik der Sitten/Zweiter Teil. Metaphysische Anfangsgründe der Tugendlehre/Vorrede\\  
	
	\noindent\textbf{Paragraphe : }
	Die Ursache dieser Irrungen ist keine andere als folgende. Der kategorische Imperativ, aus dem diese Gesetze diktatorisch hervorgehen, will denen, die bloß an physiologische Erklärungen gewohnt sind, nicht in den Kopf; unerachtet sie sich doch durch ihn unwiderstehlich gedrungen fühlen. Sich aber das nicht erklären zu können, was über jenen \match{Kreis} gänzlich hinaus liegt (die Freiheit der Willkür), so seelenerhebend auch eben dieser Vorzug des Menschen ist, einer solchen Idee fähig zu sein, wird durch die stolzen Ansprüche der spekulativen Vernunft, die sonst ihr Vermögen in andern Feldern so stark fühlt, gleichsam zum allgemeinen Aufgebot der für die Allgewalt der theoretischen Vernunft Verbündeten gereizt, sich jener Idee zu widersetzen und so den moralischen Freiheitsbegriff jetzt und vielleicht noch lange, obzwar am Ende doch vergeblich, anzufechten und, wo möglich, verdächtig zu machen. 
	
	\subsection*{tg484.2.18} 
	\textbf{Source : }Die Metaphysik der Sitten/Zweiter Teil. Metaphysische Anfangsgründe der Tugendlehre/I. Ethische Elementarlehre/II. Teil. Von den Tugendpflichten gegen andere/Beschluß der Elementarlehre. Von der innigsten Vereinigung der Liebe mit der Achtung in der Freundschaft\\  
	
	\noindent\textbf{Paragraphe : }Es ist Pflicht, so wohl gegen sich selbst, als auch gegen andere, mit seinen sittlichen Vollkommenheiten unter einander Verkehr zu treiben (officium commercii, sociabilitas); sich nicht zu isolieren (separatistam agere); zwar sich einen unbeweglichen Mittelpunkt seiner Grundsätze zu machen, aber diesen um sich gezogenen \match{Kreis} doch auch als einen, der den Teil von einem allbefassenden, der weltbürgerlichen Gesinnung, ausmacht, anzusehen; nicht eben um das Welt beste als Zweck zu befördern, sondern nur die wechselseitige, die indirekt dahin führt, die Annehmlichkeit in derselben, die Verträglichkeit, die wechselseitige Liebe und Achtung (Leutseligkeit und Wohlanständigkeit, humanitas aesthetica, et decorum) zu kultivieren, und so der Tugend die Grazien beizugesellen; welches zu bewerkstelligen selbst Tugendpflicht ist. 
	
	\unnumberedsection{Lange (1)} 
	\subsection*{tg438.2.10} 
	\textbf{Source : }Die Metaphysik der Sitten/Erster Teil. Metaphysische Anfangsgründe der Rechtslehre/1. Teil. Das Privatrecht vom äußeren Mein und Dein überhaupt/2. Hauptstück. Von der Art, etwas Äußeres zu erwerben/Episodischer Abschnitt. Von der idealen Erwerbung eines äußeren Gegenstandes der Willkür\\  
	
	\noindent\textbf{Paragraphe : }Denn setzet: die Versäumung dieses Besitzakts hätte nicht die Folge, daß ein anderer auf seinen gesetzmäßigen und ehrlichen Besitz (possessio bonae fidei) einen zu Recht beständigen (possessio irrefragabilis) gründe, und die Sache, die in seinem Besitz ist, als von ihm erworben ansehe, so würde gar keine Erwerbung peremtorisch (gesichert), sondern alle nur provisorisch (einstweilig) sein; weil die Geschichtskunde ihre Nachforschung bis zum ersten Besitzer und dessen Erwerbakt hinauf zurückzuführen nicht vermögend  ist. – Die Präsumtion, auf welcher sich die Ersitzung (usucapio) gründet, ist also nicht bloß rechtmäßig (erlaubt, iusta) als Vermutung, sondern auch rechtlich (praesumtio iuris et de iure) als Voraussetzung nach Zwangsgesetzen (suppositio legalis): wer seinen Besitzakt zu dokumentieren verabsäumt, hat seinen Anspruch auf den dermaligen Besitzer verloren, wobei die \match{Länge} der Zeit der Verabsäumung (die gar nicht bestimmt werden kann und darf) nur zum Behuf der Gewißheit dieser Unterlassung angeführt wird. Daß aber ein bisher unbekannter Besitzer, wenn jener Besitzakt (es sei auch ohne seine Schuld) unterbrochen worden, die Sache immer wiedererlangen (vindizieren) könne (dominia rerum incerta facere), widerspricht dem obigen Postulat der rechtlich-praktischen Vernunft. 
	
	\unnumberedsection{Leistung (7)} 
	\subsection*{tg430.2.26} 
	\textbf{Source : }Die Metaphysik der Sitten/Erster Teil. Metaphysische Anfangsgründe der Rechtslehre/Einleitung in die Metaphysik der Sitten\\  
	
	\noindent\textbf{Paragraphe : }Hieraus ist zu ersehen, daß alle Pflichten bloß darum, weil sie Pflichten sind, mit zur Ethik gehören; aber ihre Gesetzgebung ist darum nicht allemal in der Ethik enthalten, sondern von vielen derselben außerhalb derselben. So gebietet die Ethik, daß ich eine in einem Vertrage getane Anheischigmachung, wenn mich der andere Teil gleich nicht dazu zwingen könnte, doch erfüllen müsse: allein sie nimmt das Gesetz (pacta sunt servanda), und die diesem korrespondierende Pflicht aus der Rechtslehre als gegeben an. Also nicht in der Ethik, sondern im Ius, liegt die Gesetzgebung, daß angenommene Versprechen gehalten werden müssen. Die Ethik lehrt hernach nur, daß, wenn die Triebfeder, welche die juridische Gesetzgebung mit jener Pflicht verbindet, nämlich der äußere Zwang, auch weggelassen wird, die Idee der Pflicht allein schon zur Triebfeder hinreichend sei. Denn wäre das nicht, und die Gesetzgebung selber nicht juridisch, mithin die aus ihr entspringende Pflicht nicht eigentliche Rechtspflicht (zum Unterschiede von der Tugendpflicht), so würde man die \match{Leistung} der Treue (gemäß seinem Versprechen in einem Vertrage) mit denen Handlungen des Wohlwollens und der Verpflichtung zu ihnen in eine Klasse setzen, welches durchaus nicht geschehen muß. Es ist keine Tugendpflicht, sein Versprechen zu halten, sondern eine Rechtspflicht, zu deren Leistung man gezwungen werden kann. Aber es ist doch eine tugendhafte Handlung (Beweis der Tugend), es auch da zu tun, wo kein Zwang besorgt werden darf. Rechtslehre und Tugendlehre unterscheiden sich also nicht sowohl durch ihre verschiedene Pflichten, als vielmehr durch die Verschiedenheit der Gesetzgebung, welche die eine oder die andere Triebfeder mit dem Gesetze verbindet. 
	
	\subsection*{tg433.2.19} 
	\textbf{Source : }Die Metaphysik der Sitten/Erster Teil. Metaphysische Anfangsgründe der Rechtslehre/1. Teil. Das Privatrecht vom äußeren Mein und Dein überhaupt/1. Hauptstück\\  
	
	\noindent\textbf{Paragraphe : }b) Ich kann die \match{Leistung} von etwas durch die Willkür des andern nicht mein nennen, wenn ich bloß sagen kann, sie sei mit meinem Versprechen zugleich (pactum re initum) in meinen Besitz gekommen, sondern nur, wenn ich behaupten darf, ich bin im Besitz der Willkür des andern (diesen zur Leistung zu bestimmen), obgleich die Zeit der Leistung noch erst kommen soll; das Versprechen des letzteren gehört demnach zur Habe und Gut (obligatio activa) und ich kann sie zu dem Meinen rechnen, aber nicht bloß, wenn ich das Versprochene (wie im ersten Falle) schon in meinem Besitz habe, sondern auch, ob ich dieses gleich noch nicht besitze. Also muß ich mich, als von dem auf Zeitbedingung eingeschränkten, mithin vom empirischen Besitze unabhängig, doch im Besitz dieses Gegenstandes zu sein denken können. 
	
	\subsection*{tg436.2.16} 
	\textbf{Source : }Die Metaphysik der Sitten/Erster Teil. Metaphysische Anfangsgründe der Rechtslehre/1. Teil. Das Privatrecht vom äußeren Mein und Dein überhaupt/2. Hauptstück. Von der Art, etwas Äußeres zu erwerben/2. Abschnitt. Vom persönlichen Recht\\  
	
	\noindent\textbf{Paragraphe : }Was ist aber das Äußere, das ich durch den Vertrag erwerbe? Da es nur die Kausalität der Willkür des anderen in Ansehung einer mir versprochenen \match{Leistung} ist, so erwerbe ich dadurch unmittelbar nicht eine äußere Sache, sondern  eine Tat desselben, dadurch jene Sache in meine Gewalt gebracht wird, damit ich sie zu der meinen mache. – Durch den Vertrag also erwerbe ich das Versprechen eines anderen (nicht das Versprochene) und doch kommt etwas zu meiner äußeren Habe hinzu; ich bin vermögender (locupletior) geworden, durch Erwerbung einer aktiven Obligation auf die Freiheit und das Vermögen des anderen. – Dieses mein Recht aber ist nur ein persönliches, nämlich gegen eine bestimmte physische Person, und zwar auf ihre Kausalität (ihre Willkür) zu wirken, mir etwas zu leisten, nicht ein Sachenrecht, gegen diejenige moralische Person, welche nichts anders als die Idee der a priori vereinigten Willkür aller ist, und wodurch ich allein ein Recht gegen jeden Besitzer derselben erwerben kann; als worin alles Recht in einer Sache besteht. 
	
	\subsection*{tg436.2.20} 
	\textbf{Source : }Die Metaphysik der Sitten/Erster Teil. Metaphysische Anfangsgründe der Rechtslehre/1. Teil. Das Privatrecht vom äußeren Mein und Dein überhaupt/2. Hauptstück. Von der Art, etwas Äußeres zu erwerben/2. Abschnitt. Vom persönlichen Recht\\  
	
	\noindent\textbf{Paragraphe : }Eine Sache wird in einem Vertrage nicht durch Annehmung (acceptatio) des Versprechens, sondern nur durch Übergabe (traditio) des Versprochenen erworben. Denn alles Versprechen geht auf eine Leistung, und wenn das Versprochene eine Sache ist, kann jene nicht anders errichtet werden, als durch einen Akt, wodurch der Promissar vom Promittenten in den Besitz derselben gesetzt wird; d.i. durch die Übergabe. Vor dieser also und dem Empfang ist die \match{Leistung} noch nicht geschehen; die Sache ist von dem einen zu dem anderen noch nicht übergegangen, folglich von diesem nicht erworben worden, mithin das Recht aus einem Vertrage nur ein persönliches, und wird nur durch die Tradition ein dingliches Recht. 
	
	\subsection*{tg436.2.22} 
	\textbf{Source : }Die Metaphysik der Sitten/Erster Teil. Metaphysische Anfangsgründe der Rechtslehre/1. Teil. Das Privatrecht vom äußeren Mein und Dein überhaupt/2. Hauptstück. Von der Art, etwas Äußeres zu erwerben/2. Abschnitt. Vom persönlichen Recht\\  
	
	\noindent\textbf{Paragraphe : }Wenn ich einen Vertrag über eine Sache, z.B. über ein Pferd, das ich erwerben will, schließe, und nehme es zugleich mit in meinen Stall, oder sonst in meinen physischen Besitz, so ist es mein (vi pacti re initi), und mein Recht ist ein Recht in der Sache; lasse ich es aber in den Händen des Verkäufers, ohne mit ihm darüber besonders auszumachen, in wessen physischem Besitz (Inhabung) diese Sache vor meiner Besitznehmung (apprehensio), mithin vor dem Wechsel des Besitzes sein solle: so ist dieses Pferd noch nicht mein, und mein Recht, was ich erwerbe, ist nur ein Recht gegen eine bestimmte Person, nämlich den Verkäufer, von ihm in den Besitz gesetzt zu werden (poscendi traditionem), als subjektive Bedingung der Möglichkeit alles beliebigen Gebrauchs desselben, d.i. mein Recht ist nur ein persönliches Recht, von jenem die \match{Leistung} des Versprechens (praestatio), mich in den Besitz der Sache zu setzen, zu fordern. Nun kann ich, wenn der Vertrag nicht zugleich die Übergabe (als pactum re initum) enthält, mithin eine Zeit zwischen dem Abschluß desselben und der Besitznehmung des Erworbenen verläuft, in dieser Zeit nicht anders zum Besitz gelangen, als dadurch, daß ich einen besonderen rechtlichen, nämlich einen Besitzakt (actum possessorium) ausübe, der einen besonderen Vertrag ausmacht, und dieser ist: daß ich sage, ich werde die Sache (das Pferd) abholen lassen, wozu der Verkäufer einwilligt. Denn daß dieser eine Sache zum Gebrauche eines anderen auf eigene Gefahr in seine Gewahrsame nehmen werde, versteht sich nicht von selbst, sondern dazu gehört ein besonderer Vertrag, nach welchem der Veräußerer seiner Sache innerhalb der bestimmten Zeit noch immer Eigentümer bleibt (und alle Gefahr, die die Sache treffen möchte, tragen muß), der Erwerbende aber nur dann, wann er über diese Zeit zögert, von dem Verkäufer dafür angesehen werden kann, als sei sie ihm überliefert. Vor diesem Besitzakt ist also alles durch den Vertrag Erworbene nur ein persönliches Recht, und der Promissar kann eine äußere Sache nur durch Tradition erwerben. 
	
	\subsection*{tg481.2.4} 
	\textbf{Source : }Die Metaphysik der Sitten/Zweiter Teil. Metaphysische Anfangsgründe der Tugendlehre/I. Ethische Elementarlehre/II. Teil. Von den Tugendpflichten gegen andere/Erstes Hauptstück. Von den Pflichten gegen andere, bloß als Menschen/Erster Abschnitt. Von der Liebespflicht gegen andere Menschen\\  
	
	\noindent\textbf{Paragraphe : }Die oberste Einteilung kann die sein: in Pflichten gegen andere, so fern du sie durch \match{Leistung} derselben zugleich verbindest, und in solche, deren Beobachtung die Verbindlichkeit anderer nicht zur Folge hat. – Die erstere Leistung ist (respektiv gegen andere) verdienstlich; die der zweiten ist schuldige Pflicht. – Liebe und Achtung sind die Gefühle, welche die Ausübung dieser Pflichten begleiten. Sie können abgesondert (jede für sich allein) erwogen werden und auch so bestehen, (Liebe des Nächsten, ob dieser gleich wenig Achtung verdienen möchte; imgleichen notwendige Achtung für jeden Menschen, unerachtet er kaum der Liebe wert zu sein beurteilt würde.) Sie sind aber im Grunde dem Gesetze nach jederzeit mit einander in einer Pflicht zusammen verbunden; nur so, daß bald die eine Pflicht, bald die andere das Prinzip im Subjekt ausmacht, an welche die andere akzessorisch geknüpft ist. – So werden wir gegen einen Armen wohltätig zu sein uns für verpflichtet erkennen; aber, weil diese Gunst doch auch Abhängigkeit seines Wohls von meiner Großmut enthält, die doch den anderen erniedrigt, so ist es Pflicht, dem Empfänger durch ein Betragen, welches diese Wohltätigkeit entweder als bloße Schuldigkeit oder geringen Liebesdienst vorstellt, die Demütigung zu ersparen und ihm seine Achtung für sich selbst zu erhalten. 
	
	\subsection*{tg488.2.5} 
	\textbf{Source : }Die Metaphysik der Sitten/Zweiter Teil. Metaphysische Anfangsgründe der Tugendlehre/Beschluß. Die Religionslehre als Lehre der Pflichten gegen Gott liegt außerhalb den Grenzen der reinen Moralphilosophie\\  
	
	\noindent\textbf{Paragraphe : }Das Formale aller Religion, wenn man sie so erklärt: sie sei »der Inbegriff aller Pflichten als (instar) göttlicher Gebote«, gehört zur philosophischen Moral, indem dadurch nur die Beziehung der Vernunft auf die Idee von Gott, welche sie sich selber macht, ausgedrückt wird, und eine Religionspflicht wird alsdann noch nicht zur Pflicht gegen (erga) Gott, als ein außer unserer Idee existierendes Wesen gemacht, indem wir hiebei von der Existenz desselben noch abstrahieren. – Daß alle Menschenpflichten diesem Formalen (der Beziehung derselben auf einen göttlichen, a priori gegebenen, Willen) gemäß gedacht werden sollen, davon ist der Grund nur subjektiv-logisch. Wir können uns nämlich Verpflichtung (moralische Nötigung) nicht wohl anschaulich machen, ohne einen anderen und dessen Willen (von dem die allgemein gesetzgebende Vernunft nur der Sprecher ist), nämlich Gott, dabei zu denken. – – Allein diese Pflicht in Ansehung Gottes (eigentlich der Idee, welche wir uns von einem solchen Wesen machen) ist Pflicht des Menschen gegen sich selbst, d.i. nicht objektive die Verbindlichkeit zur \match{Leistung} gewisser Dienste an einen anderen, sondern nur subjektive zur Stärkung der moralischen Triebfeder in unserer eigenen gesetzgebenden Vernunft. 
	
	\unnumberedsection{Licht (2)} 
	\subsection*{tg469.2.5} 
	\textbf{Source : }Die Metaphysik der Sitten/Zweiter Teil. Metaphysische Anfangsgründe der Tugendlehre/I. Ethische Elementarlehre/I. Teil. Von den Pflichten gegen sich selbst überhaupt/Einleitung\\  
	
	\noindent\textbf{Paragraphe : }Wenn das verpflichtende Ich mit dem verpflichteten in einerlei Sinn genommen wird, so ist Pflicht gegen sich selbst ein sich widersprechender Begriff. Denn in dem Begriffe der Pflicht ist der einer passiven Nötigung enthalten (ich werde verbunden). Darin aber, daß es eine Pflicht gegen mich selbst ist, stelle ich mich als verbindend, mithin in einer aktiven Nötigung vor (Ich, eben dasselbe Subjekt, bin der Verbindende); und der Satz, der eine Pflicht gegen sich selbst ausspricht (ich soll mich selbst verbinden), würde eine Verbindlichkeit verbunden zu sein (passive Obligation, die doch zugleich, in demselben Sinne des Verhältnisses, eine aktive wäre), mithin einen Widerspruch enthalten. – Man kann diesen Widerspruch auch dadurch ins \match{Licht} stellen: daß man zeigt, der Verbindende (auctor obligationis) könne den Verbundenen (subiectum obligationis) jederzeit von der Verbindlichkeit (terminus obligationis) lossprechen; mithin (wenn beide ein und dasselbe Subjekt sind), er sei an eine Pflicht, die er sich auferlegt, gar nicht gebunden: welches einen Widerspruch enthält. 
	
	\subsection*{tg481.2.86} 
	\textbf{Source : }Die Metaphysik der Sitten/Zweiter Teil. Metaphysische Anfangsgründe der Tugendlehre/I. Ethische Elementarlehre/II. Teil. Von den Tugendpflichten gegen andere/Erstes Hauptstück. Von den Pflichten gegen andere, bloß als Menschen/Erster Abschnitt. Von der Liebespflicht gegen andere Menschen\\  
	
	\noindent\textbf{Paragraphe : }c) Die Schadenfreude, welche das gerade Umgekehrte der Teilnehmung ist, ist der menschlichen Natur auch nicht fremd; wiewohl, wenn sie so weit geht, das Übel oder Böses selbst bewirken zu helfen, sie als qualifizierte Schadenfreude den Menschenhaß sichtbar macht und in ihrer Gräßlichkeit erscheint. Sein Wohlsein und selbst sein Wohlverhalten stärker zu fühlen, wenn Unglück, oder Verfall  anderer in Skandale, gleichsam als die Folie unserem eigenen Wohlstande untergelegt wird, um diesen in ein desto helleres \match{Licht} zu stellen, ist freilich nach Gesetzen der Einbildungskraft, nämlich des Kontrastes, in der Natur gegründet. Aber über die Existenz solcher das allgemeine Weltbeste zerstörenden Enormitäten unmittelbar sich zu freuen, mithin dergleichen Eräugnisse auch wohl zu wünschen, ist ein geheimer Menschenhaß und das gerade Widerspiel der Nächstenliebe, die uns als Pflicht obliegt. – Der Übermut anderer bei ununterbrochenem Wohlergehn und der Eigendünkel im Wohlverhalten (eigentlich aber nur im Glück, der Verleitung zum öffentlichen Laster noch immer entwischt zu sein), welches beides der eigenliebige Mensch sich zum Verdienst anrechnet, bringen diese feindselige Freude hervor, die der Pflicht nach dem Prinzip der Teilnehmung (des ehrlichen Chremes beim Terenz) »ich bin ein Mensch; alles, was Men schen widerfährt, das trifft auch mich« gerade entgegengesetzt ist. 
	
	\unnumberedsection{Maß (6)} 
	\subsection*{tg458.2.4} 
	\textbf{Source : }Die Metaphysik der Sitten/Zweiter Teil. Metaphysische Anfangsgründe der Tugendlehre/Einleitung/X. Das oberste Prinzip der Rechtslehre war analytisch; das der Tugendlehre ist synthetisch\\  
	
	\noindent\textbf{Paragraphe : }Diese Erweiterung des Pflichtbegriffs über den der äußeren Freiheit und der Einschränkung derselben durch das bloße Förmliche ihrer durchgängigen Zusammenstimmung, wo die innere Freiheit, statt des Zwanges von außen, das Vermögen des Selbstzwanges und zwar nicht vermittelst anderer Neigungen, sondern durch reine praktische Vernunft (welche alle diese Vermittelung verschmäht), aufgestellt wird, besteht darin und erhebt sich dadurch über die Rechtspflicht: daß durch sie Zwecke aufgestellet werden, von denen überhaupt das Recht abstrahiert. – Im moralischen Imperativ, und der notwendigen Voraussetzung der Freiheit zum Behuf desselben, machen: das Gesetz, das Vermögen (es zu erfüllen) und der die Maxime bestimmende Wille alle Elemente aus, welche den Begriff der Rechtspflicht bilden. Aber in demjenigen, welcher die Tugendpflicht
	gebietet, kommt, noch über den Begriff eines Selbstzwanges, der eines Zwecks dazu, nicht den wir haben, sondern haben sollen, den also die reine praktische Vernunft in sich hat, deren höchster, unbedingter Zweck (der aber doch immer noch Pflicht ist) darin gesetzt wird: daß die Tugend ihr eigener Zweck und, bei dem Verdienst, das sie um den Menschen hat, auch ihr eigener Lohn sei. (Wobei sie, als Ideal, so glänzt, daß sie nach menschlichem Augenmaß die Heiligkeit selbst, die zur Übertretung nie versucht wird, zu verdunkeln scheint;
	
	
	16
	welches gleichwohl eine Täuschung ist, da, weil wir kein \match{Maß} für den Grad einer Stärke, als die Größe der Hindernisse haben, die da haben überwunden werden können (welche in uns die Neigungen sind ), wir die subjektive Bedingungen der Schätzung einer Größe für die objektive der Größe an sich selbst zu halten verleitet werden.) Aber mit menschlichen Zwecken, die insgesamt ihre zu bekämpfenden Hindernisse haben, verglichen, hat es seine Richtigkeit, daß der Wert der Tugend selbst, als ihres eigenen Zwecks, den Wert alles Nutzens und aller empirischen Zwecke und Vorteile weit überwiege, die sie zu ihrer Folge immerhin haben mag. 
	
	\subsection*{tg471.2.50} 
	\textbf{Source : }Die Metaphysik der Sitten/Zweiter Teil. Metaphysische Anfangsgründe der Tugendlehre/I. Ethische Elementarlehre/I. Teil. Von den Pflichten gegen sich selbst überhaupt/Erstes Buch. Von den vollkommenen Pflichten gegen sich selbst/Erstes Hauptstück. Die Pflicht des Menschen gegen sich selbst, als einem animalischen Wesen\\  
	
	\noindent\textbf{Paragraphe : }Kann man dem Wein, wenn gleich nicht als Panegyrist, doch wenigstens als Apologet, einen Gebrauch verstatten, der bis nahe an die Berauschung reicht; weil er doch die Gesellschaft zur Gesprächigkeit belebt, und damit Offenherzigkeit verbindet? – Oder kann man ihm wohl gar das Verdienst zugestehen, das zu befördern, was Seneca vom Cato rühmt: virtus eius incaluit mero? – Der Gebrauch des Opium und Branntweins sind, als Genießmittel, der Niederträchtigkeit näher, weil sie, bei dem geträumten Wohlbefinden, stumm, zurückhaltend und unmitteilbar machen, daher auch nur als Arzneimittel erlaubt sind. – Wer kann aber das \match{Maß} für einen bestimmen, der in den Zustand, wo er zum Messen keine klare Augen mehr hat, überzugehen eben in Bereitschaft ist? Der Mohammedanism, welcher den Wein ganz verbietet, hat also sehr schlecht gewählt, dafür das Opium zu erlauben. 
	
	\subsection*{tg472.2.22} 
	\textbf{Source : }Die Metaphysik der Sitten/Zweiter Teil. Metaphysische Anfangsgründe der Tugendlehre/I. Ethische Elementarlehre/I. Teil. Von den Pflichten gegen sich selbst überhaupt/Erstes Buch. Von den vollkommenen Pflichten gegen sich selbst\\  
	
	\noindent\textbf{Paragraphe : }Ich verstehe hier unter diesem Namen nicht den habsüchtigen Geiz (der Erweiterung seines Erwerbs der Mittel zum Wohlleben, über die Schranken des wahren Bedürfnisses); denn dieser kann auch als bloße Verletzung seiner Pflicht (der Wohltätigkeit) gegen andere betrachtet werden;  auch nicht den kargen Geiz, welcher, wenn er schimpflich ist, Knickerei oder Knauserei genannt wird, aber doch bloß Vernachlässigung seiner Liebespflichten gegen andere sein kann; sondern die Verengung seines eigenen Genusses der Mittel zum Wohlleben unter das \match{Maß} des wahren eigenen Bedürfnisses; dieser Geiz ist es eigentlich, der hier gemeint ist, welcher der Pflicht gegen sich selbst widerstreitet. 
	
	\subsection*{tg472.2.25} 
	\textbf{Source : }Die Metaphysik der Sitten/Zweiter Teil. Metaphysische Anfangsgründe der Tugendlehre/I. Ethische Elementarlehre/I. Teil. Von den Pflichten gegen sich selbst überhaupt/Erstes Buch. Von den vollkommenen Pflichten gegen sich selbst\\  
	
	\noindent\textbf{Paragraphe : }Nicht das \match{Maß} der Ausübung sittlicher Maximen, sondern das objektive Prinzip derselben, muß als verschieden erkannt und vorgetragen werden, wenn ein Laster von der Tugend unterschieden werden soll. – Die Maxime des habsüchtigen Geizes (als Verschwenders) ist: alle Mittel des Wohllebens in der Absicht auf den Genuß anzuschaffen und zu erhalten. – Die des kargen Geizes ist hingegen der Erwerb so wohl, als die Erhaltung aller Mittel des Wohllebens, aber ohne Absicht auf den Genuß (d.i. ohne daß dieser, sondern nur der Besitz der Zweck sei). 
	
	\subsection*{tg477.2.4} 
	\textbf{Source : }Die Metaphysik der Sitten/Zweiter Teil. Metaphysische Anfangsgründe der Tugendlehre/I. Ethische Elementarlehre/I. Teil. Von den Pflichten gegen sich selbst überhaupt/2. Buch: Die Pflichten gegen sich selbst/Erster Abschnitt. Von der Pflicht gegen sich selbst in Entwickelung und Vermehrung seiner Naturvollkommenheit, d.i. in pragmatischer Absicht\\  
	
	\noindent\textbf{Paragraphe : }Der Anbau (cultura) seiner Naturkräfte (Geistes-, Seelen- und Leibeskräfte), als Mittel zu allerlei möglichen Zwecken ist Pflicht des Menschen gegen sich selbst. – Der Mensch ist es sich selbst (als einem Vernunftwesen) schuldig, die Naturanlage und Vermögen, von denen seine Vernunft dereinst Gebrauch machen kann, nicht unbenutzt und gleichsam rosten zu lassen, sondern, gesetzt daß er auch mit dem angebornen \match{Maß} seines Vermögens für die natürlichen Bedürfnisse zufrieden sein könne, so muß ihm doch seine Vernunft dieses Zufriedensein, mit dem geringen Maß seiner Vermögen, erst durch Grundsätze anweisen, weil er, als ein Wesen, das der Zwecke (sich Gegenstände zum Zweck zu machen) fähig ist, den Gebrauch seiner Kräfte nicht bloß dem Instinkt der Natur, sondern der Freiheit, mit der er dieses Maß bestimmt, zu verdanken haben muß. Es ist also nicht Rücksicht auf den Vorteil, den die Kultur seines Vermögens (zu allerlei Zwecken) verschaffen kann; denn dieser würde vielleicht (nach Rousseauschen Grundsätzen) für die Rohigkeit des Naturbedürfnisses vorteilhaft ausfallen: sondern es ist Gebot der moralisch-praktischen Vernunft und Pflicht des Menschen gegen sich selbst, seine  Vermögen (unter denselben eins mehr als das andere, nach Verschiedenheit seiner Zwecke) anzubauen, und in pragmatischer Rücksicht ein dem Zweck seines Daseins angemessener Mensch zu sein. 
	
	\subsection*{tg487.2.5} 
	\textbf{Source : }Die Metaphysik der Sitten/Zweiter Teil. Metaphysische Anfangsgründe der Tugendlehre/II. Ethische Methodenlehre/2. Abschnitt. Die ethische Asketik\\  
	
	\noindent\textbf{Paragraphe : }Die Kultur der Tugend, d.i. die moralische Asketik, hat, in Ansehung des Prinzips der rüstigen, mutigen und wackeren Tugendübung den Wahlspruch der Stoiker: gewöhne dich, die zufälligen Lebensübel zu ertragen und die eben so überflüssigen Ergötzlichkeiten zu entbehren (assuesce incommodis et desuesce commoditatibus vitae). Es ist eine Art von Diätetik für den Menschen, sich moralisch gesund zu erhalten. Gesundheit ist aber nur ein negatives  Wohlbefinden, sie selber kann nicht gefühlt werden. Es muß etwas dazu kommen, was einen angenehmen Lebensgenuß gewährt und doch bloß moralisch ist. Das ist das jederzeit fröhliche Herz in der Idee des tugendhaften Epikurs. Denn wer sollte wohl mehr Ursache haben, frohen Muts zu sein und nicht darin selbst eine Pflicht finden, sich in eine fröhliche Gemütsstimmung zu versetzen und sie sich habituell zu machen, als der, welcher sich keiner vorsätzlichen Übertretung bewußt und, wegen des Verfalls in eine solche, gesichert ist (hic murus ahenëus esto etc. Horat.). – Die Mönchsasketik hingegen, welche aus abergläubischer Furcht, oder geheucheltem Abscheu an sich selbst, mit Selbstpeinigung und Fleischeskreuzigung zu Werke geht, zweckt auch nicht auf Tugend, sondern auf schwärmerische Entsündigung ab, sich selbst Strafe aufzulegen und, anstatt sie moralisch (d.i. in Absicht auf die Besserung) zu bereuen, sie büßen zu wollen; welches, bei einer selbstgewählten und an sich vollstreckten Strafe (denn die muß immer ein anderer auflegen), ein Widerspruch ist, und kann auch den Frohsinn, der die Tugend begleitet, nicht bewirken, vielmehr nicht ohne geheimen Haß gegen das Tugendgebot statt finden. – Die ethische Gymnastik besteht also nur in der Bekämpfung der Naturtriebe, die das \match{Maß} erreicht, über sie bei vorkommenden, der Moralität Gefahr drohenden, Fällen Meister werden zu können; mithin die wacker und, im Bewußtsein seiner wiedererworbenen Freiheit, fröhlich macht. Etwas bereuen (welches bei der Rückerinnerung ehemaliger Übertretungen unvermeidlich, ja wobei diese Erinnerung nicht schwinden zu lassen es so gar Pflicht ist) und sich eine Pönitenz auferlegen (z.B. das Fasten), nicht in diätetischer, sondern frommer Rücksicht, sind zwei sehr verschiedene, moralisch gemeinte, Vorkehrungen, von denen die letztere, welche freudenlos, finster und mürrisch ist, die Tugend selbst verhaßt macht und ihre Anhänger verjagt. Die Zucht (Disziplin), die der Mensch an sich selbst verübt, kann daher nur durch den Frohsinn, der sie begleitet, verdienstlich und exemplarisch werden. 
	
	\unnumberedsection{Maße (2)} 
	\subsection*{tg441.2.64} 
	\textbf{Source : }Die Metaphysik der Sitten/Erster Teil. Metaphysische Anfangsgründe der Rechtslehre/2. Teil. Das öffentliche Recht/1. Abschnitt. Das Staatsrecht\\  
	
	\noindent\textbf{Paragraphe : }Die Würde betreffend, nicht bloß die, welche ein Amt bei sich führen mag, sondern auch die, welche den Besitzer auch ohne besondere Bedienungen zum Gliede eines höheren Standes macht, ist der Adel, der, vom bürgerlichen Stande, in welchem das Volk ist, unterschieden, den männlichen  Nachkommen anerbt, durch diese auch wohl den weiblichen unadliger Geburt, nur so, daß die Adlig-Geborne ihrem unadligen Ehemann nicht um gekehrt diesen Rang mitteilt, sondern selbst in den bloß bürgerlichen (des Volks) zurückfällt. – Die Frage ist nun: ob der Souverän einen Adelstand, als einen erblichen Mittelstand zwischen ihm und den übrigen Staatsbürgern, zu gründen berechtigt sei. In dieser Frage kommt es nicht darauf an: ob es der Klugheit des Souveräns, wegen seines oder des Volks Vorteils, sondern nur, ob es dem Rechte des Volks gemäß sei, einen Stand von Personen über sich zu haben, die zwar selbst Untertanen, aber doch in Ansehung des Volks geborne Befehlshaber (wenigstens privilegierte) sind. – – Die Beantwortung derselben geht nun hier, eben so wie vorher, aus dem Prinzip hervor: »was das Volk (die ganze \match{Masse} der Untertanen) nicht über sich selbst und seine Genossen beschließen kann, das kann auch der Souverän nicht über das Volk beschließen«. Nun ist ein angeerbter Adel ein Rang, der vor dem Verdienst vorher geht, und dieses auch mit keinem Grunde hoffen läßt, ein Gedankending, ohne alle Realität. Denn, wenn der Vorfahr Verdienst hatte, so konnte er dieses doch nicht auf seine Nachkommen vererben, sondern diese mußten es sich immer selbst erwerben; da die Natur es nicht so fügt, daß das Talent und der Wille, welche Verdienste um den Staat möglich machen, auch anarten. Weil nun von keinem Menschen angenommen werden kann, er werde seine Freiheit wegwerfen, so ist es unmöglich, daß der allgemeine Volkswille zu einem solchen grundlosen Prärogativ zusammen stimme, mithin kann der Souverän es auch nicht geltend machen. – – Wenn indessen gleich eine solche Anomalie in das Maschinenwesen einer Regierung von alten Zeiten (des Lehnswesens, das fast gänzlich auf den Krieg angelegt war) eingeschlichen, von Untertanen, die mehr als Staatsbürger, nämlich geborne Beamte (wie etwa ein Erbprofessor) sein wollen, so kann der Staat diesen von ihm begangenen Fehler eines widerrechtlich erteilten erblichen Vorzugs nicht anders, als durch Eingehen und Nichtbesetzung der Stellen allmählich wiederum gut machen, und  so hat er provisorisch ein Recht, diese Würde dem Titel nach fortdauern zu lassen, bis selbst in der öffentlichen Meinung die Einteilung in Souverän, Adel und Volk der einzigen natürlichen in Souverän und Volk Platz gemacht haben wird. 
	
	\subsection*{tg460.2.9} 
	\textbf{Source : }Die Metaphysik der Sitten/Zweiter Teil. Metaphysische Anfangsgründe der Tugendlehre/Einleitung/XII. Ästhetische Vorbegriffe der Empfänglichkeit des Gemüts für Pflichtbegriffe überhaupt\\  
	
	\noindent\textbf{Paragraphe : }Dieses Gefühl einen moralischen Sinn zu nennen ist nicht schicklich; denn unter dem Wort Sinn wird gemeiniglich ein theoretisches, auf einen Gegenstand bezogenes Wahrnehmungsvermögen verstanden: dahin gegen das moralische Gefühl (wie Lust und Unlust überhaupt) etwas bloß Subjektives ist, was kein Erkenntnis abgibt. – Ohne alles moralische Gefühl ist kein Mensch; denn, bei völliger Unempfänglichkeit für diese Empfindung, wäre er sittlich tot und, wenn (um in der Sprache der Ärzte zu reden) die sittliche Lebenskraft keinen Reiz mehr auf dieses Gefühl bewirken könnte, so würde sich die Menschheit (gleichsam nach chemischen Gesetzen) in die bloße Tierheit auflösen und mit der \match{Masse} anderer Naturwesen unwiederbringlich vermischt werden. – Wir haben aber für das (sittlich-) Gute und Böse eben so wenig einen besonderen Sinn, als wir einen solchen für die Wahrheit haben, ob man sich gleich oft so ausdrückt, sondern Empfänglichkeit der freien Willkür für die Bewegung derselben durch praktische reine Vernunft (und ihr Gesetz), und das ist es, was wir das moralische Gefühl nennen. 
	
	\unnumberedsection{Maßstab (4)} 
	\subsection*{tg437.2.82} 
	\textbf{Source : }Die Metaphysik der Sitten/Erster Teil. Metaphysische Anfangsgründe der Rechtslehre/1. Teil. Das Privatrecht vom äußeren Mein und Dein überhaupt/2. Hauptstück. Von der Art, etwas Äußeres zu erwerben/3. Abschnitt. Von dem auf dingliche Art persönlichen Recht\\  
	
	\noindent\textbf{Paragraphe : }»Geld ist also (nach Adam Smith) derjenige Körper, dessen Veräußerung das Mittel und zugleich der \match{Maßstab} des Fleißes ist, mit welchem Menschen und Völker unter einander Verkehr treiben.« – Diese Erklärung führt den empirischen Begriff des Geldes dadurch auf den intellektuellen hinaus, daß sie nur auf die Form der wechselseitigen Leistungen  im belästigten Vertrage sieht (und von dieser ihrer Materie abstrahiert), und so auf Rechtsbegriff in der Umsetzung des Mein und Dein (commutatio late sic dicta) überhaupt, um die obige Tafel einer dogmatischen Einteilung a priori, mithin der Metaphysik des Rechts, als eines Systems, angemessen vorzustellen. 
	
	\subsection*{tg481.2.29} 
	\textbf{Source : }Die Metaphysik der Sitten/Zweiter Teil. Metaphysische Anfangsgründe der Tugendlehre/I. Ethische Elementarlehre/II. Teil. Von den Tugendpflichten gegen andere/Erstes Hauptstück. Von den Pflichten gegen andere, bloß als Menschen/Erster Abschnitt. Von der Liebespflicht gegen andere Menschen\\  
	
	\noindent\textbf{Paragraphe : }Aber einer ist mir doch näher als der andere, und ich bin im Wohlwollen mir selbst der nächste. Wie stimmt das nun mit der Formel: Liebe deinen Nächsten (deinen Mitmenschen) als dich selbst? Wenn einer mir näher ist (in der Pflicht des Wohlwollens) als der andere, ich also zum größeren Wohlwollen gegen einen als gegen den anderen verbunden, mir selber aber geständlich näher (selbst der Pflicht nach) bin, als jeder andere, so kann ich, wie es scheint, ohne mir selbst zu widersprechen, nicht sagen, ich soll jeden Menschen lieben wie mich selbst; denn der \match{Maßstab} der Selbstliebe würde keinen Unterschied in Graden zulassen. – Man siehet bald: daß hier nicht bloß das Wohlwollen des Wunsches, welches eigentlich ein bloßes Wohlgefallen am Wohl jedes anderen ist, ohne selbst dazu etwas beitragen zu dürfen (ein jeder für sich; Gott für uns alle), sondern ein tätiges, praktisches Wohlwollen, sich das Wohl und Heil des anderen zum Zweck zu machen (das Wohltun), gemeinet sei. Denn im Wünschen kann ich allen gleich wohlwollen, aber im Tun kann der Grad, nach Verschiedenheit der Geliebten (deren einer mich näher angeht als der andere), ohne die Allgemeinheit der Maxime zu verletzen, doch sehr verschieden sein. 
	
	\subsection*{tg481.2.84} 
	\textbf{Source : }Die Metaphysik der Sitten/Zweiter Teil. Metaphysische Anfangsgründe der Tugendlehre/I. Ethische Elementarlehre/II. Teil. Von den Tugendpflichten gegen andere/Erstes Hauptstück. Von den Pflichten gegen andere, bloß als Menschen/Erster Abschnitt. Von der Liebespflicht gegen andere Menschen\\  
	
	\noindent\textbf{Paragraphe : }a) Der Neid (livor), als Hang, das Wohl anderer mit Schmerz, wahrzunehmen, ob zwar dem seinigen dadurch kein Abbruch geschieht, der, wenn er zur Tat (jenes Wohl zu schmälern) ausschlägt, qualifizierter Neid, sonst aber nur Mißgunst (invidentia) heißt, ist doch nur eine indirekt-bösartige Gesinnung, nämlich ein Unwille, unser eigen Wohl durch das Wohl anderer in Schatten gestellt zu sehen, weil wir den \match{Maßstab} desselben nicht in dessen innerem Wert, sondern nur in der Vergleichung mit dem Wohl anderer, zu schätzen, und diese Schätzung zu versinnlichen wissen. – Daher spricht man auch wohl von einer beneidungswürdigen Eintracht und Glückseligkeit in einer Ehe, oder Familie u.s.w.; gleich als ob es in manchen Fällen erlaubt wäre, jemanden zu beneiden. Die Regungen des Neides liegen also in der Natur des Menschen, und nur der Ausbruch derselben macht sie zu dem scheußlichen Laster einer grämischen, sich selbst folternden und auf Zerstörung des Glücks anderer, wenigstens dem Wunsche nach, gerichteten Leidenschaft, ist mithin der Pflicht des Menschen gegen sich selbst so wohl, als gegen andere entgegengesetzt. 
	
	\subsection*{tg481.2.91} 
	\textbf{Source : }Die Metaphysik der Sitten/Zweiter Teil. Metaphysische Anfangsgründe der Tugendlehre/I. Ethische Elementarlehre/II. Teil. Von den Tugendpflichten gegen andere/Erstes Hauptstück. Von den Pflichten gegen andere, bloß als Menschen/Erster Abschnitt. Von der Liebespflicht gegen andere Menschen\\  
	
	\noindent\textbf{Paragraphe : }Alle Laster, welche selbst die menschliche Natur hassenswert machen würden, wenn man sie (als qualifiziert) in der Bedeutung von Grundsätzen nehmen wollte, sind inhuman, objektiv betrachtet, aber doch menschlich, subjektiv erwogen: d.i. wie die Erfahrung uns unsere Gattung kennen lehrt. Ob man also zwar einige derselben in der Heftigkeit des Abscheues teuflisch nennen möchte, so wie ihr Gegenstück Engelstugend genannt werden könnte: so sind beide Begriffe doch nur Ideen von einem Maximum, als \match{Maßstab} zum Behuf der Vergleichung des Grades der Moralität gedacht, indem man dem Menschen seinen Platz im Himmel oder der Hölle anweiset, ohne aus ihm ein Mittelwesen, was weder den einen dieser Plätze, noch den anderen einnimmt, zu machen. Ob es Haller, mit seinem »zweideutig Mittelding von Engeln und von Vieh«, besser getroffen habe, mag hier unausgemacht bleiben. Aber das Halbieren. in einer Zusammenstellung heterogener Dinge führt auf gar keinen bestimmten Begriff, und zu diesem kann uns in der Ordnung der Wesen nach ihrem uns unbekannten Klassenunterschiede nichts hinleiten. Die erstere Gegeneinanderstellung (von Engelstugend und teuflischem  Laster) ist Übertreibung. Die zweite, ob zwar Menschen leider! auch in viehische Laster fallen, berechtigt doch nicht, eine zu ihrer Spezies gehörige Anlage dazu ihnen beizulegen, so wenig, als die Verkrüppelung einiger Bäume im Walde ein Grund ist, sie zu einer besondern Art von Gewächsen zu machen. 
	
	\unnumberedsection{Materie (15)} 
	\subsection*{tg430.2.36} 
	\textbf{Source : }Die Metaphysik der Sitten/Erster Teil. Metaphysische Anfangsgründe der Rechtslehre/Einleitung in die Metaphysik der Sitten\\  
	
	\noindent\textbf{Paragraphe : }
	Pflicht ist diejenige Handlung, zu welcher jemand verbunden ist. Sie ist also die \match{Materie} der Verbindlichkeit, und es kann einerlei Pflicht (der Handlung nach) sein, ob wir zwar auf verschiedene Art dazu verbunden werden können. 
	
	\subsection*{tg430.2.9} 
	\textbf{Source : }Die Metaphysik der Sitten/Erster Teil. Metaphysische Anfangsgründe der Rechtslehre/Einleitung in die Metaphysik der Sitten\\  
	
	\noindent\textbf{Paragraphe : }Unter dem Willen kann die Willkür, aber auch der bloße Wunsch enthalten sein, sofern die Vernunft das Begehrungsvermögen überhaupt bestimmen kann; die Willkür, die durch reine Vernunft bestimmt werden kann, heißt  die freie Willkür. Die, welche nur durch Neigung (sinnlichen Antrieb, stimulus) bestimmbar ist, würde tierische Willkür (arbitrium brutum) sein. Die menschliche Willkür ist dagegen eine solche, welche durch Antriebe zwar affiziert, aber nicht bestimmt wird, und ist also für sich (ohne erworbene Fertigkeit der Vernunft) nicht rein, kann aber doch zu Handlungen aus reinem Willen bestimmt werden. Die Freiheit der Willkür ist jene Unabhängigkeit ihrer Bestimmung durch sinnliche Antriebe; dies ist der negative Begriff derselben. Der positive ist: das Vermögen der reinen Vernunft, für sich selbst praktisch zu sein. Dieses ist aber nicht anders möglich, als durch die Unterwerfung der Maxime einer jeden Handlung unter die Bedingung der Tauglichkeit der erstern zum allgemeinen Gesetze. Denn, als reine Vernunft, auf die Willkür, unangesehen dieser ihres Objekts, angewandt, kann sie, als Vermögen der Prinzipien (und hier praktischer Prinzipien, mithin als gesetzgebendes Vermögen), da ihr die \match{Materie} des Gesetzes abgeht, nichts mehr, als die Form der Tauglichkeit der Maxime der Willkür zum allgemeinen Gesetze selbst zum obersten Gesetze und Bestimmungsgrunde der Willkür machen, und, da die Maximen des Menschen aus subjektiven Ursachen mit jenen objektiven nicht von selbst übereinstimmen, dieses Gesetz nur schlechthin, als Imperativ des Verbots oder Gebots, vorschreiben. 
	
	\subsection*{tg433.2.10} 
	\textbf{Source : }Die Metaphysik der Sitten/Erster Teil. Metaphysische Anfangsgründe der Rechtslehre/1. Teil. Das Privatrecht vom äußeren Mein und Dein überhaupt/1. Hauptstück\\  
	
	\noindent\textbf{Paragraphe : }Denn ein Gegenstand meiner Willkür ist etwas, was zu gebrauchen ich physisch in meiner Macht habe. Sollte es nun doch rechtlich schlechterdings nicht in meiner Macht stehen, d.i. mit der Freiheit von jedermann nach einem allgemeinen Gesetz nicht zusammen bestehen können (unrecht sein), Gebrauch von demselben zu machen: so würde die Freiheit sich selbst des Gebrauchs ihrer Willkür in Ansehung eines Gegenstandes derselben berauben, dadurch, daß sie brauchbare Gegenstände außer aller Möglichkeit des Gebrauchs setzte: d.i. diese in praktischer Rücksicht vernichtete, und zur res nullius machte; obgleich die Willkür, formaliter, im Gebrauch der Sachen mit jedermanns äußeren Freiheit nach allgemeinen Gesetzen zusammenstimmete. – Da nun die reine praktische Vernunft keine andere als formale Gesetze des Gebrauchs der Willkür zum Gründe legt, und also von der \match{Materie} der Willkür, d.i. der übrigen Beschaffenheit des Objekts, wenn es nur ein Gegenstand der Willkür ist, abstrahiert, so kann sie in Ansehung eines solchen Gegenstandes kein absolutes Verbot seines Gebrauchs enthalten, weil dieses ein Widerspruch der äußeren Freiheit mit sich selbst sein würde. – Ein Gegenstand meiner Willkür aber ist das, wovon beliebigen Gebrauch zu machen ich das physische Vermögen habe, dessen Gebrauch in meiner Macht (potentia) steht: wovon noch unterschieden werden muß, denselben Gegenstand in meiner Gewalt (in potestatem meam redactum) zu haben, welches nicht bloß ein Vermögen, sondern auch einen Akt der Willkür voraus setzt. Um aber etwas bloß als Gegenstand meiner Willkür zu denken, ist hinreichend, mir bewußt zu sein, daß ich ihn in meiner Macht habe. – Also ist es eine Voraussetzung a priori der praktischen Vernunft, einen jeden  Gegenstand meiner Willkür als objektiv-mögliches Mein oder Dein anzusehen und zu behandeln. 
	
	\subsection*{tg434.2.12} 
	\textbf{Source : }Die Metaphysik der Sitten/Erster Teil. Metaphysische Anfangsgründe der Rechtslehre/1. Teil. Das Privatrecht vom äußeren Mein und Dein überhaupt\\  
	
	\noindent\textbf{Paragraphe : }1) Der \match{Materie} (dem Objekte) nach erwerbe ich entweder eine körperliche Sache (Substanz) oder die Leistung (Kausalität) eines anderen oder diese andere Person selbst, d.i. den Zustand derselben, so fern ich ein Recht erlange, über denselben zu verfügen (das Commercium mit derselben). 
	
	\subsection*{tg435.2.9} 
	\textbf{Source : }Die Metaphysik der Sitten/Erster Teil. Metaphysische Anfangsgründe der Rechtslehre/1. Teil. Das Privatrecht vom äußeren Mein und Dein überhaupt/2. Hauptstück. Von der Art, etwas Äußeres zu erwerben/1. Abschnitt. Vom Sachrecht\\  
	
	\noindent\textbf{Paragraphe : }Denn setzet, der Boden gehöre niemanden an: so werde ich jede bewegliche Sache, die sich auf ihm befindet, aus ihrem Platze stoßen können, um ihn selbst einzunehmen, bis sie sich gänzlich verliert, ohne daß der Freiheit irgend eines anderen, der jetzt gerade nicht Inhaber desselben ist, dadurch Abbruch geschieht; alles aber, was zerstört werden kann, ein Baum, Haus u.s.w. ist (wenigstens der \match{Materie} nach) beweglich, und wenn man die Sache, die ohne Zerstörung ihrer Form nicht bewegt werden kann, ein Immobile nennt, so wird das Mein und Dein an jener nicht von der Substanz, sondern dem ihr Anhängenden verstanden, welches nicht die Sache selbst ist. 
	
	\subsection*{tg437.2.40} 
	\textbf{Source : }Die Metaphysik der Sitten/Erster Teil. Metaphysische Anfangsgründe der Rechtslehre/1. Teil. Das Privatrecht vom äußeren Mein und Dein überhaupt/2. Hauptstück. Von der Art, etwas Äußeres zu erwerben/3. Abschnitt. Von dem auf dingliche Art persönlichen Recht\\  
	
	\noindent\textbf{Paragraphe : }Das Gesinde gehört nun zu dem Seinen des Hausherrn, und zwar, was die Form (den Besitzstand) betrifft, gleich als nach einem Sachenrecht; denn der Hausherr kann, wenn es ihm entläuft, es durch einseitige Willkür in seine Gewalt bringen; was aber die \match{Materie} betrifft, d.i. welchen Gebrauch er von diesen seinen Hausgenossen machen kann, so kann er sich nie als Eigentümer desselben (dominus servi) betragen: weil er nur durch Vertrag unter seine Gewalt gebracht ist, ein Vertrag aber, durch den ein Teil zum Vorteil des anderen auf seine ganze Freiheit Verzicht tut, mithin aufhört, eine Person zu sein, folglich auch keine Pflicht hat, einen Vertrag zu halten, sondern nur Gewalt  anerkennt, in sich selbst widersprechend, d.i. null und nichtig ist. (Von dem Eigentumsrecht gegen den, der sich durch ein Verbrechen seiner Persönlichkeit verlustig gemacht hat, ist hier nicht die Rede.) 
	
	\subsection*{tg437.2.73} 
	\textbf{Source : }Die Metaphysik der Sitten/Erster Teil. Metaphysische Anfangsgründe der Rechtslehre/1. Teil. Das Privatrecht vom äußeren Mein und Dein überhaupt/2. Hauptstück. Von der Art, etwas Äußeres zu erwerben/3. Abschnitt. Von dem auf dingliche Art persönlichen Recht\\  
	
	\noindent\textbf{Paragraphe : }In dieser Tafel aller Arten der Übertragung (translatio) des Seinen auf einen anderen finden sich Begriffe von Objekten, oder Werkzeugen dieser Übertragung vor, welche ganz empirisch zu sein, und selbst ihrer Möglichkeit nach, in einer  metaphysischen Rechtslehre, eigentlich nicht Platz haben, in der die Einteilungen nach Prinzipien a priori gemacht werden müssen, mithin von der \match{Materie} des Verkehrs (welche konventionell sein könnte) abstrahiert, und bloß auf die Form gesehen werden muß, dergleichen der Begriff des Geldes, im Gegensatz mit aller anderen veräußerlichen Sache, nämlich der Ware, im Titel des Kaufs und Verkaufs, oder der eines Buchs ist. – Allein es wird sich zeigen, daß jener Begriff des größten und brauchbarsten aller Mittel des Verkehrs der Menschen mit Sachen, Kauf und Verkauf (Handel) genannt, imgleichen der eines Buchs, als das des größten Verkehrs der Gedanken, sich doch in lauter intellektuelle Verhältnisse auflösen lasse, und so die Tafel der reinen Verträge nicht durch empirische Beimischung verunreinigen dürfe. 
	
	\subsection*{tg437.2.80} 
	\textbf{Source : }Die Metaphysik der Sitten/Erster Teil. Metaphysische Anfangsgründe der Rechtslehre/1. Teil. Das Privatrecht vom äußeren Mein und Dein überhaupt/2. Hauptstück. Von der Art, etwas Äußeres zu erwerben/3. Abschnitt. Von dem auf dingliche Art persönlichen Recht\\  
	
	\noindent\textbf{Paragraphe : }Wie ist es aber möglich, daß das, was anfänglich Ware war, endlich Geld ward? Wenn ein großer und machthabender Vertuer einer Materie, die er anfangs bloß zum Schmuck und Glanz seiner Diener (des Hofes) brauchte (z.B. Gold, Silber, Kupfer, oder eine Art schöner Muschelschalen, Kauris, oder auch, wie in Kongo, eine Art Matten, Makuten genannt, oder, wie am Senegal, Eisenstangen, und auf der Guineaküste selbst Negersklaven), d.i. wenn ein Landesherr die Abgaben von seinen Untertanen in dieser \match{Materie} (als Ware) einfordert, und die, deren Fleiß in Anschaffung derselben  dadurch bewegt werden soll, mit eben denselben, nach Verordnungen des Verkehrs unter und mit ihnen überhaupt (auf einem Markt, oder einer Börse), wieder lohnt. – Dadurch allein hat (meinem Bedünken nach) eine Ware ein gesetzliches Mittel des Verkehrs des Fleißes der Untertanen unter einander und hiemit auch des Staatsreichtums, d.i. Geld, werden können. 
	
	\subsection*{tg439.2.46} 
	\textbf{Source : }Die Metaphysik der Sitten/Erster Teil. Metaphysische Anfangsgründe der Rechtslehre/1. Teil. Das Privatrecht vom äußeren Mein und Dein überhaupt/3. Hauptstück. Von der subjektiv-bedingten Erwerbung durch den Ausspruch einer öffentlichen Gerichtsbarkeit\\  
	
	\noindent\textbf{Paragraphe : }Man kann den ersteren und zweiten Zustand den des Privatrechts, den letzteren und dritten aber den des 
	öffentlichen Rechts nennen. Dieses enthält nicht mehr, oder andere Pflichten der Menschen unter sich, als in jenem gedacht werden können; die \match{Materie} des Privatrechts ist eben dieselbe in beiden. Die Gesetze des letzteren betreffen also nur die rechtliche Form ihres Beisammenseins (Verfassung), in Ansehung deren diese Gesetze notwendig als öffentliche gedacht werden müssen. 
	
	\subsection*{tg447.2.7} 
	\textbf{Source : }Die Metaphysik der Sitten/Zweiter Teil. Metaphysische Anfangsgründe der Tugendlehre/Vorrede\\  
	
	\noindent\textbf{Paragraphe : }
	Geht man von diesem Grundsatze ab und fängt vom pathologischen, oder dem reinästhetischen, oder auch dem moralischen Gefühl (dem subjektivpraktischen statt des objektiven), d.i. von der \match{Materie} des Willens, dem Zweck, nicht von der Form desselben, d.i. dem Gesetz an, um von da aus die Pflichten zu bestimmen: so finden freilich keine metaphysischen Anfangsgründe der Tugendlehre statt – denn Gefühl, wodurch es auch immer erregt werden mag, ist jederzeit physisch. – Aber die Tugendlehre wird alsdenn auch in ihrer Quelle, einerlei ob in Schulen, oder Hörsälen, u.s.w., verderbt. Denn es ist nicht gleichviel, durch welche Triebfedern als Mittel man zu einer guten Absicht (der Befolgung aller Pflicht) hingeleitet werde. – – Es mag also den orakel– oder auch geniemäßig über Pflichtenlehre absprechenden vermeinten Weisheitslehrern Metaphysik noch so sehr anekeln: so ist es doch für die, welche sich dazu aufwerfen, unerläßliche Pflicht, selbst in der Tugendlehre zu jener ihren Grundsätzen zurückzugehen und auf ihren Bänken vorerst selbst die Schule zu machen. 
	
	\subsection*{tg449.2.6} 
	\textbf{Source : }Die Metaphysik der Sitten/Zweiter Teil. Metaphysische Anfangsgründe der Tugendlehre/Einleitung/I. Erörterung des Begriffs einer Tugendlehre\\  
	
	\noindent\textbf{Paragraphe : }Die Rechtslehre hatte es bloß mit der formalen Bedingung der äußeren Freiheit (durch die Zusammenstimmung mit sich selbst, wenn ihre Maxime zum allgemeinen Gesetz gemacht wurde), d.i. mit dem Recht zu tun. Die Ethik dagegen gibt noch eine \match{Materie} (einen Gegenstand der freien Willkür), einen Zweck der reinen Vernunft, der zugleich als objektiv-notwendiger Zweck, d.i. für den Menschen als Pflicht vorgestellt wird, an die Hand. – Denn, da die sinnlichen Neigungen zu Zwecken (als der Materie der Willkür) verleiten, die der Pflicht zuwider sein können, so  kann die gesetzgebende Vernunft ihrem Einfluß nicht anders wehren, als wiederum durch einen entgegengesetzten moralischen Zweck, der also von der Neigung unabhängig a priori gegeben sein muß. 
	
	\subsection*{tg454.2.2} 
	\textbf{Source : }Die Metaphysik der Sitten/Zweiter Teil. Metaphysische Anfangsgründe der Tugendlehre/Einleitung/VI. Die Ethik gibt nicht Gesetze für die Handlungen [...] sondern nur für die Maximen der Handlungen\\  
	
	\noindent\textbf{Paragraphe : }Der Pflichtbegriff steht unmittelbar in Beziehung auf ein Gesetz (wenn ich gleich noch von allem Zweck, als der \match{Materie} desselben, abstrahiere); wie denn das formale Prinzip der Pflicht im kategorischen Imperativ: »handle so, daß die Maxime deiner Handlung ein allgemeines Gesetz werden könne«, es schon anzeigt; nur daß in der Ethik dieses als das Gesetz deines eigenen Willens gedacht wird, nicht des Willens überhaupt, der auch der Wille anderer sein könnte: wo es alsdenn eine Rechtspflicht abgeben würde, die nicht in das Feld der Ethik gehört. – Die Maximen werden hier als solche subjektive Grundsätze angesehen, die sich zu einer allgemeinen Gesetzgebung bloß qualifizieren; welches nur ein negatives Prinzip (einem Gesetz überhaupt nicht zu widerstreiten) ist. – Wie kann es aber dann noch ein Gesetz für die Maxime der Handlungen geben? 
	
	\subsection*{tg454.2.3} 
	\textbf{Source : }Die Metaphysik der Sitten/Zweiter Teil. Metaphysische Anfangsgründe der Tugendlehre/Einleitung/VI. Die Ethik gibt nicht Gesetze für die Handlungen [...] sondern nur für die Maximen der Handlungen\\  
	
	\noindent\textbf{Paragraphe : }Der Begriff eines Zwecks, der zugleich Pflicht ist, welcher der Ethik eigentümlich zugehört, ist es allein, der ein Gesetz für die Maximen der Handlungen begründet, indem der subjektive Zweck (den jedermann hat) dem objektiven (den sich jedermann dazu machen soll) untergeordnet wird. Der Imperativ: »du sollst dir dieses oder jenes (z.B. die Glückseligkeit anderer) zum Zweck machen«, geht auf die \match{Materie} der Willkür (ein Objekt). Da nun keine freie Handlung möglich ist, ohne daß der Handelnde hiebei zugleich einen Zweck (als Materie der Willkür) beabsichtigte, so muß, wenn es einen Zweck gibt, der zugleich Pflicht ist, die Maxime der Handlungen, als Mittel zu Zwecken, nur die Bedingung der Qualifikation zu einer möglichen allgemeinen Gesetzgebung enthalten; wogegen der Zweck, der zugleich Pflicht ist, es zu einem Gesetz machen kann, eine solche Maxime zu haben, indessen daß für die Maxime selbst die bloße Möglichkeit, zu einer allgemeinen Gesetzgebung zusammen zu stimmen, schon genug ist. 
	
	\subsection*{tg457.2.4} 
	\textbf{Source : }Die Metaphysik der Sitten/Zweiter Teil. Metaphysische Anfangsgründe der Tugendlehre/Einleitung/IX. Was ist Tugendpflicht\\  
	
	\noindent\textbf{Paragraphe : }Aber, was zu tun Tugend ist, das ist darum noch nicht so fort eigentliche Tugendpflicht. Jenes kann bloß das Formale der Maximen betreffen, diese aber geht auf die  \match{Materie} derselben, nämlich auf einen Zweck, der zugleich als Pflicht gedacht wird. – Da aber die ethische Verbindlichkeit zu Zwecken, deren es mehrere geben kann, nur eine weite ist, weil sie da bloß ein Gesetz für die Maxime der Handlungen enthält und der Zweck die Materie (Objekt) der Willkür ist, so gibt es viele, nach Verschiedenheit des gesetzlichen Zwecks verschiedene, Pflichten, welche Tugendpflichten (officia honestatis) genannt werden; eben darum, weil sie bloß dem freien Selbstzwange, nicht dem anderer Menschen, unterworfen sind und die den Zweck bestimmen, der zugleich Pflicht ist. 
	
	\subsection*{tg489.2.4} 
	\textbf{Source : }Die Metaphysik der Sitten/Fußnoten\\  
	
	\noindent\textbf{Paragraphe : }
	
	2 Man kann Sinnlichkeit durch das Subjektive unserer Vorstellungen überhaupt erklären; denn der Verstand bezieht allererst die Vorstellungen auf ein Objekt, d.i. er allein denkt sich etwas vermittelst derselben. Nun kann das Subjektive unserer Vorstellung entweder von der Art sein, daß es auch auf ein Objekt zum Erkenntnis desselben (der Form oder \match{Materie} nach, da es im ersteren Falle reine Anschauung, im zweiten Empfindung heißt) bezogen werden kann. In diesem Fall ist die Sinnlichkeit, als Empfänglichkeit der gedachten Vorstellung, der Sinn: aber das Subjektive der Vorstellung kann gar kein Erkenntnisstück werden; weil es bloß die Beziehung derselben aufs Subjekt und nichts zur Erkenntnis des Objekts Brauchbares enthält, und alsdann heißt diese Empfänglichkeit der Vorstellung Gefühl; welches die Wirkung der Vorstellung (diese mag sinnlich oder intellektuell 
	sein) aufs Subjekt enthält und zur Sinnlichkeit gehört, obgleich die Vorstellung selbst zum Verstande oder der Vernunft gehören mag. 
	
	\unnumberedsection{Mathematik (2)} 
	\subsection*{tg461.2.3} 
	\textbf{Source : }Die Metaphysik der Sitten/Zweiter Teil. Metaphysische Anfangsgründe der Tugendlehre/Einleitung/XIII. Allgemeine Grundsätze der Metaphysik der Sitten in Behandlung einer reinen Tugendlehre\\  
	
	\noindent\textbf{Paragraphe : }Denn alle moralische Beweise können, als philosophische, nur vermittelst einer Vernunfterkenntnis aus Begriffen, nicht, wie die \match{Mathematik} sie gibt, durch die Konstruktion der Begriffe geführt werden; die letztern verstatten Mehrheit  der Beweise eines und desselben Satzes; weil in der Anschauung a priori es mehrere Bestimmungen der Beschaffenheit eines Objekts geben kann, die alle auf eben denselben Grund zurück führen. – Wenn z.B. für die Pflicht der Wahrhaftigkeit ein Beweis, erstlich aus dem Schaden, den die Lüge andern Menschen verursacht, dann aber auch aus der Nichtswürdigkeit eines Lügners und der Verletzung der Achtung gegen sich selbst, geführt werden will, so ist im ersteren eine Pflicht des Wohlwollens, nicht eine der Wahrhaftigkeit, mithin nicht diese, von der man den Beweis verlangte, sondern eine andere Pflicht bewiesen worden. – Was aber die Mehrheit der Beweise für einen und denselben Satz betrifft, womit man sich tröstet, daß die Menge der Gründe den Mangel am Gewicht eines jeden einzeln genommen ergänzen werde, so ist dieses ein sehr unphilosophischer Behelf: weil er Hinterlist und Unredlichkeit verrät; – denn verschiedene unzureichende Gründe, neben einander gestellt, ergänzen nicht der eine den Mangel des anderen zur Gewißheit, ja nicht einmal zur Wahrscheinlichkeit. Sie müssen als Grund und Folge in einer Reihe, bis zum zureichenden Grunde, fortschreiten und können auch nur auf solche Art beweisend sein. – Und gleichwohl ist dies der gewöhnliche Handgriff der Überredungskunst. 
	
	\subsection*{tg465.2.9} 
	\textbf{Source : }Die Metaphysik der Sitten/Zweiter Teil. Metaphysische Anfangsgründe der Tugendlehre/Einleitung/XVII. Vorbegriffe zur Einteilung der Tugendlehre\\  
	
	\noindent\textbf{Paragraphe : }Wie komme ich aber dazu, wird man fragen, die Einteilung der Ethik in Elementarlehre und Methodenlehre einzuführen: da ich ihrer doch in der Rechtslehre überhoben sein konnte? – Die Ursache ist: weil jene es mit weiten, diese aber mit lauter engen Pflichten zu tun hat; weshalb die letztere, welche ihrer Natur nach strenge (präzis) bestimmend sein muß, eben so wenig wie die reine \match{Mathematik} einer allgemeinen Vorschrift (Methode), wie im Urteilen verfahren werden soll, bedarf, sondern sie durch die Tatwahr macht. – Die Ethik hingegen führt, wegen des Spielraums, den sie ihren unvollkommenen Pflichten verstattet, unvermeidlich dahin, zu Fragen, welche die Urteilskraft auffordern auszumachen, wie eine Maxime in besonderen Fällen anzuwenden sei, und zwar so: daß diese wiederum eine (untergeordnete) Maxime an die Hand gebe (wo immer wiederum nach einem Prinzip der Anwendung dieser auf vorkommende Fälle gefragt werden kann); und so gerät sie in eine Kasuistik, von welcher die Rechtslehre nichts weiß. 
	
	\unnumberedsection{Mathematiker (1)} 
	\subsection*{tg486.2.10} 
	\textbf{Source : }Die Metaphysik der Sitten/Zweiter Teil. Metaphysische Anfangsgründe der Tugendlehre/II. Ethische Methodenlehre/1. Abschnitt. Die ethische Didaktik\\  
	
	\noindent\textbf{Paragraphe : }Was nun die doktrinale Methode betrifft (denn methodisch muß eine jede wissenschaftliche Lehre sein; sonst  wäre der Vortrag tumultuarisch): so kann sie auch nicht fragmentarisch, sondern muß systematisch sein, wenn die Tugendlehre eine Wissenschaft vorstellen soll. – Der Vortrag aber kann entweder akroamatisch, da alle andere, welchen er geschieht, bloße Zuhörer sind, oder erotematisch sein, wo der Lehrer das, was er seine Jünger lehren will, ihnen abfragt; und diese erotematische Methode ist wiederum entweder die, da er es ihrer Vernunft, die dialogische Lehrart, oder bloß ihrem Gedächtnisse abfragt, die katechetische Lehrart. Denn wenn jemand der Vernunft des anderen etwas abfragen will, so kann es nicht anders als dialogisch, d.i. dadurch geschehen: daß Lehrer und Schüler einander wechselseitig fragen und antworten. Der Lehrer leitet durch Fragen den Gedankengang seines Lehrjüngers dadurch, daß er die Anlage zu gewissen Begriffen in demselben durch vorgelegte Fälle bloß entwickelt (er ist die Hebamme seiner Gedanken); der Lehrling, welcher hiebei inne wird, daß er selbst zu denken vermöge, veranlaßt durch seine Gegenfragen (über Dunkelheit, oder den eingeräumten Sätzen entgegenstehende Zweifel), daß der Lehrer nach dem docendo discimus selbst lernt, wie er gut fragen müsse. (Denn es ist eine, an die Logik ergehende, noch nicht genugsam beherzigte Forderung: daß sie auch Regeln an die Hand gebe, wie man zweckmäßig suchen solle, d.i. nicht immer bloß für bestimmende, sondern auch für vorläufige Urteile (iudicia praevia), durch die man auf Gedanken gebracht wird; eine Lehre, die selbst dem \match{Mathematiker} zu Erfindungen ein Fingerzeig sein kann und die von ihm auch oft angewandt wird.) 
	
	\unnumberedsection{Menge (7)} 
	\subsection*{tg437.2.81} 
	\textbf{Source : }Die Metaphysik der Sitten/Erster Teil. Metaphysische Anfangsgründe der Rechtslehre/1. Teil. Das Privatrecht vom äußeren Mein und Dein überhaupt/2. Hauptstück. Von der Art, etwas Äußeres zu erwerben/3. Abschnitt. Von dem auf dingliche Art persönlichen Recht\\  
	
	\noindent\textbf{Paragraphe : }Der intellektuelle Begriff, dem der empirische vom Gelde untergelegt ist, ist also der von einer Sache, die, im Umlauf des Besitzes begriffen (permutatio publica), den Preis aller anderen Dinge (Waren) bestimmt, unter welche letztere so gar Wissenschaften, so fern sie anderen nicht umsonst gelehrt werden, gehören: dessen \match{Menge} also in einem Volk die Begüterung (opulentia) desselben ausmacht. Denn Preis (pretium) ist das öffentliche Urteil über den Wert (valor) einer Sache, in Verhältnis auf die proportionierte Menge desjenigen, was das allgemeine stellvertretende Mittel der gegenseitigen Vertauschung des Fleißes (des Umlaufs) ist. – Daher werden, wo der Verkehr groß ist, weder Gold noch Kupfer für eigentliches Geld, sondern nur für Ware gehalten; weil von dem ersteren zu wenig, vom anderen zu viel da ist, um es leicht in Umlauf zu bringen, und dennoch in so kleinen Teilen zu haben, als zum Umsatz gegen Ware, oder eine Menge derselben im kleinsten Erwerb nötig ist. Silber (weniger oder mehr mit Kupfer versetzt) wird daher im großen Verkehr der Welt für das eigentliche Material des Geldes und den Maßstab der Berechnung aller Preise genommen; die übrigen Metalle (noch viel mehr also die unmetallischen Materien) können nur in einem Volk von kleinem Verkehr statt finden. – Die erstern beiden, wenn sie nicht bloß gewogen, sondern auch gestempelt, d.i. mit einem Zeichen, für wie viel sie gelten sollen, versehen worden, sind gesetzliches Geld, d.i. Münze. 
	
	\subsection*{tg441.2.49} 
	\textbf{Source : }Die Metaphysik der Sitten/Erster Teil. Metaphysische Anfangsgründe der Rechtslehre/2. Teil. Das öffentliche Recht/1. Abschnitt. Das Staatsrecht\\  
	
	\noindent\textbf{Paragraphe : }Kann der Beherrscher als Obereigentümer (des Bodens), oder muß er nur als Oberbefehlshaber in Ansehung des Volks durch Gesetze betrachtet werden? Da der Boden die oberste Bedingung ist, unter der allein es möglich ist, äußere Sachen als das Seine zu haben, deren möglicher Besitz und Gebrauch das erste erwerbliche Recht ausmacht, so wird von dem Souverän, als Landesherren, besser als Obereigentümer (dominus territorii) alles solche Recht abgeleitet wer den müssen. Das Volk, als die \match{Menge} der Untertanen, gehört ihm auch zu (es ist sein Volk), aber nicht ihm, als Eigentümer (nach dem dinglichen), sondern als Oberbefehlshaber (nach dem persönlichen Recht). – Dieses Obereigentum ist aber nur eine Idee des bürgerlichen Vereins, um die notwendige Vereinigung des Privateigentums aller im Volk unter einem öffentlichen allgemeinen Besitzer, zu Bestimmung des besonderen Eigentums, nicht nach Grundsätzen der Aggregation (die von den Teilen zum Ganzen empirisch fortschreitet), sondern dem notwendigen formalen Prinzip der Einteilung (Division des Bodens) nach Rechtsbegriffen vorstellig zu machen. Nach diesen kann der Obereigentümer kein Privateigentum an irgend einem Boden haben (denn sonst machte er sich zu einer Privatperson), sondern dieses gehört nur dem Volk (und zwar nicht kollektiv- sondern distributiv genommen) zu; wovon doch ein nomadisch-beherrschtes Volk auszunehmen ist, als in welchem gar kein Privateigentum des Bodens statt findet. – Der Oberbefehlshaber kann also keine Domänen, d.i. Ländereien, zu seiner Privatbenutzung (zu Unterhaltung des Hofes) haben. Denn, weil es alsdenn auf sein eigen Gutbefinden ankäme, wie weit sie ausgebreitet sein sollten, so würde der Staat Gefahr laufen, alles Eigentum des Bodens in den Händen der Regierung zu sehen, und alle Untertanen als grunduntertänig (glebae adscripti)und Besitzer von dem, was immer nur Eigentum eines anderen ist, folglich aller Freiheit beraubt (servi) anzusehen. – Von einem Landesherrn kann man sagen: er besitzt nichts (zu eigen), außer sich selbst; denn, wenn er neben einem anderen im Staat etwas zu eigen hätte, so würde mit diesem ein Streit möglich sein, zu dessen Schlichtung kein Richter wäre. Aber man kann auch sagen: er besitzt alles; weil er das Befehlshaberrecht über das Volk hat (jedem das Seine zu Teil kommen zu lassen), dem alle äußere Sachen (divisim) zugehören. 
	
	\subsection*{tg441.2.6} 
	\textbf{Source : }Die Metaphysik der Sitten/Erster Teil. Metaphysische Anfangsgründe der Rechtslehre/2. Teil. Das öffentliche Recht/1. Abschnitt. Das Staatsrecht\\  
	
	\noindent\textbf{Paragraphe : }Der Inbegriff der Gesetze, die einer allgemeinen Bekanntmachung bedürfen, um einen rechtlichen Zustand hervorzubringen, ist das öffentliche Recht. – Dieses ist also ein System von Gesetzen für ein Volk, d.i. eine \match{Menge} von Menschen, oder für eine Menge von Völkern, die, im wechselseitigen Einflusse gegen einander stehend, des rechtlichen Zustandes unter einem sie vereinigenden Willen, einer Verfassung (constitutio) bedürfen, um dessen, was Rechtens ist, teilhaftig zu werden. – Dieser Zustand der einzelnen im Volke, in Verhältnis untereinander, heißt der bürgerliche (status civilis), und das Ganze derselben, in Beziehung auf seine eigene Glieder, der Staat (civitas), welcher, seiner Form wegen, als verbunden durch das gemeinsame Interesse aller, im rechtlichen Zustande zu sein, das gemeine Wesen (res publica latius sic dicta) genannt wird, in Verhältnis aber auf an dere Völker eine Macht (potentia) schlechthin heißt (daher das Wort Potentaten), was sich auch wegen (anmaßlich) angeerbter Vereinigung ein Stammvolk (gens) nennt, und so, unter dem allgemeinen Begriffe des öffentlichen Rechts, nicht bloß das Staats- sondern auch ein Völkerrecht (ius gentium) zu denken Anlaß gibt: welches dann, weil der Erdboden eine nicht grenzenlose, sondern sich selbst schließende Fläche ist, beides zusammen zu der Idee eines Völkerstaatsrechts (ius gentium) oder des Weltbürgerrechts (ius cosmopoliticum) unumgänglich hinleitet: so, daß, wenn unter diesen drei möglichen Formen des rechtlichen Zustandes es nur einer an dem die äußere Freiheit durch Gesetze einschränkenden Prinzip fehlt, das Gebäude aller übrigen unvermeidlich untergraben werden, und endlich einstürzen muß. 
	
	\subsection*{tg442.2.14} 
	\textbf{Source : }Die Metaphysik der Sitten/Erster Teil. Metaphysische Anfangsgründe der Rechtslehre/2. Teil. Das öffentliche Recht/2. Abschnitt. Das Völkerrecht\\  
	
	\noindent\textbf{Paragraphe : }Es gibt mancherlei Naturprodukte in einem Lande, die doch, was die \match{Menge} derselben von einer gewissen Art betrifft, zugleich als Gemächsel (artefacta) des Staats angesehen werden müssen, weil das Land sie in solcher Menge nicht liefern würde, wenn es nicht einen Staat und eine ordentliche machthabende Regierung gäbe, sondern die Bewohner im Stande der Natur wären. – Haushühner (die nützlichste Art des Geflügels), Schafe, Schweine, das Rindergeschlecht u.a.m. würden, entweder aus Mangel an Futter, oder der Raubtiere wegen, in dem Lande, wo ich lebe, entweder gar nicht, oder höchst sparsam anzutreffen sein, wenn es darin nicht eine Regierung gäbe, welche den Einwohnern ihren Erwerb und Besitz sicherte. – Eben das gilt auch von der Menschenzahl, die, eben so wie in den amerikanischen Wüsten, ja selbst dann, wenn man diesen den größten Fleiß (den jene nicht haben) beilegte, nur gering sein kann. Die Einwohner würden nur sehr dünn gesäet sein, weil keiner derselben sich, mit samt seinem Gesinde, auf einem Boden weit verbreiten könnte, der immer in Gefahr ist, von Menschen oder Wilden und Raubtieren verwüstet zu werden; mithin sich für eine so große Menge von Menschen, als jetzt auf einem Lande leben, kein hinlänglicher Unterhalt finden würde. – – Sowie man nun von Gewächsen (z.B. den Kartoffeln) und von Haustieren, weil sie, was die Menge betrifft, ein Machwerk der Menschen sind, sagen kann, daß man sie gebrauchen, verbrauchen und verzehren (töten lassen) kann: so, scheint es, könne man auch von der obersten Gewalt im Staat, dem Souverän, sagen, er  habe das Recht, seine Untertanen, die dem größten Teil nach sein eigenes Produkt sind, in den Krieg, wie auf eine Jagd, und zu einer Feldschlacht, wie auf eine Lustpartie zu führen. 
	
	\subsection*{tg444.2.4} 
	\textbf{Source : }Die Metaphysik der Sitten/Erster Teil. Metaphysische Anfangsgründe der Rechtslehre/Beschluß\\  
	
	\noindent\textbf{Paragraphe : }Man kann sagen, daß diese allgemeine und fortdauernde Friedensstiftung nicht bloß einen Teil, sondern den ganzen Endzweck der Rechtslehre innerhalb den Grenzen der bloßen Vernunft ausmache; denn der Friedenszustand ist allein der unter Gesetzen gesicherte Zustand des Mein und Dein in einer \match{Menge} einander benachbarter Menschen, mithin die in einer Verfassung zusammen sind, deren Regel aber nicht von der Erfahrung derjenigen, die sich bisher am besten dabei befunden haben, als einer Norm für andere, sondern die durch die Vernunft a priori von dem Ideal einer rechtlichen Verbindung der Menschen unter öffentlichen Gesetzen überhaupt hergenommen werden muß, weil alle Beispiele (als die nur erläutern, aber nichts beweisen können) trüglich sind, und so allerdings einer Metaphysik bedürfen, deren Notwendigkeit diejenigen, die dieser spotten, doch unvorsichtiger Weise selbst zugestehen, wenn sie z.B., wie sie es oft tun, sagen: »die beste Verfassung ist die, wo nicht die Menschen, sondern die Gesetze machthabend sind«. Denn was kann mehr metaphysisch sublimiert sein, als eben diese Idee, welche gleichwohl, nach jener ihrer eigenen Behauptung, die bewährteste objektive Realität hat, die sich auch in vorkommenden Fällen leicht darstellen läßt, und welche allein, wenn sie nicht revolutionsmäßig, durch einen Sprung, d.i. durch gewaltsame Umstürzung einer bisher bestandenen fehlerhaften – (denn da würde sich zwischeninne ein Augenblick der Vernichtung alles rechtlichen Zustandes ereignen) sondern durch allmähliche Reform nach festen Grundsätzen versucht und durchgeführt wird, in kontinuierlicher Annäherung zum höchsten politischen Gut, zum ewigen Frieden, hinleiten kann. 
	
	\subsection*{tg471.2.51} 
	\textbf{Source : }Die Metaphysik der Sitten/Zweiter Teil. Metaphysische Anfangsgründe der Tugendlehre/I. Ethische Elementarlehre/I. Teil. Von den Pflichten gegen sich selbst überhaupt/Erstes Buch. Von den vollkommenen Pflichten gegen sich selbst/Erstes Hauptstück. Die Pflicht des Menschen gegen sich selbst, als einem animalischen Wesen\\  
	
	\noindent\textbf{Paragraphe : }Der Schmaus, als förmliche Einladung zur Unmäßigkeit in beiderlei Art des Genusses, hat doch, außer dem bloß physischen Wohlleben, noch etwas zum sittlichen Zweck Abzielendes an sich, nämlich viel Menschen und lange zu wechselseitiger Mitteilung zusammen zu halten: gleichwohl aber, da eben die \match{Menge} (wenn sie, wie Chesterfield sagt, über die Zahl der Musen geht) nur eine kleine Mitteilung (mit den nächsten Beisitzern) erlaubt, mithin die Veranstaltung jenem Zweck widerspricht, so bleibt sie immer Verleitung zum Unsittlichen, nämlich der Unmäßigkeit, der Übertretung der Pflicht gegen sich selbst; auch ohne auf die physischen Nachteile der Überladung, die vielleicht vom Arzt gehoben werden können, zu sehen. Wie weit geht die sittliche Befugnis, diesen Einladungen zur Unmäßigkeit Gehör zu geben? 
	
	\subsection*{tg488.2.16} 
	\textbf{Source : }Die Metaphysik der Sitten/Zweiter Teil. Metaphysische Anfangsgründe der Tugendlehre/Beschluß. Die Religionslehre als Lehre der Pflichten gegen Gott liegt außerhalb den Grenzen der reinen Moralphilosophie\\  
	
	\noindent\textbf{Paragraphe : }Die Strafe läßt (nach dem Horaz) den vor ihr stolz schreitenden Verbrecher nicht aus den Augen, sondern hinkt ihm unablässig nach, bis sie ihn ertappt. – Das unschuldig vergossene Blut schreit um Rache. – Das Verbrechen kann nicht ungerächt bleiben; trifft die Strafe nicht den Verbrecher, so werden es seine Nachkommen entgelten müssen; oder geschieht's nicht bei seinem Leben, so muß es in einem Leben nach dem Tode
	
	
	25
	geschehen, welches ausdrücklich darum auch angenommen und gern geglaubt wird, damit der Anspruch der ewigen Gerechtigkeit ausgeglichen werde. – Ich will keine Blutschuld auf mein Land kommen lassen, dadurch, daß ich einen boshaft mordenden Duellanten, für den ihr Fürbitte tut, begnadige, sagte einmal ein wohldenkender Landesherr. – Die Sündenschuld muß bezahlt werden, und sollte sich auch ein völlig Unschuldiger zum Sühnopfer  hingeben (wo dann freilich die von ihm übernommene Leiden eigentlich nicht Strafe – denn er hat selbst nichts verbrochen – heißen könnten); aus welchen allen zu ersehen ist, daß es nicht eine die Gerechtigkeit verwaltende Person ist, der man diesen Verurteilungsspruch beilegt (denn die würde nicht so sprechen können, ohne anderen unrecht zu tun), sondern daß die bloße Gerechtigkeit, als überschwengliches, einem übersinnlichen Subjekt angedachtes Prinzip, das Recht dieses Wesens bestimme; welches zwar dem Formalen dieses Prinzips gemäß ist, dem Materialen desselben aber, dem Zweck, welcher immer die Glückseligkeit der Menschen ist, widerstreitet. – Denn, bei der etwanigen großen \match{Menge} der Verbrecher, die ihr Schuldenregister immer so fortlaufen lassen, würde die Strafgerechtigkeit den Zweck der Schöpfung nicht in der Liebe des Welturhebers (wie man sich doch denken muß), sondern in der strengen Befolgung des Rechts setzen (das Recht selbst zum Zweck machen, der in der Ehre Gottes gesetzt wird), welches, da das letztere (die Gerechtigkeit) nur die einschränkende Bedingung des ersteren (der Gütigkeit) ist, den Prinzipien der praktischen Vernunft zu widersprechen scheint, nach welchen eine Weltschöpfung hätte unterbleiben müssen, die ein, der Absicht ihres Urhebers, die nur Liebe zum Grunde haben kann, so widerstreitendes Produkt geliefert haben würde. 
	
	\unnumberedsection{Physik (1)} 
	\subsection*{tg483.2.4} 
	\textbf{Source : }Die Metaphysik der Sitten/Zweiter Teil. Metaphysische Anfangsgründe der Tugendlehre/I. Ethische Elementarlehre/II. Teil. Von den Tugendpflichten gegen andere/Zweites Hauptstück. Von den ethischen Pflichten der Menschen gegen einander in Ansehung ihres Zustandes\\  
	
	\noindent\textbf{Paragraphe : }Diese (Tugendpflichten) können zwar in der reinen Ethik keinen Anlaß zu einem besondern Hauptstück im System derselben geben, denn sie enthalten nicht Prinzipien der Verpflichtung der Menschen als solcher gegen einander, und können also von den metaphysischen Anfangsgründen der Tugendlehre eigentlich nicht einen Teil abgeben, sondern sind nur, nach Verschiedenheit der Subjekte der Anwendung des Tugendprinzips (dem Formale nach) auf in der Erfahrung vorkommende Fälle (das Materiale) modifizierte, Regeln, weshalb sie auch, wie alle empirische Einteilungen, keine gesichert-vollständige Klassifikation zulassen. Indessen, gleichwie von der Metaphysik der Natur zur \match{Physik} ein Überschritt, der seine besondern Regeln hat, verlangt wird: so wird der Metaphysik der Sitten ein Ähnliches mit Recht angesonnen: nämlich durch Anwendung reiner Pflichtprinzipien auf Fälle der Erfahrung jene gleichsam  zu schematisieren und zum moralisch-praktischen Gebrauch fertig darzulegen, – Welches Verhalten also gegen Menschen, z.B. in der moralischen Reinigkeit ihres Zustandes, oder in ihrer Verdorbenheit; welches im kultivierten, oder rohen Zustände; was den Gelehrten oder Ungelehrten, und jenen im Gebrauch ihrer Wissenschaft als umgänglichen (geschliffenen), oder in ihrem Fach unumgänglichen Gelehrten (Pedanten), pragmatischen oder mehr auf Geist und Geschmack ausgehenden; welches nach Verschiedenheit der Stände, des Alters, des Geschlechts, des Gesundheitszustandes, des der Wohlhabenheit oder Armut u.s.w. zukomme: das gibt nicht so vielerlei Arten der ethischen Verpflichtung (denn es ist nur eine, nämlich die der Tugend überhaupt), sondern nur Arten der Anwendung (Porismen) ab; die also nicht, als Abschnitte der Ethik und Glieder der Einteilung eines Systems (das a priori aus einem Vernunftbegriffe hervorgehen muß), aufgeführt, sondern nur angehängt werden können. – Aber eben diese Anwendung gehört zur Vollständigkeit der Darstellung desselben. 
	
	\unnumberedsection{Postulat (5)} 
	\subsection*{tg433.2.11} 
	\textbf{Source : }Die Metaphysik der Sitten/Erster Teil. Metaphysische Anfangsgründe der Rechtslehre/1. Teil. Das Privatrecht vom äußeren Mein und Dein überhaupt/1. Hauptstück\\  
	
	\noindent\textbf{Paragraphe : }Man kann dieses \match{Postulat} ein Erlaubnisgesetz (lex permissiva) der praktischen Vernunft nennen, was uns die Befugnis gibt, die wir aus bloßen Begriffen vom Rechte überhaupt nicht herausbringen könnten; nämlich allen andern eine Verbindlichkeit aufzulegen, die sie sonst nicht hätten, sich des Gebrauchs gewisser Gegenstände unserer Willkür zu enthalten, weil wir zuerst sie in unseren Besitz genommen haben. Die Vernunft will, daß dieses als Grundsatz gelte, und das zwar als praktische Vernunft, die sich durch dieses ihr Postulat a priori erweitert. 
	
	\subsection*{tg433.2.48} 
	\textbf{Source : }Die Metaphysik der Sitten/Erster Teil. Metaphysische Anfangsgründe der Rechtslehre/1. Teil. Das Privatrecht vom äußeren Mein und Dein überhaupt/1. Hauptstück\\  
	
	\noindent\textbf{Paragraphe : }
	Auflösung: Beide Sätze sind wahr: der erstere, wenn ich den empirischen Besitz (possessio phaenomenon), der andere, wenn ich unter diesem Wort den reinen intelligibelen Besitz (possessio noumenon) verstehe. – Aber die Möglichkeit eines intelligibelen Besitzes, mithin auch des äußeren Mein und Dein läßt sich nicht einsehen,  sondern muß aus dem \match{Postulat} der praktischen Vernunft gefolgert werden, wobei es noch besonders merkwürdig ist: daß diese, ohne Anschauungen, selbst ohne einer a priori zu bedürfen, sich durch bloße, vom Gesetz der Freiheit berechtigte, Weglassung empirischer Bedingungen erweitere und so synthetische Rechtssätze a priori aufstellen kann, deren Beweis (wie bald gezeigt werden soll) nachher in praktischer Rücksicht auf analytische Art geführt werden kann. 
	
	\subsection*{tg434.2.6} 
	\textbf{Source : }Die Metaphysik der Sitten/Erster Teil. Metaphysische Anfangsgründe der Rechtslehre/1. Teil. Das Privatrecht vom äußeren Mein und Dein überhaupt\\  
	
	\noindent\textbf{Paragraphe : }Das Prinzip der äußeren Erwerbung ist nun: Was ich (nach dem Gesetz der äußeren Freiheit) in meine Gewalt bringe, und wovon, als Objekt meiner Willkür, Gebrauch zu machen ich (nach dem \match{Postulat} der praktischen Vernunft) das Vermögen habe, endlich, was ich (gemäß der Idee eines möglichen vereinigten Willens) will, es solle mein sein, das ist mein. 
	
	\subsection*{tg435.2.25} 
	\textbf{Source : }Die Metaphysik der Sitten/Erster Teil. Metaphysische Anfangsgründe der Rechtslehre/1. Teil. Das Privatrecht vom äußeren Mein und Dein überhaupt/2. Hauptstück. Von der Art, etwas Äußeres zu erwerben/1. Abschnitt. Vom Sachrecht\\  
	
	\noindent\textbf{Paragraphe : }Gleichwohl ist jene provisorische dennoch eine wahre Erwerbung; denn, nach dem \match{Postulat} der recht lich-praktischen Vernunft, ist die Möglichkeit derselben, in welchem Zustande die Menschen neben einander sein mögen (also auch im Naturzustande), ein Prinzip des Privatrechts, nach welchem jeder zu demjenigen Zwange berechtigt ist, durch welchen es allein möglich wird, aus jenem Naturzustande heraus zu gehen, und in den bürgerlichen, der allein alle Erwerbung peremtorisch machen kann, zu treten. 
	
	\subsection*{tg439.2.50} 
	\textbf{Source : }Die Metaphysik der Sitten/Erster Teil. Metaphysische Anfangsgründe der Rechtslehre/1. Teil. Das Privatrecht vom äußeren Mein und Dein überhaupt/3. Hauptstück. Von der subjektiv-bedingten Erwerbung durch den Ausspruch einer öffentlichen Gerichtsbarkeit\\  
	
	\noindent\textbf{Paragraphe : }Aus dem Privatrecht im natürlichen Zustande geht nun das \match{Postulat} des öffentlichen Rechts hervor: du sollst, im Verhältnisse eines unvermeidlichen Nebeneinanderseins, mit allen anderen, aus jenem heraus, in einen rechtlichen Zustand, d.i. den einer austeilenden Gerechtigkeit, übergehen. – Der Grund davon läßt sich analytisch aus dem Begriffe des Rechts, im äußeren Verhältnis, im Gegensatz der Gewalt (violentia) entwickeln. 
	
	\unnumberedsection{Punkt (1)} 
	\subsection*{tg489.2.40} 
	\textbf{Source : }Die Metaphysik der Sitten/Fußnoten\\  
	
	\noindent\textbf{Paragraphe : }
	
	18 So sagt man, wenn es z.B. einen \match{Punkt} meiner Ehrenrettung oder der Selbsterhaltung betrifft: »ich bin mir das selbst schuldig«. Selbst wenn es Pflichten von minderer Bedeutung, die nämlich nicht das Notwendige, sondern nur das Verdienstliche meiner Pflichtbefolgung betreffen, spreche ich so, z.B.: »ich bin es mir selbst schuldig, meine Geschicklichkeit für den Umgang mit Menschen u.s.w. zu erweitern (mich zu kultivieren)«. 
	
	\unnumberedsection{Punktlichkeit (1)} 
	\subsection*{tg456.2.5} 
	\textbf{Source : }Die Metaphysik der Sitten/Zweiter Teil. Metaphysische Anfangsgründe der Tugendlehre/Einleitung/VIII. Exposition der Tugendpflichten als weiter Pflichten\\  
	
	\noindent\textbf{Paragraphe : }b) Kultur der Moralität in uns. Die größte moralische Vollkommenheit des Menschen ist: seine Pflicht zu tun und zwar aus Pflicht (daß das Gesetz nicht bloß die Regel sondern auch die Triebfeder der Handlungen sei). – Nun scheint dieses zwar beim ersten Anblick eine enge Verbindlichkeit zu sein und das Pflichtprinzip zu jeder Handlung nicht bloß die Legalität, sondern auch die Moralität, d.i. Gesinnung, mit der \match{Pünktlichkeit} und Strenge eines Gesetzes zu gebieten; aber in der Tat gebietet das Gesetz auch hier nur, die Maxime der Handlung, nämlich den Grund der Verpflichtung nicht in den sinnlichen Antrieben (Vorteil oder Nachteil), sondern ganz und gar im Gesetz zu suchen – mithin nicht die Handlung selbst. – – Denn es ist dem Menschen nicht möglich, so in die Tiefe seines eigenen Herzens einzuschauen, daß er jemals von der Reinigkeit seiner moralischen Absicht und der Lauterkeit seiner Gesinnung auch nur in einer Handlung völlig gewiß sein könnte; wenn er gleich über die Legalität derselben gar nicht zweifelhaft ist. Vielmals wird Schwäche, welche das Wagstück eines Verbrechens abrät, von demselben Menschen für Tugend (die den Begriff von Stärke gibt) gehalten, und wie viele mögen ein langes schuldloses Leben geführt haben, die nur Glückliche sind, so vielen Versuchungen entgangen zu sein; wie viel reiner moralischer Gehalt bei jeder Tat in der Gesinnung gelegen habe, das bleibt ihnen selbst verborgen. 
	
	\unnumberedsection{Reflexion (1)} 
	\subsection*{tg471.2.37} 
	\textbf{Source : }Die Metaphysik der Sitten/Zweiter Teil. Metaphysische Anfangsgründe der Tugendlehre/I. Ethische Elementarlehre/I. Teil. Von den Pflichten gegen sich selbst überhaupt/Erstes Buch. Von den vollkommenen Pflichten gegen sich selbst/Erstes Hauptstück. Die Pflicht des Menschen gegen sich selbst, als einem animalischen Wesen\\  
	
	\noindent\textbf{Paragraphe : }Die Geschlechtsneigung wird auch Liebe (in der engsten Bedeutung des Worts) genannt und ist in der Tat die größte Sinnenlust, die an einem Gegenstande möglich ist; – nicht bloß sinnliche Lust, wie an Gegenständen, die in der bloßen \match{Reflexion} über sie gefallen (da die Empfänglichkeit für sie Geschmack heißt), sondern die Lust aus dem Genusse einer anderen Person, die also zum Begehrungsvermögen und zwar der höchsten Stufe desselben, der Leidenschaft, gehört. Sie kann aber weder zur Liebe des Wohlgefallens, noch der des Wohlwollens gezählt werden (denn beide halten eher vom fleischlichen Genuß ab), sondern ist eine Lust von besonderer Art (sui generis) und das Brünstigsein hat mit der moralischen Liebe eigentlich nichts gemein, wiewohl sie mit der letzteren, wenn die praktische Vernunft mit ihren einschränkenden Bedingungen hinzu kommt, in enge Verbindung treten kann. 
	
	\unnumberedsection{Reihe (2)} 
	\subsection*{tg439.2.27} 
	\textbf{Source : }Die Metaphysik der Sitten/Erster Teil. Metaphysische Anfangsgründe der Rechtslehre/1. Teil. Das Privatrecht vom äußeren Mein und Dein überhaupt/3. Hauptstück. Von der subjektiv-bedingten Erwerbung durch den Ausspruch einer öffentlichen Gerichtsbarkeit\\  
	
	\noindent\textbf{Paragraphe : }Denn alles Veräußerliche muß von irgend jemand können erworben werden. Die Rechtmäßigkeit der Erwerbung aber beruht gänzlich auf der Form, nach welcher das, was im Besitz eines anderen ist, auf mich übertragen und von mir angenommen wird, d.i. auf der Förmlichkeit des rechtlichen Akts des Verkehrs (commutatio) zwischen dem Besitzer der Sache und dem Erwerbenden, ohne daß ich fragen darf, wie jener dazu gekommen sei; weil dieses schon Beleidigung sein würde (quilibet praesumitur bonus, donec etc.). Gesetzt nun, es ergäbe sich in der Folge, daß jener nicht Eigentümer sei, sondern ein anderer, so kann ich nicht sagen, daß dieser sich gerade zu an mich halten könnte (so wie auch an jeden anderen, der Inhaber der Sache sein möchte). Denn ich habe ihm nichts entwandt, sondern, z.B. das Pferd, was auf öffentlichem Markte feil geboten wurde, dem Gesetze gemäß (titulo ernti venditi) erstanden; weil der Titel der Erwerbung meinerseits unbestritten ist, ich aber (als Käufer) den Titel des Besitzes des anderen (des Verkäufers) nachzusuchen – da diese Nachforschung in der aufsteigenden \match{Reihe} ins Unendliche gehen würde – nicht verbunden, ja so gar nicht einmal befugt bin. Also bin ich, durch den gehörigbetitelten Kauf, nicht der bloß putative, sondern der wahre Eigentümer des Pferdes geworden. 
	
	\subsection*{tg445.2.65} 
	\textbf{Source : }Die Metaphysik der Sitten/Erster Teil. Metaphysische Anfangsgründe der Rechtslehre/Anhang erläutender Bemerkungen zu den metaphysischen Anhangsgründen der Rechtslehre\\  
	
	\noindent\textbf{Paragraphe : }Was endlich die Majoratsstiftung betrifft, da ein Gutsbesitzer durch Erbeseinsetzung verordnet: daß in der \match{Reihe} der auf einander folgenden Erben immer der Nächste von der Familie der Gutsherr sein solle (nach der Analogie mit einer monarchisch-erblichen Verfassung eines Staats, wo der Landesherr es ist), so kann eine solche Stiftung nicht allein mit Beistimmung aller Agnaten jederzeit aufgehoben werden und darf nicht auf ewige Zeiten – gleich als ob das Erbrecht am Boden haftete – immerwährend fortdauern, noch gesagt werden, es sei eine Verletzung der Stiftung und des Willens des Urahnherrn derselben, des Stifters, sie eingehen zu lassen: sondern der Staat hat auch hier ein Recht, ja sogar die Pflicht, bei den allmählich eintretenden Ursachen seiner eigenen Reform ein solches föderatives System seiner Untertanen, gleich als Unterkönige (nach der Analogie von Dynasten und Satrapen), wenn es erloschen ist, nicht weiter aufkommen zu lassen. 
	
	\unnumberedsection{Satz (10)} 
	\subsection*{tg429.2.7} 
	\textbf{Source : }Die Metaphysik der Sitten/Erster Teil. Metaphysische Anfangsgründe der Rechtslehre/Vorrede\\  
	
	\noindent\textbf{Paragraphe : }
	Es klingt arrogant, selbstsüchtig, und für die, welche ihrem alten System noch nicht entsagt haben, verkleinerlich, zu behaupten: »daß vor dem Entstehen der kritischen Philosophie es noch gar keine gegeben habe«. – Um nun über diese scheinbare Anmaßung absprechen zu können, kommt es auf die Frage an: ob es wohl mehr als eine Philosophie geben könne? Verschiedene Arten zu philosophieren, und zu den ersten Vernunftprinzipien zurückzugehen, um darauf, mit mehr oder weniger Glück, ein System zu gründen, hat es nicht allein gegeben, sondern es mußte viele Versuche dieser Art, deren jeder auch um die gegenwärtige sein Verdienst hat, geben; aber, da es doch, objektiv betrachtet, nur Eine menschliche Vernunft geben kann: so kann es auch nicht viel Philosophien geben, d.i. es ist nur Ein wahres System derselben aus Prinzipien möglich, so mannigfaltig und oft widerstreitend man auch über einen und denselben \match{Satz} philosophiert haben mag. So sagt der Moralist mit Recht: es gibt nur Eine Tugend und Lehre derselben, d.i. ein einziges System, das alle Tugendpflichten durch Ein Prinzip verbindet; der Chymist: es gibt nur Eine Chemie (die nach Lavoisier); der Arzneilehrer: es gibt nur Ein Prinzip zum System der Krankheitseinteilung (nach Brown), ohne doch darum, weil das neue System alle andere ausschließt, das Verdienst der älteren (Moralisten, Chemiker und Arzneilehrer) zu schmälern; weil, ohne dieser ihre Entdeckungen, oder auch mißlungene Versuche, wir zu jener Einheit des wahren Prinzips der ganzen Philosophie in einem System nicht gelanget wären. – Wenn also jemand ein System der Philosophie als sein eigenes Fabrikat ankündigt, so ist es eben so viel, als ob er sagte: »vor dieser Philosophie sei gar keine andere noch gewesen«. Denn wollte er einräumen, es wäre eine andere (und wahre) gewesen, so würde es über dieselbe Gegenstände zweierlei wahre Philosophien gegeben haben, welches sich widerspricht. – Wenn also die kritische Philosophie sich als eine solche ankündigt, vor der es überall noch gar keine Philosophie gegeben habe, so tut sie nichts anders, als was alle getan haben, tun werden, ja tun müssen, die eine Philosophie nach ihrem eigenen Plane entwerfen. 
	
	\subsection*{tg430.2.50} 
	\textbf{Source : }Die Metaphysik der Sitten/Erster Teil. Metaphysische Anfangsgründe der Rechtslehre/Einleitung in die Metaphysik der Sitten\\  
	
	\noindent\textbf{Paragraphe : }Die Freiheit der Willkür aber kann nicht durch das Vermögen der Wahl, für oder wider das Gesetz zu handeln  (libertas indifferentiae), definiert werden – wie es wohl einige versucht haben –, obzwar die Willkür als Phänomen davon in der Erfahrung häufige Beispiele gibt. Denn die Freiheit (so wie sie uns durchs moralische Gesetz allererst kundbar wird) kennen wir nur als negative Eigenschaft in uns, nämlich durch keine sinnliche Bestimmungsgründe zum Handeln genötigt zu werden. Als Noumen aber, d.i. nach dem Vermögen des Menschen bloß als Intelligenz betrachtet, wie sie in Ansehung der sinnlichen Willkür nötigend ist, mithin ihrer positiven Beschaffenheit nach, können wir sie theoretisch gar nicht darstellen. Nur das können wir wohl einsehen: daß, obgleich der Mensch, als Sinnenwesen, der Erfahrung nach ein Vermögen zeigt, dem Gesetze nicht allein gemäß, sondern auch zuwider zu wählen, dadurch doch nicht seine Freiheit als intelligiblen Wesens definiert werden könne, weil Erscheinungen kein übersinnliches Objekt (dergleichen doch die freie Willkür ist) verständlich machen können, und daß die Freiheit nimmermehr darin gesetzt werden kann, daß das vernünftige Subjekt auch eine wider seine (gesetzgebende) Vernunft streitende Wahl treffen kann, wenn gleich die Erfahrung oft genug beweist, daß es geschieht (wovon wir doch die Möglichkeit nicht begreifen können). – Denn ein anderes ist, einen \match{Satz} (der Erfahrung) einräumen, ein anderes, ihn zum Erklärungsprinzip (des Begriffs der freien Willkür) und allgemeinen Unterscheidungsmerkmal (vom arbitrio bruto s. servo) machen; weil das erstere nicht behauptet, daß das Merkmal notwendig zum Begriff gehöre, welches doch zum zweiten erforderlich ist. – Die Freiheit, in Beziehung auf die innere Gesetzgebung der Vernunft, ist eigentlich allein ein Vermögen; die Möglichkeit, von dieser abzuweichen, ein Unvermögen. Wie kann nun jenes aus diesem erklärt werden? Es ist eine Definition, die über den praktischen Begriff noch die Ausübung desselben, wie sie die Erfahrung lehrt, hinzutut, eine Bastarderklärung (definitio hybrida), welche den Begriff im falschen Lichte darstellt. 
	
	\subsection*{tg431.2.23} 
	\textbf{Source : }Die Metaphysik der Sitten/Erster Teil. Metaphysische Anfangsgründe der Rechtslehre/Einleitung in die Rechtslehre\\  
	
	\noindent\textbf{Paragraphe : }Dieser \match{Satz} will so viel sagen, als: das Recht darf nicht als aus zwei Stücken, nämlich der Verbindlichkeit nach einem Gesetze und der Befugnis dessen, der durch seine Willkür den andern verbindet, diesen dazu zu zwingen, zusammengesetzt gedacht werden, sondern man kann den Begriff des Rechts in der Möglichkeit der Verknüpfung des allgemeinen wechselseitigen Zwanges mit jedermanns Freiheit unmittelbar setzen. So, wie nämlich das Recht überhaupt nur das zum Objekte hat, was in Handlungen äußerlich ist, so ist das strikte Recht, nämlich das, dem nichts Ethisches beigemischt ist, dasjenige, welches keine andern Bestimmungsgründe der Willkür als bloß die äußern fordert; denn alsdenn ist es rein und mit keinen Tugendvorschriften vermengt. Ein striktes (enges) Recht kann man also nur das völlig äußere nennen. Dieses gründet sich nun zwar auf dem Bewußtsein der Verbindlichkeit eines jeden nach dem Gesetze, aber die Willkür darnach zu bestimmen darf und kann es, wenn es rein sein soll, sich auf dieses Bewußtsein als Triebfeder nicht berufen, sondern fußet sich deshalb auf dem Prinzip der Möglichkeit eines äußeren Zwanges, der mit der Freiheit von jedermann nach allgemeinen Gesetzen zusammen bestehen kann. – Wenn also gesagt wird: ein Gläubiger hat ein Recht, von dem Schuldner die Bezahlung seiner Schuld zu fordern, so bedeutet das nicht, er kann ihm zu Gemüte führen, daß ihn seine Vernunft selbst zu dieser Leistung verbinde, sondern ein Zwang, der jedermann nötigt, dieses zu tun, kann gar wohl mit jedermanns Freiheit, also auch mit der seinigen, nach einem allgemeinen äußeren  Gesetze zusammen bestehen: Recht und Befugnis zu zwingen bedeuten also einerlei. 
	
	\subsection*{tg431.2.66} 
	\textbf{Source : }Die Metaphysik der Sitten/Erster Teil. Metaphysische Anfangsgründe der Rechtslehre/Einleitung in die Rechtslehre\\  
	
	\noindent\textbf{Paragraphe : }Warum wird aber die Sittenlehre (Moral) gewöhnlich (namentlich vom Cicero) die Lehre von den Pflichten und nicht auch von den Rechten betitelt? da doch die einen sich auf die andern beziehen. – Der Grund ist dieser: Wir kennen unsere eigene Freiheit (von der alle moralische Gesetze, mithin auch alle Rechte sowohl als Pflichten ausgehen) nur durch den moralischen Imperativ, welcher ein pflichtgebieten der \match{Satz} ist, aus welchem nachher das Vermögen, andere zu verpflichten, d.i. der Begriff des Rechts, entwickelt werden kann. 
	
	\subsection*{tg433.2.28} 
	\textbf{Source : }Die Metaphysik der Sitten/Erster Teil. Metaphysische Anfangsgründe der Rechtslehre/1. Teil. Das Privatrecht vom äußeren Mein und Dein überhaupt/1. Hauptstück\\  
	
	\noindent\textbf{Paragraphe : }Alle Rechtssätze sind Sätze a priori, denn sie sind Vernunftgesetze (dictamina rationis). Der Rechtssatz a priori in Ansehung des empirischen Besitzes ist analytisch; denn er sagt nichts mehr, als was nach dem \match{Satz} des Widerspruchs aus dem letzteren folgt, daß nämlich, wenn ich Inhaber einer Sache (mit ihr also physisch verbunden) bin, derjenige, der sie wider meine Einwilligung affiziert (z.B. mir den Apfel aus der Hand reißt), das innere Meine (meine Freiheit) affiziere und schmälere, mithin in seiner Maxime mit dem Axiom des Rechts im geraden Widerspruch stehe. Der Satz von einem empirischen rechtmäßigen Besitz geht also nicht über das Recht einer Person in Ansehung ihrer selbst hinaus. 
	
	\subsection*{tg433.2.29} 
	\textbf{Source : }Die Metaphysik der Sitten/Erster Teil. Metaphysische Anfangsgründe der Rechtslehre/1. Teil. Das Privatrecht vom äußeren Mein und Dein überhaupt/1. Hauptstück\\  
	
	\noindent\textbf{Paragraphe : }Dagegen geht der Satz: von der Möglichkeit des Besitzes einer Sache außer mir, nach Absonderung aller Bedingungen des empirischen Besitzes im Raum und Zeit, (mithin  die Voraussetzung der Möglichkeit einer possessio noumenon) über jene einschränkende Bedingungen hinaus, und, weil er einen Besitz auch ohne Inhabung als notwendig zum Begriffe des äußeren Mein und Dein statuiert, so ist er synthetisch und nun kann es zur Aufgabe für die Vernunft dienen, zu zeigen, wie ein solcher sich über den Begriff des empirischen Besitzes erweiternde \match{Satz} a priori möglich sei. 
	
	\subsection*{tg433.2.46} 
	\textbf{Source : }Die Metaphysik der Sitten/Erster Teil. Metaphysische Anfangsgründe der Rechtslehre/1. Teil. Das Privatrecht vom äußeren Mein und Dein überhaupt/1. Hauptstück\\  
	
	\noindent\textbf{Paragraphe : }Der \match{Satz} heißt: Es ist möglich, etwas Äußeres als das Meine zu haben; ob ich gleich nicht im Besitz desselben bin. 
	
	\subsection*{tg435.2.12} 
	\textbf{Source : }Die Metaphysik der Sitten/Erster Teil. Metaphysische Anfangsgründe der Rechtslehre/1. Teil. Das Privatrecht vom äußeren Mein und Dein überhaupt/2. Hauptstück. Von der Art, etwas Äußeres zu erwerben/1. Abschnitt. Vom Sachrecht\\  
	
	\noindent\textbf{Paragraphe : }Was das erste betrifft, so gründet sich dieser \match{Satz} auf dem Postulat der praktischen Vernunft (§ 2); das zweite auf folgenden Beweis. 
	
	\subsection*{tg435.2.23} 
	\textbf{Source : }Die Metaphysik der Sitten/Erster Teil. Metaphysische Anfangsgründe der Rechtslehre/1. Teil. Das Privatrecht vom äußeren Mein und Dein überhaupt/2. Hauptstück. Von der Art, etwas Äußeres zu erwerben/1. Abschnitt. Vom Sachrecht\\  
	
	\noindent\textbf{Paragraphe : }Der empirische Titel der Erwerbung war die auf ursprüngliche Gemeinschaft des Bodens gegründete physische Besitznehmung (apprehensio physica), welchem, weil dem Besitz nach Vernunftbegriffen des Rechts nur ein Besitz in der Erscheinung untergelegt werden kann, der einer intellektuellen Besitznehmung (mit Weglassung aller empirischen  Bedingungen in Raum und Zeit) korrespondieren muß, und die den \match{Satz} gründet: »was ich nach Gesetzen der äußeren Freiheit in meine Gewalt bringe, und will, es solle mein sein, das wird mein«. 
	
	\subsection*{tg455.2.2} 
	\textbf{Source : }Die Metaphysik der Sitten/Zweiter Teil. Metaphysische Anfangsgründe der Tugendlehre/Einleitung/VII. Die ethischen Pflichten sind von weiter, dagegen die Rechtspflichten von enger Verbindlichkeit\\  
	
	\noindent\textbf{Paragraphe : }Dieser \match{Satz} ist eine Folge aus dem vorigen; denn wenn das Gesetz nur die Maxime der Handlungen, nicht die Handlungen selbst, gebieten kann, so ist's ein Zeichen, daß es der Befolgung (Observanz) einen Spielraum (latitudo) für die freie Willkür überlasse, d.i. nicht bestimmt angeben könne, wie und wie viel durch die Handlung zu dem Zweck, der zugleich Pflicht ist, gewirkt werden solle. – Es wird aber unter einer weiten Pflicht nicht eine Erlaubnis zu Ausnahmen von der Maxime der Handlungen, sondern nur die der Einschränkung einer Pflichtmaxime durch die andere (z.B. die allgemeine Nächstenliebe durch die Elternliebe) verstanden, wodurch in der Tat das Feld für die Tugendpraxis erweitert wird. – Je weiter die Pflicht, je unvollkommener also die Verbindlichkeit des Menschen zur Handlung ist, je näher er gleichwohl die Maxime der Observanz derselben (in seiner Gesinnung) der engen Pflicht (des Rechts) bringt, desto vollkommener ist seine Tugendhandlung. 
	
	\unnumberedsection{Schatzung (2)} 
	\subsection*{tg472.2.47} 
	\textbf{Source : }Die Metaphysik der Sitten/Zweiter Teil. Metaphysische Anfangsgründe der Tugendlehre/I. Ethische Elementarlehre/I. Teil. Von den Pflichten gegen sich selbst überhaupt/Erstes Buch. Von den vollkommenen Pflichten gegen sich selbst\\  
	
	\noindent\textbf{Paragraphe : }Ist nicht in dem Menschen das Gefühl der Erhabenheit seiner Bestimmung, d.i. die Gemütserhebung (elatio animi), als \match{Schätzung} seiner selbst, mit dem Eigendünkel (arrogantia), welcher der wahren Demut (humilitas moralis) gerade entgegengesetzt ist, zu nahe verwandt, als daß zu jener aufzumuntern es ratsam wäre; selbst in Vergleichung mit anderen Menschen, nicht bloß mit dem Gesetz? oder würde diese Art von Selbstverleugnung nicht vielmehr den Ausspruch anderer bis zur Geringschätzung unserer Person steigern, und so der Pflicht (der Achtung) gegen uns selbst zuwider sein? Das Bücken und Schmiegen vor einem  Menschen scheint in jedem Fall eines Menschen unwürdig zu sein. 
	
	\subsection*{tg477.2.10} 
	\textbf{Source : }Die Metaphysik der Sitten/Zweiter Teil. Metaphysische Anfangsgründe der Tugendlehre/I. Ethische Elementarlehre/I. Teil. Von den Pflichten gegen sich selbst überhaupt/2. Buch: Die Pflichten gegen sich selbst/Erster Abschnitt. Von der Pflicht gegen sich selbst in Entwickelung und Vermehrung seiner Naturvollkommenheit, d.i. in pragmatischer Absicht\\  
	
	\noindent\textbf{Paragraphe : }Auf welche von diesen physischen Vollkommenheiten vorzüglich, und in welcher Proportion, in Vergleichung gegen einander, sie sich zum Zweck zu machen es Pflicht des Menschen gegen sich selbst sei, bleibt ihrer eigenen vernünftigen Überlegung, in Ansehung der Lust zu einer gewissen Lebensart und zugleich der \match{Schätzung} seiner dazu erforderlichen Kräfte, überlassen, um darunter zu wählen (z.B. ob es ein Handwerk, oder der Kaufhandel, oder die  Gelehrsamkeit sein sollte). Denn, abgesehen von dem Bedürfnis der Selbsterhaltung, welches an sich keine Pflicht begründen kann, ist es Pflicht des Menschen gegen sich selbst, ein der Welt nützliches Glied zu sein, weil dieses auch zum Wert der Menschheit in seiner eigenen Person gehört, die er also nicht abwürdigen soll. 
	
	\unnumberedsection{Scheffel (1)} 
	\subsection*{tg437.2.77} 
	\textbf{Source : }Die Metaphysik der Sitten/Erster Teil. Metaphysische Anfangsgründe der Rechtslehre/1. Teil. Das Privatrecht vom äußeren Mein und Dein überhaupt/2. Hauptstück. Von der Art, etwas Äußeres zu erwerben/3. Abschnitt. Von dem auf dingliche Art persönlichen Recht\\  
	
	\noindent\textbf{Paragraphe : }Ein \match{Scheffel} Getreide hat den größten direkten Wert als Mittel zu menschlichen Bedürfnissen. Man kann damit Tiere füttern, die uns zur Nahrung, zur Bewegung und zur Arbeit an unserer statt, und dann auch vermittelst desselben also Menschen vermehren und erhalten, welche nicht allein jene Naturprodukte immer wieder erzeugen, sondern auch durch Kunstprodukte allen unseren Bedürfnissen zu Hülfe kommen können; zur Verfertigung unserer Wohnung, Kleidung, ausgesuchtem Genusse und aller Gemächlichkeit überhaupt, welche die Güter der Industrie ausmachen. Der Wert des Geldes ist dagegen nur indirekt. Man kann es selbst nicht genießen, oder als ein solches irgend wozu unmittelbar gebrauchen; aber doch ist es ein Mittel, was unter allen Sachen von der höchsten Brauchbarkeit ist. 
	
	\unnumberedsection{Seite (1)} 
	\subsection*{tg471.2.11} 
	\textbf{Source : }Die Metaphysik der Sitten/Zweiter Teil. Metaphysische Anfangsgründe der Tugendlehre/I. Ethische Elementarlehre/I. Teil. Von den Pflichten gegen sich selbst überhaupt/Erstes Buch. Von den vollkommenen Pflichten gegen sich selbst/Erstes Hauptstück. Die Pflicht des Menschen gegen sich selbst, als einem animalischen Wesen\\  
	
	\noindent\textbf{Paragraphe : }a) Die Selbstentleibung ist ein Verbrechen (Mord). Dieses kann nun zwar auch als Übertretung seiner Pflicht gegen andere Menschen (Eheleute, Eltern gegen Kinder, des Untertans gegen seine Obrigkeit, oder seine Mitbürger, endlich auch gegen Gott betrachtet werden, dessen uns anvertrauten Posten in der Welt der Mensch verläßt, ohne davon abgerufen zu sein) betrachtet werden; – aber hier ist nur die Rede von Verletzung einer Pflicht gegen sich selbst, ob nämlich, wenn ich auch alle jene Rücksichten bei \match{Seite} setzte, der Mensch doch zur Erhaltung seines Lebens, bloß durch seine Qualität als Person verbunden sei, und hierin eine (und zwar strenge) Pflicht gegen sich selbst anerkennen müsse. 
	
	\unnumberedsection{Seiten (1)} 
	\subsection*{tg441.2.63} 
	\textbf{Source : }Die Metaphysik der Sitten/Erster Teil. Metaphysische Anfangsgründe der Rechtslehre/2. Teil. Das öffentliche Recht/1. Abschnitt. Das Staatsrecht\\  
	
	\noindent\textbf{Paragraphe : }Was ein bürgerliches Amt anlangt, so kommt hier die Frage vor: hat der Souverän das Recht, einem, dem er ein Amt gegeben, es nach seinem Gutbefinden (ohne ein Verbrechen von \match{Seiten} des letzteren) wieder zu nehmen? Ich sage, nein! Denn, was der vereinigte Wille des Volks über seine bürgerliche Beamte nie beschließen wird, das kann auch das Staatsoberhaupt über ihn nicht beschließen. Nun will das Volk (das die Kosten tragen soll, welche die Ansetzung eines Beamten ihm machen wird) ohne allen Zweifel, daß dieser seinem ihm auferlegten Geschäfte völlig gewachsen sei; welches aber nicht anders, als durch eine hinlängliche Zeit hindurch fortgesetzte Vorbereitung und Erlernung desselben, über der er diejenige versäumt, die er zur Erlernung eines anderen ihn nährenden Geschäfts hätte verwenden können, geschehen kann; mithin würde, in der Regel, das Amt mit Leuten versehen werden, die keine dazu erforderliche Geschicklichkeit, und durch Übung erlangte reife Urteilskraft erworben hätten; welches der Absicht des Staats zuwider ist, als zu welcher auch erforderlich ist, daß jeder vom niedrigeren Amte zu höheren (die sonst lauter Untauglichen in die Hände fallen würden) steigen, mithin auch auf lebenswierige Versorgung müsse rechnen können. 
	
	\unnumberedsection{Starke (9)} 
	\subsection*{tg450.2.10} 
	\textbf{Source : }Die Metaphysik der Sitten/Zweiter Teil. Metaphysische Anfangsgründe der Tugendlehre/Einleitung/II. Erörterung des Begriffs von einem Zwecke, der zugleich Pflicht ist\\  
	
	\noindent\textbf{Paragraphe : }Der Tugend = + a ist die negative Untugend (moralische Schwäche) = 0 als logisches Gegenteil (contradictorie oppositum), das Laster aber = – a als Widerspiel (contrarie s. realiter oppositum) entgegen gesetzt und es ist eine, nicht bloß unnötige, sondern auch anstößige Frage: ob zu großen Verbrechen nicht etwa mehr \match{Stärke} der Seele als selbst zu großen Tugenden gehöre. Denn unter Stärke der Seele verstehen wir die Stärke des Vorsatzes eines Menschen, als mit Freiheit begabten Wesens, mithin so fern er seiner selbst mächtig (bei Sinnen) ist, also im gesunden Zustande des Menschen. Große Verbrechen aber sind Paroxysmen, deren Anblick den an Seele gesunden Menschen schaudern macht. Die Frage würde also etwa dahin auslaufen: ob ein Mensch im Anfall einer Krankheit mehr physische Stärke haben könne, als wenn er bei Sinnen ist; welches  man einräumen kann, ohne ihm darum mehr Seelenstärke beizulegen, wenn man unter Seele das Lebensprinzip des Menschen im freien Gebrauch seiner Kräfte versteht. Denn, weil jene bloß in der Macht der die Vernunft schwächenden Neigungen ihren Grund haben, welches keine Seelenstärke beweiset, so würde diese Frage mit der ziemlich auf einerlei hinauslaufen: ob ein Mensch im Anfall einer Krankheit mehr Stärke als im gesunden Zustande beweisen könne, welche geradezu verneinend beantwortet werden kann, weil der Mangel der Gesundheit, die im Gleichgewicht aller körperlichen Kräfte des Menschen besteht, eine Schwächung im System dieser Kräfte ist, nach welchem man allein die absolute Gesundheit beurteilen kann. 
	
	\subsection*{tg457.2.2} 
	\textbf{Source : }Die Metaphysik der Sitten/Zweiter Teil. Metaphysische Anfangsgründe der Tugendlehre/Einleitung/IX. Was ist Tugendpflicht\\  
	
	\noindent\textbf{Paragraphe : }
	Tugend ist die \match{Stärke} der Maxime des Menschen in Befolgung seiner Pflicht. – Alle Stärke wird nur durch Hindernisse erkannt, die sie überwältigen kann; bei der Tugend aber sind diese die Naturneigungen, welche mit dem sittlichen Vorsatz in Streit kommen können, und, da der Mensch es selbst ist, der seinen Maximen diese Hindernisse in den Weg legt, so ist die Tugend nicht bloß ein Selbstzwang (denn da könnte eine Naturneigung die andere zu bezwingen trachten), sondern auch ein Zwang nach einem Prinzip der innern Freiheit, mithin durch die bloße Vorstellung seiner Pflicht, nach dem formalen Gesetz derselben. 
	
	\subsection*{tg458.2.5} 
	\textbf{Source : }Die Metaphysik der Sitten/Zweiter Teil. Metaphysische Anfangsgründe der Tugendlehre/Einleitung/X. Das oberste Prinzip der Rechtslehre war analytisch; das der Tugendlehre ist synthetisch\\  
	
	\noindent\textbf{Paragraphe : }Man kann auch gar wohl sagen: der Mensch sei zur Tugend (als einer moralischen Stärke) verbunden. Denn obgleich das Vermögen (facultas) der Überwindung aller sinnlich entgegenwirkenden Antriebe, seiner Freiheit halber, schlechthin vorausgesetzt werden kann und muß: so ist doch dieses Vermögen als \match{Stärke} (robur) etwas, was erworben werden muß, dadurch, daß die moralische Triebfeder (die Vorstellung des Gesetzes) durch Betrachtung (contemplatione) der Würde des reinen Vernunftgesetzes in uns, zugleich aber auch durch Übung (exercitio) erhoben wird. 
	
	\subsection*{tg461.2.12} 
	\textbf{Source : }Die Metaphysik der Sitten/Zweiter Teil. Metaphysische Anfangsgründe der Tugendlehre/Einleitung/XIII. Allgemeine Grundsätze der Metaphysik der Sitten in Behandlung einer reinen Tugendlehre\\  
	
	\noindent\textbf{Paragraphe : }Tugend bedeutet eine moralische \match{Stärke} des Willens. Aber dies erschöpft noch nicht den Begriff; denn eine solche Stärke könnte auch einem heiligen (übermenschlichen) Wesen zukommen, in welchem kein hindernder Antrieb dem Gesetze seines Willens entgegen wirkt; das also alles dem Gesetz gemäß gerne tut. Tugend ist also die moralische Stärke des Willens eines Menschen in Befolgung seiner Pflicht: welche eine moralische Nötigung durch seine eigene gesetzgebende Vernunft ist, insofern diese sich zu einer das Gesetz ausführenden Gewalt selbst konstituiert. – Sie ist nicht selbst, oder sie zu besitzen ist nicht Pflicht (denn sonst würde es eine Verpflichtung zur Pflicht geben müssen), sondern sie gebietet und begleitet ihr Gebot durch einen sittlichen (nach Gesetzen der inneren Freiheit möglichen) Zwang; wozu aber, weil er unwiderstehlich sein soll, Stärke erforderlich ist, deren Grad wir nur durch die Große der Hindernisse, die der Mensch durch seine Neigungen sich selber schafft, schätzen können. Die Laster, als die Brut gesetzwidriger Gesinnungen, sind die Ungeheuer, die er nun zu bekämpfen hat: weshalb diese sittliche Stärke auch, als Tapferkeit (fortitudo moralis), die größte und einzige wahre Kriegsehre des Menschen ausmacht; auch wird sie die eigentliche, nämlich praktische Weisheit genannt: weil sie den Endzweck des Daseins der Menschen auf Erden zu dem ihrigen macht. – In ihrem Besitz ist der Mensch allein frei, gesund, reich, ein König u.s.w. und kann, weder durch Zufall, noch Schicksal einbüßen; weil er sich selbst besitzt und der Tugendhafte seine Tugend nicht verlieren kann. 
	
	\subsection*{tg463.2.2} 
	\textbf{Source : }Die Metaphysik der Sitten/Zweiter Teil. Metaphysische Anfangsgründe der Tugendlehre/Einleitung/XV. Zur Tugend wird zuerst erfordert die Herrschaft über sich selbst\\  
	
	\noindent\textbf{Paragraphe : }
	Affekten und Leidenschaften sind wesentlich von einander unterschieden; die erstern gehören zum Gefühl, so fern es, vor der Überlegung vorhergehend, diese selbst unmöglich oder schwerer macht. Daher heißt der Affekt jäh, oder jach (animus praeceps) und die Vernunft sagt durch den Tugendbegriff, man solle sich fassen; doch ist diese Schwäche  im Gebrauch seines Verstandes, verbunden mit der \match{Stärke} der Gemütsbewegung, nur eine Untugend und gleichsam etwas Kindisches und Schwaches, was mit dem besten Willen gar wohl zusammen bestehen kann, und das einzige Gute noch an sich hat, daß dieser Sturm bald aufhört. Ein Hang zum Affekt (z.B. Zorn) verschwistert sich daher nicht so sehr mit dem Laster, als die Leidenschaft. Leidenschaft dagegen ist die zur bleibenden Neigung gewordene sinnliche Begierde (z.B. der Haß im Gegensatz des Zorns). Die Ruhe, mit der ihr nachgehangen wird, läßt Überlegung zu und verstattet dem Gemüt, sich darüber Grundsätze zu machen und so, wenn die Neigung auf das Gesetzwidrige fällt, über sie zu brüten, sie tief zu wurzeln und das Böse dadurch (als vorsätzlich) in seine Maxime aufzunehmen; welches alsdann ein qualifiziertes Böse, d.i. ein wahres Laster ist. 
	
	\subsection*{tg464.2.2} 
	\textbf{Source : }Die Metaphysik der Sitten/Zweiter Teil. Metaphysische Anfangsgründe der Tugendlehre/Einleitung/XVI. Zur Tugend wird Apathie (als Stärke betrachtet) notwendig vorausgesetzt\\  
	
	\noindent\textbf{Paragraphe : }Dieses Wort ist, gleich als ob es Fühllosigkeit, mithin subjektive Gleichgültigkeit in Ansehung der Gegenstände der Willkür, bedeutete, in übelen Ruf gekommen; man nahm es für Schwäche. Dieser Mißdeutung kann dadurch vorgebeugt werden, daß man diejenige Affektlosigkeit, welche von der Indifferenz zu unterscheiden ist, die moralische Apathie nennt: da die Gefühle aus sinnlichen Eindrücken ihren Einfluß auf das moralische nur dadurch verlieren, daß die Achtung fürs Gesetz über sie insgesamt mächtiger wird. – Es ist nur die scheinbare \match{Stärke} eines Fieberkranken, die den lebhaften Anteil selbst am Guten bis zum Affekt steigen, oder vielmehr darin ausarten läßt. Man nennt den Affekt dieser Art Enthusiasm, und dahin ist auch die Mäßigung zu deuten, die man selbst für Tugendausübungen zu empfehlen pflegt (insani sapiens nomen habeat aequus iniqui – ultra, quam satis est virtutem si petat ipsam. Horat.). Denn sonst ist es ungereimt zu wähnen, man könne auch wohl allzuweise, allzutugendhaft sein. Der Affekt gehört immer zur Sinnlichkeit; er mag durch einen Gegenstand erregt werden, welcher es wolle. Die wahre Stärke der Tugend ist das Gemüt in Ruhe, mit einer überlegten und festen Entschließung, ihr Gesetz in Ausübung zu bringen. Das ist der Zustand der Gesundheit im moralischen Leben; dagegen der Affekt, selbst wenn er durch die Vorstellung des Guten aufgeregt wird, eine augenblicklich glänzende Erscheinung ist, welche Mattigkeit hinterläßt. – Phantastisch-tugendhaft aber kann doch der genannt werden, der keine in Ansehung der Moralität gleichgültige Dinge (adiaphora) einräumt und sich alle seine Schritte und Tritte mit Pflichten als mit Fußangeln bestreut und es nicht gleichgültig findet, ob ich mich mit Fleisch oder Fisch, mit Bier oder Wein, wenn mir beides bekömmt, nähre; eine Mikrologie, welche, wenn man sie in die Lehre der Tugend aufnähme, die Herrschaft derselben zur Tyrannei machen würde. 
	
	\subsection*{tg478.2.10} 
	\textbf{Source : }Die Metaphysik der Sitten/Zweiter Teil. Metaphysische Anfangsgründe der Tugendlehre/I. Ethische Elementarlehre/I. Teil. Von den Pflichten gegen sich selbst überhaupt/2. Buch: Die Pflichten gegen sich selbst/Zweiter Abschnitt. Von der Pflicht gegen sich selbst in Erhöhung seiner moralischen Vollkommenheit, d.i. in bloß sittlicher Absicht\\  
	
	\noindent\textbf{Paragraphe : }Die Tiefen des menschlichen Herzens sind unergründlich. Wer kennt sich gnugsam, wenn die Triebfeder zur Pflichtbeobachtung von ihm gefühlt wird, ob sie gänzlich aus der Vorstellung des Gesetzes hervorgehe, oder ob nicht manche andere, sinnliche Antriebe mitwirken, die auf den Vorteil (oder zur Verhütung eines Nachteils) angelegt sind und bei anderer Gelegenheit auch wohl dem Laster zu Diensten stehen könnten. – Was aber die Vollkommenheit als moralischen Zweck betrifft, so gibt's zwar in der Idee (objektiv) nur eine Tugend (als sittliche \match{Stärke} der Maximen), in der Tat (subjektiv) aber eine Menge derselben von heterogener Beschaffenheit, worunter es unmöglich sein dürfte, nicht irgend eine Untugend (ob sie gleich eben jener wegen den Namen des Lasters nicht zu führen pflegen) aufzufinden, wenn man sie suchen wollte. Eine Summe von Tugenden aber, deren Vollständigkeit oder Mängel das Selbsterkenntnis uns nie hinreichend einschauen läßt, kann keine andere als unvollkommene Pflicht vollkommen zu sein begründen. 
	
	\subsection*{tg482.2.52} 
	\textbf{Source : }Die Metaphysik der Sitten/Zweiter Teil. Metaphysische Anfangsgründe der Tugendlehre/I. Ethische Elementarlehre/II. Teil. Von den Tugendpflichten gegen andere/Erstes Hauptstück. Von den Pflichten gegen andere, bloß als Menschen/Zweiter Abschnitt. Von den Tugendpflichten gegen andere Menschen aus der ihnen gebührenden Achtung\\  
	
	\noindent\textbf{Paragraphe : }Die verschiedene andern zu beweisende Achtung nach Verschiedenheit der Beschaffenheit der Menschen, oder ihrer zufälligen Verhältnisse, nämlich der des Alters, des Geschlechts, der Abstammung, der \match{Stärke} oder Schwäche, oder gar des Standes und der Würde, welche zum Teil auf beliebigen Anordnungen beruhen, darf in metaphysischen Anfangsgründen der Tugendlehre nicht ausführlich dargestellt und klassifiziert werden, da es hier nur um die reinen Vernunftprinzipien derselben zu tun ist. 
	
	\subsection*{tg486.2.6} 
	\textbf{Source : }Die Metaphysik der Sitten/Zweiter Teil. Metaphysische Anfangsgründe der Tugendlehre/II. Ethische Methodenlehre/1. Abschnitt. Die ethische Didaktik\\  
	
	\noindent\textbf{Paragraphe : }Daß Tugend erworben werden müsse (nicht angeboren sei), liegt, ohne sich deshalb auf anthropologische Kenntnisse aus der Erfahrung berufen zu dürfen, schon in dem Begriffe derselben. Denn das sittliche Vermögen des Menschen wäre nicht Tugend, wenn es nicht durch die \match{Stärke} des Vorsatzes, in dem Streit mit so mächtigen entgegenstehenden Neigungen, hervorgebracht wäre. Sie ist das Produkt aus der reinen praktischen Vernunft, so fern diese im Bewußtsein ihrer Überlegenheit (aus Freiheit) über jene die Obermacht gewinnt. 
	
	\unnumberedsection{Substanz (8)} 
	\subsection*{tg435.2.26} 
	\textbf{Source : }Die Metaphysik der Sitten/Erster Teil. Metaphysische Anfangsgründe der Rechtslehre/1. Teil. Das Privatrecht vom äußeren Mein und Dein überhaupt/2. Hauptstück. Von der Art, etwas Äußeres zu erwerben/1. Abschnitt. Vom Sachrecht\\  
	
	\noindent\textbf{Paragraphe : }Es ist die Frage: wie weit erstreckt sich die Befugnis der Besitznehmung eines Bodens? So weit, als das Vermögen, ihn in seiner Gewalt zu haben, d.i. als der, so ihn sich zueignen will, ihn verteidigen kann, gleich als ob der Boden spräche: wenn ihr mich nicht beschützen könnt, so könnt ihr mir auch nicht gebieten. Darnach müßte also auch der Streit über das freie oder verschlossene Meer entschieden werden; z.B. innerhalb der Weite, wohin die Kanonen reichen, darf niemand an der Küste eines Landes, das schon einem gewissen  Staat zugehört, fischen, Bernstein aus dem Grunde der See holen, u. dergl. – Ferner: ist die Bearbeitung des Bodens (Bebauung, Beackerung, Entwässerung u. dergl.) zur Erwerbung desselben notwendig? Nein! denn, da diese Formen (der Spezifizierung) nur Akzidenzen sind, so machen sie kein Objekt eines unmittelbaren Besitzes aus, und können zu dem des Subjekts nur gehören, so fern die \match{Substanz} vorher als das Seine desselben anerkannt ist. Die Bearbeitung ist, wenn es auf die Frage von der ersten Erwerbung ankommt, nichts weiter als ein äußeres Zeichen der Besitznehmung, welches man durch viele andere, die weniger Mühe kosten, ersetzen kann. – Ferner: darf man wohl jemanden in dem Akt seiner Besitznehmung hindern, so daß keiner von beiden des Rechts der Priorität teilhaftig werde, und so der Boden immer als keinem angehörig frei bleibe? Gänzlich kann diese Hinderung nicht statt finden, weil der andere, um dieses tun zu können, sich doch auch selbst auf irgend einem benachbarten Boden befinden muß, wo er also selbst behindert werden kann zu sein, mithin eine absolute Verhinderung ein Widerspruch wäre; aber respektiv auf einen gewissen (zwischenliegenden) Boden, diesen, als neutral, zur Scheidung zweier Benachbarten unbenutzt liegen zu lassen, würde doch mit dem Rechte der Bemächtigung zusammen bestehen; aber alsdann gehört wirklich dieser Boden beiden gemeinschaftlich, und ist nicht herrenlos (res nullius), eben darum, weil er von beiden dazu gebraucht wird, um sie von einander zu scheiden. – Ferner: kann man auf einem Boden, davon kein Teil das Seine von jemanden ist, doch eine Sache als die seine haben? Ja, wie in der Mongolei jeder sein Gepäcke, was er hat, liegen lassen, oder sein Pferd, was ihm entlaufen ist, als das Seine in seinen Besitz bringen kann, weil der ganze Boden dem Volk, der Gebrauch desselben also jedem einzelnen zusteht; daß aber jemand eine bewegliche Sache auf dem Boden eines anderen als das Seine haben kann, ist zwar möglich, aber nur durch Vertrag. – Endlich ist die Frage: können zwei benachbarte  Völker (oder Familien) einander widerstehen, eine gewisse Art des Gebrauchs eines Bodens anzunehmen, z.B. die Jagdvölker dem Hirtenvolk, oder den Ackerleuten, oder diese den Pflanzern, u. dergl.? Allerdings; denn die Art, wie sie sich auf dem Erdboden überhaupt ansässig machen wollen, ist, wenn sie sich innerhalb ihrer Grenzen halten, eine Sache des bloßen Beliebens (res merae facultatis). 
	
	\subsection*{tg435.2.36} 
	\textbf{Source : }Die Metaphysik der Sitten/Erster Teil. Metaphysische Anfangsgründe der Rechtslehre/1. Teil. Das Privatrecht vom äußeren Mein und Dein überhaupt/2. Hauptstück. Von der Art, etwas Äußeres zu erwerben/1. Abschnitt. Vom Sachrecht\\  
	
	\noindent\textbf{Paragraphe : }Der Rechtsbegriff vom äußeren Mein und Dein, so fern es \match{Substanz} ist, kann, was das Wort außer mir betrifft, nicht einen anderen Ort, als wo ich bin, bedeuten: denn er ist ein Vernunftbegriff; sondern, da unter diesem nur ein reiner Verstandesbegriff subsumiert werden kann, bloß etwas von mir Unterschiedenes und den eines nicht empirischen Besitzes (der gleichsam fortdauernden Apprehension), sondern nur den des in meiner Gewalt-habens (die Verknüpfung desselben mit mir als subjektive Bedingung der Möglichkeit des Gebrauchs) des äußeren Gegenstandes, welcher ein reiner Verstandesbegriff ist, bedeuten. Nun ist die Weglassung, oder das Absehen (Abstraktion) von diesen sinnlichen Bedingungen des Besitzes, als eines Verhältnisses der Person zu Gegenständen, die keine Verbindlichkeit haben, nichts anders als das Verhältnis einer Person zu Personen, diese alle durch den Willen der ersteren, so fern er dem Axiom der äußeren Freiheit, dem Postulat des Vermögens und der allgemeinen Gesetzgebung
	des a priori als vereinigt gedachten Willens gemäß ist, in Ansehung des Gebrauchs der Sachen zu verbinden, welches also der intelligibele Besitz derselben, d.i. der durchs bloße Recht, ist, obgleich der Gegenstand (die Sache, die ich besitze) ein Sinnenobjekt ist. 
	
	\subsection*{tg435.2.37} 
	\textbf{Source : }Die Metaphysik der Sitten/Erster Teil. Metaphysische Anfangsgründe der Rechtslehre/1. Teil. Das Privatrecht vom äußeren Mein und Dein überhaupt/2. Hauptstück. Von der Art, etwas Äußeres zu erwerben/1. Abschnitt. Vom Sachrecht\\  
	
	\noindent\textbf{Paragraphe : }Daß die erste Bearbeitung, Begrenzung, oder überhaupt Formgebung eines Bodens keinen Titel der Erwerbung desselben, d.i. der Besitz des Akzidens nicht ein Grund des rechtlichen Besitzes der \match{Substanz} abgeben könne, sondern vielmehr umgekehrt das Mein und Dein nach der Regel (accessorium sequitur suum principale) aus dem Eigentum der Substanz gefolgert werden müsse, und daß der, welcher an einen Boden, der nicht schon vorher der seine war, Fleiß verwendet, seine Mühe und Arbeit gegen den ersteren verloren hat, ist für sich selbst so klar, daß man jene so alte und noch weit und breit herrschende Meinung schwerlich einer anderen Ursache zuschreiben kann, als der in geheim obwaltenden Täuschung, Sachen zu personifizieren und, gleich als ob jemand sie sich durch an sie verwandte Arbeit verbindlich machen könne, keinem anderen als ihm zu Diensten zu stehen, unmittelbar gegen sie sich ein Recht zu denken; denn wahrscheinlicherweise würde man auch nicht so leichten Fußes über die natürliche Frage (von der oben schon Erwähnung geschehen) weggeglitten sein: »wie ist ein Recht in einer Sache möglich?« Denn das Recht gegen einen jeden Besitzer einer Sache bedeutet nur die Befugnis der besonderen Willkür zum Gebrauch eines, Objekts, so fern sie als im synthetisch-allgemeinen Willen enthalten, und mit dem Gesetz desselben zusammenstimmend gedacht werden kann. 
	
	\subsection*{tg435.2.43} 
	\textbf{Source : }Die Metaphysik der Sitten/Erster Teil. Metaphysische Anfangsgründe der Rechtslehre/1. Teil. Das Privatrecht vom äußeren Mein und Dein überhaupt/2. Hauptstück. Von der Art, etwas Äußeres zu erwerben/1. Abschnitt. Vom Sachrecht\\  
	
	\noindent\textbf{Paragraphe : }Der äußere Gegenstand, welcher der \match{Substanz} nach das Seine von jemanden ist, ist dessen Eigentum (dominium), welchem alle Rechte in dieser Sache (wie Akzidenzen der Substanz) inhärieren, über welche also der Eigentümer (dominus) nach Belieben verfügen kann (ius disponendi de re sua). Aber hieraus folgt von selbst:  daß ein solcher Gegenstand nur eine körperliche Sache (gegen die man keine Verbindlichkeit hat) sein könne, daher ein Mensch sein eigener Herr (sui iuris), aber nicht Eigentümer von sich selbst (sui dominus) (über sich nach Belieben disponieren zu können) geschweige denn von anderen Menschen sein kann, weil er der Menschheit in seiner eigenen Person verantwortlich ist; wiewohl dieser Punkt, der zum Recht der Menschheit, nicht dem der Menschen gehört, hier nicht seinen eigentlichen Platz hat, sondern nur beiläufig zum besseren Verständnis des kurz vorher Gesagten angeführt wird. – Es kann ferner zwei volle Eigentümer einer und derselben Sache geben, ohne ein gemeinsames Mein und Dein, sondern nur als gemeinsame Besitzer dessen, was nur einem als das Seine zugehört, wenn, von den sogenannten Miteigentümern (condomini), einem nur der ganze Besitz ohne Gebrauch, dem anderen aber aller Gebrauch der Sache samt dem Besitz zukommt, jener also (dominus directus) diesen (dominus utilis) nur auf die Bedingung einer beharrlichen Leistung restringiert, ohne dabei seinen Gebrauch zu limitieren. 
	
	\subsection*{tg435.2.8} 
	\textbf{Source : }Die Metaphysik der Sitten/Erster Teil. Metaphysische Anfangsgründe der Rechtslehre/1. Teil. Das Privatrecht vom äußeren Mein und Dein überhaupt/2. Hauptstück. Von der Art, etwas Äußeres zu erwerben/1. Abschnitt. Vom Sachrecht\\  
	
	\noindent\textbf{Paragraphe : }Der Boden (unter welchem alles bewohnbare Land verstanden wird) ist, in Ansehung alles Beweglichen auf demselben, als Substanz, die Existenz des letzteren aber nur als Inhärenz zu betrachten und so, wie Im theoretischen Sinne die Akzidenzen nicht außerhalb der \match{Substanz} existieren können, so kann im praktischen das Bewegliche auf dem Boden nicht das Seine von jemanden sein, wenn dieser nicht vorher als im rechtlichen Besitz desselben befindlich (als das Seine desselben) angenommen wird. 
	
	\subsection*{tg442.2.13} 
	\textbf{Source : }Die Metaphysik der Sitten/Erster Teil. Metaphysische Anfangsgründe der Rechtslehre/2. Teil. Das öffentliche Recht/2. Abschnitt. Das Völkerrecht\\  
	
	\noindent\textbf{Paragraphe : }Dieses Recht scheint sich leicht dartun zu lassen; nämlich aus dem Rechte, mit dem Seinen (Eigentum) zu tun, was man will. Was jemand aber der \match{Substanz} nach selbst gemacht hat, davon hat er ein unbestrittenes Eigentum. – Hier ist also die Deduktion, so wie sie ein bloßer Jurist abfassen würde. 
	
	\subsection*{tg474.2.4} 
	\textbf{Source : }Die Metaphysik der Sitten/Zweiter Teil. Metaphysische Anfangsgründe der Tugendlehre/I. Ethische Elementarlehre/I. Teil. Von den Pflichten gegen sich selbst überhaupt/Erstes Buch. Von den vollkommenen Pflichten gegen sich selbst/Zweites Hauptstück. Die Pflicht des Menschen gegen sich selbst, bloß als einem moralischen Wesen/2. Abschnitt. Von dem ersten Gebot aller Pflichten gegen sich selbst\\  
	
	\noindent\textbf{Paragraphe : }Dieses ist: Erkenne (erforsche, ergründe) dich selbst nicht nach deiner physischen Vollkommenheit (der Tauglichkeit oder Untauglichkeit zu allerlei dir beliebigen oder auch gebotenen Zwecke), sondern nach der moralischen, in Beziehung auf deine Pflicht – dein Herz – ob es gut oder böse sei, ob die Quelle deiner Handlungen lauter oder unlauter, und was, entweder als ursprünglich zur \match{Substanz} des Menschen gehörend, oder, als abgeleitet (erworben oder zugezogen) ihm selbst zugerechnet werden kann und zum moralischen Zustande gehören mag. 
	
	\subsection*{tg488.2.15} 
	\textbf{Source : }Die Metaphysik der Sitten/Zweiter Teil. Metaphysische Anfangsgründe der Tugendlehre/Beschluß. Die Religionslehre als Lehre der Pflichten gegen Gott liegt außerhalb den Grenzen der reinen Moralphilosophie\\  
	
	\noindent\textbf{Paragraphe : }Die Idee einer göttlichen Strafgerechtigkeit wird hier personifiziert; es ist nicht ein besonderes richtendes Wesen, was sie ausübt (denn da würden Widersprüche desselben mit Rechtsprinzipien vorkommen), sondern die Gerechtigkeit, gleich als \match{Substanz} (sonst die ewige Gerechtigkeit genannt), die, wie das Fatum (Verhängnis) der alten philosophierenden Dichter, noch über dem Jupiter ist, spricht das Recht nach der eisernen, unablenkbaren Notwendigkeit aus, die für uns weiter unerforschlich ist. – Hievon jetzt einige Beispiele. 
	
	\unnumberedsection{Summe (2)} 
	\subsection*{tg437.2.78} 
	\textbf{Source : }Die Metaphysik der Sitten/Erster Teil. Metaphysische Anfangsgründe der Rechtslehre/1. Teil. Das Privatrecht vom äußeren Mein und Dein überhaupt/2. Hauptstück. Von der Art, etwas Äußeres zu erwerben/3. Abschnitt. Von dem auf dingliche Art persönlichen Recht\\  
	
	\noindent\textbf{Paragraphe : }Hierauf läßt sich vorläufig eine Realdefinition des Geldes gründen: es ist das allgemeine Mittel, den Fleiß der Menschen gegen einander zu verkehren, so: daß der Nationalreichtum, in sofern er vermittelst des Geldes erworben worden, eigentlich nur die \match{Summe} des Fleißes ist, mit dem Menschen sich untereinander lohnen, und welcher durch das in dem Volk umlaufende Geld repräsentiert wird. 
	
	\subsection*{tg437.2.85} 
	\textbf{Source : }Die Metaphysik der Sitten/Erster Teil. Metaphysische Anfangsgründe der Rechtslehre/1. Teil. Das Privatrecht vom äußeren Mein und Dein überhaupt/2. Hauptstück. Von der Art, etwas Äußeres zu erwerben/3. Abschnitt. Von dem auf dingliche Art persönlichen Recht\\  
	
	\noindent\textbf{Paragraphe : }Ein Buch ist eine Schrift (ob mit der Feder oder durch Typen, auf wenig oder viel Blättern verzeichnet, ist hier gleichgültig), welche eine Rede vorstellt, die jemand durch sichtbare Sprachzeichen an das Publikum hält. – Der, welcher zu diesem in seinem eigenen Namen spricht, heißt der Schriftsteller (autor). Der, welcher durch eine Schrift im Namen eines anderen (des Autors) öffentlich redet, ist der Verleger. Dieser, wenn er es mit jenes seiner Erlaubnis tut, ist der rechtmäßige, tut er es aber ohne dieselbe, der unrechtmäßige Verleger, d.i. der Nachdrucker. Die \match{Summe} aller Kopeien der Urschrift (Exemplare) ist der Verlag. 
	
	\unnumberedsection{Unterschied (8)} 
	\subsection*{tg430.2.57} 
	\textbf{Source : }Die Metaphysik der Sitten/Erster Teil. Metaphysische Anfangsgründe der Rechtslehre/Einleitung in die Metaphysik der Sitten\\  
	
	\noindent\textbf{Paragraphe : }Dagegen: je kleiner das Naturhindernis, je größer das Hindernis aus Gründen der Pflicht, desto mehr wird die Übertretung (als Verschuldung) zugerechnet. – Daher der Gemütszustand, ob das Subjekt die Tat im Affekt, oder mit ruhiger Überlegung verübt habe, in der Zurechnung einen \match{Unterschied} macht, der Folgen hat. 
	
	\subsection*{tg433.2.45} 
	\textbf{Source : }Die Metaphysik der Sitten/Erster Teil. Metaphysische Anfangsgründe der Rechtslehre/1. Teil. Das Privatrecht vom äußeren Mein und Dein überhaupt/1. Hauptstück\\  
	
	\noindent\textbf{Paragraphe : }Zur Kritik der rechtlich-praktischen Vernunft, im Begriffe des äußeren Mein und Dein, wird diese eigentlich durch eine Antinomie der Sätze über die Möglichkeit eines solchen Besitzes genötigt, d.i. nur durch eine unvermeidliche Dialektik, in welcher Thesis und Antithesis beide auf die Gültigkeit zweier einander widerstreitenden Bedingungen gleichen Anspruch machen, wird die Vernunft auch in ihrem praktischen (das Recht betreffenden) Gebrauch genötigt, zwischen dem Besitz als Erscheinung und dem bloß durch den Verstand denkbaren einen \match{Unterschied} zu machen. 
	
	\subsection*{tg439.2.20} 
	\textbf{Source : }Die Metaphysik der Sitten/Erster Teil. Metaphysische Anfangsgründe der Rechtslehre/1. Teil. Das Privatrecht vom äußeren Mein und Dein überhaupt/3. Hauptstück. Von der subjektiv-bedingten Erwerbung durch den Ausspruch einer öffentlichen Gerichtsbarkeit\\  
	
	\noindent\textbf{Paragraphe : }Da nun über das Mein und Dein aus dem Leihvertrage, wenn (wie es die Natur dieses Vertrages so mit sich bringt) über die mögliche Verunglückung (casus), die die Sache treffen möchte, nichts verabredet worden, er also, weil die  Einwilligung nur präsumiert worden, ein ungewisser Vertrag (pactum incertum) ist, das Urteil darüber, d.i. die Entscheidung, wen das Unglück treffen müsse, nicht aus den Bedingungen des Vertrages an sich selbst, sondern wie sie allein vor einem Gerichtshofe, der immer nur auf das Gewisse in jenem sieht (welches hier der Besitz der Sache als Eigentum ist), entschieden werden kann, so wird das Urteil im Naturzustande, d.i. nach der Sache innerer Beschaffenheit, so lauten: der Schade aus der Verunglückung einer geliehenen Sache fällt auf den Beliehenen (casum sentit commodatarius), dagegen im bürgerlichen, also vor einem Gerichtshofe, wird die Sentenz so ausfallen: der Schade fällt auf den Anleiher (casum sentit dominus), und zwar aus dem Grunde verschieden von dem Ausspruche der bloßen gesunden Vernunft, weil ein öffentlicher Richter sich nicht auf Präsumtionen von dem, was der eine oder andere Teil gedacht haben mag, einlassen kann, sondern der, welcher sich nicht die Freiheit von allem Schaden an der geliehenen Sache durch einen besonderen angehängten Vertrag ausbedungen hat, diesen selbst tragen muß. – Also ist der \match{Unterschied} zwischen dem Urteile, wie es ein Gericht fällen müßte, und dem, was die Privatvernunft eines jeden für sich zu fällen berechtigt ist, ein durchaus nicht zu übersehender Punkt in Berichtigung der Rechtsurteile. 
	
	\subsection*{tg447.2.12} 
	\textbf{Source : }Die Metaphysik der Sitten/Zweiter Teil. Metaphysische Anfangsgründe der Tugendlehre/Vorrede\\  
	
	\noindent\textbf{Paragraphe : }Ich habe an einem anderen Orte (der Berl. M. S.) den \match{Unterschied} der Lust, welche pathologisch ist, von der moralischen, wie ich glaube, auf die einfachsten Ausdrücke zurückgeführt. Die Lust nämlich, welche vor der Befolgung des Gesetzes hergehen muß, damit diesem gemäß gehandelt werde, ist pathologisch und das Verhalten folgt der Naturordnung; diejenige aber, vor welcher das Gesetz hergehen muß, damit sie empfunden werde, ist in der sittlichen Ordnung. – – Wenn dieser Unterschied nicht beobachtet wird: wenn Eudämonie (das Glückseligkeitsprinzip) statt der Eleutheronomie (des Freiheitsprinzips der inneren Gesetzgebung) zum Grundsatze aufgestellt wird, so ist die Folge davon Euthanasie (der sanfte Tod) aller Moral. 
	
	\subsection*{tg461.2.4} 
	\textbf{Source : }Die Metaphysik der Sitten/Zweiter Teil. Metaphysische Anfangsgründe der Tugendlehre/Einleitung/XIII. Allgemeine Grundsätze der Metaphysik der Sitten in Behandlung einer reinen Tugendlehre\\  
	
	\noindent\textbf{Paragraphe : }
	Zweitens. Der \match{Unterschied} der Tugend vom Laster kann nie in Graden der Befolgung gewisser Maximen, sondern muß allein in der spezifischen Qualität derselben (dem Verhältnis zum Gesetz) gesucht werden; mit andern Worten, der belobte Grundsatz (des Aristoteles), die Tugend in dem Mittleren zwischen zwei Lastern zu setzen, ist falsch.
	
	
	17
	Es sei z.B. gute Wirtschaft, als das Mittlere
	zwischen zwei Lastern, Verschwendung und Geiz, gegeben: so kann sie als Tugend nicht durch die allmähliche Verminderung des ersten beider genannten Laster (Ersparung), noch durch die Vermehrung der Ausgaben, des dem letzteren Ergebenen, als entspringend vorgestellt werden: indem sie sich gleichsam nach entgegengesetzten Richtungen in der guten Wirtschaft begegneten: sondern eine jede derselben hat ihre eigene Maxime, die der andern notwendig widerspricht. 
	
	\subsection*{tg472.2.48} 
	\textbf{Source : }Die Metaphysik der Sitten/Zweiter Teil. Metaphysische Anfangsgründe der Tugendlehre/I. Ethische Elementarlehre/I. Teil. Von den Pflichten gegen sich selbst überhaupt/Erstes Buch. Von den vollkommenen Pflichten gegen sich selbst\\  
	
	\noindent\textbf{Paragraphe : }Die vorzügliche Achtungsbezeigung in Worten und Manieren, selbst gegen einen nicht Gebietenden in der bürgerlichen Verfassung – die Reverenzen, Verbeugungen (Komplimente), höfische – den \match{Unterschied} der Stände mit sorgfältiger Pünktlichkeit bezeichnende Phrasen, – welche von der Höflichkeit (die auch sich gleich Achtenden notwendig ist) ganz unterschieden sind, – das Du, Er, Ihr und Sie, oder Ew. Wohledlen, Hochedeln, Hochedelgebornen, Wohlgebornen (ohe, iam satis est!) in der Anrede – als in welcher Pedanterei die Deutschen unter allen Völkern der Erde (die indischen Kasten vielleicht ausgenommen) es am weitesten gebracht haben, sind das nicht Beweise eines ausgebreiteten Hanges zur Kriecherei unter Menschen? (Hae nugae in seria ducunt.) Wer sich aber zum Wurm macht, kann nachher nicht klagen, daß er mit Füßen getreten wird. 
	
	\subsection*{tg489.2.15} 
	\textbf{Source : }Die Metaphysik der Sitten/Fußnoten\\  
	
	\noindent\textbf{Paragraphe : }
	
	7 Dieser \match{Unterschied} zwischen dem, was bloß formaliter, und dem, was auch materialiter unrecht ist, hat in der Rechtslehre mannigfaltigen Gebrauch. Der Feind, der, statt seine Kapitulation mit der Besatzung einer belagerten Festung ehrlich zu vollziehen, sie bei dieser ihrem Auszuge mißhandelt, oder sonst diesen Vertrag bricht, kann nicht über Unrecht klagen, wenn sein Gegner bei Gelegenheit ihm denselben Streich spielt. Aber sie tun überhaupt im höchsten Grade unrecht, weil sie dem Begriff des Rechts selber alle Gültigkeit nehmen, und alles der wilden Gewalt, gleichsam gesetzmäßig, überliefern, und so das Recht der Menschen überhaupt umstürzen. 
	
	\subsection*{tg489.2.8} 
	\textbf{Source : }Die Metaphysik der Sitten/Fußnoten\\  
	
	\noindent\textbf{Paragraphe : }
	
	4 Vorsätzlich, wenn gleich bloß leichtsinniger Weise, Unwahrheit zu sagen, pflegt zwar gewöhnlich Lüge (mendacium) genannt zu werden, weil sie wenigstens so fern auch schaden kann, daß der, welcher sie treuherzig nachsagt, als ein Leichtgläubiger anderen zum Gespötte wird. Im rechtlichen Sinne aber will man, daß nur diejenige Unwahrheit Lüge genannt werde, die einem anderen unmittelbar an seinem Rechte Abbruch tut, z.B. das falsche Vorgeben eines mit jemanden geschlossenen Vertrags, um ihn um das Seine zu bringen (falsiloquium dolosum), und dieser \match{Unterschied} sehr verwandter Begriffe ist nicht ungegründet: weil es bei der bloßen Erklärung seiner Gedanken immer dem andern frei bleibt, sie anzunehmen wofür er will, obgleich die gegründete Nachrede, daß dieser ein Mensch sei, dessen Reden man nicht glauben kann, so nahe an den Vorwurf, ihn einen Lügner zu nennen, streift, daß die Grenzlinie, die hier das, was zum Ius gehört, von dem, was der Ethik anheim fällt, nur so eben zu unterscheiden ist. 
	
	\unnumberedsection{Verbindung (6)} 
	\subsection*{tg431.2.47} 
	\textbf{Source : }Die Metaphysik der Sitten/Erster Teil. Metaphysische Anfangsgründe der Rechtslehre/Einleitung in die Rechtslehre\\  
	
	\noindent\textbf{Paragraphe : }2) Tue niemanden Unrecht (neminem laede) und solltest du darüber auch aus aller \match{Verbindung} mit andern heraus gehen und alle Gesellschaft meiden müssen (lex iuridica). 
	
	\subsection*{tg433.2.42} 
	\textbf{Source : }Die Metaphysik der Sitten/Erster Teil. Metaphysische Anfangsgründe der Rechtslehre/1. Teil. Das Privatrecht vom äußeren Mein und Dein überhaupt/1. Hauptstück\\  
	
	\noindent\textbf{Paragraphe : }Die Art also, etwas außer mir als das Meine zu haben, ist die bloß-rechtliche \match{Verbindung} des Willens des Subjekts mit jenem Gegenstande, unabhängig von dem Verhältnisse zu demselben im Raum und in der Zeit, nach dem Begriff eines intelligibelen Besitzes. – Ein Platz auf der Erde ist nicht darum ein äußeres Meine, weil ich ihn mit meinem Leibe einnehme (denn es betrifft hier nur meine äußere Freiheit, mithin nur den Besitz meiner selbst, kein Ding außer mir, und ist also nur ein inneres Recht); sondern, wenn ich ihn noch besitze, ob ich mich gleich von ihm weg und an einen andern Ort begeben habe, nur alsdenn betrifft es mein äußeres Recht, und derjenige, der die fortwährende Besetzung dieses Platzes durch meine Person zur Bedingung machen wollte, ihn als das Meine zu haben, muß entweder behaupten, es sei gar nicht möglich, etwas Äußeres als das Seine zu haben (welches dem Postulat § 2 widerstreitet), oder er verlangt, daß, um dieses zu können, ich in zwei Orten zugleich sei; welches denn aber so viel sagt, als: ich solle an einem Orte sein und auch nicht sein, wodurch er sich selbst widerspricht. 
	
	\subsection*{tg435.2.27} 
	\textbf{Source : }Die Metaphysik der Sitten/Erster Teil. Metaphysische Anfangsgründe der Rechtslehre/1. Teil. Das Privatrecht vom äußeren Mein und Dein überhaupt/2. Hauptstück. Von der Art, etwas Äußeres zu erwerben/1. Abschnitt. Vom Sachrecht\\  
	
	\noindent\textbf{Paragraphe : }Zuletzt kann noch gefragt werden: ob, wenn uns weder die Natur noch der Zufall, sondern bloß unser eigener Wille in Nachbarschaft mit einem Volk bringt, welches keine Aussicht zu einer bürgerlichen \match{Verbindung} mit ihm verspricht, wir nicht, in der Absicht, diese zu stiften und diese Menschen (Wilde) in einen rechtlichen Zustand zu versetzen (wie etwa die amerikanischen Wilden, die Hottentotten, die Neuholländer), befugt sein sollten, allenfalls mit Gewalt, oder (welches nicht viel besser ist) durch betrügerischen Kauf, Kolonien zu errichten und so Eigentümer ihres Bodens zu werden, und, ohne Rücksicht auf ihren ersten Besitz, Gebrauch von unserer Überlegenheit zu machen; zumal es die Natur selbst (als die das Leere verabscheuet) so zu fordern scheint, und große Landstriche in anderen Weltteilen an gesitteten Einwohnern sonst menschenleer geblieben wären, die jetzt herrlich bevölkert sind, oder gar auf immer bleiben müßten, und so der Zweck der Schöpfung vereitelt werden würde? Allein man sieht durch diesen Schleier der Ungerechtigkeit (Jesuitism), alle Mittel zu guten Zwecken zu billigen, leicht durch; diese Art der Erwerbung des Bodens ist also verwerflich. 
	
	\subsection*{tg437.2.12} 
	\textbf{Source : }Die Metaphysik der Sitten/Erster Teil. Metaphysische Anfangsgründe der Rechtslehre/1. Teil. Das Privatrecht vom äußeren Mein und Dein überhaupt/2. Hauptstück. Von der Art, etwas Äußeres zu erwerben/3. Abschnitt. Von dem auf dingliche Art persönlichen Recht\\  
	
	\noindent\textbf{Paragraphe : }Die natürliche Geschlechtsgemeinschaft ist nun entweder die nach der bloßen tierischen Natur (vaga libido, venus volgivaga, fornicatio), oder nach dem Gesetz. – Die letztere ist die Ehe (matrimonium), d.i. die \match{Verbindung} zweier Personen verschiedenen Geschlechts zum lebenswierigen wechselseitigen Besitz ihrer Geschlechtseigenschaften. – Der Zweck, Kinder zu erzeugen und zu erziehen, mag immer ein Zweck der Natur sein, zu welchem sie die Neigung der Geschlechter gegeneinander einpflanzte; aber daß der Mensch, der sich verehlicht, diesen Zweck sich vorsetzen müsse, wird zur Rechtmäßigkeit dieser seiner Verbindung nicht erfordert; denn sonst würde, wenn das Kinderzeugen aufhört, die Ehe sich zugleich von selbst auflösen. 
	
	\subsection*{tg437.2.41} 
	\textbf{Source : }Die Metaphysik der Sitten/Erster Teil. Metaphysische Anfangsgründe der Rechtslehre/1. Teil. Das Privatrecht vom äußeren Mein und Dein überhaupt/2. Hauptstück. Von der Art, etwas Äußeres zu erwerben/3. Abschnitt. Von dem auf dingliche Art persönlichen Recht\\  
	
	\noindent\textbf{Paragraphe : }Dieser Vertrag also der Hausherrschaft mit dem Gesinde kann nicht von solcher Beschaffenheit sein, daß der Gebrauch desselben ein Verbrauch sein würde, worüber das Urteil aber nicht bloß dem Hausherrn, sondern auch der Dienerschaft (die also nie Leibeigenschaft sein kann) zukommt; kann also nicht auf lebenslängliche, sondern allenfalls nur auf unbestimmte Zeit, binnen der ein Teil dem anderen die \match{Verbindung} aufkündigen darf, geschlossen werden. Die Kinder aber (selbst die eines durch sein Verbrechen zum Sklaven Gewordenen) sind jederzeit frei. Denn frei geboren ist jeder Mensch, weil er noch nichts verbrochen hat, und die Kosten der Erziehung bis zu seiner Volljährigkeit können ihm auch nicht als eine Schuld angerechnet werden, die er zu tilgen habe. Denn der Sklave müßte, wenn er könnte, seine Kinder auch erziehen, ohne ihnen dafür Kosten zu verrechnen; der Besitzer des Sklaven tritt also, bei dieses seinem Unvermögen, in die Stelle seiner Verbindlichkeit. 
	
	\subsection*{tg445.2.10} 
	\textbf{Source : }Die Metaphysik der Sitten/Erster Teil. Metaphysische Anfangsgründe der Rechtslehre/Anhang erläutender Bemerkungen zu den metaphysischen Anhangsgründen der Rechtslehre\\  
	
	\noindent\textbf{Paragraphe : }Die Rechtslehrer haben bisher nun zwei Gemeinplätze besetzt: den des dinglichen und den des persönlichen Rechts. Es ist natürlich, zu fragen: ob auch, da noch zwei Plätze, aus der bloßen Form der \match{Verbindung} beider zu einem Begriffe, als Glieder der Einteilung a priori, offen stehen, nämlich der eines auf persönliche Art dinglichen, imgleichen der eines auf dingliche Art persönlichen Rechts, ob nämlich ein solcher neuhinzukommender Begriff auch statthaft sei, und vor der Hand, obzwar nur problematisch, in der vollständigen Tafel der Einteilung angetroffen werden müsse. Das letztere leidet keinen Zweifel. Denn die bloß logische Einteilung (die vom Inhalt der Erkenntnis – dem Objekt – abstrahiert) ist immer Dichotomie, z.B. ein jedes Recht ist entweder ein dingliches oder ein nicht-dingliches Recht. Diejenige aber, von der hier die Rede ist, nämlich die metaphysische Einteilung, kann auch Tetrachotomie sein; weil, außer den zwei einfachen Gliedern  der Einteilung, noch zwei Verhältnisse, nämlich die der das Recht einschränkenden Bedingungen hinzukommen, unter denen das eine Recht mit dem anderen in Verbindung tritt, deren Möglichkeit einer besonderen Untersuchung bedarf. – Der Begriff eines auf persönliche Art dinglichen Rechts fällt ohne weitere Umstände weg; denn es läßt sich kein Recht einer Sache gegen eine Person denken. Nun fragt sich: ob die Umkehrung dieses Verhältnisses auch eben so undenkbar sei; oder ob dieser Begriff, nämlich der eines auf dingliche Art persönlichen Rechts, nicht allein ohne inneren Widerspruch, sondern selbst auch ein notwendiger (a priori in der Vernunft gegebener) zum Begriffe des äußeren Mein und Dein gehörender Begriff sei, Personen auf ähnliche Art als Sachen, zwar nicht in allen Stücken zu behandlen, aber sie doch zu besitzen und in vielen Verhältnissen mit ihnen als Sachen zu verfahren. 
	
	\unnumberedsection{Vereinigung (4)} 
	\subsection*{tg433.2.32} 
	\textbf{Source : }Die Metaphysik der Sitten/Erster Teil. Metaphysische Anfangsgründe der Rechtslehre/1. Teil. Das Privatrecht vom äußeren Mein und Dein überhaupt/1. Hauptstück\\  
	
	\noindent\textbf{Paragraphe : }Diese ursprüngliche Gemeinschaft des Bodens, und hiemit auch der Sachen auf demselben (communio fundi originaria), ist eine Idee, welche objektive (rechtlichpraktische) Realität hat, und ist ganz und gar von der uranfänglichen (communio primaeva) unterschieden,  welche eine Erdichtung ist; weil diese eine gestiftete Gemeinschaft hätte sein und aus einem Vertrage hervorgehen müssen, durch den alle auf den Privatbesitz Verzicht getan, und ein jeder, durch die \match{Vereinigung} seiner Besitzung mit der jedes andern, jenen in einen Gesamtbesitz verwandelt habe, und davon müßte uns die Geschichte einen Beweis geben. Ein solches Verfahren aber als ursprüngliche Besitznehmung an zusehen, und daß darauf jedes Menschen besonderer Besitz habe gegründet werden können und sollen, ist ein Widerspruch. 
	
	\subsection*{tg441.2.14} 
	\textbf{Source : }Die Metaphysik der Sitten/Erster Teil. Metaphysische Anfangsgründe der Rechtslehre/2. Teil. Das öffentliche Recht/1. Abschnitt. Das Staatsrecht\\  
	
	\noindent\textbf{Paragraphe : }Ein Staat (civitas) ist die \match{Vereinigung} einer Menge von Menschen unter Rechtsgesetzen. So fern diese als Gesetze a priori notwendig, d.i. aus Begriffen des äußeren Rechts überhaupt von selbst folgend (nicht statutarisch) sind, ist seine Form die Form eines Staats überhaupt, d.i. der Staat in der Idee, wie er nach reinen Rechtsprinzipien sein soll, welche jeder wirklichen Vereinigung zu einem gemeinen Wesen (also im Inneren) zur Richtschnur (norma) dient. 
	
	\subsection*{tg441.2.35} 
	\textbf{Source : }Die Metaphysik der Sitten/Erster Teil. Metaphysische Anfangsgründe der Rechtslehre/2. Teil. Das öffentliche Recht/1. Abschnitt. Das Staatsrecht\\  
	
	\noindent\textbf{Paragraphe : }Also sind es drei verschiedene Gewalten (potestas legislatoria, executoria, iudiciaria), wodurch der Staat (civitas) seine Autonomie hat, d.i. sich selbst nach Freiheitsgesetzen bildet und erhält. – In ihrer \match{Vereinigung} besteht das Heil des Staats (salus reipublicae suprema lex est); worunter man nicht das Wohl der Staatsbürger und ihre Glückseligkeit verstehen muß; denn die kann vielleicht (wie auch Rousseau behauptet) im Naturzustande, oder auch unter einer despotischen Regierung, viel behaglicher und erwünschter ausfallen: sondern den Zustand der größten Übereinstimmung der Verfassung mit Rechtsprinzipien versteht, als nach welchem zu streben uns die Vernunft durch einen kategorischen Imperativ verbindlich macht. 
	
	\subsection*{tg484.2.4} 
	\textbf{Source : }Die Metaphysik der Sitten/Zweiter Teil. Metaphysische Anfangsgründe der Tugendlehre/I. Ethische Elementarlehre/II. Teil. Von den Tugendpflichten gegen andere/Beschluß der Elementarlehre. Von der innigsten Vereinigung der Liebe mit der Achtung in der Freundschaft\\  
	
	\noindent\textbf{Paragraphe : }
	Freundschaft (in ihrer Vollkommenheit betrachtet) ist die \match{Vereinigung} zweier Personen durch gleiche wechselseitige Liebe und Achtung. – Man sieht leicht, daß sie ein Ideal der Teilnehmung und Mitteilung an dem Wohl eines jeden dieser durch den moralisch guten Willen Vereinigten sei, und, wenn es auch nicht das ganze Glück des Lebens bewirkt, die Aufnahme desselben in ihre beiderseitige Gesinnung die Würdigkeit enthalte, glücklich zu sein, mithin daß Freundschaft unter Menschen Pflicht derselben ist. – Daß aber Freundschaft eine bloße (aber doch praktisch-notwendige) Idee, in der Ausübung zwar unerreichbar, aber doch darnach (als einem Maximum der guten Gesinnung  gegen einander) zu streben von der Vernunft aufgegebene, nicht etwa gemeine, sondern ehrenvolle Pflicht sei, ist leicht zu ersehen. Denn, wie ist es für den Menschen in Verhältnis zu seinem Nächsten möglich, die Gleichheit eines der dazu erforderlichen Stücke eben derselben Pflicht (z.B. des wechselseitigen Wohlwollens) in dem einen, mit eben derselben Gesinnung im anderen auszumitteln, noch mehr aber, welches Verhältnis das Gefühl aus der einen Pflicht zu dem aus der andern (z.B. das aus dem Wohlwollen, zu dem aus der Achtung) in derselben Person habe, und ob, wenn die eine in der Liebe inbrünstiger ist, sie nicht eben dadurch in der Achtung des anderen etwas einbüße, so daß beiderseitig Liebe und Hochschätzung subjektiv schwerlich in das Ebenmaß des Gleichgewichts gebracht werden wird; welches doch zur Freundschaft erforderlich ist? – Denn man kann jene als Anziehung, diese als Abstoßung betrachten, und wenn das Prinzip der ersteren Annäherung gebietet, das der zweiten sich einander in geziemendem Abstande zu halten fordert; welche Einschränkung der Vertraulichkeit, durch die Regel: daß auch die besten Freunde sich unter einander nicht gemein machen sollen, ausgedrückt, eine Maxime enthält, die nicht bloß dem Höheren gegen den Niedrigen, sondern auch umgekehrt gilt. Denn der Höhere fühlt, ehe man es sich versieht, seinen Stolz gekränkt und will die Achtung des Niedrigen, etwa für einen Augenblick aufgeschoben, nicht aber aufgehoben wissen, welche aber einmal verletzt, innerlich unwiderbringlich verloren ist; wenn gleich die äußere Bezeichnung derselben (das Zeremoniell) wieder in den alten Gang gebracht wird. 
	
	\unnumberedsection{Verfahren (1)} 
	\subsection*{tg433.2.41} 
	\textbf{Source : }Die Metaphysik der Sitten/Erster Teil. Metaphysische Anfangsgründe der Rechtslehre/1. Teil. Das Privatrecht vom äußeren Mein und Dein überhaupt/1. Hauptstück\\  
	
	\noindent\textbf{Paragraphe : }Der Begriff eines bloß-rechtlichen Besitzes ist kein empirischer (von Raum und Zeitbedingungen abhängiger) Begriff, und gleichwohl hat er praktische Realität, d.i. er muß auf Gegenstände der Erfahrung, deren Erkenntnis von jenen Bedingungen abhängig ist, anwendbar sein. – Das \match{Verfahren} mit dem Rechtsbegriffe in Ansehung der letzteren, als des möglichen äußeren Mein und Dein, ist folgendes: Der Rechtsbegriff, der bloß in der Vernunft liegt, kann nicht unmittelbar auf Erfahrungsobjekte, und auf den Begriff eines empirischen Besitzes, sondern muß zunächst auf den reinen Verstandesbegriff eines Besitzes überhaupt angewandt werden, so daß, statt der Inhabung (detentio), als einer empirischen Vorstellung des Besitzes, der von allen Raumes- und Zeitbedingungen abstrahierende Begriff des Habens und nur, daß der Gegenstand als in meiner Gewalt (in potestate mea positum esse) sei, gedacht werde; da dann der Ausdruck des Äußeren nicht das Dasein in einem anderen Orte, als wo ich bin, oder meiner Willensentschließung und Annahme als in einer anderen Zeit, wie der des Angebots, sondern nur einen von mir unterschiedenen Gegenstand bedeutet. Nun will die praktische Vernunft durch ihr Rechtsgesetz, daß ich das Mein und Dein in der Anwendung auf Gegenstände nicht nach sinnlichen Bedingungen, sondern abgesehen von denselben, weil es eine Bestimmung der Willkür nach Freiheitsgesetzen betrifft, auch den Besitz desselben denke, indem nur ein Verstandesbegriff unter Rechtsbegriffe subsumiert werden kann. Also werde ich sagen: ich besitze einen Acker, ob er zwar ein ganz anderer Platz ist, als worauf ich mich wirklich befinde. Denn die Rede ist hier nur von einem intellektuellen Verhältnis zum Gegenstande, so fern ich ihn in meiner Gewalt habe (ein von Raumesbestimmungen unabhängiger Verstandesbegriff des Besitzes), und er ist mein,  weil mein, zu desselben beliebigem Gebrauch sich bestimmender, Wille dem Gesetz der äußeren Freiheit nicht widerstreitet. Gerade darin: daß, abgesehen vom Besitz in der Erscheinung (der Inhabung) dieses Gegenstandes meiner Willkür, die praktische Vernunft den Besitz nach Verstandesbegriffen, nicht nach empirischen, sondern solchen, die a priori die Bedingungen desselben enthalten können, gedacht wissen will, liegt der Grund der Gültigkeit eines solchen Begriffs vom Besitze (possessio noumenon) als einer allgemeingeltenden Gesetzgebung; denn eine solche ist in dem Ausdrucke enthalten: »dieser äußere Gegenstand ist mein«; weil allen andern dadurch eine Verbindlichkeit auferlegt wird, die sie sonst nicht hätten, sich des Gebrauchs desselben zu enthalten. 
	
	\unnumberedsection{Verhaltnis (38)} 
	\subsection*{tg430.2.43} 
	\textbf{Source : }Die Metaphysik der Sitten/Erster Teil. Metaphysische Anfangsgründe der Rechtslehre/Einleitung in die Metaphysik der Sitten\\  
	
	\noindent\textbf{Paragraphe : }Ein Widerstreit der Pflichten (collisio officiorum. s. obligationum) würde das \match{Verhältnis} derselben sein, durch welches eine derselben die andere (ganz oder zum Teil) aufhöbe. – Da aber Pflicht und Verbindlichkeit überhaupt Begriffe sind, welche die objektive praktische Notwendigkeit gewisser Handlungen ausdrücken und zwei einander entgegengesetzte Regeln nicht zugleich notwendig sein können, sondern, wenn nach einer derselben zu handeln es Pflicht ist, so ist nach der entgegengesetzten zu handeln nicht allein keine Pflicht, sondern sogar pflichtwidrig: so ist eine Kollision von Pflichten und Verbindlichkeiten gar nicht denkbar (obligationes non colliduntur). Es können aber gar wohl zwei Gründe der Verbindlichkeit (rationes obligandi), deren einer aber, oder der andere, zur Verpflichtung nicht zureichend ist (rationes obligandi non obligantes), in einem Subjekt und der Regel, die es sich vorschreibt, verbunden sein, da dann der eine nicht Pflicht ist. – Wenn zwei solcher Gründe einander widerstreiten, so sagt die praktische Philosophie nicht: daß die stärkere Verbindlichkeit die Oberhand behalte (fortior obligatio vincit), sondern  der stärkere Verpflichtungsgrund behält den Platz (fortior obligandi ratio vincit). 
	
	\subsection*{tg431.2.46} 
	\textbf{Source : }Die Metaphysik der Sitten/Erster Teil. Metaphysische Anfangsgründe der Rechtslehre/Einleitung in die Rechtslehre\\  
	
	\noindent\textbf{Paragraphe : }1) Sei ein rechtlicher Mensch (honeste vive). Die rechtliche Ehrbarkeit (honestas iuridica) bestehet darin: im \match{Verhältnis} zu anderen seinen Wert als den eines Menschen zu behaupten, welche Pflicht durch den Satz ausgedrückt wird: »mache dich anderen nicht zum bloßen Mittel, sondern sei für sie zugleich Zweck«. Diese Pflicht wird im folgenden als Verbindlichkeit aus dem Rechte der Menschheit in unserer eigenen Person erklärt werden (lex iusti). 
	
	\subsection*{tg431.2.76} 
	\textbf{Source : }Die Metaphysik der Sitten/Erster Teil. Metaphysische Anfangsgründe der Rechtslehre/Einleitung in die Rechtslehre\\  
	
	\noindent\textbf{Paragraphe : }Da die Subjekte, in Ansehung deren ein \match{Verhältnis} des Rechts zur Pflicht (es sei statthaft oder unstatthaft) gedacht wird, verschiedne Beziehungen zulassen: so wird auch in dieser Absicht eine Einteilung vorgenommen werden können. 
	
	\subsection*{tg431.2.80} 
	\textbf{Source : }Die Metaphysik der Sitten/Erster Teil. Metaphysische Anfangsgründe der Rechtslehre/Einleitung in die Rechtslehre\\  
	
	\noindent\textbf{Paragraphe : }Das rechtliche \match{Verhältnis} des Menschen zu Wesen die weder Recht noch Pflicht haben. 
	
	\subsection*{tg431.2.87} 
	\textbf{Source : }Die Metaphysik der Sitten/Erster Teil. Metaphysische Anfangsgründe der Rechtslehre/Einleitung in die Rechtslehre\\  
	
	\noindent\textbf{Paragraphe : }Denn es ist ein \match{Verhältnis} von Menschen zu Menschen. 
	
	\subsection*{tg431.2.90} 
	\textbf{Source : }Die Metaphysik der Sitten/Erster Teil. Metaphysische Anfangsgründe der Rechtslehre/Einleitung in die Rechtslehre\\  
	
	\noindent\textbf{Paragraphe : }Das rechtliche \match{Verhältnis} des Menschen zu Wesen, die lauter Pflichten und keine Rechte haben. 
	
	\subsection*{tg431.2.95} 
	\textbf{Source : }Die Metaphysik der Sitten/Erster Teil. Metaphysische Anfangsgründe der Rechtslehre/Einleitung in die Rechtslehre\\  
	
	\noindent\textbf{Paragraphe : }Das rechtliche \match{Verhältnis} des Menschen zu einem Wesen, was lauter Rechte und keine Pflicht hat (Gott). 
	
	\subsection*{tg431.2.99} 
	\textbf{Source : }Die Metaphysik der Sitten/Erster Teil. Metaphysische Anfangsgründe der Rechtslehre/Einleitung in die Rechtslehre\\  
	
	\noindent\textbf{Paragraphe : }Also findet sich nur in No. 2 ein reales \match{Verhältnis} zwischen Recht und Pflicht. Der Grund, warum es nicht auch in No. 4 angetroffen wird, ist: weil es eine transzendente
	Pflicht sein würde, d.i. eine solche, der kein äußeres verpflichtendes Subjekt korrespondierend gegeben werden kann, mithin das Verhältnis in theoretischer Rücksicht hier nur ideal, d.i. zu einem Gedankendinge ist, was wir uns selbst, aber doch nicht durch seinen ganz leeren, sondern, in Beziehung auf uns selbst und die Maximen der inneren Sittlichkeit, mithin in praktischer innerer Absicht, fruchtbaren Begriff, machen, worin denn auch unsere ganze immanente (ausführbare) Pflicht in diesem bloß gedachten Verhältnisse allein besteht. 
	
	\subsection*{tg433.2.17} 
	\textbf{Source : }Die Metaphysik der Sitten/Erster Teil. Metaphysische Anfangsgründe der Rechtslehre/1. Teil. Das Privatrecht vom äußeren Mein und Dein überhaupt/1. Hauptstück\\  
	
	\noindent\textbf{Paragraphe : }Der äußeren Gegenstände meiner Willkür können nur drei sein: 1) eine (körperliche) Sache außer mir; 2) die Willkür eines anderen zu einer bestimmten Tat (praestatio); 3) der Zustand eines anderen in \match{Verhältnis} auf mich: nach den Kategorien der Substanz, Kausalität, und Gemeinschaft zwischen mir und äußeren Gegenständen nach Freiheitsgesetzen. 
	
	\subsection*{tg433.2.31} 
	\textbf{Source : }Die Metaphysik der Sitten/Erster Teil. Metaphysische Anfangsgründe der Rechtslehre/1. Teil. Das Privatrecht vom äußeren Mein und Dein überhaupt/1. Hauptstück\\  
	
	\noindent\textbf{Paragraphe : }Wenn auch gleich ein Boden als frei, d.i. zu jedermanns Gebrauch offen angesehen, oder dafür erklärt würde, so kann man doch nicht sagen, daß er es von Natur und ursprünglich, vor allem rechtlichem Akt, frei sei, denn auch das wäre ein \match{Verhältnis} zu Sachen, nämlich dem Boden, der jedermann seinen Besitz verweigerte, sondern, weil diese Freiheit des Bodens ein Verbot für jedermann sein würde, sich desselben zu bedienen; wozu ein gemeinsamer Besitz desselben erfordert wird, der ohne Vertrag nicht statt finden kann. Ein Boden aber, der nur durch diesen frei sein kann, muß wirklich im Besitze aller derer (zusammen verbundenen) sein, die sich wechselseitig den Gebrauch desselben untersagen, oder ihn suspendieren. 
	
	\subsection*{tg433.2.44} 
	\textbf{Source : }Die Metaphysik der Sitten/Erster Teil. Metaphysische Anfangsgründe der Rechtslehre/1. Teil. Das Privatrecht vom äußeren Mein und Dein überhaupt/1. Hauptstück\\  
	
	\noindent\textbf{Paragraphe : }Ebendasselbe gilt auch von dem Begriffe des rechtlichen Besitzes einer Person, als zu der Habe des Subjekts gehörend (sein Weib, Kind, Knecht): daß nämlich diese häusliche Gemeinschaft und der wechselseitige Besitz des Zustandes aller Glieder derselben, durch die Befugnis, sich örtlich von einander zu trennen, nicht aufgehoben wird; weil es ein rechtliches \match{Verhältnis} ist, was sie verknüpft, und das äußere Mein und Dein hier, eben so wie in vorigen Fällen, gänzlich auf der Voraussetzung der Möglichkeit eines reinen Vernunftbesitzes ohne Inhabung beruht. 
	
	\subsection*{tg435.2.3} 
	\textbf{Source : }Die Metaphysik der Sitten/Erster Teil. Metaphysische Anfangsgründe der Rechtslehre/1. Teil. Das Privatrecht vom äußeren Mein und Dein überhaupt/2. Hauptstück. Von der Art, etwas Äußeres zu erwerben/1. Abschnitt. Vom Sachrecht\\  
	
	\noindent\textbf{Paragraphe : }Die gewöhnliche Erklärung des Rechts in einer Sache (ius reale, ius in re): »es sei das Recht gegen jeden Besitzer derselben«, ist eine richtige Nominaldefinition. – Aber, was ist das, was da macht, daß ich mich wegen eines äußeren Gegenstandes an jeden Inhaber desselben halten, und ihn (per vindicationem) nötigen kann, mich wieder in Besitz desselben zu setzen? Ist dieses äußere rechtliche \match{Verhältnis} meiner Willkür etwa ein unmittelbares Verhältnis zu einem körperlichen Dinge? So müßte derjenige, welcher sein Recht nicht unmittelbar auf Personen, sondern auf Sachen bezogen denkt, es sich freilich (obzwar nur auf dunkele Art) vorstellen: nämlich, weil dem Recht auf einer  Seite eine Pflicht auf der andern korrespondiert, daß die äußere Sache, ob sie zwar dem ersten Besitzer abhanden gekommen, diesem doch immer verpflichtet bleibe, d.i. sich jedem anmaßlichen anderen Besitzer weigere, weil sie jenem schon verbindlich ist, und so mein Recht, gleich einem die Sache begleitenden und vor allem fremden Angriffe bewahrenden Genius, den fremden Besitzer immer an mich weise. Es ist also ungereimt, sich Verbindlichkeit einer Person gegen Sachen und umgekehrt zu denken, wenn es gleich allenfalls erlaubt werden mag, das rechtliche Verhältnis durch ein solches Bild zu versinnlichen, und sich so auszudrücken. 
	
	\subsection*{tg435.2.5} 
	\textbf{Source : }Die Metaphysik der Sitten/Erster Teil. Metaphysische Anfangsgründe der Rechtslehre/1. Teil. Das Privatrecht vom äußeren Mein und Dein überhaupt/2. Hauptstück. Von der Art, etwas Äußeres zu erwerben/1. Abschnitt. Vom Sachrecht\\  
	
	\noindent\textbf{Paragraphe : }Unter dem Wort: Sachenrecht (ius reale) wird übrigens nicht bloß das Recht in einer Sache (ius in re) sondern auch der Inbegriff aller Gesetze, die das dingliche Mein und Dein betreffen, verstanden. – Es ist aber klar, daß ein Mensch, der auf Erden ganz allein wäre, eigentlich kein äußeres Ding als das Seine haben, oder erwerben könnte; weil zwischen ihm, als Person, und allen anderen äußeren Dingen, als Sachen, es gar kein \match{Verhältnis} der Verbindlichkeit gibt. Es gibt also, eigentlich und buchstäblich verstanden, auch kein (direktes) Recht in einer Sache, sondern nur dasjenige wird so genannt, was jemanden gegen eine Person zukommt, die mit allen anderen (im bürgerlichen Zustande) im gemeinsamen Besitz ist. 
	
	\subsection*{tg437.2.19} 
	\textbf{Source : }Die Metaphysik der Sitten/Erster Teil. Metaphysische Anfangsgründe der Rechtslehre/1. Teil. Das Privatrecht vom äußeren Mein und Dein überhaupt/2. Hauptstück. Von der Art, etwas Äußeres zu erwerben/3. Abschnitt. Von dem auf dingliche Art persönlichen Recht\\  
	
	\noindent\textbf{Paragraphe : }Aus denselben Gründen ist das \match{Verhältnis} der Verehlichten ein Verhältnis der Gleichheit des Besitzes, sowohl der Personen, die einander wechselseitig besitzen (folglich nur in Monogamie, denn in einer Polygamie gewinnt die Person, die sich weggibt, nur einen Teil desjenigen, dem sie ganz anheim fällt, und macht sich also zur bloßen Sache), als auch der Glücksgüter, wobei sie doch die Befugnis haben, sich, obgleich nur durch einen besonderen Vertrag, des Gebrauchs eines Teils derselben zu begeben. 
	
	\subsection*{tg437.2.4} 
	\textbf{Source : }Die Metaphysik der Sitten/Erster Teil. Metaphysische Anfangsgründe der Rechtslehre/1. Teil. Das Privatrecht vom äußeren Mein und Dein überhaupt/2. Hauptstück. Von der Art, etwas Äußeres zu erwerben/3. Abschnitt. Von dem auf dingliche Art persönlichen Recht\\  
	
	\noindent\textbf{Paragraphe : }Dieses Recht ist das des Besitzes eines äußeren Gegenstandes als einer Sache und des Gebrauchs desselben 
	als einer Person. – Das Mein und Dein nach diesem Recht ist das häusliche und das \match{Verhältnis} in diesem Zustande ist das der Gemeinschaft freier Wesen, die durch den wechselseitigen Einfluß (der Person des einen auf das andere) nach dem Prinzip der äußeren Freiheit (Kausalität) eine Gesellschaft von Gliedern eines Ganzen (in Gemeinschaft stehender Personen) ausmachen, welches das Hauswesen heißt. – Die Erwerbungsart dieses Zustandes und in demselben geschieht weder durch eigenmächtige Tat (facto), noch durch bloßen Vertrag (pacto), sondern durchs Gesetz (lege), welches, weil es kein Recht in einer Sache, auch nicht ein bloßes Recht gegen eine Person, sondern auch ein Besitz derselben zugleich ist, ein über alles Sachen- und persönliche hinausliegendes Recht, nämlich das Recht der Menschheit in unserer eigenen Person sein muß, welches ein natürliches Erlaubnisgesetz zur Folge hat, durch dessen Gunst uns eine solche Erwerbung möglich ist. 
	
	\subsection*{tg438.2.23} 
	\textbf{Source : }Die Metaphysik der Sitten/Erster Teil. Metaphysische Anfangsgründe der Rechtslehre/1. Teil. Das Privatrecht vom äußeren Mein und Dein überhaupt/2. Hauptstück. Von der Art, etwas Äußeres zu erwerben/Episodischer Abschnitt. Von der idealen Erwerbung eines äußeren Gegenstandes der Willkür\\  
	
	\noindent\textbf{Paragraphe : }Daß der Verstorbene nach seinem Tode (wenn er also nicht mehr ist) noch etwas besitzen könne, wäre eine Ungereimtheit zu denken, wenn der Nachlaß eine Sache wäre. Nun ist aber der gute Name ein angebornes äußeres, obzwar bloß ideales Mein oder Dein, was dem Subjekt als einer Person anhängt, von deren Natur, ob sie mit dem Tode gänzlich aufhöre zu sein, oder immer noch als solche übrig bleibe, ich abstrahieren kann und muß, weil ich, im rechtlichen \match{Verhältnis} auf andere, jede Person bloß nach ihrer Menschheit, mithin als homo noumenon wirklich betrachte, und so ist jeder Versuch, ihn nach dem Tode in übele falsche Nachrede zu bringen, immer bedenklich, obgleich eine gegründete Anklage desselben gar wohl statt findet (mithin der Grundsatz: de mortuis nihil nisi bene, unrichtig ist), weil gegen den Abwesenden, welcher sich nicht verteidigen kann, Vorwürfe auszustreuen, ohne die größte Gewißheit derselben, wenigstens ungroßmütig ist. 
	
	\subsection*{tg439.2.12} 
	\textbf{Source : }Die Metaphysik der Sitten/Erster Teil. Metaphysische Anfangsgründe der Rechtslehre/1. Teil. Das Privatrecht vom äußeren Mein und Dein überhaupt/3. Hauptstück. Von der subjektiv-bedingten Erwerbung durch den Ausspruch einer öffentlichen Gerichtsbarkeit\\  
	
	\noindent\textbf{Paragraphe : }Dieser Vertrag (donatio), wodurch ich das Mein, meine Sache (oder mein Recht) unvergolten (gratis) veräußere, enthält ein \match{Verhältnis} von mir, dem Schenkenden (donans), zu einem anderen, dem Beschenkten (donatarius), nach dem Privatrecht, wodurch das Meine auf diesen durch Annehmung des letzteren (donum) übergeht. – Es ist aber nicht zu präsumieren, daß ich hiebei gemeinet sei, zu der Haltung meines Versprechens gezwungen zu werden, und also auch meine Freiheit umsonst wegzugeben, und gleichsam mich selbst wegzuwerfen (nemo suum iactare praesumitur), welches doch nach dem Recht im bürgerlichen Zustande geschehen würde; denn da kann der Zubeschenkende mich zu Leistung des Versprechens zwingen. Es müßte also, wenn die Sache vor Gericht käme, d.i. nach einem öffentlichen Recht, entweder präsumiert werden, der  Verschenkende willigte zu diesem Zwange ein, welches ungereimt ist, oder der Gerichtshof sehe in seinem Spruch (Sentenz) gar nicht darauf, ob jener die Freiheit, von seinem Versprechen abzugehen, hat vorbehalten wollen, oder nicht, sondern auf das, was gewiß ist, nämlich das Versprechen und die Akzeptation des Promissars. Wenn also gleich der Promittent, wie wohl vermutet werden kann, gedacht hat, daß, wenn es ihn noch vor der Erfüllung gereuet, das Versprechen getan zu haben, man ihn daran nicht binden könne: so nimmt doch das Gericht an, daß er sich dieses ausdrücklich hätte vorbehalten müssen, und, wenn er es nicht getan hat, zu Erfüllung des Versprechens könne gezwungen werden, und dieses Prinzip nimmt der Gerichtshof darum an, weil ihm sonst das Rechtsprechen unendlich erschwert, oder gar unmöglich gemacht werden würde. 
	
	\subsection*{tg439.2.44} 
	\textbf{Source : }Die Metaphysik der Sitten/Erster Teil. Metaphysische Anfangsgründe der Rechtslehre/1. Teil. Das Privatrecht vom äußeren Mein und Dein überhaupt/3. Hauptstück. Von der subjektiv-bedingten Erwerbung durch den Ausspruch einer öffentlichen Gerichtsbarkeit\\  
	
	\noindent\textbf{Paragraphe : }Der rechtliche Zustand ist dasjenige \match{Verhältnis} der Menschen unter einander, welches die Bedingungen enthält,  unter denen allein jeder seines Rechts teilhaftig werden kann, und das formale Prinzip der Möglichkeit desselben, nach der Idee eines allgemein gesetzgebenden Willens betrachtet, heißt die öffentliche Gerechtigkeit, welche in Beziehung, entweder auf die Möglichkeit, oder Wirklichkeit, oder Notwendigkeit des Besitzes der Gegenstände (als der Materie der Willkür) nach Gesetzen in die beschützende (iustitia tutatrix), die wechselseitig erwerbende (iustitia commutativa) und die austeilende Gerechtigkeit (iustitia distributiva) eingeteilt werden kann. – Das Gesetz sagt hiebei erstens bloß, welches Verhalten innerlich der Form nach recht ist (lex iusti); zweitens, was als Materie noch auch äußerlich gesetzfähig, d.i. dessen Besitzstand rechtlich ist (lex iuridica); drittens, was und wovon der Ausspruch vor einem Gerichtshofe in einem besonderen Falle unter dem gegebenen Gesetze diesem gemäß, d.i. Rechtens ist (lex iustitiae), wo man denn auch jenen Gerichtshof selbst die Gerechtigkeit eines Landes nennt, und, ob eine solche sei oder nicht sei, als die wichtigste unter allen rechtlichen Angelegenheiten gefragt werden kann. 
	
	\subsection*{tg441.2.24} 
	\textbf{Source : }Die Metaphysik der Sitten/Erster Teil. Metaphysische Anfangsgründe der Rechtslehre/2. Teil. Das öffentliche Recht/1. Abschnitt. Das Staatsrecht\\  
	
	\noindent\textbf{Paragraphe : }Alle jene drei Gewalten im Staate sind Würden, und, als wesentliche aus der Idee eines Staats überhaupt zur Gründung desselben (Konstitution) notwendig hervorgehend, Staatswürden. Sie enthalten das \match{Verhältnis} eines allgemeinen Oberhaupts (der, nach Freiheitsgesetzen betrachtet, kein anderer als das vereinigte Volk selbst sein kann) zu der vereinzelten Menge ebendesselben als Untertans, d.i. des Gebietenden (imperans) gegen den Gehorsamenden (subditus). – Der Akt, wodurch sich das Volk selbst zu einem Staat konstituiert, eigentlich aber nur die Idee desselben, nach der die Rechtmäßigkeit desselben allein gedacht werden kann, ist der ursprüngliche Kontrakt, nach welchem alle (omnes et singuli) im Volk ihre äußere Freiheit aufgeben, um sie als Glieder eines gemeinen Wesens, d.i. des Volks als Staat betrachtet (universi) sofort wiederaufzunehmen, und man kann nicht sagen: der Staat, der Mensch im Staate, habe einen Teil seiner angebornen äußeren Freiheit einem Zwecke aufgeopfert, sondern er hat die wilde gesetzlose Freiheit gänzlich verlassen, um seine Freiheit überhaupt in einer gesetzlichen Abhängigkeit, d.i. in einem rechtlichen Zustande unvermindert wieder zu finden; weil diese Abhängigkeit aus seinem eigenen gesetzgebenden Willen entspringt. 
	
	\subsection*{tg441.2.94} 
	\textbf{Source : }Die Metaphysik der Sitten/Erster Teil. Metaphysische Anfangsgründe der Rechtslehre/2. Teil. Das öffentliche Recht/1. Abschnitt. Das Staatsrecht\\  
	
	\noindent\textbf{Paragraphe : }Die drei Gewalten im Staat, die aus dem Begriff eines gemeinen Wesens überhaupt (res publica latius dicta) hervorgehen, sind nur so viel Verhältnisse des vereinigten, a priori aus der Vernunft abstammenden, Volkswillens und eine reine Idee von einem Staatsoberhaupt, welche objektive praktische Realität hat. Dieses Oberhaupt (der Souverän) aber ist so fern nur ein (das gesamte Volk vorstellendes) Gedankending, als es noch an einer physischen Person mangelt, welche die höchste Staatsgewalt vorstellt, und dieser Idee Wirksamkeit auf den Volkswillen verschafft. Das \match{Verhältnis} der ersteren zum letzteren ist nun auf dreierlei verschiedene Art denkbar: entweder daß einer im Staate über alle, oder daß einige, die einander gleich sind, vereinigt, über alle andere, oder daß alle zusammen über einen jeden, mithin auch über sich selbst gebieten, d.i. die Staatsform ist entweder autokratisch, oder aristokratisch, oder demokratisch. (Der Ausdruck monarchisch, statt autokratisch, ist nicht dem Begriffe, den man hier will, angemessen; denn Monarch ist der, welcher die höchste, Autokrator aber, oder Selbstherrscher,  der, welcher alle Gewalt hat; dieser ist der Souverän, jener repräsentiert ihn bloß.) – Man wird leicht gewahr, daß die autokratische Staatsform die einfachste sei, nämlich von Einem (dem Könige) zum Volke, mithin wo nur Einer der Gesetzgeber ist. Die aristokratische ist schon aus zwei Verhältnissen zusammengesetzt: nämlich dem der Vornehmen (als Gesetzgeber) zu einander, um den Souverän zu machen, und dann das dieses Souveräns zum Volk; die demokratische aber die allerzusammengesetzteste, nämlich den Willen aller zuerst zu vereinigen, um daraus ein Volk, dann den der Staatsbürger, um ein gemeines Wesen zu bilden, und dann diesem gemeinen Wesen den Souverän, der dieser vereinigte Wille selbst ist, vorzusetzen.
	
	
	9
	Was die Handhabung des Rechts im Staat betrifft, so ist freilich die einfachste auch zugleich die beste; aber, was das Recht selbst anlangt, die gefährlichste fürs Volk, in Betracht des Despotismus, zu dem sie so sehr einladet. Das Simplifizieren ist zwar im Maschinenwerk der Vereinigung des Volks durch Zwangsgesetze die vernünftige Maxime: wenn nämlich alle im Volk passiv sind, und Einem, der über sie ist, gehorchen; aber das gibt keine Untertanen als Staatsbürger. Was die Vertröstung, womit sich das Volk befriedigen soll, betrifft: daß nämlich die Monarchie (eigentlich hier Autokratie) die beste Staatsverfassung sei, wenn der Monarch gut ist (d.i. nicht bloß den Willen, sondern auch die Einsicht dazu hat), gehört zu den tautologischen Weisheitssprüchen, und sagt nichts mehr, als: die beste Verfassung ist die, durch welche der Staatsverwalter zum besten Regenten gemacht wird, d.i. diejenige, welche die beste ist. 
	
	\subsection*{tg442.2.25} 
	\textbf{Source : }Die Metaphysik der Sitten/Erster Teil. Metaphysische Anfangsgründe der Rechtslehre/2. Teil. Das öffentliche Recht/2. Abschnitt. Das Völkerrecht\\  
	
	\noindent\textbf{Paragraphe : }Das Recht im Kriege ist gerade das im Völkerrecht, wobei die meiste Schwierigkeit ist, um sich auch nur einen Begriff davon zu machen, und ein Gesetz in diesem gesetzlosen Zustande zu denken (inter arma silent leges), ohne sich selbst zu widersprechen; es müßte denn dasjenige sein: den Krieg nach solchen Grundsätzen zu führen, nach welchen es immer noch möglich bleibt, aus jenem Naturzustande der Staaten (im äußeren \match{Verhältnis} gegen einander) herauszugehen, und in einen rechtlichen zu treten. 
	
	\subsection*{tg442.2.26} 
	\textbf{Source : }Die Metaphysik der Sitten/Erster Teil. Metaphysische Anfangsgründe der Rechtslehre/2. Teil. Das öffentliche Recht/2. Abschnitt. Das Völkerrecht\\  
	
	\noindent\textbf{Paragraphe : }Kein Krieg unabhängiger Staaten gegen einander kann ein Strafkrieg (bellum punitivum) sein, denn Strafe findet nur im Verhältnisse eines Obern (imper antis) gegen den Unterworfenen (subditum) statt, welches \match{Verhältnis} nicht das der Staaten gegen einander ist, – aber auch weder ein Ausrottungs– (bellum internecinum) noch Unterjochungskrieg (bellum subiugatorium), der eine moralische  Vertilgung eines Staats (dessen Volk nun mit dem des Überwinders entweder in eine Masse verschmelzt, oder in Knechtschaft verfällt) sein würde. Nicht als ob dieses Notmittel des Staats, zum Friedenszustande zu gelangen, an sich dem Rechte eines Staats widerspräche, sondern weil die Idee des Völkerrechts bloß den Begriff eines Antagonismus nach Prinzipien der äußeren Freiheit bei sich führt, um sich bei dem Seinen zu erhalten, aber nicht eine Art zu erwerben, als welche, durch Vergrößerung der Macht des einen Staats, für den anderen bedrohend sein kann. 
	
	\subsection*{tg442.2.4} 
	\textbf{Source : }Die Metaphysik der Sitten/Erster Teil. Metaphysische Anfangsgründe der Rechtslehre/2. Teil. Das öffentliche Recht/2. Abschnitt. Das Völkerrecht\\  
	
	\noindent\textbf{Paragraphe : }Die Menschen, welche ein Volk ausmachen, können, als Landeseingeborne, nach der Analogie der Erzeugung von einem gemeinschaftlichen Elternstamm (congeniti) vorgestellt werden, ob sie es gleich nicht sind: dennoch aber, in intellektueller und rechtlicher Bedeutung, als von einer gemeinschaftlichen Mutter (der Republik) geboren, gleichsam eine Familie (gens, natio) ausmachen, deren Glieder (Staatsbürger) alle ebenbürtig sind, und mit denen, die neben ihnen im Naturzustande leben möchten, als unedlen keine Vermischung eingehen, obgleich diese (die Wilden) ihrerseits sich wiederum wegen der gesetzlosen Freiheit, die sie gewählt haben, sich vornehmer dünken, die gleichfalls Völkerschaften, aber nicht Staaten, ausmachen. Das Recht der Staaten in \match{Verhältnis} zu einander (welches nicht ganz richtig im Deutschen das Völkerrecht genannt wird, sondern vielmehr das Staatenrecht (ius publicum civitatum) heißen sollte) ist nun dasjenige, was wir unter dem Namen des Völkerrechts zu betrachten haben: wo ein Staat, als eine moralische Person, gegen einen anderen im Zustande der natürlichen Freiheit, folglich auch dem des beständigen Krieges betrachtet, teils das Recht zum Kriege, teils das im Kriege, teils das, einander zu nötigen, aus diesem Kriegszustande herauszugehen, mithin eine den beharrlichen Frieden gründende Verfassung, d.i. das Recht nach dem Kriege zur Aufgabe macht, und führt nur das Unterscheidende von dem des Naturzustandes einzelner Menschen oder Familien (im Verhältnis gegen einander) von dem der Völker bei sich, daß im Völkerrecht nicht bloß ein Verhältnis eines Staats gegen den anderen im ganzen, sondern auch einzelner Personen des einen gegen einzelne des anderen, imgleichen gegen den ganzen anderen Staat selbst in Betrachtung kommt; welcher Unterschied aber vom Recht einzelner im  bloßen Naturzustande nur solcher Bestimmungen bedarf, die sich aus dem Begriffe des letzteren leicht folgern lassen. 
	
	\subsection*{tg442.2.7} 
	\textbf{Source : }Die Metaphysik der Sitten/Erster Teil. Metaphysische Anfangsgründe der Rechtslehre/2. Teil. Das öffentliche Recht/2. Abschnitt. Das Völkerrecht\\  
	
	\noindent\textbf{Paragraphe : }Die Elemente des Völkerrechts sind: 1) daß Staaten, im äußeren \match{Verhältnis} gegen einander betrachtet, (wie gesetzlose Wilde) von Natur in einem nicht-rechtlichen Zustande sind; 2) daß dieser Zustand ein Zustand des Krieges (des Rechts des Stärkeren), wenn gleich nicht wirklicher Krieg und immerwährende wirkliche Befehdung (Hostilität) ist, welche (indem sie es beide nicht besser haben wollen), obzwar dadurch keinem von dem anderen unrecht geschieht, doch an sich selbst im höchsten Grade unrecht ist, und aus welchem die Staaten, welche einander benachbart sind, auszugehen verbunden sind; 3) daß ein Völkerbund, nach der Idee eines ursprünglichen gesellschaftlichen Vertrages, notwendig ist, sich zwar einander nicht in die einheimische Mißhelligkeiten derselben zu mischen, aber doch gegen Angriffe der äußeren zu schützen; 4) daß die Verbindung doch keine souveräne Gewalt (wie in einer bürgerlichen Verfassung), sondern nur eine Genossenschaft (Föderalität) enthalten müsse; eine Verbündung, die zu aller Zeit aufgekündigt werden kann, mithin von Zeit zu Zeit erneuert werden muß, – ein Recht, in subsidium eines anderen und ursprünglichen Rechts, den Verfall in den Zustand des wirklichen Krieges derselben untereinander von sich abzuwehren (foedus Amphictyonum). 
	
	\subsection*{tg444.2.3} 
	\textbf{Source : }Die Metaphysik der Sitten/Erster Teil. Metaphysische Anfangsgründe der Rechtslehre/Beschluß\\  
	
	\noindent\textbf{Paragraphe : }Nun spricht die moralisch-praktische Vernunft in uns ihr unwiderstehliches Veto aus: Es soll kein Krieg sein; weder der, welcher zwischen mir und dir im Naturzustande, noch zwischen uns als Staaten, die, obzwar innerlich im gesetzlichen, doch äußerlich (in \match{Verhältnis} gegen einander) im gesetzlosen Zustande sind; – denn das ist nicht die Art, wie jedermann sein Recht suchen soll. Also ist nicht mehr die Frage; ob der ewige Friede ein Ding oder Unding sei, und ob wir uns nicht in unserem theoretischen Urteile betrügen, wenn wir das erstere annehmen, sondern wir müssen so handeln, als ob das Ding sei, was vielleicht nicht ist, auf Begründung desselben, und diejenige Konstitution, die uns dazu die tauglichste scheint (vielleicht den Republikanism aller Staaten samt und sonders) hinwirken, um ihn herbei zu führen, und dem heillosen Kriegführen, worauf, als den Hauptzweck, bisher alle Staaten, ohne Ausnahme, ihre innere Anstalten gerichtet haben, ein Ende zu machen. Und, wenn das letztere, was die Vollendung dieser Absicht betrifft, auch immer ein frommer Wunsch bliebe, so betrügen wir uns doch gewiß nicht mit der Annahme der Maxime, dahin unablässig zu wirken; denn diese ist Pflicht; das moralische Gesetz aber in uns selbst für betrüglich anzunehmen, würde den Abscheu erregenden Wunsch hervorbringen,  lieber aller Vernunft zu entbehren, und sich, seinen Grundsätzen nach, mit den übrigen Tierklassen in einen gleichen Mechanism der Natur geworfen anzusehen. 
	
	\subsection*{tg447.2.4} 
	\textbf{Source : }Die Metaphysik der Sitten/Zweiter Teil. Metaphysische Anfangsgründe der Tugendlehre/Vorrede\\  
	
	\noindent\textbf{Paragraphe : }Wenn es über irgend einen Gegenstand eine Philosophie (System der Vernunfterkenntnis aus Begriffen) gibt, so muß es für diese Philosophie auch ein System reiner, von aller Anschauungsbedingung unabhängiger Vernunftbegriffe, d.i. eine Metaphysik geben. – Es fragt sich nur: ob es für jede praktische Philosophie, als Pflichtenlehre, mithin auch für die Tugendlehre (Ethik), auch metaphysischer Anfangsgründe bedürfe, um sie, als wahre Wissenschaft (systematisch), nicht bloß als Aggregat einzeln aufgesuchter Lehren (fragmentarisch) aufstellen zu können. – Von der reinen Rechtslehre wird niemand dies Bedürfnis bezweifeln; denn sie betrifft nur das Förmliche der nach Freiheitsgesetzen im äußeren \match{Verhältnis} einzuschränkenden Willkür; abgesehen von allem Zweck (als der Materie derselben). Die Pflichtenlehre ist also hier eine bloße Wissenslehre (doctrina scientiae).
	
	
	13
	
	
	
	\subsection*{tg450.2.2} 
	\textbf{Source : }Die Metaphysik der Sitten/Zweiter Teil. Metaphysische Anfangsgründe der Tugendlehre/Einleitung/II. Erörterung des Begriffs von einem Zwecke, der zugleich Pflicht ist\\  
	
	\noindent\textbf{Paragraphe : }Man kann sich das \match{Verhältnis} des Zwecks zur Pflicht auf zweierlei Art denken: entweder, von dem Zwecke ausgehend, die Maxime der pflichtmäßigen Handlungen, oder, umgekehrt, von dieser anhebend, den Zweck ausfindig zu machen, der zugleich Pflicht ist. – Die Rechtslehre geht auf dem ersten Wege. Es wird jedermanns freier Willkür überlassen, welchen Zweck er sich für seine Handlung setzen wolle. Die Maxime derselben aber ist a priori bestimmt: daß nämlich die Freiheit des Handelnden mit jedes anderen Freiheit nach einem allgemeinen Gesetz zusammen bestehen könne. 
	
	\subsection*{tg466.2.2} 
	\textbf{Source : }Die Metaphysik der Sitten/Zweiter Teil. Metaphysische Anfangsgründe der Tugendlehre/Einleitung/XVIII.\\  
	
	\noindent\textbf{Paragraphe : }Die Einteilung, welche die praktische Vernunft zu Gründung eines Systems ihrer Begriffe in einer Ethik entwirft (die architektonische), kann nun nach zweierlei Prinzipien, einzeln oder zusammen verbunden, gemacht werden: das eine, welches das subjektive \match{Verhältnis} der Verpflichteten zu dem Verpflichtenden, der Materie nach, das andere, welches das objektive Verhältnis der ethischen Gesetze zu den Pflichten überhaupt in einem System der Form nach vorstellt. – Die erste Einteilung ist die der Wesen, in Beziehung auf welche eine ethische Verbindlichkeit gedacht werden kann; die zweite wäre die der Begriffe der reinen ethisch-praktischen Vernunft; welche zu jener ihren Pflichten gehören, die also zur Ethik, nur so fern sie Wissenschaft sein soll, also zu der methodischen Zusammensetzung aller Sätze, welche nach der ersteren aufgefunden worden, erforderlich sind. 
	
	\subsection*{tg473.2.7} 
	\textbf{Source : }Die Metaphysik der Sitten/Zweiter Teil. Metaphysische Anfangsgründe der Tugendlehre/I. Ethische Elementarlehre/I. Teil. Von den Pflichten gegen sich selbst überhaupt/Erstes Buch. Von den vollkommenen Pflichten gegen sich selbst/Zweites Hauptstück. Die Pflicht des Menschen gegen sich selbst, bloß als einem moralischen Wesen/1. Abschnitt. Von der Pflicht des Menschen gegen sich selbst, als dem angebornen Richter über sich selbst\\  
	
	\noindent\textbf{Paragraphe : }Eine solche idealische Person (der autorisierte Gewissensrichter) muß ein Herzenskündiger sein; denn der Gerichtshof ist im Inneren des Menschen aufgeschlagen – zugleich muß er aber auch allverpflichtend, d.i. eine solche Person sein, oder als eine solche gedacht werden, in \match{Verhältnis} auf welche alle Pflichten überhaupt auch als ihre Gebote anzusehen sind; weil das Gewissen über alle freie Handlungen der innere Richter ist. – – Da nun ein solches moralisches Wesen zugleich alle Gewalt (im Himmel und auf Erden) haben muß, weil es sonst nicht (was doch zum Richteramt notwendig gehört) seinen Gesetzen den ihnen angemessenen Effekt verschaffen könnte, ein solches über alles machthabende moralische Wesen aber Gott heißt: so wird das Gewissen als subjektives Prinzip einer vor Gott seiner Taten wegen zu leistenden Verantwortung gedacht werden müssen; ja es wird der letztere Begriff (wenn gleich nur auf dunkele Art) in jenem moralischen Selbstbewußtsein jederzeit enthalten sein. 
	
	\subsection*{tg481.2.25} 
	\textbf{Source : }Die Metaphysik der Sitten/Zweiter Teil. Metaphysische Anfangsgründe der Tugendlehre/I. Ethische Elementarlehre/II. Teil. Von den Tugendpflichten gegen andere/Erstes Hauptstück. Von den Pflichten gegen andere, bloß als Menschen/Erster Abschnitt. Von der Liebespflicht gegen andere Menschen\\  
	
	\noindent\textbf{Paragraphe : }Die Maxime des Wohlwollens (die praktische Menschenliebe) ist aller Menschen Pflicht gegen einander; man mag diese nun liebenswürdig finden oder nicht, nach dem ethischen Gesetz der Vollkommenheit: Liebe deinen Nebenmenschen als dich selbst. – Denn alles moralisch-praktische \match{Verhältnis} gegen Menschen ist ein Verhältnis derselben in der Vorstellung der reinen Vernunft, d.i. der freien Handlungen nach Maximen, welche sich zur allgemeinen Gesetzgebung qualifizieren, die also nicht selbstsüchtig (ex solipsismo prodeuntes) sein können. Ich will jedes anderen Wohlwollen (benevolentiam) gegen mich; ich soll also auch gegen jeden anderen wohlwollend sein. Da aber alle andere außer mir nicht alle sein, mithin die Maxime nicht die Allgemeinheit eines Gesetzes an sich haben würde, welche doch zur Verpflichtung notwendig ist: so wird das Pflichtgesetz des Wohlwollens mich als Objekt desselben im Gebot der praktischen Vernunft mit begreifen: nicht, als ob ich dadurch verbunden würde, mich selbst zu lieben (denn das geschieht ohne das unvermeidlich, und dazu gibt's also keine Verpflichtung), sondern die gesetzgebende Vernunft, welche in ihrer Idee der Menschheit überhaupt die ganze Gattung (mich also mit) einschließt, nicht der Mensch, schließt als allgemeingesetzgebend mich in der Pflicht des wechselseitigen Wohlwollens nach dem Prinzip der Gleichheit alle andere neben mir mit ein, und erlaubt es dir, dir selbst wohlzuwollen, unter der Bedingung, daß du auch jedem anderen wohl willst; weil so allein deine Maxime (des Wohltuns) sich zu einer allgemeinen Gesetzgebung qualifiziert, als worauf alles Pflichtgesetz gegründet ist. 
	
	\subsection*{tg481.2.55} 
	\textbf{Source : }Die Metaphysik der Sitten/Zweiter Teil. Metaphysische Anfangsgründe der Tugendlehre/I. Ethische Elementarlehre/II. Teil. Von den Tugendpflichten gegen andere/Erstes Hauptstück. Von den Pflichten gegen andere, bloß als Menschen/Erster Abschnitt. Von der Liebespflicht gegen andere Menschen\\  
	
	\noindent\textbf{Paragraphe : }
	Dankbarkeit ist die Verehrung einer Person wegen einer uns erwiesenen Wohltat. Das Gefühl, was mit dieser Beurteilung verbunden ist, ist das der Achtung gegen den (ihn verpflichtenden) Wohltäter, da hingegen dieser gegen den Empfänger nur als im \match{Verhältnis} der Liebe betrachtet wird. – Selbst ein bloßes herzliches Wohlwollen des anderen, ohne physische Folgen, verdient den Namen einer Tugendpflicht; welches dann den Unterschied zwischen der tätigen und bloß affektionellen Dankbarkeit begründet. 
	
	\subsection*{tg481.2.7} 
	\textbf{Source : }Die Metaphysik der Sitten/Zweiter Teil. Metaphysische Anfangsgründe der Tugendlehre/I. Ethische Elementarlehre/II. Teil. Von den Tugendpflichten gegen andere/Erstes Hauptstück. Von den Pflichten gegen andere, bloß als Menschen/Erster Abschnitt. Von der Liebespflicht gegen andere Menschen\\  
	
	\noindent\textbf{Paragraphe : }Wenn von Pflichtgesetzen (nicht von Naturgesetzen) die Rede ist und zwar im äußeren \match{Verhältnis} der Menschen gegen einander, so betrachten wir uns in einer moralischen (intelligibelen) Welt, in welcher, nach der Analogie mit der physischen, die Verbindung vernünftiger Wesen (auf Erden) durch Anziehung und Abstoßung bewirkt wird. Vermöge des Prinzips der Wechselliebe sind sie angewiesen, sich einander beständig zu nähern, durch das der Achtung, die sie einander schuldig sind, sich im Abstande von einander zu erhalten, und, sollte eine dieser großen sittlichen Kräfte sinken: »so würde dann das Nichts (der Immoralität) mit aufgesperrtem Schlund der (moralischen) Wesen ganzes Reich, wie einen Tropfen Wasser trinken« (wenn ich mich hier der Worte Hallers, nur in einer andern Beziehung, bedienen darf). 
	
	\subsection*{tg481.2.85} 
	\textbf{Source : }Die Metaphysik der Sitten/Zweiter Teil. Metaphysische Anfangsgründe der Tugendlehre/I. Ethische Elementarlehre/II. Teil. Von den Tugendpflichten gegen andere/Erstes Hauptstück. Von den Pflichten gegen andere, bloß als Menschen/Erster Abschnitt. Von der Liebespflicht gegen andere Menschen\\  
	
	\noindent\textbf{Paragraphe : }b) Undankbarkeit gegen seinen Wohltäter, welche, wenn sie gar so weit geht, seinen Wohltäter zu hassen, 
	qualifizierte Undankbarkeit, sonst aber bloß Unerkenntlichkeit heißt, ist ein zwar im öffentlichen Urteile höchst verabscheutes Laster, gleichwohl ist der Mensch desselben wegen so berüchtigt, daß man es nicht für unwahrscheinlich hält, man könne sich durch erzeigte Wohltaten wohl gar einen Feind machen. – Der Grund der Möglichkeit eines solchen Lasters liegt in der mißverstandenen Pflicht gegen sich selbst, die Wohltätigkeit anderer, weil sie uns Verbindlichkeit gegen sie auferlegt, nicht zu bedürfen und aufzufordern, sondern lieber die Beschwerden des Lebens selbst zu ertragen, als andere damit zu belästigen, mithin dadurch bei ihnen in Schulden (Verpflichtung) zu kommen; weil wir dadurch auf die niedere Stufe des Beschützten gegen seinen Beschützer zu geraten fürchten; welches der echten Selbstschätzung (auf die Würde der Menschheit in seiner eigenen Person stolz zu sein) zuwider ist. Daher Dankbarkeit gegen die, die uns im Wohltun unvermeidlich zuvor kommen mußten (gegen Vorfahren im Angedenken, oder gegen Eltern), freigebig, die aber gegen Zeitgenossen nur kärglich, ja, um dieses \match{Verhältnis} der Ungleichheit unsichtbar zu machen, wohl gar das Gegenteil derselben bewiesen wird. – Dieses ist aber alsdann ein die Menschheit empörendes Laster, nicht bloß des Schadens wegen, den ein solches Beispiel Menschen überhaupt zuziehen muß, von fernerer Wohltätigkeit abzuschrecken (denn diese können mit echtmoralischer Gesinnung, eben in der Verschmähung alles solchen Lohns ihrem Wohltun nur einen desto größeren inneren moralischen Wert setzen): sondern weil die Menschenliebe hier gleichsam auf den Kopf gestellt, und der Mangel der Liebe gar in die Befugnis, den Liebenden zu hassen, verunedelt wird. 
	
	\subsection*{tg488.2.13} 
	\textbf{Source : }Die Metaphysik der Sitten/Zweiter Teil. Metaphysische Anfangsgründe der Tugendlehre/Beschluß. Die Religionslehre als Lehre der Pflichten gegen Gott liegt außerhalb den Grenzen der reinen Moralphilosophie\\  
	
	\noindent\textbf{Paragraphe : }Aber nicht allein eben so groß, sondern noch größer (weil das Prinzip einschränkend ist) ist der Anspruch, den die göttliche Gerechtigkeit, im Urteile unserer eigenen Vernunft, und zwar als strafende an uns macht. – Denn Belohnung (praemium, remuneratio gratuita) bezieht sich gar nicht auf Gerechtigkeit gegen Wesen, die lauter Pflichten und keine Rechte gegen das andere haben, sondern bloß auf Liebe und Wohltätigkeit (benignitas); – noch weniger kann ein Anspruch auf Lohn (merces) bei einem solchen Wesen statt finden, und eine belohnende Gerechtigkeit (iustitia brabeutica) ist im \match{Verhältnis} Gottes gegen Menschen ein Widerspruch. 
	
	\subsection*{tg488.2.14} 
	\textbf{Source : }Die Metaphysik der Sitten/Zweiter Teil. Metaphysische Anfangsgründe der Tugendlehre/Beschluß. Die Religionslehre als Lehre der Pflichten gegen Gott liegt außerhalb den Grenzen der reinen Moralphilosophie\\  
	
	\noindent\textbf{Paragraphe : }Es ist aber doch in der Idee einer Gerechtigkeitsausübung eines Wesens, was über allen Abbruch an seinen Zwecken erhaben ist, etwas, was sich mit dem \match{Verhältnis} des Menschen zu Gott nicht wohl vereinigen läßt: nämlich der Begriff einer Läsion, welche an dem unumschränkten und unerreichbaren Weltherrscher begangen werden könne; denn hier ist nicht von den Rechtsverletzungen, die Menschen gegen einander verüben und worüber Gott als strafender Richter entscheide, sondern von der Verletzung, die Gott selber und seinem Recht widerfahren solle, die Rede, wovon der Begriff transzendent ist, d.i. über den Begriff aller Strafgerechtigkeit, wovon wir irgend ein Beispiel aufstellen können (d.i. der unter Menschen), ganz hinaus liegt und überschwengliche Prinzipien enthält, die mit denen, welche wir in Erfahrungsfällen gebrauchen würden, gar nicht in Zusammenstimmung  gebracht werden können, folglich für unsere praktische Vernunft gänzlich leer sind. 
	
	\subsection*{tg488.2.17} 
	\textbf{Source : }Die Metaphysik der Sitten/Zweiter Teil. Metaphysische Anfangsgründe der Tugendlehre/Beschluß. Die Religionslehre als Lehre der Pflichten gegen Gott liegt außerhalb den Grenzen der reinen Moralphilosophie\\  
	
	\noindent\textbf{Paragraphe : }Man sieht hieraus: daß in der Ethik, als reiner praktischer Philosophie der inneren Gesetzgebung, nur die moralischen Verhältnisse des Menschen gegen den Menschen für uns begreiflich sind: was aber zwischen Gott und dem Menschen hierüber für ein \match{Verhältnis} obwalte, die Grenzen derselben gänzlich übersteigt und uns schlechterdings unbegreiflich ist; wodurch dann bestätigt wird, was oben behauptet ward: daß die Ethik sich nicht über die Grenzen der wechselseitigen Menschenpflichten erweitern könne. 
	
	\subsection*{tg489.2.23} 
	\textbf{Source : }Die Metaphysik der Sitten/Fußnoten\\  
	
	\noindent\textbf{Paragraphe : }
	
	10 Ich sage hier auch nicht: eine Person als die meinige (mit dem Adjektiv) sondern als das Meine (to meum, mit dem Substantiv) zu haben. Denn ich kann sagen: dieser ist mein Vater, das bezeichnet nur mein physisches \match{Verhältnis} (der Verknüpfung) zu ihm überhaupt. Z.B. »ich habe einen Vater«. Aber ich kann nicht sagen: »ich habe ihn als das Meine«. Sage ich aber: mein Weib, so bedeutet dieses ein besonderes, nämlich rechtliches, Verhältnis des Besitzers zu einem Gegenstande (wenn es auch eine Person wäre), als Sache. Besitz (physischer) aber ist die Bedingung der Möglichkeit der Handhabung (manipulatio) eines Dinges als einer Sache; wenn dieses gleich, in einer anderen Beziehung, zugleich als Person behandelt werden muß. 
	
	\subsection*{tg489.2.46} 
	\textbf{Source : }Die Metaphysik der Sitten/Fußnoten\\  
	
	\noindent\textbf{Paragraphe : }
	
	21 Die zwiefache Persönlichkeit, in welcher der Mensch, der sich im Gewissen anklagt und richtet, sich selbst denken muß: dieses doppelte Selbst, einerseits vor den Schranken eines Gerichtshofes, der doch ihm selbst anvertraut ist, zitternd stehen zu müssen, anderseits aber das Richteramt aus angeborener Autorität selbst in Händen zu haben, bedarf einer Erläuterung, damit nicht die Vernunft mit sich selbst gar in Widerspruch gerate. – Ich, der Kläger und doch auch Angeklagter, bin eben derselbe Mensch (numero idem), aber, als Subjekt der moralischen, von dem Begriffe der Freiheit ausgehenden, Gesetzgebung, wo der Mensch einem Gesetz Untertan ist, das er sich selbst gibt (homo noumenon), ist er als ein anderer als der mit Vernunft begabte Sinnenmensch (specie diversus), aber nur in praktischer Rücksicht, zu betrachten – denn über das Kausal-\match{Verhältnis} des Intelligibelen zum Sensibelen gibt es keine Theorie – und diese spezifische Verschiedenheit ist die der Fakultäten des Menschen (der oberen und unteren), die ihn charakterisieren. Der erstere ist der Ankläger, dem entgegen ein rechtlicher Beistand des Verklagten (Sachwalter desselben) bewilligt ist. Nach Schließung der Akten tut der innere Richter, als machthabende Person, den Ausspruch über Glückseligkeit oder Elend, als moralische Folgen der Tat; in welcher Qualität wir dieser ihre Macht (als Weltherrschers) durch unsere Vernunft nicht weiter verfolgen, sondern nur das unbedingte iubeo oder veto verehren können. 
	
	\unnumberedsection{Versuch (1)} 
	\subsection*{tg443.2.4} 
	\textbf{Source : }Die Metaphysik der Sitten/Erster Teil. Metaphysische Anfangsgründe der Rechtslehre/2. Teil. Das öffentliche Recht/3. Abschnitt. Das Weltbürgerrecht\\  
	
	\noindent\textbf{Paragraphe : }Diese Vernunftidee einer friedlichen, wenngleich noch nicht freundschaftlichen, durchgängigen Gemeinschaft aller Völker auf Erden, die untereinander in wirksame Verhältnisse kommen können, ist nicht etwa philanthropisch (ethisch), sondern ein rechtliches Prinzip. Die Natur hat sie alle zusammen (vermöge der Kugelgestalt ihres Aufenthalts, als globus terraqueus) in bestimmte Grenzen eingeschlossen, und, da der Besitz des Bodens, worauf der Erdbewohner leben kann, immer nur als Besitz von einem Teil eines bestimmten Ganzen, folglich als ein solcher, auf den jeder derselben ursprünglich ein Recht hat, gedacht werden kann: so stehen alle Völker ursprünglich in einer Gemeinschaft  des Bodens, nicht aber der rechtlichen Gemeinschaft des Besitzes (communio) und hiemit des Gebrauchs, oder des Eigentums an denselben, sondern der physischen möglichen Wechselwirkung (commercium), d.i. in einem durchgängigen Verhältnisse, eines zu allen an deren, sich zum Verkehr untereinander anzubieten, und haben ein Recht, den \match{Versuch} mit demselben zu machen, ohne daß der Auswärtige ihm darum als einen Feind zu begegnen berechtigt wäre. – Dieses Recht, so fern es auf die mögliche Vereinigung aller Völker, in Absicht auf gewisse allgemeine Gesetze ihres möglichen Verkehrs, geht, kann das weltbürgerliche (ius cosmopoliticum) genannt werden. 
	
	\unnumberedsection{Warme (1)} 
	\subsection*{tg481.2.69} 
	\textbf{Source : }Die Metaphysik der Sitten/Zweiter Teil. Metaphysische Anfangsgründe der Tugendlehre/I. Ethische Elementarlehre/II. Teil. Von den Tugendpflichten gegen andere/Erstes Hauptstück. Von den Pflichten gegen andere, bloß als Menschen/Erster Abschnitt. Von der Liebespflicht gegen andere Menschen\\  
	
	\noindent\textbf{Paragraphe : }
	Mitfreude und Mitleid (sympathia moralis) sind zwar sinnliche Gefühle einer (darum ästhetisch zu nennenden) Lust oder Unlust an dem Zustande des Vergnügens so wohl als Schmerzens anderer (Mitgefühl, teilnehmende Empfindung), wozu schon die Natur in den Menschen die Empfänglichkeit gelegt hat. Aber diese als Mittel zu Beförderung des tätigen und vernünftigen Wohlwollens zu gebrauchen, ist noch eine besondere, obzwar nur bedingte, Pflicht, unter dem Namen der Menschlichkeit (humanitas); weil hier der Mensch nicht bloß als vernünftiges Wesen, sondern  auch als mit Vernunft begabtes Tier betrachtet wird. Diese kann nun in dem Vermögen und Willen, sich einander in Ansehung seiner Gefühle mitzuteilen (humanitas practica), oder bloß in der Empfänglichkeit für das gemeinsame Gefühl des Vergnügens oder Schmerzens (humanitas aesthetica), was die Natur selbst gibt, gesetzt werden. Das erstere ist frei und wird daher teilnehmend genannt (communio sentiendi liberalis) und gründet sich auf praktische Vernunft; das zweite ist unfrei (communio sentiendi illiberalis, servilis) und kann mitteilend (wie die der \match{Wärme} oder ansteckender Krankheiten), auch Mitleidenschaft heißen; weil sie sich unter nebeneinander lebenden Menschen natürlicher Weise verbreitet. Nur zu dem ersten gibt's Verbindlichkeit. 
	
	\unnumberedsection{Widerstand (10)} 
	\subsection*{tg441.2.43} 
	\textbf{Source : }Die Metaphysik der Sitten/Erster Teil. Metaphysische Anfangsgründe der Rechtslehre/2. Teil. Das öffentliche Recht/1. Abschnitt. Das Staatsrecht\\  
	
	\noindent\textbf{Paragraphe : }Ja es kann auch selbst in der Konstitution kein Artikel enthalten sein, der es einer Gewalt im Staat möglich machte, sich, im Fall der Übertretung der Konsti tutionalgesetze durch den obersten Befehlshaber, ihm zu widersetzen, mithin ihn einzuschränken. Denn der, welcher die Staatsgewalt einschränken soll, muß doch mehr, oder wenigstens gleiche Macht haben, als derjenige, welcher eingeschränkt wird, und, als ein rechtmäßiger Gebieter, der den Untertanen befähle, sich zu widersetzen, muß er sie auch schützen können,  und in jedem vorkommenden Fall rechtskräftig urteilen, mithin öffentlich den \match{Widerstand} befehligen können. Alsdann ist aber nicht jener, sondern dieser der oberste Befehlshaber; welches sich widerspricht. Der Souverän verfährt alsdann durch seinen Minister zugleich als Regent, mithin despotisch, und das Blendwerk, das Volk durch die Deputierte desselben die einschränkende Gewalt vorstellen zu lassen (da es eigentlich nur die gesetzgebende hat), kann die Despotie nicht so verstecken, daß sie aus den Mitteln, deren sich der Minister bedient, nicht hervorblickte. Das Volk, das durch seine Deputierte (im Parlament) repräsentiert wird, hat an diesen Gewährsmännern seiner Freiheit und Rechte Leute, die für sich und ihre Familien, und dieser ihre vom Minister abhängigen Versorgung, in Armeen, Flotte und Zivilämtern, lebhaft interessiert sind, und die (statt des Widerstandes gegen die Anmaßung der Regierung, dessen öffentliche Ankündigung ohnedem eine dazu schon vorbereitete Einhelligkeit im Volk bedarf, die aber im Frieden nicht erlaubt sein kann) vielmehr immer bereit sind, sich selbst der Regierung in die Hände zu spielen. – Also ist die sogenannte gemäßigte Staatsverfassung, als Konstitution des innern Rechts des Staats, ein Unding, und, anstatt zum Recht zu gehören, nur ein Klugheitsprinzip, um, so viel als möglich, dem mächtigen Übertreter der Volksrechte seine willkürliche Einflüsse auf die Regierung nicht zu erschweren, sondern unter dem Schein einer dem Volk verstatteten Opposition zu bemänteln. 
	
	\subsection*{tg441.2.44} 
	\textbf{Source : }Die Metaphysik der Sitten/Erster Teil. Metaphysische Anfangsgründe der Rechtslehre/2. Teil. Das öffentliche Recht/1. Abschnitt. Das Staatsrecht\\  
	
	\noindent\textbf{Paragraphe : }Wider das gesetzgebende Oberhaupt des Staats gibt es also keinen rechtmäßigen \match{Widerstand} des Volks; denn nur durch Unterwerfung unter seinen allgemein-gesetzgebenden Willen ist ein rechtlicher Zustand möglich; also kein Recht des Aufstandes (seditio), noch weniger des Aufruhrs (rebellio), am allerwenigsten gegen ihn, als einzelne Person (Monarch), unter dem Verwände des Mißbrauchs seiner Gewalt (tyrannis), Vergreifung an seiner Person, ja an seinem Leben (monarchomachismus sub specie tyrannicidii). Der geringste Versuch hiezu ist Hochverrat (proditio eminens), und der Verräter dieser Art kann als einer, der sein 
	Vaterland umzubringen versucht (parricida), nicht minder als mit dem Tode bestraft werden. – – Der Grund der Pflicht des Volks, einen, selbst den für unerträglich ausgegebenen Mißbrauch der obersten Gewalt dennoch zu ertragen, liegt darin: daß sein Widerstand wider die höchste Gesetzgebung selbst niemals anders, als gesetzwidrig, ja als die ganze gesetzliche Verfassung zernichtend gedacht werden muß. Denn, um zu demselben befugt zu sein, müßte ein öffentliches Gesetz vorhanden sein, welches diesen Widerstand des Volks erlaubte, d.i. die oberste Gesetzgebung enthielte eine Bestimmung in sich, nicht die oberste zu sein, und das Volk, als Untertan, in einem und demselben Urteile zum Souverän über den zu machen, dem es untertänig ist; welches sich widerspricht und wovon der Widerspruch durch die Frage alsbald in die Augen fällt: wer denn in diesem Streit zwischen Volk und Souverän Richter sein sollte (denn es sind rechtlich betrachtet doch immer zwei verschiedene moralische Personen); wo sich dann zeigt, daß das erstere es in seiner eigenen Sache sein will.
	
	
	8
	
	
	
	\subsection*{tg441.2.45} 
	\textbf{Source : }Die Metaphysik der Sitten/Erster Teil. Metaphysische Anfangsgründe der Rechtslehre/2. Teil. Das öffentliche Recht/1. Abschnitt. Das Staatsrecht\\  
	
	\noindent\textbf{Paragraphe : }
	Eine Veränderung der (fehlerhaften) Staatsverfassung, die wohl bisweilen nötig sein mag – kann also nur vom Souverän selbst durch Reform, aber nicht vom Volk, mithin durch Revolution verrichtet werden, und, wenn sie geschieht, so kann jene nur die ausübende Gewalt, nicht die gesetzgebende, treffen. – In einer Staatsverfassung, die so beschaffen ist, daß das Volk durch seine Repräsentanten (im Parlament) jener und dem Repräsentanten derselben (dem Minister) gesetzlich widerstehen kann – welche dann eine eingeschränkte Verfassung heißt – ist gleichwohl kein aktiver \match{Widerstand} (der willkürlichen Verbindung des Volks, die Regierung zu einem gewissen tätigen Verfahren zu zwingen, mithin selbst einen Akt der ausübenden Gewalt zu begehen), sondern nur ein negativer Widerstand, d.i. Weigerung des Volks (im Parlament) und erlaubt jener, in den Forderungen, die sie zur Staatsverwaltung nötig zu haben vorgibt, nicht immer zu willfahren; vielmehr wenn das letztere geschähe, so wäre es ein sicheres Zeichen, daß das Volk verderbt, seine Repräsentanten erkäuflich  und das Oberhaupt in der Regierung durch seinen Minister despotisch, dieser selbst aber ein Verräter des Volks sei. 
	
	\subsection*{tg445.2.74} 
	\textbf{Source : }Die Metaphysik der Sitten/Erster Teil. Metaphysische Anfangsgründe der Rechtslehre/Anhang erläutender Bemerkungen zu den metaphysischen Anhangsgründen der Rechtslehre\\  
	
	\noindent\textbf{Paragraphe : }Die Idee einer Staatsverfassung überhaupt, welche zugleich absolutes Gebot der nach Rechtsbegriffen urteilenden praktischen Vernunft für ein jedes Volk ist, ist heilig und unwiderstehlich; und, wenn gleich die Organisation des Staats durch sich selbst fehlerhaft wäre, so kann doch keine subalterne Gewalt in demselben dem gesetzgebenden Oberhaupte desselben tätlichen \match{Widerstand} entgegensetzen, sondern die ihm anhängenden Gebrechen müssen durch Reformen, die er an sich selbst verrichtet, allmählich gehoben werden; weil sonst bei einer entgegengesetzten Maxime des Untertans (nach eigenmächtiger Willkür zu verfahren) eine gute Verfassung selbst nur durch blinden Zufall zu Stande kommen kann. – Das Gebot: »Gehorchet der Obrigkeit, die Gewalt über euch hat«, grübelt nicht nach, wie sie zu dieser Gewalt gekommen sei (um sie allenfalls zu untergraben); denn die, welche schon da ist, unter welcher ihr lebt, ist schon im Besitz der Gesetzgebung, über die ihr zwar öffentlich vernünfteln, euch aber selbst nicht zu widerstrebenden Gesetzgebern aufwerfen könnt. 
	
	\subsection*{tg445.2.75} 
	\textbf{Source : }Die Metaphysik der Sitten/Erster Teil. Metaphysische Anfangsgründe der Rechtslehre/Anhang erläutender Bemerkungen zu den metaphysischen Anhangsgründen der Rechtslehre\\  
	
	\noindent\textbf{Paragraphe : }Unbedingte Unterwerfung des Volkswillens (der an sich unvereinigt, mithin gesetzlos ist), unter einem souveränen (alle durch Ein Gesetz vereinigenden) Willen, ist Tat, die nur durch Bemächtigung der obersten Gewalt anheben kann, und so zuerst ein öffentliches Recht begründet. – Gegen diese Machtvollkommenheit noch einen \match{Widerstand} zu erlauben (der jene oberste Gewalt einschränkete), heißt sich selbst widersprechen; denn alsdann wäre jene (welcher widerstanden werden  darf) nicht die gesetzliche oberste Gewalt, die zuerst bestimmt, was öffentlich recht sein soll oder nicht – und dieses Prinzip liegt schon a priori in der Idee einer Staatsverfassung überhaupt, d.i. in einem Begriffe der praktischen Vernunft; dem zwar adäquat kein Beispiel in der Erfahrung untergelegt werden kann, dem aber auch, als Norm keine widersprechen muß. 
	
	\subsection*{tg449.2.2} 
	\textbf{Source : }Die Metaphysik der Sitten/Zweiter Teil. Metaphysische Anfangsgründe der Tugendlehre/Einleitung/I. Erörterung des Begriffs einer Tugendlehre\\  
	
	\noindent\textbf{Paragraphe : }Der Pflichtbegriff ist an sich schon der Begriff von einer Nötigung (Zwang) der freien Willkür durchs Gesetz; dieser Zwang mag nun ein äußerer oder ein Selbstzwang sein. Der moralische Imperativ verkündigt, durch seinen kategorischen Ausspruch (das unbedingte Sollen) diesen Zwang, der also nicht auf vernünftige Wesen überhaupt (deren es etwa auch heilige geben könnte), sondern auf Menschen als vernünftige Naturwesen geht, die dazu unheilig genug sind, daß sie die Lust wohl anwandeln kann, das moralische Gesetz, ob sie gleich dessen Ansehen selbst anerkennen, doch zu übertreten und, selbst wenn sie es befolgen, es dennoch ungern (mit \match{Widerstand} ihrer Neigung) zu tun, als worin der Zwang eigentlich besteht.
	
	
	14
	– 
	
	\subsection*{tg449.2.5} 
	\textbf{Source : }Die Metaphysik der Sitten/Zweiter Teil. Metaphysische Anfangsgründe der Tugendlehre/Einleitung/I. Erörterung des Begriffs einer Tugendlehre\\  
	
	\noindent\textbf{Paragraphe : }Nun ist das Vermögen und der überlegte Vorsatz, einem starken, aber ungerechten Gegner \match{Widerstand} zu tun, die Tapferkeit (fortitudo) und, in Ansehung des Gegners der sittlichen Gesinnung in uns, Tugend (virtus, fortitudo moralis). Also ist die allgemeine Pflichtenlehre in dem Teil, der nicht die äußere Freiheit, sondern die innere unter Gesetze bringt, eine Tugendlehre. 
	
	\subsection*{tg458.2.2} 
	\textbf{Source : }Die Metaphysik der Sitten/Zweiter Teil. Metaphysische Anfangsgründe der Tugendlehre/Einleitung/X. Das oberste Prinzip der Rechtslehre war analytisch; das der Tugendlehre ist synthetisch\\  
	
	\noindent\textbf{Paragraphe : }Daß der äußere Zwang, so fern dieser ein dem Hindernisse der nach allgemeinen Gesetzen zusammenstimmenden äußeren Freiheit entgegengesetzter \match{Widerstand} (ein Hindernis des Hindernisses derselben) ist, mit Zwecken überhaupt zusammen bestehen könne, ist, nach dem Satz des Widerspruchs, klar und ich darf nicht über den Begriff der Freiheit hinausgehen, um ihn einzusehen; der Zweck, den ein jeder hat, mag sein welcher er wolle. – Also ist das oberste Rechtsprinzip ein analytischer Satz. 
	
	\subsection*{tg489.2.27} 
	\textbf{Source : }Die Metaphysik der Sitten/Fußnoten\\  
	
	\noindent\textbf{Paragraphe : }
	
	12 In jeder Bestrafung liegt etwas das Ehrgefühl des Angeklagten (mit Recht) Kränkendes; weil sie einen bloßen einseitigen Zwang enthält und so an ihm die Würde eines Staatsbürgers, als eines solchen, in einem besonderen Fall wenigstens suspendiert ist: Da er einer äußeren Pflicht unterworfen wird, der er seiner seits keinen \match{Widerstand} entgegen setzen darf. Der Vornehme und Reiche, der auf den Beutel geklopft wird, fühlt mehr seine Erniedrigung, sich unter den Willen des geringeren Mannes beugen zu müssen, als den Geldverlust. Die Strafgerechtigkeit (iustitia punitiva), da nämlich das Argument der Strafbarkeit moralisch ist (quia peccatum est), muß hier von der Strafklugheit, da es bloß pragmatisch ist (ne peccetur) und sich auf Erfahrung von dem gründet, was am stärksten wirkt, Verbrechen abzuhalten, unterschieden werden, und hat in der Topik der Rechtsbegriffe einen ganz anderen Ort, locus iusti, nicht des Conducibilis, oder des Zuträglichen in gewisser Absicht noch auch den des bloßen Honesti, dessen Ort in der Ethik aufgesucht werden muß. 
	
	\subsection*{tg489.2.31} 
	\textbf{Source : }Die Metaphysik der Sitten/Fußnoten\\  
	
	\noindent\textbf{Paragraphe : }
	
	14 Der Mensch aber findet sich doch als moralisches Wesen zugleich, wenn er sich objektiv, wozu er durch seine reine praktische Vernunft bestimmt ist, (nach der Menschheit in seiner eigenen Person) betrachtet, heilig genug, um das innere Gesetz ungern zu übertreten; denn es gibt keinen so verruchten Menschen, der bei dieser Übertretung in sich nicht einen \match{Widerstand} fühlete und eine Verabscheuung seiner selbst, bei der er sich selbst Zwang antun muß. – Das Phänomen nun: daß der Mensch auf diesem Scheidewege (wo die schöne Fabel den Herkules zwischen Tugend und Wohllust hinstellt) mehr Hang zeigt, der Neigung als dem Gesetz Gehör zu geben, zu erklären ist unmöglich: weil wir, was geschieht, nur erklären können, indem wir es von einer Ursache nach Gesetzen der Natur ableiten; wobei wir aber die Willkür nicht als frei denken würden. – Dieser wechselseitig entgegengesetzte Selbstzwang aber und die Unvermeidlichkeit desselben gibt doch die unbegreifliche Eigenschaft der Freiheit selbst zu erkennen. 
	
	\unnumberedsection{Wirkung (19)} 
	\subsection*{tg430.2.31} 
	\textbf{Source : }Die Metaphysik der Sitten/Erster Teil. Metaphysische Anfangsgründe der Rechtslehre/Einleitung in die Metaphysik der Sitten\\  
	
	\noindent\textbf{Paragraphe : }Auf diesem (in praktischer Rücksicht) positiven Begriffe der Freiheit gründen sich unbedingte praktische Gesetze, welche moralisch heißen, die in Ansehung unser, deren Willkür sinnlich affiziert und so dem reinen Willen nicht von selbst angemessen, sondern oft widerstrebend ist, Imperativen (Gebote oder Verbote) und zwar kategorische (unbedingte) Imperativen sind, wodurch sie sich von den technischen (den Kunst-Vorschriften), als die jederzeit nur bedingt gebieten, unterscheiden, nach denen gewisse Handlungen erlaubt oder unerlaubt, d.i. moralisch möglich oder unmöglich, einige derselben aber, oder ihr Gegenteil moralisch notwendig, d.i. verbindlich sind, woraus dann für jene der Begriff einer Pflicht entspringt, deren Befolgung oder Übertretung zwar auch mit einer Lust oder Unlust von besonderer Art (der eines moralischen Gefühls) verbunden ist, auf welche wir aber (weil sie nicht den Grund der praktischen Gesetze, sondern nur die subjektive \match{Wirkung} im Gemüt bei der Bestimmung unserer Willkür durch jene betreffen und (ohne jener ihrer Gültigkeit oder Einflusse objektiv, d.i. im Urteil der Vernunft, etwas hinzuzutun oder zu benehmen) nach Verschiedenheit der Subjekte verschieden sein kann) in praktischen Gesetzen der Vernunft gar nicht Rücksicht nehmen. 
	
	\subsection*{tg430.2.38} 
	\textbf{Source : }Die Metaphysik der Sitten/Erster Teil. Metaphysische Anfangsgründe der Rechtslehre/Einleitung in die Metaphysik der Sitten\\  
	
	\noindent\textbf{Paragraphe : }
	Tat heißt eine Handlung, sofern sie unter Gesetzen der Verbindlichkeit steht, folglich auch, sofern das Subjekt in derselben nach der Freiheit seiner Willkür betrachtet wird. Der Handelnde wird durch einen solchen Akt als Urheber der \match{Wirkung} betrachtet, und diese, zusamt der Handlung selbst, können ihm zugerechnet werden, wenn man vorher das Gesetz kennt, kraft welches auf ihnen eine Verbindlichkeit ruhet. 
	
	\subsection*{tg430.2.4} 
	\textbf{Source : }Die Metaphysik der Sitten/Erster Teil. Metaphysische Anfangsgründe der Rechtslehre/Einleitung in die Metaphysik der Sitten\\  
	
	\noindent\textbf{Paragraphe : }Mit dem Begehren oder Verabscheuen ist erstlich jederzeit Lust oder Unlust, deren Empfänglichkeit man Gefühl nennt, verbunden, aber nicht immer umgekehrt. Denn es kann eine Lust geben, welche mit gar keinem Begehren des Gegenstandes, sondern mit der bloßen Vorstellung, die man sich von einem Gegenstande macht (gleichgültig, ob das Objekt derselben existiere oder nicht), schon verknüpft ist. Auch geht, zweitens, nicht immer die Lust oder Unlust an dem Gegenstande des Begehrens vor dem Begehren vorher und darf nicht allemal als Ursache, sondern kann auch als \match{Wirkung} desselben angesehen werden. 
	
	\subsection*{tg430.2.6} 
	\textbf{Source : }Die Metaphysik der Sitten/Erster Teil. Metaphysische Anfangsgründe der Rechtslehre/Einleitung in die Metaphysik der Sitten\\  
	
	\noindent\textbf{Paragraphe : }Man kann die Lust, welche mit dem Begehren (des Gegenstandes, dessen Vorstellung das Gefühl so affiziert) notwendig verbunden ist, praktische Lust nennen: sie mag nun Ursache oder \match{Wirkung} vom Begehren sein. Dagegen würde man die Lust, die mit dem Begehren des Gegenstandes nicht notwendig verbunden ist, die also im Grunde nicht eine Lust an der Existenz des Objekts der Vorstellung ist, sondern bloß an der Vorstellung allein haftet, bloß kontemplative Lust oder untätiges Wohlgefallen nennen können. Das Gefühl der letztern Art von Lust nennen wir Geschmack. Von diesem wird also in einer praktischen Philosophie, nicht als von einem einheimischen Begriffe, sondern allenfalls nur episodisch die Rede sein. Was aber die praktische Lust betrifft, so wird die Bestimmung des Begehrungsvermögens, vor welcher diese Lust, als Ursache, notwendig vorhergehen muß, im engen Verstande Begierde, die habituelle Begierde aber Neigung heißen, und, weil die Verbindung der Lust mit dem Begehrungsvermögen, sofern diese Verknüpfung durch den Verstand nach einer allgemeinen Regel (allenfalls auch nur für das Subjekt) gültig zu sein geurteilt wird, Interesse heißt, so wird die praktische Lust in diesem Falle ein Interesse der Neigung, dagegen wenn die Lust nur auf eine vorhergehende Bestimmung des Begehrungsvermögens folgen kann, so wird sie eine intellektuelle Lust und das Interesse an dem Gegenstande ein Vernunftinteresse genannt werden müssen; denn wäre das Interesse sinnlich und nicht bloß  auf reine Vernunftprinzipien gegründet, so müßte Empfindung mit Lust verbunden sein und so das Begehrungsvermögen bestimmen können. Obgleich, wo ein bloß reines Vernunftinteresse angenommen werden muß, ihm kein Interesse der Neigung untergeschoben werden kann, so können wir doch, um dem Sprachgebrauche gefällig zu sein, einer Neigung, selbst zu dem, was nur Objekt einer intellektuellen Lust sein kann, ein habituelles Begehren aus reinem Vernunftinteresse einräumen, welche alsdenn aber nicht die Ursache, sondern die Wirkung des letztern Interesse sein würde, und die wir die sinnenfreie Neigung (propensio intellectualis) nennen könnten. 
	
	\subsection*{tg431.2.20} 
	\textbf{Source : }Die Metaphysik der Sitten/Erster Teil. Metaphysische Anfangsgründe der Rechtslehre/Einleitung in die Rechtslehre\\  
	
	\noindent\textbf{Paragraphe : }Der Widerstand, der dem Hindernisse einer \match{Wirkung} entgegengesetzt wird, ist eine Beförderung dieser Wirkung und stimmt mit ihr zusammen. Nun ist alles, was Unrecht ist, ein Hindernis der Freiheit nach allgemeinen Gesetzen; der Zwang aber ist ein Hindernis oder Widerstand, der der Freiheit geschieht. Folglich: wenn ein gewisser Gebrauch der Freiheit selbst ein Hindernis der Freiheit nach allgemeinen Gesetzen (d.i. unrecht) ist, so ist der Zwang, der diesem entgegengesetzt wird, als Verhinderung eines Hindernisses der Freiheit mit der Freiheit nach allgemeinen Gesetzen  zusammen stimmend, d.i. recht: mithin ist mit dem Rechte zugleich eine Befugnis, den, der ihm Abbruch tut, zu zwingen, nach dem Satze des Widerspruchs verknüpft. 
	
	\subsection*{tg431.2.39} 
	\textbf{Source : }Die Metaphysik der Sitten/Erster Teil. Metaphysische Anfangsgründe der Rechtslehre/Einleitung in die Rechtslehre\\  
	
	\noindent\textbf{Paragraphe : }Es ist klar: daß diese Behauptung nicht objektiv, nach dem, was ein Gesetz vorschreiben, sondern bloß subjektiv, wie vor Gericht die Sentenz gefället werden würde, zu verstehen sei. Es kann nämlich kein Strafgesetz geben, welches demjenigen den Tod zuerkennete, der im Schiffbruche, mit einem andern in gleicher Lebensgefahr schwebend, diesen von dem Brette, worauf er sich gerettet hat, wegstieße, um sich selbst zu retten. Denn die durchs Gesetz angedrohete Strafe könnte doch nicht größer sein, als die des Verlusts des Lebens des ersteren. Nun kann ein solches Strafgesetz die beabsichtigte \match{Wirkung} gar nicht haben; denn die Bedrohung mit einem Übel, was noch ungewiß ist (dem Tode durch den richterlichen Ausspruch), kann die Furcht vor dem Übel, was gewiß ist (nämlich dem Ersaufen), nicht überwiegen. Also ist die Tat der gewalttätigen Selbsterhaltung nicht etwa als unsträflich (inculpabile), sondern nur als unstrafbar (inpunibile) zu beurteilen und diese subjektive Straflosigkeit wird, durch eine wunderliche Verwechselung, von den Rechtslehrern für eine objektive (Gesetzmäßigkeit) gehalten. 
	
	\subsection*{tg441.2.98} 
	\textbf{Source : }Die Metaphysik der Sitten/Erster Teil. Metaphysische Anfangsgründe der Rechtslehre/2. Teil. Das öffentliche Recht/1. Abschnitt. Das Staatsrecht\\  
	
	\noindent\textbf{Paragraphe : }Die Staatsformen sind nur der Buchstabe (littera) der ursprünglichen Gesetzgebung im bürgerlichen Zustande, und sie mögen also bleiben, so lange sie, als zum Maschinenwesen der Staatsverfassung gehörend, durch alte und lange Gewohnheit (also nur subjektiv) für notwendig gehalten  werden. Aber der Geist jenes ursprünglichen Vertrages (anima pacti originarii) enthält die Verbindlichkeit der konstituierenden Gewalt, die Regierungsart jener Idee angemessen zu machen, und so sie, wenn es nicht auf einmal geschehen kann, allmählich und kontinuierlich dahin zu verändern, daß sie mit der einzig rechtmäßigen Verfassung, nämlich der einer reinen Republik, ihrer \match{Wirkung} nach zusammenstimme, und jene alte empirische (statutarische) Formen, welche bloß die Untertänigkeit des Volks zu bewirken dienten, sich in die ursprüngliche (rationale) auflösen, welche allein die Freiheit zum Prinzip, ja zur Bedingung alles Zwanges macht, der zu einer rechtlichen Verfassung, im eigentlichen Sinne des Staats, erforderlich ist, und dahin auch dem Buchstaben nach endlich führen wird. – Dies ist die einzige bleibende Staatsverfassung, wo das Gesetz selbstherrschend ist, und an keiner besonderen Person hängt; der letzte Zweck alles öffentlichen Rechts, der Zustand, in welchem allein jedem das Seine peremtorisch zugeteilt werden kann; indessen, daß, so lange jene Staatsformen dem Buchstaben nach eben so viel verschiedene, mit der obersten Gewalt bekleidete, moralische Personen vorstellen sollen, nur ein provisorisches inneres Recht, und kein absolut-rechtlicher Zustand, der bürgerlichen Gesellschaft zugestanden werden kann. 
	
	\subsection*{tg450.2.7} 
	\textbf{Source : }Die Metaphysik der Sitten/Zweiter Teil. Metaphysische Anfangsgründe der Tugendlehre/Einleitung/II. Erörterung des Begriffs von einem Zwecke, der zugleich Pflicht ist\\  
	
	\noindent\textbf{Paragraphe : }Tugend ist aber auch nicht bloß als Fertigkeit und (wie die Preisschrift des Hofpred. Cochius sich ausdrückt) für eine lange, durch Übung erworbene, Gewohnheit moralisch-guter Handlungen zu erklären und zu würdigen. Denn wenn diese nicht eine \match{Wirkung} überlegter, fester und immer mehr geläuterter Grundsätze ist, so ist sie, wie ein jeder andere Mechanism aus technisch-praktischer Vernunft, weder auf alle Fälle gerüstet, noch vor der Veränderung, die neue Anlockungen bewirken können, hinreichend gesichert. 
	
	\subsection*{tg451.2.2} 
	\textbf{Source : }Die Metaphysik der Sitten/Zweiter Teil. Metaphysische Anfangsgründe der Tugendlehre/Einleitung/III. Von dem Grunde, sich einen Zweck, der zugleich Pflicht ist, zu Denken\\  
	
	\noindent\textbf{Paragraphe : }
	Zweck ist ein Gegenstand der freien Willkür, dessen Vorstellung diese zu einer Handlung bestimmt, wodurch jener hervorgebracht wird. Eine jede Handlung hat also ihren Zweck und, da niemand einen Zweck haben kann, ohne sich den Gegenstand seiner Willkür selbst zum Zweck zu machen, so ist es ein Akt der Freiheit des handelnden Subjekts, nicht eine \match{Wirkung} der Natur, irgend einen Zweck der Handlungen zu haben. Weil aber dieser Akt, der einen Zweck bestimmt, ein praktisches Prinzip ist, welches nicht die Mittel (mithin nicht bedingt) sondern den Zweck selbst (folglich unbedingt) gebietet, so ist es ein kategorischer Imperativ der reinen praktischen Vernunft, mithin ein solcher, der einen Pflichtbegriff mit dem eines Zwecks überhaupt verbindet. 
	
	\subsection*{tg453.2.4} 
	\textbf{Source : }Die Metaphysik der Sitten/Zweiter Teil. Metaphysische Anfangsgründe der Tugendlehre/Einleitung/V. Erläuterung dieser zwei Begriffe\\  
	
	\noindent\textbf{Paragraphe : }Wenn von der dem Menschen überhaupt (eigentlich der Menschheit) zugehörigen Vollkommenheit gesagt wird: daß, sie sich zum Zweck zu machen, an sich selbst Pflicht sei, so muß sie in demjenigen gesetzt werden, was \match{Wirkung} von seiner Tat sein kann, nicht was bloß Geschenk ist, das er der Natur verdanken muß; denn sonst wäre sie nicht Pflicht. Sie kann also nichts anders sein als Kultur seines Vermögens (oder der Naturanlage), in welchem der Verstand, als Vermögen der Begriffe, mithin auch deren, die auf Pflicht gehen, das oberste ist, zugleich aber auch seines Willens (sittlicher Denkungsart), aller Pflicht überhaupt ein Gnüge zu tun. 1) Es ist ihm Pflicht: sich aus der Rohigkeit seiner  Natur, aus der Tierheit (quoad actum), immer mehr zur Menschheit, durch die er allein fähig ist, sich Zwecke zu setzen, empor zu arbeiten: seine Unwissenheit durch Belehrung zu ergänzen und seine Irrtümer zu verbessern, und dieses ist ihm nicht bloß die technisch-praktische Vernunft zu seinen anderweitigen Absichten (der Kunst) anrätig, sondern die moralisch-praktische gebietet es ihm schlechthin und macht diesen Zweck ihm zur Pflicht, um der Menschheit, die in ihm wohnt, würdig zu sein. 2) Die Kultur seines Willens bis zur reinesten Tugendgesinnung, da nämlich das Gesetz zugleich die Triebfeder seiner pflichtmäßigen Handlungen wird, zu erheben und ihm aus Pflicht zu gehorchen, welches innere moralisch-praktische Vollkommenheit ist, die, weil es ein Gefühl der Wirkung ist, welche der in ihm selbst gesetzgebende Wille auf das Vermögen ausübt, darnach zu handeln, das moralische Gefühl, gleichsam ein besonderer Sinn (sensus moralis), ist, der zwar freilich oft schwärmerisch, als ob er (gleich dem Genius des Sokrates) vor der Vernunft vorhergehe, oder auch ihr Urteil gar entbehren könne, mißbraucht wird, doch aber eine sittliche Vollkommenheit ist, jeden besonderen Zweck, der zugleich Pflicht ist, sich zum Gegenstande zu machen. 
	
	\subsection*{tg456.2.13} 
	\textbf{Source : }Die Metaphysik der Sitten/Zweiter Teil. Metaphysische Anfangsgründe der Tugendlehre/Einleitung/VIII. Exposition der Tugendpflichten als weiter Pflichten\\  
	
	\noindent\textbf{Paragraphe : }b) Moralisches Wohlsein anderer (salubritas moralis) gehört auch zu der Glückseligkeit anderer, die zu befördern für uns Pflicht, aber nur negative Pflicht ist. Der Schmerz, den ein Mensch von Gewissensbissen fühlt, obzwar sein Ursprung moralisch ist, ist doch, der \match{Wirkung} nach, physisch, wie der Gram, die Furcht und jeder andere krankhafte Zustand.  Zu verhüten, daß jenen dieser innere Vorwurf nicht verdienterweise treffe, ist nun zwar eben nicht meine Pflicht, sondern seine Sache; wohl aber, nichts zu tun, was, nach der Natur des Menschen, Verleitung sein könnte zu dem, worüber ihn sein Gewissen nachher peinigen kann, welches man Skandal nennt. – Aber es sind keine bestimmte Grenzen, innerhalb welcher sich diese Sorgfalt für die moralische Zufriedenheit anderer halten ließe; daher ruht auf ihr nur eine weite Verbindlichkeit. 
	
	\subsection*{tg460.2.2} 
	\textbf{Source : }Die Metaphysik der Sitten/Zweiter Teil. Metaphysische Anfangsgründe der Tugendlehre/Einleitung/XII. Ästhetische Vorbegriffe der Empfänglichkeit des Gemüts für Pflichtbegriffe überhaupt\\  
	
	\noindent\textbf{Paragraphe : }Es sind solche moralische Beschaffenheiten, die, wenn man sie nicht besitzt, es auch keine Pflicht geben kann, sich in ihren Besitz zu setzen. – Sie sind das moralische Gefühl, das Gewissen, die Liebe des Nächsten und die Achtung für sich selbst (Selbstschätzung), welche zu haben es keine Verbindlichkeit gibt; weil sie als subjektive Bedingungen der Empfänglichkeit für den Pflichtbegriff, nicht als objektive Bedingungen der Moralität zum Grunde liegen. Sie sind insgesamt ästhetisch und vorhergehende, aber natürliche Gemütsanlagen (praedispositio), durch Pflichtbegriffe affiziert zu werden; welche Anlagen zu haben nicht als Pflicht angesehen werden kann, sondern die jeder Mensch hat und kraft deren er verpflichtet werden kann. – Das Bewußtsein derselben ist nicht empirischen Ursprungs, sondern kann nur auf das eines moralischen Gesetzes, als \match{Wirkung} desselben aufs Gemüt, folgen. 
	
	\subsection*{tg460.2.7} 
	\textbf{Source : }Die Metaphysik der Sitten/Zweiter Teil. Metaphysische Anfangsgründe der Tugendlehre/Einleitung/XII. Ästhetische Vorbegriffe der Empfänglichkeit des Gemüts für Pflichtbegriffe überhaupt\\  
	
	\noindent\textbf{Paragraphe : }Dieses ist die Empfänglichkeit für Lust oder Unlust, bloß aus dem Bewußtsein der Übereinstimmung oder des Widerstreits unserer Handlung mit dem Pflichtgesetze. Alle Bestimmung der Willkür aber geht von der Vorstellung der möglichen Handlung durch das Gefühl der Lust oder Unlust, an ihr oder ihrer \match{Wirkung} ein Interesse zu nehmen, zur Tat; wo der ästhetische Zustand (der Affizierung des inneren Sinnes) nun entweder ein pathologisches oder moralisches Gefühl ist. – Das erstere ist dasjenige Gefühl, welches vor der Vorstellung des Gesetzes vorhergeht, das letztere das, was nur auf diese folgen kann. 
	
	\subsection*{tg471.2.22} 
	\textbf{Source : }Die Metaphysik der Sitten/Zweiter Teil. Metaphysische Anfangsgründe der Tugendlehre/I. Ethische Elementarlehre/I. Teil. Von den Pflichten gegen sich selbst überhaupt/Erstes Buch. Von den vollkommenen Pflichten gegen sich selbst/Erstes Hauptstück. Die Pflicht des Menschen gegen sich selbst, als einem animalischen Wesen\\  
	
	\noindent\textbf{Paragraphe : }Ein Mann empfand schon die Wasserscheu, als \match{Wirkung} von dem Biß eines tollen Hundes, und, nachdem er sich darüber so erklärt hatte: er habe noch nie erfahren, daß jemand daran geheilt worden sei, brachte er sich selbst um, damit, wie er in einer hinterlassenen Schrift sagte, er nicht in seiner Hundewut (zu welcher er schon den Anfall fühlte) andere Menschen auch unglücklich machte; es fragt sich, ob er damit unrecht tat? 
	
	\subsection*{tg471.2.28} 
	\textbf{Source : }Die Metaphysik der Sitten/Zweiter Teil. Metaphysische Anfangsgründe der Tugendlehre/I. Ethische Elementarlehre/I. Teil. Von den Pflichten gegen sich selbst überhaupt/Erstes Buch. Von den vollkommenen Pflichten gegen sich selbst/Erstes Hauptstück. Die Pflicht des Menschen gegen sich selbst, als einem animalischen Wesen\\  
	
	\noindent\textbf{Paragraphe : }So wie die Liebe zum Leben von der Natur zur Erhaltung der Person, so ist die Liebe zum Geschlecht von ihr zur Erhaltung der Art bestimmt; d.i. eine jede von beiden ist 
	Naturzweck, unter welchem man diejenige Verknüpfung der Ursache mit einer \match{Wirkung} versteht, in welcher jene, auch ohne ihr dazu einen Verstand beizulegen, diese doch nach der Analogie mit einem solchen, also gleichsam absichtlich Menschen hervorbringend gedacht wird. Es fragt sich nun, ob der Gebrauch des letzteren Vermögens, in Ansehung der Person selbst, die es ausübt, unter einem einschränkenden Pflichtgesetz stehe, oder ob diese, auch ohne jenen Zweck zu beabsichtigen, den Gebrauch ihrer Geschlechtseigenschaften der bloßen tierischen Lust zu widmen befugt sei, ohne damit einer Pflicht gegen sich selbst zuwider zu handeln. – In der Rechtslehre wird bewiesen, daß der Mensch sich einer anderen Person dieser Lust zu – Gefallen, ohne besondere Einschränkung durch einen rechtlichen Vertrag, nicht bedienen könne; wo dann zwei Personen wechselseitig einander verpflichten. Hier aber ist die Frage: ob in Ansehung dieses Genusses eine Pflicht des Menschen gegen sich selbst obwalte, deren Übertretung eine Schändung (nicht bloß Abwürdigung) der Menschheit in seiner eigenen Person sei. Der Trieb zu jenem wird Fleischeslust (auch Wohllust schlechthin) genannt. Das Laster, welches dadurch erzeugt wird, heißt Unkeuschheit, die Tugend aber, in Ansehung dieser sinnlichen Antriebe, wird Keuschheit genannt, die nun hier als Pflicht des Menschen gegen sich selbst vorgestellt werden soll. Unnatürlich heißt eine Wohllust, wenn der Mensch dazu, nicht durch den wirklichen Gegenstand) sondern durch die Einbildung von demselben, also zweckwidrig, ihn sich selbst schaffend, gereizt wird. Denn sie bewirkt alsdann eine Begierde wider den Zweck der Natur, und zwar einen noch wichtigem, als selbst der der Liebe zum Leben ist, weil dieser nur auf Erhaltung des Individuum, jener aber auf die der ganzen Spezies abzielt. – 
	
	\subsection*{tg473.2.4} 
	\textbf{Source : }Die Metaphysik der Sitten/Zweiter Teil. Metaphysische Anfangsgründe der Tugendlehre/I. Ethische Elementarlehre/I. Teil. Von den Pflichten gegen sich selbst überhaupt/Erstes Buch. Von den vollkommenen Pflichten gegen sich selbst/Zweites Hauptstück. Die Pflicht des Menschen gegen sich selbst, bloß als einem moralischen Wesen/1. Abschnitt. Von der Pflicht des Menschen gegen sich selbst, als dem angebornen Richter über sich selbst\\  
	
	\noindent\textbf{Paragraphe : }Ein jeder Pflichtbegriff enthält objektive Nötigung durchs Gesetz (als moralischen unsere Freiheit einschränkenden Imperativ) und gehört dem praktischen Verstande zu, der die Regel gibt; die innere Zurechnung aber einer Tat, als eines unter dem Gesetz stehenden Falles (in meritum aut demeritum) gehört zur Urteilskraft (iudicium), welche, als das subjektive Prinzip der Zurechnung der Handlung, ob sie als Tat (unter einem Gesetz stehende Handlung) geschehen sei oder nicht, rechtskräftig urteilt; worauf denn der Schluß der Vernunft (die Sentenz), d.i. die Verknüpfung  der rechtlichen \match{Wirkung} mit der Handlung (die Verurteilung oder Lossprechung) folgt: welches alles vor Gericht (coram iudicio), als einer dem Gesetz Effekt verschaffenden moralischen Person, Gerichtshof (forum) genannt, geschiehet. – Das Bewußtsein eines inneren Gerichtshofes im Menschen (»vor welchem sich seine Gedanken einander verklagen oder entschuldigen«) ist das Gewissen. 
	
	\subsection*{tg489.2.10} 
	\textbf{Source : }Die Metaphysik der Sitten/Fußnoten\\  
	
	\noindent\textbf{Paragraphe : }
	
	5 Selbst nicht, wie es möglich ist, daß Gott freie Wesen erschaffe; denn da wären, wie es scheint, alle künftige Handlungen derselben, durch jenen ersten Akt vorherbestimmt, in der Kette der Naturnotwendigkeit enthalten, mithin nicht frei. Daß sie aber (wir Menschen) doch frei sind, beweiset der kategorische Imperativ in moralisch-praktischer Absicht, wie durch einen Machtspruch der Vernunft, ohne daß diese doch die Möglichkeit dieses Verhältnisses einer Ursache zur \match{Wirkung} in theoretischer begreiflich machen kann, weil beide übersinnlich sind. – Was man ihr hiebei allein zumuten kann, wäre bloß: daß sie beweist, es sei in dem Begriffe von einer Schöpfung freier Wesen kein Widerspruch; und dieses kann dadurch gar wohl geschehen, daß gezeigt wird: der Widerspruch eräugne sich nur dann, wenn mit der Kategorie der Kausalität zugleich die Zeitbedingung, die im Verhältnis zu Sinnenobjekten nicht vermieden werden kann (daß nämlich der Grund einer Wirkung vor dieser vorhergehe), auch in das Verhältnis des Übersinnlichen zu einander hinüber gezogen wird (welches auch wirklich, wenn jener Kausalbegriff in theoretischer Absicht objektive Realität bekommen soll, geschehen müßte), er – der Widerspruch – aber verschwinde, wenn, in moralisch-praktischer, mithin nicht-sinnlicher Absicht, die reine Kategorie (ohne ein ihr untergelegtes Schema) im Schöpfungsbegriffe gebraucht wird. 
	
	\subsection*{tg489.2.17} 
	\textbf{Source : }Die Metaphysik der Sitten/Fußnoten\\  
	
	\noindent\textbf{Paragraphe : }
	
	8 Weil die Entthronung eines Monarchen doch auch als freiwillige Ablegung der Krone und Niederlegung seiner Gewalt, mit Zurückgebung derselben an das Volk, gedacht werden kann, oder auch als eine, ohne Vergreifung an der höchsten Person, vorgenommene Verlassung derselben, wodurch sie in den Privatstand versetzt werden würde, so hat das Verbrechen des Volks, welches sie erzwang, doch noch wenigstens den Vorwand des Notrechts (casus necessitatis) für sich, niemals aber das mindeste Recht, ihn, das Oberhaupt, wegen der vorigen Verwaltung zu strafen; weil alles, was er vorher in der Qualität eines Oberhaupts tat, als äußerlich rechtmäßig geschehen angesehen werden muß, und er selbst, als Quell der Gesetze betrachtet, nicht unrecht tun kann. Unter allen Greueln einer Staatsumwälzung durch Aufruhr ist selbst die Ermordung des Monarchen noch nicht das Ärgste: denn noch kann man sich vorstellen, sie geschehe vom Volk aus Furcht, er könne, wenn er am Leben bleibt, sich wieder ermannen, und jenes die verdiente Strafe fühlen lassen, und solle also nicht eine Verfügung der Strafgerechtigkeit, sondern bloß der Selbsterhaltung sein. Die formale Hinrichtung ist es, was die mit Ideen des Menschenrechts erfüllete Seele mit einem Schaudern ergreift, das man wiederholentlich fühlt, so bald und so oft man sich diesen Auftritt denkt, wie das Schicksal Karls I. oder Ludwigs XVI. Wie erklärt man sich aber dieses Gefühl, was hier nicht ästhetisch (ein Mitgefühl, \match{Wirkung} der Einbildungskraft, die sich in die Stelle des Leidenden versetzt), sondern moralisch, der gänzlichen Umkehrung aller Rechtsbegriffe ist? Es wird als Verbrechen, was ewig bleibt, und nie ausgetilgt werden kann (crimen immortale, inexpiabile), angesehen, und scheint demjenigen ähnlich zu sein, was die Theologen diejenige Sünde nennen, welche weder in dieser noch in jener Welt vergeben werden kann. Die Erklärung dieses Phänomens im menschlichen Gemüte scheint aus folgenden Reflexionen über sich selbst, die selbst auf die staatsrechtlichen Prinzipien ein Licht werfen, hervorzugehen. 
	
	\subsection*{tg489.2.54} 
	\textbf{Source : }Die Metaphysik der Sitten/Fußnoten\\  
	
	\noindent\textbf{Paragraphe : }
	
	25 Die Hypothese von einem künftigen Leben darf hier nicht einmal eingemischt werden, um jene drohende Strafe als vollständig in der Vollziehung vorzustellen. Denn der Mensch, seiner Moralität nach betrachtet, wird, als übersinnlicher Gegenstand vor einem übersinnlichen Richter, nicht nach Zeitbedingungen beurteilt; es ist nur von seiner Existenz die Rede. Sein Erdenleben, es sei kurz oder lang, oder gar ewig, ist nur das Dasein desselben in der Erscheinung und der Begriff der Gerechtigkeit bedarf keiner näheren Bestimmung; wie denn auch der Glaube an ein künftiges Leben eigentlich nicht vorausgeht, um die Strafgerechtigkeit an ihm ihre \match{Wirkung} sehen zu lassen, sondern vielmehr umgekehrt aus der Notwendigkeit der Bestrafung auf ein künftiges Leben die Folgerung gezogen wird.
	
	
	\unnumberedsection{Wurzel (1)} 
	\subsection*{tg450.2.3} 
	\textbf{Source : }Die Metaphysik der Sitten/Zweiter Teil. Metaphysische Anfangsgründe der Tugendlehre/Einleitung/II. Erörterung des Begriffs von einem Zwecke, der zugleich Pflicht ist\\  
	
	\noindent\textbf{Paragraphe : }Die Ethik aber nimmt einen entgegengesetzten Weg. Sie kann nicht von den Zwecken ausgehen, die der Mensch sich setzen mag und darnach über seine zu nehmende Maximen,  d.i. über seine Pflicht, verfügen; denn das wären empirische Gründe der Maximen, die keinen Pflichtbegriff abgeben, als welcher (das kategorische Sollen) in der reinen Vernunft allein seine \match{Wurzel} hat; wie denn auch, wenn die Maximen nach jenen Zwecken (welche alle selbstsüchtig sind) genommen werden sollten, vom Pflichtbegriff eigentlich gar nicht die Rede sein könnte. – Also wird in der Ethik der Pflichtbegriff auf Zwecke leiten und die Maximen, in Ansehung der Zwecke, die wir uns setzen sollen, nach moralischen Grundsätzen begründen müssen. 
	
	\unnumberedsection{Zahl (1)} 
	\subsection*{tg441.2.73} 
	\textbf{Source : }Die Metaphysik der Sitten/Erster Teil. Metaphysische Anfangsgründe der Rechtslehre/2. Teil. Das öffentliche Recht/1. Abschnitt. Das Staatsrecht\\  
	
	\noindent\textbf{Paragraphe : }So viel also der Mörder sind, die den Mord verübt, oder auch befohlen, oder dazu mitgewirkt haben, so viele müssen auch den Tod leiden; so will es die Gerechtigkeit als Idee der richterlichen Gewalt nach allgemeinen a priori begründeten Gesetzen. – Wenn aber doch die \match{Zahl} der Komplizen (correi) zu einer solchen Tat so groß ist, daß der Staat, um keine solche Verbrecher zu haben, bald dahin kommen könnte, keine Untertanen mehr zu haben, und sich doch nicht auflösen, d.i. in den noch viel ärgeren, aller äußeren Gerechtigkeit entbehrenden Naturzustand übergehen (vornehmlich nicht durch das Spektakel einer Schlachtbank das Gefühl des Volks abstumpfen) will, so muß es auch der Souverän in seiner Macht haben, in diesem Notfall (casus necessitatis) selbst den Richter zu machen (vorzustellen) und ein Urteil zu sprechen, welches, statt der Lebensstrafe, eine andere den Verbrechern zuerkennt, bei der die Volksmenge noch erhalten wird; dergleichen die Deportation ist: Dieses selbst aber nicht als nach einem öffentlichen Gesetz, sondern durch einen Machtspruch, d.i. einen Akt des Majestätsrechts, der, als Begnadigung, nur immer in einzelnen Fällen ausgeübt werden kann. 
	
	\unnumberedsection{Zeichen (3)} 
	\subsection*{tg445.2.19} 
	\textbf{Source : }Die Metaphysik der Sitten/Erster Teil. Metaphysische Anfangsgründe der Rechtslehre/Anhang erläutender Bemerkungen zu den metaphysischen Anhangsgründen der Rechtslehre\\  
	
	\noindent\textbf{Paragraphe : }Etwas Äußeres als das Seine haben heißt es rechtlich besitzen; Besitz aber ist die Bedingung der Möglichkeit des Gebrauchs. Wenn diese Bedingung bloß als die physische gedacht wird, so heißt der Besitz Inhabung. – Rechtmäßige Inhabung reicht nun zwar allein nicht zu, um deshalb den Gegenstand für das Meine auszugeben, oder es dazu zu machen; wenn ich aber, es sei aus welchem Grunde es wolle, befugt bin, auf die Inhabung eines Gegenstandes zu dringen, der meiner Gewalt entwischt oder entrissen ist, so ist dieser Rechtsbegriff ein \match{Zeichen} (wie Wirkung von ihrer Ursache), daß ich mich für befugt halte, ihn als das Meine, mich aber auch als im intelligibelen Besitz desselben befindlich gegen ihn zu verhalten und diesen Gegenstand so zu gebrauchen. 
	
	\subsection*{tg445.2.36} 
	\textbf{Source : }Die Metaphysik der Sitten/Erster Teil. Metaphysische Anfangsgründe der Rechtslehre/Anhang erläutender Bemerkungen zu den metaphysischen Anhangsgründen der Rechtslehre\\  
	
	\noindent\textbf{Paragraphe : }»Das Recht der Ersitzung (usucapio) soll, nach S. 131 ff. durchs Naturrecht begründet werden. Denn nähme man nicht an, daß durch den ehrlichen Besitz eine ideale Erwerbung, wie sie hier genannt wird, begründet werde, so wäre gar keine Erwerbung peremtorisch gesichert. (Aber Hr. K. nimmt ja selbst im Naturstande eine nur provisorische Erwerbung an, und dringt deswegen auf die juristische Notwendigkeit der bürgerlichen Verfassung. – – Ich behaupte mich als ehrlicher Besitzer aber nur gegen den, der nicht beweisen kann, daß er eher als ich ehrlicher Besitzer derselben Sache war, und mit seinem Willen zu sein nicht aufgehört hat.)« – – Davon ist nun hier nicht die Rede, sondern ob ich mich auch als Eigentümer behaupten kann, wenn sich gleich ein Prätendent als früherer wahrer Eigentümer der Sache melden sollte, die Erkundung aber seiner Existenz als Besitzers und seines Besitzstandes als Eigentümers schlechterdings unmöglich war; welches letztere alsdann zutrifft, wenn dieser gar kein öffentlich gültiges \match{Zeichen} seines ununterbrochenen Besitzes (es sei aus  eigener Schuld oder auch ohne sie), z.B. durch Einschreibung in Matrikeln, oder unwidersprochene Stimmgebung als Eigentümer in bürgerlichen Versammlungen, von sich gegeben hat. 
	
	\subsection*{tg445.2.37} 
	\textbf{Source : }Die Metaphysik der Sitten/Erster Teil. Metaphysische Anfangsgründe der Rechtslehre/Anhang erläutender Bemerkungen zu den metaphysischen Anhangsgründen der Rechtslehre\\  
	
	\noindent\textbf{Paragraphe : }Denn die Frage ist hier: wer soll seine rechtmäßige Erwerbung beweisen? Dem Besitzer kann diese Verbindlichkeit (onus probandi) nicht aufgebürdet werden; denn er ist, so weit wie seine konstatierte Geschichte reicht, im Besitz derselben. Der frühere angebliche Eigentümer der Sache ist durch eine Zwischenzeit, innerhalb deren er keine bürgerlich gültige \match{Zeichen} seines Eigentums gab, von der Reihe der auf einander folgenden Besitzer nach Rechtsprinzipien ganz abgeschnitten. Diese Unterlassung irgend eines öffentlichen Besitzakts macht ihn zu einem unbetitelten Prätendenten. (Dagegen heißt es hier, wie bei der Theologie, conservatio est continua creatio.) Wenn sich auch ein bisher nicht manifestierter, obzwar hinten nach mit aufgefundenen Dokumenten versehener Prätendent vorfände, so würde doch wiederum auch bei diesem der Zweifel vorwalten, ob nicht ein noch älterer Prätendent dereinst auftreten, und seine Ansprüche auf den früheren Besitz gründen könnte. – Auf die Länge der Zeit des Besitzes kommt es hiebei gar nicht an, um die Sache endlich zu ersitzen (acquirere per usucapionem). Denn es ist ungereimt, anzunehmen, daß ein Unrecht dadurch, daß es lange gewährt hat, nach gerade ein Recht werde. Der (noch so lange) Gebrauch setzt das Recht in der Sache voraus: weit gefehlt, daß dieses sich auf jenen gründen sollte. Also ist die Ersitzung (usucapio) als Erwerbung durch den langen Gebrauch einer Sache ein sich selbst widersprechender Begriff. Die Verjährung der Ansprüche als Erhaltungsart (conservatio possessionis meae per praescriptionem) ist es nicht weniger: indessen doch ein von dem vorigen unterschiedener Begriff, was das Argument der Zueignung betrifft. Es ist nämlich ein negativer Grund, d.i. der gänzliche Nichtgebrauch seines Rechts, selbst nicht einmal der, welcher nötig ist, um sich als Besitzer zu manifestieren, für eine Verzichttuung auf dieselbe (derelictio), welche ein rechtlicher Akt, d.i. Gebrauch seines Rechts gegen einen  anderen ist, um durch Ausschließung desselben vom Anspruche (per praescriptionem) das Objekt desselben zu erwerben, welches einen Widerspruch enthält. 
	
	\unnumberedsection{Zeit (19)} 
	\subsection*{tg429.2.9} 
	\textbf{Source : }Die Metaphysik der Sitten/Erster Teil. Metaphysische Anfangsgründe der Rechtslehre/Vorrede\\  
	
	\noindent\textbf{Paragraphe : }Von der allermindesten Bedeutung aber in Ansehung des Geistes dieser Philosophie ist wohl der Unfug, den einige Nachäffer derselben mit den Wörtern stiften, die in der Kritik d. r. V. selbst nicht wohl durch andere gangbare zu ersetzen sind, sie auch außerhalb derselben zum öffentlichen Gedankenverkehr zu brauchen, und welcher allerdings gezüchtigt zu werden verdient, wie Hr. Nicolai tut, wiewohl er über die gänzliche Entbehrung derselben in ihrem eigentümlichen Felde, gleich als einer überall bloß versteckten Armseligkeit an Gedanken, kein Urteil zu haben sich selbst bescheiden wird. – Indessen läßt sich über den unpopulären Pedanten freilich viel lustiger lachen, als über den unkritischen Ignoranten (denn in der Tat kann der Metaphysiker, welcher seinem Systeme steif anhängt, ohne sich an alle Kritik zu kehren, zur letzteren Klasse gezählt werden, ob er zwar nur willkürlich ignoriert, was er nicht aufkommen lassen will, weil es zu seiner älteren Schule nicht gehört). Wenn aber, nach Shaftesburys Behauptung, es ein nicht zu verachtender Probierstein für die Wahrheit einer (vornehmlich praktischen) Lehre ist, wenn sie das Belachen aushält, so müßte wohl an den kritischen Philosophen mit der \match{Zeit} die Reihe kommen, zuletzt, und so auch am besten, zu lachen; wenn er die papierne Systeme derer, die eine lange Zeit das große Wort führten, nach einander einstürzen, und alle Anhänger derselben sich verlaufen sieht: ein Schicksal, was jenen unvermeidlich bevorsteht. 
	
	\subsection*{tg430.2.10} 
	\textbf{Source : }Die Metaphysik der Sitten/Erster Teil. Metaphysische Anfangsgründe der Rechtslehre/Einleitung in die Metaphysik der Sitten\\  
	
	\noindent\textbf{Paragraphe : }Diese Gesetze der Freiheit heißen, zum Unterschiede von Naturgesetzen, moralisch. Sofern sie nur auf bloße äußere Handlungen und deren Gesetzmäßigkeit gehen, heißen sie juridisch; fordern sie aber auch, daß sie (die Gesetze) selbst die Bestimmungsgründe der Handlungen sein sollen, so sind sie ethisch, und alsdann sagt man: die Übereinstimmung mit den ersteren ist die Legalität, die mit den zweiten die Moralität der Handlung. Die Freiheit, auf die sich die erstern Gesetze beziehen, kann nur die Freiheit im äußeren Gebrauche, diejenige aber, auf die sich die letztere beziehen, die Freiheit sowohl im äußern als innern Gebrauche der Willkür sein, sofern sie durch Vernunftgesetze bestimmt wird. So sagt man in der theoretischen Philosophie:  im Räume sind nur die Gegenstände äußerer Sinne, in der \match{Zeit} aber alle, sowohl die Gegenstände äußerer, als des inneren Sinnes; weil die Vorstellungen beider doch Vorstellungen sind, und sofern insgesamt zum inneren Sinne gehören. Eben so mag die Freiheit im äußeren oder inneren Gebrauche der Willkür betrachtet werden, so müssen doch ihre Gesetze, als reine praktische Vernunftgesetze für die freie Willkür überhaupt, zugleich innere Bestimmungsgründe derselben sein: obgleich sie nicht immer in dieser Beziehung betrachtet werden dürfen. 
	
	\subsection*{tg431.2.8} 
	\textbf{Source : }Die Metaphysik der Sitten/Erster Teil. Metaphysische Anfangsgründe der Rechtslehre/Einleitung in die Rechtslehre\\  
	
	\noindent\textbf{Paragraphe : }Diese Frage möchte wohl den Rechtsgelehrten, wenn er nicht in Tautologie verfallen, oder, statt einer allgemeinen Auflösung, auf das, was in irgend einem Lande die Gesetze zu irgend einer \match{Zeit} wollen, verweisen will, eben so in Verlegenheit setzen, als die berufene Aufforderung: Was ist Wahrheit? den Logiker, Was Rechtens sei (quid sit iuris), d.i. was die Gesetze an einem gewissen Ort und zu einer gewissen Zeit sagen oder gesagt haben, kann er noch wohl angeben; aber, ob das, was sie wollten, auch recht sei, und das allgemeine Kriterium, woran man überhaupt Recht sowohl als Unrecht (iustum et iniustum) erkennen könne, bleibt ihm wohl verborgen, wenn er nicht eine Zeitlang jene empirischen Prinzipien verläßt, die Quellen jener Urteile in der bloßen Vernunft sucht (wiewohl ihm dazu jene Gesetze vortrefflich zum Leitfaden dienen können), um zu einer möglichen positiven Gesetzgebung die Grundlage zu errichten. Eine bloß empirische Rechtslehre ist (wie der hölzerne Kopf in Phädrus' Fabel) ein Kopf, der schön sein mag, nur schade! daß er kein Gehirn hat.  Der Begriff des Rechts, sofern er sich auf eine ihm korrespondierende Verbindlichkeit bezieht (d.i. der moralische Begriff derselben), betrifft erstlich nur das äußere und zwar praktische Verhältnis einer Person gegen eine andere, sofern ihre Handlungen als Facta aufeinander (unmittelbar, oder mittelbar) Einfluß haben können. Aber zweitens bedeutet er nicht das Verhältnis der Willkür auf den Wunsch (folglich auch auf das bloße Bedürfnis) des anderen, wie etwa in den Handlungen der Wohltätigkeit oder Hartherzigkeit, sondern lediglich auf die Willkür des anderen. Drittens in diesem wechselseitigen Verhältnis der Willkür kommt auch gar nicht die Materie der Willkür, d.i. der Zweck, den ein jeder mit dem Objekt, was er will, zur Absicht hat, in Betrachtung, z.B. es wird nicht gefragt, ob jemand bei der Ware, die er zu seinem eigenen Handel von mir kauft, auch seinen Vorteil finden möge, oder nicht, sondern nur nach der Form im Verhältnis der beiderseitigen Willkür, sofern sie bloß als frei betrachtet wird, und ob durch die Handlung eines von beiden sich mit der Freiheit des andern nach einem allgemeinen Gesetze zusammen vereinigen lasse. 
	
	\subsection*{tg433.2.43} 
	\textbf{Source : }Die Metaphysik der Sitten/Erster Teil. Metaphysische Anfangsgründe der Rechtslehre/1. Teil. Das Privatrecht vom äußeren Mein und Dein überhaupt/1. Hauptstück\\  
	
	\noindent\textbf{Paragraphe : }Dieses kann auch auf den Fall angewendet werden, da ich ein Versprechen akzeptiert habe; denn da wird meine Habe und Besitz an dem Versprochenen da durch nicht aufgehoben, daß der Versprechende zu einer \match{Zeit} sagte: diese Sache soll dein sein, eine Zeit hernach aber von ebenderselben Sache  sagt: ich will jetzt, die Sache solle nicht dein sein. Denn es hat mit solchen intellektuellen Verhältnissen die Bewandtnis, als ob jener ohne eine Zeit zwischen beiden Deklarationen seines Willens gesagt hätte, sie soll dein sein, und auch, sie soll nicht dein sein, was sich dann selbst widerspricht. 
	
	\subsection*{tg434.2.5} 
	\textbf{Source : }Die Metaphysik der Sitten/Erster Teil. Metaphysische Anfangsgründe der Rechtslehre/1. Teil. Das Privatrecht vom äußeren Mein und Dein überhaupt\\  
	
	\noindent\textbf{Paragraphe : }Nichts Äußeres ist ursprünglich mein; wohl aber kann es ursprünglich, d.i. ohne es von dem Seinen irgend eines anderen abzuleiten, erworben sein, – Der Zustand der Gemeinschaft des Mein und Dein (communio) kann nie als ursprünglich gedacht, sondern muß (durch einen äußeren rechtlichen Akt) erworben werden; obwohl der Besitz eines äußeren Gegenstandes ursprünglich und gemeinsam sein kann. Auch wenn man sich (problematisch) eine ursprüngliche Gemeinschaft (communio mei et tui originaria) denkt: so muß sie doch von der uranfänglichen (communio primaeva) unterschieden werden, welche, als in der ersten \match{Zeit} der Rechtsverhältnisse unter Menschen gestiftet, angenommen wird, und nicht, wie die erstere, auf Prinzipien, sondern nur auf Geschichte gegründet werden kann: wobei die letztere doch immer als erworben und abgeleitet (communio derivativa) gedacht werden müßte. 
	
	\subsection*{tg434.2.8} 
	\textbf{Source : }Die Metaphysik der Sitten/Erster Teil. Metaphysische Anfangsgründe der Rechtslehre/1. Teil. Das Privatrecht vom äußeren Mein und Dein überhaupt\\  
	
	\noindent\textbf{Paragraphe : }Die ursprüngliche Erwerbung eines äußeren Gegenstandes der Willkür heißt Bemächtigung (occupatio) und kann nicht anders, als an körperlichen Dingen (Substanzen) statt finden. Wo nun eine solche statt findet, bedarf sie zur Bedingung des empirischen Besitzes die Priorität der \match{Zeit} vor jedem anderen, der sich einer Sache bemächtigen will (qui prior tempore potior iure). Sie ist als ursprünglich auch nur die Folge von einseitiger Willkür; denn wäre dazu eine doppelseitige erforderlich, so würde sie von dem Vertrag zweier (oder mehrerer) Personen, folglich von dem Seinen anderer abgeleitet sein. – Wie ein solcher Akt der Willkür, als jener ist, das Seine für jemanden begründen könne, ist nicht leicht einzusehen, – Indessen ist die erste Erwerbung doch darum so fort nicht die ursprüngliche. Denn die Erwerbung eines öffentlichen rechtlichen Zustandes durch Vereinigung des Willens aller zu einer allgemeinen Gesetzgebung wäre eine solche, vor der keine vorhergehen darf, und doch wäre sie von dem besonderen Willen eines jeden abgeleitet und allseitig: da eine ursprüngliche Erwerbung nur aus dem einseitigen Willen hervorgehen kann. 
	
	\subsection*{tg436.2.11} 
	\textbf{Source : }Die Metaphysik der Sitten/Erster Teil. Metaphysische Anfangsgründe der Rechtslehre/1. Teil. Das Privatrecht vom äußeren Mein und Dein überhaupt/2. Hauptstück. Von der Art, etwas Äußeres zu erwerben/2. Abschnitt. Vom persönlichen Recht\\  
	
	\noindent\textbf{Paragraphe : }
	Aber weder durch den besonderen Willen des Promittenten, noch den des Promissars (als Akzeptanten), geht das Seine des ersteren zu dem letzteren über, sondern nur durch den vereinigten Willen beider, mithin so fern beider Wille zugleich deklariert wird. Nun ist dies aber durch empirische Actus der Deklaration, die einander notwendig in der \match{Zeit} folgen müssen, und niemals zugleich sind, unmöglich. Denn, wenn ich versprochen habe und der andere nun akzeptieren will, so kann ich während der Zwischenzeit (so kurz sie auch sein mag) es mich gereuen lassen, weil ich vor der Akzeptation noch frei bin; so wie anderseits der Akzeptant, eben darum, an seine auf das Versprechen folgende Gegenerklärung auch sich nicht für gebunden halten darf. – Die äußern Förmlichkeiten (solennia) bei Schließung des Vertrags (der Handschlag, oder die Zerbrechung eines von beiden Personen angefaßten Strohhalms (stipula)), und alle hin und her geschehene Bestätigungen seiner vorherigen Erklärung beweisen vielmehr die Verlegenheit der Paziszenten, wie und auf welche Art sie die immer nur aufeinander folgenden Erklärungen als in einem Augenblicke zugleich existierend vorstellig machen wollen, was ihnen doch nicht gelingt; weil es immer nur in der Zeit einander folgende Actus sind, wo, wenn der eine Akt ist, der andere entweder noch nicht, oder nicht mehr ist. 
	
	\subsection*{tg436.2.21} 
	\textbf{Source : }Die Metaphysik der Sitten/Erster Teil. Metaphysische Anfangsgründe der Rechtslehre/1. Teil. Das Privatrecht vom äußeren Mein und Dein überhaupt/2. Hauptstück. Von der Art, etwas Äußeres zu erwerben/2. Abschnitt. Vom persönlichen Recht\\  
	
	\noindent\textbf{Paragraphe : }Der Vertrag, auf den unmittelbar die Übergabe folgt (pactum re initum), schließt alle Zwischenzeit zwischen der Schließung und Vollziehung aus, und bedarf keines besonderen noch zu erwartenden Akts, wodurch das Seine des einen auf den anderen übertragen wird. Aber, wenn zwischen jenen beiden noch eine (bestimmte oder unbestimmte) \match{Zeit} zur Übergabe bewilligt ist, fragt sich: ob die Sache schon vor dieser durch den Vertrag das Seine des Akzeptanten geworden, und das Recht des letzteren ein Recht in der Sache sei, oder ob noch ein besonderer Vertrag, der allein die Übergabe betrifft, dazu kommen müsse, mithin das Recht durch die bloße Akzeptation nur ein persönliches sei, und allererst durch die Übergabe ein Recht in der Sache werde? – Daß es sich hiemit wirklich so, wie das letztere besagt, verhalte, erhellet aus nachfolgendem: 
	
	\subsection*{tg437.2.32} 
	\textbf{Source : }Die Metaphysik der Sitten/Erster Teil. Metaphysische Anfangsgründe der Rechtslehre/1. Teil. Das Privatrecht vom äußeren Mein und Dein überhaupt/2. Hauptstück. Von der Art, etwas Äußeres zu erwerben/3. Abschnitt. Von dem auf dingliche Art persönlichen Recht\\  
	
	\noindent\textbf{Paragraphe : }Aus dieser Pflicht entspringt auch notwendig das Recht der Eltern zur Handhabung und Bildung des Kindes, so lange es des eigenen Gebrauchs seiner Gliedmaßen, imgleichen des Verstandesgebrauchs, noch nicht mächtig ist, außer der Ernährung und Pflege es zu erziehen, und sowohl pragmatisch, damit es künftig sich selbst erhalten und fortbringen  könne, als auch moralisch, weil sonst die Schuld ihrer Verwahrlosung auf die Eltern fallen würde, – es zu bilden; alles bis zur \match{Zeit} der Entlassung (emancipatio), da diese, sowohl ihrem väterlichen Recht zu befehlen, als auch allem Anspruch auf Kostenerstattung für ihre bisherige Verpflegung und Mühe entsagen, wofür, und nach vollendeter Erziehung, sie der Kinder ihre Verbindlichkeit (gegen die Eltern) nur als bloße Tugendpflicht, nämlich als Dankbarkeit, in Anschlag bringen können. 
	
	\subsection*{tg438.2.4} 
	\textbf{Source : }Die Metaphysik der Sitten/Erster Teil. Metaphysische Anfangsgründe der Rechtslehre/1. Teil. Das Privatrecht vom äußeren Mein und Dein überhaupt/2. Hauptstück. Von der Art, etwas Äußeres zu erwerben/Episodischer Abschnitt. Von der idealen Erwerbung eines äußeren Gegenstandes der Willkür\\  
	
	\noindent\textbf{Paragraphe : }Ich nenne diejenige Erwerbung ideal, die keine Kausalität in der \match{Zeit} enthält, mithin eine bloße Idee der reinen Vernunft zum Grunde hat. Sie ist nichtsdestoweniger wahre, nicht eingebildete, Erwerbung, und heißt nur darum nicht real, weil der Erwerbakt nicht empirisch ist, indem das Subjekt von einem anderen, der entweder noch nicht ist (von dem man bloß die Möglichkeit annimmt, daß er sei), oder, indem dieser eben aufhört zu sein, oder, wenn er nicht mehr ist, erwirbt, mithin die Gelangung zum Besitz eine bloße praktische Idee der Vernunft ist. – Es sind die drei Erwerbungsarten: 1) durch Ersitzung, 2) durch Beerbung, 3) durch unsterbliches Verdienst (meritum immortale), d.i. der Anspruch auf den guten Namen nach dem Tode. Alle drei können zwar nur im öffentlichen rechtlichen Zustande ihren Effekt haben, gründen sich aber nicht nur auf der Konstitution desselben und willkürlichen Statuten, sondern sind auch a priori im Naturzustande, und zwar notwendig zuvor, denkbar, um hernach die Gesetze in der bürgerlichen Verfassung darnach einzurichten (sunt iuris naturae). 
	
	\subsection*{tg438.2.8} 
	\textbf{Source : }Die Metaphysik der Sitten/Erster Teil. Metaphysische Anfangsgründe der Rechtslehre/1. Teil. Das Privatrecht vom äußeren Mein und Dein überhaupt/2. Hauptstück. Von der Art, etwas Äußeres zu erwerben/Episodischer Abschnitt. Von der idealen Erwerbung eines äußeren Gegenstandes der Willkür\\  
	
	\noindent\textbf{Paragraphe : }Ich erwerbe das Eigentum eines anderen bloß durch den langen Besitz (usucapio); nicht weil ich dieses seine Einwilligung  dazu rechtmäßig voraussetzen darf (per consensum praesumtum), noch weil ich, da er nicht widerspricht, annehmen kann, er habe seine Sache aufgegeben (rem derelictam), sondern, weil, wenn es auch einen wahren und auf diese Sache als Eigentümer Anspruch Machenden (Prätendenten) gäbe, ich ihn doch bloß durch meinen langen Besitz ausschließen, sein bisheriges Dasein ignorieren, und gar, als ob er zur \match{Zeit} meines Besitzes nur als Gedankending existierte, verfahren darf: wenn ich gleich von seiner Wirklichkeit so wohl, als der seines Anspruchs hinterher benachrichtigt sein möchte. – Man nennt diese Art der Erwerbung, nicht ganz richtig, die durch Verjährung (per praescriptionem); denn die Ausschließung ist nur als die Folge von jener anzusehen; die Erwerbung muß vorhergegangen sein. – Die Möglichkeit, auf diese Art zu erwerben ist nun zu beweisen. 
	
	\subsection*{tg441.2.50} 
	\textbf{Source : }Die Metaphysik der Sitten/Erster Teil. Metaphysische Anfangsgründe der Rechtslehre/2. Teil. Das öffentliche Recht/1. Abschnitt. Das Staatsrecht\\  
	
	\noindent\textbf{Paragraphe : }Hieraus folgt: daß es auch keine Korporation im Staat, keinen Stand und Orden, geben könne, der als Eigentümer den Boden zur alleinigen Benutzung den folgenden Generationen (ins Unendliche) nach gewissen Statuten überliefern könne. Der Staat kann sie zu aller \match{Zeit} aufheben, nur unter der Bedingung, die Überlebenden zu entschädigen. Der Ritterorden (als Korporation, oder auch bloß Rang einzelner, vorzüglich beehrter, Personen); der Orden der Geistlichkeit, die Kirche genannt, können nie durch diese Vorrechte, womit sie begünstigt worden, ein auf Nachfolger übertragbares Eigentum am Boden, sondern nur die einstweilige Benutzung desselben erwerben. Die Komtureien auf einer, die Kirchengüter auf der anderen Seite können, wenn die öffentliche Meinung wegen der Mittel, durch die Kriegsehre
	den Staat wider die Lauigkeit in Verteidigung desselben zu schützen, oder die Menschen in demselben durch Seelmessen, Gebete und eine Menge zu bestellender Seelsorger, um sie vor dem ewigen Feuer zu bewahren, anzutreiben, aufgehört hat, ohne Bedenken (doch unter der vorgenannten Bedingung) aufgehoben werden. Die, so hier in die Reform fallen, können nicht klagen, daß ihnen ihr Eigentum genommen werde; denn der Grund ihres bisherigen Besitzes lag nur in der Volksmeinung, und mußte auch, so lange diese fortwährte, gelten. So bald diese aber erlosch, und zwar auch nur in dem Urteil derjenigen, welche auf Leitung desselben durch ihr Verdienst den größten Anspruch haben, so mußte, gleichsam als durch eine Appellation desselben an den Staat (a rege male informato ad regem melius informandum), das vermeinte Eigentum aufhören. 
	
	\subsection*{tg442.2.48} 
	\textbf{Source : }Die Metaphysik der Sitten/Erster Teil. Metaphysische Anfangsgründe der Rechtslehre/2. Teil. Das öffentliche Recht/2. Abschnitt. Das Völkerrecht\\  
	
	\noindent\textbf{Paragraphe : }Unter einem Kongreß wird hier aber nur eine willkürliche, zu aller \match{Zeit} ablösliche Zusammentretung verschiedener Staaten, nicht eine solche Verbindung, welche (so wie die der amerikanischen Staaten) auf einer Staatsverfassung gegründet, und daher unauflöslich ist, verstanden; – durch welchen allein die Idee eines zu errichtenden öffentlichen Rechts der Völker, ihre Streitigkeiten auf zivile Art, gleichsam durch einen Prozeß, nicht auf barbarische (nach Art der Wilden), nämlich durch Krieg zu entscheiden, realisiert werden kann. 
	
	\subsection*{tg445.2.50} 
	\textbf{Source : }Die Metaphysik der Sitten/Erster Teil. Metaphysische Anfangsgründe der Rechtslehre/Anhang erläutender Bemerkungen zu den metaphysischen Anhangsgründen der Rechtslehre\\  
	
	\noindent\textbf{Paragraphe : }Die wohltätige Anstalt für Arme, Invalide und Kranke, welche auf dem Staatsvermögen fundiert worden (in Stiften und Hospitälern), ist allerdings unablöslich. Wenn aber nicht der Buchstabe sondern der Sinn des Willens des Testators den Vorzug haben soll, so können sich wohl Zeitumstände ereignen, welche die Aufhebung einer solchen Stiftung wenigstens ihrer Form nach anrätig machen. – So hat man gefunden: daß der Arme und Kranke (den vom Narrenhospital ausgenommen) besser und wohlfeiler versorgt werde, wenn ihm die Beihülfe in einer gewissen (dem Bedürfnisse der \match{Zeit} proportionierten) Geldsumme, wofür er sich, wo er will, bei seinen Verwandten oder sonst Bekannten, einmieten kann, gereicht wird, als wenn – wie im Hospital von Greenwich – prächtige und dennoch die Freiheit sehr beschränkende, mit einem kostbaren Personale  versehenen Anstalten dazu getroffen werden. – Da kann man nun nicht sagen, der Staat nehme dem zum Genuß dieser Stiftung berechtigten Volke das Seine, sondern er befördert es vielmehr, indem er weisere Mittel zur Erhaltung desselben wählt. 
	
	\subsection*{tg445.2.54} 
	\textbf{Source : }Die Metaphysik der Sitten/Erster Teil. Metaphysische Anfangsgründe der Rechtslehre/Anhang erläutender Bemerkungen zu den metaphysischen Anhangsgründen der Rechtslehre\\  
	
	\noindent\textbf{Paragraphe : }Die Frage ist hier: ob die Kirche dem Staat oder der Staat der Kirche als das Seine angehören könne; denn zwei oberste Gewalten können einander ohne Widerspruch nicht untergeordnet sein. – Daß nur die erstere Verfassung (politico-hierarchica) Bestand an sich haben könne, ist an sich klar: denn alle bürgerliche Verfassung ist von dieser Welt, weil sie eine irdische Gewalt (der Menschen) ist, die sich samt ihren Folgen in der Erfahrung dokumentieren läßt. Die Gläubigen, deren Reich im Himmel und in jener Welt ist, müssen, in so fern man ihnen eine sich auf dieses beziehende Verfassung (hierarchico-politica) zugesteht, sich den Leiden dieser \match{Zeit} unter der Obergewalt der Weltmenschen unterwerfen. – Also findet nur die erstere Verfassung statt. 
	
	\subsection*{tg445.2.57} 
	\textbf{Source : }Die Metaphysik der Sitten/Erster Teil. Metaphysische Anfangsgründe der Rechtslehre/Anhang erläutender Bemerkungen zu den metaphysischen Anhangsgründen der Rechtslehre\\  
	
	\noindent\textbf{Paragraphe : }Selbst Stiftungen zu ewigen Zeiten für Arme, oder Schulanstalten, sobald sie einen gewissen, von dem Stifter nach  seiner Idee bestimmten entworfenen Zuschnitt haben, können nicht auf ewige Zeiten fundiert und der Boden damit belästigt werden; sondern der Staat muß die Freiheit haben, sie nach dem Bedürfnisse der \match{Zeit} einzurichten. – Daß es schwerer hält, diese Idee allerwärts auszuführen (z.B. die Pauperbursche die Unzulänglichkeit des wohltätig errichteten Schulfonds durch bettelhaftes Singen ergänzen zu müssen,) darf niemanden wundern; denn der, welcher gutmütiger- aber doch zugleich etwas ehrbegierigerweise eine Stiftung macht, will, daß sie nicht ein anderer nach seinen Begriffen umändere, sondern Er darin unsterblich sei. Das ändert aber nicht die Beschaffenheit der Sache selbst und das Recht des Staats, ja die Pflicht desselben zum Umändern einer jeden Stiftung, wenn sie der Erhaltung und dem Fortschreiten desselben zum Besseren entgegen ist, kann daher niemals als auf ewig begründet betrachtet werden. 
	
	\subsection*{tg471.2.36} 
	\textbf{Source : }Die Metaphysik der Sitten/Zweiter Teil. Metaphysische Anfangsgründe der Tugendlehre/I. Ethische Elementarlehre/I. Teil. Von den Pflichten gegen sich selbst überhaupt/Erstes Buch. Von den vollkommenen Pflichten gegen sich selbst/Erstes Hauptstück. Die Pflicht des Menschen gegen sich selbst, als einem animalischen Wesen\\  
	
	\noindent\textbf{Paragraphe : }Ist es z.B. zur \match{Zeit} der Schwangerschaft, – ist es bei der Sterilität des Weibes (Alters oder Krankheit wegen), oder wenn dieses keinen Anreiz dazu bei sich findet, nicht dem Naturzwecke und hiemit auch der Pflicht gegen sich selbst, an einem oder dem anderen Teil, eben so wie bei der unnatürlichen Wohllust, zuwider, von seinen Geschlechtseigenschaften Gebrauch zu machen; oder gibt es hier ein Erlaubnisgesetz der moralisch-praktischen Vernunft, welches in der Kollision ihrer Bestimmungsgründe etwas, an sich zwar Unerlaubtes, doch zur Verhütung einer noch größeren Übertretung (gleichsam nachsichtlich) erlaubt macht? – Von wo an kann man die Einschränkung einer weiten Verbindlichkeit zum Purism (einer Pedanterei in Ansehung der Pflichtbeobachtung, was die Weite derselben betrifft) zählen, und den tierischen Neigungen, mit Gefahr der Verlassung des Vernunftgesetzes, einen Spielraum verstatten? 
	
	\subsection*{tg471.2.45} 
	\textbf{Source : }Die Metaphysik der Sitten/Zweiter Teil. Metaphysische Anfangsgründe der Tugendlehre/I. Ethische Elementarlehre/I. Teil. Von den Pflichten gegen sich selbst überhaupt/Erstes Buch. Von den vollkommenen Pflichten gegen sich selbst/Erstes Hauptstück. Die Pflicht des Menschen gegen sich selbst, als einem animalischen Wesen\\  
	
	\noindent\textbf{Paragraphe : }Die tierische Unmäßigkeit, im Genuß der Nahrung, ist der Mißbrauch der Genießmittel, wodurch das Vermögen des intellektuellen Gebrauchs derselben gehemmt oder erschöpft wird. Versoffenheit und Gefräßigkeit sind die Laster, die unter diese Rubrik gehören. Im Zustande der Betrunkenheit ist der Mensch nur wie ein Tier, nicht als Mensch, zu behandeln; durch die Überladung mit Speisen und in einem solchen Zustande ist er für Handlungen, wozu Gewandtheit und Überlegung im Gebrauch seiner Kräfte erfordert wird, auf eine gewisse \match{Zeit} gelähmt. – Daß sich in einen solchen Zustand zu versetzen Verletzung einer Pflicht wider sich selbst sei, fällt von selbst in die Augen. Die erste dieser Erniedrigungen, selbst unter die tierische Natur, wird gewöhnlich durch gegorene Getränke, aber auch durch andere betäubende Mittel, als den Mohnsaft und andere Produkte des Gewächsreichs, bewirkt, und wird dadurch verführerisch, daß dadurch auf eine Weile geträumte Glückseligkeit und Sorgenfreiheit, ja wohl auch eingebildete Stärke hervorgebracht, Niedergeschlagenheit aber und Schwäche, und, was das Schlimmste ist, Notwendigkeit, dieses Betäubungsmittel zu wiederholen, ja wohl gar damit zu steigern, eingeführt wird. Die Gefräßigkeit ist sofern noch unter jener tierischen Sinnenbelustigung, daß sie bloß den Sinn  als passive Beschaffenheit und nicht einmal die Einbildungskraft, welche doch noch ein tätiges Spiel der Vorstellungen, wie im vorerwähnten Genuß der Fall ist, beschäftigt; mithin sich dem des Viehes noch mehr nähert. 
	
	\subsection*{tg489.2.13} 
	\textbf{Source : }Die Metaphysik der Sitten/Fußnoten\\  
	
	\noindent\textbf{Paragraphe : }
	
	6 Daß man aber hiebei ja nicht auf Vorempfindung eines künftigen Lebens und unsichtbare Verhältnisse zu abgeschiedenen Seelen schwärmerisch schließe, denn es ist hier von nichts weiter, als dem reinmoralischen und rechtlichen Verhältnis, was unter Menschen auch im Leben statt hat, die Rede, worin sie, als intelligibele Wesen, stehen, indem man alles Physische (zu ihrer Existenz in Raum und \match{Zeit} Gehörende) logisch davon absondert, d.i. davon abstrahiert, nicht aber die Menschen diese ihre Natur ausziehen und sie Geister werden läßt, in welchem Zustande sie die Beleidigung durch ihre Verleumder fühleten. – Der, welcher nach hundert Jahren mir etwas Böses fälschlich nachsagt, beleidigt mich schon jetzt; denn im reinen Rechtsverhältnisse, welches ganz intellektuell ist, wird von allen physischen Bedingungen (der Zeit) abstrahiert, und der Ehrenräuber (Kalumniant) ist eben sowohl strafbar, als ob er es in meiner Lebzeit getan hätte; nur durch kein Kriminalgericht, sondern nur dadurch, daß ihm, nach dem Recht der Wiedervergeltung, durch die öffentliche Meinung derselbe Verlust der Ehre zugefügt wird, die er an eine in anderen schmälerte. – Selbst das Plagiat, welches ein Schriftsteller an Verstorbenen verübt, ob es zwar die Ehre des Verstorbenen nicht befleckt, sondern diesem nur einen Teil derselben entwendet, wird doch mit Recht als Läsion desselben (Menschenraub) geahndet. 
	
	\unnumberedsection{Zeitalter (2)} 
	\subsection*{tg445.2.53} 
	\textbf{Source : }Die Metaphysik der Sitten/Erster Teil. Metaphysische Anfangsgründe der Rechtslehre/Anhang erläutender Bemerkungen zu den metaphysischen Anhangsgründen der Rechtslehre\\  
	
	\noindent\textbf{Paragraphe : }Die Geistlichkeit, welche sich fleischlich nicht fortpflanzt (die katholische), besitzt, mit Begünstigung des Staats, Ländereien und daran haftende Untertanen, die einem geistlichen Staate (Kirche genannt) angehören, welchem die Weltliche durch Vermächtnis zum Heil ihrer Seelen sich als ihr Eigentum hingegeben haben, und so hat der Klerus als ein besonderer Stand einen Besitztum, der sich von einem \match{Zeitalter} zum anderen gesetzmäßig vererben läßt und durch päpstliche Bullen hinreichend dokumentiert ist. – Kann man nun wohl annehmen, daß dieses Verhältnis derselben zu den Laien, durch die Machtvollkommenheit des weltlichen Staats, geradezu den ersteren könne genommen werden, und würde das nicht so viel sein, als jemanden mit Gewalt das Seine nehmen; wie es doch von Ungläubigen der französischen Republik versucht wird. 
	
	\subsection*{tg445.2.60} 
	\textbf{Source : }Die Metaphysik der Sitten/Erster Teil. Metaphysische Anfangsgründe der Rechtslehre/Anhang erläutender Bemerkungen zu den metaphysischen Anhangsgründen der Rechtslehre\\  
	
	\noindent\textbf{Paragraphe : }Der Adel eines Landes, das selbst nicht unter einer aristokratischen, sondern monarchischen Verfassung steht, mag immer ein, für ein gewisses \match{Zeitalter} erlaubtes, und den Umständen nach notwendiges Institut sein; aber daß dieser Stand auf ewig könne begründet werden, und ein Staatsoberhaupt nicht solle die Befugnis haben, diesen Standesvorzug gänzlich aufzuheben, oder, wenn er es tut, man sagen könne, er nehme seinem (adligen) Untertan das Seine, was ihm erblich zukommt, kann keinesweges behauptet werden. Er ist eine temporäre, vom Staat autorisierte, Zunftgenossenschaft, die sich nach den Zeitumständen bequemen muß, und dem allgemeinen Menschenrechte, das so lange suspendiert war, nicht Abbruch tun darf. – Denn der Rang des Edelmanns im Staate ist von der Konstitution selber nicht allein abhängig, sondern ist nur ein Akzidenz derselben, was nur durch Inhärenz in demselben existieren kann (ein Edelmann kann ja als ein solcher, nur im Staate, nicht im Stande der Natur gedacht werden). Wenn also der Staat seine Konstitution abändert, so kann der, welcher  hiemit jenen Titel und Vorrang einbüßt, nicht sagen, es sei ihm das Seine genommen; weil er es nur unter der Bedingung der Fortdauer dieser Staatsform das Seine nennen konnte: der Staat aber diese abzuändern (z.B. in den Republikanism umzuformen) das Recht hat. – Die Orden, und der Vorzug, gewisse Zeichen desselben zu tragen, geben also kein ewiges Recht dieses Besitzes. 
	
	\unnumberedchapter{Vetement} 
	\unnumberedsection{Anhanger (1)} 
	\subsection*{tg429.2.9} 
	\textbf{Source : }Die Metaphysik der Sitten/Erster Teil. Metaphysische Anfangsgründe der Rechtslehre/Vorrede\\  
	
	\noindent\textbf{Paragraphe : }Von der allermindesten Bedeutung aber in Ansehung des Geistes dieser Philosophie ist wohl der Unfug, den einige Nachäffer derselben mit den Wörtern stiften, die in der Kritik d. r. V. selbst nicht wohl durch andere gangbare zu ersetzen sind, sie auch außerhalb derselben zum öffentlichen Gedankenverkehr zu brauchen, und welcher allerdings gezüchtigt zu werden verdient, wie Hr. Nicolai tut, wiewohl er über die gänzliche Entbehrung derselben in ihrem eigentümlichen Felde, gleich als einer überall bloß versteckten Armseligkeit an Gedanken, kein Urteil zu haben sich selbst bescheiden wird. – Indessen läßt sich über den unpopulären Pedanten freilich viel lustiger lachen, als über den unkritischen Ignoranten (denn in der Tat kann der Metaphysiker, welcher seinem Systeme steif anhängt, ohne sich an alle Kritik zu kehren, zur letzteren Klasse gezählt werden, ob er zwar nur willkürlich ignoriert, was er nicht aufkommen lassen will, weil es zu seiner älteren Schule nicht gehört). Wenn aber, nach Shaftesburys Behauptung, es ein nicht zu verachtender Probierstein für die Wahrheit einer (vornehmlich praktischen) Lehre ist, wenn sie das Belachen aushält, so müßte wohl an den kritischen Philosophen mit der Zeit die Reihe kommen, zuletzt, und so auch am besten, zu lachen; wenn er die papierne Systeme derer, die eine lange Zeit das große Wort führten, nach einander einstürzen, und alle \match{Anhänger} derselben sich verlaufen sieht: ein Schicksal, was jenen unvermeidlich bevorsteht. 
	
	\unnumberedsection{Geschmack (3)} 
	\subsection*{tg471.2.37} 
	\textbf{Source : }Die Metaphysik der Sitten/Zweiter Teil. Metaphysische Anfangsgründe der Tugendlehre/I. Ethische Elementarlehre/I. Teil. Von den Pflichten gegen sich selbst überhaupt/Erstes Buch. Von den vollkommenen Pflichten gegen sich selbst/Erstes Hauptstück. Die Pflicht des Menschen gegen sich selbst, als einem animalischen Wesen\\  
	
	\noindent\textbf{Paragraphe : }Die Geschlechtsneigung wird auch Liebe (in der engsten Bedeutung des Worts) genannt und ist in der Tat die größte Sinnenlust, die an einem Gegenstande möglich ist; – nicht bloß sinnliche Lust, wie an Gegenständen, die in der bloßen Reflexion über sie gefallen (da die Empfänglichkeit für sie \match{Geschmack} heißt), sondern die Lust aus dem Genusse einer anderen Person, die also zum Begehrungsvermögen und zwar der höchsten Stufe desselben, der Leidenschaft, gehört. Sie kann aber weder zur Liebe des Wohlgefallens, noch der des Wohlwollens gezählt werden (denn beide halten eher vom fleischlichen Genuß ab), sondern ist eine Lust von besonderer Art (sui generis) und das Brünstigsein hat mit der moralischen Liebe eigentlich nichts gemein, wiewohl sie mit der letzteren, wenn die praktische Vernunft mit ihren einschränkenden Bedingungen hinzu kommt, in enge Verbindung treten kann. 
	
	\subsection*{tg477.2.6} 
	\textbf{Source : }Die Metaphysik der Sitten/Zweiter Teil. Metaphysische Anfangsgründe der Tugendlehre/I. Ethische Elementarlehre/I. Teil. Von den Pflichten gegen sich selbst überhaupt/2. Buch: Die Pflichten gegen sich selbst/Erster Abschnitt. Von der Pflicht gegen sich selbst in Entwickelung und Vermehrung seiner Naturvollkommenheit, d.i. in pragmatischer Absicht\\  
	
	\noindent\textbf{Paragraphe : }
	Seelenkräfte sind diejenige, welche dem Verstande und der Regel, die er zu Befriedigung beliebiger Absichten braucht, zu Gebote stehen, und so fern an dem Leitfaden der Erfahrung geführt werden. Dergleichen ist das Gedächtnis, die Einbildungskraft u. dgl., worauf Gelahrtheit, \match{Geschmack} (innere und äußere Verschönerung) etc. gegründet werden können, welche zu mannigfaltiger Absicht die Werkzeuge darbieten. 
	
	\subsection*{tg483.2.4} 
	\textbf{Source : }Die Metaphysik der Sitten/Zweiter Teil. Metaphysische Anfangsgründe der Tugendlehre/I. Ethische Elementarlehre/II. Teil. Von den Tugendpflichten gegen andere/Zweites Hauptstück. Von den ethischen Pflichten der Menschen gegen einander in Ansehung ihres Zustandes\\  
	
	\noindent\textbf{Paragraphe : }Diese (Tugendpflichten) können zwar in der reinen Ethik keinen Anlaß zu einem besondern Hauptstück im System derselben geben, denn sie enthalten nicht Prinzipien der Verpflichtung der Menschen als solcher gegen einander, und können also von den metaphysischen Anfangsgründen der Tugendlehre eigentlich nicht einen Teil abgeben, sondern sind nur, nach Verschiedenheit der Subjekte der Anwendung des Tugendprinzips (dem Formale nach) auf in der Erfahrung vorkommende Fälle (das Materiale) modifizierte, Regeln, weshalb sie auch, wie alle empirische Einteilungen, keine gesichert-vollständige Klassifikation zulassen. Indessen, gleichwie von der Metaphysik der Natur zur Physik ein Überschritt, der seine besondern Regeln hat, verlangt wird: so wird der Metaphysik der Sitten ein Ähnliches mit Recht angesonnen: nämlich durch Anwendung reiner Pflichtprinzipien auf Fälle der Erfahrung jene gleichsam  zu schematisieren und zum moralisch-praktischen Gebrauch fertig darzulegen, – Welches Verhalten also gegen Menschen, z.B. in der moralischen Reinigkeit ihres Zustandes, oder in ihrer Verdorbenheit; welches im kultivierten, oder rohen Zustände; was den Gelehrten oder Ungelehrten, und jenen im Gebrauch ihrer Wissenschaft als umgänglichen (geschliffenen), oder in ihrem Fach unumgänglichen Gelehrten (Pedanten), pragmatischen oder mehr auf Geist und \match{Geschmack} ausgehenden; welches nach Verschiedenheit der Stände, des Alters, des Geschlechts, des Gesundheitszustandes, des der Wohlhabenheit oder Armut u.s.w. zukomme: das gibt nicht so vielerlei Arten der ethischen Verpflichtung (denn es ist nur eine, nämlich die der Tugend überhaupt), sondern nur Arten der Anwendung (Porismen) ab; die also nicht, als Abschnitte der Ethik und Glieder der Einteilung eines Systems (das a priori aus einem Vernunftbegriffe hervorgehen muß), aufgeführt, sondern nur angehängt werden können. – Aber eben diese Anwendung gehört zur Vollständigkeit der Darstellung desselben. 
	
	\unnumberedsection{Große (3)} 
	\subsection*{tg430.2.56} 
	\textbf{Source : }Die Metaphysik der Sitten/Erster Teil. Metaphysische Anfangsgründe der Rechtslehre/Einleitung in die Metaphysik der Sitten\\  
	
	\noindent\textbf{Paragraphe : }
	Subjektiv ist der Grad der Zurechnungsfähigkeit (imputabilitas) der Handlungen nach der \match{Größe} der Hindernisse zu schätzen, die dabei haben überwunden werden müssen. – Je größer die Naturhindernisse (der Sinnlichkeit), je kleiner das moralische Hindernis (der Pflicht), desto mehr wird die gute Tat zum Verdienst angerechnet. Z.B. wenn ich einen mir ganz fremden Menschen mit meiner beträchtlichen Aufopferung aus großer Not rette. 
	
	\subsection*{tg445.2.14} 
	\textbf{Source : }Die Metaphysik der Sitten/Erster Teil. Metaphysische Anfangsgründe der Rechtslehre/Anhang erläutender Bemerkungen zu den metaphysischen Anhangsgründen der Rechtslehre\\  
	
	\noindent\textbf{Paragraphe : }Ob nun jener Begriff »als neues Phänomen am juristischen Himmel« eine stella mirabilis (eine bis zum Stern erster \match{Größe} wachsende, vorher nie gesehene, allmählich aber wieder verschwindende,  vielleicht einmal wiederkehrende Erscheinung), oder bloß eine Sternschnuppe sei? das soll jetzt untersucht werden. 
	
	\subsection*{tg472.2.36} 
	\textbf{Source : }Die Metaphysik der Sitten/Zweiter Teil. Metaphysische Anfangsgründe der Tugendlehre/I. Ethische Elementarlehre/I. Teil. Von den Pflichten gegen sich selbst überhaupt/Erstes Buch. Von den vollkommenen Pflichten gegen sich selbst\\  
	
	\noindent\textbf{Paragraphe : }Das Bewußtsein und Gefühl der Geringfähigkeit seines moralischen Werts in Vergleichung mit dem Gesetz
	ist die Demut (humilitas moralis). Die Überredung von einer \match{Größe} dieses seinen Werts, aber nur aus Mangel der Vergleichung mit dem Gesetz, kann der Tugendstolz (arrogantia moralis) genannt werden. – Die Entsagung alles Anspruchs auf irgend einen moralischen Wert seiner selbst, in der Überredung, sich eben dadurch einen geborgten zu erwerben, ist die sittlich-falsche Kriecherei (humilitas spuria). 
	
	\unnumberedsection{Mantel (1)} 
	\subsection*{tg439.2.16} 
	\textbf{Source : }Die Metaphysik der Sitten/Erster Teil. Metaphysische Anfangsgründe der Rechtslehre/1. Teil. Das Privatrecht vom äußeren Mein und Dein überhaupt/3. Hauptstück. Von der subjektiv-bedingten Erwerbung durch den Ausspruch einer öffentlichen Gerichtsbarkeit\\  
	
	\noindent\textbf{Paragraphe : }Wenn ich, z.B. bei einfallendem Regen, in ein Haus eintrete, und erbitte mir einen \match{Mantel} zu leihen, der aber, etwa durch unvorsichtige Ausgießung abfärbender Materien aus dem Fenster, auf immer verdorben, oder, wenn er, indem ich ihn in einem anderen Hause, wo ich eintrete, ablege, mir gestohlen wird, so muß doch die Behauptung jedem Menschen als ungereimt auffallen, ich hätte nichts weiter zu tun, als jenen, so wie er ist, zurückzuschicken, oder den geschehenen Diebstahl nur zu melden; allenfalls sei es noch eine Höflichkeit, den Eigentümer dieses Verlustes wegen zu beklagen, da er aus seinem Recht nichts fordern könne. – Ganz anders lautet es, wenn ich bei der Erbittung dieses Gebrauchs zugleich auf den Fall, daß die Sache unter meinen Händen verunglückte, mir zum voraus erbäte, auch diese Gefahr zu übernehmen, weil ich arm und den Verlust zu ersetzen unvermögend wäre. Niemand wird das letztere überflüssig und lächerlich finden, außer etwa, wenn der Anleihende ein bekanntlich vermögender und wohldenkender Mann wäre, weil es alsdann beinahe Beleidigung sein würde, die großmütige Erfassung meiner Schuld in diesem Falle nicht zu präsumieren. 
	
	\unnumberedsection{Maß (6)} 
	\subsection*{tg458.2.4} 
	\textbf{Source : }Die Metaphysik der Sitten/Zweiter Teil. Metaphysische Anfangsgründe der Tugendlehre/Einleitung/X. Das oberste Prinzip der Rechtslehre war analytisch; das der Tugendlehre ist synthetisch\\  
	
	\noindent\textbf{Paragraphe : }Diese Erweiterung des Pflichtbegriffs über den der äußeren Freiheit und der Einschränkung derselben durch das bloße Förmliche ihrer durchgängigen Zusammenstimmung, wo die innere Freiheit, statt des Zwanges von außen, das Vermögen des Selbstzwanges und zwar nicht vermittelst anderer Neigungen, sondern durch reine praktische Vernunft (welche alle diese Vermittelung verschmäht), aufgestellt wird, besteht darin und erhebt sich dadurch über die Rechtspflicht: daß durch sie Zwecke aufgestellet werden, von denen überhaupt das Recht abstrahiert. – Im moralischen Imperativ, und der notwendigen Voraussetzung der Freiheit zum Behuf desselben, machen: das Gesetz, das Vermögen (es zu erfüllen) und der die Maxime bestimmende Wille alle Elemente aus, welche den Begriff der Rechtspflicht bilden. Aber in demjenigen, welcher die Tugendpflicht
	gebietet, kommt, noch über den Begriff eines Selbstzwanges, der eines Zwecks dazu, nicht den wir haben, sondern haben sollen, den also die reine praktische Vernunft in sich hat, deren höchster, unbedingter Zweck (der aber doch immer noch Pflicht ist) darin gesetzt wird: daß die Tugend ihr eigener Zweck und, bei dem Verdienst, das sie um den Menschen hat, auch ihr eigener Lohn sei. (Wobei sie, als Ideal, so glänzt, daß sie nach menschlichem Augenmaß die Heiligkeit selbst, die zur Übertretung nie versucht wird, zu verdunkeln scheint;
	
	
	16
	welches gleichwohl eine Täuschung ist, da, weil wir kein \match{Maß} für den Grad einer Stärke, als die Größe der Hindernisse haben, die da haben überwunden werden können (welche in uns die Neigungen sind ), wir die subjektive Bedingungen der Schätzung einer Größe für die objektive der Größe an sich selbst zu halten verleitet werden.) Aber mit menschlichen Zwecken, die insgesamt ihre zu bekämpfenden Hindernisse haben, verglichen, hat es seine Richtigkeit, daß der Wert der Tugend selbst, als ihres eigenen Zwecks, den Wert alles Nutzens und aller empirischen Zwecke und Vorteile weit überwiege, die sie zu ihrer Folge immerhin haben mag. 
	
	\subsection*{tg471.2.50} 
	\textbf{Source : }Die Metaphysik der Sitten/Zweiter Teil. Metaphysische Anfangsgründe der Tugendlehre/I. Ethische Elementarlehre/I. Teil. Von den Pflichten gegen sich selbst überhaupt/Erstes Buch. Von den vollkommenen Pflichten gegen sich selbst/Erstes Hauptstück. Die Pflicht des Menschen gegen sich selbst, als einem animalischen Wesen\\  
	
	\noindent\textbf{Paragraphe : }Kann man dem Wein, wenn gleich nicht als Panegyrist, doch wenigstens als Apologet, einen Gebrauch verstatten, der bis nahe an die Berauschung reicht; weil er doch die Gesellschaft zur Gesprächigkeit belebt, und damit Offenherzigkeit verbindet? – Oder kann man ihm wohl gar das Verdienst zugestehen, das zu befördern, was Seneca vom Cato rühmt: virtus eius incaluit mero? – Der Gebrauch des Opium und Branntweins sind, als Genießmittel, der Niederträchtigkeit näher, weil sie, bei dem geträumten Wohlbefinden, stumm, zurückhaltend und unmitteilbar machen, daher auch nur als Arzneimittel erlaubt sind. – Wer kann aber das \match{Maß} für einen bestimmen, der in den Zustand, wo er zum Messen keine klare Augen mehr hat, überzugehen eben in Bereitschaft ist? Der Mohammedanism, welcher den Wein ganz verbietet, hat also sehr schlecht gewählt, dafür das Opium zu erlauben. 
	
	\subsection*{tg472.2.22} 
	\textbf{Source : }Die Metaphysik der Sitten/Zweiter Teil. Metaphysische Anfangsgründe der Tugendlehre/I. Ethische Elementarlehre/I. Teil. Von den Pflichten gegen sich selbst überhaupt/Erstes Buch. Von den vollkommenen Pflichten gegen sich selbst\\  
	
	\noindent\textbf{Paragraphe : }Ich verstehe hier unter diesem Namen nicht den habsüchtigen Geiz (der Erweiterung seines Erwerbs der Mittel zum Wohlleben, über die Schranken des wahren Bedürfnisses); denn dieser kann auch als bloße Verletzung seiner Pflicht (der Wohltätigkeit) gegen andere betrachtet werden;  auch nicht den kargen Geiz, welcher, wenn er schimpflich ist, Knickerei oder Knauserei genannt wird, aber doch bloß Vernachlässigung seiner Liebespflichten gegen andere sein kann; sondern die Verengung seines eigenen Genusses der Mittel zum Wohlleben unter das \match{Maß} des wahren eigenen Bedürfnisses; dieser Geiz ist es eigentlich, der hier gemeint ist, welcher der Pflicht gegen sich selbst widerstreitet. 
	
	\subsection*{tg472.2.25} 
	\textbf{Source : }Die Metaphysik der Sitten/Zweiter Teil. Metaphysische Anfangsgründe der Tugendlehre/I. Ethische Elementarlehre/I. Teil. Von den Pflichten gegen sich selbst überhaupt/Erstes Buch. Von den vollkommenen Pflichten gegen sich selbst\\  
	
	\noindent\textbf{Paragraphe : }Nicht das \match{Maß} der Ausübung sittlicher Maximen, sondern das objektive Prinzip derselben, muß als verschieden erkannt und vorgetragen werden, wenn ein Laster von der Tugend unterschieden werden soll. – Die Maxime des habsüchtigen Geizes (als Verschwenders) ist: alle Mittel des Wohllebens in der Absicht auf den Genuß anzuschaffen und zu erhalten. – Die des kargen Geizes ist hingegen der Erwerb so wohl, als die Erhaltung aller Mittel des Wohllebens, aber ohne Absicht auf den Genuß (d.i. ohne daß dieser, sondern nur der Besitz der Zweck sei). 
	
	\subsection*{tg477.2.4} 
	\textbf{Source : }Die Metaphysik der Sitten/Zweiter Teil. Metaphysische Anfangsgründe der Tugendlehre/I. Ethische Elementarlehre/I. Teil. Von den Pflichten gegen sich selbst überhaupt/2. Buch: Die Pflichten gegen sich selbst/Erster Abschnitt. Von der Pflicht gegen sich selbst in Entwickelung und Vermehrung seiner Naturvollkommenheit, d.i. in pragmatischer Absicht\\  
	
	\noindent\textbf{Paragraphe : }Der Anbau (cultura) seiner Naturkräfte (Geistes-, Seelen- und Leibeskräfte), als Mittel zu allerlei möglichen Zwecken ist Pflicht des Menschen gegen sich selbst. – Der Mensch ist es sich selbst (als einem Vernunftwesen) schuldig, die Naturanlage und Vermögen, von denen seine Vernunft dereinst Gebrauch machen kann, nicht unbenutzt und gleichsam rosten zu lassen, sondern, gesetzt daß er auch mit dem angebornen \match{Maß} seines Vermögens für die natürlichen Bedürfnisse zufrieden sein könne, so muß ihm doch seine Vernunft dieses Zufriedensein, mit dem geringen Maß seiner Vermögen, erst durch Grundsätze anweisen, weil er, als ein Wesen, das der Zwecke (sich Gegenstände zum Zweck zu machen) fähig ist, den Gebrauch seiner Kräfte nicht bloß dem Instinkt der Natur, sondern der Freiheit, mit der er dieses Maß bestimmt, zu verdanken haben muß. Es ist also nicht Rücksicht auf den Vorteil, den die Kultur seines Vermögens (zu allerlei Zwecken) verschaffen kann; denn dieser würde vielleicht (nach Rousseauschen Grundsätzen) für die Rohigkeit des Naturbedürfnisses vorteilhaft ausfallen: sondern es ist Gebot der moralisch-praktischen Vernunft und Pflicht des Menschen gegen sich selbst, seine  Vermögen (unter denselben eins mehr als das andere, nach Verschiedenheit seiner Zwecke) anzubauen, und in pragmatischer Rücksicht ein dem Zweck seines Daseins angemessener Mensch zu sein. 
	
	\subsection*{tg487.2.5} 
	\textbf{Source : }Die Metaphysik der Sitten/Zweiter Teil. Metaphysische Anfangsgründe der Tugendlehre/II. Ethische Methodenlehre/2. Abschnitt. Die ethische Asketik\\  
	
	\noindent\textbf{Paragraphe : }Die Kultur der Tugend, d.i. die moralische Asketik, hat, in Ansehung des Prinzips der rüstigen, mutigen und wackeren Tugendübung den Wahlspruch der Stoiker: gewöhne dich, die zufälligen Lebensübel zu ertragen und die eben so überflüssigen Ergötzlichkeiten zu entbehren (assuesce incommodis et desuesce commoditatibus vitae). Es ist eine Art von Diätetik für den Menschen, sich moralisch gesund zu erhalten. Gesundheit ist aber nur ein negatives  Wohlbefinden, sie selber kann nicht gefühlt werden. Es muß etwas dazu kommen, was einen angenehmen Lebensgenuß gewährt und doch bloß moralisch ist. Das ist das jederzeit fröhliche Herz in der Idee des tugendhaften Epikurs. Denn wer sollte wohl mehr Ursache haben, frohen Muts zu sein und nicht darin selbst eine Pflicht finden, sich in eine fröhliche Gemütsstimmung zu versetzen und sie sich habituell zu machen, als der, welcher sich keiner vorsätzlichen Übertretung bewußt und, wegen des Verfalls in eine solche, gesichert ist (hic murus ahenëus esto etc. Horat.). – Die Mönchsasketik hingegen, welche aus abergläubischer Furcht, oder geheucheltem Abscheu an sich selbst, mit Selbstpeinigung und Fleischeskreuzigung zu Werke geht, zweckt auch nicht auf Tugend, sondern auf schwärmerische Entsündigung ab, sich selbst Strafe aufzulegen und, anstatt sie moralisch (d.i. in Absicht auf die Besserung) zu bereuen, sie büßen zu wollen; welches, bei einer selbstgewählten und an sich vollstreckten Strafe (denn die muß immer ein anderer auflegen), ein Widerspruch ist, und kann auch den Frohsinn, der die Tugend begleitet, nicht bewirken, vielmehr nicht ohne geheimen Haß gegen das Tugendgebot statt finden. – Die ethische Gymnastik besteht also nur in der Bekämpfung der Naturtriebe, die das \match{Maß} erreicht, über sie bei vorkommenden, der Moralität Gefahr drohenden, Fällen Meister werden zu können; mithin die wacker und, im Bewußtsein seiner wiedererworbenen Freiheit, fröhlich macht. Etwas bereuen (welches bei der Rückerinnerung ehemaliger Übertretungen unvermeidlich, ja wobei diese Erinnerung nicht schwinden zu lassen es so gar Pflicht ist) und sich eine Pönitenz auferlegen (z.B. das Fasten), nicht in diätetischer, sondern frommer Rücksicht, sind zwei sehr verschiedene, moralisch gemeinte, Vorkehrungen, von denen die letztere, welche freudenlos, finster und mürrisch ist, die Tugend selbst verhaßt macht und ihre Anhänger verjagt. Die Zucht (Disziplin), die der Mensch an sich selbst verübt, kann daher nur durch den Frohsinn, der sie begleitet, verdienstlich und exemplarisch werden. 
	
	\unnumberedsection{Mode (1)} 
	\subsection*{tg482.2.47} 
	\textbf{Source : }Die Metaphysik der Sitten/Zweiter Teil. Metaphysische Anfangsgründe der Tugendlehre/I. Ethische Elementarlehre/II. Teil. Von den Tugendpflichten gegen andere/Erstes Hauptstück. Von den Pflichten gegen andere, bloß als Menschen/Zweiter Abschnitt. Von den Tugendpflichten gegen andere Menschen aus der ihnen gebührenden Achtung\\  
	
	\noindent\textbf{Paragraphe : }Die leichtfertige Tadelsucht und der Hang, andere zum Gelächter bloß zu stellen, die Spottsucht, um die Fehler eines anderen zum unmittelbaren Gegenstande seiner Belustigung zu machen, ist Bosheit, und von dem 
	Scherz, der Vertraulichkeit unter Freunden, sie nur zum Schein als Fehler, in der Tat aber als Vorzüge des Muts, bisweilen auch außer der Regel der \match{Mode} zu sein, zu belachen (welches dann kein Hohnlachen ist), gänzlich unterschieden. Wirkliche Fehler aber, oder, gleich als ob sie wirklich wären, angedichtete, welche die Person ihrer verdienten Achtung zu berauben abgezweckt sind, dem Gelächter bloß zu stellen, und der Hang dazu, die bittere Spottsucht (spiritus causticus), hat etwas von teuflischer Freude an sich und ist darum eben eine desto härtere Verletzung der Pflicht der Achtung gegen andere Menschen. 
	
	\unnumberedsection{Reinigung (1)} 
	\subsection*{tg443.2.7} 
	\textbf{Source : }Die Metaphysik der Sitten/Erster Teil. Metaphysische Anfangsgründe der Rechtslehre/2. Teil. Das öffentliche Recht/3. Abschnitt. Das Weltbürgerrecht\\  
	
	\noindent\textbf{Paragraphe : }Wenn Anbauung in solcher Entlegenheit vom Sitz des ersteren geschieht, daß keines derselben im Gebrauch seines Bodens dem anderen Eintrag tut, so ist das Recht dazu nicht zu bezweifeln; wenn es aber Hirten- oder Jagdvölker  sind (wie die Hottentotten, Tungusen und die meisten amerikanischen Nationen), deren Unterhalt von großen öden Landstrecken abhängt, so würde dies nicht mit Gewalt, sondern nur durch Vertrag, und selbst dieser nicht mit Benutzung der Unwissenheit jener Einwohner in Ansehung der Abtretung solcher Ländereien, geschehen können; obzwar die Rechtfertigungsgründe scheinbar genug sind, daß eine solche Gewalttätigkeit zum Weltbesten gereiche; teils durch Kultur roher Völker (wie der Vorwand, durch den selbst Büsching die blutige Einführung der christlichen Religion in Deutschland entschuldigen will), teils zur \match{Reinigung} seines eigenen Landes von verderbten Menschen und gehoffter Besserung derselben, oder ihrer Nachkommenschaft, in einem anderen Weltteile (wie in Neuholland); denn alle diese vermeintlich gute Absichten können doch den Flecken der Ungerechtigkeit in den dazu gebrauchten Mitteln nicht abwaschen. – Wendet man hiegegen ein: daß, bei solcher Bedenklichkeit, mit der Gewalt den Anfang zu Gründung eines gesetzlichen Zustandes zu machen, vielleicht die ganze Erde noch in gesetzlosem Zustande sein würde: so kann das eben so wenig jene Rechtsbedingung aufheben, als der Vorwand der Staatrevolutionisten, daß es auch, wenn Verfassungen verunartet sind, dem Volk zustehe, sie mit Gewalt umzuformen, und überhaupt einmal für allemal ungerecht zu sein, um nachher die Gerechtigkeit desto sicherer zu gründen und aufblühen zu machen. 
	
	\unnumberedsection{Rolle (1)} 
	\subsection*{tg438.2.24} 
	\textbf{Source : }Die Metaphysik der Sitten/Erster Teil. Metaphysische Anfangsgründe der Rechtslehre/1. Teil. Das Privatrecht vom äußeren Mein und Dein überhaupt/2. Hauptstück. Von der Art, etwas Äußeres zu erwerben/Episodischer Abschnitt. Von der idealen Erwerbung eines äußeren Gegenstandes der Willkür\\  
	
	\noindent\textbf{Paragraphe : }Daß durch ein tadelloses Leben und einen dasselbe beschließenden Tod der Mensch einen (negativ-) guten Namen  als das Seine, welches ihm übrig bleibt, erwerbe, wenn er als homo phaenomenon nicht mehr existiert, und daß die Überlebenden (angehörige, oder fremde) ihn auch vor Recht zu verteidigen befugt sind (weil unerwiesene Anklage sie insgesamt wegen ähnlicher Begegnung auf ihren Sterbefall in Gefahr bringt), daß er, sage ich, ein solches Recht erwerben könne, ist eine sonderbare, nichtsdestoweniger unleugbare Erscheinung der a priori gesetzgebenden Vernunft, die ihr Gebot und Verbot auch über die Grenze des Lebens hinaus erstreckt. – Wenn jemand von einem Verstorbenen ein Verbrechen verbreitet, das diesen im Leben ehrlos, oder nur verächtlich gemacht haben würde: so kann ein jeder, welcher einen Beweis führen kann, daß diese Beschuldigung vorsätzlich unwahr und gelogen sei, den, welcher jenen in böse Nachrede bringt, für einen Kalumnianten öffentlich erklären, mithin ihn selbst ehrlos machen; welches er nicht tun dürfte, wenn er nicht mit Recht voraussetzte, daß der Verstorbene dadurch beleidigt wäre, ob er gleich tot ist, und daß diesem durch jene Apologie Genugtuung widerfahre, ob er gleich nicht mehr existiert.
	
	
	6
	Die Befugnis, die \match{Rolle}  des Apologeten für den Verstorbenen zu spielen, darf dieser auch nicht beweisen; denn jeder Mensch maßt sie sich unvermeidlich an, als nicht bloß zur Tugendpflicht (ethisch betrachtet), sondern so gar zum Recht der Menschheit überhaupt gehörig: und es bedarf hiezu keiner besonderen persönlichen Nachteile, die etwa Freunden und Anverwandten aus einem solchen Schandfleck am Verstorbenen erwachsen dürften, um jenen zu einer solchen Rüge zu berechtigen. – Daß also eine solche ideale Erwerbung und ein Recht des Menschen nach seinem Tode gegen die Überlebenden gegründet sei, ist nicht zu streiten, ob schon die Möglichkeit desselben keiner Deduktion fähig ist. 
	
	\unnumberedsection{Schmuck (1)} 
	\subsection*{tg437.2.80} 
	\textbf{Source : }Die Metaphysik der Sitten/Erster Teil. Metaphysische Anfangsgründe der Rechtslehre/1. Teil. Das Privatrecht vom äußeren Mein und Dein überhaupt/2. Hauptstück. Von der Art, etwas Äußeres zu erwerben/3. Abschnitt. Von dem auf dingliche Art persönlichen Recht\\  
	
	\noindent\textbf{Paragraphe : }Wie ist es aber möglich, daß das, was anfänglich Ware war, endlich Geld ward? Wenn ein großer und machthabender Vertuer einer Materie, die er anfangs bloß zum \match{Schmuck} und Glanz seiner Diener (des Hofes) brauchte (z.B. Gold, Silber, Kupfer, oder eine Art schöner Muschelschalen, Kauris, oder auch, wie in Kongo, eine Art Matten, Makuten genannt, oder, wie am Senegal, Eisenstangen, und auf der Guineaküste selbst Negersklaven), d.i. wenn ein Landesherr die Abgaben von seinen Untertanen in dieser Materie (als Ware) einfordert, und die, deren Fleiß in Anschaffung derselben  dadurch bewegt werden soll, mit eben denselben, nach Verordnungen des Verkehrs unter und mit ihnen überhaupt (auf einem Markt, oder einer Börse), wieder lohnt. – Dadurch allein hat (meinem Bedünken nach) eine Ware ein gesetzliches Mittel des Verkehrs des Fleißes der Untertanen unter einander und hiemit auch des Staatsreichtums, d.i. Geld, werden können. 
	
	\unnumberedsection{Stoff (1)} 
	\subsection*{tg437.2.93} 
	\textbf{Source : }Die Metaphysik der Sitten/Erster Teil. Metaphysische Anfangsgründe der Rechtslehre/1. Teil. Das Privatrecht vom äußeren Mein und Dein überhaupt/2. Hauptstück. Von der Art, etwas Äußeres zu erwerben/3. Abschnitt. Von dem auf dingliche Art persönlichen Recht\\  
	
	\noindent\textbf{Paragraphe : }Die Verwechselung des persönlichen Rechts mit dem Sachenrecht ist noch in einem anderen, unter den Verdingungsvertrag gehörigen, Falle (B, II, α), nämlich dem der Einmietung (ius incolatus), ein \match{Stoff} zu Streitigkeiten. – Es fragt sich nämlich: ist der Eigentümer, wenn er sein an jemanden vermietetes Haus (oder seinen Grund) vor Ablauf der Mietszeit an einen anderen verkauft, verbunden, die Bedingung der fortdauernden Miete dem Kaufkontrakte beizufügen, oder kann man sagen: Kauf bricht Miete (doch in einer durch den Gebrauch bestimmten Zeit der Aufkündigung)? – Im ersteren Fall hätte das Haus wirklich eine Belästigung (onus) auf sich liegend, ein Recht in dieser Sache, das der Mieter sich an derselben (dem Hause) erworben hätte; welches auch wohl geschehen kann (durch Ingrossation des Mietskontrakts auf das Haus), aber alsdenn kein bloßer Mietskontrakt sein würde, sondern wozu noch ein anderer Vertrag (dazu sich nicht viel Vermieter verstehen würden) hinzukommen müßte. Also gilt der Satz: »Kauf bricht Miete «, d.i. das volle Recht in einer Sache (das Eigentum) überwiegt alles persönliche Recht, was mit ihm  nicht zusammen bestehen kann; wobei doch die Klage aus dem Grunde des letzteren dem Mieter offen bleibt, ihn wegen des aus der Zerreißung des Kontrakts entspringenden Nachteils schadenfrei zu halten. 
	
	\unnumberedchapter{Zoologie} 
	\unnumberedsection{Art (64)} 
	\subsection*{tg430.2.15} 
	\textbf{Source : }Die Metaphysik der Sitten/Erster Teil. Metaphysische Anfangsgründe der Rechtslehre/Einleitung in die Metaphysik der Sitten\\  
	
	\noindent\textbf{Paragraphe : }Wenn die Sittenlehre nichts als Glückseligkeitslehre wäre, so würde es ungereimt sein, zum Behuf derselben sich nach Prinzipien a priori umzusehen. Denn so scheinbar es immer auch lauten mag: daß die Vernunft noch vor der Erfahrung einsehen könne, durch welche Mittel man zum dauerhaften Genuß wahrer Freuden des Lebens gelangen könne, so ist doch alles, was man darüber a priori lehrt, entweder tautologisch, oder ganz grundlos angenommen. Nur die Erfahrung kann lehren, was uns Freude bringe. Die natürlichen Triebe zur Nahrung, zum Geschlecht, zur Ruhe, zur Bewegung, und (bei der Entwickelung unserer Naturanlagen) die Triebe zur Ehre, zur Erweiterung unserer Erkenntnis u. d. gl., können allein und einem jeden nur auf seine besondere \match{Art} zu erkennen geben, worin er jene Freuden zu setzen, ebendieselbe kann ihm auch die Mittel lehren, wodurch er sie zu suchen habe. Alles scheinbare Vernünfteln a priori ist hier im Grunde nichts, als durch Induktion zur Allgemeinheit erhobene Erfahrung, welche Allgemeinheit (secundum principia generalia non universalia) noch dazu so kümmerlich ist, daß man einem jeden unendlich viel Ausnahmen erlauben muß, um jene Wahl seiner Lebensweise seiner besondern Neigung und seiner Empfänglichkeit für die Vergnügen anzupassen, und am Ende doch nur durch seinen, oder anderer ihren Schaden klug zu werden. 
	
	\subsection*{tg430.2.17} 
	\textbf{Source : }Die Metaphysik der Sitten/Erster Teil. Metaphysische Anfangsgründe der Rechtslehre/Einleitung in die Metaphysik der Sitten\\  
	
	\noindent\textbf{Paragraphe : }Wenn daher ein System der Erkenntnis a priori aus bloßen Begriffen Metaphysik heißt, so wird eine praktische Philosophie, welche nicht Natur, sondern die Freiheit der Willkür zum Objekte hat, eine Metaphysik der Sitten voraussetzen und bedürfen: d.i. eine solche zu haben ist selbst Pflicht, und jeder Mensch hat sie auch, obzwar gemeiniglich nur auf dunkle \match{Art} in sich; denn wie könnte er, ohne Prinzipien a priori, eine allgemeine Gesetzgebung in sich zu haben glauben? So wie es aber in einer Metaphysik der Natur auch Prinzipien der Anwendung jener allgemeinen obersten Grundsätze von einer Natur überhaupt auf Gegenstände der Erfahrung geben muß, so wird es auch eine Metaphysik der Sitten daran nicht können mangeln lassen, und wir werden oft die besondere Natur des Menschen, die nur durch Erfahrung erkannt wird, zum Gegenstande nehmen müssen, um an ihr die Folgerungen aus den allgemeinen moralischen Prinzipien zu zeigen, ohne daß jedoch dadurch der Reinigkeit der letzteren etwas benommen,  noch ihr Ursprung a priori dadurch zweifelhaft gemacht wird. – Das will so viel sagen, als: eine Metaphysik der Sitten kann nicht auf Anthropologie gegründet, aber doch auf sie angewandt werden. 
	
	\subsection*{tg430.2.23} 
	\textbf{Source : }Die Metaphysik der Sitten/Erster Teil. Metaphysische Anfangsgründe der Rechtslehre/Einleitung in die Metaphysik der Sitten\\  
	
	\noindent\textbf{Paragraphe : }Alle Gesetzgebung also (sie mag auch in Ansehung der Handlung, die sie zur Pflicht macht, mit einer anderen übereinkommen, z.B. die Handlungen mögen in allen Fällen äußere sein) kann doch in Ansehung der Triebfedern unterschieden sein. Diejenige, welche eine Handlung zur Pflicht, und diese Pflicht zugleich zur Triebfeder macht, ist ethisch. Diejenige aber, welche das letztere nicht im Gesetze mit einschließt, mithin auch eine andere Triebfeder, als die Idee der Pflicht selbst, zuläßt, ist juridisch. Man sieht in Ansehung der letztern leicht ein, daß diese von der Idee der Pflicht unterschiedene Triebfeder von den pathologischen Bestimmungsgründen der Willkür der Neigungen und Abneigungen und unter diesen von denen der letzteren \match{Art} hergenommen sein müssen, weil es eine Gesetzgebung, welche nötigend, nicht eine Anlockung, die einladend ist, sein soll. 
	
	\subsection*{tg430.2.27} 
	\textbf{Source : }Die Metaphysik der Sitten/Erster Teil. Metaphysische Anfangsgründe der Rechtslehre/Einleitung in die Metaphysik der Sitten\\  
	
	\noindent\textbf{Paragraphe : }
	Die ethische Gesetzgebung (die Pflichten mögen allenfalls auch äußere sein) ist diejenige, welche nicht äußerlich sein kann; die juridische ist, welche auch äußerlich sein kann. So ist es eine äußerliche Pflicht, sein vertragsmäßiges Versprechen zu halten; aber das Gebot, dieses bloß darum zu tun, weil es Pflicht ist, ohne auf eine andere Triebfeder Rücksicht zu nehmen, ist bloß zur innern Gesetzgebung gehörig. Also nicht als besondere \match{Art} von Pflicht (eine besondere Art Handlungen, zu denen man verbunden ist) – denn es ist in der Ethik sowohl als im Rechte eine äußere Pflicht –, sondern weil die Gesetzgebung, im angeführten Falle, eine innere ist und keinen äußeren Gesetzgeber haben kann, wird die Verbindlichkeit zur Ethik gezählt. Aus eben dem Grunde werden die Pflichten des Wohlwollens, ob sie gleich äußere Pflichten (Verbindlichkeiten zu äußeren Handlungen) sind, doch zur Ethik gezählt, weil ihre Gesetzgebung nur innerlich sein kann. – Die Ethik hat freilich auch ihre besondern Pflichten (z.B. die gegen sich selbst), aber hat doch auch mit dem Rechte Pflichten, aber nur nicht die Art der Verpflichtung gemein. Denn Handlungen bloß darum, weil es Pflichten sind, ausüben, und den Grundsatz der Pflicht selbst, woher sie auch komme, zur hinreichenden Triebfeder der Willkür zu machen, ist das Eigentümliche der ethischen Gesetzgebung. So gibt es also zwar viele direkt-ethische Pflichten, aber die innere Gesetzgebung macht auch die übrigen, alle und insgesamt, zu indirekt-ethischen. 
	
	\subsection*{tg430.2.31} 
	\textbf{Source : }Die Metaphysik der Sitten/Erster Teil. Metaphysische Anfangsgründe der Rechtslehre/Einleitung in die Metaphysik der Sitten\\  
	
	\noindent\textbf{Paragraphe : }Auf diesem (in praktischer Rücksicht) positiven Begriffe der Freiheit gründen sich unbedingte praktische Gesetze, welche moralisch heißen, die in Ansehung unser, deren Willkür sinnlich affiziert und so dem reinen Willen nicht von selbst angemessen, sondern oft widerstrebend ist, Imperativen (Gebote oder Verbote) und zwar kategorische (unbedingte) Imperativen sind, wodurch sie sich von den technischen (den Kunst-Vorschriften), als die jederzeit nur bedingt gebieten, unterscheiden, nach denen gewisse Handlungen erlaubt oder unerlaubt, d.i. moralisch möglich oder unmöglich, einige derselben aber, oder ihr Gegenteil moralisch notwendig, d.i. verbindlich sind, woraus dann für jene der Begriff einer Pflicht entspringt, deren Befolgung oder Übertretung zwar auch mit einer Lust oder Unlust von besonderer \match{Art} (der eines moralischen Gefühls) verbunden ist, auf welche wir aber (weil sie nicht den Grund der praktischen Gesetze, sondern nur die subjektive Wirkung im Gemüt bei der Bestimmung unserer Willkür durch jene betreffen und (ohne jener ihrer Gültigkeit oder Einflusse objektiv, d.i. im Urteil der Vernunft, etwas hinzuzutun oder zu benehmen) nach Verschiedenheit der Subjekte verschieden sein kann) in praktischen Gesetzen der Vernunft gar nicht Rücksicht nehmen. 
	
	\subsection*{tg430.2.36} 
	\textbf{Source : }Die Metaphysik der Sitten/Erster Teil. Metaphysische Anfangsgründe der Rechtslehre/Einleitung in die Metaphysik der Sitten\\  
	
	\noindent\textbf{Paragraphe : }
	Pflicht ist diejenige Handlung, zu welcher jemand verbunden ist. Sie ist also die Materie der Verbindlichkeit, und es kann einerlei Pflicht (der Handlung nach) sein, ob wir zwar auf verschiedene \match{Art} dazu verbunden werden können. 
	
	\subsection*{tg430.2.41} 
	\textbf{Source : }Die Metaphysik der Sitten/Erster Teil. Metaphysische Anfangsgründe der Rechtslehre/Einleitung in die Metaphysik der Sitten\\  
	
	\noindent\textbf{Paragraphe : }
	Recht oder Unrecht(rectum aut minus rectum) überhaupt ist eine Tat, sofern sie pflichtmäßig oder pflichtwidrig (factum licitum aut illicitum) ist; die Pflicht selbst mag, ihrem Inhalte oder ihrem Ursprunge nach, sein, von welcher \match{Art} sie wolle. Eine pflichtwidrige Tat heißt Übertretung (reatus). 
	
	\subsection*{tg430.2.6} 
	\textbf{Source : }Die Metaphysik der Sitten/Erster Teil. Metaphysische Anfangsgründe der Rechtslehre/Einleitung in die Metaphysik der Sitten\\  
	
	\noindent\textbf{Paragraphe : }Man kann die Lust, welche mit dem Begehren (des Gegenstandes, dessen Vorstellung das Gefühl so affiziert) notwendig verbunden ist, praktische Lust nennen: sie mag nun Ursache oder Wirkung vom Begehren sein. Dagegen würde man die Lust, die mit dem Begehren des Gegenstandes nicht notwendig verbunden ist, die also im Grunde nicht eine Lust an der Existenz des Objekts der Vorstellung ist, sondern bloß an der Vorstellung allein haftet, bloß kontemplative Lust oder untätiges Wohlgefallen nennen können. Das Gefühl der letztern \match{Art} von Lust nennen wir Geschmack. Von diesem wird also in einer praktischen Philosophie, nicht als von einem einheimischen Begriffe, sondern allenfalls nur episodisch die Rede sein. Was aber die praktische Lust betrifft, so wird die Bestimmung des Begehrungsvermögens, vor welcher diese Lust, als Ursache, notwendig vorhergehen muß, im engen Verstande Begierde, die habituelle Begierde aber Neigung heißen, und, weil die Verbindung der Lust mit dem Begehrungsvermögen, sofern diese Verknüpfung durch den Verstand nach einer allgemeinen Regel (allenfalls auch nur für das Subjekt) gültig zu sein geurteilt wird, Interesse heißt, so wird die praktische Lust in diesem Falle ein Interesse der Neigung, dagegen wenn die Lust nur auf eine vorhergehende Bestimmung des Begehrungsvermögens folgen kann, so wird sie eine intellektuelle Lust und das Interesse an dem Gegenstande ein Vernunftinteresse genannt werden müssen; denn wäre das Interesse sinnlich und nicht bloß  auf reine Vernunftprinzipien gegründet, so müßte Empfindung mit Lust verbunden sein und so das Begehrungsvermögen bestimmen können. Obgleich, wo ein bloß reines Vernunftinteresse angenommen werden muß, ihm kein Interesse der Neigung untergeschoben werden kann, so können wir doch, um dem Sprachgebrauche gefällig zu sein, einer Neigung, selbst zu dem, was nur Objekt einer intellektuellen Lust sein kann, ein habituelles Begehren aus reinem Vernunftinteresse einräumen, welche alsdenn aber nicht die Ursache, sondern die Wirkung des letztern Interesse sein würde, und die wir die sinnenfreie Neigung (propensio intellectualis) nennen könnten. 
	
	\subsection*{tg431.2.24} 
	\textbf{Source : }Die Metaphysik der Sitten/Erster Teil. Metaphysische Anfangsgründe der Rechtslehre/Einleitung in die Rechtslehre\\  
	
	\noindent\textbf{Paragraphe : }Das Gesetz eines mit jedermanns Freiheit notwendig zusammenstimmenden wechselseitigen Zwanges, unter dem Prinzip der allgemeinen Freiheit, ist gleichsam die Konstruktion jenes Begriffs, d.i. Darstellung desselben in einer reinen Anschauung a priori, nach der Analogie der Möglichkeit freier Bewegungen der Körper unter dem Gesetze der Gleichheit der Wirkung und Gegenwirkung. So wie wir nun in der reinen Mathematik die Eigenschaften ihres Objekts nicht unmittelbar vom Begriffe ableiten, sondern nur durch die Konstruktion des Begriffs entdecken können, so ist's nicht sowohl der Begriff des Rechts, als vielmehr der, unter allgemeine Gesetze gebrachte, mit ihm zusammenstimmende durchgängig wechselseitige und gleiche Zwang, der die Darstellung jenes Begriffs möglich macht. Dieweil aber diesem dynamischen Begriffe noch ein bloß formaler, in der reinen Mathematik (z.B. der Geometrie) zum Grunde liegt: so hat die Vernunft dafür gesorgt, den Verstand auch mit Anschauungen a priori, zum Behuf der Konstruktion des Rechtsbegriffs, so viel möglich zu versorgen. – Das Rechte (rectum) wird, als das Gerade, teils dem Krummen, teils dem Schiefen entgegen gesetzt. Das erste ist die innere Beschaffenheit einer Linie von der Art, daß es zwischen zweien gegebenen Punkten nur eine einzige, das zweite aber die Lage zweier einander durchschneidenden oder zusammenstoßenden Linien, von deren \match{Art} es auch nur eine einzige (die senkrechte) geben kann, die sich nicht mehr nach einer Seite, als der andern hinneigt, und die den Raum von beiden Seiten gleich abteilt, nach welcher Analogie auch die Rechtslehre das Seine einem jeden (mit mathematischer Genauigkeit) bestimmt wissen will, welches in der Tugendlehre nicht erwartet werden darf, als welche einen gewissen Raum zu Ausnahmen (latitudinem) nicht verweigern kann. – Aber, ohne ins Gebiet der Ethik einzugreifen, gibt es zwei Fälle, die auf Rechtsentscheidung  Anspruch machen, für die aber keiner, der sie entscheide, ausgefunden werden kann, und die gleichsam in Epikurs Intermundia hingehören. – Diese müssen wir zuvörderst aus der eigentlichen Rechtslehre, zu der wir bald schreiten wollen, aussondern, damit ihre schwankenden Prinzipien nicht auf die festen Grundsätze der erstern Einfluß bekommen. 
	
	\subsection*{tg431.2.31} 
	\textbf{Source : }Die Metaphysik der Sitten/Erster Teil. Metaphysische Anfangsgründe der Rechtslehre/Einleitung in die Rechtslehre\\  
	
	\noindent\textbf{Paragraphe : }Die Billigkeit (objektiv betrachtet) ist keinesweges ein Grund zur Aufforderung bloß an die ethische Pflicht anderer (ihr Wohlwollen und Gütigkeit), sondern der, welcher aus diesem Grunde etwas fordert, fußt sich auf sein Recht, nur daß ihm die für den Richter erforderlichen Bedingungen mangeln, nach welchen dieser bestimmen könnte, wie viel, oder auf welche \match{Art} dem Anspruche desselben genug getan werden könne. Der in einer auf gleiche Vorteile eingegangenen Maskopei dennoch mehr getan, dabei aber wohl gar durch Unglücksfälle mehr verloren hat, als die übrigen Glieder, kann nach der Billigkeit von der Gesellschaft  mehr fordern, als bloß zu gleichen Teilen mit ihnen zu gehen. Allein nach dem eigentlichen (strikten) Recht, weil, wenn man sich in seinem Fall einen Richter denkt, dieser keine bestimmte Angaben (data) hat, um, wie viel nach dem Kontrakt ihm zukomme, auszumachen, würde er mit seiner Forderung abzuweisen sein. Der Hausdiener, dem sein bis zu Ende des Jahres laufender Lohn in einer binnen der Zeit verschlechterten Münzsorte bezahlt wird, womit er das nicht ausrichten kann, was er bei Schließung des Kontrakts sich dafür anschaffen konnte, kann, bei gleichem Zahlwert, aber ungleichem Geldwert, sich nicht auf sein Recht berufen, deshalb schadlos gehalten zu werden, sondern nur die Billigkeit zum Grunde aufrufen (eine stumme Gottheit, die nicht gehöret werden kann); weil nichts hierüber im Kontrakt bestimmt war, ein Richter aber nach unbestimmten Bedingungen nicht sprechen kann. 
	
	\subsection*{tg433.2.42} 
	\textbf{Source : }Die Metaphysik der Sitten/Erster Teil. Metaphysische Anfangsgründe der Rechtslehre/1. Teil. Das Privatrecht vom äußeren Mein und Dein überhaupt/1. Hauptstück\\  
	
	\noindent\textbf{Paragraphe : }Die \match{Art} also, etwas außer mir als das Meine zu haben, ist die bloß-rechtliche Verbindung des Willens des Subjekts mit jenem Gegenstande, unabhängig von dem Verhältnisse zu demselben im Raum und in der Zeit, nach dem Begriff eines intelligibelen Besitzes. – Ein Platz auf der Erde ist nicht darum ein äußeres Meine, weil ich ihn mit meinem Leibe einnehme (denn es betrifft hier nur meine äußere Freiheit, mithin nur den Besitz meiner selbst, kein Ding außer mir, und ist also nur ein inneres Recht); sondern, wenn ich ihn noch besitze, ob ich mich gleich von ihm weg und an einen andern Ort begeben habe, nur alsdenn betrifft es mein äußeres Recht, und derjenige, der die fortwährende Besetzung dieses Platzes durch meine Person zur Bedingung machen wollte, ihn als das Meine zu haben, muß entweder behaupten, es sei gar nicht möglich, etwas Äußeres als das Seine zu haben (welches dem Postulat § 2 widerstreitet), oder er verlangt, daß, um dieses zu können, ich in zwei Orten zugleich sei; welches denn aber so viel sagt, als: ich solle an einem Orte sein und auch nicht sein, wodurch er sich selbst widerspricht. 
	
	\subsection*{tg433.2.48} 
	\textbf{Source : }Die Metaphysik der Sitten/Erster Teil. Metaphysische Anfangsgründe der Rechtslehre/1. Teil. Das Privatrecht vom äußeren Mein und Dein überhaupt/1. Hauptstück\\  
	
	\noindent\textbf{Paragraphe : }
	Auflösung: Beide Sätze sind wahr: der erstere, wenn ich den empirischen Besitz (possessio phaenomenon), der andere, wenn ich unter diesem Wort den reinen intelligibelen Besitz (possessio noumenon) verstehe. – Aber die Möglichkeit eines intelligibelen Besitzes, mithin auch des äußeren Mein und Dein läßt sich nicht einsehen,  sondern muß aus dem Postulat der praktischen Vernunft gefolgert werden, wobei es noch besonders merkwürdig ist: daß diese, ohne Anschauungen, selbst ohne einer a priori zu bedürfen, sich durch bloße, vom Gesetz der Freiheit berechtigte, Weglassung empirischer Bedingungen erweitere und so synthetische Rechtssätze a priori aufstellen kann, deren Beweis (wie bald gezeigt werden soll) nachher in praktischer Rücksicht auf analytische \match{Art} geführt werden kann. 
	
	\subsection*{tg433.2.57} 
	\textbf{Source : }Die Metaphysik der Sitten/Erster Teil. Metaphysische Anfangsgründe der Rechtslehre/1. Teil. Das Privatrecht vom äußeren Mein und Dein überhaupt/1. Hauptstück\\  
	
	\noindent\textbf{Paragraphe : }Das Naturrecht im Zustande einer bürgerlichen Verfassung (d.i. dasjenige, was für die letztere aus Prinzipien a priori abgeleitet werden kann) kann durch die statutarischen Gesetze der letzteren nicht Abbruch leiden, und so bleibt das rechtliche Prinzip in Kraft: »der, welcher nach einer Maxime verfährt, nach der es unmöglich wird, einen Gegenstand meiner Willkür als das Meine zu haben, lädiert mich«; denn bürgerliche Verfassung ist allein der rechtliche Zustand, durch welchen jedem das Seine nur gesichert, eigentlich aber nicht ausgemacht und bestimmt wird. – Alle Garantie setzt also das Seine von jemanden (dem es gesichert wird) schon voraus. Mithin muß vor der bürgerlichen Verfassung (oder von ihr abgesehen) ein äußeres Mein und Dein als möglich angenommen werden, und zugleich ein Recht, jedermann, mit dem wir irgend auf eine \match{Art} in Verkehr kommen könnten, zu nötigen, mit uns in eine Verfassung zusammen zu treten, worin jenes gesichert werden kann. – Ein Besitz in Erwartung und Vorbereitung eines solchen Zustandes, der allein auf einem Gesetz des gemeinsamen Willens gegründet werden kann, der also zu der Möglichkeit des letzteren zusammenstimmt, ist ein provisorisch-rechtlicher Besitz, wogegen derjenige, der in  einem solchen wirklichen Zustande angetroffen wird, ein peremtorischer Besitz sein würde. – Vor dem Eintritt in diesen Zustand, zu dem das Subjekt bereit ist, widersteht er denen mit Recht, die dazu sich nicht bequemen und ihn in seinem einstweiligen Besitz stören wollen; weil der Wille aller anderen, außer ihm selbst, der ihm eine Verbindlichkeit aufzulegen denkt, von einem gewissen Besitz abzustehen, bloß einseitig ist, mithin eben so wenig gesetzliche Kraft (als die nur im allgemeinen Willen angetroffen wird) zum Widersprechen hat, als jener zum Behaupten, indessen daß der letztere doch dies voraus hat, zur Einführung und Errichtung eines bürgerlichen Zustandes zusammenzustimmen. – Mit einem Worte: die Art, etwas Äußeres als das Seine im Naturzustande zu haben, ist ein physischer Besitz, der die rechtliche Präsumtion für sich hat, ihn, durch Vereinigung mit dem Willen aller in einer öffentlichen Gesetzgebung, zu einem rechtlichen zu machen, und gilt in der Erwartung komparativ für einen rechtlichen. 
	
	\subsection*{tg434.2.14} 
	\textbf{Source : }Die Metaphysik der Sitten/Erster Teil. Metaphysische Anfangsgründe der Rechtslehre/1. Teil. Das Privatrecht vom äußeren Mein und Dein überhaupt\\  
	
	\noindent\textbf{Paragraphe : }3) Nachdem Rechtsgrunde (titulus) der Erwerbung; welches eigentlich kein besonderes Glied der Einteilung der Rechte, aber doch ein Moment der \match{Art} ihrer Ausübung ist: entweder durch den Akt einer einseitigen, oder doppelseitigen, oder allseitigen Willkür, wodurch etwas Äußeres (facto, pacto, lege) er worben wird. 
	
	\subsection*{tg435.2.19} 
	\textbf{Source : }Die Metaphysik der Sitten/Erster Teil. Metaphysische Anfangsgründe der Rechtslehre/1. Teil. Das Privatrecht vom äußeren Mein und Dein überhaupt/2. Hauptstück. Von der Art, etwas Äußeres zu erwerben/1. Abschnitt. Vom Sachrecht\\  
	
	\noindent\textbf{Paragraphe : }Die Möglichkeit, auf solche \match{Art} zu erwerben, läßt sich auf keine Weise einsehen, noch durch Gründe dartun, sondern ist die unmittelbare Folge aus dem Postulat der praktischen Vernunft. Derselbe Wille aber kann doch eine äußere Erwerbung nicht anders berechtigen, als nur so fern er in einem a priori vereinigten (d.i. durch die Vereinigung der Willkür aller, die in ein praktisches Verhältnis gegen einander kommen können) absolut gebietenden Willen enthalten ist; denn der einseitige Wille (wozu auch der doppelseitige, aber doch besondere Wille gehört) kann nicht jedermann eine Verbindlichkeit auflegen, die an sich zufällig ist, sondern dazu wird ein allseitiger nicht zufällig, sondern a priori, mithin notwendig vereinigter und darum allein gesetzgebender Wille erfordert; denn nur nach dieses seinem Prinzip ist Übereinstimmung der freien Willkür eines jeden mit der Freiheit von jedermann, mithin ein Recht überhaupt, und also auch ein äußeres Mein und Dein möglich. 
	
	\subsection*{tg435.2.26} 
	\textbf{Source : }Die Metaphysik der Sitten/Erster Teil. Metaphysische Anfangsgründe der Rechtslehre/1. Teil. Das Privatrecht vom äußeren Mein und Dein überhaupt/2. Hauptstück. Von der Art, etwas Äußeres zu erwerben/1. Abschnitt. Vom Sachrecht\\  
	
	\noindent\textbf{Paragraphe : }Es ist die Frage: wie weit erstreckt sich die Befugnis der Besitznehmung eines Bodens? So weit, als das Vermögen, ihn in seiner Gewalt zu haben, d.i. als der, so ihn sich zueignen will, ihn verteidigen kann, gleich als ob der Boden spräche: wenn ihr mich nicht beschützen könnt, so könnt ihr mir auch nicht gebieten. Darnach müßte also auch der Streit über das freie oder verschlossene Meer entschieden werden; z.B. innerhalb der Weite, wohin die Kanonen reichen, darf niemand an der Küste eines Landes, das schon einem gewissen  Staat zugehört, fischen, Bernstein aus dem Grunde der See holen, u. dergl. – Ferner: ist die Bearbeitung des Bodens (Bebauung, Beackerung, Entwässerung u. dergl.) zur Erwerbung desselben notwendig? Nein! denn, da diese Formen (der Spezifizierung) nur Akzidenzen sind, so machen sie kein Objekt eines unmittelbaren Besitzes aus, und können zu dem des Subjekts nur gehören, so fern die Substanz vorher als das Seine desselben anerkannt ist. Die Bearbeitung ist, wenn es auf die Frage von der ersten Erwerbung ankommt, nichts weiter als ein äußeres Zeichen der Besitznehmung, welches man durch viele andere, die weniger Mühe kosten, ersetzen kann. – Ferner: darf man wohl jemanden in dem Akt seiner Besitznehmung hindern, so daß keiner von beiden des Rechts der Priorität teilhaftig werde, und so der Boden immer als keinem angehörig frei bleibe? Gänzlich kann diese Hinderung nicht statt finden, weil der andere, um dieses tun zu können, sich doch auch selbst auf irgend einem benachbarten Boden befinden muß, wo er also selbst behindert werden kann zu sein, mithin eine absolute Verhinderung ein Widerspruch wäre; aber respektiv auf einen gewissen (zwischenliegenden) Boden, diesen, als neutral, zur Scheidung zweier Benachbarten unbenutzt liegen zu lassen, würde doch mit dem Rechte der Bemächtigung zusammen bestehen; aber alsdann gehört wirklich dieser Boden beiden gemeinschaftlich, und ist nicht herrenlos (res nullius), eben darum, weil er von beiden dazu gebraucht wird, um sie von einander zu scheiden. – Ferner: kann man auf einem Boden, davon kein Teil das Seine von jemanden ist, doch eine Sache als die seine haben? Ja, wie in der Mongolei jeder sein Gepäcke, was er hat, liegen lassen, oder sein Pferd, was ihm entlaufen ist, als das Seine in seinen Besitz bringen kann, weil der ganze Boden dem Volk, der Gebrauch desselben also jedem einzelnen zusteht; daß aber jemand eine bewegliche Sache auf dem Boden eines anderen als das Seine haben kann, ist zwar möglich, aber nur durch Vertrag. – Endlich ist die Frage: können zwei benachbarte  Völker (oder Familien) einander widerstehen, eine gewisse \match{Art} des Gebrauchs eines Bodens anzunehmen, z.B. die Jagdvölker dem Hirtenvolk, oder den Ackerleuten, oder diese den Pflanzern, u. dergl.? Allerdings; denn die Art, wie sie sich auf dem Erdboden überhaupt ansässig machen wollen, ist, wenn sie sich innerhalb ihrer Grenzen halten, eine Sache des bloßen Beliebens (res merae facultatis). 
	
	\subsection*{tg435.2.27} 
	\textbf{Source : }Die Metaphysik der Sitten/Erster Teil. Metaphysische Anfangsgründe der Rechtslehre/1. Teil. Das Privatrecht vom äußeren Mein und Dein überhaupt/2. Hauptstück. Von der Art, etwas Äußeres zu erwerben/1. Abschnitt. Vom Sachrecht\\  
	
	\noindent\textbf{Paragraphe : }Zuletzt kann noch gefragt werden: ob, wenn uns weder die Natur noch der Zufall, sondern bloß unser eigener Wille in Nachbarschaft mit einem Volk bringt, welches keine Aussicht zu einer bürgerlichen Verbindung mit ihm verspricht, wir nicht, in der Absicht, diese zu stiften und diese Menschen (Wilde) in einen rechtlichen Zustand zu versetzen (wie etwa die amerikanischen Wilden, die Hottentotten, die Neuholländer), befugt sein sollten, allenfalls mit Gewalt, oder (welches nicht viel besser ist) durch betrügerischen Kauf, Kolonien zu errichten und so Eigentümer ihres Bodens zu werden, und, ohne Rücksicht auf ihren ersten Besitz, Gebrauch von unserer Überlegenheit zu machen; zumal es die Natur selbst (als die das Leere verabscheuet) so zu fordern scheint, und große Landstriche in anderen Weltteilen an gesitteten Einwohnern sonst menschenleer geblieben wären, die jetzt herrlich bevölkert sind, oder gar auf immer bleiben müßten, und so der Zweck der Schöpfung vereitelt werden würde? Allein man sieht durch diesen Schleier der Ungerechtigkeit (Jesuitism), alle Mittel zu guten Zwecken zu billigen, leicht durch; diese \match{Art} der Erwerbung des Bodens ist also verwerflich. 
	
	\subsection*{tg436.2.11} 
	\textbf{Source : }Die Metaphysik der Sitten/Erster Teil. Metaphysische Anfangsgründe der Rechtslehre/1. Teil. Das Privatrecht vom äußeren Mein und Dein überhaupt/2. Hauptstück. Von der Art, etwas Äußeres zu erwerben/2. Abschnitt. Vom persönlichen Recht\\  
	
	\noindent\textbf{Paragraphe : }
	Aber weder durch den besonderen Willen des Promittenten, noch den des Promissars (als Akzeptanten), geht das Seine des ersteren zu dem letzteren über, sondern nur durch den vereinigten Willen beider, mithin so fern beider Wille zugleich deklariert wird. Nun ist dies aber durch empirische Actus der Deklaration, die einander notwendig in der Zeit folgen müssen, und niemals zugleich sind, unmöglich. Denn, wenn ich versprochen habe und der andere nun akzeptieren will, so kann ich während der Zwischenzeit (so kurz sie auch sein mag) es mich gereuen lassen, weil ich vor der Akzeptation noch frei bin; so wie anderseits der Akzeptant, eben darum, an seine auf das Versprechen folgende Gegenerklärung auch sich nicht für gebunden halten darf. – Die äußern Förmlichkeiten (solennia) bei Schließung des Vertrags (der Handschlag, oder die Zerbrechung eines von beiden Personen angefaßten Strohhalms (stipula)), und alle hin und her geschehene Bestätigungen seiner vorherigen Erklärung beweisen vielmehr die Verlegenheit der Paziszenten, wie und auf welche \match{Art} sie die immer nur aufeinander folgenden Erklärungen als in einem Augenblicke zugleich existierend vorstellig machen wollen, was ihnen doch nicht gelingt; weil es immer nur in der Zeit einander folgende Actus sind, wo, wenn der eine Akt ist, der andere entweder noch nicht, oder nicht mehr ist. 
	
	\subsection*{tg437.2.33} 
	\textbf{Source : }Die Metaphysik der Sitten/Erster Teil. Metaphysische Anfangsgründe der Rechtslehre/1. Teil. Das Privatrecht vom äußeren Mein und Dein überhaupt/2. Hauptstück. Von der Art, etwas Äußeres zu erwerben/3. Abschnitt. Von dem auf dingliche Art persönlichen Recht\\  
	
	\noindent\textbf{Paragraphe : }Aus dieser Persönlichkeit der erstern folgt nun auch, daß, da die Kinder nie als Eigentum der Eltern angesehen werden können, aber doch zum Mein und Dein derselben gehören (weil sie gleich den Sachen im Besitz der Eltern sind, und aus jedes anderen Besitz, selbst wider ihren Willen, in diesen zurückgebracht werden können), das Recht der ersteren kein bloßes Sachenrecht, mithin nicht veräußerlich (ius personalissimum), aber auch nicht ein bloß persönliches, sondern ein auf dingliche \match{Art} persönliches Recht ist. 
	
	\subsection*{tg437.2.34} 
	\textbf{Source : }Die Metaphysik der Sitten/Erster Teil. Metaphysische Anfangsgründe der Rechtslehre/1. Teil. Das Privatrecht vom äußeren Mein und Dein überhaupt/2. Hauptstück. Von der Art, etwas Äußeres zu erwerben/3. Abschnitt. Von dem auf dingliche Art persönlichen Recht\\  
	
	\noindent\textbf{Paragraphe : }Hiebei fällt also in die Augen, daß der Titel eines auf dingliche \match{Art} persönlichen Rechts in der Rechtslehre noch über dem des Sachen- und persönlichen Rechts notwendig hinzukommen müsse, jene bisherige Einteilung also nicht vollständig gewesen ist, weil, wenn von dem Recht der Eltern an den Kindern, als einem Stück ihres Hauses, die Rede ist, jene sich nicht bloß auf die Pflicht der Kinder berufen dürfen, zurückzukehren, wenn sie entlaufen sind, sondern sich ihrer als Sachen (verlaufener Haustiere) zu bemächtigen, und sie einzufangen berechtigt sind. 
	
	\subsection*{tg437.2.45} 
	\textbf{Source : }Die Metaphysik der Sitten/Erster Teil. Metaphysische Anfangsgründe der Rechtslehre/1. Teil. Das Privatrecht vom äußeren Mein und Dein überhaupt/2. Hauptstück. Von der Art, etwas Äußeres zu erwerben/3. Abschnitt. Von dem auf dingliche Art persönlichen Recht\\  
	
	\noindent\textbf{Paragraphe : }Man sieht also auch hier, wie unter beiden vorigen Titeln, daß es ein auf dingliche \match{Art} persönliches Recht (der Herrschaft über das Gesinde) gebe; weil man sie zurück holen, und als das äußere Seine von jedem Besitzer abfordern kann, ehe noch die Gründe, welche sie dazu vermocht haben mögen, und ihr Recht untersucht werden dürfen. 
	
	\subsection*{tg437.2.76} 
	\textbf{Source : }Die Metaphysik der Sitten/Erster Teil. Metaphysische Anfangsgründe der Rechtslehre/1. Teil. Das Privatrecht vom äußeren Mein und Dein überhaupt/2. Hauptstück. Von der Art, etwas Äußeres zu erwerben/3. Abschnitt. Von dem auf dingliche Art persönlichen Recht\\  
	
	\noindent\textbf{Paragraphe : }
	Geld ist eine Sache, deren Gebrauch nur dadurch möglich ist, daß man sie veräußert. Dies ist eine gute Namenerklärung desselben (nach Achenwall), nämlich hinreichend zur Unterscheidung dieser \match{Art} Gegenstände der Willkür von allen andern; aber sie gibt uns keinen Aufschluß über die Möglichkeit einer solchen Sache. Doch sieht man so viel daraus: daß erstlich diese Veräußerung im Verkehr nicht als Verschenkung, sondern als zur wechselseitigen
	Erwerbung (durch ein pactum onerosum) beabsichtigt ist; zweitens daß, da es als ein (in einem Volke) allgemein beliebtes bloßes Mittel des Handels, was an sich keinen Wert hat, im Gegensatz einer Sache, als Ware (d.i. desjenigen, was einen solchen hat, und sich auf das besondere Bedürfnis eines oder des anderen im Volk bezieht), gedacht wird, es alle Ware repräsentiert. 
	
	\subsection*{tg437.2.80} 
	\textbf{Source : }Die Metaphysik der Sitten/Erster Teil. Metaphysische Anfangsgründe der Rechtslehre/1. Teil. Das Privatrecht vom äußeren Mein und Dein überhaupt/2. Hauptstück. Von der Art, etwas Äußeres zu erwerben/3. Abschnitt. Von dem auf dingliche Art persönlichen Recht\\  
	
	\noindent\textbf{Paragraphe : }Wie ist es aber möglich, daß das, was anfänglich Ware war, endlich Geld ward? Wenn ein großer und machthabender Vertuer einer Materie, die er anfangs bloß zum Schmuck und Glanz seiner Diener (des Hofes) brauchte (z.B. Gold, Silber, Kupfer, oder eine \match{Art} schöner Muschelschalen, Kauris, oder auch, wie in Kongo, eine Art Matten, Makuten genannt, oder, wie am Senegal, Eisenstangen, und auf der Guineaküste selbst Negersklaven), d.i. wenn ein Landesherr die Abgaben von seinen Untertanen in dieser Materie (als Ware) einfordert, und die, deren Fleiß in Anschaffung derselben  dadurch bewegt werden soll, mit eben denselben, nach Verordnungen des Verkehrs unter und mit ihnen überhaupt (auf einem Markt, oder einer Börse), wieder lohnt. – Dadurch allein hat (meinem Bedünken nach) eine Ware ein gesetzliches Mittel des Verkehrs des Fleißes der Untertanen unter einander und hiemit auch des Staatsreichtums, d.i. Geld, werden können. 
	
	\subsection*{tg438.2.8} 
	\textbf{Source : }Die Metaphysik der Sitten/Erster Teil. Metaphysische Anfangsgründe der Rechtslehre/1. Teil. Das Privatrecht vom äußeren Mein und Dein überhaupt/2. Hauptstück. Von der Art, etwas Äußeres zu erwerben/Episodischer Abschnitt. Von der idealen Erwerbung eines äußeren Gegenstandes der Willkür\\  
	
	\noindent\textbf{Paragraphe : }Ich erwerbe das Eigentum eines anderen bloß durch den langen Besitz (usucapio); nicht weil ich dieses seine Einwilligung  dazu rechtmäßig voraussetzen darf (per consensum praesumtum), noch weil ich, da er nicht widerspricht, annehmen kann, er habe seine Sache aufgegeben (rem derelictam), sondern, weil, wenn es auch einen wahren und auf diese Sache als Eigentümer Anspruch Machenden (Prätendenten) gäbe, ich ihn doch bloß durch meinen langen Besitz ausschließen, sein bisheriges Dasein ignorieren, und gar, als ob er zur Zeit meines Besitzes nur als Gedankending existierte, verfahren darf: wenn ich gleich von seiner Wirklichkeit so wohl, als der seines Anspruchs hinterher benachrichtigt sein möchte. – Man nennt diese \match{Art} der Erwerbung, nicht ganz richtig, die durch Verjährung (per praescriptionem); denn die Ausschließung ist nur als die Folge von jener anzusehen; die Erwerbung muß vorhergegangen sein. – Die Möglichkeit, auf diese Art zu erwerben ist nun zu beweisen. 
	
	\subsection*{tg439.2.26} 
	\textbf{Source : }Die Metaphysik der Sitten/Erster Teil. Metaphysische Anfangsgründe der Rechtslehre/1. Teil. Das Privatrecht vom äußeren Mein und Dein überhaupt/3. Hauptstück. Von der subjektiv-bedingten Erwerbung durch den Ausspruch einer öffentlichen Gerichtsbarkeit\\  
	
	\noindent\textbf{Paragraphe : }Ist die Sache mir abhanden gekommen (res amissa) und so von einem anderen auf ehrliche \match{Art} (bona fide), als ein vermeinter Fund, oder durch förmliche Veräußerung des Besitzers, der sich als Eigentümer führt, an mich gekommen, obgleich dieser nicht Eigentümer ist, so fragt sich, ob, da ich von einem Nichteigentümer (a non domino) eine Sache nicht erwerben kann, ich durch jenen von allem Recht in dieser Sache ausgeschlossen werde, und bloß ein persönliches gegen den unrechtmäßigen Besitzer übrig behalte. – Das letztere ist offenbar der Fall, wenn die Erwerbung bloß nach ihren inneren berechtigenden Gründen (im Naturzustande), nicht nach der Konvenienz eines Gerichtshofes beurteilt wird. 
	
	\subsection*{tg441.2.21} 
	\textbf{Source : }Die Metaphysik der Sitten/Erster Teil. Metaphysische Anfangsgründe der Rechtslehre/2. Teil. Das öffentliche Recht/1. Abschnitt. Das Staatsrecht\\  
	
	\noindent\textbf{Paragraphe : }Diese Abhängigkeit von dem Willen anderer, und Ungleichheit, ist gleichwohl keinesweges der Freiheit und Gleichheit derselben als Menschen, die zusammen ein Volk ausmachen, entgegen: vielmehr kann, bloß den Bedingungen derselben gemäß, dieses Volk ein Staat werden, und in eine bürgerliche Verfassung eintreten. In dieser Verfassung aber das Recht der Stimmgebung zu haben, d.i. Staatsbürger, nicht bloß Staatsgenosse zu sein, dazu qualifizieren sich nicht alle mit gleichem Recht. Denn daraus, daß sie fordern können, von allen anderen nach Gesetzen der natürlichen Freiheit und Gleichheit als passive Teile des Staats behandelt zu werden, folgt nicht das Recht, auch als aktive Glieder den Staat selbst  zu behandeln, zu organisieren oder zu Einführung gewisser Gesetze mitzuwirken: sondern nur, daß, welcherlei \match{Art} die positiven Gesetze, wozu sie stimmen, auch sein möchten, sie doch den natürlichen der Freiheit und der dieser angemessenen Gleichheit aller im Volk, sich nämlich aus diesem passiven Zustande zu dem aktiven empor arbeiten zu können, nicht zuwider sein müssen. 
	
	\subsection*{tg441.2.44} 
	\textbf{Source : }Die Metaphysik der Sitten/Erster Teil. Metaphysische Anfangsgründe der Rechtslehre/2. Teil. Das öffentliche Recht/1. Abschnitt. Das Staatsrecht\\  
	
	\noindent\textbf{Paragraphe : }Wider das gesetzgebende Oberhaupt des Staats gibt es also keinen rechtmäßigen Widerstand des Volks; denn nur durch Unterwerfung unter seinen allgemein-gesetzgebenden Willen ist ein rechtlicher Zustand möglich; also kein Recht des Aufstandes (seditio), noch weniger des Aufruhrs (rebellio), am allerwenigsten gegen ihn, als einzelne Person (Monarch), unter dem Verwände des Mißbrauchs seiner Gewalt (tyrannis), Vergreifung an seiner Person, ja an seinem Leben (monarchomachismus sub specie tyrannicidii). Der geringste Versuch hiezu ist Hochverrat (proditio eminens), und der Verräter dieser \match{Art} kann als einer, der sein 
	Vaterland umzubringen versucht (parricida), nicht minder als mit dem Tode bestraft werden. – – Der Grund der Pflicht des Volks, einen, selbst den für unerträglich ausgegebenen Mißbrauch der obersten Gewalt dennoch zu ertragen, liegt darin: daß sein Widerstand wider die höchste Gesetzgebung selbst niemals anders, als gesetzwidrig, ja als die ganze gesetzliche Verfassung zernichtend gedacht werden muß. Denn, um zu demselben befugt zu sein, müßte ein öffentliches Gesetz vorhanden sein, welches diesen Widerstand des Volks erlaubte, d.i. die oberste Gesetzgebung enthielte eine Bestimmung in sich, nicht die oberste zu sein, und das Volk, als Untertan, in einem und demselben Urteile zum Souverän über den zu machen, dem es untertänig ist; welches sich widerspricht und wovon der Widerspruch durch die Frage alsbald in die Augen fällt: wer denn in diesem Streit zwischen Volk und Souverän Richter sein sollte (denn es sind rechtlich betrachtet doch immer zwei verschiedene moralische Personen); wo sich dann zeigt, daß das erstere es in seiner eigenen Sache sein will.
	
	
	8
	
	
	
	\subsection*{tg441.2.51} 
	\textbf{Source : }Die Metaphysik der Sitten/Erster Teil. Metaphysische Anfangsgründe der Rechtslehre/2. Teil. Das öffentliche Recht/1. Abschnitt. Das Staatsrecht\\  
	
	\noindent\textbf{Paragraphe : }Auf diesem ursprünglich erworbenen Grundeigentum beruht das Recht des Oberbefehlshabers, als Obereigentümers (des Landesherrn), die Privateigentümer des Bodens zu beschatzen, d.i. Abgaben durch die Landtaxe, Akzise und Zölle, oder Dienstleistung (dergleichen die Stellung der Mannschaft zum Kriegsdienst ist) zu fordern: so doch, daß das Volk sich selber beschatzt, weil dieses die einzige \match{Art} ist, hiebei nach Rechtsgesetzen zu verfahren, wenn es durch das Korps der Deputierten desselben geschieht, auch als gezwungene (von dem bisher bestandenen Gesetz abweichende) Anleihe, nach dem Majestätsrechte, als in einem Falle, da der Staat in Gefahr seiner Auflösung kommt, erlaubt ist. 
	
	\subsection*{tg441.2.58} 
	\textbf{Source : }Die Metaphysik der Sitten/Erster Teil. Metaphysische Anfangsgründe der Rechtslehre/2. Teil. Das öffentliche Recht/1. Abschnitt. Das Staatsrecht\\  
	
	\noindent\textbf{Paragraphe : }Was die Erhaltung der aus Not oder Scham ausgesetzten, oder wohl gar darum ermordeten Kinder betrifft, so hat der Staat ein Recht, das Volk mit der Pflicht zu belasten, diesen, obzwar unwillkommenen Zuwachs des Staatsvermögens nicht wissentlich um kommen zu lassen. Ob dieses aber durch Besteurung der Hagestolzen beiderlei Geschlechts (worunter die vermögende Ledige verstanden werden), als solche, die daran doch zum Teil schuld sind, vermittelst dazu errichteter Findelhäuser, oder auf andere \match{Art} mit Recht geschehen könne (ein anderes Mittel, es zu verhüten, möchte es aber schwerlich geben), ist eine Aufgabe, deren Lösung, ohne entweder wider das Recht, oder die Moralität zu verstoßen, bisher noch nicht gelungen ist. 
	
	\subsection*{tg441.2.71} 
	\textbf{Source : }Die Metaphysik der Sitten/Erster Teil. Metaphysische Anfangsgründe der Rechtslehre/2. Teil. Das öffentliche Recht/1. Abschnitt. Das Staatsrecht\\  
	
	\noindent\textbf{Paragraphe : }Welche \match{Art} aber und welcher Grad der Bestrafung ist es, welche die öffentliche Gerechtigkeit sich zum Prinzip und Richtmaße macht? Kein anderes, als das Prinzip der Gleichheit (im Stande des Züngleins an der Wage der Gerechtigkeit), sich nicht mehr auf die eine, als auf die andere Seite hinzuneigen. Also: was für unverschuldetes Übel du einem  anderen im Volk zufügst, das tust du dir selbst an. Beschimpfst du ihn, so beschimpfst du dich selbst; bestiehlst du ihn, so bestiehlst du dich selbst; schlägst du ihn, so schlägst du dich selbst; tötest du ihn, so tötest du dich selbst. Nur das Wiedervergeltungsrecht (ius talionis), aber, wohl zu verstehen, vor den Schranken des Gerichts (nicht in deinem Privaturteil), kann die Qualität und Quantität der Strafe bestimmt angeben; alle andere sind hin und her schwankend, und können, anderer sich einmischenden Rücksichten wegen, keine Angemessenheit mit dem Spruch der reinen und strengen Gerechtigkeit enthalten. – Nun scheint es zwar, daß der Unterschied der Stände das Prinzip der Wiedervergeltung Gleiches mit Gleichem nicht verstatte; aber, wenn es gleich nicht nach dem Buchstaben möglich sein kann, so kann es doch der Wirkung nach, respektive auf die Empfindungsart der Vornehmeren, immer geltend bleiben. – So hat z.B. Geldstrafe wegen einer Verbalinjurie gar kein Verhältnis zur Beleidigung, denn, der des Geldes viel hat, kann diese sich wohl einmal zur Lust erlauben; aber die Kränkung der Ehrliebe des einen kann doch dem Wehtun des Hochmuts des anderen sehr gleich kommen: wenn dieser nicht allein öffentlich abzubitten, sondern jenem, ob er zwar niedriger ist, etwa zugleich die Hand zu küssen, durch Urteil und Recht genötigt würde. Eben so, wenn der gewalttätige Vornehme für die Schläge, die er dem niederen aber schuldlosen Staatsbürger zumißt, außer der Abbitte noch zu einem einsamen und beschwerlichen Arrest verurteilt würde, weil hiemit, außer der Ungemächlichkeit, noch die Eitelkeit des Täters schmerzhaft angegriffen, und so durch Beschämung Gleiches mit Gleichem gehörig vergolten würde. – Was heißt das aber: »bestiehlst du ihn, so bestiehlst du dich selbst «? Wer da stiehlt, macht aller anderer Eigentum unsicher; er beraubt sich also (nach dem Recht der Wiedervergeltung) der Sicherheit alles möglichen Eigentums; er hat nichts und kann auch nichts erwerben, will aber doch leben; welches nun nicht anders möglich ist, als daß ihn andere ernähren. Weil dieses aber der Staat nicht umsonst tun wird, so muß er diesem seine  Kräfte zu ihm beliebigen Arbeiten (Karren- oder Zuchthausarbeit) überlassen, und kommt auf gewisse Zeit, oder, nach Befinden, auch auf immer, in den Sklavenstand. – Hat er aber gemordet, so muß er sterben. Es gibt hier Kein Surrogat zur Befriedigung der Gerechtigkeit. Es ist keine Gleichartigkeit zwischen einem noch so kummervollen Leben und dem Tode, also auch keine Gleichheit des Verbrechens und der Wiedervergeltung, als durch den am Täter gerichtlich vollzogenen, doch von aller Mißhandlung, welche die Menschheit in der leidenden Person zum Scheusal machen könnte, befreieten Tod. – Selbst, wenn sich die bürgerliche Gesellschaft mit aller Glieder Einstimmung auflösete (z.B. das eine Insel bewohnende Volk beschlösse, auseinander zu gehen, und sich in alle Welt zu zerstreuen), müßte der letzte im Gefängnis befindliche Mörder vorher hingerichtet werden, damit jedermann das widerfahre, was seine Taten wert sind, und die Blutschuld nicht auf dem Volke hafte, das auf diese Bestrafung nicht gedrungen hat; weil es als Teilnehmer an dieser öffentlichen Verletzung der Gerechtigkeit betrachtet werden kann. 
	
	\subsection*{tg441.2.72} 
	\textbf{Source : }Die Metaphysik der Sitten/Erster Teil. Metaphysische Anfangsgründe der Rechtslehre/2. Teil. Das öffentliche Recht/1. Abschnitt. Das Staatsrecht\\  
	
	\noindent\textbf{Paragraphe : }Diese Gleichheit der Strafen, die allein durch die Erkenntnis des Richters auf den Tod, nach dem strengen Wiedervergeltungsrechte, möglich ist, offenbaret sich daran, daß dadurch allein proportionierlich mit der inneren Bösartigkeit der Verbrecher das Todesurteil über alle (selbst wenn es nicht einen Mord, sondern ein anderes nur mit dem Tode zu tilgendes Staatsverbrechen beträfe) ausgesprochen wird. – Setzet: daß, wie in der letzten schottischen Rebellion, da verschiedene Teilnehmer an derselben (wie Balmerino und andere) durch ihre Empörung nichts als eine dem Hause Stuart schuldige Pflicht auszuüben glaubten, andere dagegen Privatabsichten hegten, von dem höchsten Gericht das Urteil so gesprochen worden wäre: ein jeder solle die Freiheit der Wahl zwischen dem Tode und der Karrenstrafe haben: so sage ich, der ehrliche Mann wählt den Tod, der Schelm aber die Karre; so bringt es die Natur des menschlichen Gemüts mit sich. Denn der erstere kennt etwas, was er noch höher schätzt, als selbst das Leben: nämlich die
	Ehre; der andere hält ein mit Schande bedecktes Leben doch immer noch für besser, als gar nicht zu sein (animam praeferre pudori. Juven.). Der erstere ist nun ohne Widerrede weniger strafbar als der andere, und so werden sie durch den über alle gleich verhängten Tod ganz proportionierlich bestraft, jener gelinde, nach seiner Empfindungsart, und dieser hart, nach der seinigen; da hingegen, wenn durchgängig auf die Karrenstrafe erkannt würde, der erstere zu hart, der andere, für seine Niederträchtigkeit, gar zu gelinde bestraft wäre; und so ist auch hier im Ausspruche über eine im Komplott vereinigte Zahl von Verbrechern der beste Ausgleicher, vor der öffentlichen Gerechtigkeit, der Tod. – Überdem hat man nie gehört, daß ein wegen Mordes zum Tode Verurteilter sich beschwert hätte, daß ihm damit zu viel und also unrecht geschehe, jeder würde ihm ins Gesicht lachen, wenn er sich dessen äußerte. – Man müßte sonst annehmen, daß, wenn dem Verbrecher gleich nach dem Gesetz nicht unrecht geschieht, doch die gesetzgebende Gewalt im Staat diese \match{Art} von Strafe zu verhängen nicht befugt, und, wenn sie es tut, mit sich selbst im Widerspruch sei. 
	
	\subsection*{tg441.2.76} 
	\textbf{Source : }Die Metaphysik der Sitten/Erster Teil. Metaphysische Anfangsgründe der Rechtslehre/2. Teil. Das öffentliche Recht/1. Abschnitt. Das Staatsrecht\\  
	
	\noindent\textbf{Paragraphe : }Es gibt indessen zwei todeswürdige Verbrechen, in Ansehung deren, ob die Gesetzgebung auch die Befugnis habe, sie mit der Todesstrafe zu belegen, noch zweifelhaft bleibt. Zu beiden verleitet das Ehrgefühl. Das eine ist das der Geschlechtsehre, das andere der Kriegsehre, und zwar der wahren Ehre, welche jeder dieser zwei Menschenklassen als Pflicht obliegt. Das eine Verbrechen ist der mütterliche Kindesmord (infanticidium maternale); das andere der Kriegsgesellenmord (commilitonicidium), der Duell. – Da die Gesetzgebung die Schmach einer unehelichen Geburt nicht wegnehmen, und eben so wenig den Fleck, welcher aus dem Verdacht der Feigheit, der auf einen untergeordneten Kriegsbefehlshaber fällt, welcher einer verächtlichen Begegnung nicht eine über die Todesfurcht erhobene eigene Gewalt entgegensetzt, wegwischen kann: so scheint es, daß Menschen in diesen Fällen sich im Naturzustande befinden und Tötung (homicidium), die alsdann nicht einmal Mord (homicidium dolosum) heißen müßte, in beiden zwar allerdings strafbar sei, von der obersten Macht aber mit dem Tode nicht könne bestraft werden. Das uneheliche auf die Weit gekommene Kind ist außer dem Gesetz (denn das heißt Ehe), mithin auch außer dem Schutz desselben, geboren. Es ist in das gemeine Wesen gleichsam eingeschlichen (wie verbotene Ware), so daß dieses seine Existenz (weil es billig auf diese \match{Art} nicht hätte existieren sollen), mithin auch seine Vernichtung ignorieren kann,  und die Schande der Mutter, wenn ihre uneheliche Niederkunft bekannt, wird, kann keine Verordnung heben. – Der zum Unter-Befehlshaber eingesetzte Kriegesmann, dem ein Schimpf angetan wird, sieht sich eben so wohl durch die öffentliche Meinung der Mitgenossen seines Standes genötigt, sich Genugtuung, und, wie im Naturzustande, Bestrafung des Beleidigers, nicht durchs Gesetz, vor einem Gerichtshofe, sondern durch den Duell, darin er sich selbst der Lebensgefahr aussetzt, zu verschaffen, um seinen Kriegsmut zu beweisen, als worauf die Ehre seines Standes wesentlich beruht, sollte es auch mit der Tötung seines Gegners verbunden sein, die In diesem Kampfe, der öffentlich und mit beiderseitiger Einwilligung, doch auch ungern, geschieht, eigentlich nicht Mord (homicidium dolosum) genannt werden kann. – – Was ist nun in beiden (zur Kriminalgerechtigkeit gehörigen) Fällen Rechtens? – Hier kommt die Strafgerechtigkeit gar sehr ins Gedränge: entweder den Ehrbegriff (der hier kein Wahn ist) durchs Gesetz für nichtig zu erklären, und so mit dem Tode zu strafen, oder von dem Verbrechen die angemessene Todesstrafe wegzunehmen, und so entweder grausam oder nachsichtig zu sein. Die Auflösung dieses Knotens ist: daß der kategorische Imperativ der Strafgerechtigkeit (die gesetzwidrige Tötung eines anderen müsse mit dem Tode bestraft werden) bleibt, die Gesetzgebung selber aber (mithin auch die bürgerliche Verfassung), so lange noch als barbarisch und unausgebildet, daran Schuld ist, daß die Triebfedern der Ehre im Volk (subjektiv) nicht mit den Maßregeln zusammen treffen wollen, die (objektiv) ihrer Absicht gemäß sind, so daß die öffentliche, vom Staat ausgehende, Gerechtigkeit in Ansehung der aus dem Volk eine Ungerechtigkeit wird. 
	
	\subsection*{tg441.2.94} 
	\textbf{Source : }Die Metaphysik der Sitten/Erster Teil. Metaphysische Anfangsgründe der Rechtslehre/2. Teil. Das öffentliche Recht/1. Abschnitt. Das Staatsrecht\\  
	
	\noindent\textbf{Paragraphe : }Die drei Gewalten im Staat, die aus dem Begriff eines gemeinen Wesens überhaupt (res publica latius dicta) hervorgehen, sind nur so viel Verhältnisse des vereinigten, a priori aus der Vernunft abstammenden, Volkswillens und eine reine Idee von einem Staatsoberhaupt, welche objektive praktische Realität hat. Dieses Oberhaupt (der Souverän) aber ist so fern nur ein (das gesamte Volk vorstellendes) Gedankending, als es noch an einer physischen Person mangelt, welche die höchste Staatsgewalt vorstellt, und dieser Idee Wirksamkeit auf den Volkswillen verschafft. Das Verhältnis der ersteren zum letzteren ist nun auf dreierlei verschiedene \match{Art} denkbar: entweder daß einer im Staate über alle, oder daß einige, die einander gleich sind, vereinigt, über alle andere, oder daß alle zusammen über einen jeden, mithin auch über sich selbst gebieten, d.i. die Staatsform ist entweder autokratisch, oder aristokratisch, oder demokratisch. (Der Ausdruck monarchisch, statt autokratisch, ist nicht dem Begriffe, den man hier will, angemessen; denn Monarch ist der, welcher die höchste, Autokrator aber, oder Selbstherrscher,  der, welcher alle Gewalt hat; dieser ist der Souverän, jener repräsentiert ihn bloß.) – Man wird leicht gewahr, daß die autokratische Staatsform die einfachste sei, nämlich von Einem (dem Könige) zum Volke, mithin wo nur Einer der Gesetzgeber ist. Die aristokratische ist schon aus zwei Verhältnissen zusammengesetzt: nämlich dem der Vornehmen (als Gesetzgeber) zu einander, um den Souverän zu machen, und dann das dieses Souveräns zum Volk; die demokratische aber die allerzusammengesetzteste, nämlich den Willen aller zuerst zu vereinigen, um daraus ein Volk, dann den der Staatsbürger, um ein gemeines Wesen zu bilden, und dann diesem gemeinen Wesen den Souverän, der dieser vereinigte Wille selbst ist, vorzusetzen.
	
	
	9
	Was die Handhabung des Rechts im Staat betrifft, so ist freilich die einfachste auch zugleich die beste; aber, was das Recht selbst anlangt, die gefährlichste fürs Volk, in Betracht des Despotismus, zu dem sie so sehr einladet. Das Simplifizieren ist zwar im Maschinenwerk der Vereinigung des Volks durch Zwangsgesetze die vernünftige Maxime: wenn nämlich alle im Volk passiv sind, und Einem, der über sie ist, gehorchen; aber das gibt keine Untertanen als Staatsbürger. Was die Vertröstung, womit sich das Volk befriedigen soll, betrifft: daß nämlich die Monarchie (eigentlich hier Autokratie) die beste Staatsverfassung sei, wenn der Monarch gut ist (d.i. nicht bloß den Willen, sondern auch die Einsicht dazu hat), gehört zu den tautologischen Weisheitssprüchen, und sagt nichts mehr, als: die beste Verfassung ist die, durch welche der Staatsverwalter zum besten Regenten gemacht wird, d.i. diejenige, welche die beste ist. 
	
	\subsection*{tg442.2.14} 
	\textbf{Source : }Die Metaphysik der Sitten/Erster Teil. Metaphysische Anfangsgründe der Rechtslehre/2. Teil. Das öffentliche Recht/2. Abschnitt. Das Völkerrecht\\  
	
	\noindent\textbf{Paragraphe : }Es gibt mancherlei Naturprodukte in einem Lande, die doch, was die Menge derselben von einer gewissen \match{Art} betrifft, zugleich als Gemächsel (artefacta) des Staats angesehen werden müssen, weil das Land sie in solcher Menge nicht liefern würde, wenn es nicht einen Staat und eine ordentliche machthabende Regierung gäbe, sondern die Bewohner im Stande der Natur wären. – Haushühner (die nützlichste Art des Geflügels), Schafe, Schweine, das Rindergeschlecht u.a.m. würden, entweder aus Mangel an Futter, oder der Raubtiere wegen, in dem Lande, wo ich lebe, entweder gar nicht, oder höchst sparsam anzutreffen sein, wenn es darin nicht eine Regierung gäbe, welche den Einwohnern ihren Erwerb und Besitz sicherte. – Eben das gilt auch von der Menschenzahl, die, eben so wie in den amerikanischen Wüsten, ja selbst dann, wenn man diesen den größten Fleiß (den jene nicht haben) beilegte, nur gering sein kann. Die Einwohner würden nur sehr dünn gesäet sein, weil keiner derselben sich, mit samt seinem Gesinde, auf einem Boden weit verbreiten könnte, der immer in Gefahr ist, von Menschen oder Wilden und Raubtieren verwüstet zu werden; mithin sich für eine so große Menge von Menschen, als jetzt auf einem Lande leben, kein hinlänglicher Unterhalt finden würde. – – Sowie man nun von Gewächsen (z.B. den Kartoffeln) und von Haustieren, weil sie, was die Menge betrifft, ein Machwerk der Menschen sind, sagen kann, daß man sie gebrauchen, verbrauchen und verzehren (töten lassen) kann: so, scheint es, könne man auch von der obersten Gewalt im Staat, dem Souverän, sagen, er  habe das Recht, seine Untertanen, die dem größten Teil nach sein eigenes Produkt sind, in den Krieg, wie auf eine Jagd, und zu einer Feldschlacht, wie auf eine Lustpartie zu führen. 
	
	\subsection*{tg442.2.22} 
	\textbf{Source : }Die Metaphysik der Sitten/Erster Teil. Metaphysische Anfangsgründe der Rechtslehre/2. Teil. Das öffentliche Recht/2. Abschnitt. Das Völkerrecht\\  
	
	\noindent\textbf{Paragraphe : }Was die tätige Verletzung betrifft, die ein Recht zum Kriege gibt, so gehört dazu die selbstgenommene Genugtuung für die Beleidigung des einen Volks durch das Volk des anderen Staats, die Wiedervergeltung (retorsio), ohne eine Erstattung (durch friedliche Wege) bei dem anderen Staate zu suchen, womit, der Förmlichkeit nach, der Ausbruch des Krieges, ohne vorhergehende Aufkündigung des Friedens (Kriegsankündigung), eine Ähnlichkeit hat; weil, wenn man einmal ein Recht im Kriegszustande finden will, etwas Analogisches mit einem Vertrag angenommen werden muß, nämlich Annahme der Erklärung des anderen Teils, daß beide ihr Recht auf diese \match{Art} suchen wollen. 
	
	\subsection*{tg442.2.26} 
	\textbf{Source : }Die Metaphysik der Sitten/Erster Teil. Metaphysische Anfangsgründe der Rechtslehre/2. Teil. Das öffentliche Recht/2. Abschnitt. Das Völkerrecht\\  
	
	\noindent\textbf{Paragraphe : }Kein Krieg unabhängiger Staaten gegen einander kann ein Strafkrieg (bellum punitivum) sein, denn Strafe findet nur im Verhältnisse eines Obern (imper antis) gegen den Unterworfenen (subditum) statt, welches Verhältnis nicht das der Staaten gegen einander ist, – aber auch weder ein Ausrottungs– (bellum internecinum) noch Unterjochungskrieg (bellum subiugatorium), der eine moralische  Vertilgung eines Staats (dessen Volk nun mit dem des Überwinders entweder in eine Masse verschmelzt, oder in Knechtschaft verfällt) sein würde. Nicht als ob dieses Notmittel des Staats, zum Friedenszustande zu gelangen, an sich dem Rechte eines Staats widerspräche, sondern weil die Idee des Völkerrechts bloß den Begriff eines Antagonismus nach Prinzipien der äußeren Freiheit bei sich führt, um sich bei dem Seinen zu erhalten, aber nicht eine \match{Art} zu erwerben, als welche, durch Vergrößerung der Macht des einen Staats, für den anderen bedrohend sein kann. 
	
	\subsection*{tg442.2.27} 
	\textbf{Source : }Die Metaphysik der Sitten/Erster Teil. Metaphysische Anfangsgründe der Rechtslehre/2. Teil. Das öffentliche Recht/2. Abschnitt. Das Völkerrecht\\  
	
	\noindent\textbf{Paragraphe : }Verteidigungsmittel aller \match{Art} sind dem bekriegten Staat erlaubt, nur nicht solche, deren Gebrauch die Untertanen desselben, Staatsbürger zu sein, unfähig machen würde; denn alsdann machte er sich selbst zugleich unfähig, im Staatenverhältnisse nach dem Völkerrecht für eine Person zu gelten (die gleicher Rechte mit andern teilhaftig wäre). Darunter gehört: seine eigne Untertanen zu Spionen, diese, ja auch Auswärtige zu Meuchelmördern, Giftmischern (in welche Klasse auch wohl die so genannten Scharfschützen, welche einzelnen im Hinterhalte auflauern, gehören möchten), oder auch nur zur Verbreitung falscher Nachrichten, zu gebrauchen: mit einem Wort, sich solcher heimtückischen Mittel zu bedienen, die das Vertrauen, welches zur künftigen Gründung eines dauerhaften Friedens erforderlich ist, vernichten würden. 
	
	\subsection*{tg442.2.48} 
	\textbf{Source : }Die Metaphysik der Sitten/Erster Teil. Metaphysische Anfangsgründe der Rechtslehre/2. Teil. Das öffentliche Recht/2. Abschnitt. Das Völkerrecht\\  
	
	\noindent\textbf{Paragraphe : }Unter einem Kongreß wird hier aber nur eine willkürliche, zu aller Zeit ablösliche Zusammentretung verschiedener Staaten, nicht eine solche Verbindung, welche (so wie die der amerikanischen Staaten) auf einer Staatsverfassung gegründet, und daher unauflöslich ist, verstanden; – durch welchen allein die Idee eines zu errichtenden öffentlichen Rechts der Völker, ihre Streitigkeiten auf zivile Art, gleichsam durch einen Prozeß, nicht auf barbarische (nach \match{Art} der Wilden), nämlich durch Krieg zu entscheiden, realisiert werden kann. 
	
	\subsection*{tg445.2.10} 
	\textbf{Source : }Die Metaphysik der Sitten/Erster Teil. Metaphysische Anfangsgründe der Rechtslehre/Anhang erläutender Bemerkungen zu den metaphysischen Anhangsgründen der Rechtslehre\\  
	
	\noindent\textbf{Paragraphe : }Die Rechtslehrer haben bisher nun zwei Gemeinplätze besetzt: den des dinglichen und den des persönlichen Rechts. Es ist natürlich, zu fragen: ob auch, da noch zwei Plätze, aus der bloßen Form der Verbindung beider zu einem Begriffe, als Glieder der Einteilung a priori, offen stehen, nämlich der eines auf persönliche \match{Art} dinglichen, imgleichen der eines auf dingliche Art persönlichen Rechts, ob nämlich ein solcher neuhinzukommender Begriff auch statthaft sei, und vor der Hand, obzwar nur problematisch, in der vollständigen Tafel der Einteilung angetroffen werden müsse. Das letztere leidet keinen Zweifel. Denn die bloß logische Einteilung (die vom Inhalt der Erkenntnis – dem Objekt – abstrahiert) ist immer Dichotomie, z.B. ein jedes Recht ist entweder ein dingliches oder ein nicht-dingliches Recht. Diejenige aber, von der hier die Rede ist, nämlich die metaphysische Einteilung, kann auch Tetrachotomie sein; weil, außer den zwei einfachen Gliedern  der Einteilung, noch zwei Verhältnisse, nämlich die der das Recht einschränkenden Bedingungen hinzukommen, unter denen das eine Recht mit dem anderen in Verbindung tritt, deren Möglichkeit einer besonderen Untersuchung bedarf. – Der Begriff eines auf persönliche Art dinglichen Rechts fällt ohne weitere Umstände weg; denn es läßt sich kein Recht einer Sache gegen eine Person denken. Nun fragt sich: ob die Umkehrung dieses Verhältnisses auch eben so undenkbar sei; oder ob dieser Begriff, nämlich der eines auf dingliche Art persönlichen Rechts, nicht allein ohne inneren Widerspruch, sondern selbst auch ein notwendiger (a priori in der Vernunft gegebener) zum Begriffe des äußeren Mein und Dein gehörender Begriff sei, Personen auf ähnliche Art als Sachen, zwar nicht in allen Stücken zu behandlen, aber sie doch zu besitzen und in vielen Verhältnissen mit ihnen als Sachen zu verfahren. 
	
	\subsection*{tg445.2.13} 
	\textbf{Source : }Die Metaphysik der Sitten/Erster Teil. Metaphysische Anfangsgründe der Rechtslehre/Anhang erläutender Bemerkungen zu den metaphysischen Anhangsgründen der Rechtslehre\\  
	
	\noindent\textbf{Paragraphe : }Die Definition des auf dingliche \match{Art} persönlichen Rechts ist nun kurz und gut diese: »es ist das Recht des Menschen, eine Person außer sich als das Seine
	
	
	
	10
	zu haben«. Ich sage mit Fleiß: eine Person; denn einen anderen Menschen, der durch Verbrechen seine Persönlichkeit eingebüßt hat (zum Leibeigenen geworden ist), könnte man wohl als das Seine haben; von diesem Sachenrecht ist aber hier nicht die Rede. 
	
	\subsection*{tg445.2.23} 
	\textbf{Source : }Die Metaphysik der Sitten/Erster Teil. Metaphysische Anfangsgründe der Rechtslehre/Anhang erläutender Bemerkungen zu den metaphysischen Anhangsgründen der Rechtslehre\\  
	
	\noindent\textbf{Paragraphe : }Eben so kann der Mann mit dem Weibe kein Kind, als ihr beiderseitiges Machwerk (res artificialis), zeugen, ohne daß beide Teile sich gegen dieses und gegen einander die Verbindlichkeit zuziehen, es zu erhalten: welches doch auch die Erwerbung eines Menschen gleich als einer Sache, aber nur der Form nach (einem bloß auf dingliche \match{Art} persönlichem Rechte angemessen) ist. Die Eltern
	
	
	11
	haben ein Recht gegen jeden Besitzer des Kindes, das aus ihrer Gewalt gebracht worden (ius in re), und zugleich ein Recht, es zu allen Leistungen und aller Befolgung ihrer Befehle zu nötigen, die einer möglichen gesetzlichen Freiheit nicht zuwider sind (ius ad rem): folglich auch ein persönliches Recht gegen dasselbe. 
	
	\subsection*{tg450.2.2} 
	\textbf{Source : }Die Metaphysik der Sitten/Zweiter Teil. Metaphysische Anfangsgründe der Tugendlehre/Einleitung/II. Erörterung des Begriffs von einem Zwecke, der zugleich Pflicht ist\\  
	
	\noindent\textbf{Paragraphe : }Man kann sich das Verhältnis des Zwecks zur Pflicht auf zweierlei \match{Art} denken: entweder, von dem Zwecke ausgehend, die Maxime der pflichtmäßigen Handlungen, oder, umgekehrt, von dieser anhebend, den Zweck ausfindig zu machen, der zugleich Pflicht ist. – Die Rechtslehre geht auf dem ersten Wege. Es wird jedermanns freier Willkür überlassen, welchen Zweck er sich für seine Handlung setzen wolle. Die Maxime derselben aber ist a priori bestimmt: daß nämlich die Freiheit des Handelnden mit jedes anderen Freiheit nach einem allgemeinen Gesetz zusammen bestehen könne. 
	
	\subsection*{tg450.2.4} 
	\textbf{Source : }Die Metaphysik der Sitten/Zweiter Teil. Metaphysische Anfangsgründe der Tugendlehre/Einleitung/II. Erörterung des Begriffs von einem Zwecke, der zugleich Pflicht ist\\  
	
	\noindent\textbf{Paragraphe : }Dahingestellt: was denn das für ein Zweck sei, der an sich selbst Pflicht ist, und wie ein solcher möglich sei, ist hier nur noch zu zeigen nötig, daß und warum eine Pflicht dieser \match{Art} den Namen einer Tugendpflicht führe. 
	
	\subsection*{tg453.2.7} 
	\textbf{Source : }Die Metaphysik der Sitten/Zweiter Teil. Metaphysische Anfangsgründe der Tugendlehre/Einleitung/V. Erläuterung dieser zwei Begriffe\\  
	
	\noindent\textbf{Paragraphe : }Glückseligkeit, d.i. Zufriedenheit mit seinem Zustande, sofern man der Fortdauer derselben gewiß ist, sich zu wünschen und zu suchen ist der menschlichen Natur unvermeidlich; eben darum aber auch nicht ein Zweck, der zugleich Pflicht ist. – Da einige noch einen Unterschied zwischen einer moralischen und physischen Glückseligkeit machen (deren erstere in der Zufriedenheit mit seiner Person und ihrem eigenen sittlichen Verhalten, also mit dem was man tut, die andere mit dem was die Natur beschert, mithin was man als fremde Gabe genießt, bestehe): so muß man bemerken, daß, ohne den Mißbrauch des Worts hier zu rügen (das schon einen Widerspruch in sich enthält), die erstere \match{Art} zu empfinden allein zum vorigen Titel, nämlich dem der  Vollkommenheit, gehöre. – Denn der, welcher sich im bloßen Bewußtsein seiner Rechtschaffenheit glücklich fühlen soll, besitzt schon diejenige Vollkommenheit, die im vorigen Titel für denjenigen Zweck erklärt war, der zugleich Pflicht ist. 
	
	\subsection*{tg456.2.4} 
	\textbf{Source : }Die Metaphysik der Sitten/Zweiter Teil. Metaphysische Anfangsgründe der Tugendlehre/Einleitung/VIII. Exposition der Tugendpflichten als weiter Pflichten\\  
	
	\noindent\textbf{Paragraphe : }Allein diese Pflicht ist bloß ethisch, d.i. von weiter Verbindlichkeit. Wie weit man in Bearbeitung (Erweiterung oder Berichtigung seines Verstandesvermögens, d.i. in Kenntnissen oder in Kunstfähigkeit) gehen solle, schreibt kein Vernunftprinzip bestimmt vor, auch macht die Verschiedenheit der Lagen, worein Menschen kommen können, die Wahl der \match{Art} der Beschäftigung, dazu er sein Talent anbauen soll, sehr willkürlich. – Es ist also hier kein Gesetz der Vernunft für die Handlungen, sondern bloß für die Maxime der Handlungen,  welche so lautet: »baue deine Gemüts- und Leibeskräfte zur Tauglichkeit für alle Zwecke an, die dir aufstoßen können«, ungewiß, welche davon einmal die deinigen werden könnten. 
	
	\subsection*{tg459.2.2} 
	\textbf{Source : }Die Metaphysik der Sitten/Zweiter Teil. Metaphysische Anfangsgründe der Tugendlehre/Einleitung/XI. Das Schema der Tugendpflichten\\  
	
	\noindent\textbf{Paragraphe : }Das Schema der Tugendpflichten kann obigen Grundsätzen gemäß auf folgende \match{Art} verzeichnet werden: 
	
	\subsection*{tg460.2.27} 
	\textbf{Source : }Die Metaphysik der Sitten/Zweiter Teil. Metaphysische Anfangsgründe der Tugendlehre/Einleitung/XII. Ästhetische Vorbegriffe der Empfänglichkeit des Gemüts für Pflichtbegriffe überhaupt\\  
	
	\noindent\textbf{Paragraphe : }Achtung (reverentia) ist eben sowohl etwas bloß Subjektives; ein Gefühl eigener Art, nicht ein Urteil über einen Gegenstand, den zu bewirken, oder zu befördern, es eine Pflicht gäbe. Denn sie könnte, als Pflicht betrachtet, nur durch die Achtung, die wir vor ihr haben, vorgestellt werden. Zu dieser also eine Pflicht zu haben würde so viel sagen, als zur Pflicht verpflichtet werden. – Wenn es demnach heißt: Der Mensch hat eine Pflicht der Selbstschätzung, so ist das unrichtig gesagt und es müßte vielmehr heißen: das Gesetz in ihm zwingt ihm unvermeidlich Achtung für sein eigenes Wesen ab und dieses Gefühl (welches von eigner \match{Art} ist) ist ein Grund gewisser Pflichten, d.i. gewisser Handlungen, die mit der Pflicht gegen sich selbst zusammen bestehen können, nicht: er habe eine Pflicht der Achtung gegen sich; denn er muß Achtung vor dem Gesetz in sich selbst haben, um sich nur eine Pflicht überhaupt denken zu können. 
	
	\subsection*{tg461.2.3} 
	\textbf{Source : }Die Metaphysik der Sitten/Zweiter Teil. Metaphysische Anfangsgründe der Tugendlehre/Einleitung/XIII. Allgemeine Grundsätze der Metaphysik der Sitten in Behandlung einer reinen Tugendlehre\\  
	
	\noindent\textbf{Paragraphe : }Denn alle moralische Beweise können, als philosophische, nur vermittelst einer Vernunfterkenntnis aus Begriffen, nicht, wie die Mathematik sie gibt, durch die Konstruktion der Begriffe geführt werden; die letztern verstatten Mehrheit  der Beweise eines und desselben Satzes; weil in der Anschauung a priori es mehrere Bestimmungen der Beschaffenheit eines Objekts geben kann, die alle auf eben denselben Grund zurück führen. – Wenn z.B. für die Pflicht der Wahrhaftigkeit ein Beweis, erstlich aus dem Schaden, den die Lüge andern Menschen verursacht, dann aber auch aus der Nichtswürdigkeit eines Lügners und der Verletzung der Achtung gegen sich selbst, geführt werden will, so ist im ersteren eine Pflicht des Wohlwollens, nicht eine der Wahrhaftigkeit, mithin nicht diese, von der man den Beweis verlangte, sondern eine andere Pflicht bewiesen worden. – Was aber die Mehrheit der Beweise für einen und denselben Satz betrifft, womit man sich tröstet, daß die Menge der Gründe den Mangel am Gewicht eines jeden einzeln genommen ergänzen werde, so ist dieses ein sehr unphilosophischer Behelf: weil er Hinterlist und Unredlichkeit verrät; – denn verschiedene unzureichende Gründe, neben einander gestellt, ergänzen nicht der eine den Mangel des anderen zur Gewißheit, ja nicht einmal zur Wahrscheinlichkeit. Sie müssen als Grund und Folge in einer Reihe, bis zum zureichenden Grunde, fortschreiten und können auch nur auf solche \match{Art} beweisend sein. – Und gleichwohl ist dies der gewöhnliche Handgriff der Überredungskunst. 
	
	\subsection*{tg464.2.2} 
	\textbf{Source : }Die Metaphysik der Sitten/Zweiter Teil. Metaphysische Anfangsgründe der Tugendlehre/Einleitung/XVI. Zur Tugend wird Apathie (als Stärke betrachtet) notwendig vorausgesetzt\\  
	
	\noindent\textbf{Paragraphe : }Dieses Wort ist, gleich als ob es Fühllosigkeit, mithin subjektive Gleichgültigkeit in Ansehung der Gegenstände der Willkür, bedeutete, in übelen Ruf gekommen; man nahm es für Schwäche. Dieser Mißdeutung kann dadurch vorgebeugt werden, daß man diejenige Affektlosigkeit, welche von der Indifferenz zu unterscheiden ist, die moralische Apathie nennt: da die Gefühle aus sinnlichen Eindrücken ihren Einfluß auf das moralische nur dadurch verlieren, daß die Achtung fürs Gesetz über sie insgesamt mächtiger wird. – Es ist nur die scheinbare Stärke eines Fieberkranken, die den lebhaften Anteil selbst am Guten bis zum Affekt steigen, oder vielmehr darin ausarten läßt. Man nennt den Affekt dieser \match{Art} Enthusiasm, und dahin ist auch die Mäßigung zu deuten, die man selbst für Tugendausübungen zu empfehlen pflegt (insani sapiens nomen habeat aequus iniqui – ultra, quam satis est virtutem si petat ipsam. Horat.). Denn sonst ist es ungereimt zu wähnen, man könne auch wohl allzuweise, allzutugendhaft sein. Der Affekt gehört immer zur Sinnlichkeit; er mag durch einen Gegenstand erregt werden, welcher es wolle. Die wahre Stärke der Tugend ist das Gemüt in Ruhe, mit einer überlegten und festen Entschließung, ihr Gesetz in Ausübung zu bringen. Das ist der Zustand der Gesundheit im moralischen Leben; dagegen der Affekt, selbst wenn er durch die Vorstellung des Guten aufgeregt wird, eine augenblicklich glänzende Erscheinung ist, welche Mattigkeit hinterläßt. – Phantastisch-tugendhaft aber kann doch der genannt werden, der keine in Ansehung der Moralität gleichgültige Dinge (adiaphora) einräumt und sich alle seine Schritte und Tritte mit Pflichten als mit Fußangeln bestreut und es nicht gleichgültig findet, ob ich mich mit Fleisch oder Fisch, mit Bier oder Wein, wenn mir beides bekömmt, nähre; eine Mikrologie, welche, wenn man sie in die Lehre der Tugend aufnähme, die Herrschaft derselben zur Tyrannei machen würde. 
	
	\subsection*{tg469.2.17} 
	\textbf{Source : }Die Metaphysik der Sitten/Zweiter Teil. Metaphysische Anfangsgründe der Tugendlehre/I. Ethische Elementarlehre/I. Teil. Von den Pflichten gegen sich selbst überhaupt/Einleitung\\  
	
	\noindent\textbf{Paragraphe : }Die Einteilung kann nur in Ansehung des Objekts der Pflicht, nicht in Ansehung des sich verpflichtenden Subjekts, gemacht werden. Das verpflichtete so wohl als das verpflichtende Subjekt ist immer nur der Mensch, und wenn es uns, in theoretischer Rücksicht, gleich erlaubt ist, im Menschen Seele und Körper als Naturbeschaffenheiten des Menschen von einander zu unterscheiden, so ist es doch nicht erlaubt, sie als verschiedene den Menschen verpflichtende Substanzen zu denken, um zur Einteilung in Pflichten gegen den Körper und gegen die Seele berechtigt zu sein. – Wir sind, weder durch Erfahrung, noch durch Schlüsse der Vernunft, hinreichend darüber belehrt, ob der Mensch eine Seele (als in ihm wohnende, vom Körper unterschiedene und von diesem unabhängig zu denken vermögende, d.i. geistige Substanz) enthalte, oder ob nicht vielmehr das Leben eine Eigenschaft der Materie sein möge, und wenn es sich auch auf die erstere \match{Art} verhielte, so würde doch keine Pflicht des Men schen gegen einen Körper (als verpflichtendes Subjekt), ob er gleich der menschliche ist, denkbar sein. 
	
	\subsection*{tg471.2.37} 
	\textbf{Source : }Die Metaphysik der Sitten/Zweiter Teil. Metaphysische Anfangsgründe der Tugendlehre/I. Ethische Elementarlehre/I. Teil. Von den Pflichten gegen sich selbst überhaupt/Erstes Buch. Von den vollkommenen Pflichten gegen sich selbst/Erstes Hauptstück. Die Pflicht des Menschen gegen sich selbst, als einem animalischen Wesen\\  
	
	\noindent\textbf{Paragraphe : }Die Geschlechtsneigung wird auch Liebe (in der engsten Bedeutung des Worts) genannt und ist in der Tat die größte Sinnenlust, die an einem Gegenstande möglich ist; – nicht bloß sinnliche Lust, wie an Gegenständen, die in der bloßen Reflexion über sie gefallen (da die Empfänglichkeit für sie Geschmack heißt), sondern die Lust aus dem Genusse einer anderen Person, die also zum Begehrungsvermögen und zwar der höchsten Stufe desselben, der Leidenschaft, gehört. Sie kann aber weder zur Liebe des Wohlgefallens, noch der des Wohlwollens gezählt werden (denn beide halten eher vom fleischlichen Genuß ab), sondern ist eine Lust von besonderer \match{Art} (sui generis) und das Brünstigsein hat mit der moralischen Liebe eigentlich nichts gemein, wiewohl sie mit der letzteren, wenn die praktische Vernunft mit ihren einschränkenden Bedingungen hinzu kommt, in enge Verbindung treten kann. 
	
	\subsection*{tg471.2.44} 
	\textbf{Source : }Die Metaphysik der Sitten/Zweiter Teil. Metaphysische Anfangsgründe der Tugendlehre/I. Ethische Elementarlehre/I. Teil. Von den Pflichten gegen sich selbst überhaupt/Erstes Buch. Von den vollkommenen Pflichten gegen sich selbst/Erstes Hauptstück. Die Pflicht des Menschen gegen sich selbst, als einem animalischen Wesen\\  
	
	\noindent\textbf{Paragraphe : }Das Laster in dieser \match{Art} der Unmäßigkeit wird hier nicht aus dem Schaden, oder den körperlichen Schmerzen (solchen Krankheiten), die der Mensch sich dadurch zuzieht, beurteilt; denn da wäre es ein Prinzip des Wohlbefindens und der Behaglichkeit (folglich der Glückseligkeit), wodurch ihm entgegen gearbeitet werden sollte, welches aber nie eine Pflicht, sondern nur eine Klugheitsregel begründen kann: wenigstens wäre es kein Prinzip einer direkten Pflicht. 
	
	\subsection*{tg471.2.51} 
	\textbf{Source : }Die Metaphysik der Sitten/Zweiter Teil. Metaphysische Anfangsgründe der Tugendlehre/I. Ethische Elementarlehre/I. Teil. Von den Pflichten gegen sich selbst überhaupt/Erstes Buch. Von den vollkommenen Pflichten gegen sich selbst/Erstes Hauptstück. Die Pflicht des Menschen gegen sich selbst, als einem animalischen Wesen\\  
	
	\noindent\textbf{Paragraphe : }Der Schmaus, als förmliche Einladung zur Unmäßigkeit in beiderlei \match{Art} des Genusses, hat doch, außer dem bloß physischen Wohlleben, noch etwas zum sittlichen Zweck Abzielendes an sich, nämlich viel Menschen und lange zu wechselseitiger Mitteilung zusammen zu halten: gleichwohl aber, da eben die Menge (wenn sie, wie Chesterfield sagt, über die Zahl der Musen geht) nur eine kleine Mitteilung (mit den nächsten Beisitzern) erlaubt, mithin die Veranstaltung jenem Zweck widerspricht, so bleibt sie immer Verleitung zum Unsittlichen, nämlich der Unmäßigkeit, der Übertretung der Pflicht gegen sich selbst; auch ohne auf die physischen Nachteile der Überladung, die vielleicht vom Arzt gehoben werden können, zu sehen. Wie weit geht die sittliche Befugnis, diesen Einladungen zur Unmäßigkeit Gehör zu geben? 
	
	\subsection*{tg472.2.34} 
	\textbf{Source : }Die Metaphysik der Sitten/Zweiter Teil. Metaphysische Anfangsgründe der Tugendlehre/I. Ethische Elementarlehre/I. Teil. Von den Pflichten gegen sich selbst überhaupt/Erstes Buch. Von den vollkommenen Pflichten gegen sich selbst\\  
	
	\noindent\textbf{Paragraphe : }Allein der Mensch als Person betrachtet, d.i. als Subjekt einer moralisch-praktischen Vernunft, ist über allen Preis erhaben; denn als ein solcher (homo noumenon) ist er nicht bloß als Mittel zu anderer ihren, ja selbst seinen eigenen Zwecken, sondern als Zweck an sich seihst zu schätzen, d.i. er besitzt eine Würde (einen absoluten innern Wert), wodurch er allen andern vernünftigen Weltwesen Achtung für ihn abnötigt, sich mit jedem anderen dieser \match{Art} messen und auf den Fuß der Gleichheit schätzen kann. 
	
	\subsection*{tg472.2.47} 
	\textbf{Source : }Die Metaphysik der Sitten/Zweiter Teil. Metaphysische Anfangsgründe der Tugendlehre/I. Ethische Elementarlehre/I. Teil. Von den Pflichten gegen sich selbst überhaupt/Erstes Buch. Von den vollkommenen Pflichten gegen sich selbst\\  
	
	\noindent\textbf{Paragraphe : }Ist nicht in dem Menschen das Gefühl der Erhabenheit seiner Bestimmung, d.i. die Gemütserhebung (elatio animi), als Schätzung seiner selbst, mit dem Eigendünkel (arrogantia), welcher der wahren Demut (humilitas moralis) gerade entgegengesetzt ist, zu nahe verwandt, als daß zu jener aufzumuntern es ratsam wäre; selbst in Vergleichung mit anderen Menschen, nicht bloß mit dem Gesetz? oder würde diese \match{Art} von Selbstverleugnung nicht vielmehr den Ausspruch anderer bis zur Geringschätzung unserer Person steigern, und so der Pflicht (der Achtung) gegen uns selbst zuwider sein? Das Bücken und Schmiegen vor einem  Menschen scheint in jedem Fall eines Menschen unwürdig zu sein. 
	
	\subsection*{tg475.2.4} 
	\textbf{Source : }Die Metaphysik der Sitten/Zweiter Teil. Metaphysische Anfangsgründe der Tugendlehre/I. Ethische Elementarlehre/I. Teil. Von den Pflichten gegen sich selbst überhaupt/Erstes Buch. Von den vollkommenen Pflichten gegen sich selbst/Zweites Hauptstück. Die Pflicht des Menschen gegen sich selbst, bloß als einem moralischen Wesen/Episodischer Abschnitt. Von der Amphibolie der moralischen Reflexionsbegriffe\\  
	
	\noindent\textbf{Paragraphe : }Diese vermeinte Pflicht kann nun auf unpersönliche, oder zwar persönliche aber schlechterdings unsichtbare (den äußeren Sinnen nicht darzustellende) Gegenstände bezogen werden. – Die erstere (außermenschliche) können der bloße Naturstoff, oder der zur Fortpflanzung organisierte, aber empfindungslose, oder der mit Empfindung und Willkür begabte Teil der Natur (Mineralien, Pflanzen, Tiere) sein; die zweite (übermenschliche) können als geistige Wesen (Engel, Gott) gedacht werden. – Ob zwischen Wesen beider \match{Art} und den Menschen ein Pflichtverhältnis, und welches dazwischen statt finde, wird nun gefragt. 
	
	\subsection*{tg477.2.11} 
	\textbf{Source : }Die Metaphysik der Sitten/Zweiter Teil. Metaphysische Anfangsgründe der Tugendlehre/I. Ethische Elementarlehre/I. Teil. Von den Pflichten gegen sich selbst überhaupt/2. Buch: Die Pflichten gegen sich selbst/Erster Abschnitt. Von der Pflicht gegen sich selbst in Entwickelung und Vermehrung seiner Naturvollkommenheit, d.i. in pragmatischer Absicht\\  
	
	\noindent\textbf{Paragraphe : }Die Pflicht des Menschen gegen sich selbst in Ansehung seiner physischen Vollkommenheit ist aber nur weite und unvollkommene Pflicht: weil sie zwar ein Gesetz für die Maxime der Handlungen enthält, in Ansehung der Handlungen selbst aber, ihrer \match{Art} und ihrem Grade nach, nichts bestimmt, sondern der freien Willkür einen Spielraum verstattet. 
	
	\subsection*{tg481.2.60} 
	\textbf{Source : }Die Metaphysik der Sitten/Zweiter Teil. Metaphysische Anfangsgründe der Tugendlehre/I. Ethische Elementarlehre/II. Teil. Von den Tugendpflichten gegen andere/Erstes Hauptstück. Von den Pflichten gegen andere, bloß als Menschen/Erster Abschnitt. Von der Liebespflicht gegen andere Menschen\\  
	
	\noindent\textbf{Paragraphe : }Dankbarkeit aber muß auch noch besonders als heilige Pflicht, d.i. als eine solche, deren Verletzung die moralische Triebfeder zum Wohltun in dem Grundsatze selbst vernichten kann (als skandalöses Beispiel), angesehen werden. Denn heilig ist derjenige moralische Gegenstand, in Ansehung dessen die Verbindlichkeit durch keinen ihr gemäßen Akt völlig getilgt werden kann (wobei der Verpflichtete immer noch verpflichtet bleibt). Alle andere ist gemeine Pflicht. – Man kann aber durch keine Vergeltung einer empfangenen Wohltat über dieselbe quittieren; weil der Empfänger den Vorzug des Verdienstes, den der Geber hat, nämlich der erste im Wohlwollen gewesen zu sein, diesem nie abgewinnen kann. – Aber, auch ohne einen solchen Akt (des Wohltuns) ist selbst das bloße herzliche Wohlwollen schon Grund der Verpflichtung zur Dankbarkeit. Eine dankbare Gesinnung dieser \match{Art} wird Erkenntlichkeit genannt. 
	
	\subsection*{tg481.2.71} 
	\textbf{Source : }Die Metaphysik der Sitten/Zweiter Teil. Metaphysische Anfangsgründe der Tugendlehre/I. Ethische Elementarlehre/II. Teil. Von den Tugendpflichten gegen andere/Erstes Hauptstück. Von den Pflichten gegen andere, bloß als Menschen/Erster Abschnitt. Von der Liebespflicht gegen andere Menschen\\  
	
	\noindent\textbf{Paragraphe : }In der Tat, wenn ein anderer leidet und ich mich durch seinen Schmerz, dem ich doch nicht abhelfen kann, auch (vermittelst der Einbildungskraft) anstecken lasse, so leiden ihrer zwei; ob zwar das Übel eigentlich (in der Natur) nur Einen trifft. Es kann aber unmöglich Pflicht sein, die Übel in der Welt zu vermehren, mithin auch nicht, aus Mitleid wohl zu tun; wie dann dieses auch eine beleidigende \match{Art} des Wohltuns sein würde, indem es ein Wohlwollen ausdrückt, was sich auf den Unwürdigen bezieht und Barmherzigkeit genannt wird, unter Menschen, welche mit ihrer Würdigkeit, glücklich zu sein, eben nicht prahlen dürfen, und respektiv gegen einander gar nicht vorkommen sollte. 
	
	\subsection*{tg482.2.32} 
	\textbf{Source : }Die Metaphysik der Sitten/Zweiter Teil. Metaphysische Anfangsgründe der Tugendlehre/I. Ethische Elementarlehre/II. Teil. Von den Tugendpflichten gegen andere/Erstes Hauptstück. Von den Pflichten gegen andere, bloß als Menschen/Zweiter Abschnitt. Von den Tugendpflichten gegen andere Menschen aus der ihnen gebührenden Achtung\\  
	
	\noindent\textbf{Paragraphe : }Der Hochmut (superbia und, wie dieses Wort es ausdrückt, die Neigung, immer oben zu schwimmen) ist eine  \match{Art} von Ehrbegierde (ambitio), nach welcher wir anderen Menschen ansinnen, sich selbst in Vergleichung mit uns gering zu schätzen, und ist also ein der Achtung, worauf jeder Mensch gesetzmäßigen Anspruch machen kann, widerstreitendes Laster. 
	
	\subsection*{tg486.2.28} 
	\textbf{Source : }Die Metaphysik der Sitten/Zweiter Teil. Metaphysische Anfangsgründe der Tugendlehre/II. Ethische Methodenlehre/1. Abschnitt. Die ethische Didaktik\\  
	
	\noindent\textbf{Paragraphe : }4. L. Das beweist nun wohl, daß du noch so ziemlich ein gutes Herz hast; laß aber sehen, ob du dabei auch guten Verstand zeigest. – Würdest du wohl dem Faulenzer weiche Polster verschaffen, damit er im süßen Nichtstun sein Leben dahin bringe, oder dem Trunkenbolde es an Wein, und was sonst zur Berauschung gehört, nicht ermangeln lassen, dem Betrüger eine einnehmende Gestalt und Manieren geben, um andere zu überlisten, oder dem Gewalttätigen Kühnheit und starke Faust, um andere überwältigen zu können? Das sind ja so viel Mittel, die ein jeder sich wünscht, um nach seiner \match{Art} glücklich zu sein. S. Nein, das nicht. 
	
	\subsection*{tg486.2.36} 
	\textbf{Source : }Die Metaphysik der Sitten/Zweiter Teil. Metaphysische Anfangsgründe der Tugendlehre/II. Ethische Methodenlehre/1. Abschnitt. Die ethische Didaktik\\  
	
	\noindent\textbf{Paragraphe : }In dieser katechetischen Moralunterweisung würde es zur sittlichen Bildung von großem Nutzen sein, bei jeder Pflichtzergliederung einige kasuistische Fragen aufzuwerfen und die versammelten Kinder ihren Verstand versuchen zu lassen, wie ein jeder von ihnen die ihm vorgelegte verfängliche Aufgabe aufzulösen meinete. – Nicht allein, daß dieses eine, der Fähigkeit des Ungebildeten am meisten angemessene, Kultur der Vernunft ist (weil diese in Fragen, die, was Pflicht ist, betreffen, weit leichter entscheiden kann, als in Ansehung der spekulativen) und so den Verstand der Jugend überhaupt zu schärfen die schicklichste \match{Art} ist: sondern vornehmlich deswegen, weil es in der Natur des Menschen liegt, das zu lieben, worin und in dessen Bearbeitung er es bis zu einer Wissenschaft  (mit der er nun Bescheid weiß) gebracht hat, und so der Lehrling durch dergleichen Übungen unvermerkt in das Interesse der Sittlichkeit gezogen wird. 
	
	\subsection*{tg487.2.5} 
	\textbf{Source : }Die Metaphysik der Sitten/Zweiter Teil. Metaphysische Anfangsgründe der Tugendlehre/II. Ethische Methodenlehre/2. Abschnitt. Die ethische Asketik\\  
	
	\noindent\textbf{Paragraphe : }Die Kultur der Tugend, d.i. die moralische Asketik, hat, in Ansehung des Prinzips der rüstigen, mutigen und wackeren Tugendübung den Wahlspruch der Stoiker: gewöhne dich, die zufälligen Lebensübel zu ertragen und die eben so überflüssigen Ergötzlichkeiten zu entbehren (assuesce incommodis et desuesce commoditatibus vitae). Es ist eine \match{Art} von Diätetik für den Menschen, sich moralisch gesund zu erhalten. Gesundheit ist aber nur ein negatives  Wohlbefinden, sie selber kann nicht gefühlt werden. Es muß etwas dazu kommen, was einen angenehmen Lebensgenuß gewährt und doch bloß moralisch ist. Das ist das jederzeit fröhliche Herz in der Idee des tugendhaften Epikurs. Denn wer sollte wohl mehr Ursache haben, frohen Muts zu sein und nicht darin selbst eine Pflicht finden, sich in eine fröhliche Gemütsstimmung zu versetzen und sie sich habituell zu machen, als der, welcher sich keiner vorsätzlichen Übertretung bewußt und, wegen des Verfalls in eine solche, gesichert ist (hic murus ahenëus esto etc. Horat.). – Die Mönchsasketik hingegen, welche aus abergläubischer Furcht, oder geheucheltem Abscheu an sich selbst, mit Selbstpeinigung und Fleischeskreuzigung zu Werke geht, zweckt auch nicht auf Tugend, sondern auf schwärmerische Entsündigung ab, sich selbst Strafe aufzulegen und, anstatt sie moralisch (d.i. in Absicht auf die Besserung) zu bereuen, sie büßen zu wollen; welches, bei einer selbstgewählten und an sich vollstreckten Strafe (denn die muß immer ein anderer auflegen), ein Widerspruch ist, und kann auch den Frohsinn, der die Tugend begleitet, nicht bewirken, vielmehr nicht ohne geheimen Haß gegen das Tugendgebot statt finden. – Die ethische Gymnastik besteht also nur in der Bekämpfung der Naturtriebe, die das Maß erreicht, über sie bei vorkommenden, der Moralität Gefahr drohenden, Fällen Meister werden zu können; mithin die wacker und, im Bewußtsein seiner wiedererworbenen Freiheit, fröhlich macht. Etwas bereuen (welches bei der Rückerinnerung ehemaliger Übertretungen unvermeidlich, ja wobei diese Erinnerung nicht schwinden zu lassen es so gar Pflicht ist) und sich eine Pönitenz auferlegen (z.B. das Fasten), nicht in diätetischer, sondern frommer Rücksicht, sind zwei sehr verschiedene, moralisch gemeinte, Vorkehrungen, von denen die letztere, welche freudenlos, finster und mürrisch ist, die Tugend selbst verhaßt macht und ihre Anhänger verjagt. Die Zucht (Disziplin), die der Mensch an sich selbst verübt, kann daher nur durch den Frohsinn, der sie begleitet, verdienstlich und exemplarisch werden. 
	
	\subsection*{tg489.2.4} 
	\textbf{Source : }Die Metaphysik der Sitten/Fußnoten\\  
	
	\noindent\textbf{Paragraphe : }
	
	2 Man kann Sinnlichkeit durch das Subjektive unserer Vorstellungen überhaupt erklären; denn der Verstand bezieht allererst die Vorstellungen auf ein Objekt, d.i. er allein denkt sich etwas vermittelst derselben. Nun kann das Subjektive unserer Vorstellung entweder von der \match{Art} sein, daß es auch auf ein Objekt zum Erkenntnis desselben (der Form oder Materie nach, da es im ersteren Falle reine Anschauung, im zweiten Empfindung heißt) bezogen werden kann. In diesem Fall ist die Sinnlichkeit, als Empfänglichkeit der gedachten Vorstellung, der Sinn: aber das Subjektive der Vorstellung kann gar kein Erkenntnisstück werden; weil es bloß die Beziehung derselben aufs Subjekt und nichts zur Erkenntnis des Objekts Brauchbares enthält, und alsdann heißt diese Empfänglichkeit der Vorstellung Gefühl; welches die Wirkung der Vorstellung (diese mag sinnlich oder intellektuell 
	sein) aufs Subjekt enthält und zur Sinnlichkeit gehört, obgleich die Vorstellung selbst zum Verstande oder der Vernunft gehören mag. 
	
	\unnumberedsection{Fall (24)} 
	\subsection*{tg430.2.51} 
	\textbf{Source : }Die Metaphysik der Sitten/Erster Teil. Metaphysische Anfangsgründe der Rechtslehre/Einleitung in die Metaphysik der Sitten\\  
	
	\noindent\textbf{Paragraphe : }
	
	Gesetz (ein moralischpraktisches) ist ein Satz, der einen kategorischen Imperativ (Gebot) enthält. Der Gebietende (imperans) durch ein Gesetz ist der Gesetzgeber (legislator). Er ist Urheber (autor) der Verbindlichkeit nach dem Gesetze, aber nicht immer Urheber des Gesetzes. Im letzteren \match{Fall} würde das Gesetz positiv (zufällig) und willkürlich sein. Das Gesetz, was uns a priori und unbedingt durch unsere eigene Vernunft verbindet, kann auch als aus dem Willen eines höchsten Gesetzgebers, d.i. eines solchen, der lauter Rechte und keine Pflichten hat (mithin dem göttlichen Willen), hervorgehend ausgedrückt werden, welches aber nur die Idee von einem moralischen Wesen bedeutet, dessen Wille für alle Gesetz ist, ohne ihn doch als Urheber desselben zu denken. 
	
	\subsection*{tg431.2.38} 
	\textbf{Source : }Die Metaphysik der Sitten/Erster Teil. Metaphysische Anfangsgründe der Rechtslehre/Einleitung in die Rechtslehre\\  
	
	\noindent\textbf{Paragraphe : }Dieses vermeinte Recht soll eine Befugnis sein, im \match{Fall} der Gefahr des Verlusts meines eigenen Lebens, einem anderen, der mir nichts zu Leide tat, das Leben zu nehmen. Es fällt in die Augen, daß hierin ein Widerspruch der Rechtslehre mit sich selbst enthalten sein müsse – denn es ist hier nicht von einem ungerechten Angreifer auf mein Leben, dem ich durch Beraubung des seinen zuvorkomme (ius inculpatae tutelae), die Rede, wo die Anempfehlung der Mäßigung (moderamen) nicht einmal zum Recht, sondern nur zur Ethik gehört, sondern von einer erlaubten Gewalttätigkeit gegen den, der keine gegen mich ausübte. 
	
	\subsection*{tg433.2.43} 
	\textbf{Source : }Die Metaphysik der Sitten/Erster Teil. Metaphysische Anfangsgründe der Rechtslehre/1. Teil. Das Privatrecht vom äußeren Mein und Dein überhaupt/1. Hauptstück\\  
	
	\noindent\textbf{Paragraphe : }Dieses kann auch auf den \match{Fall} angewendet werden, da ich ein Versprechen akzeptiert habe; denn da wird meine Habe und Besitz an dem Versprochenen da durch nicht aufgehoben, daß der Versprechende zu einer Zeit sagte: diese Sache soll dein sein, eine Zeit hernach aber von ebenderselben Sache  sagt: ich will jetzt, die Sache solle nicht dein sein. Denn es hat mit solchen intellektuellen Verhältnissen die Bewandtnis, als ob jener ohne eine Zeit zwischen beiden Deklarationen seines Willens gesagt hätte, sie soll dein sein, und auch, sie soll nicht dein sein, was sich dann selbst widerspricht. 
	
	\subsection*{tg437.2.93} 
	\textbf{Source : }Die Metaphysik der Sitten/Erster Teil. Metaphysische Anfangsgründe der Rechtslehre/1. Teil. Das Privatrecht vom äußeren Mein und Dein überhaupt/2. Hauptstück. Von der Art, etwas Äußeres zu erwerben/3. Abschnitt. Von dem auf dingliche Art persönlichen Recht\\  
	
	\noindent\textbf{Paragraphe : }Die Verwechselung des persönlichen Rechts mit dem Sachenrecht ist noch in einem anderen, unter den Verdingungsvertrag gehörigen, Falle (B, II, α), nämlich dem der Einmietung (ius incolatus), ein Stoff zu Streitigkeiten. – Es fragt sich nämlich: ist der Eigentümer, wenn er sein an jemanden vermietetes Haus (oder seinen Grund) vor Ablauf der Mietszeit an einen anderen verkauft, verbunden, die Bedingung der fortdauernden Miete dem Kaufkontrakte beizufügen, oder kann man sagen: Kauf bricht Miete (doch in einer durch den Gebrauch bestimmten Zeit der Aufkündigung)? – Im ersteren \match{Fall} hätte das Haus wirklich eine Belästigung (onus) auf sich liegend, ein Recht in dieser Sache, das der Mieter sich an derselben (dem Hause) erworben hätte; welches auch wohl geschehen kann (durch Ingrossation des Mietskontrakts auf das Haus), aber alsdenn kein bloßer Mietskontrakt sein würde, sondern wozu noch ein anderer Vertrag (dazu sich nicht viel Vermieter verstehen würden) hinzukommen müßte. Also gilt der Satz: »Kauf bricht Miete «, d.i. das volle Recht in einer Sache (das Eigentum) überwiegt alles persönliche Recht, was mit ihm  nicht zusammen bestehen kann; wobei doch die Klage aus dem Grunde des letzteren dem Mieter offen bleibt, ihn wegen des aus der Zerreißung des Kontrakts entspringenden Nachteils schadenfrei zu halten. 
	
	\subsection*{tg438.2.16} 
	\textbf{Source : }Die Metaphysik der Sitten/Erster Teil. Metaphysische Anfangsgründe der Rechtslehre/1. Teil. Das Privatrecht vom äußeren Mein und Dein überhaupt/2. Hauptstück. Von der Art, etwas Äußeres zu erwerben/Episodischer Abschnitt. Von der idealen Erwerbung eines äußeren Gegenstandes der Willkür\\  
	
	\noindent\textbf{Paragraphe : }»Es ist möglich, durch Erbeseinsetzung zu erwerben.« – Denn der Erblasser Caius verspricht und erklärt in seinem letzten Willen dem Titus, der nichts von jenem Versprechen weiß, seine Habe solle im Sterbefall auf diesen übergehen, und bleibt also, so lange er lebt, alleiniger Eigentümer derselben. Nun kann zwar durch den bloßen einseitigen Willen nichts auf den anderen übergehen: sondern es wird über dem Versprechen noch Annehmung (acceptatio) des anderen Teils dazu erfordert und ein gleichzeitiger Wille (voluntas simultanea), welcher jedoch hier mangelt; denn so lange Caius lebt, kann Titus nicht ausdrücklich akzeptieren, um dadurch zu erwerben; weil jener nur auf den \match{Fall} des Todes versprochen hat (denn sonst wäre das Eigentum einen Augenblick gemeinschaftlich, welches nicht der Wille des Erblassers ist). – Dieser aber erwirbt doch stillschweigend ein eigentümliches Recht an der Verlassenschaft als ein Sachenrecht, nämlich ausschlüßlich sie zu akzeptieren (ius in re iacente), daher diese in dem gedachten Zeitpunkt hereditas iacens heißt. Da nun jeder Mensch notwendigerweise (weil er dadurch wohl gewinnen, nie aber verlieren kann) ein solches Recht, mithin auch stillschweigend akzeptiert und Titus nach dem Tode des Caius in diesem Falle ist, so kann er die Erbschaft durch Annahme des Versprechens erwerben, und sie ist nicht etwa mittlerweile ganz herrenlos (res nullius),  sondern nur erledigt (res vacua) gewesen; weil er ausschlüßlich das Recht der Wahl hatte, ob er die hinterlassene Habe zu der seinigen machen wollte, oder nicht. 
	
	\subsection*{tg441.2.34} 
	\textbf{Source : }Die Metaphysik der Sitten/Erster Teil. Metaphysische Anfangsgründe der Rechtslehre/2. Teil. Das öffentliche Recht/1. Abschnitt. Das Staatsrecht\\  
	
	\noindent\textbf{Paragraphe : }
	Also kann nur das Volk, durch seine von ihm selbst abgeordnete Stellvertreter (die Jury), über jeden in demselben, obwohl nur mittelbar, richten. – Es wäre auch unter der Würde des Staatsoberhaupts, den Richter zu spielen, d.i. sich in die Möglichkeit zu versetzen, Unrecht zu tun, und so in den \match{Fall} der Appellation (a rege male informato ad regem melius informandum) zu geraten. 
	
	\subsection*{tg441.2.43} 
	\textbf{Source : }Die Metaphysik der Sitten/Erster Teil. Metaphysische Anfangsgründe der Rechtslehre/2. Teil. Das öffentliche Recht/1. Abschnitt. Das Staatsrecht\\  
	
	\noindent\textbf{Paragraphe : }Ja es kann auch selbst in der Konstitution kein Artikel enthalten sein, der es einer Gewalt im Staat möglich machte, sich, im \match{Fall} der Übertretung der Konsti tutionalgesetze durch den obersten Befehlshaber, ihm zu widersetzen, mithin ihn einzuschränken. Denn der, welcher die Staatsgewalt einschränken soll, muß doch mehr, oder wenigstens gleiche Macht haben, als derjenige, welcher eingeschränkt wird, und, als ein rechtmäßiger Gebieter, der den Untertanen befähle, sich zu widersetzen, muß er sie auch schützen können,  und in jedem vorkommenden Fall rechtskräftig urteilen, mithin öffentlich den Widerstand befehligen können. Alsdann ist aber nicht jener, sondern dieser der oberste Befehlshaber; welches sich widerspricht. Der Souverän verfährt alsdann durch seinen Minister zugleich als Regent, mithin despotisch, und das Blendwerk, das Volk durch die Deputierte desselben die einschränkende Gewalt vorstellen zu lassen (da es eigentlich nur die gesetzgebende hat), kann die Despotie nicht so verstecken, daß sie aus den Mitteln, deren sich der Minister bedient, nicht hervorblickte. Das Volk, das durch seine Deputierte (im Parlament) repräsentiert wird, hat an diesen Gewährsmännern seiner Freiheit und Rechte Leute, die für sich und ihre Familien, und dieser ihre vom Minister abhängigen Versorgung, in Armeen, Flotte und Zivilämtern, lebhaft interessiert sind, und die (statt des Widerstandes gegen die Anmaßung der Regierung, dessen öffentliche Ankündigung ohnedem eine dazu schon vorbereitete Einhelligkeit im Volk bedarf, die aber im Frieden nicht erlaubt sein kann) vielmehr immer bereit sind, sich selbst der Regierung in die Hände zu spielen. – Also ist die sogenannte gemäßigte Staatsverfassung, als Konstitution des innern Rechts des Staats, ein Unding, und, anstatt zum Recht zu gehören, nur ein Klugheitsprinzip, um, so viel als möglich, dem mächtigen Übertreter der Volksrechte seine willkürliche Einflüsse auf die Regierung nicht zu erschweren, sondern unter dem Schein einer dem Volk verstatteten Opposition zu bemänteln. 
	
	\subsection*{tg441.2.65} 
	\textbf{Source : }Die Metaphysik der Sitten/Erster Teil. Metaphysische Anfangsgründe der Rechtslehre/2. Teil. Das öffentliche Recht/1. Abschnitt. Das Staatsrecht\\  
	
	\noindent\textbf{Paragraphe : }Ohne alle Würde kann nun wohl kein Mensch im Staate sein, denn er hat wenigstens die des Staatsbürgers; außer, wenn er sich durch sein eigenes Verbrechen darum gebracht hat, da er dann zwar im Leben erhalten, aber zum bloßen Werkzeuge der Willkür eines anderen (entweder des Staats, oder eines anderen Staatsbürgers) gemacht wird. Wer nun das letztere ist (was er aber nur durch Urteil und Recht werden kann), ist ein Leibeigener (servus in sensu stricto) und gehört zum Eigentum (dominium) eines anderen, der daher nicht bloß sein Herr (herus), sondern auch sein Eigentümer (dominus) ist, der ihn als eine Sache veräußern und nach Belieben (nur nicht zu schandbaren Zwecken) brauchen, und über seine Kräfte, wenn gleich nicht über sein Leben und Gliedmaßen verfügen (disponieren) kann. Durch einen Vertrag kann sich niemand zu einer solchen Abhängigkeit verbinden, dadurch er aufhört, eine Person zu sein; denn nur als Person kann er einen Vertrag machen. Nun scheint es zwar, ein Mensch könne sich zu gewissen, der Qualität nach erlaubten, dem Grad nach aber unbestimmten Diensten gegen einen andern (für Lohn, Kost, oder Schutz) verpflichten, durch einen Verdingungsvertrag (locatio conductio), und er werde dadurch bloß Untertan (subiectus), nicht Leibeigener (servus); allein das ist nur ein falscher Schein. Denn, wenn sein Herr befugt ist, die Kräfte seines Untertans nach Belieben zu benutzen, so kann er sie auch (wie es mit den Negern auf den Zuckerinseln der \match{Fall} ist) erschöpfen, bis zum Tode oder der Verzweiflung, und jener hat sich seinem Herrn wirklich als Eigentum weggegeben; welches unmöglich ist. – Er kann sich also nur zu, der Qualität und dem Grade nach bestimmten, Arbeiten verdingen: entweder als Tagelöhner, oder ansässiger Untertan; im letzteren Fall, daß er teils, für den Gebrauch des Bodens seines Herrn, statt des Tagelohns. Dienste auf demselben Boden, teils für die eigene Benutzung  desselben bestimmte Abgaben (einen Zins) nach einem Pachtvertrage leistet, ohne sich dabei zum Gutsuntertan (glebae adscriptus) zu machen, als wodurch er seine Persönlichkeit einbüßen würde, mithin eine Zeit- oder Erbpacht gründen kann. Er mag nun aber auch durch sein Verbrechen ein persönlicher Untertan geworden sein, so kann diese Untertänigkeit ihm doch nicht anerben, weil er sie sich nur durch seine eigene Schuld zugezogen hat, und eben so wenig kann der von einem Leibeigenen Erzeugte, wegen der Erziehungskosten, die er gemacht hat, in Anspruch genommen werden, weil Erziehung eine absolute Naturpflicht der Eltern, und, im Falle daß diese Leibeigene waren, der Herren ist, welche mit dem Besitz ihrer Untertanen auch die Pflichten derselben übernommen haben. 
	
	\subsection*{tg445.2.41} 
	\textbf{Source : }Die Metaphysik der Sitten/Erster Teil. Metaphysische Anfangsgründe der Rechtslehre/Anhang erläutender Bemerkungen zu den metaphysischen Anhangsgründen der Rechtslehre\\  
	
	\noindent\textbf{Paragraphe : }Was das Recht der Beerbung anlangt, so hat den Herrn Rezensenten diesesmal sein Scharfblick, den Nerven des Beweises meiner Behauptung zu treffen, verlassen. – Ich sage ja nicht S. 135: »daß ein jeder Mensch notwendigerweise jede ihm angebotene Sache, durch deren Annehmung er nur gewinnen, nichts verlieren kann, annehme« (denn solche Sachen gibt es gar nicht), sondern daß ein jeder das Recht des Angebots in demselben Augenblick unvermeidlich und stillschweigend, dabei aber doch gültig, immer wirklich annehme: wenn es nämlich die Natur der Sache so mit sich bringt, daß der Widerruf schlechterdings unmöglich ist, nämlich im Augenblicke seines Todes; denn da kann der Promittent nicht widerrufen, und der Promissar ist, ohne irgend einen rechtlichen Akt begehen zu dürfen, in demselben Augenblick Akzeptant, nicht der versprochenen Erbschaft, sondern des Rechts, sie anzunehmen oder auszuschlagen. In diesem Augenblicke sieht er sich bei Eröffnung des Testaments, daß er, schon vor der Akzeptation der Erbschaft, vermögender geworden ist, als er war; denn er hat ausschließlich die Befugnis zu akzeptieren erworben, welche schon ein Vermö gensumstand ist. – Daß hiebei ein bürgerlicher Zustand vorausgesetzt wird, um etwas zu dem Seinen eines anderen zu machen, wenn man nicht mehr da ist, dieser Übergang des Besitztums, aus der  Totenhand, ändert in Ansehung der Möglichkeit der Erwerbung nach allgemeinen Prinzipien des Naturrechts nichts, wenn gleich der Anwendung derselben auf den vorkommenden \match{Fall} eine bürgerliche Verfassung zum Grunde gelegt werden muß. – Eine Sache nämlich, die ohne Bedingung anzunehmen oder auszuschlagen in meiner freien Wahl gestellt wird, heißt res iacens. Wenn der Eigentümer einer Sache mir etwas, z.B. ein Möbel des Hauses, aus dem ich auszuziehen eben im Begriff bin, umsonst anbietet (verspricht, es soll mein sein), so habe ich, so lange er nicht widerruft (welches, wenn er darüber stirbt, unmöglich ist), ausschließlich ein Recht zur Akzeptation des Angebotenen (ius in re iacente), d.i. ich allein kann es annehmen oder ausschlagen, wie es mir beliebt: und dieses Recht, ausschließlich zu wählen, erlange ich nicht vermittelst eines besonderen rechtlichen Akts meiner Deklaration, ich wolle, dieses Recht solle mir zustehen, sondern ohne denselben (lege). – Ich kann also zwar mich dahin erklären, ich wolle, die Sache solle mir nicht angehören (weil diese Annahme mir Verdrießlichkeiten mit anderen zuziehen dürfte), aber ich kann nicht wollen, ausschließlich die Wahl zu haben, ob sie mir angehören solle oder nicht; denn dieses Recht (des Annehmens oder Ausschlagens) habe ich ohne alle Deklaration meiner Annahme, unmittelbar durchs Angebot: denn wenn ich sogar die Wahl zu haben ausschlagen könnte, so würde ich wählen, nicht zu wählen; welches ein Widerspruch ist. Dieses Recht zu wählen geht nun im Augenblicke des Todes des Erb-Lassers auf mich über, durch dessen Vermächtnis (institutio heredis) ich zwar noch nichts von der Habe und Gut des Erb-Lassers, aber doch den bloß-rechtlichen (intelligibelen) Besitz dieser Habe oder eines Teils derselben erwerbe: deren Annahme ich mich nun zum Vorteil anderer begeben kann, mithin dieser Besitz keinen Augenblick unterbrochen ist, sondern die Sukzession als eine stetige Reihenfolge, vom Sterbenden zum eingesetzten Erben durch seine Akzeptation übergeht und so der Satz: testamenta sunt iuris naturae, wider alle Zweifel befestigt wird. 
	
	\subsection*{tg447.2.5} 
	\textbf{Source : }Die Metaphysik der Sitten/Zweiter Teil. Metaphysische Anfangsgründe der Tugendlehre/Vorrede\\  
	
	\noindent\textbf{Paragraphe : }In dieser Philosophie (der Tugendlehre) scheint es nun der Idee derselben gerade zuwider zu sein, bis zu metaphysischen Anfangsgründen zurückzugehen, um den Pflichtbegriff, von allem Empirischen (jedem Gefühl) gereinigt, doch zur Triebfeder zu machen. Denn was kann man sich für einen Begriff von einer Kraft und herkulischer Stärke  machen, um die lastergebärende Neigungen zu überwältigen, wenn die Tugend ihre Waffen aus der Rüstkammer der Metaphysik entlehnen soll? welche eine Sache der Spekulation ist, die nur wenig Menschen zu handhaben wissen. Daher fallen auch alle Tugendlehren, in Hörsälen, von Kanzeln und in Volksbüchern, wenn sie mit metaphysischen Brocken ausgeschmückt werden, ins Lächerliche. – Aber darum ist es doch nicht unnütz, vielweniger lächerlich, den ersten Gründen der Tugendlehre in einer Metaphysik nachzuspüren; denn irgend einer muß doch als Philosoph auf die ersten Gründe dieses Pflichtbegriffs hinausgehen: weil sonst weder Sicherheit noch Lauterkeit für die Tugendlehre überhaupt zu erwarten wäre. Sich desfalls auf ein gewisses Gefühl, welches man, seiner davon erwarteten Wirkung halber, moralisch nennt, zu verlassen, kann auch wohl dem Volkslehrer gnügen: indem dieser zum Probierstein einer Tugendpflicht, ob sie es sei oder nicht, die Aufgabe zu beherzigen verlangt: »wie, wenn nun ein jeder in jedem \match{Fall} deine Maxime zum allgemeinen Gesetz machte, würde eine solche wohl mit sich selbst zusammenstimmen können?« Aber, wenn es bloß Gefühl wäre, was auch diesen Satz zum Probierstein zu nehmen uns zur Pflicht machte, so wäre diese doch alsdann nicht durch die Vernunft diktiert, sondern nur instinktmäßig, mithin blindlings dafür angenommen. 
	
	\subsection*{tg460.2.14} 
	\textbf{Source : }Die Metaphysik der Sitten/Zweiter Teil. Metaphysische Anfangsgründe der Tugendlehre/Einleitung/XII. Ästhetische Vorbegriffe der Empfänglichkeit des Gemüts für Pflichtbegriffe überhaupt\\  
	
	\noindent\textbf{Paragraphe : }Eben so ist das Gewissen nicht etwas Erwerbliches und es gibt keine Pflicht, sich eines anzuschaffen; sondern jeder Mensch, als sittliches Wesen, hat ein solches ursprünglich in sich. Zum Gewissen verbunden zu sein, würde so viel sagen als: die Pflicht auf sich haben, Pflichten anzuerkennen. Denn Gewissen ist die dem Menschen in jedem \match{Fall} eines Gesetzes seine Pflicht zum Lossprechen oder Verurteilen vorhaltende praktische Vernunft. Seine Beziehung  also ist nicht die auf ein Objekt, sondern bloß aufs Subjekt (das moralische Gefühl durch ihren Akt zu affizieren); also eine unausbleibliche Tatsache, nicht eine Obliegenheit und Pflicht. Wenn man daher sagt: Dieser Mensch hat kein Gewissen, so meint man damit: er kehrt sich nicht an den Ausspruch desselben. Denn hätte er wirklich keines, so würde er sich auch nichts als pflichtmäßig zurechnen, oder als pflichtwidrig vorwerfen, mithin auch selbst die Pflicht, ein Gewissen zu haben, sich gar nicht denken können. 
	
	\subsection*{tg460.2.15} 
	\textbf{Source : }Die Metaphysik der Sitten/Zweiter Teil. Metaphysische Anfangsgründe der Tugendlehre/Einleitung/XII. Ästhetische Vorbegriffe der Empfänglichkeit des Gemüts für Pflichtbegriffe überhaupt\\  
	
	\noindent\textbf{Paragraphe : }Die mancherlei Einteilungen des Gewissens gehe ich noch hier vorbei und bemerke nur, was aus dem eben Angeführten folgt: daß nämlich ein irrendes Gewissen ein Unding sei. Denn in dem objektiven Urteile, ob etwas Pflicht sei oder nicht, kann man wohl bisweilen irren; aber im subjektiven, ob ich es mit meiner praktischen (hier richtenden) Vernunft zum Behuf jenes Urteils verglichen habe, kann ich nicht irren, weil ich alsdann praktisch gar nicht geurteilt haben würde: in welchem \match{Fall} weder Irrtum noch Wahrheit statt hat. Gewissenlosigkeit ist nicht Mangel des Gewissens, sondern Hang, sich an dessen Urteil nicht zu kehren. Wenn aber jemand sich bewußt ist, nach Gewissen gehandelt zu haben, so kann von ihm, was Schuld oder Unschuld betrifft, nichts mehr verlangt werden. Es liegt ihm nur ob, seinen Verstand über das, was Pflicht ist oder nicht, aufzuklären; wenn es aber zur Tat kommt oder gekommen ist, so spricht das Gewissen unwillkürlich und unvermeidlich. Nach Gewissen zu handeln kann also selbst nicht Pflicht sein, weil es sonst noch ein zweites Gewissen geben müßte, um sich des Akts des ersteren bewußt zu werden. 
	
	\subsection*{tg462.2.6} 
	\textbf{Source : }Die Metaphysik der Sitten/Zweiter Teil. Metaphysische Anfangsgründe der Tugendlehre/Einleitung/XIV. Vom Prinzip der Absonderung der Tugendlehre von der Rechtslehre\\  
	
	\noindent\textbf{Paragraphe : }Zur inneren Freiheit aber werden zwei Stücke erfordert: seiner selbst in einem gegebenen \match{Fall} Meister (animus sui compos) und über sich selbst Herr zu sein (imperium in semetipsum), d.i. seine Affekten zu zähmen und seine Leidenschaften zu beherrschen. – Die Gemütsart (indoles) in diesen beiden Zuständen ist edel (erecta), im entgegengesetzten Fall aber unedel (indoles abiecta, serva). 
	
	\subsection*{tg471.2.14} 
	\textbf{Source : }Die Metaphysik der Sitten/Zweiter Teil. Metaphysische Anfangsgründe der Tugendlehre/I. Ethische Elementarlehre/I. Teil. Von den Pflichten gegen sich selbst überhaupt/Erstes Buch. Von den vollkommenen Pflichten gegen sich selbst/Erstes Hauptstück. Die Pflicht des Menschen gegen sich selbst, als einem animalischen Wesen\\  
	
	\noindent\textbf{Paragraphe : }Sich eines integrierenden Teils als Organs berauben (verstümmeln), z.B. einen Zahn zu verschenken, oder zu verkaufen, um ihn in die Kinnlade eines andern zu pflanzen, oder die Kastration mit sich vornehmen zu lassen, um als Sänger bequemer leben zu können, u. dgl. gehört zum partialen Selbstmorde; aber nicht, ein abgestorbenes oder die Absterbung drohendes, und hiemit dem Leben nachteiliges Organ durch Amputation, oder, was zwar ein Teil, aber kein Organ des Körpers ist, z.B. die Haare, sich abnehmen zu lassen, kann zum Verbrechen an seiner eigenen Person nicht gerechnet werden; wiewohl der letztere \match{Fall} nicht ganz schuldfrei ist, wenn er zum äußeren Erwerb beabsichtigt wird. 
	
	\subsection*{tg471.2.23} 
	\textbf{Source : }Die Metaphysik der Sitten/Zweiter Teil. Metaphysische Anfangsgründe der Tugendlehre/I. Ethische Elementarlehre/I. Teil. Von den Pflichten gegen sich selbst überhaupt/Erstes Buch. Von den vollkommenen Pflichten gegen sich selbst/Erstes Hauptstück. Die Pflicht des Menschen gegen sich selbst, als einem animalischen Wesen\\  
	
	\noindent\textbf{Paragraphe : }Wer sich die Pocken einimpfen zu lassen beschließt, wagt sein Leben aufs Ungewisse: ob er es zwar tut, um sein Leben zu erhalten, und ist so fern in einem weit bedenklicheren \match{Fall} des Pflichtgesetzes, als der Seefahrer, welcher doch wenigstens den Sturm nicht macht, dem er sich anvertraut, statt dessen jener die Krankheit, die ihn in Todesgefahr bringt, sich selbst zuzieht. Ist also die Pockeninokulation erlaubt? 
	
	\subsection*{tg471.2.45} 
	\textbf{Source : }Die Metaphysik der Sitten/Zweiter Teil. Metaphysische Anfangsgründe der Tugendlehre/I. Ethische Elementarlehre/I. Teil. Von den Pflichten gegen sich selbst überhaupt/Erstes Buch. Von den vollkommenen Pflichten gegen sich selbst/Erstes Hauptstück. Die Pflicht des Menschen gegen sich selbst, als einem animalischen Wesen\\  
	
	\noindent\textbf{Paragraphe : }Die tierische Unmäßigkeit, im Genuß der Nahrung, ist der Mißbrauch der Genießmittel, wodurch das Vermögen des intellektuellen Gebrauchs derselben gehemmt oder erschöpft wird. Versoffenheit und Gefräßigkeit sind die Laster, die unter diese Rubrik gehören. Im Zustande der Betrunkenheit ist der Mensch nur wie ein Tier, nicht als Mensch, zu behandeln; durch die Überladung mit Speisen und in einem solchen Zustande ist er für Handlungen, wozu Gewandtheit und Überlegung im Gebrauch seiner Kräfte erfordert wird, auf eine gewisse Zeit gelähmt. – Daß sich in einen solchen Zustand zu versetzen Verletzung einer Pflicht wider sich selbst sei, fällt von selbst in die Augen. Die erste dieser Erniedrigungen, selbst unter die tierische Natur, wird gewöhnlich durch gegorene Getränke, aber auch durch andere betäubende Mittel, als den Mohnsaft und andere Produkte des Gewächsreichs, bewirkt, und wird dadurch verführerisch, daß dadurch auf eine Weile geträumte Glückseligkeit und Sorgenfreiheit, ja wohl auch eingebildete Stärke hervorgebracht, Niedergeschlagenheit aber und Schwäche, und, was das Schlimmste ist, Notwendigkeit, dieses Betäubungsmittel zu wiederholen, ja wohl gar damit zu steigern, eingeführt wird. Die Gefräßigkeit ist sofern noch unter jener tierischen Sinnenbelustigung, daß sie bloß den Sinn  als passive Beschaffenheit und nicht einmal die Einbildungskraft, welche doch noch ein tätiges Spiel der Vorstellungen, wie im vorerwähnten Genuß der \match{Fall} ist, beschäftigt; mithin sich dem des Viehes noch mehr nähert. 
	
	\subsection*{tg472.2.10} 
	\textbf{Source : }Die Metaphysik der Sitten/Zweiter Teil. Metaphysische Anfangsgründe der Tugendlehre/I. Ethische Elementarlehre/I. Teil. Von den Pflichten gegen sich selbst überhaupt/Erstes Buch. Von den vollkommenen Pflichten gegen sich selbst\\  
	
	\noindent\textbf{Paragraphe : }Der Mensch, als moralisches Wesen (homo noumenon), kann sich selbst, als physisches Wesen (homo phaenomenon), nicht als bloßes Mittel (Sprachmaschine) brauchen, das an den inneren Zweck (der Gedankenmitteilung) nicht gebunden wäre, sondern ist an die Bedingung der Übereinstimmung mit der Erklärung (declaratio) des ersteren gebunden,  und gegen sich selbst zur Wahrhaftigkeit verpflichtet. – Wenn er z.B. den Glauben an einen künftigen Weltrichter lügt, indem er wirklich keinen solchen in sich findet, aber, indem er sich überredet, es könne doch nicht schaden, wohl aber nutzen, einen solchen in Gedanken einem Herzenskündiger zu bekennen, um auf allen \match{Fall} seine Gunst zu erheucheln. Oder, wenn er zwar desfalls nicht im Zweifel ist, aber sich doch mit innerer Verehrung seines Gesetzes schmeichelt, da er doch keine andere Triebfeder, als die der Furcht vor Strafe, bei sich fühlt. 
	
	\subsection*{tg472.2.29} 
	\textbf{Source : }Die Metaphysik der Sitten/Zweiter Teil. Metaphysische Anfangsgründe der Tugendlehre/I. Ethische Elementarlehre/I. Teil. Von den Pflichten gegen sich selbst überhaupt/Erstes Buch. Von den vollkommenen Pflichten gegen sich selbst\\  
	
	\noindent\textbf{Paragraphe : }Da hier nur von Pflichten gegen sich selbst die Rede ist und Habsucht (Unersättlichkeit im Erwerb), um zu verschwenden, eben so wohl als Knauserei (Peinlichkeit im Vertun), Selbstsucht (solipsismus) zum Grunde haben, und beide, die Verschwendung so wohl als die Kargheit, bloß darum verwerflich zu sein scheinen, weil sie auf Armut hinaus laufen, bei dem einen auf nicht erwartete, bei dem anderen auf willkürliche (armselig leben zu wollen), – so ist die Frage: ob sie, die eine so wohl als die andere, überhaupt Laster und nicht vielmehr beide bloße Unklugheit genannt werden sollen, mithin nicht ganz und gar außerhalb den Grenzen der Pflicht gegen sich selbst liegen mögen. Die Kargheit aber ist nicht bloß mißverstandene Sparsamkeit, sondern sklavische Unterwerfung seiner selbst unter die Glücksgüter, ihrer nicht Herr zu sein, welches Verletzung der Pflicht gegen sich selbst ist. Sie ist der Liberalität (liberalitas moralis) der Denkungsart überhaupt (nicht der Freigebigkeit (liberalitas sumtuosa), welche nur eine Anwendung derselben auf einen besonderen \match{Fall} ist), d.i. dem Prinzip der Unabhängigkeit von allem anderen, außer von dem Gesetz, entgegengesetzt, und Defraudation, die das Subjekt an sich selbst begeht. Aber was ist das für ein Gesetz, dessen innerer Gesetzgeber selbst nicht weiß, wo es anzuwenden ist? Soll ich meinem Munde abbrechen, oder nur dem äußeren Aufwande? im Alter oder schon in der Jugend? oder ist Sparsamkeit überhaupt eine Tugend? 
	
	\subsection*{tg481.2.77} 
	\textbf{Source : }Die Metaphysik der Sitten/Zweiter Teil. Metaphysische Anfangsgründe der Tugendlehre/I. Ethische Elementarlehre/II. Teil. Von den Tugendpflichten gegen andere/Erstes Hauptstück. Von den Pflichten gegen andere, bloß als Menschen/Erster Abschnitt. Von der Liebespflicht gegen andere Menschen\\  
	
	\noindent\textbf{Paragraphe : }Würde es mit dem Wohl der Welt überhaupt nicht besser stehen, wenn alle Moralität der Menschen nur auf Rechtspflichten, doch mit der größten Gewissenhaftigkeit, eingeschränkt, das Wohlwollen aber unter die Adiaphora gezählt würde? Es ist nicht so leicht zu übersehen, welche Folge es auf die Glückseligkeit der Menschen haben dürfte. Aber in diesem \match{Fall} würde es doch wenigstens an einer großen moralischen Zierde der Welt, nämlich der Menschenliebe fehlen, welche also für sich, auch ohne die Vorteile (der Glückseligkeit) zu berechnen, die Welt als ein schönes moralisches Ganze in ihrer ganzen Vollkommenheit darzustellen erfordert wird. 
	
	\subsection*{tg486.2.30} 
	\textbf{Source : }Die Metaphysik der Sitten/Zweiter Teil. Metaphysische Anfangsgründe der Tugendlehre/II. Ethische Methodenlehre/1. Abschnitt. Die ethische Didaktik\\  
	
	\noindent\textbf{Paragraphe : }6. L. Um nun zu wissen, wie du es anfängst, um der Glückseligkeit teilhaftig und doch auch nicht unwürdig  zu werden, dazu liegt die Regel und Anweisung ganz allein in deiner Vernunft; das heißt so viel als: du hast nicht nötig, diese Regel deines Verhaltens von der Erfahrung, oder von anderen durch ihre Unterweisung abzulernen; deine eigene Vernunft lehrt und gebietet dir geradezu, was du zu tun hast. Z.B. wenn dir ein \match{Fall} vorkömmt, da du durch eine fein ausgedachte Lüge dir, oder deinen Freunden, einen großen Vorteil verschaffen kannst, ja noch dazu dadurch auch keinem anderen schadest, was sagt dazu deine Vernunft? S. Ich soll nicht lügen; der Vorteil für mich und meinen Freund mag so groß sein, wie er immer wolle. Lügen ist niederträchtig und macht den Menschen unwürdig glücklich zu sein. – Hier ist eine unbedingte Nötigung durch ein Vernunftgebot (oder Verbot), dem ich gehorchen muß: wogegen alle meine Neigungen verstummen müssen. L. Wie nennt man diese unmittelbar durch die Vernunft dem Menschen auferlegte Notwendigkeit, einem Gesetze derselben gemäß zu handeln? S. Sie heißt Pflicht. L. Also ist dem Menschen die Beobachtung seiner Pflicht die allgemeine und einzige Bedingung der Würdigkeit glücklich zu sein, und diese ist mit jener ein und dasselbe. 
	
	\subsection*{tg489.2.18} 
	\textbf{Source : }Die Metaphysik der Sitten/Fußnoten\\  
	
	\noindent\textbf{Paragraphe : }Eine jede Übertretung des Gesetzes kann und muß nicht anders, als so erklärt werden, daß sie aus einer Maxime des Verbrechers (sich eine solche Untat zur Regel zu machen) entspringe; denn, wenn man sie von einem sinnlichen Antrieb ableitete, so wäre sie nicht von ihm, als einem freien Wesen, begangen, und könnte ihm nicht zugerechnet werden; wie es aber dem Subjekt möglich ist, eine solche Maxime wider das klare Verbot der gesetzgebenden Vernunft zu fassen, läßt sich schlechterdings nicht erklären; denn nur die Begebenheiten nach dem Mechanism der Natur sind erklärungsfähig. Nun kann der Verbrecher seine Untat entweder nach der Maxime einer angenommenen objektiven Regel (als allgemein geltend), oder nur als Ausnahme von der Regel (sich davon gelegentlich zu dispensieren) begehen; im letzteren \match{Fall} weicht er nur (ob zwar vorsätzlich) vom Gesetz ab; er kann seine eigene Übertretung zugleich verabscheuen, und, ohne dem Gesetz förmlich den Gehorsam aufzukündigen, es nur umgehen wollen; im ersteren aber verwirft er die Autorität des Gesetzes selbst, dessen Gültigkeit er sich doch vor seiner Vernunft nicht ableugnen kann, und macht es sich zur Regel, wider dasselbe zu handeln; seine Maxime ist also nicht bloß ermangelungsweise (negative) sondern sogar abbruchsweise (contrarie) oder, wie man sich ausdrückt, diametraliter, als Widerspruch (gleichsam feindselig) dem Gesetz entgegen. So viel wir einsehen, ist ein dergleichen Verbrechen einer förmlichen (ganz nutzlosen) Bosheit zu begehen Menschen unmöglich, und doch (ob zwar bloße Idee des Äußerst-bösen) in einem System der Moral nicht zu übergehen. 
	
	\subsection*{tg489.2.27} 
	\textbf{Source : }Die Metaphysik der Sitten/Fußnoten\\  
	
	\noindent\textbf{Paragraphe : }
	
	12 In jeder Bestrafung liegt etwas das Ehrgefühl des Angeklagten (mit Recht) Kränkendes; weil sie einen bloßen einseitigen Zwang enthält und so an ihm die Würde eines Staatsbürgers, als eines solchen, in einem besonderen \match{Fall} wenigstens suspendiert ist: Da er einer äußeren Pflicht unterworfen wird, der er seiner seits keinen Widerstand entgegen setzen darf. Der Vornehme und Reiche, der auf den Beutel geklopft wird, fühlt mehr seine Erniedrigung, sich unter den Willen des geringeren Mannes beugen zu müssen, als den Geldverlust. Die Strafgerechtigkeit (iustitia punitiva), da nämlich das Argument der Strafbarkeit moralisch ist (quia peccatum est), muß hier von der Strafklugheit, da es bloß pragmatisch ist (ne peccetur) und sich auf Erfahrung von dem gründet, was am stärksten wirkt, Verbrechen abzuhalten, unterschieden werden, und hat in der Topik der Rechtsbegriffe einen ganz anderen Ort, locus iusti, nicht des Conducibilis, oder des Zuträglichen in gewisser Absicht noch auch den des bloßen Honesti, dessen Ort in der Ethik aufgesucht werden muß. 
	
	\subsection*{tg489.2.48} 
	\textbf{Source : }Die Metaphysik der Sitten/Fußnoten\\  
	
	\noindent\textbf{Paragraphe : }
	
	22
	Beispiel, ein deutsches Wort, was man gemeiniglich für Exempel als ihm gleichgeltend braucht, ist mit diesem nicht von einerlei Bedeutung. Woran ein Exempel nehmen und zur Verständlichkeit eines Ausdrucks ein Beispiel anführen, sind ganz verschiedene Begriffe, Das Exempel ist ein besonderer \match{Fall} von einer praktischen Regel, sofern diese die Tunlichkeit oder Untunlichkeit einer Handlung vorstellt. Hingegen ein Beispiel ist nur das Besondere (concretum), als unter dem Allgemeinen nach Begriffen (abstractum) enthalten vorgestellt, und bloß theoretische Darstellung eines Begriffs. 
	
	\subsection*{tg489.2.52} 
	\textbf{Source : }Die Metaphysik der Sitten/Fußnoten\\  
	
	\noindent\textbf{Paragraphe : }
	
	24 Zwar hat später hin ein großer moralisch gesetzgebender Weise das Schwören als ungereimt, und zugleich beinahe an Blasphemie grenzend, ganz und gar verboten; allein in politischer Rücksicht glaubt man noch immer dieses mechanischen, zur Verwaltung der öffentlichen Gerechtigkeit dienlichen, Mittels schlechterdings nicht entbehren zu können und hat milde Auslegungen ausgedacht, um jenem Verbot auszuweichen. – Da es eine Ungereimtheit wäre, im Ernst zu schwören, daß ein Gott sei (weil man diesen schon postuliert haben muß, um überhaupt nur schwören zu können), so bleibt noch die Frage: ob nicht ein Eid möglich und geltend sei, da man nur auf den \match{Fall} daß ein Gott sei (ohne, wie Protagoras, darüber etwas auszumachen) schwöre. – In der Tat mögen wohl alle redlich und zugleich mit Besonnenheit abgelegten Eide in keinem anderen Sinne getan worden sein. – Denn, daß einer sich erböte, schlechthin zu beschwören, daß ein Gott sei, scheint zwar kein bedenkliches Anerbieten zu sein, er mag ihn glauben oder nicht. Ist einer (wird der Betrüger sagen), so habe ich's getroffen; ist keiner, so zieht mich auch keiner zur Verantwortung, und ich bringe mich durch solchen Eid in keine Gefahr. – Ist denn aber keine Gefahr dabei, wenn ein solcher ist, auf einer vorsätzlichen und, selbst um Gott zu täuschen, angelegten Lüge betroffen zu werden? 
	
	\unnumberedsection{Familie (9)} 
	\subsection*{tg437.2.38} 
	\textbf{Source : }Die Metaphysik der Sitten/Erster Teil. Metaphysische Anfangsgründe der Rechtslehre/1. Teil. Das Privatrecht vom äußeren Mein und Dein überhaupt/2. Hauptstück. Von der Art, etwas Äußeres zu erwerben/3. Abschnitt. Von dem auf dingliche Art persönlichen Recht\\  
	
	\noindent\textbf{Paragraphe : }Die Kinder des Hauses, die mit den Eltern zusammen eine \match{Familie} ausmachten, werden, auch ohne allen Vertrag  der Aufkündigung ihrer bisherigen Abhängigkeit, durch die bloße Gelangung zu dem Vermögen ihrer Selbsterhaltung (so wie es, teils als natürliche Volljährigkeit) dem allgemeinen Laufe der Natur überhaupt, teils ihrer besonderen Naturbeschaffenheit gemäß, eintritt, mündig (maiorennes). d.i. ihre eigene Herren (sui iuris), und erwerben dieses Recht ohne besonderen rechtlichen Akt, mithin bloß durchs Gesetz (lege) – sind den Eltern für ihre Erziehung nichts schuldig, so wie gegenseitig die letzteren ihrer Verbindlichkeit gegen diese auf ebendieselbe Art loswerden, hiemit beide ihre natürliche Freiheit gewinnen oder wieder gewinnen – die häusliche Gesellschaft aber, welche nach dem Gesetz notwendig war, nunmehr aufgelöset wird. 
	
	\subsection*{tg437.2.39} 
	\textbf{Source : }Die Metaphysik der Sitten/Erster Teil. Metaphysische Anfangsgründe der Rechtslehre/1. Teil. Das Privatrecht vom äußeren Mein und Dein überhaupt/2. Hauptstück. Von der Art, etwas Äußeres zu erwerben/3. Abschnitt. Von dem auf dingliche Art persönlichen Recht\\  
	
	\noindent\textbf{Paragraphe : }Beide Teile können nun wirklich ebendasselbe Hauswesen, aber in einer anderen Form der Verpflichtung, nämlich als Verknüpfung des Hausherren mit dem Gesinde (den Dienern oder Dienerinnen des Hauses), mithin eben diese häusliche Gesellschaft, aber jetzt als hausherrliche (societas herilis) erhalten, durch einen Vertrag, den der erstere mit den mündig gewordenen Kindern, oder, wenn die \match{Familie} keine Kinder hat, mit anderen freien Personen (der Hausgenossenschaft) eine häusliche Gesellschaft stiften, welche eine ungleiche Gesellschaft (des Gebietenden, oder der Herrschaft und der Gehorchenden, d.i. der Dienerschaft (imperantis et subiecti domestici)) sein würde. 
	
	\subsection*{tg437.2.7} 
	\textbf{Source : }Die Metaphysik der Sitten/Erster Teil. Metaphysische Anfangsgründe der Rechtslehre/1. Teil. Das Privatrecht vom äußeren Mein und Dein überhaupt/2. Hauptstück. Von der Art, etwas Äußeres zu erwerben/3. Abschnitt. Von dem auf dingliche Art persönlichen Recht\\  
	
	\noindent\textbf{Paragraphe : }Die Erwerbung nach diesem Gesetz ist dem Gegenstande nach dreierlei: Der Mann erwirbt ein Weib, das Paar erwirbt Kinder und die \match{Familie} Gesinde. – Alles dieses Erwerbliche ist zugleich unveräußerlich und das Recht des Besitzers dieser Gegenstände das allerpersönlichste. 
	
	\subsection*{tg442.2.4} 
	\textbf{Source : }Die Metaphysik der Sitten/Erster Teil. Metaphysische Anfangsgründe der Rechtslehre/2. Teil. Das öffentliche Recht/2. Abschnitt. Das Völkerrecht\\  
	
	\noindent\textbf{Paragraphe : }Die Menschen, welche ein Volk ausmachen, können, als Landeseingeborne, nach der Analogie der Erzeugung von einem gemeinschaftlichen Elternstamm (congeniti) vorgestellt werden, ob sie es gleich nicht sind: dennoch aber, in intellektueller und rechtlicher Bedeutung, als von einer gemeinschaftlichen Mutter (der Republik) geboren, gleichsam eine \match{Familie} (gens, natio) ausmachen, deren Glieder (Staatsbürger) alle ebenbürtig sind, und mit denen, die neben ihnen im Naturzustande leben möchten, als unedlen keine Vermischung eingehen, obgleich diese (die Wilden) ihrerseits sich wiederum wegen der gesetzlosen Freiheit, die sie gewählt haben, sich vornehmer dünken, die gleichfalls Völkerschaften, aber nicht Staaten, ausmachen. Das Recht der Staaten in Verhältnis zu einander (welches nicht ganz richtig im Deutschen das Völkerrecht genannt wird, sondern vielmehr das Staatenrecht (ius publicum civitatum) heißen sollte) ist nun dasjenige, was wir unter dem Namen des Völkerrechts zu betrachten haben: wo ein Staat, als eine moralische Person, gegen einen anderen im Zustande der natürlichen Freiheit, folglich auch dem des beständigen Krieges betrachtet, teils das Recht zum Kriege, teils das im Kriege, teils das, einander zu nötigen, aus diesem Kriegszustande herauszugehen, mithin eine den beharrlichen Frieden gründende Verfassung, d.i. das Recht nach dem Kriege zur Aufgabe macht, und führt nur das Unterscheidende von dem des Naturzustandes einzelner Menschen oder Familien (im Verhältnis gegen einander) von dem der Völker bei sich, daß im Völkerrecht nicht bloß ein Verhältnis eines Staats gegen den anderen im ganzen, sondern auch einzelner Personen des einen gegen einzelne des anderen, imgleichen gegen den ganzen anderen Staat selbst in Betrachtung kommt; welcher Unterschied aber vom Recht einzelner im  bloßen Naturzustande nur solcher Bestimmungen bedarf, die sich aus dem Begriffe des letzteren leicht folgern lassen. 
	
	\subsection*{tg445.2.24} 
	\textbf{Source : }Die Metaphysik der Sitten/Erster Teil. Metaphysische Anfangsgründe der Rechtslehre/Anhang erläutender Bemerkungen zu den metaphysischen Anhangsgründen der Rechtslehre\\  
	
	\noindent\textbf{Paragraphe : }Endlich, wenn bei eintretender Volljährigkeit die Pflicht der Eltern zur Erhaltung ihrer Kinder aufhört, so haben jene noch das Recht, diese als ihren Befehlen unterworfene Hausgenossen zu Erhaltung des Hauswesens zu brauchen, bis zur Entlassung derselben; welches eine Pflicht der Eltern gegen diese ist, die aus der natürlichen Beschränkung des Rechts der ersteren folgt. Bis dahin sind sie zwar Hausgenossen und gehören zur Familie, aber von nun an gehören sie zur Dienerschaft (famulatus) in derselben, die folglich nicht anders als durch Vertrag zu dem Seinen des Hausherrn (als seine Domestiken) hinzu kommen können. – Eben so kann auch eine Dienerschaft 
	außer der \match{Familie} zu dem Seinen des Hausherren nach einem auf dingliche Art persönlichen Rechte gemacht und als Gesinde (famulatus domesticus) durch Vertrag erworben werden. Ein solcher Vertrag ist nicht der einer bloßen Verdingung (locatio conductio operae) sondern der Hingebung seiner Person in den Besitz des Hausherrn, Vermietung (locatio conductio personae), welche darin von jener Verdingung unterschieden ist, daß das Gesinde sich zu allem Erlaubten versteht, was das Wohl des Hauswesens betrifft und ihm nicht, als bestellte und spezifisch bestimmte Arbeit, aufgetragen wird: Anstatt daß der zur bestimmten Arbeit Gedungene (Handwerker oder Tagelöhner) sich nicht zu dem Seinen des anderen hingibt und so auch kein Hausgenosse ist. – Des letzteren, weil er nicht im rechtlichen Besitz des anderen ist, der ihn zu gewissen Leistungen verpflichtet, kann der Hausherr, wenn jener auch sein häuslicher Einwohner (inquilinus) wäre, sich nicht (via facti) als einer Sache bemächtigen, sondern muß nach dem persönlichen Recht, auf die Leistung des Versprochenen dringen, welche ihm durch Rechtsmittel (via iuris) zu Gebote stehen. – – So viel zur Erläuterung und Verteidigung eines befremdlichen, neu hinzukommenden, Rechtstitels in der natürlichen Gesetzlehre, der doch, stillschweigend immer in Gebrauch gewesen ist. 
	
	\subsection*{tg445.2.46} 
	\textbf{Source : }Die Metaphysik der Sitten/Erster Teil. Metaphysische Anfangsgründe der Rechtslehre/Anhang erläutender Bemerkungen zu den metaphysischen Anhangsgründen der Rechtslehre\\  
	
	\noindent\textbf{Paragraphe : }
	Stiftung (sanctio testamentaria beneficii perpetui) ist die freiwillige, durch den Staat bestätigte, für gewisse auf einander folgende Glieder desselben, bis zu ihrem gänzlichen Aussterben, errichtete wohltätige Anstalt. – Sie heißt ewig, wenn die Verordnung zu Erhaltung derselben mit der Konstitution des Staats selbst vereinigt ist (denn der Staat muß für ewig angesehen werden); ihre Wohltätigkeit aber ist entweder für das Volk überhaupt oder für einen nach gewissen besonderen Grundsätzen vereinigten Teil desselben, einen Stand oder für eine \match{Familie} und die ewige Fortdauer ihrer Deszendenten abgezweckt. Ein Beispiel vom ersteren sind die Hospitäler, vom zweiten die Kirchen, vom dritten die Orden (geistliche und weltliche), vom vierten die Majorate. 
	
	\subsection*{tg445.2.65} 
	\textbf{Source : }Die Metaphysik der Sitten/Erster Teil. Metaphysische Anfangsgründe der Rechtslehre/Anhang erläutender Bemerkungen zu den metaphysischen Anhangsgründen der Rechtslehre\\  
	
	\noindent\textbf{Paragraphe : }Was endlich die Majoratsstiftung betrifft, da ein Gutsbesitzer durch Erbeseinsetzung verordnet: daß in der Reihe der auf einander folgenden Erben immer der Nächste von der \match{Familie} der Gutsherr sein solle (nach der Analogie mit einer monarchisch-erblichen Verfassung eines Staats, wo der Landesherr es ist), so kann eine solche Stiftung nicht allein mit Beistimmung aller Agnaten jederzeit aufgehoben werden und darf nicht auf ewige Zeiten – gleich als ob das Erbrecht am Boden haftete – immerwährend fortdauern, noch gesagt werden, es sei eine Verletzung der Stiftung und des Willens des Urahnherrn derselben, des Stifters, sie eingehen zu lassen: sondern der Staat hat auch hier ein Recht, ja sogar die Pflicht, bei den allmählich eintretenden Ursachen seiner eigenen Reform ein solches föderatives System seiner Untertanen, gleich als Unterkönige (nach der Analogie von Dynasten und Satrapen), wenn es erloschen ist, nicht weiter aufkommen zu lassen. 
	
	\subsection*{tg481.2.83} 
	\textbf{Source : }Die Metaphysik der Sitten/Zweiter Teil. Metaphysische Anfangsgründe der Tugendlehre/I. Ethische Elementarlehre/II. Teil. Von den Tugendpflichten gegen andere/Erstes Hauptstück. Von den Pflichten gegen andere, bloß als Menschen/Erster Abschnitt. Von der Liebespflicht gegen andere Menschen\\  
	
	\noindent\textbf{Paragraphe : }Sie machen die abscheuliche \match{Familie} des Neides, der Undankbarkeit und der Schadenfreude aus. – Der Haß ist aber hier nicht offen und gewalttätig, sondern geheim und verschleiert, welches zu der Pflichtvergessenheit gegen seinen Nächsten noch Niederträchtigkeit hinzutut, und so zugleich die Pflicht gegen sich selbst verletzt. 
	
	\subsection*{tg481.2.84} 
	\textbf{Source : }Die Metaphysik der Sitten/Zweiter Teil. Metaphysische Anfangsgründe der Tugendlehre/I. Ethische Elementarlehre/II. Teil. Von den Tugendpflichten gegen andere/Erstes Hauptstück. Von den Pflichten gegen andere, bloß als Menschen/Erster Abschnitt. Von der Liebespflicht gegen andere Menschen\\  
	
	\noindent\textbf{Paragraphe : }a) Der Neid (livor), als Hang, das Wohl anderer mit Schmerz, wahrzunehmen, ob zwar dem seinigen dadurch kein Abbruch geschieht, der, wenn er zur Tat (jenes Wohl zu schmälern) ausschlägt, qualifizierter Neid, sonst aber nur Mißgunst (invidentia) heißt, ist doch nur eine indirekt-bösartige Gesinnung, nämlich ein Unwille, unser eigen Wohl durch das Wohl anderer in Schatten gestellt zu sehen, weil wir den Maßstab desselben nicht in dessen innerem Wert, sondern nur in der Vergleichung mit dem Wohl anderer, zu schätzen, und diese Schätzung zu versinnlichen wissen. – Daher spricht man auch wohl von einer beneidungswürdigen Eintracht und Glückseligkeit in einer Ehe, oder \match{Familie} u.s.w.; gleich als ob es in manchen Fällen erlaubt wäre, jemanden zu beneiden. Die Regungen des Neides liegen also in der Natur des Menschen, und nur der Ausbruch derselben macht sie zu dem scheußlichen Laster einer grämischen, sich selbst folternden und auf Zerstörung des Glücks anderer, wenigstens dem Wunsche nach, gerichteten Leidenschaft, ist mithin der Pflicht des Menschen gegen sich selbst so wohl, als gegen andere entgegengesetzt. 
	
	\unnumberedsection{Feder (1)} 
	\subsection*{tg437.2.85} 
	\textbf{Source : }Die Metaphysik der Sitten/Erster Teil. Metaphysische Anfangsgründe der Rechtslehre/1. Teil. Das Privatrecht vom äußeren Mein und Dein überhaupt/2. Hauptstück. Von der Art, etwas Äußeres zu erwerben/3. Abschnitt. Von dem auf dingliche Art persönlichen Recht\\  
	
	\noindent\textbf{Paragraphe : }Ein Buch ist eine Schrift (ob mit der \match{Feder} oder durch Typen, auf wenig oder viel Blättern verzeichnet, ist hier gleichgültig), welche eine Rede vorstellt, die jemand durch sichtbare Sprachzeichen an das Publikum hält. – Der, welcher zu diesem in seinem eigenen Namen spricht, heißt der Schriftsteller (autor). Der, welcher durch eine Schrift im Namen eines anderen (des Autors) öffentlich redet, ist der Verleger. Dieser, wenn er es mit jenes seiner Erlaubnis tut, ist der rechtmäßige, tut er es aber ohne dieselbe, der unrechtmäßige Verleger, d.i. der Nachdrucker. Die Summe aller Kopeien der Urschrift (Exemplare) ist der Verlag. 
	
	\unnumberedsection{Fruchtbarkeit (1)} 
	\subsection*{tg475.2.13} 
	\textbf{Source : }Die Metaphysik der Sitten/Zweiter Teil. Metaphysische Anfangsgründe der Tugendlehre/I. Ethische Elementarlehre/I. Teil. Von den Pflichten gegen sich selbst überhaupt/Erstes Buch. Von den vollkommenen Pflichten gegen sich selbst/Zweites Hauptstück. Die Pflicht des Menschen gegen sich selbst, bloß als einem moralischen Wesen/Episodischer Abschnitt. Von der Amphibolie der moralischen Reflexionsbegriffe\\  
	
	\noindent\textbf{Paragraphe : }
	In Ansehung dessen, was ganz über unsere Erfahrungsgrenze hinaus liegt, aber doch seiner Möglichkeit nach in unseren Ideen angetroffen wird, z.B. der Idee von Gott, haben wir eben so wohl auch eine Pflicht, welche Religionspflicht genannt wird, die nämlich »der Erkenntnis aller unserer Pflichten als (instar) göttlicher Gebote«. Aber dieses ist nicht das Bewußtsein einer Pflicht gegen Gott. Denn, da diese Idee ganz aus unserer eigenen Vernunft hervorgeht, und von uns, es sei in theoretischer Absicht, um sich die Zweckmäßigkeit im Weltganzen zu erklären, oder auch, um zur Triebfeder in unserem Verhalten zu dienen, von uns selbst gemacht wird, so haben wir hiebei nicht ein gegebenes Wesen vor uns, gegen welches uns Verpflichtung obläge: denn da müßte dessen Wirklichkeit allererst durch Erfahrung bewiesen (geoffenbart) sein; sondern es ist Pflicht des Menschen gegen sich selbst, diese unumgänglich der Vernunft sich darbietende Idee auf das moralische Gesetz in uns, wo es von der größten sittlichen \match{Fruchtbarkeit} ist, anzuwenden. In diesem (praktischen) Sinn kann es also so lauten: Religion zu haben ist Pflicht des Menschen gegen sich selbst. 
	
	\unnumberedsection{Gattung (4)} 
	\subsection*{tg481.2.25} 
	\textbf{Source : }Die Metaphysik der Sitten/Zweiter Teil. Metaphysische Anfangsgründe der Tugendlehre/I. Ethische Elementarlehre/II. Teil. Von den Tugendpflichten gegen andere/Erstes Hauptstück. Von den Pflichten gegen andere, bloß als Menschen/Erster Abschnitt. Von der Liebespflicht gegen andere Menschen\\  
	
	\noindent\textbf{Paragraphe : }Die Maxime des Wohlwollens (die praktische Menschenliebe) ist aller Menschen Pflicht gegen einander; man mag diese nun liebenswürdig finden oder nicht, nach dem ethischen Gesetz der Vollkommenheit: Liebe deinen Nebenmenschen als dich selbst. – Denn alles moralisch-praktische Verhältnis gegen Menschen ist ein Verhältnis derselben in der Vorstellung der reinen Vernunft, d.i. der freien Handlungen nach Maximen, welche sich zur allgemeinen Gesetzgebung qualifizieren, die also nicht selbstsüchtig (ex solipsismo prodeuntes) sein können. Ich will jedes anderen Wohlwollen (benevolentiam) gegen mich; ich soll also auch gegen jeden anderen wohlwollend sein. Da aber alle andere außer mir nicht alle sein, mithin die Maxime nicht die Allgemeinheit eines Gesetzes an sich haben würde, welche doch zur Verpflichtung notwendig ist: so wird das Pflichtgesetz des Wohlwollens mich als Objekt desselben im Gebot der praktischen Vernunft mit begreifen: nicht, als ob ich dadurch verbunden würde, mich selbst zu lieben (denn das geschieht ohne das unvermeidlich, und dazu gibt's also keine Verpflichtung), sondern die gesetzgebende Vernunft, welche in ihrer Idee der Menschheit überhaupt die ganze \match{Gattung} (mich also mit) einschließt, nicht der Mensch, schließt als allgemeingesetzgebend mich in der Pflicht des wechselseitigen Wohlwollens nach dem Prinzip der Gleichheit alle andere neben mir mit ein, und erlaubt es dir, dir selbst wohlzuwollen, unter der Bedingung, daß du auch jedem anderen wohl willst; weil so allein deine Maxime (des Wohltuns) sich zu einer allgemeinen Gesetzgebung qualifiziert, als worauf alles Pflichtgesetz gegründet ist. 
	
	\subsection*{tg481.2.91} 
	\textbf{Source : }Die Metaphysik der Sitten/Zweiter Teil. Metaphysische Anfangsgründe der Tugendlehre/I. Ethische Elementarlehre/II. Teil. Von den Tugendpflichten gegen andere/Erstes Hauptstück. Von den Pflichten gegen andere, bloß als Menschen/Erster Abschnitt. Von der Liebespflicht gegen andere Menschen\\  
	
	\noindent\textbf{Paragraphe : }Alle Laster, welche selbst die menschliche Natur hassenswert machen würden, wenn man sie (als qualifiziert) in der Bedeutung von Grundsätzen nehmen wollte, sind inhuman, objektiv betrachtet, aber doch menschlich, subjektiv erwogen: d.i. wie die Erfahrung uns unsere \match{Gattung} kennen lehrt. Ob man also zwar einige derselben in der Heftigkeit des Abscheues teuflisch nennen möchte, so wie ihr Gegenstück Engelstugend genannt werden könnte: so sind beide Begriffe doch nur Ideen von einem Maximum, als Maßstab zum Behuf der Vergleichung des Grades der Moralität gedacht, indem man dem Menschen seinen Platz im Himmel oder der Hölle anweiset, ohne aus ihm ein Mittelwesen, was weder den einen dieser Plätze, noch den anderen einnimmt, zu machen. Ob es Haller, mit seinem »zweideutig Mittelding von Engeln und von Vieh«, besser getroffen habe, mag hier unausgemacht bleiben. Aber das Halbieren. in einer Zusammenstellung heterogener Dinge führt auf gar keinen bestimmten Begriff, und zu diesem kann uns in der Ordnung der Wesen nach ihrem uns unbekannten Klassenunterschiede nichts hinleiten. Die erstere Gegeneinanderstellung (von Engelstugend und teuflischem  Laster) ist Übertreibung. Die zweite, ob zwar Menschen leider! auch in viehische Laster fallen, berechtigt doch nicht, eine zu ihrer Spezies gehörige Anlage dazu ihnen beizulegen, so wenig, als die Verkrüppelung einiger Bäume im Walde ein Grund ist, sie zu einer besondern Art von Gewächsen zu machen. 
	
	\subsection*{tg482.2.14} 
	\textbf{Source : }Die Metaphysik der Sitten/Zweiter Teil. Metaphysische Anfangsgründe der Tugendlehre/I. Ethische Elementarlehre/II. Teil. Von den Tugendpflichten gegen andere/Erstes Hauptstück. Von den Pflichten gegen andere, bloß als Menschen/Zweiter Abschnitt. Von den Tugendpflichten gegen andere Menschen aus der ihnen gebührenden Achtung\\  
	
	\noindent\textbf{Paragraphe : }Andere verachten (contemnere), d.i. ihnen die dem Menschen überhaupt schuldige Achtung weigern, ist auf alle Fälle pflichtwidrig; denn es sind Menschen. Sie vergleichungsweise mit anderen innerlich geringschätzen (despicatui habere) ist zwar bisweilen unvermeidlich, aber die äußere Bezeigung der Geringschätzung ist doch Beleidigung. – Was gefährlich ist, ist kein Gegenstand der Verachtung und so ist es auch nicht der Lasterhafte; und, wenn die Überlegenheit über die Angriffe desselben mich berechtigt zu sagen: ich verachte jenen, so bedeutet das nur so viel, als: es ist keine Gefahr dabei, wenn ich gleich gar keine Verteidigung gegen ihn veranstaltete, weil er sich in seiner Verworfenheit selbst darstellt. Nichts desto weniger kann ich selbst dem Lasterhaften als Menschen nicht alle Achtung versagen, die ihm wenigstens in der Qualität eines Menschen nicht entzogen werden kann; ob er zwar durch seine Tat sich derselben unwürdig macht. So kann es schimpfliche, die Menschheit selbst entehrende Strafen geben (wie das Vierteilen, von Hunden zerreißen lassen, Nasen und Ohren abschneiden), die nicht bloß dem Ehrliebenden (der auf Achtung anderer Anspruch macht, was ein jeder tun muß) schmerzhafter sind, als der Verlust der Güter und des Lebens, sondern auch dem Zuschauer Schamröte abjagen, zu einer \match{Gattung} zu gehören, mit der man so verfahren darf. 
	
	\subsection*{tg482.2.42} 
	\textbf{Source : }Die Metaphysik der Sitten/Zweiter Teil. Metaphysische Anfangsgründe der Tugendlehre/I. Ethische Elementarlehre/II. Teil. Von den Tugendpflichten gegen andere/Erstes Hauptstück. Von den Pflichten gegen andere, bloß als Menschen/Zweiter Abschnitt. Von den Tugendpflichten gegen andere Menschen aus der ihnen gebührenden Achtung\\  
	
	\noindent\textbf{Paragraphe : }Die geflissentliche Verbreitung (propalatio) desjenigen, die Ehre eines andern Schmälernden, was auch nicht zur öffentlichen Gerichtsbarkeit gehört, es mag übrigens auch wahr sein, ist Verringerung der Achtung für die Menschheit überhaupt, um endlich auf unsere \match{Gattung} selbst den Schatten der Nichtswürdigkeit zu werfen, und Misanthropie (Menschenscheu) oder Verachtung zur herrschenden Denkungsart zu machen, oder sein moralisches Gefühl durch den öfteren Anblick derselben abzustumpfen und sich daran zu gewöhnen. Es ist also Tugendpflicht, statt einer hämischen Lust an der Bloßstellung der Fehler anderer, um sich dadurch die Meinung, gut, wenigstens nicht schlechter als alle andere Menschen zu sein, zu sichern, den Schleier der Menschenliebe, nicht bloß durch Milderung unserer Urteile, sondern auch durch Verschweigung derselben, über die Fehler anderer zu werfen; weil Beispiele der Achtung, welche uns andere geben, auch die Bestrebung rege machen können, sie gleichmäßig zu verdienen. – Um deswillen ist die Ausspähungssucht der Sitten anderer (allotrio-episcopia) auch für sich selbst schon ein beleidigender Vorwitz der Menschenkunde, welchem jedermann sich mit Recht als Verletzung der ihm schuldigen Achtung widersetzen kann. 
	
	\unnumberedsection{Gemeine (1)} 
	\subsection*{tg441.2.59} 
	\textbf{Source : }Die Metaphysik der Sitten/Erster Teil. Metaphysische Anfangsgründe der Rechtslehre/2. Teil. Das öffentliche Recht/1. Abschnitt. Das Staatsrecht\\  
	
	\noindent\textbf{Paragraphe : }Da auch das Kirchenwesen, welches von der Religion, als innerer Gesinnung, die ganz außer dem Wirkungskreise der bürgerlichen Macht ist, sorgfältig unterschieden werden muß (als Anstalt zum öffentlichen Gottesdienst für das Volk, aus welchem dieser auch seinen Ursprung hat, es sei Meinung oder Überzeugung), ein wahres Staatsbedürfnis  wird, sich auch als Untertanen einer höchsten unsichtbaren Macht, der sie huldigen müssen, und die mit der bürgerlichen oft in einen sehr ungleichen Streit kommen kann, zu betrachten: so hat der Staat das Recht, nicht etwa der inneren Konstitutionalgesetzgebung, das Kirchenwesen nach seinem Sinne, wie es ihm vorteilhaft dünkt, einzurichten, den Glauben und gottesdienstliche Formen (ritus) dem Volk vorzuschreiben, oder zu befehlen (denn dieses muß gänzlich den Lehrern und Vorstehern, die es sich selbst gewählt hat, überlassen bleiben), sondern nur das negative Recht, den Einfluß der öffentlichen Lehrer auf das sichtbare, politische gemeine Wesen, der der öffentlichen Ruhe nachteilig sein möchte, abzuhalten, mithin bei dem inneren Streit, oder dem der verschiedenen Kirchen unter einander, die bürgerliche Eintracht nicht in Gefahr kommen zu lassen, welches also ein Recht der Polizei ist. Daß eine Kirche einen gewissen Glauben, und welchen sie haben, oder daß sie ihn unabänderlich erhalten müsse, und sich nicht selbst reformieren dürfe, sind Einmischungen der obrigkeitlichen Gewalt, die unter ihrer Würde sind; weil sie sich dabei, als einem Schulgezänke, auf den Fuß der Gleichheit mit ihren Untertanen einläßt (der Monarch sich zum Priester macht), die ihr geradezu sagen können, daß sie hievon nichts verstehe; vornehmlich was des letztere, nämlich das Verbot innerer Reformen, betrifft; – denn, was das gesamte Volk nicht über sich selbst beschließen kann, das kann auch der Gesetzgeber nicht über das Volk beschließen. Nun kann aber kein Volk beschließen, in seinen den Glauben betreffenden Einsichten (der Aufklärung) niemals weiter fortzuschreiten, mithin auch sich in Ansehung des Kirchenwesens nie zu reformieren; weil dies der Menschheit in seiner eigenen Person, mithin dem höchsten Rechte desselben entgegen sein würde. Also kann es auch keine obrigkeitliche Gewalt über das Volk beschließen. – – Was aber die Kosten der Erhaltung des Kirchenwesens betrifft, so können diese, aus ebenderselben Ursache, nicht dem Staat, sondern müssen dem Teil des Volks, der sich zu einem oder dem anderen Glauben bekennt, d.i. nur der \match{Gemeine} zu Lasten kommen. 
	
	\unnumberedsection{Geschlecht (3)} 
	\subsection*{tg435.2.28} 
	\textbf{Source : }Die Metaphysik der Sitten/Erster Teil. Metaphysische Anfangsgründe der Rechtslehre/1. Teil. Das Privatrecht vom äußeren Mein und Dein überhaupt/2. Hauptstück. Von der Art, etwas Äußeres zu erwerben/1. Abschnitt. Vom Sachrecht\\  
	
	\noindent\textbf{Paragraphe : }Die Unbestimmtheit, in Ansehung der Quantität sowohl als der Qualität des äußeren erwerblichen Objekts, macht diese Aufgabe (der einzigen ursprünglichen äußeren Erwerbung) unter allen zur schweresten sie aufzulösen. Irgend eine ursprüngliche Erwerbung des Äußeren aber muß es indessen doch geben; denn abgeleitet kann nicht alle sein. Daher kann man diese Aufgabe auch nicht als unauflöslich und als an sich unmöglich aufgeben. Aber,  wenn sie auch durch den ursprünglichen Vertrag aufgelöset wird, so wird, wenn dieser sich nicht aufs ganze menschliche \match{Geschlecht} erstreckt, die Erwerbung doch immer nur provisorisch bleiben. 
	
	\subsection*{tg437.2.16} 
	\textbf{Source : }Die Metaphysik der Sitten/Erster Teil. Metaphysische Anfangsgründe der Rechtslehre/1. Teil. Das Privatrecht vom äußeren Mein und Dein überhaupt/2. Hauptstück. Von der Art, etwas Äußeres zu erwerben/3. Abschnitt. Von dem auf dingliche Art persönlichen Recht\\  
	
	\noindent\textbf{Paragraphe : }Denn der natürliche Gebrauch, den ein \match{Geschlecht} von den Geschlechtsorganen des anderen macht, ist ein Genuß, zu dem sich ein Teil dem anderen hingibt. In diesem Akt macht sich ein Mensch selbst zur Sache, welches dem Rechte der Menschheit an seiner eigenen Person widerstreitet. Nur unter der einzigen Bedingung ist dieses möglich, daß,  indem die eine Person von der anderen, gleich als Sache, erworben wird, diese gegenseitig wiederum jene erwerbe, denn so gewinnt sie wiederum sich selbst und stellt ihre Persönlichkeit wieder her. Es ist aber der Erwerb eines Gliedmaßes am Menschen zugleich Erwerbung der ganzen Person – weil diese eine absolute Einheit ist –; folglich ist die Hingebung und Annehmung eines Geschlechts zum Genuß des andern nicht allein unter der Bedingung der Ehe zulässig, sondern auch allein unter derselben möglich. Daß aber dieses persönliche Recht es doch zugleich auf dingliche Art sei, gründet sich darauf, weil, wenn eines der Eheleute sich verlaufen, oder sich in eines anderen Besitz gegeben hat, das andere es jederzeit und unweigerlich, gleich als eine Sache, in seine Gewalt zurückzubringen berechtigt ist. 
	
	\subsection*{tg471.2.28} 
	\textbf{Source : }Die Metaphysik der Sitten/Zweiter Teil. Metaphysische Anfangsgründe der Tugendlehre/I. Ethische Elementarlehre/I. Teil. Von den Pflichten gegen sich selbst überhaupt/Erstes Buch. Von den vollkommenen Pflichten gegen sich selbst/Erstes Hauptstück. Die Pflicht des Menschen gegen sich selbst, als einem animalischen Wesen\\  
	
	\noindent\textbf{Paragraphe : }So wie die Liebe zum Leben von der Natur zur Erhaltung der Person, so ist die Liebe zum \match{Geschlecht} von ihr zur Erhaltung der Art bestimmt; d.i. eine jede von beiden ist 
	Naturzweck, unter welchem man diejenige Verknüpfung der Ursache mit einer Wirkung versteht, in welcher jene, auch ohne ihr dazu einen Verstand beizulegen, diese doch nach der Analogie mit einem solchen, also gleichsam absichtlich Menschen hervorbringend gedacht wird. Es fragt sich nun, ob der Gebrauch des letzteren Vermögens, in Ansehung der Person selbst, die es ausübt, unter einem einschränkenden Pflichtgesetz stehe, oder ob diese, auch ohne jenen Zweck zu beabsichtigen, den Gebrauch ihrer Geschlechtseigenschaften der bloßen tierischen Lust zu widmen befugt sei, ohne damit einer Pflicht gegen sich selbst zuwider zu handeln. – In der Rechtslehre wird bewiesen, daß der Mensch sich einer anderen Person dieser Lust zu – Gefallen, ohne besondere Einschränkung durch einen rechtlichen Vertrag, nicht bedienen könne; wo dann zwei Personen wechselseitig einander verpflichten. Hier aber ist die Frage: ob in Ansehung dieses Genusses eine Pflicht des Menschen gegen sich selbst obwalte, deren Übertretung eine Schändung (nicht bloß Abwürdigung) der Menschheit in seiner eigenen Person sei. Der Trieb zu jenem wird Fleischeslust (auch Wohllust schlechthin) genannt. Das Laster, welches dadurch erzeugt wird, heißt Unkeuschheit, die Tugend aber, in Ansehung dieser sinnlichen Antriebe, wird Keuschheit genannt, die nun hier als Pflicht des Menschen gegen sich selbst vorgestellt werden soll. Unnatürlich heißt eine Wohllust, wenn der Mensch dazu, nicht durch den wirklichen Gegenstand) sondern durch die Einbildung von demselben, also zweckwidrig, ihn sich selbst schaffend, gereizt wird. Denn sie bewirkt alsdann eine Begierde wider den Zweck der Natur, und zwar einen noch wichtigem, als selbst der der Liebe zum Leben ist, weil dieser nur auf Erhaltung des Individuum, jener aber auf die der ganzen Spezies abzielt. – 
	
	\unnumberedsection{Gift (1)} 
	\subsection*{tg471.2.21} 
	\textbf{Source : }Die Metaphysik der Sitten/Zweiter Teil. Metaphysische Anfangsgründe der Tugendlehre/I. Ethische Elementarlehre/I. Teil. Von den Pflichten gegen sich selbst überhaupt/Erstes Buch. Von den vollkommenen Pflichten gegen sich selbst/Erstes Hauptstück. Die Pflicht des Menschen gegen sich selbst, als einem animalischen Wesen\\  
	
	\noindent\textbf{Paragraphe : }Kann man es einem großen unlängst verstorbenen Monarchen zum verbrecherischen Vorhaben anrechnen, daß er ein behend wirkendes \match{Gift} bei sich führte, vermutlich damit, wenn er in dem Kriege, den er persönlich führte, gefangen würde, er nicht etwa genötigt sei, Bedingungen der Auslösung einzugehn, die seinem Staate nachteilig sein könnten; denn diese Absicht kann man ihm unterlegen, ohne daß man nötig hat, hierunter einen bloßen Stolz zu vermuten? 
	
	\unnumberedsection{Große (2)} 
	\subsection*{tg450.2.10} 
	\textbf{Source : }Die Metaphysik der Sitten/Zweiter Teil. Metaphysische Anfangsgründe der Tugendlehre/Einleitung/II. Erörterung des Begriffs von einem Zwecke, der zugleich Pflicht ist\\  
	
	\noindent\textbf{Paragraphe : }Der Tugend = + a ist die negative Untugend (moralische Schwäche) = 0 als logisches Gegenteil (contradictorie oppositum), das Laster aber = – a als Widerspiel (contrarie s. realiter oppositum) entgegen gesetzt und es ist eine, nicht bloß unnötige, sondern auch anstößige Frage: ob zu großen Verbrechen nicht etwa mehr Stärke der Seele als selbst zu großen Tugenden gehöre. Denn unter Stärke der Seele verstehen wir die Stärke des Vorsatzes eines Menschen, als mit Freiheit begabten Wesens, mithin so fern er seiner selbst mächtig (bei Sinnen) ist, also im gesunden Zustande des Menschen. \match{Große} Verbrechen aber sind Paroxysmen, deren Anblick den an Seele gesunden Menschen schaudern macht. Die Frage würde also etwa dahin auslaufen: ob ein Mensch im Anfall einer Krankheit mehr physische Stärke haben könne, als wenn er bei Sinnen ist; welches  man einräumen kann, ohne ihm darum mehr Seelenstärke beizulegen, wenn man unter Seele das Lebensprinzip des Menschen im freien Gebrauch seiner Kräfte versteht. Denn, weil jene bloß in der Macht der die Vernunft schwächenden Neigungen ihren Grund haben, welches keine Seelenstärke beweiset, so würde diese Frage mit der ziemlich auf einerlei hinauslaufen: ob ein Mensch im Anfall einer Krankheit mehr Stärke als im gesunden Zustande beweisen könne, welche geradezu verneinend beantwortet werden kann, weil der Mangel der Gesundheit, die im Gleichgewicht aller körperlichen Kräfte des Menschen besteht, eine Schwächung im System dieser Kräfte ist, nach welchem man allein die absolute Gesundheit beurteilen kann. 
	
	\subsection*{tg461.2.12} 
	\textbf{Source : }Die Metaphysik der Sitten/Zweiter Teil. Metaphysische Anfangsgründe der Tugendlehre/Einleitung/XIII. Allgemeine Grundsätze der Metaphysik der Sitten in Behandlung einer reinen Tugendlehre\\  
	
	\noindent\textbf{Paragraphe : }Tugend bedeutet eine moralische Stärke des Willens. Aber dies erschöpft noch nicht den Begriff; denn eine solche Stärke könnte auch einem heiligen (übermenschlichen) Wesen zukommen, in welchem kein hindernder Antrieb dem Gesetze seines Willens entgegen wirkt; das also alles dem Gesetz gemäß gerne tut. Tugend ist also die moralische Stärke des Willens eines Menschen in Befolgung seiner Pflicht: welche eine moralische Nötigung durch seine eigene gesetzgebende Vernunft ist, insofern diese sich zu einer das Gesetz ausführenden Gewalt selbst konstituiert. – Sie ist nicht selbst, oder sie zu besitzen ist nicht Pflicht (denn sonst würde es eine Verpflichtung zur Pflicht geben müssen), sondern sie gebietet und begleitet ihr Gebot durch einen sittlichen (nach Gesetzen der inneren Freiheit möglichen) Zwang; wozu aber, weil er unwiderstehlich sein soll, Stärke erforderlich ist, deren Grad wir nur durch die \match{Große} der Hindernisse, die der Mensch durch seine Neigungen sich selber schafft, schätzen können. Die Laster, als die Brut gesetzwidriger Gesinnungen, sind die Ungeheuer, die er nun zu bekämpfen hat: weshalb diese sittliche Stärke auch, als Tapferkeit (fortitudo moralis), die größte und einzige wahre Kriegsehre des Menschen ausmacht; auch wird sie die eigentliche, nämlich praktische Weisheit genannt: weil sie den Endzweck des Daseins der Menschen auf Erden zu dem ihrigen macht. – In ihrem Besitz ist der Mensch allein frei, gesund, reich, ein König u.s.w. und kann, weder durch Zufall, noch Schicksal einbüßen; weil er sich selbst besitzt und der Tugendhafte seine Tugend nicht verlieren kann. 
	
	\unnumberedsection{Klaße (2)} 
	\subsection*{tg429.2.9} 
	\textbf{Source : }Die Metaphysik der Sitten/Erster Teil. Metaphysische Anfangsgründe der Rechtslehre/Vorrede\\  
	
	\noindent\textbf{Paragraphe : }Von der allermindesten Bedeutung aber in Ansehung des Geistes dieser Philosophie ist wohl der Unfug, den einige Nachäffer derselben mit den Wörtern stiften, die in der Kritik d. r. V. selbst nicht wohl durch andere gangbare zu ersetzen sind, sie auch außerhalb derselben zum öffentlichen Gedankenverkehr zu brauchen, und welcher allerdings gezüchtigt zu werden verdient, wie Hr. Nicolai tut, wiewohl er über die gänzliche Entbehrung derselben in ihrem eigentümlichen Felde, gleich als einer überall bloß versteckten Armseligkeit an Gedanken, kein Urteil zu haben sich selbst bescheiden wird. – Indessen läßt sich über den unpopulären Pedanten freilich viel lustiger lachen, als über den unkritischen Ignoranten (denn in der Tat kann der Metaphysiker, welcher seinem Systeme steif anhängt, ohne sich an alle Kritik zu kehren, zur letzteren \match{Klasse} gezählt werden, ob er zwar nur willkürlich ignoriert, was er nicht aufkommen lassen will, weil es zu seiner älteren Schule nicht gehört). Wenn aber, nach Shaftesburys Behauptung, es ein nicht zu verachtender Probierstein für die Wahrheit einer (vornehmlich praktischen) Lehre ist, wenn sie das Belachen aushält, so müßte wohl an den kritischen Philosophen mit der Zeit die Reihe kommen, zuletzt, und so auch am besten, zu lachen; wenn er die papierne Systeme derer, die eine lange Zeit das große Wort führten, nach einander einstürzen, und alle Anhänger derselben sich verlaufen sieht: ein Schicksal, was jenen unvermeidlich bevorsteht. 
	
	\subsection*{tg430.2.26} 
	\textbf{Source : }Die Metaphysik der Sitten/Erster Teil. Metaphysische Anfangsgründe der Rechtslehre/Einleitung in die Metaphysik der Sitten\\  
	
	\noindent\textbf{Paragraphe : }Hieraus ist zu ersehen, daß alle Pflichten bloß darum, weil sie Pflichten sind, mit zur Ethik gehören; aber ihre Gesetzgebung ist darum nicht allemal in der Ethik enthalten, sondern von vielen derselben außerhalb derselben. So gebietet die Ethik, daß ich eine in einem Vertrage getane Anheischigmachung, wenn mich der andere Teil gleich nicht dazu zwingen könnte, doch erfüllen müsse: allein sie nimmt das Gesetz (pacta sunt servanda), und die diesem korrespondierende Pflicht aus der Rechtslehre als gegeben an. Also nicht in der Ethik, sondern im Ius, liegt die Gesetzgebung, daß angenommene Versprechen gehalten werden müssen. Die Ethik lehrt hernach nur, daß, wenn die Triebfeder, welche die juridische Gesetzgebung mit jener Pflicht verbindet, nämlich der äußere Zwang, auch weggelassen wird, die Idee der Pflicht allein schon zur Triebfeder hinreichend sei. Denn wäre das nicht, und die Gesetzgebung selber nicht juridisch, mithin die aus ihr entspringende Pflicht nicht eigentliche Rechtspflicht (zum Unterschiede von der Tugendpflicht), so würde man die Leistung der Treue (gemäß seinem Versprechen in einem Vertrage) mit denen Handlungen des Wohlwollens und der Verpflichtung zu ihnen in eine \match{Klasse} setzen, welches durchaus nicht geschehen muß. Es ist keine Tugendpflicht, sein Versprechen zu halten, sondern eine Rechtspflicht, zu deren Leistung man gezwungen werden kann. Aber es ist doch eine tugendhafte Handlung (Beweis der Tugend), es auch da zu tun, wo kein Zwang besorgt werden darf. Rechtslehre und Tugendlehre unterscheiden sich also nicht sowohl durch ihre verschiedene Pflichten, als vielmehr durch die Verschiedenheit der Gesetzgebung, welche die eine oder die andere Triebfeder mit dem Gesetze verbindet. 
	
	\unnumberedsection{Lauf (1)} 
	\subsection*{tg486.2.31} 
	\textbf{Source : }Die Metaphysik der Sitten/Zweiter Teil. Metaphysische Anfangsgründe der Tugendlehre/II. Ethische Methodenlehre/1. Abschnitt. Die ethische Didaktik\\  
	
	\noindent\textbf{Paragraphe : }7. L. Wenn wir uns aber auch eines solchen guten und tätigen Willens, durch den wir uns würdig (wenigstens nicht unwürdig) halten, glücklich zu sein, auch bewußt sind, können wir darauf auch die sichere Hoffnung gründen, dieser Glückseligkeit teilhaftig zu werden? S. Nein! darauf allein nicht; denn es steht nicht immer in unserem Vermögen, sie uns zu verschaffen, und der \match{Lauf} der Natur richtet sich auch nicht so von selbst nach dem Verdienst, sondern das Glück des Lebens (unsere Wohlfahrt überhaupt) hängt von Umständen ab, die bei weitem nicht alle in des Menschen Gewalt sind. Also bleibt unsere Glückseligkeit immer nur ein Wunsch, ohne daß, wenn nicht irgend eine andere Macht hinzukommt, dieser jemals Hoffnung werden kann. 
	
	\unnumberedsection{Maul (1)} 
	\subsection*{tg445.2.22} 
	\textbf{Source : }Die Metaphysik der Sitten/Erster Teil. Metaphysische Anfangsgründe der Rechtslehre/Anhang erläutender Bemerkungen zu den metaphysischen Anhangsgründen der Rechtslehre\\  
	
	\noindent\textbf{Paragraphe : }Ohne diese Bedingung ist der fleischliche Genuß dem Grundsatz (wenn gleich nicht immer der Wirkung nach) kannibalisch.  Ob, mit \match{Maul} und Zähnen, der weibliche Teil durch Schwängerung, und daraus vielleicht erfolgende, für ihn tödliche, Niederkunft, der männliche aber durch, von öfteren Ansprüchen des Weibes an das Geschlechtsvermögen des Mannes herrührende Erschöpfungen aufgezehrt wird, ist bloß in der Manier zu genießen unterschieden, und ein Teil ist in Ansehung des anderen, bei diesem wechselseitigen Gebrauche der Geschlechtsorganen, wirklich eine verbrauchbare Sache (res fungibilis); zu welcher also sich vermittelst eines Vertrags zu machen es ein gesetzwidriger Vertrag (pactum turpe) sein würde. 
	
	\unnumberedsection{Ordnung (4)} 
	\subsection*{tg441.2.41} 
	\textbf{Source : }Die Metaphysik der Sitten/Erster Teil. Metaphysische Anfangsgründe der Rechtslehre/2. Teil. Das öffentliche Recht/1. Abschnitt. Das Staatsrecht\\  
	
	\noindent\textbf{Paragraphe : }Der Ursprung der obersten Gewalt ist für das Volk, das unter derselben steht, in praktischer Absicht unerforschlich: d.i. der Untertan soll nicht über diesen Ursprung, als ein noch in Ansehung des ihr schuldigen Gehorsams zu bezweifelndes Recht (ius controversum), werktätig vernünfteln. Denn, da das Volk, um rechtskräftig über die oberste Staatsgewalt (summum imperium) zu urteilen, schon als unter einem allgemein gesetzgebenden Willen vereint angesehen werden muß, so kann und darf es nicht anders urteilen, als das gegenwärtige Staatsoberhaupt (summus imperans) es will. – Ob ursprünglich ein wirklicher Vertrag der Unterwerfung unter denselben (pactum subiectionis  civilis) als ein Faktum vorher gegangen, oder ob die Gewalt vorherging, und das Gesetz nur hintennach gekommen sei, oder auch in dieser \match{Ordnung} sich habe folgen sollen: das sind für das Volk, das nun schon unter dem bürgerlichen Gesetze steht, ganz zweckleere, und doch den Staat mit Gefahr bedrohende Vernünfteleien; denn, wollte der Untertan, der den letzteren Ursprung nun ergrübelt hätte, sich jener jetzt herrschenden Autorität widersetzen, so würde er nach den Gesetzen derselben, d.i. mit allem Recht, bestraft, vertilgt, oder (als vogelfrei, exlex) ausgestoßen werden. – Ein Gesetz, das so heilig (unverletzlich) ist, daß es, praktisch, auch nur in Zweifel zu @article{ID,
		author = {author},
		title = {title},
		journaltitle = {journaltitle},
		date = {date},
		OPTtranslator = {translator},
		OPTannotator = {annotator},
		OPTcommentator = {commentator},
		OPTsubtitle = {subtitle},
		OPTtitleaddon = {titleaddon},
		OPTeditor = {editor},
		OPTeditora = {editora},
		OPTeditorb = {editorb},
		OPTeditorc = {editorc},
		OPTjournalsubtitle = {journalsubtitle},
		OPTissuetitle = {issuetitle},
		OPTissuesubtitle = {issuesubtitle},
		OPTlanguage = {language},
		OPToriglanguage = {origlanguage},
		OPTseries = {series},
		OPTvolume = {volume},
		OPTnumber = {number},
		OPTeid = {eid},
		OPTissue = {issue},
		OPTmonth = {month},
		OPTpages = {pages},
		OPTversion = {version},
		OPTnote = {note},
		OPTissn = {issn},
		OPTaddendum = {addendum},
		OPTpubstate = {pubstate},
		OPTdoi = {doi},
		OPTeprint = {eprint},
		OPTeprintclass = {eprintclass},
		OPTeprinttype = {eprinttype},
		OPTurl = {url},
		OPTurldate = {urldate},
	}
	ziehen, mithin seinen Effekt einen Augenblick zu suspendieren, schon ein Verbrechen ist, wird so vorgestellt, als ob es nicht von Menschen, aber doch von irgend einem höchsten tadelfreien Gesetzgeber herkommen müsse, und das ist die Bedeutung des Satzes: »alle Obrigkeit ist von Gott«, welcher nicht einen Geschichtsgrund der bürgerlichen Verfassung, sondern eine Idee, als praktisches Vernunftprinzip, aussagt: der jetzt bestehenden gesetzgebenden Gewalt gehorchen zu sollen; ihr Ursprung mag sein, welcher er wolle. 
	
	\subsection*{tg441.2.46} 
	\textbf{Source : }Die Metaphysik der Sitten/Erster Teil. Metaphysische Anfangsgründe der Rechtslehre/2. Teil. Das öffentliche Recht/1. Abschnitt. Das Staatsrecht\\  
	
	\noindent\textbf{Paragraphe : }Übrigens, wenn eine Revolution einmal gelungen, und eine neue Verfassung gegründet ist, so kann die Unrechtmäßigkeit des Beginnens und der Vollführung derselben die Untertanen von der Verbindlichkeit, der neuen \match{Ordnung} der Dinge sich, als gute Staatsbürger, zu fügen, nicht befreien, und sie können sich nicht weigern, derjenigen Obrigkeit ehrlich zu gehorchen, die jetzt die Gewalt hat. Der entthronte Monarch (der jene Umwälzung überlebt) kann wegen seiner vorigen Geschäftsführung nicht in Anspruch genommen, noch weniger aber gestraft werden, wenn er, in  den Stand eines Staatsbürgers zurückgetreten, seine und des Staats Ruhe dem Wagstück vorzieht, sich von diesem zu entfernen, um als Prätendent das Abenteuer der Wiedererlangung desselben, es sei durch ingeheim angestiftete Gegenrevolution, oder durch Beistand anderer Mächte, zu bestehen. Wenn er aber das letztere vorzieht, so bleibt ihm, weil der Aufruhr, der ihn aus seinem Besitz vertrieb, ungerecht war, sein Recht an demselben unbenommen. Ob aber andere Mächte das Recht haben, sich, diesem verunglückten Oberhaupt zum Besten, in ein Staatenbündnis zu vereinigen, bloß um jenes vom Volk begangene Verbrechen nicht ungeahndet, noch als Skandal für alle Staaten bestehen zu lassen, mithin eine in jedem anderen Staat durch Revolution zu Stande gekommene Verfassung in ihre alte mit Gewalt zurückzubringen berechtigt und berufen sein, das gehört zum Völkerrecht. 
	
	\subsection*{tg486.2.32} 
	\textbf{Source : }Die Metaphysik der Sitten/Zweiter Teil. Metaphysische Anfangsgründe der Tugendlehre/II. Ethische Methodenlehre/1. Abschnitt. Die ethische Didaktik\\  
	
	\noindent\textbf{Paragraphe : }8. L. Hat die Vernunft wohl Gründe für sich, eine solche, die Glückseligkeit nach Verdienst und Schuld der  Menschen austeilende, über die ganze Natur gebietende und die Welt mit höchster Weisheit regierende Macht als wirklich anzunehmen, d.i. an Gott zu glauben? S. Ja; denn wir sehen an den Werken der Natur, die wir beurteilen können, so ausgebreitete und tiefe Weisheit, die wir uns nicht anders als durch eine unaussprechlich große Kunst eines Weltschöpfers erklären können, von welchem wir uns denn auch, was die sittliche \match{Ordnung} betrifft, in der doch die höchste Zierde der Welt besteht, eine nicht minder weise Regierung zu versprechen Ursache haben: nämlich, daß, wenn wir uns nicht selbst der Glückseligkeit unwürdig machen, welches durch Übertretung unserer Pflicht geschieht, wir auch hoffen können, ihrer teilhaftig zu werden. 
	
	\subsection*{tg486.2.35} 
	\textbf{Source : }Die Metaphysik der Sitten/Zweiter Teil. Metaphysische Anfangsgründe der Tugendlehre/II. Ethische Methodenlehre/1. Abschnitt. Die ethische Didaktik\\  
	
	\noindent\textbf{Paragraphe : }Wenn dieses nun weislich und pünktlich nach Verschiedenheit der Stufen des Alters, des Geschlechts und des Standes, die der Mensch nach und nach betritt, aus der eigenen Vernunft des Menschen entwickelt worden,  so ist noch etwas, was den Beschluß machen muß, was die Seele inniglich bewegt und den Menschen auf eine Stelle setzt, wo er sich selbst nicht anders als mit der größten Bewunderung der ihm beiwohnenden ursprünglichen Anlagen betrachten kann, und wovon der Eindruck nie erlischt. – Wenn ihm nämlich beim Schlüsse seiner Unterweisung seine Pflichten in ihrer \match{Ordnung} noch einmal summarisch vorerzählt (rekapituliert), wenn er, bei jeder derselben, darauf aufmerksam gemacht wird, daß alle Übel, Drangsale und Leiden des Lebens, selbst Bedrohung mit dem Tode, die ihn darüber, daß er seiner Pflicht treu gehorcht, treffen mögen, ihm doch das Bewußtsein, über sie alle erhoben und Meister zu sein, nicht rauben können, so liegt ihm nun die Frage ganz nahe: was ist das in dir, was sich getrauen darf, mit allen Kräften der Natur in dir und um dich in Kampf zu treten und sie, wenn sie mit deinen sittlichen Grundsätzen in Streit kommen, zu besiegen? Wenn diese Frage, deren Auflösung das Vermögen der spekulativen Vernunft gänzlich übersteigt und die sich dennoch von selbst einstellt, ans Herz gelegt wird, so muß selbst die Unbegreiflichkeit in diesem Selbsterkenntnisse der Seele eine Erhebung geben, die sie zum Heilighalten ihrer Pflicht nur desto stärker belebt, je mehr sie angefochten wird. 
	
	\unnumberedsection{Pferd (3)} 
	\subsection*{tg436.2.22} 
	\textbf{Source : }Die Metaphysik der Sitten/Erster Teil. Metaphysische Anfangsgründe der Rechtslehre/1. Teil. Das Privatrecht vom äußeren Mein und Dein überhaupt/2. Hauptstück. Von der Art, etwas Äußeres zu erwerben/2. Abschnitt. Vom persönlichen Recht\\  
	
	\noindent\textbf{Paragraphe : }Wenn ich einen Vertrag über eine Sache, z.B. über ein Pferd, das ich erwerben will, schließe, und nehme es zugleich mit in meinen Stall, oder sonst in meinen physischen Besitz, so ist es mein (vi pacti re initi), und mein Recht ist ein Recht in der Sache; lasse ich es aber in den Händen des Verkäufers, ohne mit ihm darüber besonders auszumachen, in wessen physischem Besitz (Inhabung) diese Sache vor meiner Besitznehmung (apprehensio), mithin vor dem Wechsel des Besitzes sein solle: so ist dieses \match{Pferd} noch nicht mein, und mein Recht, was ich erwerbe, ist nur ein Recht gegen eine bestimmte Person, nämlich den Verkäufer, von ihm in den Besitz gesetzt zu werden (poscendi traditionem), als subjektive Bedingung der Möglichkeit alles beliebigen Gebrauchs desselben, d.i. mein Recht ist nur ein persönliches Recht, von jenem die Leistung des Versprechens (praestatio), mich in den Besitz der Sache zu setzen, zu fordern. Nun kann ich, wenn der Vertrag nicht zugleich die Übergabe (als pactum re initum) enthält, mithin eine Zeit zwischen dem Abschluß desselben und der Besitznehmung des Erworbenen verläuft, in dieser Zeit nicht anders zum Besitz gelangen, als dadurch, daß ich einen besonderen rechtlichen, nämlich einen Besitzakt (actum possessorium) ausübe, der einen besonderen Vertrag ausmacht, und dieser ist: daß ich sage, ich werde die Sache (das Pferd) abholen lassen, wozu der Verkäufer einwilligt. Denn daß dieser eine Sache zum Gebrauche eines anderen auf eigene Gefahr in seine Gewahrsame nehmen werde, versteht sich nicht von selbst, sondern dazu gehört ein besonderer Vertrag, nach welchem der Veräußerer seiner Sache innerhalb der bestimmten Zeit noch immer Eigentümer bleibt (und alle Gefahr, die die Sache treffen möchte, tragen muß), der Erwerbende aber nur dann, wann er über diese Zeit zögert, von dem Verkäufer dafür angesehen werden kann, als sei sie ihm überliefert. Vor diesem Besitzakt ist also alles durch den Vertrag Erworbene nur ein persönliches Recht, und der Promissar kann eine äußere Sache nur durch Tradition erwerben. 
	
	\subsection*{tg439.2.28} 
	\textbf{Source : }Die Metaphysik der Sitten/Erster Teil. Metaphysische Anfangsgründe der Rechtslehre/1. Teil. Das Privatrecht vom äußeren Mein und Dein überhaupt/3. Hauptstück. Von der subjektiv-bedingten Erwerbung durch den Ausspruch einer öffentlichen Gerichtsbarkeit\\  
	
	\noindent\textbf{Paragraphe : }
	Hierwider erheben sich aber folgende Rechtsgründe: Alle Erwerbung von einem, der nicht Eigentümer der Sache ist (a non domino), ist null und nichtig. Ich kann von dem Seinen eines anderen nicht mehr auf mich ableiten, als er selbst rechtmäßig gehabt hat, und, ob ich gleich, was die Form der Erwerbung (modus acquirendi) betrifft, ganz rechtlich verfahre, wenn ich ein gestohlen Pferd, was auf dem Markte feilsteht, erhandle, so fehlt doch der Titel der Erwerbung; denn das \match{Pferd} war nicht das Seine des eigentlichen Verkäufers. Ich mag immer ein ehrlicher Besitzer desselben (possessor bonae fidei) sein, so bin ich doch nur ein sich dünkender Eigentümer (dominus putativus) und der wahre Eigentümer hat ein Recht der Wiedererlangung (rem suam vindicandi). 
	
	\subsection*{tg439.2.29} 
	\textbf{Source : }Die Metaphysik der Sitten/Erster Teil. Metaphysische Anfangsgründe der Rechtslehre/1. Teil. Das Privatrecht vom äußeren Mein und Dein überhaupt/3. Hauptstück. Von der subjektiv-bedingten Erwerbung durch den Ausspruch einer öffentlichen Gerichtsbarkeit\\  
	
	\noindent\textbf{Paragraphe : }Wenn gefragt wird, was (im Naturzustande) unter Menschen, nach Prinzipien der Gerechtigkeit im Verkehr derselben untereinander (iustitia commutativa) in Erwerbung äußerer Sachen an sich Rechtens sei, so muß man eingestehen: daß, wer dieses zur Absicht hat, durchaus nötig habe, noch nachzuforschen, ob die Sache, die er erwerben will, nicht schon einem anderen angehöre; nämlich, wenn er gleich die formalen Bedingungen der Ableitung der Sache von dem Seinen des anderen genau beobachtet (das \match{Pferd} auf dem Markte ordentlich erhandelt) hat, er dennoch höchstens nur ein persönliches Recht in Ansehung einer Sache (ius ad rem) habe erwerben können, so lange es ihm noch unbekannt ist, ob nicht ein anderer (als der Verkäufer) der wahre Eigentümer derselben sei; so daß, wenn sich einer vorfindet, der sein vorhergehendes Eigentum daran dokumentieren könnte, dem vermeinten neuen Eigentümer nichts übrig bliebe, als den Nutzen, so er, als ehrlicher Besitzer, bisher daraus gezogen hat, bis auf diesen Augenblick rechtmäßig genossen zu haben. – Da nun in der Reihe der von einander ihr Recht ableitenden sich dünkenden Eigentümer den schlechthin ersten (Stammeigentümer) auszufinden mehrenteils unmöglich ist: so kann kein Verkehr mit äußeren Sachen, so gut er auch mit den formalen Bedingungen dieser Art von Gerechtigkeit (iustitia  commutativa) übereinstimmen möchte, einen sicheren Erwerb gewähren. 
	
	\unnumberedsection{Schmarotzer (1)} 
	\subsection*{tg484.2.20} 
	\textbf{Source : }Die Metaphysik der Sitten/Zweiter Teil. Metaphysische Anfangsgründe der Tugendlehre/I. Ethische Elementarlehre/II. Teil. Von den Tugendpflichten gegen andere/Beschluß der Elementarlehre. Von der innigsten Vereinigung der Liebe mit der Achtung in der Freundschaft\\  
	
	\noindent\textbf{Paragraphe : }Es frägt sich aber hiebei: ob man auch mit Lasterhaften Umgang pflegen dürfe? Die Zusammenkunft mit ihnen kann man nicht vermeiden; man müßte denn sonst aus der Welt gehen, und selbst unser Urteil über sie ist nicht kompetent. – Wo aber das Laster ein Skandal, d.i. ein öffentlich gegebenes Beispiel der Verachtung strenger Pflichtgesetze ist, mithin Ehrlosigkeit bei sich führt: da muß, wenn gleich das Landesgesetz es nicht bestraft, der Umgang, der bis dahin statt fand, abgebrochen, oder so viel möglich gemieden werden; weil die fernere Fortsetzung desselben die Tugend um alle Ehre bringt, und sie für jeden zu Kauf stellt, der reich genug ist, um den \match{Schmarotzer} durch die Vergnügungen der Üppigkeit zu bestechen. 
	
	\unnumberedsection{Stimme (5)} 
	\subsection*{tg441.2.75} 
	\textbf{Source : }Die Metaphysik der Sitten/Erster Teil. Metaphysische Anfangsgründe der Rechtslehre/2. Teil. Das öffentliche Recht/1. Abschnitt. Das Staatsrecht\\  
	
	\noindent\textbf{Paragraphe : }Strafe erleidet jemand nicht, weil er sie, sondern weil er eine strafbare Handlung gewollt hat; denn es ist keine Strafe, wenn einem geschieht, was er will, und es ist unmöglich, gestraft werden zu wollen. – Sagen: ich will gestraft werden, wenn ich jemand ermorde, heißt nichts mehr, als: ich unterwerfe mich samt allen übrigen den Gesetzen, welche natürlicherweise, wenn es Verbrecher im Volk gibt, auch Strafgesetze sein werden. Ich, als Mitgesetzgeber, der das Strafgesetz diktiert, kann unmöglich dieselbe Person sein, die, als Untertan, nach dem Gesetz bestraft wird; denn als ein solcher, nämlich als Verbrecher, kann ich unmöglich eine \match{Stimme} in der Gesetzgebung haben (der Gesetzgeber ist heilig). Wenn ich also ein Strafgesetz gegen mich, als einen Verbrecher, abfasse, so ist es in mir die reine rechtlich-gesetzgebende Vernunft (homo noumenon), die mich als einen des Verbrechens Fähigen, folglich als eine andere Person (homo phaenomenon), samt allen übrigen in einem Bürgerverein dem Strafgesetze unterwirft. Mit andern Worten: nicht das Volk (jeder einzelne in demselben), sondern das Gericht (die öffentliche Gerechtigkeit), mithin ein anderer als der Verbrecher, diktiert die Todesstrafe, und im Sozialkontrakt ist gar nicht das Versprechen enthalten, sich strafen zu lassen, und so über sich selbst und sein Leben zu disponieren. Denn, wenn der Befugnis zu strafen ein Versprechen des Missetäters zum Grunde liegen müßte, sich  strafen lassen zu wollen, so müßte es diesem auch überlassen werden, sich straffällig zu finden, und der Verbrecher würde sein eigener Richter sein. – Der Hauptpunkt des Irrtums (prôton pseudos) dieses Sophisms besteht darin: daß das eigene Urteil des Verbrechers (das man seiner Vernunft notwendig zutrauen muß), des Lebens verlustig werden zu müssen, für einen Beschluß des Willens ansieht, es sich selbst zu nehmen, und so sich die Rechtsvollziehung mit der Rechtsbeurteilung in einer und derselben Person vereinigt vorstellt. 
	
	\subsection*{tg442.2.16} 
	\textbf{Source : }Die Metaphysik der Sitten/Erster Teil. Metaphysische Anfangsgründe der Rechtslehre/2. Teil. Das öffentliche Recht/2. Abschnitt. Das Völkerrecht\\  
	
	\noindent\textbf{Paragraphe : }Wir werden also wohl dieses Recht von der Pflicht des Souveräns gegen das Volk (nicht umgekehrt) abzuleiten haben; wobei dieses dafür angesehen werden muß, daß es seine \match{Stimme} dazu gegeben habe, in welcher Qualität es, obzwar passiv (mit sich machen läßt), doch auch selbsttätig ist, und den Souverän selbst vorstellt. 
	
	\subsection*{tg460.2.16} 
	\textbf{Source : }Die Metaphysik der Sitten/Zweiter Teil. Metaphysische Anfangsgründe der Tugendlehre/Einleitung/XII. Ästhetische Vorbegriffe der Empfänglichkeit des Gemüts für Pflichtbegriffe überhaupt\\  
	
	\noindent\textbf{Paragraphe : }Die Pflicht ist hier nur, sein Gewissen zu kultivieren, die Aufmerksamkeit auf die \match{Stimme} des inneren Richters zu schärfen und alle Mittel anzuwenden (mithin nur indirekte Pflicht), um ihm Gehör zu verschaffen. 
	
	\subsection*{tg473.2.5} 
	\textbf{Source : }Die Metaphysik der Sitten/Zweiter Teil. Metaphysische Anfangsgründe der Tugendlehre/I. Ethische Elementarlehre/I. Teil. Von den Pflichten gegen sich selbst überhaupt/Erstes Buch. Von den vollkommenen Pflichten gegen sich selbst/Zweites Hauptstück. Die Pflicht des Menschen gegen sich selbst, bloß als einem moralischen Wesen/1. Abschnitt. Von der Pflicht des Menschen gegen sich selbst, als dem angebornen Richter über sich selbst\\  
	
	\noindent\textbf{Paragraphe : }Jeder Mensch hat Gewissen, und findet sich durch einen inneren Richter beobachtet, bedroht und überhaupt im Respekt (mit Furcht verbundener Achtung) gehalten, und diese über die Gesetze in ihm wachende Gewalt ist nicht etwas, was er sich selbst (willkürlich) macht, sondern es ist seinem Wesen einverleibt. Es folgt ihm wie sein Schatten, wenn er zu entfliehen gedenkt. Er kann sich zwar durch Lüste und Zerstreuungen betäuben, oder in Schlaf bringen, aber nicht vermeiden, dann und wann zu sich selbst zu kommen, oder zu erwachen, wo er alsbald die furchtbare \match{Stimme} desselben vernimmt. Er kann es, in seiner äußersten Verworfenheit, allenfalls dahin bringen, sich daran gar nicht mehr zu kehren, aber sie zu hören kann er doch nicht vermeiden. 
	
	\subsection*{tg489.2.33} 
	\textbf{Source : }Die Metaphysik der Sitten/Fußnoten\\  
	
	\noindent\textbf{Paragraphe : }
	
	15 Je weniger der Mensch physisch, je mehr er dagegen moralisch (durch die bloße Vorstellung der Pflicht) kann gezwungen werden, desto freier ist er. – Der, so, z.B., von genugsam fester Entschließung und starker Seele ist, eine Lustbarkeit, die er sich vorgenommen hat, nicht aufzugeben, man mag ihm noch so viel Schaden vorstellen, den er sich dadurch zuzieht, aber auf die Vorstellung, daß er hiebei eine Amtspflicht verabsäume, oder einen kranken Vater vernachlässige, von seinem Vorsatz unbedenklich, obzwar sehr ungern, absteht, beweist eben damit seine Freiheit im höchsten Grade, daß er der \match{Stimme} der Pflicht nicht widerstehen kann. 
	
	\unnumberedsection{Tier (2)} 
	\subsection*{tg456.2.3} 
	\textbf{Source : }Die Metaphysik der Sitten/Zweiter Teil. Metaphysische Anfangsgründe der Tugendlehre/Einleitung/VIII. Exposition der Tugendpflichten als weiter Pflichten\\  
	
	\noindent\textbf{Paragraphe : }a) Physische, d.i. Kultur aller Vermögen überhaupt, zu Beförderung der durch die Vernunft vorgelegten Zwecke. Daß dieses Pflicht, mithin an sich selbst Zweck sei, und jener Bearbeitung, auch ohne Rücksicht auf den Vorteil, den sie uns gewährt, nicht ein bedingter (pragmatischer), sondern unbedingter (moralischer) Imperativ zum Grunde liege, ist hieraus zu ersehen. Das Vermögen, sich überhaupt irgend einen Zweck zu setzen, ist das Charakteristische der Menschheit (zum Unterschiede von der Tierheit). Mit dem Zwecke der Menschheit in unserer eigenen Person ist also auch der Vernunftwille, mithin die Pflicht verbunden, sich um die Menschheit durch Kultur überhaupt verdient zu machen, sich das Vermögen zu Ausführung allerlei möglichen Zwecke, so fern dieses in dem Menschen selbst anzutreffen ist, zu verschaffen oder es zu fördern, d.i. eine Pflicht zur Kultur der rohen Anlagen seiner Natur, als wodurch das \match{Tier} sich allererst zum Menschen erhebt: mithin Pflicht an sich selbst. 
	
	\subsection*{tg481.2.69} 
	\textbf{Source : }Die Metaphysik der Sitten/Zweiter Teil. Metaphysische Anfangsgründe der Tugendlehre/I. Ethische Elementarlehre/II. Teil. Von den Tugendpflichten gegen andere/Erstes Hauptstück. Von den Pflichten gegen andere, bloß als Menschen/Erster Abschnitt. Von der Liebespflicht gegen andere Menschen\\  
	
	\noindent\textbf{Paragraphe : }
	Mitfreude und Mitleid (sympathia moralis) sind zwar sinnliche Gefühle einer (darum ästhetisch zu nennenden) Lust oder Unlust an dem Zustande des Vergnügens so wohl als Schmerzens anderer (Mitgefühl, teilnehmende Empfindung), wozu schon die Natur in den Menschen die Empfänglichkeit gelegt hat. Aber diese als Mittel zu Beförderung des tätigen und vernünftigen Wohlwollens zu gebrauchen, ist noch eine besondere, obzwar nur bedingte, Pflicht, unter dem Namen der Menschlichkeit (humanitas); weil hier der Mensch nicht bloß als vernünftiges Wesen, sondern  auch als mit Vernunft begabtes \match{Tier} betrachtet wird. Diese kann nun in dem Vermögen und Willen, sich einander in Ansehung seiner Gefühle mitzuteilen (humanitas practica), oder bloß in der Empfänglichkeit für das gemeinsame Gefühl des Vergnügens oder Schmerzens (humanitas aesthetica), was die Natur selbst gibt, gesetzt werden. Das erstere ist frei und wird daher teilnehmend genannt (communio sentiendi liberalis) und gründet sich auf praktische Vernunft; das zweite ist unfrei (communio sentiendi illiberalis, servilis) und kann mitteilend (wie die der Wärme oder ansteckender Krankheiten), auch Mitleidenschaft heißen; weil sie sich unter nebeneinander lebenden Menschen natürlicher Weise verbreitet. Nur zu dem ersten gibt's Verbindlichkeit. 
	
	\unnumberedsection{Wurm (1)} 
	\subsection*{tg472.2.48} 
	\textbf{Source : }Die Metaphysik der Sitten/Zweiter Teil. Metaphysische Anfangsgründe der Tugendlehre/I. Ethische Elementarlehre/I. Teil. Von den Pflichten gegen sich selbst überhaupt/Erstes Buch. Von den vollkommenen Pflichten gegen sich selbst\\  
	
	\noindent\textbf{Paragraphe : }Die vorzügliche Achtungsbezeigung in Worten und Manieren, selbst gegen einen nicht Gebietenden in der bürgerlichen Verfassung – die Reverenzen, Verbeugungen (Komplimente), höfische – den Unterschied der Stände mit sorgfältiger Pünktlichkeit bezeichnende Phrasen, – welche von der Höflichkeit (die auch sich gleich Achtenden notwendig ist) ganz unterschieden sind, – das Du, Er, Ihr und Sie, oder Ew. Wohledlen, Hochedeln, Hochedelgebornen, Wohlgebornen (ohe, iam satis est!) in der Anrede – als in welcher Pedanterei die Deutschen unter allen Völkern der Erde (die indischen Kasten vielleicht ausgenommen) es am weitesten gebracht haben, sind das nicht Beweise eines ausgebreiteten Hanges zur Kriecherei unter Menschen? (Hae nugae in seria ducunt.) Wer sich aber zum \match{Wurm} macht, kann nachher nicht klagen, daß er mit Füßen getreten wird. 
	
\end{document}