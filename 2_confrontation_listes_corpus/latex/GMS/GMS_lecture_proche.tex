\documentclass[a4paper,12pt,twoside]{book}
\usepackage{fontspec}
\usepackage{xunicode}
\usepackage[german]{babel}


\usepackage{xcolor}

\newcommand{\match}[1]{\textcolor{red}{\textbf{#1}}}





\usepackage{fancyhdr}
\usepackage{etoolbox} % For defining robust commands

\pagestyle{fancy}

\fancyhead[LE]{\thepage}
\fancyhead[RE]{\leftmark}

\fancyhead[LO]{\rightmark}
\fancyhead[RO]{\thepage}

% Define a new command for unnumbered chapters
\newcommand{\unnumberedchapter}[1]{
	\chapter*{#1}
	\addcontentsline{toc}{chapter}{#1}
	\markboth{#1}{#1}
}

% Similarly, for sections
\newcommand{\unnumberedsection}[1]{
	\section*{#1}
	\addcontentsline{toc}{section}{#1}
	\markright{#1}
}

\usepackage[hidelinks]{hyperref} %
\usepackage[numbered]{bookmark}%va avec hyperref; marche mieux pour les signets. l'option numbered: les signets dans le pdf sont numérotés

\author{Eglantine Gaglione - M1 HN}
\title{GMS : vocabulaire thématique}

\begin{document}
	
	\maketitle
	
	\tableofcontents
	
	\unnumberedchapter{Agriculture} 
	\unnumberedsection{Abfall (1)} 
	\subsection*{tg176.2.5} 
	\textbf{Source : }Grundlegung zur Metaphysik der Sitten/Zweiter Abschnitt: Übergang von der populären sittlichen Weltweisheit zur Metaphysik der Sitten\\  
	
	\noindent\textbf{Paragraphe : }Man kann auch denen, die alle Sittlichkeit, als bloßes Hirngespinst einer durch Eigendünkel sich selbst übersteigenden menschlichen Einbildung, verlachen, keinen gewünschteren Dienst tun, als ihnen einzuräumen, daß die Begriffe der Pflicht (so wie man sich auch aus Gemächlichkeit  gerne überredet, daß es auch mit allen übrigen Begriffen bewandt sei) lediglich aus der Erfahrung gezogen werden mußten; denn da bereitet man jenen einen sichern Triumph. Ich will aus Menschenliebe einräumen, daß noch die meisten unserer Handlungen pflichtmäßig sein; sieht man aber ihr Tichten und Trachten näher an, so stößt man allenthalben auf das liebe Selbst, was immer hervorsticht, worauf, und nicht auf das strenge Gebot der Pflicht, welches mehrmalen Selbstverleugnung erfodern würde, sich ihre Absicht stützet. Man braucht auch eben kein Feind der Tugend, sondern nur ein kaltblütiger Beobachter zu sein, der den lebhaftesten Wunsch für das Gute nicht so fort für dessen Wirklichkeit hält, um (vornehmlich mit zunehmenden Jahren und einer durch Erfahrung teils gewitzigten, teils zum Beobachten geschärften Urteilskraft) in gewissen Augenblicken zweifelhaft zu werden, ob auch wirklich in der Welt irgend wahre Tugend angetroffen werde. Und hier kann uns nun nichts für den gänzlichen \match{Abfall} von unseren Ideen der Pflicht bewahren und gegründete Achtung gegen ihr Gesetz in der Seele erhalten, als die klare Überzeugung, daß, wenn es auch niemals Handlungen gegeben habe, die aus solchen reinen Quellen entsprungen wären, dennoch hier auch davon gar nicht die Rede sei, ob dies oder jenes geschehe, sondern die Vernunft für sich selbst und unabhängig von allen Erscheinungen gebiete, was geschehen soll, mithin Handlungen, von denen die Welt vielleicht bisher noch gar kein Beispiel gegeben hat, an deren Tunlichkeit sogar der, so alles auf Erfahrung gründet, sehr zweifeln möchte, dennoch durch Vernunft unnachlaßlich geboten sei, und daß z.B. reine Redlichkeit in der Freundschaft um nichts weniger von jedem Menschen gefodert werden könne, wenn es gleich bis jetzt gar keinen redlichen Freund gegeben haben möchte, weil diese Pflicht als Pflicht überhaupt, vor aller Erfahrung, in der Idee einer den Willen durch Gründe a priori bestimmenden Vernunft liegt. 
	
	\unnumberedsection{Fallen (4)} 
	\subsection*{tg174.2.7} 
	\textbf{Source : }Grundlegung zur Metaphysik der Sitten/Vorrede\\  
	
	\noindent\textbf{Paragraphe : }Alle Gewerbe, Handwerke und Künste, haben durch die Verteilung der Arbeiten gewonnen, da nämlich nicht einer alles macht, sondern jeder sich auf gewisse Arbeit, die sich, ihrer Behandlungsweise nach, von andern merklich unterscheidet, einschränkt, um sie in der größten Vollkommenheit und mit mehrerer Leichtigkeit leisten zu können. Wo die Arbeiten so nicht unterschieden und verteilt werden, wo jeder ein Tausendkünstler ist, da liegen die Gewerbe noch in der größten Barbarei. Aber ob dieses zwar für sich ein der Erwägung nicht unwürdiges Objekt wäre, zu fragen: ob die reine Philosophie in allen ihren Teilen nicht ihren besondern Mann erheische, und es um das Ganze des gelehrten Gewerbes nicht besser stehen würde, wenn die, so das Empirische mit dem Rationalen, dem Geschmacke des Publikums gemäß, nach allerlei ihnen selbst unbekannten Verhältnissen gemischt, zu verkaufen gewohnt sind, die sich Selbstdenker, andere aber, die den bloß rationalen Teil zubereiten, Grübler nennen, gewarnt würden, nicht zwei Geschäfte zugleich zu treiben, die in der Art, sie zu behandeln, gar sehr verschieden sind, zu deren jedem vielleicht ein besonderes Talent erfodert wird, und deren Verbindung in einer Person nur Stümper hervorbringt: so frage ich hier doch nur, ob nicht die Natur der Wissenschaft es erfodere, den empirischen von dem rationalen Teil jederzeit sorgfältig abzusondern, und vor der eigentlichen (empirischen) Physik eine Metaphysik der Natur, vor der praktischen Anthropologie aber eine Metaphysik der Sitten voranzuschicken, die von allem Empirischen sorgfältig gesäubert sein müßte, um zu  wissen, wie viel reine Vernunft in beiden \match{Fällen} leisten könne, und aus welchen Quellen sie selbst diese ihre Belehrung a priori schöpfe, es mag übrigens das letztere Geschäfte von allen Sittenlehrern (deren Name Legion heißt), oder nur von einigen, die Beruf dazu fühlen, getrieben werden. 
	
	\subsection*{tg174.2.9} 
	\textbf{Source : }Grundlegung zur Metaphysik der Sitten/Vorrede\\  
	
	\noindent\textbf{Paragraphe : }Also unterscheiden sich die moralischen Gesetze, samt ihren Prinzipien, unter allem praktischen Erkenntnisse von allem übrigen, darin irgend etwas Empirisches ist, nicht allein wesentlich, sondern alle Moralphilosophie beruht gänzlich auf ihrem reinen Teil, und, auf den Menschen angewandt, entlehnt sie nicht das mindeste von der Kenntnis desselben (Anthropologie), sondern gibt ihm, als vernünftigem Wesen, Gesetze a priori, die freilich noch durch Erfahrung geschärfte Urteilskraft erfodern, um teils zu unterscheiden,  in welchen \match{Fällen} sie ihre Anwendung haben, teils ihnen Eingang in den Willen des Menschen und Nachdruck zur Ausübung zu verschaffen, da diese, als selbst mit so viel Neigungen affiziert, der Idee einer praktischen reinen Vernunft zwar fähig, aber nicht so leicht vermögend ist, sie in seinem Lebenswandel in concreto wirksam zu machen. 
	
	\subsection*{tg175.2.22} 
	\textbf{Source : }Grundlegung zur Metaphysik der Sitten/Erster Abschnitt: Übergang von der gemeinen sittlichen Vernunfterkenntnis zur philosophischen\\  
	
	\noindent\textbf{Paragraphe : }So sind wir denn in der moralischen Erkenntnis der gemeinen Menschenvernunft bis zu ihrem Prinzip gelangt, welches sie sich zwar freilich nicht so in einer allgemeinen  Form abgesondert denkt, aber doch jederzeit wirklich vor Augen hat und zum Richtmaße ihrer Beurteilung braucht. Es wäre hier leicht zu zeigen, wie sie, mit diesem Kompasse in der Hand, in allen vorkommenden \match{Fällen} sehr gut Bescheid wisse, zu unterscheiden, was gut, was böse, pflichtmäßig, oder pflichtwidrig sei, wenn man, ohne sie im mindesten etwas Neues zu lehren, sie nur, wie Sokrates tat, auf ihr eigenes Prinzip aufmerksam macht, und daß es also keiner Wissenschaft und Philosophie bedürfe, um zu wissen, was man zu tun habe, um ehrlich und gut, ja sogar, um weise und tugendhaft zu sein. Das ließe sich auch wohl schon zum voraus vermuten, daß die Kenntnis dessen, was zu tun, mithin auch zu wissen jedem Menschen obliegt, auch jedes, selbst des gemeinsten Menschen Sache sein werde. Hier kann man es doch nicht ohne Bewunderung ansehen, wie das praktische Beurteilungsvermögen vor dem theoretischen im gemeinen Menschenverstande so gar viel voraus habe. In dem letzteren, wenn die gemeine Vernunft es wagt, von den Erfahrungsgesetzen und den Wahrnehmungen der Sinne abzugehen, gerät sie in lauter Unbegreiflichkeiten und Widersprüche mit sich selbst, wenigstens in ein Chaos von Ungewißheit, Dunkelheit und Unbestand. Im praktischen aber fängt die Beurteilungskraft denn eben allererst an, sich recht vorteilhaft zu zeigen, wenn der gemeine Verstand alle sinnliche Triebfedern von praktischen Gesetzen ausschließt. Er wird alsdenn so gar subtil, es mag sein, daß er mit seinem Gewissen, oder anderen Ansprüchen in Beziehung auf das, was recht heißen soll, schikanieren, oder auch den Wert der Handlungen zu seiner eigenen Belehrung aufrichtig bestimmen will, und, was das meiste ist, er kann im letzteren Falle sich eben so gut Hoffnung machen, es recht zu treffen, als es sich immer ein Philosoph versprechen mag, ja ist beinahe noch sicherer hierin, als selbst der letztere, weil dieser doch kein anderes Prinzip als jener haben, sein Urteil aber, durch eine Menge fremder, nicht zur Sache gehöriger Erwägungen, leicht verwirren und von der geraden Richtung abweichend machen kann. Wäre es demnach nicht ratsamer,  es in moralischen Dingen bei dem gemeinen Vernunfturteil bewenden zu lassen, und höchstens nur Philosophie anzubringen, um das System der Sitten desto vollständiger und faßlicher, imgleichen die Regeln derselben zum Gebrauche (noch mehr aber zum Disputieren) bequemer darzustellen, nicht aber, um selbst in praktischer Absicht den gemeinen Menschenverstand von seiner glücklichen Einfalt abzubringen, und ihn durch Philosophie auf einen neuen Weg der Untersuchung und Belehrung zu bringen? 
	
	\subsection*{tg176.2.27} 
	\textbf{Source : }Grundlegung zur Metaphysik der Sitten/Zweiter Abschnitt: Übergang von der populären sittlichen Weltweisheit zur Metaphysik der Sitten\\  
	
	\noindent\textbf{Paragraphe : }Die Imperativen der Klugheit würden, wenn es nur so leicht wäre, einen bestimmten Begriff von Glückseligkeit zu geben, mit denen der Geschicklichkeit ganz und gar übereinkommen und eben sowohl analytisch sein. Denn es würde eben sowohl hier, als dort, heißen: wer den Zweck will, will auch (der Vernunft gemäß notwendig) die einzigen Mittel, die dazu in seiner Gewalt sind. Allein es ist ein Unglück, daß der Begriff der Glückseligkeit ein so unbestimmter Begriff ist, daß, obgleich jeder Mensch zu dieser zu gelangen wünscht, er doch niemals bestimmt und mit sich selbst einstimmig sagen kann, was er eigentlich wünsche und wolle. Die Ursache davon ist: daß alle Elemente, die zum Begriff der Glückseligkeit gehören, insgesamt empirisch sind, d.i. aus der Erfahrung müssen entlehnt werden, daß gleichwohl zur Idee der Glückseligkeit ein absolutes Ganze, ein Maximum des Wohlbefindens, in meinem gegenwärtigen und jedem zukünftigen Zustande erforderlich ist. Nun ist's unmöglich, daß das einsehendste und zugleich allervermögendste, aber doch endliche Wesen sich einen bestimmten Begriff von dem mache, was er hier eigentlich wolle. Will er Reichtum, wie viel Sorge, Neid und Nachstellung könnte er sich dadurch nicht auf den Hals ziehen. Will er viel Erkenntnis und Einsicht, vielleicht könnte das ein nur um desto schärferes Auge werden, um die Übel, die sich für ihn jetzt noch verbergen und doch nicht vermieden werden können, ihm nur  um desto schrecklicher zu zeigen, oder seinen Begierden, die ihm schon genug zu schaffen machen, noch mehr Bedürfnisse aufzubürden. Will er ein langes Leben, wer steht ihm dafür, daß es nicht ein langes Elend sein würde? Will er wenigstens Gesundheit, wie oft hat noch Ungemächlichkeit des Körpers von Ausschweifung abgehalten, darein unbeschränkte Gesundheit würde haben fallen lassen, u.s.w. Kurz, er ist nicht vermögend, nach irgend einem Grundsatze, mit völliger Gewißheit zu bestimmen, was ihn wahrhaftig glücklich machen werde, darum, weil hiezu Allwissenheit erforderlich sein würde. Man kann also nicht nach bestimmten Prinzipien handeln, um glücklich zu sein, sondern nur nach empirischen Ratschlägen, z.B. der Diät, der Sparsamkeit, der Höflichkeit, der Zurückhaltung u.s.w., von welchen die Erfahrung lehrt, daß sie das Wohlbefinden im Durchschnitt am meisten befördern. Hieraus folgt, daß die Imperativen der Klugheit, genau zu reden, gar nicht gebieten, d.i. Handlungen objektiv als praktisch-notwendig darstellen können, daß sie eher für Anratungen (consilia) als Gebote (praecepta) der Vernunft zu halten sind, daß die Aufgabe: sicher und allgemein zu bestimmen, welche Handlung die Glückseligkeit eines vernünftigen Wesens befördern werde, völlig unauflöslich, mithin kein Imperativ in Ansehung derselben möglich sei, der im strengen Verstande geböte, das zu tun, was glücklich macht, weil Glückseligkeit nicht ein Ideal der Vernunft, sondern der Einbildungskraft ist, was bloß auf empirischen Gründen beruht, von denen man vergeblich erwartet, daß sie eine Handlung bestimmen sollten, dadurch die Totalität einer in der Tat unendlichen Reihe von Folgen erreicht würde. Dieser Imperativ der Klugheit würde indessen, wenn man annimmt, die Mittel zur Glückseligkeit ließen sich sicher angeben, ein analytisch-praktischer Satz sein; denn er ist von dem Imperativ der Geschicklichkeit nur darin unterschieden, daß bei diesem der Zweck bloß möglich, bei jenem aber gegeben ist; da beide aber bloß die Mittel zu demjenigen gebieten, von dem man voraussetzt, daß man es als Zweck wollte: so ist der  Imperativ, der das Wollen der Mittel für den, der den Zweck will, gebietet, in beiden \match{Fällen} analytisch. Es ist also in Ansehung der Möglichkeit eines solchen Imperativs auch keine Schwierigkeit. 
	
	\unnumberedsection{Feld (2)} 
	\subsection*{tg174.2.11} 
	\textbf{Source : }Grundlegung zur Metaphysik der Sitten/Vorrede\\  
	
	\noindent\textbf{Paragraphe : }Man denke doch ja nicht, daß man das, was hier gefodert wird, schon an der Propädeutik des berühmten Wolff vor seiner Moralphilosophie, nämlich der von ihm so genannten allgemeinen praktischen Weltweisheit, habe, und hier also nicht eben ein ganz neues \match{Feld} einzuschlagen sei.  Eben darum, weil sie eine allgemeine praktische Weltweisheit sein sollte, hat sie keinen Willen von irgend einer besondern Art, etwa einen solchen, der ohne alle empirische Bewegungsgründe, völlig aus Prinzipien a priori, bestimmt werde, und den man einen reinen Willen nennen könnte, sondern das Wollen überhaupt in Betrachtung gezogen, mit allen Handlungen und Bedingungen, die ihm in dieser allgemeinen Bedeutung zukommen, und dadurch unterscheidet sie sich von einer Metaphysik der Sitten, eben so wie die allgemeine Logik von der Transzendentalphilosophie, von denen die erstere die Handlungen und Regeln des Denkens überhaupt, diese aber bloß die besondern Handlungen und Regeln des reinen Denkens, d.i. desjenigen, wodurch Gegenstände völlig a priori erkannt werden, vorträgt. Denn die Metaphysik der Sitten soll die Idee und die Prinzipien eines möglichen reinen Willens untersuchen, und nicht die Handlungen und Bedingungen des menschlichen Wollens überhaupt, welche größtenteils aus der Psychologie geschöpft werden. Daß in der all gemeinen praktischen Weltweisheit (wiewohl wider alle Befugnis) auch von moralischen Gesetzen und Pflicht geredet wird, macht keinen Einwurf wider meine Behauptung aus. Denn die Verfasser jener Wissenschaft bleiben ihrer Idee von derselben auch hierin treu; sie unterscheiden nicht die Bewegungsgründe, die, als solche, völlig a priori bloß durch Vernunft vorgestellt werden und eigentlich moralisch sind, von den empirischen, die der Verstand bloß durch Vergleichung der Erfahrungen zu allgemeinen Begriffen erhebt, sondern betrachten sie, ohne auf den Unterschied ihrer Quellen zu achten, nur nach der größeren oder kleineren Summe derselben (indem sie alle als gleichartig angesehen werden), und machen sich dadurch ihren Begriff von Verbindlichkeit, der freilich nichts weniger als moralisch, aber doch so beschaffen ist, als es in einer Philosophie, die über den Ursprung aller möglichen praktischen Begriffe, ob sie auch a priori oder bloß a posteriori stattfinden, gar nicht urteilt, nur verlangt werden kann. 
	
	\subsection*{tg175.2.24} 
	\textbf{Source : }Grundlegung zur Metaphysik der Sitten/Erster Abschnitt: Übergang von der gemeinen sittlichen Vernunfterkenntnis zur philosophischen\\  
	
	\noindent\textbf{Paragraphe : }So wird also die gemeine Menschenvernunft nicht durch irgend ein Bedürfnis der Spekulation (welches ihr, so lange sie sich genügt, bloße gesunde Vernunft zu sein, niemals anwandelt), sondern selbst aus praktischen Gründen  angetrieben, aus ihrem Kreise zu gehen, und einen Schritt ins \match{Feld} einer praktischen Philosophie zu tun, um daselbst, wegen der Quelle ihres Prinzips und richtigen Bestimmung desselben in Gegenhaltung mit den Maximen, die sich auf Bedürfnis und Neigung fußen, Erkundigung und deutliche Anweisung zu bekommen, damit sie aus der Verlegenheit wegen beiderseitiger Ansprüche herauskomme, und nicht Gefahr laufe, durch die Zweideutigkeit, in die sie leicht gerät, um alle echte sittliche Grundsätze gebracht zu werden. Also entspinnt sich eben sowohl in der praktischen gemeinen Vernunft, wenn sie sich kultiviert, unvermerkt eine Dialektik, welche sie nötigt, in der Philosophie Hülfe zu suchen, als es ihr im theoretischen Gebrauche widerfährt, und die erstere wird daher wohl eben so wenig, als die andere, irgendwo sonst, als in einer vollständigen Kritik unserer Vernunft, Ruhe finden. 
	
	\unnumberedsection{Leser (1)} 
	\subsection*{tg174.2.12} 
	\textbf{Source : }Grundlegung zur Metaphysik der Sitten/Vorrede\\  
	
	\noindent\textbf{Paragraphe : }Im Vorsatze nun, eine Metaphysik der Sitten dereinst zu liefern, lasse ich diese Grundlegung vorangehen. Zwar gibt  es eigentlich keine andere Grundlage derselben, als die Kritik einer reinen praktischen Vernunft, so wie zur Metaphysik die schon gelieferte Kritik der reinen spekulativen Vernunft. Allein, teils ist jene nicht von so äußerster Notwendigkeit, als diese, weil die menschliche Vernunft im Moralischen, selbst beim gemeinsten Verstande, leicht zu großer Richtigkeit und Ausführlichkeit gebracht werden kann, da sie hingegen im theoretischen, aber reinen Gebrauch ganz und gar dialektisch ist; teils erfodere ich zur Kritik einer reinen praktischen Vernunft, daß, wenn sie vollendet sein soll, ihre Einheit mit der spekulativen in einem gemeinschaftlichen Prinzip zugleich müsse dargestellt werden können, weil es doch am Ende nur eine und dieselbe Vernunft sein kann, die bloß in der Anwendung unterschieden sein muß. Zu einer solchen Vollständigkeit konnte ich es aber hier noch nicht bringen, ohne Betrachtungen von ganz anderer Art herbeizuziehen und den \match{Leser} zu verwirren. Um deswillen habe ich mich, statt der Benennung einer Kritik der reinen praktischen Vernunft, der von einer Grundlegung zur Metaphysik der Sitten bedient. 
	
	\unnumberedsection{Richtschnur (1)} 
	\subsection*{tg187.2.26} 
	\textbf{Source : }Grundlegung zur Metaphysik der Sitten/Fußnoten\\  
	
	\noindent\textbf{Paragraphe : }
	
	13 Man denke ja nicht, daß hier das triviale: quod tibi non vis fieri etc. zur \match{Richtschnur} oder Prinzip dienen könne. Denn es ist, obzwar mit verschiedenen Einschränkungen, nur aus jenem abgeleitet; es kann kein allgemeines Gesetz sein, denn es enthält nicht den Grund der Pflichten gegen sich selbst, nicht der Liebespflichten gegen andere (denn mancher würde es gerne eingehen, daß andere ihm nicht wohltun sollen, wenn er es nur überhoben sein dürfte, ihnen Wohltat zu erzeigen), endlich nicht der schuldigen Pflichten gegen einander; denn der Verbrecher würde aus diesem Grunde gegen seine strafenden Richter argumentieren, u.s.w. 
	
	\unnumberedsection{Vermehrung (1)} 
	\subsection*{tg176.2.80} 
	\textbf{Source : }Grundlegung zur Metaphysik der Sitten/Zweiter Abschnitt: Übergang von der populären sittlichen Weltweisheit zur Metaphysik der Sitten\\  
	
	\noindent\textbf{Paragraphe : }Nun folgt hieraus unstreitig: daß jedes vernünftige Wesen, als Zweck an sich selbst, sich in Ansehung aller Gesetze, denen es nur immer unterworfen sein mag, zugleich als allgemein gesetzgebend müsse ansehen können, weil eben diese Schicklichkeit seiner Maximen zur allgemeinen Gesetzgebung es als Zweck an sich selbst auszeichnet, imgleichen, daß dieses seine Würde (Prärogativ) vor allen bloßen Naturwesen es mit sich bringe, seine Maximen jederzeit aus dem Gesichtspunkte seiner selbst, zugleich aber auch jedes andern vernünftigen als gesetzgebenden Wesens (die darum auch Personen heißen), nehmen zu müssen. Nun ist auf solche Weise eine Welt vernünftiger Wesen (mundus intelligibilis) als ein Reich der Zwecke möglich, und zwar durch die eigene Gesetzgebung aller Personen als Glieder. Demnach muß ein jedes vernünftige Wesen so handeln, als ob es durch seine Maximen jederzeit ein gesetzgebendes Glied im allgemeinen Reiche der Zwecke wäre. Das formale Prinzip dieser Maximen ist: handle so, als ob deine Maxime zugleich zum allgemeinen Gesetze (aller vernünftigen Wesen) dienen sollte. Ein Reich der Zwecke ist also nur möglich nach der Analogie mit einem Reiche der Natur, jenes aber nur nach Maximen, d.i. sich selbst auferlegten Regeln, diese nur nach Gesetzen äußerlich genötigter wirkenden Ursachen. Demunerachtet gibt man doch auch dem Naturganzen, ob es schon als Maschine angesehen wird, dennoch, so fern es auf vernünftige Wesen, als seine Zwecke, Beziehung hat, aus diesem Grunde den Namen eines Reichs der Natur. Ein solches Reich der Zwecke würde nun durch Maximen, deren Regel der kategorische Imperativ aller vernünftigen Wesen vorschreibt, wirklich zu Stande kommen, wenn sie allgemein befolgt würden. Allein, obgleich das vernünftige Wesen darauf nicht rechnen kann, daß, wenn es auch gleich  diese Maxime selbst pünktlich befolgte, darum jedes andere eben derselben treu sein würde, imgleichen, daß das Reich der Natur und die zweckmäßige Anordnung desselben, mit ihm, als einem schicklichen Gliede, zu einem durch ihn selbst möglichen Reiche der Zwecke zusammenstimmen, d.i. seine Erwartung der Glückseligkeit begünstigen werde: so bleibt doch jenes Gesetz: handle nach Maximen eines allgemein gesetzgebenden Gliedes zu einem bloß möglichen Reiche der Zwecke, in seiner vollen Kraft, weil es kategorisch gebietend ist. Und hierin liegt eben das Paradoxon; daß bloß die Würde der Menschheit, als vernünftiger Natur, ohne irgend einen andern dadurch zu erreichenden Zweck, oder Vorteil, mithin die Achtung für eine bloße Idee, dennoch zur unnachlaßlichen Vorschrift des Willens dienen sollte, und daß gerade in dieser Unabhängigkeit der Maxime von allen solchen Triebfedern die Erhabenheit derselben bestehe, und die Würdigkeit eines jeden vernünftigen Subjekts, ein gesetzgebendes Glied im Reiche der Zwecke zu sein; denn sonst würde es nur als dem Naturgesetze seiner Bedürfnis unterworfen vorgestellt werden müssen. Obgleich auch das Naturreich sowohl, als das Reich der Zwecke, als unter einem Oberhaupte vereinigt gedacht würde, und dadurch das letztere nicht mehr bloße Idee bliebe, sondern wahre Realität erhielte, so würde hiedurch zwar jener der Zuwachs einer starken Triebfeder, niemals aber \match{Vermehrung} ihres innern Werts zu statten kommen; denn, diesem ungeachtet, müßte doch selbst dieser alleinige unumschränkte Gesetzgeber immer so vorgestellt werden, wie er den Wert der vernünftigen Wesen, nur nach ihrem uneigennützigen, bloß aus jener Idee ihnen selbst vorgeschriebenen Verhalten, beurteilte. Das Wesen der Dinge ändert sich durch ihre äußere Verhältnisse nicht, und was, ohne an das letztere zu denken, den absoluten Wert des Menschen allein ausmacht, darnach muß er auch, von wem es auch sei, selbst vom höchsten Wesen, beurteilt werden. Moralität ist also das Verhältnis der Handlungen zur Autonomie des Willens, das ist, zur möglichen allgemeinen Gesetzgebung durch die  Maximen desselben. Die Handlung, die mit der Autonomie des Willens zusammen bestehen kann, ist erlaubt; die nicht damit stimmt, ist unerlaubt. Der Wille, dessen Maximen notwendig mit den Gesetzen der Autonomie zusammenstimmen, ist ein heiliger, schlechterdings guter Wille. Die Abhängigkeit eines nicht schlechterdings guten Willens vom Prinzip der Autonomie (die moralische Nötigung) ist Verbindlichkeit. Diese kann also auf ein heiliges Wesen nicht gezogen werden. Die objektive Notwendigkeit einer Handlung aus Verbindlichkeit heißt Pflicht. 
	
	\unnumberedchapter{Alimentation} 
	\unnumberedsection{Alter (1)} 
	\subsection*{tg187.2.8} 
	\textbf{Source : }Grundlegung zur Metaphysik der Sitten/Fußnoten\\  
	
	\noindent\textbf{Paragraphe : }
	
	4 Ich habe einen Brief vom sel. vortrefflichen Sulzer, worin er mich frägt: was doch die Ursache sein möge, warum die Lehren der Tugend, so viel Überzeugendes sie auch für die Vernunft haben, doch so wenig ausrichten. Meine Antwort wurde durch die Zurüstung dazu, um sie vollständig zu geben, verspätet. Allein es ist keine andere, als daß die Lehrer selbst ihre Begriffe nicht ins Reine gebracht haben, und, indem sie es zu gut machen wollen, dadurch, daß sie allerwärts Bewegursachen zum Sittlichguten auftreiben, um die Arznei recht kräftig zu machen, sie sie verderben. Denn die gemeinste Beobachtung zeigt, daß, wenn man eine Handlung der Rechtschaffenheit vorstellt, wie sie von aller Absicht auf irgend einen Vorteil, in dieser oder einer andern Welt, abgesondert, selbst unter den größten Versuchungen der Not, oder der Anlockung, mit standhafter Seele ausgeübt worden, sie jede ähnliche Handlung, die nur im mindesten durch eine fremde Triebfeder affiziert war, weit hinter sich lasse und verdunkle, die Seele erhebe und den Wunsch errege, auch so handeln zu können. Selbst Kinder von mittlerem \match{Alter} fühlen diesen Eindruck, und ihnen sollte man Pflichten auch niemals anders vorstellen. 
	
	\unnumberedsection{Geschmack (1)} 
	\subsection*{tg175.2.12} 
	\textbf{Source : }Grundlegung zur Metaphysik der Sitten/Erster Abschnitt: Übergang von der gemeinen sittlichen Vernunfterkenntnis zur philosophischen\\  
	
	\noindent\textbf{Paragraphe : }Dagegen, sein Leben zu erhalten, ist Pflicht, und überdem hat jedermann dazu noch eine unmittelbare Neigung. Aber um deswillen hat die oft ängstliche Sorgfalt, die der größte Teil der Menschen dafür trägt, doch keinen innern Wert, und die Maxime derselben keinen moralischen Gehalt. Sie bewahren ihr Leben zwar pflichtmäßig, aber nicht aus Pflicht. Dagegen, wenn Widerwärtigkeiten und hoffnungsloser Gram den \match{Geschmack} am Leben gänzlich weggenommen haben; wenn der Unglückliche, stark an Seele, über sein Schicksal mehr entrüstet, als kleinmütig oder niedergeschlagen, den Tod wünscht, und sein Leben doch erhält, ohne es zu lieben, nicht aus Neigung, oder Furcht, sondern aus Pflicht: alsdenn hat seine Maxime einen moralischen Gehalt. 
	
	\unnumberedsection{Hals (2)} 
	\subsection*{tg175.2.8} 
	\textbf{Source : }Grundlegung zur Metaphysik der Sitten/Erster Abschnitt: Übergang von der gemeinen sittlichen Vernunfterkenntnis zur philosophischen\\  
	
	\noindent\textbf{Paragraphe : }In der Tat finden wir auch, daß, je mehr eine kultivierte Vernunft sich mit der Absicht auf den Genuß des Lebens  und der Glückseligkeit abgibt, desto weiter der Mensch von der wahren Zufriedenheit abkomme, woraus bei vielen, und zwar den Versuchtesten im Gebrauche derselben, wenn sie nur aufrichtig genug sind, es zu gestehen, ein gewisser Grad von Misologie, d.i. Haß der Vernunft entspringt, weil sie nach dem Überschlage alles Vorteils, den sie, ich will nicht sagen von der Erfindung aller Künste des gemeinen Luxus, sondern so gar von den Wissenschaften (die ihnen am Ende auch ein Luxus des Verstandes zu sein scheinen) ziehen, dennoch finden, daß sie sich in der Tat nur mehr Mühseligkeit auf den \match{Hals} gezogen, als an Glückseligkeit gewonnen haben, und darüber endlich den gemeinern Schlag der Menschen, welcher der Leitung des bloßen Naturinstinkts näher ist, und der seiner Vernunft nicht viel Einfluß auf sein Tun und Lassen verstattet, eher beneiden, als geringschätzen. Und so weit muß man gestehen, daß das Urteil derer, die die ruhmredige Hochpreisungen der Vorteile, die uns die Vernunft in Ansehung der Glückseligkeit und Zufriedenheit des Lebens verschaffen sollte, sehr mäßigen und sogar unter Null herabsetzen, keinesweges grämisch, oder gegen die Güte der Weltregierung undankbar sei, sondern daß diesen Urteilen ingeheim die Idee von einer andern und viel würdigern Absicht ihrer Existenz zum Grunde liege, zu welcher, und nicht der Glückseligkeit, die Vernunft ganz eigentlich bestimmt sei, und welcher darum, als oberster Bedingung, die Privatabsicht des Menschen größtenteils nachstehen muß. 
	
	\subsection*{tg176.2.27} 
	\textbf{Source : }Grundlegung zur Metaphysik der Sitten/Zweiter Abschnitt: Übergang von der populären sittlichen Weltweisheit zur Metaphysik der Sitten\\  
	
	\noindent\textbf{Paragraphe : }Die Imperativen der Klugheit würden, wenn es nur so leicht wäre, einen bestimmten Begriff von Glückseligkeit zu geben, mit denen der Geschicklichkeit ganz und gar übereinkommen und eben sowohl analytisch sein. Denn es würde eben sowohl hier, als dort, heißen: wer den Zweck will, will auch (der Vernunft gemäß notwendig) die einzigen Mittel, die dazu in seiner Gewalt sind. Allein es ist ein Unglück, daß der Begriff der Glückseligkeit ein so unbestimmter Begriff ist, daß, obgleich jeder Mensch zu dieser zu gelangen wünscht, er doch niemals bestimmt und mit sich selbst einstimmig sagen kann, was er eigentlich wünsche und wolle. Die Ursache davon ist: daß alle Elemente, die zum Begriff der Glückseligkeit gehören, insgesamt empirisch sind, d.i. aus der Erfahrung müssen entlehnt werden, daß gleichwohl zur Idee der Glückseligkeit ein absolutes Ganze, ein Maximum des Wohlbefindens, in meinem gegenwärtigen und jedem zukünftigen Zustande erforderlich ist. Nun ist's unmöglich, daß das einsehendste und zugleich allervermögendste, aber doch endliche Wesen sich einen bestimmten Begriff von dem mache, was er hier eigentlich wolle. Will er Reichtum, wie viel Sorge, Neid und Nachstellung könnte er sich dadurch nicht auf den \match{Hals} ziehen. Will er viel Erkenntnis und Einsicht, vielleicht könnte das ein nur um desto schärferes Auge werden, um die Übel, die sich für ihn jetzt noch verbergen und doch nicht vermieden werden können, ihm nur  um desto schrecklicher zu zeigen, oder seinen Begierden, die ihm schon genug zu schaffen machen, noch mehr Bedürfnisse aufzubürden. Will er ein langes Leben, wer steht ihm dafür, daß es nicht ein langes Elend sein würde? Will er wenigstens Gesundheit, wie oft hat noch Ungemächlichkeit des Körpers von Ausschweifung abgehalten, darein unbeschränkte Gesundheit würde haben fallen lassen, u.s.w. Kurz, er ist nicht vermögend, nach irgend einem Grundsatze, mit völliger Gewißheit zu bestimmen, was ihn wahrhaftig glücklich machen werde, darum, weil hiezu Allwissenheit erforderlich sein würde. Man kann also nicht nach bestimmten Prinzipien handeln, um glücklich zu sein, sondern nur nach empirischen Ratschlägen, z.B. der Diät, der Sparsamkeit, der Höflichkeit, der Zurückhaltung u.s.w., von welchen die Erfahrung lehrt, daß sie das Wohlbefinden im Durchschnitt am meisten befördern. Hieraus folgt, daß die Imperativen der Klugheit, genau zu reden, gar nicht gebieten, d.i. Handlungen objektiv als praktisch-notwendig darstellen können, daß sie eher für Anratungen (consilia) als Gebote (praecepta) der Vernunft zu halten sind, daß die Aufgabe: sicher und allgemein zu bestimmen, welche Handlung die Glückseligkeit eines vernünftigen Wesens befördern werde, völlig unauflöslich, mithin kein Imperativ in Ansehung derselben möglich sei, der im strengen Verstande geböte, das zu tun, was glücklich macht, weil Glückseligkeit nicht ein Ideal der Vernunft, sondern der Einbildungskraft ist, was bloß auf empirischen Gründen beruht, von denen man vergeblich erwartet, daß sie eine Handlung bestimmen sollten, dadurch die Totalität einer in der Tat unendlichen Reihe von Folgen erreicht würde. Dieser Imperativ der Klugheit würde indessen, wenn man annimmt, die Mittel zur Glückseligkeit ließen sich sicher angeben, ein analytisch-praktischer Satz sein; denn er ist von dem Imperativ der Geschicklichkeit nur darin unterschieden, daß bei diesem der Zweck bloß möglich, bei jenem aber gegeben ist; da beide aber bloß die Mittel zu demjenigen gebieten, von dem man voraussetzt, daß man es als Zweck wollte: so ist der  Imperativ, der das Wollen der Mittel für den, der den Zweck will, gebietet, in beiden Fällen analytisch. Es ist also in Ansehung der Möglichkeit eines solchen Imperativs auch keine Schwierigkeit. 
	
	\unnumberedsection{Herz (2)} 
	\subsection*{tg175.2.13} 
	\textbf{Source : }Grundlegung zur Metaphysik der Sitten/Erster Abschnitt: Übergang von der gemeinen sittlichen Vernunfterkenntnis zur philosophischen\\  
	
	\noindent\textbf{Paragraphe : }
	Wohltätig sein, wo man kann, ist Pflicht, und überdem gibt es manche so teilnehmend gestimmte Seelen, daß sie, auch ohne einen andern Bewegungsgrund der Eitelkeit, oder des Eigennutzes, ein inneres Vergnügen daran finden, Freude um sich zu verbreiten, und die sich an der Zufriedenheit anderer, so fern sie ihr Werk ist, ergötzen können. Aber ich behaupte, daß in solchem Falle dergleichen Handlung, so pflichtmäßig, so liebenswürdig sie auch ist, dennoch keinen wahren sittlichen Wert habe, sondern mit andern Neigungen zu gleichen Paaren gehe, z. E. der Neigung nach Ehre, die, wenn sie glücklicherweise auf das trifft, was in der Tat gemeinnützig und pflichtmäßig, mithin ehrenwert ist, Lob und Aufmunterung, aber nicht Hochschätzung verdient; denn der Maxime fehlt der sittliche Gehalt, nämlich solche Handlungen nicht aus Neigung, sondern aus Pflicht zu tun. Gesetzt also, das Gemüt jenes Menschenfreundes wäre vom eigenen Gram umwölkt, der alle Teilnehmung an anderer Schicksal auslöscht, er hätte immer noch Vermögen, andern Notleidenden wohlzutun, aber fremde Not rührte ihn nicht, weil er mit seiner eigenen gnug beschäftigt ist, und nun, da keine Neigung ihn mehr dazu anreizt, risse er sich doch aus dieser tödlichen Unempfindlichkeit heraus, und täte die Handlung ohne alle Neigung, lediglich aus Pflicht, alsdenn hat sie allererst ihren echten moralischen Wert. Noch mehr: wenn die Natur diesem oder jenem überhaupt wenig Sympathie ins \match{Herz} gelegt hätte, wenn er (übrigens ein ehrlicher Mann) von Temperament kalt und gleichgültig gegen die Leiden anderer wäre, vielleicht, weil er, selbst gegen seine eigene mit der besondern Gabe der Geduld und aushaltenden Stärke versehen, dergleichen bei jedem andern auch voraussetzt, oder gar fordert; wenn die Natur einen solchen Mann (welcher wahrlich nicht ihr schlechtestes Produkt sein würde) nicht eigentlich zum Menschenfreunde gebildet hätte, würde er denn nicht noch in sich einen Quell finden, sich selbst einen weit höhern Wert zu geben, als der eines gutartigen Temperaments sein mag? Allerdings! gerade da hebt der Wert des Charakters an, der  moralisch und ohne alle Vergleichung der höchste ist, nämlich daß er wohltue, nicht aus Neigung, sondern aus Pflicht. 
	
	\subsection*{tg176.2.11} 
	\textbf{Source : }Grundlegung zur Metaphysik der Sitten/Zweiter Abschnitt: Übergang von der populären sittlichen Weltweisheit zur Metaphysik der Sitten\\  
	
	\noindent\textbf{Paragraphe : }Es ist aber eine solche völlig isolierte Metaphysik der Sitten, die mit keiner Anthropologie, mit keiner Theologie, mit keiner Physik, oder Hyperphysik, noch weniger mit verborgenen Qualitäten (die man hypophysisch nennen könnte) vermischt ist, nicht allein ein unentbehrliches Substrat aller theoretischen sicher bestimmten Erkenntnis der  Pflichten, sondern zugleich ein Desiderat von der höchsten Wichtigkeit zur wirklichen Vollziehung ihrer Vorschriften. Denn die reine und mit keinem fremden Zusatze von empirischen Anreizen vermischte Vorstellung der Pflicht, und überhaupt des sittlichen Gesetzes, hat auf das menschliche \match{Herz} durch den Weg der Vernunft allein (die hiebei zuerst inne wird, daß sie für sich selbst auch praktisch sein kann) einen so viel mächtigern Einfluß, als alle andere Triebfedern
	
	
	4
	, die man aus dem empirischen Felde aufbieten mag, daß sie im Bewußtsein ihrer Würde die letzteren verachtet, und nach und nach ihr Meister werden kann; an dessen Statt eine vermischte Sittenlehre, die aus Triebfedern von Gefühlen und Neigungen und zugleich aus Vernunftbegriffen zusammengesetzt ist, das Gemüt zwischen Bewegursachen, die sich unter kein Prinzip bringen lassen, die nur sehr zufällig zum Guten, öfters aber auch zum Bösen leiten können, schwankend machen muß. 
	
	\unnumberedchapter{Botanique} 
	\unnumberedsection{Abteilung (1)} 
	\subsection*{tg176.2.41} 
	\textbf{Source : }Grundlegung zur Metaphysik der Sitten/Zweiter Abschnitt: Übergang von der populären sittlichen Weltweisheit zur Metaphysik der Sitten\\  
	
	\noindent\textbf{Paragraphe : }Dieses sind nun einige von den vielen wirklichen oder wenigstens von uns dafür gehaltenen Pflichten, deren \match{Abteilung} aus dem einigen angeführten Prinzip klar in die Augen fällt. Man muß wollen können, daß eine Maxime unserer Handlung ein allgemeines Gesetz werde: dies ist der Kanon der moralischen Beurteilung derselben überhaupt. Einige Handlungen sind so beschaffen, daß ihre Maxime ohne Widerspruch nicht einmal als allgemeines Naturgesetz 
	gedacht werden kann; weit gefehlt, daß man noch wollen könne, es sollte ein solches werden. Bei andern ist zwar jene innere Unmöglichkeit nicht anzutreffen, aber es ist doch unmöglich, zu wollen, daß ihre Maxime zur Allgemeinheit eines Naturgesetzes erhoben werde, weil ein solcher Wille sich selbst widersprechen würde. Man sieht leicht: daß die erstere der strengen oder engeren (unnachlaßlichen) Pflicht, die zweite nur der weiteren (verdienstlichen) Pflicht widerstreite, und so alle Pflichten, was die Art der Verbindlichkeit (nicht das Objekt ihrer Handlung) betrifft, durch diese Beispiele in ihrer Abhängigkeit von dem einigen Prinzip vollständig aufgestellt worden. 
	
	\unnumberedsection{Auge (1)} 
	\subsection*{tg176.2.27} 
	\textbf{Source : }Grundlegung zur Metaphysik der Sitten/Zweiter Abschnitt: Übergang von der populären sittlichen Weltweisheit zur Metaphysik der Sitten\\  
	
	\noindent\textbf{Paragraphe : }Die Imperativen der Klugheit würden, wenn es nur so leicht wäre, einen bestimmten Begriff von Glückseligkeit zu geben, mit denen der Geschicklichkeit ganz und gar übereinkommen und eben sowohl analytisch sein. Denn es würde eben sowohl hier, als dort, heißen: wer den Zweck will, will auch (der Vernunft gemäß notwendig) die einzigen Mittel, die dazu in seiner Gewalt sind. Allein es ist ein Unglück, daß der Begriff der Glückseligkeit ein so unbestimmter Begriff ist, daß, obgleich jeder Mensch zu dieser zu gelangen wünscht, er doch niemals bestimmt und mit sich selbst einstimmig sagen kann, was er eigentlich wünsche und wolle. Die Ursache davon ist: daß alle Elemente, die zum Begriff der Glückseligkeit gehören, insgesamt empirisch sind, d.i. aus der Erfahrung müssen entlehnt werden, daß gleichwohl zur Idee der Glückseligkeit ein absolutes Ganze, ein Maximum des Wohlbefindens, in meinem gegenwärtigen und jedem zukünftigen Zustande erforderlich ist. Nun ist's unmöglich, daß das einsehendste und zugleich allervermögendste, aber doch endliche Wesen sich einen bestimmten Begriff von dem mache, was er hier eigentlich wolle. Will er Reichtum, wie viel Sorge, Neid und Nachstellung könnte er sich dadurch nicht auf den Hals ziehen. Will er viel Erkenntnis und Einsicht, vielleicht könnte das ein nur um desto schärferes \match{Auge} werden, um die Übel, die sich für ihn jetzt noch verbergen und doch nicht vermieden werden können, ihm nur  um desto schrecklicher zu zeigen, oder seinen Begierden, die ihm schon genug zu schaffen machen, noch mehr Bedürfnisse aufzubürden. Will er ein langes Leben, wer steht ihm dafür, daß es nicht ein langes Elend sein würde? Will er wenigstens Gesundheit, wie oft hat noch Ungemächlichkeit des Körpers von Ausschweifung abgehalten, darein unbeschränkte Gesundheit würde haben fallen lassen, u.s.w. Kurz, er ist nicht vermögend, nach irgend einem Grundsatze, mit völliger Gewißheit zu bestimmen, was ihn wahrhaftig glücklich machen werde, darum, weil hiezu Allwissenheit erforderlich sein würde. Man kann also nicht nach bestimmten Prinzipien handeln, um glücklich zu sein, sondern nur nach empirischen Ratschlägen, z.B. der Diät, der Sparsamkeit, der Höflichkeit, der Zurückhaltung u.s.w., von welchen die Erfahrung lehrt, daß sie das Wohlbefinden im Durchschnitt am meisten befördern. Hieraus folgt, daß die Imperativen der Klugheit, genau zu reden, gar nicht gebieten, d.i. Handlungen objektiv als praktisch-notwendig darstellen können, daß sie eher für Anratungen (consilia) als Gebote (praecepta) der Vernunft zu halten sind, daß die Aufgabe: sicher und allgemein zu bestimmen, welche Handlung die Glückseligkeit eines vernünftigen Wesens befördern werde, völlig unauflöslich, mithin kein Imperativ in Ansehung derselben möglich sei, der im strengen Verstande geböte, das zu tun, was glücklich macht, weil Glückseligkeit nicht ein Ideal der Vernunft, sondern der Einbildungskraft ist, was bloß auf empirischen Gründen beruht, von denen man vergeblich erwartet, daß sie eine Handlung bestimmen sollten, dadurch die Totalität einer in der Tat unendlichen Reihe von Folgen erreicht würde. Dieser Imperativ der Klugheit würde indessen, wenn man annimmt, die Mittel zur Glückseligkeit ließen sich sicher angeben, ein analytisch-praktischer Satz sein; denn er ist von dem Imperativ der Geschicklichkeit nur darin unterschieden, daß bei diesem der Zweck bloß möglich, bei jenem aber gegeben ist; da beide aber bloß die Mittel zu demjenigen gebieten, von dem man voraussetzt, daß man es als Zweck wollte: so ist der  Imperativ, der das Wollen der Mittel für den, der den Zweck will, gebietet, in beiden Fällen analytisch. Es ist also in Ansehung der Möglichkeit eines solchen Imperativs auch keine Schwierigkeit. 
	
	\unnumberedsection{Ordnung (5)} 
	\subsection*{tg176.2.25} 
	\textbf{Source : }Grundlegung zur Metaphysik der Sitten/Zweiter Abschnitt: Übergang von der populären sittlichen Weltweisheit zur Metaphysik der Sitten\\  
	
	\noindent\textbf{Paragraphe : }Das Wollen nach diesen dreierlei Prinzipien wird auch durch die Ungleichheit der Nötigung des Willens deutlich unterschieden. Um diese nun auch merklich zu machen, glaube ich, daß man sie in ihrer \match{Ordnung} am angemessensten so benennen würde, wenn man sagte: sie wären entweder Regeln der Geschicklichkeit, oder Ratschläge der  Klugheit, oder Gebote (Gesetze) der Sittlichkeit. Denn nur das Gesetz führt den Begriff einer unbedingten und zwar objektiven und mithin allgemein gültigen Notwendigkeit bei sich, und Gebote sind Gesetze, denen gehorcht, d.i. auch wider Neigung Folge geleistet werden muß. Die Ratgebung enthält zwar Notwendigkeit, die aber bloß unter subjektiver gefälliger Bedingung, ob dieser oder jener Mensch dieses oder jenes zu seiner Glückseligkeit zähle, gelten kann; dagegen der kategorische Imperativ durch keine Bedingung eingeschränkt wird, und als absolut- obgleich praktisch-notwendig ganz eigentlich ein Gebot heißen kann. Man könnte die ersteren Imperative auch technisch (zur Kunst gehörig), die zweiten pragmatisch
	
	
	
	7
	(zur Wohlfahrt), die dritten moralisch (zum freien Verhalten überhaupt, d.i. zu den Sitten gehörig) nennen. 
	
	\subsection*{tg183.2.6} 
	\textbf{Source : }Grundlegung zur Metaphysik der Sitten/Dritter Abschnitt: Übergang von der Metaphysik der Sitten zur Kritik der reinen praktischen Vernunft/Von dem Interesse, welches den Ideen der Sittlichkeit anhängt\\  
	
	\noindent\textbf{Paragraphe : }Es zeigt sich hier, man muß es frei gestehen, eine Art von Zirkel, aus dem, wie es scheint, nicht heraus zu kommen ist. Wir nehmen uns in der \match{Ordnung} der wirkenden Ursachen  als frei an, um uns in der Ordnung der Zwecke unter sittlichen Gesetzen zu denken, und wir denken uns nachher als diesen Gesetzen unterworfen, weil wir uns die Freiheit des Willens beigelegt haben, denn Freiheit und eigene Gesetzgebung des Willens sind beides Autonomie, mithin Wechselbegriffe, davon aber einer eben um deswillen nicht dazu gebraucht werden kann, um den anderen zu erklären und von ihm Grund anzugeben, sondern höchstens nur, um, in logischer Absicht, verschieden scheinende Vorstellungen von eben demselben Gegenstande auf einen einzigen Begriff (wie verschiedne Brüche gleiches Inhalts auf die kleinsten Ausdrücke) zu bringen. 
	
	\subsection*{tg184.2.4} 
	\textbf{Source : }Grundlegung zur Metaphysik der Sitten/Dritter Abschnitt: Übergang von der Metaphysik der Sitten zur Kritik der reinen praktischen Vernunft/Wie ist ein kategorischer Imperativ möglich\\  
	
	\noindent\textbf{Paragraphe : }Der praktische Gebrauch der gemeinen Menschenvernunft bestätigt die Richtigkeit dieser Deduktion. Es ist niemand, selbst der ärgste Bösewicht, wenn er nur sonst Vernunft  zu brauchen gewohnt ist, der nicht, wenn man ihm Beispiele der Redlichkeit in Absichten, der Standhaftigkeit in Befolgung guter Maximen, der Teilnehmung und des allgemeinen Wohlwollens (und noch dazu mit großen Aufopferungen von Vorteilen und Gemächlichkeit verbunden) vorlegt, nicht wünsche, daß er auch so gesinnt sein möchte. Er kann es aber nur wegen seiner Neigungen und Antriebe nicht wohl in sich zu Stande bringen; wobei er dennoch zugleich wünscht, von solchen ihm selbst lästigen Neigungen frei zu sein. Er beweiset hiedurch also, daß er mit einem Willen, der von Antrieben der Sinnlichkeit frei ist, sich in Gedanken in eine ganz andere \match{Ordnung} der Dinge versetze, als die seiner Begierden im Felde der Sinnlichkeit, weil er von jenem Wunsche keine Vergnügung der Begierden, mithin keinen für irgend eine seiner wirklichen oder sonst erdenklichen Neigungen befriedigenden Zustand (denn dadurch würde selbst die Idee, welche ihm den Wunsch ablockt, ihre Vorzüglichkeit einbüßen), sondern nur einen größeren inneren Wert seiner Person erwarten kann. Diese bessere Person glaubt er aber zu sein, wenn er sich in den Standpunkt eines Gliedes der Verstandeswelt versetzt, dazu die Idee der Freiheit, d.i. Unabhängigkeit von bestimmenden Ursachen der Sinnenwelt ihn unwillkürlich nötigt, und in welchem er sich eines guten Willens bewußt ist, der für seinen bösen Willen, als Gliedes der Sinnenwelt, nach seinem eigenen Geständnisse das Gesetz ausmacht, dessen Ansehen er kennt, indem er es übertritt. Das moralische Sollen ist also eigenes notwendiges Wollen als Gliedes einer intelligibelen Welt, und wird nur so fern von ihm als Sollen gedacht, als er sich zugleich wie ein Glied der Sinnenwelt betrachtet. 
	
	\subsection*{tg185.2.7} 
	\textbf{Source : }Grundlegung zur Metaphysik der Sitten/Dritter Abschnitt: Übergang von der Metaphysik der Sitten zur Kritik der reinen praktischen Vernunft/Von der äußersten Grenze aller praktischen Philosophie\\  
	
	\noindent\textbf{Paragraphe : }
	Der Rechtsanspruch aber, selbst der gemeinen Menschenvernunft, auf Freiheit des Willens, gründet sich auf das Bewußtsein und die zugestandene Voraussetzung der Unabhängigkeit der Vernunft, von bloß subjektiv-bestimmten Ursachen, die insgesamt das ausmachen, was bloß zur Empfindung, mithin unter die allgemeine Benennung der Sinnlichkeit, gehört. Der Mensch, der sich auf solche Weise als Intelligenz betrachtet, setzt sich dadurch in eine andere \match{Ordnung} der Dinge und in ein Verhältnis zu bestimmenden Gründen von ganz anderer Art, wenn er sich als Intelligenz mit einem Willen, folglich mit Kausalität begabt, denkt, als wenn er sich wie Phänomen in der Sinnenwelt (welches er wirklich auch ist) wahrnimmt, und seine Kausalität, äußerer Bestimmung nach, Naturgesetzen unterwirft. Nun wird er bald inne, daß beides zugleich stattfinden könne, ja sogar müsse. Denn, daß ein Ding in der Erscheinung (das zur Sinnenwelt gehörig) gewissen Gesetzen unterworfen ist, von welchen eben dasselbe, als Ding oder Wesen an sich selbst, unabhängig ist, enthält nicht den mindesten Widerspruch; daß er sich selbst aber auf diese zwiefache Art vorstellen und denken müsse, beruht, was das erste betrifft, auf dem Bewußtsein seiner selbst als durch Sinne affizierten Gegenstandes, was das zweite anlangt, auf dem Bewußtsein seiner selbst als Intelligenz, d.i. als unabhängig im Vernunftgebrauch von sinnlichen Eindrücken (mithin als zur Verstandeswelt gehörig). 
	
	\subsection*{tg185.2.9} 
	\textbf{Source : }Grundlegung zur Metaphysik der Sitten/Dritter Abschnitt: Übergang von der Metaphysik der Sitten zur Kritik der reinen praktischen Vernunft/Von der äußersten Grenze aller praktischen Philosophie\\  
	
	\noindent\textbf{Paragraphe : }Dadurch, daß die praktische Vernunft sich in eine Verstandeswelt hinein denkt, überschreitet sie gar nicht ihre Grenzen, wohl aber, wenn sie sich hineinschauen, hineinempfinden wollte. Jenes ist nur ein negativer Gedanke, in Ansehung der Sinnenwelt, die der Vernunft in Bestimmung des Willens keine Gesetze gibt, und nur in diesem einzigen Punkte positiv, daß jene Freiheit, als negative Bestimmung, zugleich mit einem (positiven) Vermögen und sogar mit einer Kausalität der Vernunft verbunden sei, welche wir einen Willen nennen, so zu handeln, daß das Prinzip der Handlungen der wesentlichen Beschaffenheit einer Vernunftursache, d.i. der Bedingung der Allgemeingültigkeit der Maxime, als eines Gesetzes, gemäß sei. Würde sie aber noch ein Objekt des Willens, d.i. eine Bewegursache aus der Verstandeswelt herholen, so überschritte sie ihre Grenzen, und maßte sich an, etwas zu kennen, wovon sie nichts weiß. Der Begriff einer Verstandeswelt ist also nur ein Standpunkt, den die Vernunft sich genötigt sieht außer den Erscheinungen zu nehmen, um sich selbst als praktisch zu denken, welches, wenn die Einflüsse der Sinnlichkeit für den Menschen bestimmend wären, nicht möglich sein würde, welches aber doch notwendig ist, wofern ihm nicht das Bewußtsein seiner selbst, als Intelligenz, mithin als vernünftige und durch Vernunft tätige, d.i. frei wirkende Ursache, abgesprochen werden soll. Dieser Gedanke  führt freilich die Idee einer anderen \match{Ordnung} und Gesetzgebung, als die des Naturmechanismus, der die Sinnenwelt trifft, herbei, und macht den Begriff einer intelligibelen Welt (d.i. das Ganze vernünftiger Wesen, als Dinge an sich selbst) notwendig, aber ohne die mindeste Anmaßung, hier weiter, als bloß ihrer formalen Bedingung nach, d.i. der Allgemeinheit der Maxime des Willens, als Gesetze, mithin der Autonomie des letzteren, die allein mit der Freiheit desselben bestehen kann, gemäß zu denken; da hingegen alle Gesetze, die auf ein Objekt bestimmt sind, Heteronomie geben, die nur an Naturgesetzen angetroffen werden und auch nur die Sinnenwelt treffen kann. 
	
	\unnumberedchapter{Monde} 
	\unnumberedsection{Dunkelheit (1)} 
	\subsection*{tg175.2.22} 
	\textbf{Source : }Grundlegung zur Metaphysik der Sitten/Erster Abschnitt: Übergang von der gemeinen sittlichen Vernunfterkenntnis zur philosophischen\\  
	
	\noindent\textbf{Paragraphe : }So sind wir denn in der moralischen Erkenntnis der gemeinen Menschenvernunft bis zu ihrem Prinzip gelangt, welches sie sich zwar freilich nicht so in einer allgemeinen  Form abgesondert denkt, aber doch jederzeit wirklich vor Augen hat und zum Richtmaße ihrer Beurteilung braucht. Es wäre hier leicht zu zeigen, wie sie, mit diesem Kompasse in der Hand, in allen vorkommenden Fällen sehr gut Bescheid wisse, zu unterscheiden, was gut, was böse, pflichtmäßig, oder pflichtwidrig sei, wenn man, ohne sie im mindesten etwas Neues zu lehren, sie nur, wie Sokrates tat, auf ihr eigenes Prinzip aufmerksam macht, und daß es also keiner Wissenschaft und Philosophie bedürfe, um zu wissen, was man zu tun habe, um ehrlich und gut, ja sogar, um weise und tugendhaft zu sein. Das ließe sich auch wohl schon zum voraus vermuten, daß die Kenntnis dessen, was zu tun, mithin auch zu wissen jedem Menschen obliegt, auch jedes, selbst des gemeinsten Menschen Sache sein werde. Hier kann man es doch nicht ohne Bewunderung ansehen, wie das praktische Beurteilungsvermögen vor dem theoretischen im gemeinen Menschenverstande so gar viel voraus habe. In dem letzteren, wenn die gemeine Vernunft es wagt, von den Erfahrungsgesetzen und den Wahrnehmungen der Sinne abzugehen, gerät sie in lauter Unbegreiflichkeiten und Widersprüche mit sich selbst, wenigstens in ein Chaos von Ungewißheit, \match{Dunkelheit} und Unbestand. Im praktischen aber fängt die Beurteilungskraft denn eben allererst an, sich recht vorteilhaft zu zeigen, wenn der gemeine Verstand alle sinnliche Triebfedern von praktischen Gesetzen ausschließt. Er wird alsdenn so gar subtil, es mag sein, daß er mit seinem Gewissen, oder anderen Ansprüchen in Beziehung auf das, was recht heißen soll, schikanieren, oder auch den Wert der Handlungen zu seiner eigenen Belehrung aufrichtig bestimmen will, und, was das meiste ist, er kann im letzteren Falle sich eben so gut Hoffnung machen, es recht zu treffen, als es sich immer ein Philosoph versprechen mag, ja ist beinahe noch sicherer hierin, als selbst der letztere, weil dieser doch kein anderes Prinzip als jener haben, sein Urteil aber, durch eine Menge fremder, nicht zur Sache gehöriger Erwägungen, leicht verwirren und von der geraden Richtung abweichend machen kann. Wäre es demnach nicht ratsamer,  es in moralischen Dingen bei dem gemeinen Vernunfturteil bewenden zu lassen, und höchstens nur Philosophie anzubringen, um das System der Sitten desto vollständiger und faßlicher, imgleichen die Regeln derselben zum Gebrauche (noch mehr aber zum Disputieren) bequemer darzustellen, nicht aber, um selbst in praktischer Absicht den gemeinen Menschenverstand von seiner glücklichen Einfalt abzubringen, und ihn durch Philosophie auf einen neuen Weg der Untersuchung und Belehrung zu bringen? 
	
	\unnumberedsection{Erhebung (1)} 
	\subsection*{tg176.2.9} 
	\textbf{Source : }Grundlegung zur Metaphysik der Sitten/Zweiter Abschnitt: Übergang von der populären sittlichen Weltweisheit zur Metaphysik der Sitten\\  
	
	\noindent\textbf{Paragraphe : }Diese Herablassung zu Volksbegriffen ist allerdings sehr rühmlich, wenn die \match{Erhebung} zu den Prinzipien der reinen Vernunft zuvor geschehen und zur völligen Befriedigung erreicht ist, und das würde heißen, die Lehre der Sitten zuvor auf Metaphysik gründen, ihr aber, wenn sie fest steht, nachher durch Popularität Eingang verschaffen. Es ist aber äußerst ungereimt, dieser in der ersten Untersuchung, worauf alle Richtigkeit der Grundsätze ankommt, schon willfahren zu wollen. Nicht allein, daß dieses Verfahren auf das höchst seltene Verdienst einer wahren philosophischen Popularität niemals Anspruch machen kann, indem es gar keine Kunst ist, gemeinverständlich zu sein, wenn man dabei auf alle gründliche Einsicht Verzicht tut: so bringt es einen ekelhaften Mischmasch von zusammengestoppelten Beobachtungen und halbvernünftelnden Prinzipien zum Vorschein, daran sich schale Köpfe laben, weil es doch etwas gar Brauchbares fürs alltägliche Geschwätz ist, wo Einsehende aber Verwirrung fühlen, und unzufrieden, ohne sich doch helfen zu können, ihre Augen wegwenden, obgleich Philosophen, die das Blendwerk ganz wohl durchschauen, wenig  Gehör finden, wenn sie auf einige Zeit von der vorgeblichen Popularität abrufen, um nur allererst nach erworbener bestimmter Einsicht mit Recht populär sein zu dürfen. 
	
	\unnumberedsection{Gebiet (1)} 
	\subsection*{tg176.2.47} 
	\textbf{Source : }Grundlegung zur Metaphysik der Sitten/Zweiter Abschnitt: Übergang von der populären sittlichen Weltweisheit zur Metaphysik der Sitten\\  
	
	\noindent\textbf{Paragraphe : }Die Frage ist also diese: ist es ein notwendiges Gesetz für alle vernünftige Wesen, ihre Handlungen jederzeit nach solchen Maximen zu beurteilen, von denen sie selbst wollen können, daß sie zu allgemeinen Gesetzen dienen sollen? Wenn es ein solches ist, so muß es (völlig a priori) schon mit dem Begriffe des Willens eines vernünftigen Wesens überhaupt verbunden sein. Um aber diese Verknüpfung zu entdecken, muß man, so sehr man sich auch sträubt, einen Schritt hinaus tun, nämlich zur Metaphysik, obgleich in ein \match{Gebiet} derselben, welches von dem der spekulativen Philosophie unterschieden ist, nämlich in die Metaphysik der Sitten. In einer praktischen Philosophie, wo es uns nicht darum zu tun ist, Gründe anzunehmen, von dem, was geschieht, sondern Gesetze von dem, was geschehen soll, ob es gleich niemals geschieht, d.i. objektiv-praktische Gesetze: da haben wir nicht nötig, über die Gründe Untersuchung anzustellen, warum etwas gefällt oder mißfällt, wie das Vergnügen der bloßen Empfindung vom Geschmacke, und ob dieser von einem allgemeinen Wohlgefallen der Vernunft unterschieden sei; worauf Gefühl der Lust und Unlust beruhe, und wie hieraus Begierden und Neigungen, aus diesen aber, durch Mitwirkung der Vernunft, Maximen entspringen; denn das gehört alles zu einer empirischen Seelenlehre, welche den zweiten Teil der Naturlehre ausmachen würde, wenn man sie als Philosophie der Natur betrachtet, so fern sie auf empirischen Gesetzen gegründet ist. Hier aber ist vom objektiv-praktischen Gesetze die Rede, mithin von dem Verhältnisse eines Willens zu sich selbst, so fern er sich bloß durch Vernunft bestimmt, da denn alles, was aufs  Empirische Beziehung hat, von selbst wegfällt; weil, wenn die Vernunft für sich allein das Verhalten bestimmt (wovon wir die Möglichkeit jetzt eben untersuchen wollen), sie dieses notwendig a priori tun muß. 
	
	\unnumberedsection{Grenze (4)} 
	\subsection*{tg185.2.10} 
	\textbf{Source : }Grundlegung zur Metaphysik der Sitten/Dritter Abschnitt: Übergang von der Metaphysik der Sitten zur Kritik der reinen praktischen Vernunft/Von der äußersten Grenze aller praktischen Philosophie\\  
	
	\noindent\textbf{Paragraphe : }Aber alsdenn würde die Vernunft alle ihre \match{Grenze} überschreiten, wenn sie es sich zu erklären unterfinge, wie reine Vernunft praktisch sein könne, welches völlig einerlei mit der Aufgabe sein würde, zu erklären, wie Freiheit möglich sei.
	
	
	\subsection*{tg185.2.16} 
	\textbf{Source : }Grundlegung zur Metaphysik der Sitten/Dritter Abschnitt: Übergang von der Metaphysik der Sitten zur Kritik der reinen praktischen Vernunft/Von der äußersten Grenze aller praktischen Philosophie\\  
	
	\noindent\textbf{Paragraphe : }Hier ist nun die oberste \match{Grenze} aller moralischen Nachforschung; welche aber zu bestimmen auch schon darum von großer Wichtigkeit ist, damit die Vernunft nicht einerseits in der Sinnenwelt, auf eine den Sitten schädliche Art, nach der obersten Bewegursache und einem begreiflichen aber empirischen Interesse herumsuche, anderer Seits aber, damit sie auch nicht in dem für sie leeren Raum transzendenter Begriffe, unter dem Namen der intelligibelen Welt, kraftlos ihre Flügel schwinge, ohne von der Stelle zu kommen, und sich unter Hirngespinsten verliere, übrigens bleibt die Idee einer reinen Verstandeswelt, als eines Ganzen aller Intelligenzen, wozu wir selbst, als vernünftige Wesen (obgleich andererseits zugleich Glieder der Sinnenwelt) gehören, immer eine brauchbare und erlaubte Idee zum Behufe eines vernünftigen Glaubens, wenn gleich alles Wissen an der Grenze derselben ein Ende hat, um durch das  herrliche Ideal eines allgemeinen Reichs der Zwecke an sich selbst (vernünftiger Wesen), zu welchen wir nur alsdann als Glieder gehören können, wenn wir uns nach Maximen der Freiheit, als ob sie Gesetze der Natur wären, sorgfältig verhalten, ein lebhaftes Interesse an dem moralischen Gesetze in uns zu bewirken. 
	
	\subsection*{tg185.2.6} 
	\textbf{Source : }Grundlegung zur Metaphysik der Sitten/Dritter Abschnitt: Übergang von der Metaphysik der Sitten zur Kritik der reinen praktischen Vernunft/Von der äußersten Grenze aller praktischen Philosophie\\  
	
	\noindent\textbf{Paragraphe : }Doch kann man hier noch nicht sagen, daß die \match{Grenze} der praktischen Philosophie anfange. Denn jene Beilegung der Streitigkeit gehört gar nicht ihr zu, sondern sie fodert nur von der spekulativen Vernunft, daß diese die Uneinigkeit, darin sie sich in theoretischen Fragen selbst verwickelt, zu Ende bringe, damit praktische Vernunft Ruhe und Sicherheit für äußere Angriffe habe, die ihr den Boden, worauf sie sich anbauen will, streitig machen könnten. 
	
	\subsection*{tg186.2.2} 
	\textbf{Source : }Grundlegung zur Metaphysik der Sitten/Schlußanmerkung\\  
	
	\noindent\textbf{Paragraphe : }Der spekulative Gebrauch der Vernunft, in Ansehung der Natur, führt auf absolute Notwendigkeit irgend einer obersten Ursache der Welt; der praktische Gebrauch der Vernunft, in Absicht auf die Freiheit, führt auch auf absolute Notwendigkeit, aber nur der Gesetze der Handlungen eines vernünftigen Wesens, als eines solchen. Nun ist es ein wesentliches Prinzip alles Gebrauchs unserer Vernunft, ihr Erkenntnis bis zum Bewußtsein ihrer Notwendigkeit zu treiben (denn ohne diese wäre sie nicht Erkenntnis der Vernunft). Es ist aber auch eine eben so wesentliche Einschränkung eben derselben Vernunft, daß sie weder die Notwendigkeit dessen, was da ist, oder was geschieht, noch dessen, was geschehen soll, einsehen kann, wenn nicht eine Bedingung, unter der es da ist, oder geschieht, oder geschehen soll, zum Grunde gelegt wird. Auf diese Weise aber wird, durch die beständige Nachfrage nach der Bedingung, die Befriedigung der Vernunft nur immer weiter aufgeschoben. Daher sucht sie rastlos das Unbedingtnotwendige, und sieht sich genötigt, es anzunehmen, ohne irgend ein Mittel, es sich begreiflich zu machen; glücklich gnug, wenn sie nur den Begriff ausfindig machen kann, der sich mit dieser Voraussetzung verträgt. Es ist also kein Tadel für unsere Deduktion des obersten Prinzips der Moralität, sondern ein Vorwurf, den man der menschlichen Vernunft überhaupt machen müßte, daß sie ein unbedingtes praktisches Gesetz (dergleichen der kategorische Imperativ sein muß) seiner absoluten Notwendigkeit nach nicht begreiflich machen kann; denn, daß sie dieses nicht durch eine  Bedingung, nämlich vermittelst irgend eines zum Grunde gelegten Interesse, tun will, kann ihr nicht verdacht werden, weil es alsdenn kein moralisches, d.i. oberstes Gesetz der Freiheit, sein würde. Und so begreifen wir zwar nicht die praktische unbedingte Notwendigkeit des moralischen Imperativs, wir begreifen aber doch seine Unbegreiflichkeit, welches alles ist, was billigermaßen von einer Philosophie, die bis zur \match{Grenze} der menschlichen Vernunft in Prinzipien strebt, gefodert werden kann. 
	
	\unnumberedsection{Grund (16)} 
	\subsection*{tg174.2.8} 
	\textbf{Source : }Grundlegung zur Metaphysik der Sitten/Vorrede\\  
	
	\noindent\textbf{Paragraphe : }Da meine Absicht hier eigentlich auf die sittliche Weltweisheit gerichtet ist, so schränke ich die vorgelegte Frage nur darauf ein: ob man nicht meine, daß es von der äußersten Notwendigkeit sei, einmal eine reine Moralphilosophie zu bearbeiten, die von allem, was nur empirisch sein mag und zur Anthropologie gehört, völlig gesäubert wäre; denn, daß es eine solche geben müsse, leuchtet von selbst aus der gemeinen Idee der Pflicht und der sittlichen Gesetze ein. Jedermann muß eingestehen, daß ein Gesetz, wenn es moralisch, d.i. als \match{Grund} einer Verbindlichkeit, gelten soll, absolute Notwendigkeit bei sich führen müsse; daß das Gebot: du sollst nicht lügen, nicht etwa bloß für Menschen gelte, andere vernünftige Wesen sich aber daran nicht zu kehren hätten; und so alle übrige eigentliche Sittengesetze; daß mithin der Grund der Verbindlichkeit hier nicht in der Natur des Menschen, oder den Umständen in der Welt, darin er gesetzt ist, gesucht werden müsse, sondern a priori lediglich in Begriffen der reinen Vernunft, und daß jede andere Vorschrift, die sich auf Prinzipien der bloßen Erfahrung gründet, und sogar eine in gewissem Betracht allgemeine Vorschrift, so fern sie sich dem mindesten Teile, vielleicht nur einem Bewegungsgrunde nach, auf empirische Gründe stützt, zwar eine praktische Regel, niemals aber ein moralisches Gesetz heißen kann. 
	
	\subsection*{tg176.2.26} 
	\textbf{Source : }Grundlegung zur Metaphysik der Sitten/Zweiter Abschnitt: Übergang von der populären sittlichen Weltweisheit zur Metaphysik der Sitten\\  
	
	\noindent\textbf{Paragraphe : }Nun entsteht die Frage: wie sind alle diese Imperative möglich? Diese Frage verlangt nicht zu wissen, wie die Vollziehung der Handlung, welche der Imperativ gebietet, sondern wie bloß die Nötigung des Willens, die der Imperativ in der Aufgabe ausdrückt, gedacht werden könne. Wie ein Imperativ der Geschicklichkeit möglich sei, bedarf wohl keiner besondern Erörterung. Wer den Zweck will, will (so fern die Vernunft auf seine Handlungen entscheidenden Einfluß hat) auch das dazu unentbehrlich notwendige Mittel, das in seiner Gewalt ist. Dieser Satz ist, was das Wollen betrifft, analytisch; denn in dem Wollen eines Objekts, als meiner Wirkung, wird schon meine Kausalität, als handelnder Ursache, d.i. der Gebrauch der Mittel, gedacht, und der Imperativ zieht den Begriff notwendiger Handlungen zu diesem Zwecke schon aus dem Begriff eines Wollens dieses Zwecks heraus (die Mittel selbst zu einer vorgesetzten Absicht  zu bestimmen, dazu gehören allerdings synthetische Sätze, die aber nicht den \match{Grund} betreffen, den Actus des Willens, sondern das Objekt wirklich zu machen). Daß, um eine Linie nach einem sichern Prinzip in zwei gleiche Teile zu teilen, ich aus den Enden derselben zwei Kreuzbogen machen müsse, das lehrt die Mathematik freilich nur durch synthetische Sätze; aber daß, wenn ich weiß, durch solche Handlung allein könne die gedachte Wirkung geschehen, ich, wenn ich die Wirkung vollständig will, auch die Handlung wolle, die dazu erfoderlich ist, ist ein analytischer Satz; denn etwas als eine auf gewisse Art durch mich mögliche Wirkung, und mich, in Ansehung ihrer, auf dieselbe Art handelnd vorstellen, ist ganz einerlei. 
	
	\subsection*{tg176.2.30} 
	\textbf{Source : }Grundlegung zur Metaphysik der Sitten/Zweiter Abschnitt: Übergang von der populären sittlichen Weltweisheit zur Metaphysik der Sitten\\  
	
	\noindent\textbf{Paragraphe : }Zweitens ist bei diesem kategorischen Imperativ oder Gesetze der Sittlichkeit der \match{Grund} der Schwierigkeit (die Möglichkeit desselben einzusehen) auch sehr groß. Er ist ein synthetisch-praktischer Satz
	
	
	8
	a priori, und da die Möglichkeit der Sätze dieser Art einzusehen so viel Schwierigkeit im theoretischen Erkenntnisse hat, so läßt sich leicht abnehmen, daß sie im praktischen nicht weniger haben werde. 
	
	\subsection*{tg176.2.48} 
	\textbf{Source : }Grundlegung zur Metaphysik der Sitten/Zweiter Abschnitt: Übergang von der populären sittlichen Weltweisheit zur Metaphysik der Sitten\\  
	
	\noindent\textbf{Paragraphe : }Der Wille wird als ein Vermögen gedacht, der Vorstellung gewisser Gesetze gemäß sich selbst zum Handeln zu bestimmen. Und ein solches Vermögen kann nur in vernünftigen Wesen anzutreffen sein. Nun ist das, was dem Willen zum objektiven Grunde seiner Selbstbestimmung dient, der Zweck, und dieser, wenn er durch bloße Vernunft gegeben wird, muß für alle vernünftige Wesen gleich gelten. Was dagegen bloß den \match{Grund} der Möglichkeit der Handlung enthält, deren Wirkung Zweck ist, heißt das Mittel. Der subjektive Grund des Begehrens ist die Triebfeder, der objektive des Wollens der Bewegungsgrund; daher der Unterschied zwischen subjektiven Zwecken, die auf Triebfedern beruhen, und objektiven, die auf Bewegungsgründe ankommen, welche für jedes vernünftige Wesen gelten. Praktische Prinzipien sind formal, wenn sie von allen subjektiven Zwecken abstrahieren; sie sind aber material, wenn sie diese, mithin gewisse Triebfedern, zum Grunde legen. Die Zwecke, die sich ein vernünftiges Wesen als Wirkungen seiner Handlung nach Belieben vorsetzt (materiale Zwecke), sind insgesamt nur relativ; denn nur bloß ihr Verhältnis auf ein besonders geartetes Begehrungsvermögen des Subjekts gibt ihnen den Wert, der daher keine allgemeine für alle vernünftige Wesen, und auch nicht für jedes Wollen gültige und notwendige Prinzipien, d.i. praktische Gesetze, an die Hand geben kann. Daher sind alle diese relative Zwecke nur der Grund von hypothetischen Imperativen. 
	
	\subsection*{tg176.2.49} 
	\textbf{Source : }Grundlegung zur Metaphysik der Sitten/Zweiter Abschnitt: Übergang von der populären sittlichen Weltweisheit zur Metaphysik der Sitten\\  
	
	\noindent\textbf{Paragraphe : }Gesetzt aber, es gäbe etwas, dessen Dasein an sich selbst einen absoluten Wert hat, was, als Zweck an sich selbst, ein \match{Grund} bestimmter Gesetze sein könnte, so würde in ihm, und nur in ihm allein, der Grund eines möglichen kategorischen Imperativs, d.i. praktischen Gesetzes, liegen. 
	
	\subsection*{tg176.2.51} 
	\textbf{Source : }Grundlegung zur Metaphysik der Sitten/Zweiter Abschnitt: Übergang von der populären sittlichen Weltweisheit zur Metaphysik der Sitten\\  
	
	\noindent\textbf{Paragraphe : }Wenn es denn also ein oberstes praktisches Prinzip, und, in Ansehung des menschlichen Willens, einen kategorischen Imperativ geben soll, so muß es ein solches sein, das aus der Vorstellung dessen, was notwendig für jedermann Zweck ist, weil es Zweck an sich selbst ist, ein objektives Prinzip des Willens ausmacht, mithin zum allgemeinen praktischen Gesetz dienen kann. Der \match{Grund} dieses Prinzips ist: die vernünftige Natur existiert als Zweck an sich selbst. So stellt sich notwendig der Mensch sein eignes Dasein vor; so fern ist es also ein subjektives Prinzip menschlicher Handlungen. So stellt sich aber auch jedes andere vernünftige Wesen sein Dasein, zufolge eben desselben Vernunftgrundes, der auch für mich gilt, vor;
	
	
	12
	also ist es zugleich ein objektives Prinzip, woraus, als einem obersten praktischen Grunde, alle Gesetze des Willens müssen abgeleitet werden können. Der praktische Imperativ wird also folgender sein: Handle so, daß du die Menschheit, sowohl in deiner Person, als in der Person eines jeden andern, jederzeit zugleich als Zweck, niemals bloß als Mittel brauchest. Wir wollen sehen, ob sich dieses bewerkstelligen lasse. 
	
	\subsection*{tg176.2.57} 
	\textbf{Source : }Grundlegung zur Metaphysik der Sitten/Zweiter Abschnitt: Übergang von der populären sittlichen Weltweisheit zur Metaphysik der Sitten\\  
	
	\noindent\textbf{Paragraphe : }Dieses Prinzip der Menschheit und jeder vernünftigen Natur überhaupt, als Zwecks an sich selbst (welche die oberste einschränkende Bedingung der Freiheit der Handlungen eines jeden Menschen ist), ist nicht aus der Erfahrung entlehnt, erstlich, wegen seiner Allgemeinheit, da es auf alle vernünftige Wesen überhaupt geht, worüber etwas zu bestimmen keine Erfahrung zureicht; zweitens, weil darin die Menschheit nicht als Zweck der Menschen (subjektiv), d.i. als Gegenstand, den man sich von selbst wirklich zum Zwecke macht, sondern als objektiver Zweck, der, wir mögen Zwecke haben, welche wir wollen, als Gesetz die oberste einschränkende Bedingung aller subjektiven Zwecke ausmachen soll, vorgestellt wird, mithin aus reiner Vernunft entspringen muß. Es liegt nämlich der \match{Grund} aller praktischen Gesetzgebung objektiv in der Regel und der Form der Allgemeinheit, die sie ein Gesetz (allenfalls Naturgesetz) zu sein fähig macht (nach dem ersten Prinzip), subjektiv aber im Zwecke; das Subjekt aller Zwecke aber ist jedes vernünftige Wesen, als Zweck an sich selbst (nach dem zweiten Prinzip): hieraus folgt nun das dritte praktische Prinzip des Willens, als oberste Bedingung der Zusammenstimmung desselben mit der allgemeinen praktischen Vernunft, die Idee des Willens jedes vernünftigen Wesens als eines allgemein gesetzgebenden Willens.
	
	
	\subsection*{tg176.2.62} 
	\textbf{Source : }Grundlegung zur Metaphysik der Sitten/Zweiter Abschnitt: Übergang von der populären sittlichen Weltweisheit zur Metaphysik der Sitten\\  
	
	\noindent\textbf{Paragraphe : }Es ist nun kein Wunder, wenn wir auf alle bisherige Bemühungen, die jemals unternommen worden, um das Prinzip der Sittlichkeit ausfündig zu machen, zurücksehen, warum sie insgesamt haben fehlschlagen müssen. Man sahe den Menschen durch seine Pflicht an Gesetze gebunden, man ließ es sich aber nicht ein fallen, daß er nur seiner eigenen und dennoch allgemeinen Gesetzgebung unterworfen sei, und daß er nur verbunden sei, seinem eigenen, dem Naturzwecke nach aber allgemein gesetzgebenden, Willen gemäß zu handeln. Denn, wenn man sich ihn nur als einem Gesetz (welches es auch sei) unterworfen dachte: so mußte dieses irgend ein Interesse als Reiz oder Zwang bei sich führen, weil es nicht als Gesetz aus seinem Willen entsprang, sondern dieser gesetzmäßig von etwas anderm genötiget wurde, auf gewisse Weise zu handeln. Durch diese ganz notwendige Folgerung aber war alle Arbeit, einen obersten \match{Grund} der Pflicht zu finden, unwiederbringlich verloren. Denn man bekam niemals Pflicht, sondern Notwendigkeit der Handlung aus einem gewissen Interesse heraus.  Dieses mochte nun ein eigenes oder fremdes Interesse sein. Aber alsdann mußte der Imperativ jederzeit bedingt ausfallen, und konnte zum moralischen Gebote gar nicht taugen. Ich will also diesen Grundsatz das Prinzip der Autonomie des Willens, im Gegensatz mit jedem andern, das ich deshalb zur Heteronomie zähle, nennen. 
	
	\subsection*{tg176.2.73} 
	\textbf{Source : }Grundlegung zur Metaphysik der Sitten/Zweiter Abschnitt: Übergang von der populären sittlichen Weltweisheit zur Metaphysik der Sitten\\  
	
	\noindent\textbf{Paragraphe : }Und was ist es denn nun, was die sittlich gute Gesinnung oder die Tugend berechtigt, so hohe Ansprüche zu machen? Es ist nichts Geringeres als der Anteil, den sie dem vernünftigen Wesen an der allgemeinen Gesetzgebung verschafft, und es hiedurch zum Gliede in einem möglichen Reiche der Zwecke tauglich macht, wozu es durch seine eigene Natur schon bestimmt war, als Zweck an sich selbst und eben darum als gesetzgebend im Reiche der Zwecke, in Ansehung aller Naturgesetze als frei, nur denjenigen allein gehorchend, die es selbst gibt und nach welchen seine Maximen zu einer allgemeinen Gesetzgebung (der er sich zugleich selbst unterwirft) gehören können. Denn es hat nichts einen Wert, als den, welchen ihm das Gesetz bestimmt. Die Gesetzgebung selbst aber, die allen Wert bestimmt, muß eben darum eine Würde, d.i. unbedingten, unvergleichbaren Wert haben, für welchen das Wort Achtung allein den geziemenden Ausdruck der Schätzung abgibt, die ein vernünftiges Wesen über sie anzustellen hat. Autonomie ist also der \match{Grund} der Würde der menschlichen und jeder vernünftigen Natur. 
	
	\subsection*{tg179.2.4} 
	\textbf{Source : }Grundlegung zur Metaphysik der Sitten/Zweiter Abschnitt: Übergang von der populären sittlichen Weltweisheit zur Metaphysik der Sitten/Einteilung aller möglichen Prinzipien der Sittlichkeit aus dem angenommenen Grundbegriffe der Heteronomie\\  
	
	\noindent\textbf{Paragraphe : }
	Empirische Prinzipien taugen überall nicht dazu, um moralische Gesetze darauf zu gründen. Denn die Allgemeinheit, mit der sie für alle vernünftige Wesen ohne Unterschied gelten sollen, die unbedingte praktische Notwendigkeit, die ihnen dadurch auferlegt wird, fällt weg, wenn der \match{Grund} derselben von der besonderen Einrichtung der menschlichen Natur, oder den zufälligen Umständen hergenommen wird, darin sie gesetzt ist. Doch  ist das Prinzip der eigenen Glückseligkeit am meisten verwerflich, nicht bloß deswegen, weil es falsch ist, und die Erfahrung dem Vorgeben, als ob das Wohlbefinden sich jederzeit nach dem Wohlverhalten richte, widerspricht, auch nicht bloß, weil es gar nichts zur Gründung der Sittlichkeit beiträgt, indem es ganz was anderes ist, einen glücklichen, als einen guten Menschen, und diesen klug und auf seinen Vorteil abgewitzt, als ihn tugendhaft zu machen: sondern, weil es der Sittlichkeit Triebfedern unterlegt, die sie eher untergraben und ihre ganze Erhabenheit zernichten, indem sie die Bewegursachen zur Tugend mit denen zum Laster in eine Klasse stellen und nur den Kalkül besser ziehen lehren, den spezifischen Unterschied beider aber ganz und gar auslöschen: dagegen das moralische Gefühl, dieser vermeintliche besondere Sinn
	
	
	16
	(so seicht auch die Berufung auf selbigen ist, indem diejenigen, die nicht denken können, selbst in dem, was bloß auf allgemeine Gesetze ankommt, sich durchs Fühlen auszuhelfen glauben, so wenig auch Gefühle, die dem Grade nach von Natur unendlich von einander unterschieden sind, einen gleichen Maßstab des Guten und Bösen abgeben; auch einer durch sein Gefühl für andere gar nicht gültig urteilen kann), dennoch der Sittlichkeit und ihrer Würde dadurch näher bleibt, daß er der Tugend die Ehre beweist, das Wohlgefallen und die Hochschätzung für sie ihr unmittelbar zuzuschreiben, und ihr nicht gleichsam ins Gesicht sagt, daß es nicht ihre Schönheit, sondern nur der Vorteil sei, der uns an sie knüpfe. 
	
	\subsection*{tg179.2.9} 
	\textbf{Source : }Grundlegung zur Metaphysik der Sitten/Zweiter Abschnitt: Übergang von der populären sittlichen Weltweisheit zur Metaphysik der Sitten/Einteilung aller möglichen Prinzipien der Sittlichkeit aus dem angenommenen Grundbegriffe der Heteronomie\\  
	
	\noindent\textbf{Paragraphe : }Der schlechterdings gute Wille, dessen Prinzip ein kategorischer Imperativ sein muß, wird also, in Ansehung aller Objekte unbestimmt, bloß die Form des Wollens überhaupt enthalten, und zwar als Autonomie, d.i. die Tauglichkeit der Maxime eines jeden guten Willens, sich selbst zum allgemeinen Gesetze zu machen, ist selbst das alleinige Gesetz, das sich der Wille eines jeden vernünftigen Wesens selbst auferlegt, ohne irgend eine Triebfeder und Interesse derselben als \match{Grund} unterzulegen. 
	
	\subsection*{tg182.2.2} 
	\textbf{Source : }Grundlegung zur Metaphysik der Sitten/Dritter Abschnitt: Übergang von der Metaphysik der Sitten zur Kritik der reinen praktischen Vernunft/Freiheit muß als Eigenschaft des Willens aller vernünftigen Wesen vorausgesetzt werden\\  
	
	\noindent\textbf{Paragraphe : }Es ist nicht genug, daß wir unserem Willen, es sei aus welchem Grunde, Freiheit zuschreiben, wenn wir nicht ebendieselbe auch allen vernünftigen Wesen beizulegen hinreichenden \match{Grund} haben. Denn da Sittlichkeit für uns bloß als für vernünftige Wesen zum Gesetze dient, so muß sie auch für alle vernünftige Wesen gelten, und da sie lediglich aus der Eigenschaft der Freiheit abgeleitet werden muß, so muß auch Freiheit als Eigenschaft des Willens aller vernünftigen  Wesen bewiesen werden, und es ist nicht genug, sie aus gewissen vermeintlichen Erfahrungen von der menschlichen Natur darzutun (wiewohl dieses auch schlechterdings unmöglich ist und lediglich a priori dargetan werden kann), sondern man muß sie als zur Tätigkeit vernünftiger und mit einem Willen begabter Wesen überhaupt beweisen. Ich sage nun: Ein jedes Wesen, das nicht anders als unter der Idee der Freiheit handeln kann, ist eben darum, in praktischer Rücksicht, wirklich frei, d.i. es gelten für dasselbe alle Gesetze, die mit der Freiheit unzertrennlich verbunden sind, eben so, als ob sein Wille auch an sich selbst, und in der theoretischen Philosophie gültig, für frei erklärt würde.
	
	
	17
	Nun behaupte ich: daß wir jedem vernünftigen Wesen, das einen Willen hat, notwendig auch die Idee der Freiheit leihen müssen, unter der es allein handle. Denn in einem solchen Wesen denken wir uns eine Vernunft, die praktisch ist, d.i. Kausalität in Ansehung ihrer Objekte hat. Nun kann man sich unmöglich eine Vernunft denken, die mit ihrem eigenen Bewußtsein in Ansehung ihrer Urteile anderwärts her eine Lenkung empfinge, denn alsdenn würde das Subjekt nicht seiner Vernunft, sondern einem Antriebe, die Bestimmung der Urteilskraft zuschreiben. Sie muß sich selbst als Urheberin ihrer Prinzipien ansehen, unabhängig von fremden Einflüssen, folglich muß sie als praktische Vernunft, oder als Wille eines vernünftigen Wesens, von ihr selbst als frei angesehen werden; d.i. der Wille desselben kann nur unter der Idee der Freiheit ein eigener Wille sein, und muß also in praktischer Absicht allen vernünftigen Wesen beigelegt werden. 
	
	\subsection*{tg183.2.13} 
	\textbf{Source : }Grundlegung zur Metaphysik der Sitten/Dritter Abschnitt: Übergang von der Metaphysik der Sitten zur Kritik der reinen praktischen Vernunft/Von dem Interesse, welches den Ideen der Sittlichkeit anhängt\\  
	
	\noindent\textbf{Paragraphe : }Nun ist der Verdacht, den wir oben rege machten, gehoben, als wäre ein geheimer Zirkel in unserem Schlüsse aus der Freiheit auf die Autonomie und aus dieser aufs sittliche Gesetz enthalten, daß wir nämlich vielleicht die Idee der Freiheit nur um des sittlichen Gesetzes willen zum Grunde legten, um dieses nachher aus der Freiheit wiederum zu schließen, mithin von jenem gar keinen \match{Grund} angeben könnten, sondern es nur als Erbittung eines Prinzips, das uns gutgesinnte Seelen wohl gerne einräumen werden, welches wir aber niemals als einen erweislichen Satz aufstellen könnten. Denn jetzt sehen wir, daß, wenn wir uns als frei denken, so versetzen wir uns als Glieder in die Verstandeswelt, und erkennen die Autonomie des Willens, samt ihrer Folge, der Moralität; denken wir uns aber als verpflichtet, so betrachten wir uns als zur Sinnenwelt und doch zugleich zur Verstandeswelt gehörig. 
	
	\subsection*{tg183.2.6} 
	\textbf{Source : }Grundlegung zur Metaphysik der Sitten/Dritter Abschnitt: Übergang von der Metaphysik der Sitten zur Kritik der reinen praktischen Vernunft/Von dem Interesse, welches den Ideen der Sittlichkeit anhängt\\  
	
	\noindent\textbf{Paragraphe : }Es zeigt sich hier, man muß es frei gestehen, eine Art von Zirkel, aus dem, wie es scheint, nicht heraus zu kommen ist. Wir nehmen uns in der Ordnung der wirkenden Ursachen  als frei an, um uns in der Ordnung der Zwecke unter sittlichen Gesetzen zu denken, und wir denken uns nachher als diesen Gesetzen unterworfen, weil wir uns die Freiheit des Willens beigelegt haben, denn Freiheit und eigene Gesetzgebung des Willens sind beides Autonomie, mithin Wechselbegriffe, davon aber einer eben um deswillen nicht dazu gebraucht werden kann, um den anderen zu erklären und von ihm \match{Grund} anzugeben, sondern höchstens nur, um, in logischer Absicht, verschieden scheinende Vorstellungen von eben demselben Gegenstande auf einen einzigen Begriff (wie verschiedne Brüche gleiches Inhalts auf die kleinsten Ausdrücke) zu bringen. 
	
	\subsection*{tg185.2.5} 
	\textbf{Source : }Grundlegung zur Metaphysik der Sitten/Dritter Abschnitt: Übergang von der Metaphysik der Sitten zur Kritik der reinen praktischen Vernunft/Von der äußersten Grenze aller praktischen Philosophie\\  
	
	\noindent\textbf{Paragraphe : }Es ist aber unmöglich, diesem Widerspruch zu entgehen, wenn das Subjekt, was sich frei dünkt, sich selbst in demselben Sinne, oder in eben dem selben Verhältnisse dächte, wenn es sich frei nennt, als wenn es sich in Absicht auf die nämliche Handlung dem Naturgesetze unterworfen annimmt. Daher ist es eine unnachlaßliche Aufgabe der spekulativen Philosophie: wenigstens zu zeigen, daß ihre Täuschung wegen des Widerspruchs darin beruhe, daß wir den Menschen in einem anderen Sinne und Verhältnisse denken, wenn wir ihn frei nennen, als wenn wir ihn, als Stück der Natur, dieser ihren Gesetzen für unterworfen halten, und daß beide nicht allein gar wohl beisammen stehen können, sondern auch, als notwendig vereinigt, in demselben Subjekt gedacht werden müssen, weil sonst nicht \match{Grund} angegeben werden könnte, warum wir die Vernunft mit einer Idee belästigen sollten, die, ob sie sich gleich ohne Widerspruch mit einer anderen genugsam bewährten vereinigen läßt, dennoch uns in ein Geschäfte verwickelt, wodurch die Vernunft in ihrem theoretischen Gebrauche sehr in die Enge gebracht wird. Diese Pflicht liegt aber bloß der spekulativen Philosophie ob, damit sie der praktischen freie Bahn schaffe. Also ist es nicht in das Belieben des Philosophen gesetzt, ob er den scheinbaren Widerstreit heben, oder ihn unangerührt lassen will; denn im letzteren Falle ist die Theorie hierüber bonum vacans, in dessen Besitz sich der Fatalist mit Grunde setzen und alle Moral aus ihrem ohne Titel besessenem vermeinten Eigentum verjagen kann. 
	
	\subsection*{tg187.2.26} 
	\textbf{Source : }Grundlegung zur Metaphysik der Sitten/Fußnoten\\  
	
	\noindent\textbf{Paragraphe : }
	
	13 Man denke ja nicht, daß hier das triviale: quod tibi non vis fieri etc. zur Richtschnur oder Prinzip dienen könne. Denn es ist, obzwar mit verschiedenen Einschränkungen, nur aus jenem abgeleitet; es kann kein allgemeines Gesetz sein, denn es enthält nicht den \match{Grund} der Pflichten gegen sich selbst, nicht der Liebespflichten gegen andere (denn mancher würde es gerne eingehen, daß andere ihm nicht wohltun sollen, wenn er es nur überhoben sein dürfte, ihnen Wohltat zu erzeigen), endlich nicht der schuldigen Pflichten gegen einander; denn der Verbrecher würde aus diesem Grunde gegen seine strafenden Richter argumentieren, u.s.w. 
	
	\unnumberedsection{Hang (2)} 
	\subsection*{tg176.2.44} 
	\textbf{Source : }Grundlegung zur Metaphysik der Sitten/Zweiter Abschnitt: Übergang von der populären sittlichen Weltweisheit zur Metaphysik der Sitten\\  
	
	\noindent\textbf{Paragraphe : }Bei der Absicht, dazu zu gelangen, ist es von der äußersten Wichtigkeit, sich dieses zur Warnung dienen zu lassen, daß man es sich ja nicht in den Sinn kommen lasse, die Realität dieses Prinzips aus der besondern Eigenschaft der menschlichen Natur ableiten zu wollen. Denn Pflicht soll praktisch-unbedingte Notwendigkeit der Handlung sein; sie muß also für alle vernünftige Wesen (auf die nur überall ein Imperativ treffen kann) gelten, und allein darum auch für allen menschlichen Willen ein Gesetz sein. Was dagegen aus der besondern Naturanlage der Menschheit, was aus gewissen Gefühlen und Hange, ja so gar, wo möglich, aus einer besonderen Richtung, die der menschlichen Vernunft eigen wäre, und nicht notwendig für den Willen eines jeden vernünftigen Wesens gelten müßte, abgeleitet wird, das kann zwar eine Maxime für uns, aber kein Gesetz abgeben, ein subjektiv Prinzip, nach welchem wir handeln zu dürfen \match{Hang} und Neigung haben, aber nicht ein objektives, nach welchem wir angewiesen wären zu handeln, wenn gleich aller unser Hang, Neigung und Natureinrichtung dawider wäre, so gar, daß es um desto mehr die Erhabenheit und innere Würde des Gebots in einer Pflicht beweiset, je weniger die subjektiven Ursachen dafür, je mehr sie dagegen sein, ohne doch deswegen die Nötigung durchs Gesetz nur im mindesten zu schwächen, und seiner Gültigkeit etwas zu benehmen. 
	
	\subsection*{tg179.2.5} 
	\textbf{Source : }Grundlegung zur Metaphysik der Sitten/Zweiter Abschnitt: Übergang von der populären sittlichen Weltweisheit zur Metaphysik der Sitten/Einteilung aller möglichen Prinzipien der Sittlichkeit aus dem angenommenen Grundbegriffe der Heteronomie\\  
	
	\noindent\textbf{Paragraphe : }Unter den rationalen, oder Vernunftgründen der Sittlichkeit ist doch der ontologische Begriff der Vollkommenheit (so leer, so unbestimmt, mithin unbrauchbar er auch ist, um in dem unermeßlichen Felde möglicher Realität  die für uns schickliche größte Summe auszufinden, so sehr er auch, um die Realität, von der hier die Rede ist, spezifisch von jeder anderen zu unterscheiden, einen unvermeidlichen \match{Hang} hat, sich im Zirkel zu drehen, und die Sittlichkeit, die er erklären soll, ingeheim vorauszusetzen nicht vermeiden kann) dennoch besser als der theologische Begriff, sie von einem göttlichen allervollkommensten Willen abzuleiten, nicht bloß deswegen, weil wir seine Vollkommenheit doch nicht anschauen, sondern sie von unseren Begriffen, unter denen der der Sittlichkeit der vornehmste ist, allein ableiten können, sondern weil, wenn wir dieses nicht tun (wie es denn, wenn es geschähe, ein grober Zirkel im Erklären sein würde), der uns noch übrige Begriff seines Willens aus den Eigenschaften der Ehr- und Herrschbegierde, mit den furchtbaren Vorstellungen der Macht und des Racheifers verbunden, zu einem System der Sitten, welches der Moralität gerade entgegen gesetzt wäre, die Grundlage machen müßte. 
	
	\unnumberedsection{Himmel (1)} 
	\subsection*{tg176.2.40} 
	\textbf{Source : }Grundlegung zur Metaphysik der Sitten/Zweiter Abschnitt: Übergang von der populären sittlichen Weltweisheit zur Metaphysik der Sitten\\  
	
	\noindent\textbf{Paragraphe : }Noch denkt ein vierter, dem es wohl geht, indessen er sieht, daß andere mit großen Mühseligkeiten zu kämpfen haben (denen er auch wohl helfen könnte): was geht's mich an? mag doch ein jeder so glücklich sein, als es der \match{Himmel} will, oder er sich selbst machen kann, ich werde ihm nichts entziehen, ja nicht einmal beneiden; nur zu seinem Wohlbefinden, oder seinem Beistande in der Not, habe ich nicht Lust, etwas beizutragen! Nun könnte allerdings, wenn eine solche Denkungsart ein allgemeines Naturgesetz würde, das menschliche Geschlecht gar wohl bestehen, und ohne Zweifel noch besser, als wenn jedermann Von Teilnehmung und Wohlwollen schwatzt, auch sich beeifert, gelegentlich dergleichen auszuüben, dagegen aber auch, wo er nur kann, betrügt, das Recht der Menschen verkauft, oder ihm sonst Abbruch tut. Aber, obgleich es möglich ist, daß nach jener Maxime ein allgemeines Naturgesetz wohl bestehen könnte: so ist es doch unmöglich, zu wollen, daß ein solches Prinzip als Naturgesetz allenthalben gelte. Denn ein Wille, der dieses beschlösse, würde sich selbst widerstreiten, indem der Fälle sich doch manche eräugnen können, wo er anderer Liebe und Teilnehmung bedarf, und wo er, durch ein solches aus seinem eigenen Willen entsprungenes Naturgesetz, sich selbst alle Hoffnung des Beistandes, den er sich wünscht, rauben würde. 
	
	\unnumberedsection{Materie (4)} 
	\subsection*{tg176.2.24} 
	\textbf{Source : }Grundlegung zur Metaphysik der Sitten/Zweiter Abschnitt: Übergang von der populären sittlichen Weltweisheit zur Metaphysik der Sitten\\  
	
	\noindent\textbf{Paragraphe : }Endlich gibt es einen Imperativ, der, ohne irgend eine andere durch ein gewisses Verhalten zu erreichende Absicht als Bedingung zum Grunde zu legen, dieses Verhalten unmittelbar gebietet. Dieser Imperativ ist kategorisch. Er betrifft nicht die \match{Materie} der Handlung und das, was aus ihr erfolgen soll, sondern die Form und das Prinzip, woraus sie selbst folgt, und das Wesentlich-Gute derselben besteht in der Gesinnung, der Erfolg mag sein, welcher er wolle. Dieser Imperativ mag der der Sittlichkeit heißen. 
	
	\subsection*{tg176.2.77} 
	\textbf{Source : }Grundlegung zur Metaphysik der Sitten/Zweiter Abschnitt: Übergang von der populären sittlichen Weltweisheit zur Metaphysik der Sitten\\  
	
	\noindent\textbf{Paragraphe : }3) eine vollständige Bestimmung aller Maximen durch jene Formel, nämlich: daß alle Maximen aus eigener Gesetzgebung zu einem möglichen Reiche der Zwecke, als einem Reiche der Natur
	
	
	15
	, zusammenstimmen sollen. Der Fortgang geschieht hier, wie durch die Kategorien der Einheit der Form des Willens (der Allgemeinheit desselben), der Vielheit der \match{Materie} (der Objekte, d.i. der Zwecke), und der Allheit oder Totalität des Systems derselben. Man tut aber besser, wenn man in der sittlichen Beurteilung immer nach der strengen Methode verfährt, und die allgemeine Formel des kategorischen Imperativs zum Grunde legt: handle nach der Maxime, die sich selbst zugleich zum allgemeinen Gesetze machen kann. Will man aber dem sittlichen Gesetze zugleich Eingang verschaffen: so ist sehr nützlich, ein und eben dieselbe Handlung durch benannte drei Begriffe zu führen, und sie dadurch, so viel sich tun läßt, der Anschauung zu nähern. 
	
	\subsection*{tg176.2.79} 
	\textbf{Source : }Grundlegung zur Metaphysik der Sitten/Zweiter Abschnitt: Übergang von der populären sittlichen Weltweisheit zur Metaphysik der Sitten\\  
	
	\noindent\textbf{Paragraphe : }Die vernünftige Natur nimmt sich dadurch vor den übrigen aus, daß sie ihr selbst einen Zweck setzt. Dieser würde die \match{Materie} eines jeden guten Willens sein. Da aber, in der Idee eines ohne einschränkende Bedingung (der Erreichung dieses oder jenes Zwecks) schlechterdings guten Willens, durchaus von allem zu bewirkenden Zwecke abstrahiert werden muß (als der jeden Willen nur relativ gut machen würde), so wird der Zweck hier nicht als ein zu bewirkender, sondern selbständiger Zweck, mithin nur negativ, gedacht werden müssen, d.i. dem niemals zuwider gehandelt, der also niemals bloß als Mittel, sondern jederzeit zugleich als Zweck in jedem Wollen geschätzt werden muß. Dieser kann nun nichts anders als das Subjekt aller möglichen Zwecke selbst sein, weil dieses zugleich das Subjekt eines möglichen schlechterdings guten Willens ist; denn dieser kann, ohne Widerspruch, keinem andern Gegenstande nachgesetzt werden. Das Prinzip: handle in Beziehung auf ein jedes vernünftiges Wesen (auf dich selbst und andere) so, daß es in deiner Maxime zugleich als Zweck an sich selbst gelte, ist demnach mit dem Grundsatze: handle nach einer Maxime, die ihre eigene allgemeine Gültigkeit für jedes vernünftige Wesen zugleich in sich enthält, im Grunde einerlei. Denn, daß ich meine Maxime im Gebrauche der Mittel zu jedem Zwecke auf die Bedingung ihrer Allgemeingültigkeit, als eines Gesetzes für jedes Subjekt einschränken soll, sagt eben so viel, als: das Subjekt der Zwecke, d.i. das vernünftige  Wesen selbst, muß niemals bloß als Mittel, sondern als oberste einschränkende Bedingung im Gebrauche aller Mittel, d.i. jederzeit zugleich als Zweck, allen Maximen der Handlungen zum Grunde gelegt werden. 
	
	\subsection*{tg185.2.14} 
	\textbf{Source : }Grundlegung zur Metaphysik der Sitten/Dritter Abschnitt: Übergang von der Metaphysik der Sitten zur Kritik der reinen praktischen Vernunft/Von der äußersten Grenze aller praktischen Philosophie\\  
	
	\noindent\textbf{Paragraphe : }Die Frage also: wie ein kategorischer Imperativ möglich sei, kann zwar so weit beantwortet werden, als man die einzige Voraussetzung angeben kann, unter der er allein möglich ist, nämlich die Idee der Freiheit, imgleichen als man  die Notwendigkeit dieser Voraussetzung einsehen kann, welches zum praktischen Gebrauche der Vernunft, d.i. zur Überzeugung von der Gültigkeit dieses Imperativs, mithin auch des sittlichen Gesetzes, hinreichend ist, aber wie diese Voraussetzung selbst möglich sei, läßt sich durch keine menschliche Vernunft jemals einsehen. Unter Voraussetzung der Freiheit des Willens einer Intelligenz aber ist die Autonomie desselben, als die formale Bedingung, unter der er allein bestimmt werden kann, eine notwendige Folge. Diese Freiheit des Willens vorauszusetzen, ist auch nicht allein (ohne in Widerspruch mit dem Prinzip der Naturnotwendigkeit in der Verknüpfung der Erscheinungen der Sinnenwelt zu geraten) ganz wohl möglich (wie die spekulative Philosophie zeigen kann), sondern auch, sie praktisch, d.i. in der Idee allen seinen willkürlichen Handlungen, als Bedingung, unterzulegen, ist einem vernünftigen Wesen, das sich seiner Kausalität durch Vernunft, mithin eines Willens (der von Begierden unterschieden ist) bewußt ist, ohne weitere Bedingung notwendig. Wie nun aber reine Vernunft, ohne andere Triebfedern, die irgend woher sonsten genommen sein mögen, für sich selbst praktisch sein, d.i. wie das bloße Prinzip der Allgemeingültigkeit aller ihrer Maximen als Gesetze (welches freilich die Form einer reinen praktischen Vernunft sein würde), ohne alle \match{Materie} (Gegenstand) des Willens, woran man zum voraus irgend ein Interesse nehmen dürfe, für sich selbst eine Triebfeder abgeben, und ein Interesse, welches rein moralisch heißen würde, bewirken, oder mit anderen Worten: wie reine Vernunft praktisch sein könne, das zu erklären, dazu ist alle menschliche Vernunft gänzlich unvermögend, und alle Mühe und Arbeit, hievon Erklärung zu suchen, ist verloren. 
	
	\unnumberedsection{Nachahmung (1)} 
	\subsection*{tg176.2.7} 
	\textbf{Source : }Grundlegung zur Metaphysik der Sitten/Zweiter Abschnitt: Übergang von der populären sittlichen Weltweisheit zur Metaphysik der Sitten\\  
	
	\noindent\textbf{Paragraphe : }Man könnte auch der Sittlichkeit nicht übler raten, als wenn man sie von Beispielen entlehnen wollte. Denn jedes Beispiel, was mir davon vorgestellt wird, muß selbst zuvor nach Prinzipien der Moralität beurteilt werden, ob es auch würdig sei, zum ursprünglichen Beispiele, d.i. zum Muster zu dienen, keinesweges aber kann es den Begriff derselben zu oberst an die Hand geben. Selbst der Heilige des Evangelii muß zuvor mit unserm Ideal der sittlichen Vollkommenheit verglichen werden, ehe man ihn dafür erkennt; auch sagt er von sich selbst: was nennt ihr mich (den ihr sehet) gut, niemand ist gut (das Urbild des Guten) als der einige Gott (den ihr nicht sehet). Woher haben wir aber den Begriff von Gott, als dem höchsten Gut? Lediglich aus der Idee, die die Vernunft a priori von sittlicher Vollkommenheit entwirft, und mit dem Begriffe eines freien Willens unzertrennlich verknüpft. \match{Nachahmung} findet im Sittlichen gar nicht statt, und Beispiele dienen nur zur Aufmunterung, d.i. sie setzen die Tunlichkeit dessen, was das Gesetz gebietet, außer Zweifel, sie machen das, was die praktische Regel allgemeiner ausdrückt, anschaulich, können aber niemals  berechtigen, ihr wahres Original, das in der Vernunft liegt, bei Seite zu setzen und sich nach Beispielen zu richten. 
	
	\unnumberedsection{Qülle (2)} 
	\subsection*{tg174.2.10} 
	\textbf{Source : }Grundlegung zur Metaphysik der Sitten/Vorrede\\  
	
	\noindent\textbf{Paragraphe : }Eine Metaphysik der Sitten ist also unentbehrlich notwendig, nicht bloß aus einem Bewegungsgrunde der Spekulation, um die \match{Quelle} der a priori in unserer Vernunft liegenden praktischen Grundsätze zu erforschen, sondern weil die Sitten selber allerlei Verderbnis unterworfen bleiben, so lange jener Leitfaden und oberste Norm ihrer richtigen Beurteilung fehlt. Denn bei dem, was moralisch gut sein soll, ist es nicht genug, daß es dem sittlichen Gesetze gemäß sei, sondern es muß auch um desselben willen geschehen; widrigenfalls ist jene Gemäßheit nur sehr zufällig und mißlich, weil der unsittliche Grund zwar dann und wann gesetzmäßige, mehrmalen aber gesetzwidrige Handlungen hervorbringen wird. Nun ist aber das sittliche Gesetz, in seiner Reinigkeit und Echtheit (woran eben im Praktischen am meisten gelegen ist), nirgend anders, als in einer reinen Philosophie zu suchen, also muß diese (Metaphysik) vorangehen, und ohne sie kann es überall keine Moralphilosophie geben; selbst verdient diejenige, welche jene reine Prinzipien unter die empirischen mischt, den Namen einer Philosophie nicht (denn dadurch unterscheidet diese sich eben von der gemeinen Vernunfterkenntnis, daß sie, was diese nur vermengt begreift, in abgesonderter Wissenschaft vorträgt), viel weniger einer Moralphilosophie, weil sie eben durch diese Vermengung so gar der Reinigkeit der Sitten selbst Abbruch tut und ihrem eigenen Zwecke zuwider verfährt. 
	
	\subsection*{tg175.2.24} 
	\textbf{Source : }Grundlegung zur Metaphysik der Sitten/Erster Abschnitt: Übergang von der gemeinen sittlichen Vernunfterkenntnis zur philosophischen\\  
	
	\noindent\textbf{Paragraphe : }So wird also die gemeine Menschenvernunft nicht durch irgend ein Bedürfnis der Spekulation (welches ihr, so lange sie sich genügt, bloße gesunde Vernunft zu sein, niemals anwandelt), sondern selbst aus praktischen Gründen  angetrieben, aus ihrem Kreise zu gehen, und einen Schritt ins Feld einer praktischen Philosophie zu tun, um daselbst, wegen der \match{Quelle} ihres Prinzips und richtigen Bestimmung desselben in Gegenhaltung mit den Maximen, die sich auf Bedürfnis und Neigung fußen, Erkundigung und deutliche Anweisung zu bekommen, damit sie aus der Verlegenheit wegen beiderseitiger Ansprüche herauskomme, und nicht Gefahr laufe, durch die Zweideutigkeit, in die sie leicht gerät, um alle echte sittliche Grundsätze gebracht zu werden. Also entspinnt sich eben sowohl in der praktischen gemeinen Vernunft, wenn sie sich kultiviert, unvermerkt eine Dialektik, welche sie nötigt, in der Philosophie Hülfe zu suchen, als es ihr im theoretischen Gebrauche widerfährt, und die erstere wird daher wohl eben so wenig, als die andere, irgendwo sonst, als in einer vollständigen Kritik unserer Vernunft, Ruhe finden. 
	
	\unnumberedsection{Seite (3)} 
	\subsection*{tg176.2.8} 
	\textbf{Source : }Grundlegung zur Metaphysik der Sitten/Zweiter Abschnitt: Übergang von der populären sittlichen Weltweisheit zur Metaphysik der Sitten\\  
	
	\noindent\textbf{Paragraphe : }Wenn es denn keinen echten obersten Grundsatz der Sittlichkeit gibt, der nicht unabhängig von aller Erfahrung bloß auf reiner Vernunft beruhen müßte, so glaube ich, es sei nicht nötig, auch nur zu fragen, ob es gut sei, diese Begriffe, so wie sie, samt den ihnen zugehörigen Prinzipien, a priori feststehen, im allgemeinen (in abstracto) vorzutragen, wofern das Erkenntnis sich vom gemeinen unterscheiden und philosophisch heißen soll. Aber in unsern Zeiten möchte dieses wohl nötig sein. Denn, wenn man Stimmen sammelte, ob reine von allem Empirischen abgesonderte Vernunfterkenntnis, mithin Metaphysik der Sitten, oder populäre praktische Philosophie vorzuziehen sei, so errät man bald, auf welche \match{Seite} das Übergewicht fallen werde. 
	
	\subsection*{tg184.2.2} 
	\textbf{Source : }Grundlegung zur Metaphysik der Sitten/Dritter Abschnitt: Übergang von der Metaphysik der Sitten zur Kritik der reinen praktischen Vernunft/Wie ist ein kategorischer Imperativ möglich\\  
	
	\noindent\textbf{Paragraphe : }Das vernünftige Wesen zählt sich als Intelligenz zur Verstandeswelt, und, bloß als eine zu dieser gehörige wirkende Ursache, nennt es seine Kausalität einen Willen. Von der anderen \match{Seite} ist es sich seiner doch auch als eines Stücks der Sinnenwelt bewußt, in welcher seine Handlungen, als bloße Erscheinungen jener Kausalität, angetroffen werden, deren Möglichkeit aber aus dieser, die wir nicht kennen, nicht eingesehen werden kann, sondern an deren Statt jene Handlungen als bestimmt durch andere Erscheinungen, nämlich Begierden und Neigungen, als zur Sinnenwelt gehörig, eingesehen werden müssen. Als bloßen Gliedes der Verstandeswelt würden also alle meine Handlungen dem  Prinzip der Autonomie des reinen Willens vollkommen gemäß sein; als bloßen Stücks der Sinnenwelt würden sie gänzlich dem Naturgesetz der Begierden und Neigungen, mithin der Heteronomie der Natur gemäß genommen werden müssen. (Die ersteren würden auf dem obersten Prinzip der Sittlichkeit, die zweiten der Glückseligkeit, beruhen.) Weil aber die Verstandeswelt den Grund der Sinnenwelt, mithin auch der Gesetze derselben, enthält, also in Ansehung meines Willens (der ganz zur Verstandeswelt gehört) unmittelbar gesetzgebend ist, und also auch als solche gedacht werden muß, so werde ich mich als Intelligenz, obgleich andererseits wie ein zur Sinnenwelt gehöriges Wesen, dennoch dem Gesetze der ersteren, d.i. der Vernunft, die in der Idee der Freiheit das Gesetz derselben enthält, und also der Autonomie des Willens unterworfen erkennen, folglich die Gesetze der Verstandeswelt für mich als Imperativen und die diesem Prinzip gemäße Handlungen als Pflichten ansehen müssen. 
	
	\subsection*{tg185.2.2} 
	\textbf{Source : }Grundlegung zur Metaphysik der Sitten/Dritter Abschnitt: Übergang von der Metaphysik der Sitten zur Kritik der reinen praktischen Vernunft/Von der äußersten Grenze aller praktischen Philosophie\\  
	
	\noindent\textbf{Paragraphe : }Alle Menschen denken sich dem Willen nach als frei. Daher kommen alle Urteile über Handlungen als solche, die hätten geschehen sollen, ob sie gleich nicht geschehen sind. Gleichwohl ist diese Freiheit kein Erfahrungsbegriff, und kann es auch nicht sein, weil er immer bleibt, obgleich die Erfahrung das Gegenteil von denjenigen Foderungen zeigt, die unter Voraussetzung derselben als notwendig vorgestellt werden. Auf der anderen \match{Seite} ist es eben so notwendig, daß alles, was geschieht, nach Naturgesetzen unausbleiblich bestimmt sei, und diese Naturnotwendigkeit ist auch kein Erfahrungsbegriff, eben darum, weil er den Begriff der Notwendigkeit, mithin einer Erkenntnis a priori, bei sich führet. Aber dieser Begriff von einer Natur wird durch Erfahrung bestätigt, und muß selbst unvermeidlich vorausgesetzt werden, wenn Erfahrung, d.i. nach allgemeinen Gesetzen zusammenhängende Erkenntnis der Gegenstände der Sinne, möglich sein soll. Daher ist Freiheit nur eine Idee der Vernunft, deren objektive Realität an sich zweifelhaft ist, Natur aber ein Verstandesbegriff, der seine Realität an Beispielen der Erfahrung beweiset und notwendig beweisen muß. 
	
	\unnumberedsection{Starke (1)} 
	\subsection*{tg175.2.13} 
	\textbf{Source : }Grundlegung zur Metaphysik der Sitten/Erster Abschnitt: Übergang von der gemeinen sittlichen Vernunfterkenntnis zur philosophischen\\  
	
	\noindent\textbf{Paragraphe : }
	Wohltätig sein, wo man kann, ist Pflicht, und überdem gibt es manche so teilnehmend gestimmte Seelen, daß sie, auch ohne einen andern Bewegungsgrund der Eitelkeit, oder des Eigennutzes, ein inneres Vergnügen daran finden, Freude um sich zu verbreiten, und die sich an der Zufriedenheit anderer, so fern sie ihr Werk ist, ergötzen können. Aber ich behaupte, daß in solchem Falle dergleichen Handlung, so pflichtmäßig, so liebenswürdig sie auch ist, dennoch keinen wahren sittlichen Wert habe, sondern mit andern Neigungen zu gleichen Paaren gehe, z. E. der Neigung nach Ehre, die, wenn sie glücklicherweise auf das trifft, was in der Tat gemeinnützig und pflichtmäßig, mithin ehrenwert ist, Lob und Aufmunterung, aber nicht Hochschätzung verdient; denn der Maxime fehlt der sittliche Gehalt, nämlich solche Handlungen nicht aus Neigung, sondern aus Pflicht zu tun. Gesetzt also, das Gemüt jenes Menschenfreundes wäre vom eigenen Gram umwölkt, der alle Teilnehmung an anderer Schicksal auslöscht, er hätte immer noch Vermögen, andern Notleidenden wohlzutun, aber fremde Not rührte ihn nicht, weil er mit seiner eigenen gnug beschäftigt ist, und nun, da keine Neigung ihn mehr dazu anreizt, risse er sich doch aus dieser tödlichen Unempfindlichkeit heraus, und täte die Handlung ohne alle Neigung, lediglich aus Pflicht, alsdenn hat sie allererst ihren echten moralischen Wert. Noch mehr: wenn die Natur diesem oder jenem überhaupt wenig Sympathie ins Herz gelegt hätte, wenn er (übrigens ein ehrlicher Mann) von Temperament kalt und gleichgültig gegen die Leiden anderer wäre, vielleicht, weil er, selbst gegen seine eigene mit der besondern Gabe der Geduld und aushaltenden \match{Stärke} versehen, dergleichen bei jedem andern auch voraussetzt, oder gar fordert; wenn die Natur einen solchen Mann (welcher wahrlich nicht ihr schlechtestes Produkt sein würde) nicht eigentlich zum Menschenfreunde gebildet hätte, würde er denn nicht noch in sich einen Quell finden, sich selbst einen weit höhern Wert zu geben, als der eines gutartigen Temperaments sein mag? Allerdings! gerade da hebt der Wert des Charakters an, der  moralisch und ohne alle Vergleichung der höchste ist, nämlich daß er wohltue, nicht aus Neigung, sondern aus Pflicht. 
	
	\unnumberedsection{Stelle (2)} 
	\subsection*{tg176.2.70} 
	\textbf{Source : }Grundlegung zur Metaphysik der Sitten/Zweiter Abschnitt: Übergang von der populären sittlichen Weltweisheit zur Metaphysik der Sitten\\  
	
	\noindent\textbf{Paragraphe : }
	Im Reiche der Zwecke hat alles entweder einen Preis, oder eine Würde. Was einen Preis hat, an dessen \match{Stelle} kann auch etwas anderes, als Äquivalent, gesetzt werden; was dagegen über allen Preis erhaben ist, mithin kein Äquivalent verstattet, das hat eine Würde. 
	
	\subsection*{tg176.2.72} 
	\textbf{Source : }Grundlegung zur Metaphysik der Sitten/Zweiter Abschnitt: Übergang von der populären sittlichen Weltweisheit zur Metaphysik der Sitten\\  
	
	\noindent\textbf{Paragraphe : }Nun ist Moralität die Bedingung, unter der allein ein vernünftiges Wesen Zweck an sich selbst sein kann; weil nur durch sie es möglich ist, ein gesetzgebend Glied im Reiche der Zwecke zu sein. Also ist Sittlichkeit und die Menschheit, so fern sie derselben fähig ist, dasjenige, was allein Würde hat. Geschicklichkeit und Fleiß im Arbeiten haben einen Marktpreis; Witz, lebhafte Einbildungskraft und Launen einen Affektionspreis; dagegen Treue im Versprechen, Wohlwollen aus Grundsätzen (nicht aus Instinkt) haben einen innern Wert. Die Natur sowohl als Kunst enthalten nichts, was sie, in Ermangelung derselben, an ihre \match{Stelle} setzen könnten; denn ihr Wert besteht nicht in den Wirkungen, die daraus entspringen, im Vorteil und Nutzen, den sie schaffen, sondern in den Gesinnungen, d.i. den Maximen des Willens, die sich auf diese Art in Handlungen zu offenbaren bereit sind, obgleich auch der Erfolg sie nicht begünstigte. Diese Handlungen bedürfen auch keiner Empfehlung von irgend einer subjektiven Disposition oder Geschmack, sie mit unmittelbarer Gunst und Wohlgefallen anzusehen, keines unmittelbaren Hanges oder Gefühles für dieselbe: sie stellen den Willen, der sie ausübt, als Gegenstand einer unmittelbaren Achtung dar, dazu nichts als Vernunft gefodert wird, um sie dem Willen aufzuerlegen, nicht von  ihm zu erschmeicheln, welches letztere bei Pflichten ohnedem ein Widerspruch wäre. Diese Schätzung gibt also den Wert einer solchen Denkungsart als Würde zu erkennen, und setzt sie über allen Preis unendlich weg, mit dem sie gar nicht in Anschlag und Vergleichung gebracht werden kann, ohne sich gleichsam an der Heiligkeit derselben zu vergreifen. 
	
	\unnumberedsection{Welt (9)} 
	\subsection*{tg176.2.5} 
	\textbf{Source : }Grundlegung zur Metaphysik der Sitten/Zweiter Abschnitt: Übergang von der populären sittlichen Weltweisheit zur Metaphysik der Sitten\\  
	
	\noindent\textbf{Paragraphe : }Man kann auch denen, die alle Sittlichkeit, als bloßes Hirngespinst einer durch Eigendünkel sich selbst übersteigenden menschlichen Einbildung, verlachen, keinen gewünschteren Dienst tun, als ihnen einzuräumen, daß die Begriffe der Pflicht (so wie man sich auch aus Gemächlichkeit  gerne überredet, daß es auch mit allen übrigen Begriffen bewandt sei) lediglich aus der Erfahrung gezogen werden mußten; denn da bereitet man jenen einen sichern Triumph. Ich will aus Menschenliebe einräumen, daß noch die meisten unserer Handlungen pflichtmäßig sein; sieht man aber ihr Tichten und Trachten näher an, so stößt man allenthalben auf das liebe Selbst, was immer hervorsticht, worauf, und nicht auf das strenge Gebot der Pflicht, welches mehrmalen Selbstverleugnung erfodern würde, sich ihre Absicht stützet. Man braucht auch eben kein Feind der Tugend, sondern nur ein kaltblütiger Beobachter zu sein, der den lebhaftesten Wunsch für das Gute nicht so fort für dessen Wirklichkeit hält, um (vornehmlich mit zunehmenden Jahren und einer durch Erfahrung teils gewitzigten, teils zum Beobachten geschärften Urteilskraft) in gewissen Augenblicken zweifelhaft zu werden, ob auch wirklich in der \match{Welt} irgend wahre Tugend angetroffen werde. Und hier kann uns nun nichts für den gänzlichen Abfall von unseren Ideen der Pflicht bewahren und gegründete Achtung gegen ihr Gesetz in der Seele erhalten, als die klare Überzeugung, daß, wenn es auch niemals Handlungen gegeben habe, die aus solchen reinen Quellen entsprungen wären, dennoch hier auch davon gar nicht die Rede sei, ob dies oder jenes geschehe, sondern die Vernunft für sich selbst und unabhängig von allen Erscheinungen gebiete, was geschehen soll, mithin Handlungen, von denen die Welt vielleicht bisher noch gar kein Beispiel gegeben hat, an deren Tunlichkeit sogar der, so alles auf Erfahrung gründet, sehr zweifeln möchte, dennoch durch Vernunft unnachlaßlich geboten sei, und daß z.B. reine Redlichkeit in der Freundschaft um nichts weniger von jedem Menschen gefodert werden könne, wenn es gleich bis jetzt gar keinen redlichen Freund gegeben haben möchte, weil diese Pflicht als Pflicht überhaupt, vor aller Erfahrung, in der Idee einer den Willen durch Gründe a priori bestimmenden Vernunft liegt. 
	
	\subsection*{tg176.2.80} 
	\textbf{Source : }Grundlegung zur Metaphysik der Sitten/Zweiter Abschnitt: Übergang von der populären sittlichen Weltweisheit zur Metaphysik der Sitten\\  
	
	\noindent\textbf{Paragraphe : }Nun folgt hieraus unstreitig: daß jedes vernünftige Wesen, als Zweck an sich selbst, sich in Ansehung aller Gesetze, denen es nur immer unterworfen sein mag, zugleich als allgemein gesetzgebend müsse ansehen können, weil eben diese Schicklichkeit seiner Maximen zur allgemeinen Gesetzgebung es als Zweck an sich selbst auszeichnet, imgleichen, daß dieses seine Würde (Prärogativ) vor allen bloßen Naturwesen es mit sich bringe, seine Maximen jederzeit aus dem Gesichtspunkte seiner selbst, zugleich aber auch jedes andern vernünftigen als gesetzgebenden Wesens (die darum auch Personen heißen), nehmen zu müssen. Nun ist auf solche Weise eine \match{Welt} vernünftiger Wesen (mundus intelligibilis) als ein Reich der Zwecke möglich, und zwar durch die eigene Gesetzgebung aller Personen als Glieder. Demnach muß ein jedes vernünftige Wesen so handeln, als ob es durch seine Maximen jederzeit ein gesetzgebendes Glied im allgemeinen Reiche der Zwecke wäre. Das formale Prinzip dieser Maximen ist: handle so, als ob deine Maxime zugleich zum allgemeinen Gesetze (aller vernünftigen Wesen) dienen sollte. Ein Reich der Zwecke ist also nur möglich nach der Analogie mit einem Reiche der Natur, jenes aber nur nach Maximen, d.i. sich selbst auferlegten Regeln, diese nur nach Gesetzen äußerlich genötigter wirkenden Ursachen. Demunerachtet gibt man doch auch dem Naturganzen, ob es schon als Maschine angesehen wird, dennoch, so fern es auf vernünftige Wesen, als seine Zwecke, Beziehung hat, aus diesem Grunde den Namen eines Reichs der Natur. Ein solches Reich der Zwecke würde nun durch Maximen, deren Regel der kategorische Imperativ aller vernünftigen Wesen vorschreibt, wirklich zu Stande kommen, wenn sie allgemein befolgt würden. Allein, obgleich das vernünftige Wesen darauf nicht rechnen kann, daß, wenn es auch gleich  diese Maxime selbst pünktlich befolgte, darum jedes andere eben derselben treu sein würde, imgleichen, daß das Reich der Natur und die zweckmäßige Anordnung desselben, mit ihm, als einem schicklichen Gliede, zu einem durch ihn selbst möglichen Reiche der Zwecke zusammenstimmen, d.i. seine Erwartung der Glückseligkeit begünstigen werde: so bleibt doch jenes Gesetz: handle nach Maximen eines allgemein gesetzgebenden Gliedes zu einem bloß möglichen Reiche der Zwecke, in seiner vollen Kraft, weil es kategorisch gebietend ist. Und hierin liegt eben das Paradoxon; daß bloß die Würde der Menschheit, als vernünftiger Natur, ohne irgend einen andern dadurch zu erreichenden Zweck, oder Vorteil, mithin die Achtung für eine bloße Idee, dennoch zur unnachlaßlichen Vorschrift des Willens dienen sollte, und daß gerade in dieser Unabhängigkeit der Maxime von allen solchen Triebfedern die Erhabenheit derselben bestehe, und die Würdigkeit eines jeden vernünftigen Subjekts, ein gesetzgebendes Glied im Reiche der Zwecke zu sein; denn sonst würde es nur als dem Naturgesetze seiner Bedürfnis unterworfen vorgestellt werden müssen. Obgleich auch das Naturreich sowohl, als das Reich der Zwecke, als unter einem Oberhaupte vereinigt gedacht würde, und dadurch das letztere nicht mehr bloße Idee bliebe, sondern wahre Realität erhielte, so würde hiedurch zwar jener der Zuwachs einer starken Triebfeder, niemals aber Vermehrung ihres innern Werts zu statten kommen; denn, diesem ungeachtet, müßte doch selbst dieser alleinige unumschränkte Gesetzgeber immer so vorgestellt werden, wie er den Wert der vernünftigen Wesen, nur nach ihrem uneigennützigen, bloß aus jener Idee ihnen selbst vorgeschriebenen Verhalten, beurteilte. Das Wesen der Dinge ändert sich durch ihre äußere Verhältnisse nicht, und was, ohne an das letztere zu denken, den absoluten Wert des Menschen allein ausmacht, darnach muß er auch, von wem es auch sei, selbst vom höchsten Wesen, beurteilt werden. Moralität ist also das Verhältnis der Handlungen zur Autonomie des Willens, das ist, zur möglichen allgemeinen Gesetzgebung durch die  Maximen desselben. Die Handlung, die mit der Autonomie des Willens zusammen bestehen kann, ist erlaubt; die nicht damit stimmt, ist unerlaubt. Der Wille, dessen Maximen notwendig mit den Gesetzen der Autonomie zusammenstimmen, ist ein heiliger, schlechterdings guter Wille. Die Abhängigkeit eines nicht schlechterdings guten Willens vom Prinzip der Autonomie (die moralische Nötigung) ist Verbindlichkeit. Diese kann also auf ein heiliges Wesen nicht gezogen werden. Die objektive Notwendigkeit einer Handlung aus Verbindlichkeit heißt Pflicht. 
	
	\subsection*{tg183.2.11} 
	\textbf{Source : }Grundlegung zur Metaphysik der Sitten/Dritter Abschnitt: Übergang von der Metaphysik der Sitten zur Kritik der reinen praktischen Vernunft/Von dem Interesse, welches den Ideen der Sittlichkeit anhängt\\  
	
	\noindent\textbf{Paragraphe : }Um deswillen muß ein vernünftiges Wesen sich selbst, als Intelligenz (also nicht von Seiten seiner untern Kräfte), nicht als zur Sinnen-, sondern zur Verstandeswelt gehörig, ansehen; mithin hat es zwei Standpunkte, daraus es sich selbst betrachten, und Gesetze des Gebrauchs seiner Kräfte, folglich aller seiner Handlungen, erkennen kann, einmal, so fern es zur Sinnenwelt gehört, unter Naturgesetzen (Heteronomie), zweitens, als zur intelligibelen \match{Welt} gehörig, unter Gesetzen, die, von der Natur unabhängig, nicht empirisch, sondern bloß in der Vernunft gegründet sein. 
	
	\subsection*{tg183.2.12} 
	\textbf{Source : }Grundlegung zur Metaphysik der Sitten/Dritter Abschnitt: Übergang von der Metaphysik der Sitten zur Kritik der reinen praktischen Vernunft/Von dem Interesse, welches den Ideen der Sittlichkeit anhängt\\  
	
	\noindent\textbf{Paragraphe : }Als ein vernünftiges, mithin zur intelligibelen \match{Welt} gehöriges Wesen kann der Mensch die Kausalität seines eigenen Willens niemals anders als unter der Idee der Freiheit denken; denn Unabhängigkeit von den bestimmten Ursachen der Sinnenwelt (dergleichen die Vernunft jederzeit sich selbst beilegen muß) ist Freiheit. Mit der Idee der Freiheit ist nun  der Begriff der Autonomie unzertrennlich verbunden, mit diesem aber das allgemeine Prinzip der Sittlichkeit, welches in der Idee allen Handlungen vernünftiger Wesen eben so zum Grunde liegt, als Naturgesetz allen Erscheinungen. 
	
	\subsection*{tg183.2.8} 
	\textbf{Source : }Grundlegung zur Metaphysik der Sitten/Dritter Abschnitt: Übergang von der Metaphysik der Sitten zur Kritik der reinen praktischen Vernunft/Von dem Interesse, welches den Ideen der Sittlichkeit anhängt\\  
	
	\noindent\textbf{Paragraphe : }Es ist eine Bemerkung, welche anzustellen eben kein subtiles Nachdenken erfodert wird, sondern von der man annehmen kann, daß sie wohl der gemeinste Verstand, obzwar, nach seiner Art, durch eine dunkele Unterscheidung der Urteilskraft, die er Gefühl nennt, machen mag: daß alle Vorstellungen, die uns ohne unsere Willkür kommen (wie die der Sinne), uns die Gegenstände nicht anders zu erkennen geben, als sie uns affizieren, wobei, was sie an sich sein mögen, uns unbekannt bleibt, mithin daß, was diese Art Vorstellungen betrifft, wir dadurch, auch bei der angestrengtesten Aufmerksamkeit und Deutlichkeit, die der Verstand nur immer hinzufügen mag, doch bloß zur Erkenntnis der Erscheinungen, niemals der Dinge an sich selbst gelangen können. Sobald dieser Unterschied (allenfalls bloß durch die bemerkte Verschiedenheit zwischen den Vorstellungen, die uns anders woher gegeben werden, und dabei wir leidend sind, von denen, die wir lediglich aus uns selbst hervorbringen, und dabei wir unsere Tätigkeit beweisen) einmal gemacht ist, so folgt von selbst, daß man hinter den Erscheinungen doch noch etwas anderes, was nicht Erscheinung  ist, nämlich die Dinge an sich, einräumen und annehmen müsse, ob wir gleich uns von selbst bescheiden, daß, da sie uns niemals bekannt werden können, sondern immer nur, wie sie uns affizieren, wir ihnen nicht näher treten, und, was sie an sich sind, niemals wissen können. Dieses muß eine, obzwar rohe, Unterscheidung einer Sinnenwelt von der Verstandeswelt abgeben, davon die erstere, nach Verschiedenheit der Sinnlichkeit in mancherlei Weltbeschauern, auch sehr verschieden sein kann, indessen die zweite, die ihr zum Grunde liegt, immer dieselbe bleibt. So gar sich selbst und zwar nach der Kenntnis, die der Mensch durch innere Empfindung von sich hat, darf er sich nicht anmaßen zu erkennen, wie er an sich selbst sei. Denn da er doch sich selbst nicht gleichsam schafft, und seinen Begriff nicht a priori, sondern empirisch bekömmt, so ist natürlich, daß er auch von sich durch den innern Sinn und folglich nur durch die Erscheinung seiner Natur, und die Art, wie sein Bewußtsein affiziert wird, Kundschaft einziehen könne, indessen er doch notwendiger Weise über diese aus lauter Erscheinungen zusammengesetzte Beschaffenheit seines eigenen Subjekts noch etwas anderes zum Grunde Liegendes, nämlich sein Ich, so wie es an sich selbst beschaffen sein mag, annehmen, und sich also in Absicht auf die bloße Wahrnehmung und Empfänglichkeit der Empfindungen zur Sinnenwelt, in Ansehung dessen aber, was in ihm reine Tätigkeit sein mag (dessen, was gar nicht durch Affizierung der Sinne, sondern unmittelbar zum Bewußtsein gelangt), sich zur intellektuellen \match{Welt} zählen muß, die er doch nicht weiter kennt. 
	
	\subsection*{tg184.2.3} 
	\textbf{Source : }Grundlegung zur Metaphysik der Sitten/Dritter Abschnitt: Übergang von der Metaphysik der Sitten zur Kritik der reinen praktischen Vernunft/Wie ist ein kategorischer Imperativ möglich\\  
	
	\noindent\textbf{Paragraphe : }Und so sind kategorische Imperativen möglich, dadurch, daß die Idee der Freiheit mich zu einem Gliede einer intelligibelen \match{Welt} macht, wodurch, wenn ich solches allein wäre, alle meine Handlungen der Autonomie des Willens jederzeit gemäß sein würden, da ich mich aber zugleich als Glied der Sinnenwelt anschaue, gemäß sein sollen, welches kategorische Sollen einen synthetischen Satz a priori vorstellt, dadurch, daß über meinen durch sinnliche Begierden affizierten Willen noch die Idee ebendesselben, aber zur Verstandeswelt gehörigen, reinen, für sich selbst praktischen Willens hinzukommt, welcher die oberste Bedingung des ersteren nach der Vernunft enthält; ohngefähr so, wie zu den Anschauungen der Sinnenwelt Begriffe des Verstandes, die für sich selbst nichts als gesetzliche Form überhaupt bedeuten, hinzu kommen, und dadurch synthetische Sätze a priori, auf welchen alle Erkenntnis einer Natur beruht, möglich ma chen. 
	
	\subsection*{tg185.2.15} 
	\textbf{Source : }Grundlegung zur Metaphysik der Sitten/Dritter Abschnitt: Übergang von der Metaphysik der Sitten zur Kritik der reinen praktischen Vernunft/Von der äußersten Grenze aller praktischen Philosophie\\  
	
	\noindent\textbf{Paragraphe : }Es ist eben dasselbe, als ob ich zu ergründen suchte, wie Freiheit selbst als Kausalität eines Willens möglich sei. Denn da verlasse ich den philosophischen Erklärungsgrund, und habe keinen anderen. Zwar könnte ich nun in der intelligibelen Welt, die mir noch übrig bleibt, in der \match{Welt} der Intelligenzen herumschwärmen; aber, ob ich gleich davon  eine Idee habe, die ihren guten Grund hat, so habe ich doch von ihr nicht die mindeste Kenntnis, und kann auch zu dieser durch alle Bestrebung meines natürlichen Vernunftvermögens niemals gelangen. Sie bedeutet nur ein Etwas, das da übrig bleibt, wenn ich alles, was zur Sinnenwelt gehöret, von den Bestimmungsgründen meines Willens ausgeschlossen habe, bloß um das Prinzip der Bewegursachen aus dem Felde der Sinnlichkeit einzuschränken, dadurch, daß ich es begrenze, und zeige, daß es nicht alles in allem in sich fasse, sondern daß außer ihm noch mehr sei; dieses Mehrere aber kenne ich nicht weiter. Von der reinen Vernunft, die dieses Ideal denkt, bleibt nach Absonderung aller Materie, d.i. Erkenntnis der Objekte, mir nichts, als die Form übrig, nämlich das praktische Gesetz der Allgemeingültigkeit der Maximen, und, diesem gemäß, die Vernunft in Beziehung auf eine reine Verstandeswelt als mögliche wirkende, d.i. als den Willen bestimmende, Ursache zu denken; die Triebfeder muß hier gänzlich fehlen; es müßte denn diese Idee einer intelligibelen Welt selbst die Triebfeder, oder dasjenige sein, woran die Vernunft ursprünglich ein Interesse nähme; welches aber begreiflich zu machen gerade die Aufgabe ist, die wir nicht auflösen können. 
	
	\subsection*{tg185.2.9} 
	\textbf{Source : }Grundlegung zur Metaphysik der Sitten/Dritter Abschnitt: Übergang von der Metaphysik der Sitten zur Kritik der reinen praktischen Vernunft/Von der äußersten Grenze aller praktischen Philosophie\\  
	
	\noindent\textbf{Paragraphe : }Dadurch, daß die praktische Vernunft sich in eine Verstandeswelt hinein denkt, überschreitet sie gar nicht ihre Grenzen, wohl aber, wenn sie sich hineinschauen, hineinempfinden wollte. Jenes ist nur ein negativer Gedanke, in Ansehung der Sinnenwelt, die der Vernunft in Bestimmung des Willens keine Gesetze gibt, und nur in diesem einzigen Punkte positiv, daß jene Freiheit, als negative Bestimmung, zugleich mit einem (positiven) Vermögen und sogar mit einer Kausalität der Vernunft verbunden sei, welche wir einen Willen nennen, so zu handeln, daß das Prinzip der Handlungen der wesentlichen Beschaffenheit einer Vernunftursache, d.i. der Bedingung der Allgemeingültigkeit der Maxime, als eines Gesetzes, gemäß sei. Würde sie aber noch ein Objekt des Willens, d.i. eine Bewegursache aus der Verstandeswelt herholen, so überschritte sie ihre Grenzen, und maßte sich an, etwas zu kennen, wovon sie nichts weiß. Der Begriff einer Verstandeswelt ist also nur ein Standpunkt, den die Vernunft sich genötigt sieht außer den Erscheinungen zu nehmen, um sich selbst als praktisch zu denken, welches, wenn die Einflüsse der Sinnlichkeit für den Menschen bestimmend wären, nicht möglich sein würde, welches aber doch notwendig ist, wofern ihm nicht das Bewußtsein seiner selbst, als Intelligenz, mithin als vernünftige und durch Vernunft tätige, d.i. frei wirkende Ursache, abgesprochen werden soll. Dieser Gedanke  führt freilich die Idee einer anderen Ordnung und Gesetzgebung, als die des Naturmechanismus, der die Sinnenwelt trifft, herbei, und macht den Begriff einer intelligibelen \match{Welt} (d.i. das Ganze vernünftiger Wesen, als Dinge an sich selbst) notwendig, aber ohne die mindeste Anmaßung, hier weiter, als bloß ihrer formalen Bedingung nach, d.i. der Allgemeinheit der Maxime des Willens, als Gesetze, mithin der Autonomie des letzteren, die allein mit der Freiheit desselben bestehen kann, gemäß zu denken; da hingegen alle Gesetze, die auf ein Objekt bestimmt sind, Heteronomie geben, die nur an Naturgesetzen angetroffen werden und auch nur die Sinnenwelt treffen kann. 
	
	\subsection*{tg187.2.14} 
	\textbf{Source : }Grundlegung zur Metaphysik der Sitten/Fußnoten\\  
	
	\noindent\textbf{Paragraphe : }
	
	7 Mich deucht, die eigentliche Bedeutung des Worts pragmatisch könne so am genauesten bestimmt werden. Denn pragmatisch werden die Sanktionen genannt, welche eigentlich nicht aus dem Rechte der Staaten, als notwendige Gesetze, sondern aus der Vorsorge für die allgemeine Wohlfahrt fließen. Pragmatisch ist eine Geschichte abgefaßt, wenn sie klug macht, d.i. die \match{Welt} belehrt, wie sie ihren Vorteil besser, oder wenigstens eben so gut, als die Vorwelt, besorgen könne. 
	
	\unnumberedsection{Wolke (1)} 
	\subsection*{tg176.2.46} 
	\textbf{Source : }Grundlegung zur Metaphysik der Sitten/Zweiter Abschnitt: Übergang von der populären sittlichen Weltweisheit zur Metaphysik der Sitten\\  
	
	\noindent\textbf{Paragraphe : }Alles also, was empirisch ist, ist, als Zutat zum Prinzip der Sittlichkeit, nicht allein dazu ganz untauglich, sondern der Lauterkeit der Sitten selbst höchst nachteilig, an welchen der eigentliche und über allen Preis erhabene Wert eines schlechterdings guten Willens eben darin besteht, daß das Prinzip der Handlung von allen Einflüssen zufälliger Gründe, die nur Erfahrung an die Hand geben kann, frei sei. Wider diese Nachlässigkeit oder gar niedrige Denkungsart, in Aufsuchung des Prinzips unter empirischen Bewegursachen und Gesetzen, kann man auch nicht zu viel und zu oft Warnungen ergehen lassen, indem die menschliche Vernunft in ihrer Ermüdung gern auf diesem Polster ausruht, und in dem Traume süßer Vorspiegelungen (die sie doch statt der Juno eine \match{Wolke} umarmen lassen) der Sittlichkeit einen aus Gliedern ganz verschiedener Abstammung zusammengeflickten Bastard unterschiebt, der allem ähnlich sieht, was man daran sehen will, nur der Tugend  nicht, für den, der sie einmal in ihrer wahren Gestalt erblickt hat.
	
	
	11
	
	
	
	\unnumberedchapter{Sciences exactes} 
	\unnumberedsection{Ableitung (1)} 
	\subsection*{tg176.2.14} 
	\textbf{Source : }Grundlegung zur Metaphysik der Sitten/Zweiter Abschnitt: Übergang von der populären sittlichen Weltweisheit zur Metaphysik der Sitten\\  
	
	\noindent\textbf{Paragraphe : }Ein jedes Ding der Natur wirkt nach Gesetzen. Nur ein vernünftiges Wesen hat das Vermögen, nach der Vorstellung der Gesetze, d.i. nach Prinzipien, zu handeln, oder einen Willen. Da zur \match{Ableitung} der Handlungen von Gesetzen Vernunft erfodert wird, so ist der Wille nichts anders, als praktische Vernunft. Wenn die Vernunft den Willen unausbleiblich bestimmt, so sind die Handlungen eines solchen Wesens, die als objektiv notwendig erkannt werden, auch subjektiv notwendig, d.i. der Wille ist ein Vermögen, nur dasjenige zu wählen, was die Vernunft, unabhängig von der Neigung, als praktisch notwendig, d.i. als gut erkennt. Bestimmt aber die Vernunft für sich allein den Willen nicht hinlänglich, ist dieser noch subjektiven Bedingungen (gewissen Triebfedern) unterworfen, die nicht immer mit den objektiven übereinstimmen; mit einem Worte, ist der Wille nicht an sich völlig der Vernunft gemäß (wie es bei Menschen wirklich ist): so sind die Handlungen, die objektiv als notwendig erkannt werden, subjektiv zufällig, und die Bestimmung eines solchen Willens, objektiven Gesetzen gemäß, ist Nötigung; d.i. das Verhältnis der objektiven Gesetze zu einem nicht durchaus guten Willen wird vorgestellt als die Bestimmung des Willens eines vernünftigen Wesens zwar durch Gründe der Vernunft, denen aber dieser Wille seiner Natur nach nicht notwendig folgsam ist. 
	
	\unnumberedsection{Auflosung (1)} 
	\subsection*{tg176.2.28} 
	\textbf{Source : }Grundlegung zur Metaphysik der Sitten/Zweiter Abschnitt: Übergang von der populären sittlichen Weltweisheit zur Metaphysik der Sitten\\  
	
	\noindent\textbf{Paragraphe : }Dagegen, wie der Imperativ der Sittlichkeit möglich sei, ist ohne Zweifel die einzige einer \match{Auflösung} bedürftige Frage, da er gar nicht hypothetisch ist und also die objektiv vorgestellte Notwendigkeit sich auf keine Voraussetzung stützen kann, wie bei den hypothetischen Imperativen. Nur ist immer hiebei nicht aus der Acht zu lassen, daß es durch kein Beispiel, mithin empirisch auszumachen sei, ob es überall irgend einen dergleichen Imperativ gebe, sondern zu besorgen, daß alle, die kategorisch scheinen, doch versteckter Weise hypothetisch sein mögen. Z.B. wenn es heißt: du sollst nichts betrüglich versprechen; und man nimmt an, daß die Notwendigkeit dieser Unterlassung nicht etwa bloße Ratgebung zu Vermeidung irgend eines andern Übels sei, so daß es etwa hieße: du sollst nicht lügenhaft versprechen, damit du nicht, wenn es offenbar wird, dich um den Kredit bringest; sondern eine Handlung dieser Art müsse für sich selbst als böse betrachtet werden, der Imperativ des Verbots sei also kategorisch: so kann man doch in keinem Beispiel mit Gewißheit dartun, daß der Wille hier ohne andere Triebfeder, bloß durchs Gesetz, bestimmt werde, ob es gleich so scheint; denn es ist immer möglich, daß ingeheim Furcht für Beschämung, vielleicht auch dunkle Besorgnis anderer Gefahren, Einfluß auf den Willen haben möge. Wer kann das Nichtsein einer Ursache durch Erfahrung beweisen, da diese nichts weiter lehrt, als daß wir jene nicht wahrnehmen? Auf solchen Fall aber würde der sogenannte moralische Imperativ, der als ein solcher kategorisch und unbedingt erscheint, in der Tat nur eine pragmatische Vorschrift sein, die uns auf unsern Vorteil aufmerksam macht, und uns bloß lehrt, diesen in Acht zu nehmen. 
	
	\unnumberedsection{Aufschub (1)} 
	\subsection*{tg179.2.7} 
	\textbf{Source : }Grundlegung zur Metaphysik der Sitten/Zweiter Abschnitt: Übergang von der populären sittlichen Weltweisheit zur Metaphysik der Sitten/Einteilung aller möglichen Prinzipien der Sittlichkeit aus dem angenommenen Grundbegriffe der Heteronomie\\  
	
	\noindent\textbf{Paragraphe : }Übrigens glaube ich einer weitläuftigen Widerlegung aller dieser Lehrbegriffe überhoben sein zu können. Sie ist so leicht, sie ist von denen selbst, deren Amt es erfodert, sich doch für eine dieser Theorien zu erklären (weil Zuhörer den \match{Aufschub} des Urteils nicht wohl leiden mögen), selbst vermutlich so wohl eingesehen, daß dadurch nur überflüssige Arbeit geschehen würde. Was uns aber hier mehr interessiert, ist, zu wissen: daß diese Prinzipien überall nichts als Heteronomie des Willens zum ersten Grunde der Sittlichkeit  aufstellen, und eben darum notwendig ihres Zwecks verfehlen müssen. 
	
	\unnumberedsection{Beweis (1)} 
	\subsection*{tg174.2.14} 
	\textbf{Source : }Grundlegung zur Metaphysik der Sitten/Vorrede\\  
	
	\noindent\textbf{Paragraphe : }Gegenwärtige Grundlegung ist aber nichts mehr, als die Aufsuchung und Festsetzung des obersten Prinzips der Moralität, welche allein ein, in seiner Absicht, ganzes und von aller anderen sittlichen Untersuchung abzusonderndes Geschäfte ausmacht. Zwar würden meine Behauptungen, über diese wichtige und bisher bei weitem noch nicht zur Gnugtuung erörterte Hauptfrage, durch Anwendung desselben Prinzips auf das ganze System, viel Licht, und, durch die Zulänglichkeit, die es allenthalben blicken  läßt, große Bestätigung erhalten: allein ich mußte mich dieses Vorteils begeben, der auch im Grunde mehr eigenliebig, als gemeinnützig sein würde, weil die Leichtigkeit im Gebrauche und die scheinbare Zulänglichkeit eines Prinzips keinen ganz sicheren \match{Beweis} von der Richtigkeit desselben abgibt, vielmehr eine gewisse Parteilichkeit erweckt, es nicht für sich selbst, ohne alle Rücksicht auf die Folge, nach aller Strenge zu untersuchen und zu wägen. 
	
	\unnumberedsection{Daürhaftigkeit (1)} 
	\subsection*{tg175.2.23} 
	\textbf{Source : }Grundlegung zur Metaphysik der Sitten/Erster Abschnitt: Übergang von der gemeinen sittlichen Vernunfterkenntnis zur philosophischen\\  
	
	\noindent\textbf{Paragraphe : }Es ist eine herrliche Sache um die Unschuld, nur es ist auch wiederum sehr schlimm, daß sie sich nicht wohl bewahren läßt und leicht verführt wird. Deswegen bedarf selbst die Weisheit – die sonst wohl mehr im Tun und Lassen, als im Wissen besteht – doch auch der Wissenschaft, nicht um von ihr zu lernen, sondern ihrer Vorschrift Eingang und \match{Dauerhaftigkeit} zu verschaffen. Der Mensch fühlt in sich selbst ein mächtiges Gegengewicht gegen alle Gebote der Pflicht, die ihm die Vernunft so hochachtungswürdig vorstellt, an seinen Bedürfnissen und Neigungen, deren ganze Befriedigung er unter dem Namen der Glückseligkeit zusammenfaßt. Nun gebietet die Vernunft, ohne doch dabei den Neigungen etwas zu verheißen, unnachlaßlich, mithin gleichsam mit Zurücksetzung und Nichtachtung jener so ungestümen und dabei so billig scheinenden Ansprüche (die sich durch kein Gebot wollen aufheben lassen), ihre Vorschriften. Hieraus entspringt aber eine natürliche Dialektik, d.i. ein Hang, wider jene strenge Gesetze der Pflicht zu vernünfteln, und ihre Gültigkeit, wenigstens ihre Reinigkeit und Strenge in Zweifel zu ziehen, und sie, wo möglich, unsern Wünschen und Neigungen angemessener zu machen, d.i. sie im Grunde zu verderben und um ihre ganze Würde zu bringen, welches denn doch selbst die gemeine praktische Vernunft am Ende nicht gutheißen kann. 
	
	\unnumberedsection{Deduktion (1)} 
	\subsection*{tg186.2.2} 
	\textbf{Source : }Grundlegung zur Metaphysik der Sitten/Schlußanmerkung\\  
	
	\noindent\textbf{Paragraphe : }Der spekulative Gebrauch der Vernunft, in Ansehung der Natur, führt auf absolute Notwendigkeit irgend einer obersten Ursache der Welt; der praktische Gebrauch der Vernunft, in Absicht auf die Freiheit, führt auch auf absolute Notwendigkeit, aber nur der Gesetze der Handlungen eines vernünftigen Wesens, als eines solchen. Nun ist es ein wesentliches Prinzip alles Gebrauchs unserer Vernunft, ihr Erkenntnis bis zum Bewußtsein ihrer Notwendigkeit zu treiben (denn ohne diese wäre sie nicht Erkenntnis der Vernunft). Es ist aber auch eine eben so wesentliche Einschränkung eben derselben Vernunft, daß sie weder die Notwendigkeit dessen, was da ist, oder was geschieht, noch dessen, was geschehen soll, einsehen kann, wenn nicht eine Bedingung, unter der es da ist, oder geschieht, oder geschehen soll, zum Grunde gelegt wird. Auf diese Weise aber wird, durch die beständige Nachfrage nach der Bedingung, die Befriedigung der Vernunft nur immer weiter aufgeschoben. Daher sucht sie rastlos das Unbedingtnotwendige, und sieht sich genötigt, es anzunehmen, ohne irgend ein Mittel, es sich begreiflich zu machen; glücklich gnug, wenn sie nur den Begriff ausfindig machen kann, der sich mit dieser Voraussetzung verträgt. Es ist also kein Tadel für unsere \match{Deduktion} des obersten Prinzips der Moralität, sondern ein Vorwurf, den man der menschlichen Vernunft überhaupt machen müßte, daß sie ein unbedingtes praktisches Gesetz (dergleichen der kategorische Imperativ sein muß) seiner absoluten Notwendigkeit nach nicht begreiflich machen kann; denn, daß sie dieses nicht durch eine  Bedingung, nämlich vermittelst irgend eines zum Grunde gelegten Interesse, tun will, kann ihr nicht verdacht werden, weil es alsdenn kein moralisches, d.i. oberstes Gesetz der Freiheit, sein würde. Und so begreifen wir zwar nicht die praktische unbedingte Notwendigkeit des moralischen Imperativs, wir begreifen aber doch seine Unbegreiflichkeit, welches alles ist, was billigermaßen von einer Philosophie, die bis zur Grenze der menschlichen Vernunft in Prinzipien strebt, gefodert werden kann. 
	
	\unnumberedsection{Durchschnitt (1)} 
	\subsection*{tg176.2.27} 
	\textbf{Source : }Grundlegung zur Metaphysik der Sitten/Zweiter Abschnitt: Übergang von der populären sittlichen Weltweisheit zur Metaphysik der Sitten\\  
	
	\noindent\textbf{Paragraphe : }Die Imperativen der Klugheit würden, wenn es nur so leicht wäre, einen bestimmten Begriff von Glückseligkeit zu geben, mit denen der Geschicklichkeit ganz und gar übereinkommen und eben sowohl analytisch sein. Denn es würde eben sowohl hier, als dort, heißen: wer den Zweck will, will auch (der Vernunft gemäß notwendig) die einzigen Mittel, die dazu in seiner Gewalt sind. Allein es ist ein Unglück, daß der Begriff der Glückseligkeit ein so unbestimmter Begriff ist, daß, obgleich jeder Mensch zu dieser zu gelangen wünscht, er doch niemals bestimmt und mit sich selbst einstimmig sagen kann, was er eigentlich wünsche und wolle. Die Ursache davon ist: daß alle Elemente, die zum Begriff der Glückseligkeit gehören, insgesamt empirisch sind, d.i. aus der Erfahrung müssen entlehnt werden, daß gleichwohl zur Idee der Glückseligkeit ein absolutes Ganze, ein Maximum des Wohlbefindens, in meinem gegenwärtigen und jedem zukünftigen Zustande erforderlich ist. Nun ist's unmöglich, daß das einsehendste und zugleich allervermögendste, aber doch endliche Wesen sich einen bestimmten Begriff von dem mache, was er hier eigentlich wolle. Will er Reichtum, wie viel Sorge, Neid und Nachstellung könnte er sich dadurch nicht auf den Hals ziehen. Will er viel Erkenntnis und Einsicht, vielleicht könnte das ein nur um desto schärferes Auge werden, um die Übel, die sich für ihn jetzt noch verbergen und doch nicht vermieden werden können, ihm nur  um desto schrecklicher zu zeigen, oder seinen Begierden, die ihm schon genug zu schaffen machen, noch mehr Bedürfnisse aufzubürden. Will er ein langes Leben, wer steht ihm dafür, daß es nicht ein langes Elend sein würde? Will er wenigstens Gesundheit, wie oft hat noch Ungemächlichkeit des Körpers von Ausschweifung abgehalten, darein unbeschränkte Gesundheit würde haben fallen lassen, u.s.w. Kurz, er ist nicht vermögend, nach irgend einem Grundsatze, mit völliger Gewißheit zu bestimmen, was ihn wahrhaftig glücklich machen werde, darum, weil hiezu Allwissenheit erforderlich sein würde. Man kann also nicht nach bestimmten Prinzipien handeln, um glücklich zu sein, sondern nur nach empirischen Ratschlägen, z.B. der Diät, der Sparsamkeit, der Höflichkeit, der Zurückhaltung u.s.w., von welchen die Erfahrung lehrt, daß sie das Wohlbefinden im \match{Durchschnitt} am meisten befördern. Hieraus folgt, daß die Imperativen der Klugheit, genau zu reden, gar nicht gebieten, d.i. Handlungen objektiv als praktisch-notwendig darstellen können, daß sie eher für Anratungen (consilia) als Gebote (praecepta) der Vernunft zu halten sind, daß die Aufgabe: sicher und allgemein zu bestimmen, welche Handlung die Glückseligkeit eines vernünftigen Wesens befördern werde, völlig unauflöslich, mithin kein Imperativ in Ansehung derselben möglich sei, der im strengen Verstande geböte, das zu tun, was glücklich macht, weil Glückseligkeit nicht ein Ideal der Vernunft, sondern der Einbildungskraft ist, was bloß auf empirischen Gründen beruht, von denen man vergeblich erwartet, daß sie eine Handlung bestimmen sollten, dadurch die Totalität einer in der Tat unendlichen Reihe von Folgen erreicht würde. Dieser Imperativ der Klugheit würde indessen, wenn man annimmt, die Mittel zur Glückseligkeit ließen sich sicher angeben, ein analytisch-praktischer Satz sein; denn er ist von dem Imperativ der Geschicklichkeit nur darin unterschieden, daß bei diesem der Zweck bloß möglich, bei jenem aber gegeben ist; da beide aber bloß die Mittel zu demjenigen gebieten, von dem man voraussetzt, daß man es als Zweck wollte: so ist der  Imperativ, der das Wollen der Mittel für den, der den Zweck will, gebietet, in beiden Fällen analytisch. Es ist also in Ansehung der Möglichkeit eines solchen Imperativs auch keine Schwierigkeit. 
	
	\unnumberedsection{Ende (4)} 
	\subsection*{tg174.2.12} 
	\textbf{Source : }Grundlegung zur Metaphysik der Sitten/Vorrede\\  
	
	\noindent\textbf{Paragraphe : }Im Vorsatze nun, eine Metaphysik der Sitten dereinst zu liefern, lasse ich diese Grundlegung vorangehen. Zwar gibt  es eigentlich keine andere Grundlage derselben, als die Kritik einer reinen praktischen Vernunft, so wie zur Metaphysik die schon gelieferte Kritik der reinen spekulativen Vernunft. Allein, teils ist jene nicht von so äußerster Notwendigkeit, als diese, weil die menschliche Vernunft im Moralischen, selbst beim gemeinsten Verstande, leicht zu großer Richtigkeit und Ausführlichkeit gebracht werden kann, da sie hingegen im theoretischen, aber reinen Gebrauch ganz und gar dialektisch ist; teils erfodere ich zur Kritik einer reinen praktischen Vernunft, daß, wenn sie vollendet sein soll, ihre Einheit mit der spekulativen in einem gemeinschaftlichen Prinzip zugleich müsse dargestellt werden können, weil es doch am \match{Ende} nur eine und dieselbe Vernunft sein kann, die bloß in der Anwendung unterschieden sein muß. Zu einer solchen Vollständigkeit konnte ich es aber hier noch nicht bringen, ohne Betrachtungen von ganz anderer Art herbeizuziehen und den Leser zu verwirren. Um deswillen habe ich mich, statt der Benennung einer Kritik der reinen praktischen Vernunft, der von einer Grundlegung zur Metaphysik der Sitten bedient. 
	
	\subsection*{tg176.2.53} 
	\textbf{Source : }Grundlegung zur Metaphysik der Sitten/Zweiter Abschnitt: Übergang von der populären sittlichen Weltweisheit zur Metaphysik der Sitten\\  
	
	\noindent\textbf{Paragraphe : }
	Erstlich, nach dem Begriffe der notwendigen Pflicht gegen sich selbst, derjenige, der mit Selbstmorde umgeht, sich fragen, ob seine Handlung mit der Idee der Menschheit, als Zwecks an sich selbst, zusammen bestehen könne. Wenn er, um einem beschwerlichen Zustande zu entfliehen, sich selbst zerstört, so bedient er sich einer Person, bloß als eines Mittels, zu Erhaltung eines erträglichen Zustandes bis zu \match{Ende} des Lebens. Der Mensch aber ist keine Sache, mithin nicht etwas, das bloß als Mittel gebraucht werden kann, sondern muß bei allen seinen Handlungen jederzeit als Zweck an sich selbst betrachtet werden. Also kann ich über den Menschen in meiner Person nichts disponieren, ihn zu verstümmeln, zu verderben, oder zu töten. (Die nähere Bestimmung dieses Grundsatzes zur Vermeidung alles Mißverstandes, z.B. der Amputation der Glieder, um mich zu erhalten, der Gefahr, der ich mein Leben aussetze, um mein Leben zu erhalten etc., muß ich hier vorbeigehen; sie gehört zur eigentlichen Moral.) 
	
	\subsection*{tg185.2.16} 
	\textbf{Source : }Grundlegung zur Metaphysik der Sitten/Dritter Abschnitt: Übergang von der Metaphysik der Sitten zur Kritik der reinen praktischen Vernunft/Von der äußersten Grenze aller praktischen Philosophie\\  
	
	\noindent\textbf{Paragraphe : }Hier ist nun die oberste Grenze aller moralischen Nachforschung; welche aber zu bestimmen auch schon darum von großer Wichtigkeit ist, damit die Vernunft nicht einerseits in der Sinnenwelt, auf eine den Sitten schädliche Art, nach der obersten Bewegursache und einem begreiflichen aber empirischen Interesse herumsuche, anderer Seits aber, damit sie auch nicht in dem für sie leeren Raum transzendenter Begriffe, unter dem Namen der intelligibelen Welt, kraftlos ihre Flügel schwinge, ohne von der Stelle zu kommen, und sich unter Hirngespinsten verliere, übrigens bleibt die Idee einer reinen Verstandeswelt, als eines Ganzen aller Intelligenzen, wozu wir selbst, als vernünftige Wesen (obgleich andererseits zugleich Glieder der Sinnenwelt) gehören, immer eine brauchbare und erlaubte Idee zum Behufe eines vernünftigen Glaubens, wenn gleich alles Wissen an der Grenze derselben ein \match{Ende} hat, um durch das  herrliche Ideal eines allgemeinen Reichs der Zwecke an sich selbst (vernünftiger Wesen), zu welchen wir nur alsdann als Glieder gehören können, wenn wir uns nach Maximen der Freiheit, als ob sie Gesetze der Natur wären, sorgfältig verhalten, ein lebhaftes Interesse an dem moralischen Gesetze in uns zu bewirken. 
	
	\subsection*{tg185.2.6} 
	\textbf{Source : }Grundlegung zur Metaphysik der Sitten/Dritter Abschnitt: Übergang von der Metaphysik der Sitten zur Kritik der reinen praktischen Vernunft/Von der äußersten Grenze aller praktischen Philosophie\\  
	
	\noindent\textbf{Paragraphe : }Doch kann man hier noch nicht sagen, daß die Grenze der praktischen Philosophie anfange. Denn jene Beilegung der Streitigkeit gehört gar nicht ihr zu, sondern sie fodert nur von der spekulativen Vernunft, daß diese die Uneinigkeit, darin sie sich in theoretischen Fragen selbst verwickelt, zu \match{Ende} bringe, damit praktische Vernunft Ruhe und Sicherheit für äußere Angriffe habe, die ihr den Boden, worauf sie sich anbauen will, streitig machen könnten. 
	
	\unnumberedsection{Erhebung (1)} 
	\subsection*{tg176.2.9} 
	\textbf{Source : }Grundlegung zur Metaphysik der Sitten/Zweiter Abschnitt: Übergang von der populären sittlichen Weltweisheit zur Metaphysik der Sitten\\  
	
	\noindent\textbf{Paragraphe : }Diese Herablassung zu Volksbegriffen ist allerdings sehr rühmlich, wenn die \match{Erhebung} zu den Prinzipien der reinen Vernunft zuvor geschehen und zur völligen Befriedigung erreicht ist, und das würde heißen, die Lehre der Sitten zuvor auf Metaphysik gründen, ihr aber, wenn sie fest steht, nachher durch Popularität Eingang verschaffen. Es ist aber äußerst ungereimt, dieser in der ersten Untersuchung, worauf alle Richtigkeit der Grundsätze ankommt, schon willfahren zu wollen. Nicht allein, daß dieses Verfahren auf das höchst seltene Verdienst einer wahren philosophischen Popularität niemals Anspruch machen kann, indem es gar keine Kunst ist, gemeinverständlich zu sein, wenn man dabei auf alle gründliche Einsicht Verzicht tut: so bringt es einen ekelhaften Mischmasch von zusammengestoppelten Beobachtungen und halbvernünftelnden Prinzipien zum Vorschein, daran sich schale Köpfe laben, weil es doch etwas gar Brauchbares fürs alltägliche Geschwätz ist, wo Einsehende aber Verwirrung fühlen, und unzufrieden, ohne sich doch helfen zu können, ihre Augen wegwenden, obgleich Philosophen, die das Blendwerk ganz wohl durchschauen, wenig  Gehör finden, wenn sie auf einige Zeit von der vorgeblichen Popularität abrufen, um nur allererst nach erworbener bestimmter Einsicht mit Recht populär sein zu dürfen. 
	
	\unnumberedsection{Feld (2)} 
	\subsection*{tg174.2.11} 
	\textbf{Source : }Grundlegung zur Metaphysik der Sitten/Vorrede\\  
	
	\noindent\textbf{Paragraphe : }Man denke doch ja nicht, daß man das, was hier gefodert wird, schon an der Propädeutik des berühmten Wolff vor seiner Moralphilosophie, nämlich der von ihm so genannten allgemeinen praktischen Weltweisheit, habe, und hier also nicht eben ein ganz neues \match{Feld} einzuschlagen sei.  Eben darum, weil sie eine allgemeine praktische Weltweisheit sein sollte, hat sie keinen Willen von irgend einer besondern Art, etwa einen solchen, der ohne alle empirische Bewegungsgründe, völlig aus Prinzipien a priori, bestimmt werde, und den man einen reinen Willen nennen könnte, sondern das Wollen überhaupt in Betrachtung gezogen, mit allen Handlungen und Bedingungen, die ihm in dieser allgemeinen Bedeutung zukommen, und dadurch unterscheidet sie sich von einer Metaphysik der Sitten, eben so wie die allgemeine Logik von der Transzendentalphilosophie, von denen die erstere die Handlungen und Regeln des Denkens überhaupt, diese aber bloß die besondern Handlungen und Regeln des reinen Denkens, d.i. desjenigen, wodurch Gegenstände völlig a priori erkannt werden, vorträgt. Denn die Metaphysik der Sitten soll die Idee und die Prinzipien eines möglichen reinen Willens untersuchen, und nicht die Handlungen und Bedingungen des menschlichen Wollens überhaupt, welche größtenteils aus der Psychologie geschöpft werden. Daß in der all gemeinen praktischen Weltweisheit (wiewohl wider alle Befugnis) auch von moralischen Gesetzen und Pflicht geredet wird, macht keinen Einwurf wider meine Behauptung aus. Denn die Verfasser jener Wissenschaft bleiben ihrer Idee von derselben auch hierin treu; sie unterscheiden nicht die Bewegungsgründe, die, als solche, völlig a priori bloß durch Vernunft vorgestellt werden und eigentlich moralisch sind, von den empirischen, die der Verstand bloß durch Vergleichung der Erfahrungen zu allgemeinen Begriffen erhebt, sondern betrachten sie, ohne auf den Unterschied ihrer Quellen zu achten, nur nach der größeren oder kleineren Summe derselben (indem sie alle als gleichartig angesehen werden), und machen sich dadurch ihren Begriff von Verbindlichkeit, der freilich nichts weniger als moralisch, aber doch so beschaffen ist, als es in einer Philosophie, die über den Ursprung aller möglichen praktischen Begriffe, ob sie auch a priori oder bloß a posteriori stattfinden, gar nicht urteilt, nur verlangt werden kann. 
	
	\subsection*{tg175.2.24} 
	\textbf{Source : }Grundlegung zur Metaphysik der Sitten/Erster Abschnitt: Übergang von der gemeinen sittlichen Vernunfterkenntnis zur philosophischen\\  
	
	\noindent\textbf{Paragraphe : }So wird also die gemeine Menschenvernunft nicht durch irgend ein Bedürfnis der Spekulation (welches ihr, so lange sie sich genügt, bloße gesunde Vernunft zu sein, niemals anwandelt), sondern selbst aus praktischen Gründen  angetrieben, aus ihrem Kreise zu gehen, und einen Schritt ins \match{Feld} einer praktischen Philosophie zu tun, um daselbst, wegen der Quelle ihres Prinzips und richtigen Bestimmung desselben in Gegenhaltung mit den Maximen, die sich auf Bedürfnis und Neigung fußen, Erkundigung und deutliche Anweisung zu bekommen, damit sie aus der Verlegenheit wegen beiderseitiger Ansprüche herauskomme, und nicht Gefahr laufe, durch die Zweideutigkeit, in die sie leicht gerät, um alle echte sittliche Grundsätze gebracht zu werden. Also entspinnt sich eben sowohl in der praktischen gemeinen Vernunft, wenn sie sich kultiviert, unvermerkt eine Dialektik, welche sie nötigt, in der Philosophie Hülfe zu suchen, als es ihr im theoretischen Gebrauche widerfährt, und die erstere wird daher wohl eben so wenig, als die andere, irgendwo sonst, als in einer vollständigen Kritik unserer Vernunft, Ruhe finden. 
	
	\unnumberedsection{Folge (1)} 
	\subsection*{tg176.2.25} 
	\textbf{Source : }Grundlegung zur Metaphysik der Sitten/Zweiter Abschnitt: Übergang von der populären sittlichen Weltweisheit zur Metaphysik der Sitten\\  
	
	\noindent\textbf{Paragraphe : }Das Wollen nach diesen dreierlei Prinzipien wird auch durch die Ungleichheit der Nötigung des Willens deutlich unterschieden. Um diese nun auch merklich zu machen, glaube ich, daß man sie in ihrer Ordnung am angemessensten so benennen würde, wenn man sagte: sie wären entweder Regeln der Geschicklichkeit, oder Ratschläge der  Klugheit, oder Gebote (Gesetze) der Sittlichkeit. Denn nur das Gesetz führt den Begriff einer unbedingten und zwar objektiven und mithin allgemein gültigen Notwendigkeit bei sich, und Gebote sind Gesetze, denen gehorcht, d.i. auch wider Neigung \match{Folge} geleistet werden muß. Die Ratgebung enthält zwar Notwendigkeit, die aber bloß unter subjektiver gefälliger Bedingung, ob dieser oder jener Mensch dieses oder jenes zu seiner Glückseligkeit zähle, gelten kann; dagegen der kategorische Imperativ durch keine Bedingung eingeschränkt wird, und als absolut- obgleich praktisch-notwendig ganz eigentlich ein Gebot heißen kann. Man könnte die ersteren Imperative auch technisch (zur Kunst gehörig), die zweiten pragmatisch
	
	
	
	7
	(zur Wohlfahrt), die dritten moralisch (zum freien Verhalten überhaupt, d.i. zu den Sitten gehörig) nennen. 
	
	\unnumberedsection{Formel (6)} 
	\subsection*{tg176.2.15} 
	\textbf{Source : }Grundlegung zur Metaphysik der Sitten/Zweiter Abschnitt: Übergang von der populären sittlichen Weltweisheit zur Metaphysik der Sitten\\  
	
	\noindent\textbf{Paragraphe : }Die Vorstellung eines objektiven Prinzips, sofern es für einen Willen nötigend ist, heißt ein Gebot (der Vernunft) und die \match{Formel} des Gebots heißt Imperativ. 
	
	\subsection*{tg176.2.31} 
	\textbf{Source : }Grundlegung zur Metaphysik der Sitten/Zweiter Abschnitt: Übergang von der populären sittlichen Weltweisheit zur Metaphysik der Sitten\\  
	
	\noindent\textbf{Paragraphe : }Bei dieser Aufgabe wollen wir zuerst versuchen, ob nicht vielleicht der bloße Begriff eines kategorischen Imperativs auch die \match{Formel} desselben an die Hand gebe, die den Satz enthält, der allein ein kategorischer Imperativ sein kann; denn wie ein solches absolutes Gebot möglich sei, wenn wir auch gleich wissen, wie es lautet, wird noch besondere und schwere Bemühung erfodern, die wir aber zum letzten Abschnitte aussetzen. 
	
	\subsection*{tg176.2.59} 
	\textbf{Source : }Grundlegung zur Metaphysik der Sitten/Zweiter Abschnitt: Übergang von der populären sittlichen Weltweisheit zur Metaphysik der Sitten\\  
	
	\noindent\textbf{Paragraphe : }Die Imperativen nach der vorigen Vorstellungsart, nämlich der allgemein einer Naturordnung ähnlichen Gesetzmäßigkeit der Handlungen, oder des allgemeinen Zwecksvorzuges vernünftiger Wesen an sich selbst, schlössen zwar von ihrem gebietenden Ansehen alle Beimischung irgend eines Interesse, als Triebfeder, aus, eben dadurch, daß sie als kategorisch vorgestellt wurden; sie wurden aber nur als kategorisch angenommen, weil man dergleichen annehmen mußte, wenn man den Begriff von Pflicht erklären wollte. Daß es aber praktische Sätze gäbe, die kategorisch geböten, könnte für sich nicht bewiesen werden, so wenig, wie es überhaupt in diesem Abschnitte auch hier noch nicht geschehen kann; allein eines hätte doch geschehen können, nämlich: daß die Lossagung von allem Interesse beim Wollen aus Pflicht, als das spezifische Unterscheidungszeichen des kategorischen vom hypothetischen Imperativ, in dem Imperativ selbst, durch irgend eine Bestimmung, die er enthielte, mit angedeutet würde, und dieses geschieht in gegenwärtiger dritten \match{Formel} des Prinzips, nämlich der Idee des Willens eines jeden vernünftigen Wesens, als allgemein-gesetzgebenden Willens. 
	
	\subsection*{tg176.2.75} 
	\textbf{Source : }Grundlegung zur Metaphysik der Sitten/Zweiter Abschnitt: Übergang von der populären sittlichen Weltweisheit zur Metaphysik der Sitten\\  
	
	\noindent\textbf{Paragraphe : }1) eine Form, welche in der Allgemeinheit besteht, und da ist die \match{Formel} des sittlichen Imperativs so ausgedrückt:  daß die Maximen so müssen gewählt werden, als ob sie wie allgemeine Naturgesetze gelten sollten; 
	
	\subsection*{tg176.2.78} 
	\textbf{Source : }Grundlegung zur Metaphysik der Sitten/Zweiter Abschnitt: Übergang von der populären sittlichen Weltweisheit zur Metaphysik der Sitten\\  
	
	\noindent\textbf{Paragraphe : }Wir können nunmehr da endigen, von wo wir im Anfange ausgingen, nämlich dem Begriffe eines unbedingt guten Willens. Der Wille ist schlechterdings gut, der nicht böse sein, mithin dessen Maxime, wenn sie zu einem allgemeinen Gesetze gemacht wird, sich selbst niemals widerstreiten kann. Dieses Prinzip ist also auch sein oberstes Gesetz: handle jederzeit nach derjenigen Maxime, deren Allgemeinheit als Gesetzes du zugleich wollen kannst; dieses ist  die einzige Bedingung, unter der ein Wille niemals mit sich selbst im Widerstreite sein kann, und ein solcher Imperativ ist kategorisch. Weil die Gültigkeit des Willens, als eines allgemeinen Gesetzes für mögliche Handlungen, mit der allgemeinen Verknüpfung des Daseins der Dinge nach allgemeinen Gesetzen, die das Formale der Natur überhaupt ist, Analogie hat, so kann der kategorische Imperativ auch so ausgedrückt werden: Handle nach Maximen, die sich selbst zugleich als allgemeine Naturgesetze zum Gegenstande haben können. So ist also die \match{Formel} eines schlechterdings guten Willens beschaffen. 
	
	\subsection*{tg187.2.28} 
	\textbf{Source : }Grundlegung zur Metaphysik der Sitten/Fußnoten\\  
	
	\noindent\textbf{Paragraphe : }
	
	14 Ich kann hier, Beispiele zur Erläuterung dieses Prinzips anzuführen, überhoben sein, denn die, so zuerst den kategorischen Imperativ und seine \match{Formel} erläuterten, können hier alle zu eben dem Zwecke dienen. 
	
	\unnumberedsection{Glied (5)} 
	\subsection*{tg176.2.66} 
	\textbf{Source : }Grundlegung zur Metaphysik der Sitten/Zweiter Abschnitt: Übergang von der populären sittlichen Weltweisheit zur Metaphysik der Sitten\\  
	
	\noindent\textbf{Paragraphe : }Es gehört aber ein vernünftiges Wesen als \match{Glied} zum Reiche der Zwecke, wenn es darin zwar allgemein gesetzgebend, aber auch diesen Gesetzen selbst unterworfen ist. Es gehört dazu als Oberhaupt, wenn es als gesetzgebend keinem Willen eines andern unterworfen ist. 
	
	\subsection*{tg176.2.72} 
	\textbf{Source : }Grundlegung zur Metaphysik der Sitten/Zweiter Abschnitt: Übergang von der populären sittlichen Weltweisheit zur Metaphysik der Sitten\\  
	
	\noindent\textbf{Paragraphe : }Nun ist Moralität die Bedingung, unter der allein ein vernünftiges Wesen Zweck an sich selbst sein kann; weil nur durch sie es möglich ist, ein gesetzgebend \match{Glied} im Reiche der Zwecke zu sein. Also ist Sittlichkeit und die Menschheit, so fern sie derselben fähig ist, dasjenige, was allein Würde hat. Geschicklichkeit und Fleiß im Arbeiten haben einen Marktpreis; Witz, lebhafte Einbildungskraft und Launen einen Affektionspreis; dagegen Treue im Versprechen, Wohlwollen aus Grundsätzen (nicht aus Instinkt) haben einen innern Wert. Die Natur sowohl als Kunst enthalten nichts, was sie, in Ermangelung derselben, an ihre Stelle setzen könnten; denn ihr Wert besteht nicht in den Wirkungen, die daraus entspringen, im Vorteil und Nutzen, den sie schaffen, sondern in den Gesinnungen, d.i. den Maximen des Willens, die sich auf diese Art in Handlungen zu offenbaren bereit sind, obgleich auch der Erfolg sie nicht begünstigte. Diese Handlungen bedürfen auch keiner Empfehlung von irgend einer subjektiven Disposition oder Geschmack, sie mit unmittelbarer Gunst und Wohlgefallen anzusehen, keines unmittelbaren Hanges oder Gefühles für dieselbe: sie stellen den Willen, der sie ausübt, als Gegenstand einer unmittelbaren Achtung dar, dazu nichts als Vernunft gefodert wird, um sie dem Willen aufzuerlegen, nicht von  ihm zu erschmeicheln, welches letztere bei Pflichten ohnedem ein Widerspruch wäre. Diese Schätzung gibt also den Wert einer solchen Denkungsart als Würde zu erkennen, und setzt sie über allen Preis unendlich weg, mit dem sie gar nicht in Anschlag und Vergleichung gebracht werden kann, ohne sich gleichsam an der Heiligkeit derselben zu vergreifen. 
	
	\subsection*{tg176.2.80} 
	\textbf{Source : }Grundlegung zur Metaphysik der Sitten/Zweiter Abschnitt: Übergang von der populären sittlichen Weltweisheit zur Metaphysik der Sitten\\  
	
	\noindent\textbf{Paragraphe : }Nun folgt hieraus unstreitig: daß jedes vernünftige Wesen, als Zweck an sich selbst, sich in Ansehung aller Gesetze, denen es nur immer unterworfen sein mag, zugleich als allgemein gesetzgebend müsse ansehen können, weil eben diese Schicklichkeit seiner Maximen zur allgemeinen Gesetzgebung es als Zweck an sich selbst auszeichnet, imgleichen, daß dieses seine Würde (Prärogativ) vor allen bloßen Naturwesen es mit sich bringe, seine Maximen jederzeit aus dem Gesichtspunkte seiner selbst, zugleich aber auch jedes andern vernünftigen als gesetzgebenden Wesens (die darum auch Personen heißen), nehmen zu müssen. Nun ist auf solche Weise eine Welt vernünftiger Wesen (mundus intelligibilis) als ein Reich der Zwecke möglich, und zwar durch die eigene Gesetzgebung aller Personen als Glieder. Demnach muß ein jedes vernünftige Wesen so handeln, als ob es durch seine Maximen jederzeit ein gesetzgebendes \match{Glied} im allgemeinen Reiche der Zwecke wäre. Das formale Prinzip dieser Maximen ist: handle so, als ob deine Maxime zugleich zum allgemeinen Gesetze (aller vernünftigen Wesen) dienen sollte. Ein Reich der Zwecke ist also nur möglich nach der Analogie mit einem Reiche der Natur, jenes aber nur nach Maximen, d.i. sich selbst auferlegten Regeln, diese nur nach Gesetzen äußerlich genötigter wirkenden Ursachen. Demunerachtet gibt man doch auch dem Naturganzen, ob es schon als Maschine angesehen wird, dennoch, so fern es auf vernünftige Wesen, als seine Zwecke, Beziehung hat, aus diesem Grunde den Namen eines Reichs der Natur. Ein solches Reich der Zwecke würde nun durch Maximen, deren Regel der kategorische Imperativ aller vernünftigen Wesen vorschreibt, wirklich zu Stande kommen, wenn sie allgemein befolgt würden. Allein, obgleich das vernünftige Wesen darauf nicht rechnen kann, daß, wenn es auch gleich  diese Maxime selbst pünktlich befolgte, darum jedes andere eben derselben treu sein würde, imgleichen, daß das Reich der Natur und die zweckmäßige Anordnung desselben, mit ihm, als einem schicklichen Gliede, zu einem durch ihn selbst möglichen Reiche der Zwecke zusammenstimmen, d.i. seine Erwartung der Glückseligkeit begünstigen werde: so bleibt doch jenes Gesetz: handle nach Maximen eines allgemein gesetzgebenden Gliedes zu einem bloß möglichen Reiche der Zwecke, in seiner vollen Kraft, weil es kategorisch gebietend ist. Und hierin liegt eben das Paradoxon; daß bloß die Würde der Menschheit, als vernünftiger Natur, ohne irgend einen andern dadurch zu erreichenden Zweck, oder Vorteil, mithin die Achtung für eine bloße Idee, dennoch zur unnachlaßlichen Vorschrift des Willens dienen sollte, und daß gerade in dieser Unabhängigkeit der Maxime von allen solchen Triebfedern die Erhabenheit derselben bestehe, und die Würdigkeit eines jeden vernünftigen Subjekts, ein gesetzgebendes Glied im Reiche der Zwecke zu sein; denn sonst würde es nur als dem Naturgesetze seiner Bedürfnis unterworfen vorgestellt werden müssen. Obgleich auch das Naturreich sowohl, als das Reich der Zwecke, als unter einem Oberhaupte vereinigt gedacht würde, und dadurch das letztere nicht mehr bloße Idee bliebe, sondern wahre Realität erhielte, so würde hiedurch zwar jener der Zuwachs einer starken Triebfeder, niemals aber Vermehrung ihres innern Werts zu statten kommen; denn, diesem ungeachtet, müßte doch selbst dieser alleinige unumschränkte Gesetzgeber immer so vorgestellt werden, wie er den Wert der vernünftigen Wesen, nur nach ihrem uneigennützigen, bloß aus jener Idee ihnen selbst vorgeschriebenen Verhalten, beurteilte. Das Wesen der Dinge ändert sich durch ihre äußere Verhältnisse nicht, und was, ohne an das letztere zu denken, den absoluten Wert des Menschen allein ausmacht, darnach muß er auch, von wem es auch sei, selbst vom höchsten Wesen, beurteilt werden. Moralität ist also das Verhältnis der Handlungen zur Autonomie des Willens, das ist, zur möglichen allgemeinen Gesetzgebung durch die  Maximen desselben. Die Handlung, die mit der Autonomie des Willens zusammen bestehen kann, ist erlaubt; die nicht damit stimmt, ist unerlaubt. Der Wille, dessen Maximen notwendig mit den Gesetzen der Autonomie zusammenstimmen, ist ein heiliger, schlechterdings guter Wille. Die Abhängigkeit eines nicht schlechterdings guten Willens vom Prinzip der Autonomie (die moralische Nötigung) ist Verbindlichkeit. Diese kann also auf ein heiliges Wesen nicht gezogen werden. Die objektive Notwendigkeit einer Handlung aus Verbindlichkeit heißt Pflicht. 
	
	\subsection*{tg184.2.3} 
	\textbf{Source : }Grundlegung zur Metaphysik der Sitten/Dritter Abschnitt: Übergang von der Metaphysik der Sitten zur Kritik der reinen praktischen Vernunft/Wie ist ein kategorischer Imperativ möglich\\  
	
	\noindent\textbf{Paragraphe : }Und so sind kategorische Imperativen möglich, dadurch, daß die Idee der Freiheit mich zu einem Gliede einer intelligibelen Welt macht, wodurch, wenn ich solches allein wäre, alle meine Handlungen der Autonomie des Willens jederzeit gemäß sein würden, da ich mich aber zugleich als \match{Glied} der Sinnenwelt anschaue, gemäß sein sollen, welches kategorische Sollen einen synthetischen Satz a priori vorstellt, dadurch, daß über meinen durch sinnliche Begierden affizierten Willen noch die Idee ebendesselben, aber zur Verstandeswelt gehörigen, reinen, für sich selbst praktischen Willens hinzukommt, welcher die oberste Bedingung des ersteren nach der Vernunft enthält; ohngefähr so, wie zu den Anschauungen der Sinnenwelt Begriffe des Verstandes, die für sich selbst nichts als gesetzliche Form überhaupt bedeuten, hinzu kommen, und dadurch synthetische Sätze a priori, auf welchen alle Erkenntnis einer Natur beruht, möglich ma chen. 
	
	\subsection*{tg184.2.4} 
	\textbf{Source : }Grundlegung zur Metaphysik der Sitten/Dritter Abschnitt: Übergang von der Metaphysik der Sitten zur Kritik der reinen praktischen Vernunft/Wie ist ein kategorischer Imperativ möglich\\  
	
	\noindent\textbf{Paragraphe : }Der praktische Gebrauch der gemeinen Menschenvernunft bestätigt die Richtigkeit dieser Deduktion. Es ist niemand, selbst der ärgste Bösewicht, wenn er nur sonst Vernunft  zu brauchen gewohnt ist, der nicht, wenn man ihm Beispiele der Redlichkeit in Absichten, der Standhaftigkeit in Befolgung guter Maximen, der Teilnehmung und des allgemeinen Wohlwollens (und noch dazu mit großen Aufopferungen von Vorteilen und Gemächlichkeit verbunden) vorlegt, nicht wünsche, daß er auch so gesinnt sein möchte. Er kann es aber nur wegen seiner Neigungen und Antriebe nicht wohl in sich zu Stande bringen; wobei er dennoch zugleich wünscht, von solchen ihm selbst lästigen Neigungen frei zu sein. Er beweiset hiedurch also, daß er mit einem Willen, der von Antrieben der Sinnlichkeit frei ist, sich in Gedanken in eine ganz andere Ordnung der Dinge versetze, als die seiner Begierden im Felde der Sinnlichkeit, weil er von jenem Wunsche keine Vergnügung der Begierden, mithin keinen für irgend eine seiner wirklichen oder sonst erdenklichen Neigungen befriedigenden Zustand (denn dadurch würde selbst die Idee, welche ihm den Wunsch ablockt, ihre Vorzüglichkeit einbüßen), sondern nur einen größeren inneren Wert seiner Person erwarten kann. Diese bessere Person glaubt er aber zu sein, wenn er sich in den Standpunkt eines Gliedes der Verstandeswelt versetzt, dazu die Idee der Freiheit, d.i. Unabhängigkeit von bestimmenden Ursachen der Sinnenwelt ihn unwillkürlich nötigt, und in welchem er sich eines guten Willens bewußt ist, der für seinen bösen Willen, als Gliedes der Sinnenwelt, nach seinem eigenen Geständnisse das Gesetz ausmacht, dessen Ansehen er kennt, indem er es übertritt. Das moralische Sollen ist also eigenes notwendiges Wollen als Gliedes einer intelligibelen Welt, und wird nur so fern von ihm als Sollen gedacht, als er sich zugleich wie ein \match{Glied} der Sinnenwelt betrachtet. 
	
	\unnumberedsection{Grad (2)} 
	\subsection*{tg175.2.8} 
	\textbf{Source : }Grundlegung zur Metaphysik der Sitten/Erster Abschnitt: Übergang von der gemeinen sittlichen Vernunfterkenntnis zur philosophischen\\  
	
	\noindent\textbf{Paragraphe : }In der Tat finden wir auch, daß, je mehr eine kultivierte Vernunft sich mit der Absicht auf den Genuß des Lebens  und der Glückseligkeit abgibt, desto weiter der Mensch von der wahren Zufriedenheit abkomme, woraus bei vielen, und zwar den Versuchtesten im Gebrauche derselben, wenn sie nur aufrichtig genug sind, es zu gestehen, ein gewisser \match{Grad} von Misologie, d.i. Haß der Vernunft entspringt, weil sie nach dem Überschlage alles Vorteils, den sie, ich will nicht sagen von der Erfindung aller Künste des gemeinen Luxus, sondern so gar von den Wissenschaften (die ihnen am Ende auch ein Luxus des Verstandes zu sein scheinen) ziehen, dennoch finden, daß sie sich in der Tat nur mehr Mühseligkeit auf den Hals gezogen, als an Glückseligkeit gewonnen haben, und darüber endlich den gemeinern Schlag der Menschen, welcher der Leitung des bloßen Naturinstinkts näher ist, und der seiner Vernunft nicht viel Einfluß auf sein Tun und Lassen verstattet, eher beneiden, als geringschätzen. Und so weit muß man gestehen, daß das Urteil derer, die die ruhmredige Hochpreisungen der Vorteile, die uns die Vernunft in Ansehung der Glückseligkeit und Zufriedenheit des Lebens verschaffen sollte, sehr mäßigen und sogar unter Null herabsetzen, keinesweges grämisch, oder gegen die Güte der Weltregierung undankbar sei, sondern daß diesen Urteilen ingeheim die Idee von einer andern und viel würdigern Absicht ihrer Existenz zum Grunde liege, zu welcher, und nicht der Glückseligkeit, die Vernunft ganz eigentlich bestimmt sei, und welcher darum, als oberster Bedingung, die Privatabsicht des Menschen größtenteils nachstehen muß. 
	
	\subsection*{tg183.2.9} 
	\textbf{Source : }Grundlegung zur Metaphysik der Sitten/Dritter Abschnitt: Übergang von der Metaphysik der Sitten zur Kritik der reinen praktischen Vernunft/Von dem Interesse, welches den Ideen der Sittlichkeit anhängt\\  
	
	\noindent\textbf{Paragraphe : }Dergleichen Schluß muß der nachdenkende Mensch von allen Dingen, die ihm vorkommen mögen, fällen; vermutlich ist er auch im gemeinsten Verstande anzutreffen, der, wie bekannt, sehr geneigt ist, hinter den Gegenständen der Sinne noch immer etwas Unsichtbares, für sich selbst Tätiges, zu erwarten, es aber wiederum dadurch verdirbt, daß er dieses Unsichtbare sich bald wiederum versinnlicht, d.i. zum Gegenstande der Anschauung machen will, und dadurch also nicht um einen \match{Grad} klüger wird. 
	
	\unnumberedsection{Grundsatz (3)} 
	\subsection*{tg175.2.7} 
	\textbf{Source : }Grundlegung zur Metaphysik der Sitten/Erster Abschnitt: Übergang von der gemeinen sittlichen Vernunfterkenntnis zur philosophischen\\  
	
	\noindent\textbf{Paragraphe : }In den Naturanlagen eines organisierten, d.i. zweckmäßig zum Leben eingerichteten Wesens nehmen wir es als \match{Grundsatz} an, daß kein Werkzeug zu irgend einem Zwecke in demselben angetroffen werde, als was auch zu demselben das schicklichste und ihm am meisten angemessen ist. Wäre nun an einem Wesen, das Vernunft und einen Willen hat, seine Erhaltung, sein Wohlergehen, mit einem Worte seine Glückseligkeit, der eigentliche Zweck der Natur, so hätte sie ihre Veranstaltung dazu sehr schlecht getroffen, sich die Vernunft des Geschöpfs zur Ausrichterin dieser ihrer Absicht zu ersehen. Denn alle Handlungen, die es in dieser Absicht auszuüben hat, und die ganze Regel seines Verhaltens würden ihm weit genauer durch Instinkt vorgezeichnet, und jener Zweck weit sicherer dadurch haben erhalten werden können, als es jemals durch Vernunft geschehen kann, und, sollte diese ja obenein dem begünstigten Geschöpf er teilt worden sein, so würde sie ihm nur dazu haben dienen müssen, um über die glückliche Anlage seiner Natur Betrachtungen anzustellen, sie zu bewundern, sich ihrer zu erfreuen und der wohltätigen Ursache dafür dankbar zu sein; nicht aber, um sein Begehrungsvermögen jener schwachen und trüglichen Leitung zu unterwerfen und in der Naturabsicht zu pfuschen; mit einem Worte, sie würde verhütet haben, daß Vernunft nicht in praktischen Gebrauch ausschlüge, und die Vermessenheit hätte, mit ihren schwachen Einsichten ihr selbst den Entwurf der Glückseligkeit und der Mittel, dazu zu gelangen, auszudenken; die Natur würde nicht allein die Wahl der Zwecke, sondern auch der Mittel selbst übernommen, und beide mit weiser Vorsorge lediglich dem Instinkte anvertraut haben. 
	
	\subsection*{tg176.2.62} 
	\textbf{Source : }Grundlegung zur Metaphysik der Sitten/Zweiter Abschnitt: Übergang von der populären sittlichen Weltweisheit zur Metaphysik der Sitten\\  
	
	\noindent\textbf{Paragraphe : }Es ist nun kein Wunder, wenn wir auf alle bisherige Bemühungen, die jemals unternommen worden, um das Prinzip der Sittlichkeit ausfündig zu machen, zurücksehen, warum sie insgesamt haben fehlschlagen müssen. Man sahe den Menschen durch seine Pflicht an Gesetze gebunden, man ließ es sich aber nicht ein fallen, daß er nur seiner eigenen und dennoch allgemeinen Gesetzgebung unterworfen sei, und daß er nur verbunden sei, seinem eigenen, dem Naturzwecke nach aber allgemein gesetzgebenden, Willen gemäß zu handeln. Denn, wenn man sich ihn nur als einem Gesetz (welches es auch sei) unterworfen dachte: so mußte dieses irgend ein Interesse als Reiz oder Zwang bei sich führen, weil es nicht als Gesetz aus seinem Willen entsprang, sondern dieser gesetzmäßig von etwas anderm genötiget wurde, auf gewisse Weise zu handeln. Durch diese ganz notwendige Folgerung aber war alle Arbeit, einen obersten Grund der Pflicht zu finden, unwiederbringlich verloren. Denn man bekam niemals Pflicht, sondern Notwendigkeit der Handlung aus einem gewissen Interesse heraus.  Dieses mochte nun ein eigenes oder fremdes Interesse sein. Aber alsdann mußte der Imperativ jederzeit bedingt ausfallen, und konnte zum moralischen Gebote gar nicht taugen. Ich will also diesen \match{Grundsatz} das Prinzip der Autonomie des Willens, im Gegensatz mit jedem andern, das ich deshalb zur Heteronomie zähle, nennen. 
	
	\subsection*{tg176.2.8} 
	\textbf{Source : }Grundlegung zur Metaphysik der Sitten/Zweiter Abschnitt: Übergang von der populären sittlichen Weltweisheit zur Metaphysik der Sitten\\  
	
	\noindent\textbf{Paragraphe : }Wenn es denn keinen echten obersten \match{Grundsatz} der Sittlichkeit gibt, der nicht unabhängig von aller Erfahrung bloß auf reiner Vernunft beruhen müßte, so glaube ich, es sei nicht nötig, auch nur zu fragen, ob es gut sei, diese Begriffe, so wie sie, samt den ihnen zugehörigen Prinzipien, a priori feststehen, im allgemeinen (in abstracto) vorzutragen, wofern das Erkenntnis sich vom gemeinen unterscheiden und philosophisch heißen soll. Aber in unsern Zeiten möchte dieses wohl nötig sein. Denn, wenn man Stimmen sammelte, ob reine von allem Empirischen abgesonderte Vernunfterkenntnis, mithin Metaphysik der Sitten, oder populäre praktische Philosophie vorzuziehen sei, so errät man bald, auf welche Seite das Übergewicht fallen werde. 
	
	\unnumberedsection{Materie (4)} 
	\subsection*{tg176.2.24} 
	\textbf{Source : }Grundlegung zur Metaphysik der Sitten/Zweiter Abschnitt: Übergang von der populären sittlichen Weltweisheit zur Metaphysik der Sitten\\  
	
	\noindent\textbf{Paragraphe : }Endlich gibt es einen Imperativ, der, ohne irgend eine andere durch ein gewisses Verhalten zu erreichende Absicht als Bedingung zum Grunde zu legen, dieses Verhalten unmittelbar gebietet. Dieser Imperativ ist kategorisch. Er betrifft nicht die \match{Materie} der Handlung und das, was aus ihr erfolgen soll, sondern die Form und das Prinzip, woraus sie selbst folgt, und das Wesentlich-Gute derselben besteht in der Gesinnung, der Erfolg mag sein, welcher er wolle. Dieser Imperativ mag der der Sittlichkeit heißen. 
	
	\subsection*{tg176.2.77} 
	\textbf{Source : }Grundlegung zur Metaphysik der Sitten/Zweiter Abschnitt: Übergang von der populären sittlichen Weltweisheit zur Metaphysik der Sitten\\  
	
	\noindent\textbf{Paragraphe : }3) eine vollständige Bestimmung aller Maximen durch jene Formel, nämlich: daß alle Maximen aus eigener Gesetzgebung zu einem möglichen Reiche der Zwecke, als einem Reiche der Natur
	
	
	15
	, zusammenstimmen sollen. Der Fortgang geschieht hier, wie durch die Kategorien der Einheit der Form des Willens (der Allgemeinheit desselben), der Vielheit der \match{Materie} (der Objekte, d.i. der Zwecke), und der Allheit oder Totalität des Systems derselben. Man tut aber besser, wenn man in der sittlichen Beurteilung immer nach der strengen Methode verfährt, und die allgemeine Formel des kategorischen Imperativs zum Grunde legt: handle nach der Maxime, die sich selbst zugleich zum allgemeinen Gesetze machen kann. Will man aber dem sittlichen Gesetze zugleich Eingang verschaffen: so ist sehr nützlich, ein und eben dieselbe Handlung durch benannte drei Begriffe zu führen, und sie dadurch, so viel sich tun läßt, der Anschauung zu nähern. 
	
	\subsection*{tg176.2.79} 
	\textbf{Source : }Grundlegung zur Metaphysik der Sitten/Zweiter Abschnitt: Übergang von der populären sittlichen Weltweisheit zur Metaphysik der Sitten\\  
	
	\noindent\textbf{Paragraphe : }Die vernünftige Natur nimmt sich dadurch vor den übrigen aus, daß sie ihr selbst einen Zweck setzt. Dieser würde die \match{Materie} eines jeden guten Willens sein. Da aber, in der Idee eines ohne einschränkende Bedingung (der Erreichung dieses oder jenes Zwecks) schlechterdings guten Willens, durchaus von allem zu bewirkenden Zwecke abstrahiert werden muß (als der jeden Willen nur relativ gut machen würde), so wird der Zweck hier nicht als ein zu bewirkender, sondern selbständiger Zweck, mithin nur negativ, gedacht werden müssen, d.i. dem niemals zuwider gehandelt, der also niemals bloß als Mittel, sondern jederzeit zugleich als Zweck in jedem Wollen geschätzt werden muß. Dieser kann nun nichts anders als das Subjekt aller möglichen Zwecke selbst sein, weil dieses zugleich das Subjekt eines möglichen schlechterdings guten Willens ist; denn dieser kann, ohne Widerspruch, keinem andern Gegenstande nachgesetzt werden. Das Prinzip: handle in Beziehung auf ein jedes vernünftiges Wesen (auf dich selbst und andere) so, daß es in deiner Maxime zugleich als Zweck an sich selbst gelte, ist demnach mit dem Grundsatze: handle nach einer Maxime, die ihre eigene allgemeine Gültigkeit für jedes vernünftige Wesen zugleich in sich enthält, im Grunde einerlei. Denn, daß ich meine Maxime im Gebrauche der Mittel zu jedem Zwecke auf die Bedingung ihrer Allgemeingültigkeit, als eines Gesetzes für jedes Subjekt einschränken soll, sagt eben so viel, als: das Subjekt der Zwecke, d.i. das vernünftige  Wesen selbst, muß niemals bloß als Mittel, sondern als oberste einschränkende Bedingung im Gebrauche aller Mittel, d.i. jederzeit zugleich als Zweck, allen Maximen der Handlungen zum Grunde gelegt werden. 
	
	\subsection*{tg185.2.14} 
	\textbf{Source : }Grundlegung zur Metaphysik der Sitten/Dritter Abschnitt: Übergang von der Metaphysik der Sitten zur Kritik der reinen praktischen Vernunft/Von der äußersten Grenze aller praktischen Philosophie\\  
	
	\noindent\textbf{Paragraphe : }Die Frage also: wie ein kategorischer Imperativ möglich sei, kann zwar so weit beantwortet werden, als man die einzige Voraussetzung angeben kann, unter der er allein möglich ist, nämlich die Idee der Freiheit, imgleichen als man  die Notwendigkeit dieser Voraussetzung einsehen kann, welches zum praktischen Gebrauche der Vernunft, d.i. zur Überzeugung von der Gültigkeit dieses Imperativs, mithin auch des sittlichen Gesetzes, hinreichend ist, aber wie diese Voraussetzung selbst möglich sei, läßt sich durch keine menschliche Vernunft jemals einsehen. Unter Voraussetzung der Freiheit des Willens einer Intelligenz aber ist die Autonomie desselben, als die formale Bedingung, unter der er allein bestimmt werden kann, eine notwendige Folge. Diese Freiheit des Willens vorauszusetzen, ist auch nicht allein (ohne in Widerspruch mit dem Prinzip der Naturnotwendigkeit in der Verknüpfung der Erscheinungen der Sinnenwelt zu geraten) ganz wohl möglich (wie die spekulative Philosophie zeigen kann), sondern auch, sie praktisch, d.i. in der Idee allen seinen willkürlichen Handlungen, als Bedingung, unterzulegen, ist einem vernünftigen Wesen, das sich seiner Kausalität durch Vernunft, mithin eines Willens (der von Begierden unterschieden ist) bewußt ist, ohne weitere Bedingung notwendig. Wie nun aber reine Vernunft, ohne andere Triebfedern, die irgend woher sonsten genommen sein mögen, für sich selbst praktisch sein, d.i. wie das bloße Prinzip der Allgemeingültigkeit aller ihrer Maximen als Gesetze (welches freilich die Form einer reinen praktischen Vernunft sein würde), ohne alle \match{Materie} (Gegenstand) des Willens, woran man zum voraus irgend ein Interesse nehmen dürfe, für sich selbst eine Triebfeder abgeben, und ein Interesse, welches rein moralisch heißen würde, bewirken, oder mit anderen Worten: wie reine Vernunft praktisch sein könne, das zu erklären, dazu ist alle menschliche Vernunft gänzlich unvermögend, und alle Mühe und Arbeit, hievon Erklärung zu suchen, ist verloren. 
	
	\unnumberedsection{Mathematik (1)} 
	\subsection*{tg187.2.6} 
	\textbf{Source : }Grundlegung zur Metaphysik der Sitten/Fußnoten\\  
	
	\noindent\textbf{Paragraphe : }
	
	3 Man kann, wenn man will, (so wie die reine \match{Mathematik} von der angewandten, die reine Logik von der angewandten unterschieden wird, also) die reine Philosophie der Sitten (Metaphysik) von der angewandten (nämlich auf die menschliche Natur) unterscheiden. Durch diese Benennung wird man auch so fort erinnert, daß die sittlichen Prinzipien nicht auf die Eigenheiten der menschlichen Natur gegründet, sondern für sich a priori bestehend sein müssen, aus solchen aber, wie für jede vernünftige Natur, also auch für die menschliche, praktische Regeln müssen abgeleitet werden können. 
	
	\unnumberedsection{Menge (1)} 
	\subsection*{tg175.2.22} 
	\textbf{Source : }Grundlegung zur Metaphysik der Sitten/Erster Abschnitt: Übergang von der gemeinen sittlichen Vernunfterkenntnis zur philosophischen\\  
	
	\noindent\textbf{Paragraphe : }So sind wir denn in der moralischen Erkenntnis der gemeinen Menschenvernunft bis zu ihrem Prinzip gelangt, welches sie sich zwar freilich nicht so in einer allgemeinen  Form abgesondert denkt, aber doch jederzeit wirklich vor Augen hat und zum Richtmaße ihrer Beurteilung braucht. Es wäre hier leicht zu zeigen, wie sie, mit diesem Kompasse in der Hand, in allen vorkommenden Fällen sehr gut Bescheid wisse, zu unterscheiden, was gut, was böse, pflichtmäßig, oder pflichtwidrig sei, wenn man, ohne sie im mindesten etwas Neues zu lehren, sie nur, wie Sokrates tat, auf ihr eigenes Prinzip aufmerksam macht, und daß es also keiner Wissenschaft und Philosophie bedürfe, um zu wissen, was man zu tun habe, um ehrlich und gut, ja sogar, um weise und tugendhaft zu sein. Das ließe sich auch wohl schon zum voraus vermuten, daß die Kenntnis dessen, was zu tun, mithin auch zu wissen jedem Menschen obliegt, auch jedes, selbst des gemeinsten Menschen Sache sein werde. Hier kann man es doch nicht ohne Bewunderung ansehen, wie das praktische Beurteilungsvermögen vor dem theoretischen im gemeinen Menschenverstande so gar viel voraus habe. In dem letzteren, wenn die gemeine Vernunft es wagt, von den Erfahrungsgesetzen und den Wahrnehmungen der Sinne abzugehen, gerät sie in lauter Unbegreiflichkeiten und Widersprüche mit sich selbst, wenigstens in ein Chaos von Ungewißheit, Dunkelheit und Unbestand. Im praktischen aber fängt die Beurteilungskraft denn eben allererst an, sich recht vorteilhaft zu zeigen, wenn der gemeine Verstand alle sinnliche Triebfedern von praktischen Gesetzen ausschließt. Er wird alsdenn so gar subtil, es mag sein, daß er mit seinem Gewissen, oder anderen Ansprüchen in Beziehung auf das, was recht heißen soll, schikanieren, oder auch den Wert der Handlungen zu seiner eigenen Belehrung aufrichtig bestimmen will, und, was das meiste ist, er kann im letzteren Falle sich eben so gut Hoffnung machen, es recht zu treffen, als es sich immer ein Philosoph versprechen mag, ja ist beinahe noch sicherer hierin, als selbst der letztere, weil dieser doch kein anderes Prinzip als jener haben, sein Urteil aber, durch eine \match{Menge} fremder, nicht zur Sache gehöriger Erwägungen, leicht verwirren und von der geraden Richtung abweichend machen kann. Wäre es demnach nicht ratsamer,  es in moralischen Dingen bei dem gemeinen Vernunfturteil bewenden zu lassen, und höchstens nur Philosophie anzubringen, um das System der Sitten desto vollständiger und faßlicher, imgleichen die Regeln derselben zum Gebrauche (noch mehr aber zum Disputieren) bequemer darzustellen, nicht aber, um selbst in praktischer Absicht den gemeinen Menschenverstand von seiner glücklichen Einfalt abzubringen, und ihn durch Philosophie auf einen neuen Weg der Untersuchung und Belehrung zu bringen? 
	
	\unnumberedsection{Norm (1)} 
	\subsection*{tg174.2.10} 
	\textbf{Source : }Grundlegung zur Metaphysik der Sitten/Vorrede\\  
	
	\noindent\textbf{Paragraphe : }Eine Metaphysik der Sitten ist also unentbehrlich notwendig, nicht bloß aus einem Bewegungsgrunde der Spekulation, um die Quelle der a priori in unserer Vernunft liegenden praktischen Grundsätze zu erforschen, sondern weil die Sitten selber allerlei Verderbnis unterworfen bleiben, so lange jener Leitfaden und oberste \match{Norm} ihrer richtigen Beurteilung fehlt. Denn bei dem, was moralisch gut sein soll, ist es nicht genug, daß es dem sittlichen Gesetze gemäß sei, sondern es muß auch um desselben willen geschehen; widrigenfalls ist jene Gemäßheit nur sehr zufällig und mißlich, weil der unsittliche Grund zwar dann und wann gesetzmäßige, mehrmalen aber gesetzwidrige Handlungen hervorbringen wird. Nun ist aber das sittliche Gesetz, in seiner Reinigkeit und Echtheit (woran eben im Praktischen am meisten gelegen ist), nirgend anders, als in einer reinen Philosophie zu suchen, also muß diese (Metaphysik) vorangehen, und ohne sie kann es überall keine Moralphilosophie geben; selbst verdient diejenige, welche jene reine Prinzipien unter die empirischen mischt, den Namen einer Philosophie nicht (denn dadurch unterscheidet diese sich eben von der gemeinen Vernunfterkenntnis, daß sie, was diese nur vermengt begreift, in abgesonderter Wissenschaft vorträgt), viel weniger einer Moralphilosophie, weil sie eben durch diese Vermengung so gar der Reinigkeit der Sitten selbst Abbruch tut und ihrem eigenen Zwecke zuwider verfährt. 
	
	\unnumberedsection{Physik (1)} 
	\subsection*{tg174.2.6} 
	\textbf{Source : }Grundlegung zur Metaphysik der Sitten/Vorrede\\  
	
	\noindent\textbf{Paragraphe : }Auf solche Weise entspringt die Idee einer zwiefachen Metaphysik, einer Metaphysik der Natur und einer Metaphysik der Sitten. Die \match{Physik} wird also ihren empirischen, aber auch einen rationalen Teil haben; die Ethik gleichfalls; wiewohl hier der empirische Teil besonders praktische Anthropologie, der rationale aber eigentlich Moral heißen könnte. 
	
	\unnumberedsection{Reihe (1)} 
	\subsection*{tg176.2.37} 
	\textbf{Source : }Grundlegung zur Metaphysik der Sitten/Zweiter Abschnitt: Übergang von der populären sittlichen Weltweisheit zur Metaphysik der Sitten\\  
	
	\noindent\textbf{Paragraphe : }1) Einer, der durch eine \match{Reihe} von Übeln, die bis zur Hoffnungslosigkeit angewachsen ist, einen Überdruß am Leben empfindet, ist noch so weit im Besitze seiner Vernunft, daß er sich selbst fragen kann, ob es auch nicht etwa der Pflicht gegen sich selbst zuwider sei, sich das Leben zu nehmen. Nun versucht er: ob die Maxime seiner Handlung wohl ein allgemeines Naturgesetz werden könne. Seine Maxime aber ist: ich mache es mir aus Selbstliebe zum Prinzip, wenn das Leben bei seiner langem Frist mehr Übel droht, als es Annehmlichkeit verspricht, es mir abzukürzen. Es frägt sich nur noch, ob dieses Prinzip der Selbstliebe ein allgemeines Naturgesetz werden könne. Da sieht man aber bald, daß eine Natur, deren Gesetz es wäre, durch dieselbe Empfindung, deren Bestimmung es ist, zur Beförderung des Lebens anzutreiben, das Leben selbst zu zerstören, ihr selbst widersprechen und also nicht als Natur bestehen würde, mithin jene Maxime unmöglich als allgemeines Naturgesetz stattfinden könne, und folglich dem obersten Prinzip aller Pflicht gänzlich widerstreite. 
	
	\unnumberedsection{Richtschnur (1)} 
	\subsection*{tg187.2.26} 
	\textbf{Source : }Grundlegung zur Metaphysik der Sitten/Fußnoten\\  
	
	\noindent\textbf{Paragraphe : }
	
	13 Man denke ja nicht, daß hier das triviale: quod tibi non vis fieri etc. zur \match{Richtschnur} oder Prinzip dienen könne. Denn es ist, obzwar mit verschiedenen Einschränkungen, nur aus jenem abgeleitet; es kann kein allgemeines Gesetz sein, denn es enthält nicht den Grund der Pflichten gegen sich selbst, nicht der Liebespflichten gegen andere (denn mancher würde es gerne eingehen, daß andere ihm nicht wohltun sollen, wenn er es nur überhoben sein dürfte, ihnen Wohltat zu erzeigen), endlich nicht der schuldigen Pflichten gegen einander; denn der Verbrecher würde aus diesem Grunde gegen seine strafenden Richter argumentieren, u.s.w. 
	
	\unnumberedsection{Satz (7)} 
	\subsection*{tg175.2.16} 
	\textbf{Source : }Grundlegung zur Metaphysik der Sitten/Erster Abschnitt: Übergang von der gemeinen sittlichen Vernunfterkenntnis zur philosophischen\\  
	
	\noindent\textbf{Paragraphe : }Der zweite \match{Satz} ist: eine Handlung aus Pflicht hat ihren moralischen Wert nicht in der Absicht, welche dadurch erreicht werden soll, sondern in der Maxime, nach der sie beschlossen wird, hängt also nicht von der Wirklichkeit des Gegenstandes der Handlung ab, sondern bloß von dem Prinzip des Wollens, nach welchem die Handlung, unangesehen aller Gegenstände des Begehrungsvermögens, geschehen ist. Daß die Absichten, die wir bei Handlungen haben mögen, und ihre Wirkungen, als Zwecke und Triebfedern des Willens, den Handlungen keinen unbedingten und moralischen Wert erteilen können, ist aus dem Vorigen klar. Worin kann also dieser Wert liegen, wenn er nicht im Willen, in Beziehung auf deren verhoffte Wirkung, bestehen soll? Er kann nirgend anders liegen, als im Prinzip des Willens, unangesehen der Zwecke, die durch solche Handlung bewirkt werden können; denn der Wille ist mitten inne zwischen seinem Prinzip a priori, welches formell ist, und zwischen seiner Triebfeder a posteriori, welche materiell ist, gleichsam auf einem Scheidewege, und, da er doch irgend wodurch muß bestimmt werden, so wird er durch das formelle Prinzip des Wollens überhaupt bestimmt werden müssen, wenn eine Handlung aus Pflicht geschieht, da ihm alles materielle Prinzip entzogen worden. 
	
	\subsection*{tg176.2.26} 
	\textbf{Source : }Grundlegung zur Metaphysik der Sitten/Zweiter Abschnitt: Übergang von der populären sittlichen Weltweisheit zur Metaphysik der Sitten\\  
	
	\noindent\textbf{Paragraphe : }Nun entsteht die Frage: wie sind alle diese Imperative möglich? Diese Frage verlangt nicht zu wissen, wie die Vollziehung der Handlung, welche der Imperativ gebietet, sondern wie bloß die Nötigung des Willens, die der Imperativ in der Aufgabe ausdrückt, gedacht werden könne. Wie ein Imperativ der Geschicklichkeit möglich sei, bedarf wohl keiner besondern Erörterung. Wer den Zweck will, will (so fern die Vernunft auf seine Handlungen entscheidenden Einfluß hat) auch das dazu unentbehrlich notwendige Mittel, das in seiner Gewalt ist. Dieser \match{Satz} ist, was das Wollen betrifft, analytisch; denn in dem Wollen eines Objekts, als meiner Wirkung, wird schon meine Kausalität, als handelnder Ursache, d.i. der Gebrauch der Mittel, gedacht, und der Imperativ zieht den Begriff notwendiger Handlungen zu diesem Zwecke schon aus dem Begriff eines Wollens dieses Zwecks heraus (die Mittel selbst zu einer vorgesetzten Absicht  zu bestimmen, dazu gehören allerdings synthetische Sätze, die aber nicht den Grund betreffen, den Actus des Willens, sondern das Objekt wirklich zu machen). Daß, um eine Linie nach einem sichern Prinzip in zwei gleiche Teile zu teilen, ich aus den Enden derselben zwei Kreuzbogen machen müsse, das lehrt die Mathematik freilich nur durch synthetische Sätze; aber daß, wenn ich weiß, durch solche Handlung allein könne die gedachte Wirkung geschehen, ich, wenn ich die Wirkung vollständig will, auch die Handlung wolle, die dazu erfoderlich ist, ist ein analytischer Satz; denn etwas als eine auf gewisse Art durch mich mögliche Wirkung, und mich, in Ansehung ihrer, auf dieselbe Art handelnd vorstellen, ist ganz einerlei. 
	
	\subsection*{tg176.2.61} 
	\textbf{Source : }Grundlegung zur Metaphysik der Sitten/Zweiter Abschnitt: Übergang von der populären sittlichen Weltweisheit zur Metaphysik der Sitten\\  
	
	\noindent\textbf{Paragraphe : }Also würde das Prinzip eines jeden menschlichen Willens, als eines durch alle seine Maximen allgemein gesetzgebenden Willens,
	
	
	14
	wenn es sonst mit ihm nur seine Richtigkeit hätte, sich zum kategorischen Imperativ darin gar wohl schicken, daß es, eben um der Idee der allgemeinen Gesetzgebung willen, sich auf kein Interesse gründet und also unter allen möglichen imperativen allein unbedingt sein kann; oder noch besser, indem wir den \match{Satz} umkehren: wenn es einen kategorischen Imperativ gibt (d.i. ein Gesetz für jeden Willen eines vernünftigen Wesens), so kann er nur gebieten, alles aus der Maxime seines Willens, als eines solchen, zu tun, der zugleich sich selbst als allgemein gesetzgebend zum Gegenstande haben könnte; denn alsdenn nur ist das praktische Prinzip und der Imperativ, dem er gehorcht, unbedingt, weil er gar kein Interesse zum Grunde haben kann. 
	
	\subsection*{tg177.2.2} 
	\textbf{Source : }Grundlegung zur Metaphysik der Sitten/Zweiter Abschnitt: Übergang von der populären sittlichen Weltweisheit zur Metaphysik der Sitten/Die Autonomie des Willens als oberstes Prinzip der Sittlichkeit\\  
	
	\noindent\textbf{Paragraphe : }Autonomie des Willens ist die Beschaffenheit des Willens, dadurch derselbe ihm selbst (unabhängig von aller Beschaffenheit der Gegenstände des Wollens) ein Gesetz ist. Das Prinzip der Autonomie ist also: nicht anders zu wählen, als so, daß die Maximen seiner Wahl in demselben Wollen  zugleich als allgemeines Gesetz mit begriffen sein. Daß diese praktische Regel ein Imperativ sei, d.i. der Wille jedes vernünftigen Wesens an sie als Bedingung notwendig gebunden sei, kann durch bloße Zergliederung der in ihm vorkommenden Begriffe nicht bewiesen werden, weil es ein synthetischer \match{Satz} ist; man müßte über die Erkenntnis der Objekte und zu einer Kritik des Subjekts, d.i. der reinen praktischen Vernunft, hinausgehen, denn völlig a priori muß dieser synthetische Satz, der apodiktisch gebietet, erkannt werden können, dieses Geschäft aber gehört nicht in gegenwärtigen Abschnitt. Allein, daß gedachtes Prinzip der Autonomie das alleinige Prinzip der Moral sei, läßt sich durch bloße Zergliederung der Begriffe der Sittlichkeit gar wohl dartun. Denn dadurch findet sich, daß ihr Prinzip ein kategorischer Imperativ sein müsse, dieser aber nichts mehr oder weniger als gerade diese Autonomie gebiete. 
	
	\subsection*{tg179.2.10} 
	\textbf{Source : }Grundlegung zur Metaphysik der Sitten/Zweiter Abschnitt: Übergang von der populären sittlichen Weltweisheit zur Metaphysik der Sitten/Einteilung aller möglichen Prinzipien der Sittlichkeit aus dem angenommenen Grundbegriffe der Heteronomie\\  
	
	\noindent\textbf{Paragraphe : }
	Wie ein solcher synthetischer praktischer \match{Satz} a priori möglich und warum er notwendig sei, ist eine Aufgabe, deren Auflösung nicht mehr binnen den Grenzen der Metaphysik der Sitten liegt, auch haben wir seine Wahrheit hier nicht behauptet, vielweniger vorgegeben, einen Beweis derselben in unserer Gewalt zu haben. Wir zeigten nur durch Entwickelung des einmal allgemein im Schwange gehenden Begriffs der Sittlichkeit: daß eine Autonomie des Willens demselben, unvermeidlicher Weise, anhänge, oder vielmehr zum Grunde liege. Wer also Sittlichkeit für Etwas, und nicht für eine chimärische Idee ohne Wahrheit, hält, muß das angeführte Prinzip derselben zugleich einräumen. Dieser Abschnitt war also, eben so, wie der erste, bloß analytisch. Daß nun Sittlichkeit kein Hirngespinst sei, welches alsdenn folgt, wenn der kategorische Imperativ und mit ihm die Autonomie des Willens wahr, und als ein Prinzip a priori schlechterdings notwendig ist, erfodert einen möglichen synthetischen Gebrauch der reinen praktischen Vernunft, den wir aber nicht wagen dürfen, ohne eine Kritik dieses Vernunftvermögens selbst voranzuschicken, von welcher wir in dem letzten Abschnitte die zu unserer Absicht hinlängliche Hauptzüge darzustellen haben. 
	
	\subsection*{tg183.2.13} 
	\textbf{Source : }Grundlegung zur Metaphysik der Sitten/Dritter Abschnitt: Übergang von der Metaphysik der Sitten zur Kritik der reinen praktischen Vernunft/Von dem Interesse, welches den Ideen der Sittlichkeit anhängt\\  
	
	\noindent\textbf{Paragraphe : }Nun ist der Verdacht, den wir oben rege machten, gehoben, als wäre ein geheimer Zirkel in unserem Schlüsse aus der Freiheit auf die Autonomie und aus dieser aufs sittliche Gesetz enthalten, daß wir nämlich vielleicht die Idee der Freiheit nur um des sittlichen Gesetzes willen zum Grunde legten, um dieses nachher aus der Freiheit wiederum zu schließen, mithin von jenem gar keinen Grund angeben könnten, sondern es nur als Erbittung eines Prinzips, das uns gutgesinnte Seelen wohl gerne einräumen werden, welches wir aber niemals als einen erweislichen \match{Satz} aufstellen könnten. Denn jetzt sehen wir, daß, wenn wir uns als frei denken, so versetzen wir uns als Glieder in die Verstandeswelt, und erkennen die Autonomie des Willens, samt ihrer Folge, der Moralität; denken wir uns aber als verpflichtet, so betrachten wir uns als zur Sinnenwelt und doch zugleich zur Verstandeswelt gehörig. 
	
	\subsection*{tg187.2.24} 
	\textbf{Source : }Grundlegung zur Metaphysik der Sitten/Fußnoten\\  
	
	\noindent\textbf{Paragraphe : }
	
	12 Diesen \match{Satz} stelle ich hier als Postulat auf. Im letzten Abschnitte wird man die Gründe dazu finden. 
	
	\unnumberedsection{Schatzung (4)} 
	\subsection*{tg175.2.10} 
	\textbf{Source : }Grundlegung zur Metaphysik der Sitten/Erster Abschnitt: Übergang von der gemeinen sittlichen Vernunfterkenntnis zur philosophischen\\  
	
	\noindent\textbf{Paragraphe : }Um aber den Begriff eines an sich selbst hochzuschätzenden und ohne weitere Absicht guten Willens, so wie er schon dem natürlichen gesunden Verstande beiwohnet und nicht so wohl gelehret als vielmehr nur aufgeklärt zu werden bedarf, diesen Begriff, der in der \match{Schätzung} des ganzen Werts unserer Handlungen immer obenan steht und die Bedingung alles übrigen ausmacht, zu entwickeln: wollen wir den Begriff der Pflicht vor uns nehmen, der den eines guten Willens, obzwar unter gewissen subjektiven Einschränkungen und Hindernissen, enthält, die aber doch, weit gefehlt, daß sie ihn verstecken und unkenntlich machen sollten, ihn vielmehr durch Abstechung heben und desto heller hervorscheinen lassen. 
	
	\subsection*{tg175.2.21} 
	\textbf{Source : }Grundlegung zur Metaphysik der Sitten/Erster Abschnitt: Übergang von der gemeinen sittlichen Vernunfterkenntnis zur philosophischen\\  
	
	\noindent\textbf{Paragraphe : }Was ich also zu tun habe, damit mein Wollen sittlich gut sei, darzu brauche ich gar keine weit ausholende Scharfsinnigkeit. Unerfahren in Ansehung des Weltlaufs, unfähig, auf alle sich eräugnende Vorfälle desselben gefaßt zu sein, frage ich mich nur: Kannst du auch wollen, daß deine Maxime ein allgemeines Gesetz werde? wo nicht, so ist sie verwerflich, und das zwar nicht um eines dir, oder auch anderen, daraus bevorstehenden Nachteils willen, sondern weil sie nicht als Prinzip in eine mögliche allgemeine Gesetzgebung passen kann, für diese aber zwingt mir die Vernunft unmittelbare Achtung ab, von der ich zwar jetzt noch nicht einsehe, worauf sie sich gründe (welches der Philosoph untersuchen mag), wenigstens aber doch so viel verstehe: daß es eine \match{Schätzung} des Wertes sei, welcher allen Wert dessen, was durch Neigung angepriesen wird, weit überwiegt, und daß die Notwendigkeit meiner Handlungen aus reiner Achtung fürs praktische Gesetz dasjenige sei, was die Pflicht ausmacht, der jeder andere Bewegungsgrund weichen muß, weil sie die Bedingung eines an sich guten Willens ist, dessen Wert über alles geht. 
	
	\subsection*{tg175.2.6} 
	\textbf{Source : }Grundlegung zur Metaphysik der Sitten/Erster Abschnitt: Übergang von der gemeinen sittlichen Vernunfterkenntnis zur philosophischen\\  
	
	\noindent\textbf{Paragraphe : }Es liegt gleichwohl in dieser Idee von dem absoluten Werte des bloßen Willens, ohne einigen Nutzen bei \match{Schätzung} desselben in Anschlag zu bringen, etwas so Befremdliches, daß, unerachtet aller Einstimmung selbst der gemeinen  Vernunft mit derselben, dennoch ein Verdacht entspringen muß, daß vielleicht bloß hochfliegende Phantasterei ingeheim zum Grunde liege, und die Natur in ihrer Absicht, warum sie unserm Willen Vernunft zur Regiererin beigelegt habe, falsch verstanden sein möge. Daher wollen wir diese Idee aus diesem Gesichtspunkte auf die Prüfung stellen. 
	
	\subsection*{tg176.2.73} 
	\textbf{Source : }Grundlegung zur Metaphysik der Sitten/Zweiter Abschnitt: Übergang von der populären sittlichen Weltweisheit zur Metaphysik der Sitten\\  
	
	\noindent\textbf{Paragraphe : }Und was ist es denn nun, was die sittlich gute Gesinnung oder die Tugend berechtigt, so hohe Ansprüche zu machen? Es ist nichts Geringeres als der Anteil, den sie dem vernünftigen Wesen an der allgemeinen Gesetzgebung verschafft, und es hiedurch zum Gliede in einem möglichen Reiche der Zwecke tauglich macht, wozu es durch seine eigene Natur schon bestimmt war, als Zweck an sich selbst und eben darum als gesetzgebend im Reiche der Zwecke, in Ansehung aller Naturgesetze als frei, nur denjenigen allein gehorchend, die es selbst gibt und nach welchen seine Maximen zu einer allgemeinen Gesetzgebung (der er sich zugleich selbst unterwirft) gehören können. Denn es hat nichts einen Wert, als den, welchen ihm das Gesetz bestimmt. Die Gesetzgebung selbst aber, die allen Wert bestimmt, muß eben darum eine Würde, d.i. unbedingten, unvergleichbaren Wert haben, für welchen das Wort Achtung allein den geziemenden Ausdruck der \match{Schätzung} abgibt, die ein vernünftiges Wesen über sie anzustellen hat. Autonomie ist also der Grund der Würde der menschlichen und jeder vernünftigen Natur. 
	
	\unnumberedsection{Seite (3)} 
	\subsection*{tg176.2.7} 
	\textbf{Source : }Grundlegung zur Metaphysik der Sitten/Zweiter Abschnitt: Übergang von der populären sittlichen Weltweisheit zur Metaphysik der Sitten\\  
	
	\noindent\textbf{Paragraphe : }Man könnte auch der Sittlichkeit nicht übler raten, als wenn man sie von Beispielen entlehnen wollte. Denn jedes Beispiel, was mir davon vorgestellt wird, muß selbst zuvor nach Prinzipien der Moralität beurteilt werden, ob es auch würdig sei, zum ursprünglichen Beispiele, d.i. zum Muster zu dienen, keinesweges aber kann es den Begriff derselben zu oberst an die Hand geben. Selbst der Heilige des Evangelii muß zuvor mit unserm Ideal der sittlichen Vollkommenheit verglichen werden, ehe man ihn dafür erkennt; auch sagt er von sich selbst: was nennt ihr mich (den ihr sehet) gut, niemand ist gut (das Urbild des Guten) als der einige Gott (den ihr nicht sehet). Woher haben wir aber den Begriff von Gott, als dem höchsten Gut? Lediglich aus der Idee, die die Vernunft a priori von sittlicher Vollkommenheit entwirft, und mit dem Begriffe eines freien Willens unzertrennlich verknüpft. Nachahmung findet im Sittlichen gar nicht statt, und Beispiele dienen nur zur Aufmunterung, d.i. sie setzen die Tunlichkeit dessen, was das Gesetz gebietet, außer Zweifel, sie machen das, was die praktische Regel allgemeiner ausdrückt, anschaulich, können aber niemals  berechtigen, ihr wahres Original, das in der Vernunft liegt, bei \match{Seite} zu setzen und sich nach Beispielen zu richten. 
	
	\subsection*{tg184.2.2} 
	\textbf{Source : }Grundlegung zur Metaphysik der Sitten/Dritter Abschnitt: Übergang von der Metaphysik der Sitten zur Kritik der reinen praktischen Vernunft/Wie ist ein kategorischer Imperativ möglich\\  
	
	\noindent\textbf{Paragraphe : }Das vernünftige Wesen zählt sich als Intelligenz zur Verstandeswelt, und, bloß als eine zu dieser gehörige wirkende Ursache, nennt es seine Kausalität einen Willen. Von der anderen \match{Seite} ist es sich seiner doch auch als eines Stücks der Sinnenwelt bewußt, in welcher seine Handlungen, als bloße Erscheinungen jener Kausalität, angetroffen werden, deren Möglichkeit aber aus dieser, die wir nicht kennen, nicht eingesehen werden kann, sondern an deren Statt jene Handlungen als bestimmt durch andere Erscheinungen, nämlich Begierden und Neigungen, als zur Sinnenwelt gehörig, eingesehen werden müssen. Als bloßen Gliedes der Verstandeswelt würden also alle meine Handlungen dem  Prinzip der Autonomie des reinen Willens vollkommen gemäß sein; als bloßen Stücks der Sinnenwelt würden sie gänzlich dem Naturgesetz der Begierden und Neigungen, mithin der Heteronomie der Natur gemäß genommen werden müssen. (Die ersteren würden auf dem obersten Prinzip der Sittlichkeit, die zweiten der Glückseligkeit, beruhen.) Weil aber die Verstandeswelt den Grund der Sinnenwelt, mithin auch der Gesetze derselben, enthält, also in Ansehung meines Willens (der ganz zur Verstandeswelt gehört) unmittelbar gesetzgebend ist, und also auch als solche gedacht werden muß, so werde ich mich als Intelligenz, obgleich andererseits wie ein zur Sinnenwelt gehöriges Wesen, dennoch dem Gesetze der ersteren, d.i. der Vernunft, die in der Idee der Freiheit das Gesetz derselben enthält, und also der Autonomie des Willens unterworfen erkennen, folglich die Gesetze der Verstandeswelt für mich als Imperativen und die diesem Prinzip gemäße Handlungen als Pflichten ansehen müssen. 
	
	\subsection*{tg185.2.2} 
	\textbf{Source : }Grundlegung zur Metaphysik der Sitten/Dritter Abschnitt: Übergang von der Metaphysik der Sitten zur Kritik der reinen praktischen Vernunft/Von der äußersten Grenze aller praktischen Philosophie\\  
	
	\noindent\textbf{Paragraphe : }Alle Menschen denken sich dem Willen nach als frei. Daher kommen alle Urteile über Handlungen als solche, die hätten geschehen sollen, ob sie gleich nicht geschehen sind. Gleichwohl ist diese Freiheit kein Erfahrungsbegriff, und kann es auch nicht sein, weil er immer bleibt, obgleich die Erfahrung das Gegenteil von denjenigen Foderungen zeigt, die unter Voraussetzung derselben als notwendig vorgestellt werden. Auf der anderen \match{Seite} ist es eben so notwendig, daß alles, was geschieht, nach Naturgesetzen unausbleiblich bestimmt sei, und diese Naturnotwendigkeit ist auch kein Erfahrungsbegriff, eben darum, weil er den Begriff der Notwendigkeit, mithin einer Erkenntnis a priori, bei sich führet. Aber dieser Begriff von einer Natur wird durch Erfahrung bestätigt, und muß selbst unvermeidlich vorausgesetzt werden, wenn Erfahrung, d.i. nach allgemeinen Gesetzen zusammenhängende Erkenntnis der Gegenstände der Sinne, möglich sein soll. Daher ist Freiheit nur eine Idee der Vernunft, deren objektive Realität an sich zweifelhaft ist, Natur aber ein Verstandesbegriff, der seine Realität an Beispielen der Erfahrung beweiset und notwendig beweisen muß. 
	
	\unnumberedsection{Seiten (1)} 
	\subsection*{tg183.2.11} 
	\textbf{Source : }Grundlegung zur Metaphysik der Sitten/Dritter Abschnitt: Übergang von der Metaphysik der Sitten zur Kritik der reinen praktischen Vernunft/Von dem Interesse, welches den Ideen der Sittlichkeit anhängt\\  
	
	\noindent\textbf{Paragraphe : }Um deswillen muß ein vernünftiges Wesen sich selbst, als Intelligenz (also nicht von \match{Seiten} seiner untern Kräfte), nicht als zur Sinnen-, sondern zur Verstandeswelt gehörig, ansehen; mithin hat es zwei Standpunkte, daraus es sich selbst betrachten, und Gesetze des Gebrauchs seiner Kräfte, folglich aller seiner Handlungen, erkennen kann, einmal, so fern es zur Sinnenwelt gehört, unter Naturgesetzen (Heteronomie), zweitens, als zur intelligibelen Welt gehörig, unter Gesetzen, die, von der Natur unabhängig, nicht empirisch, sondern bloß in der Vernunft gegründet sein. 
	
	\unnumberedsection{Starke (1)} 
	\subsection*{tg175.2.13} 
	\textbf{Source : }Grundlegung zur Metaphysik der Sitten/Erster Abschnitt: Übergang von der gemeinen sittlichen Vernunfterkenntnis zur philosophischen\\  
	
	\noindent\textbf{Paragraphe : }
	Wohltätig sein, wo man kann, ist Pflicht, und überdem gibt es manche so teilnehmend gestimmte Seelen, daß sie, auch ohne einen andern Bewegungsgrund der Eitelkeit, oder des Eigennutzes, ein inneres Vergnügen daran finden, Freude um sich zu verbreiten, und die sich an der Zufriedenheit anderer, so fern sie ihr Werk ist, ergötzen können. Aber ich behaupte, daß in solchem Falle dergleichen Handlung, so pflichtmäßig, so liebenswürdig sie auch ist, dennoch keinen wahren sittlichen Wert habe, sondern mit andern Neigungen zu gleichen Paaren gehe, z. E. der Neigung nach Ehre, die, wenn sie glücklicherweise auf das trifft, was in der Tat gemeinnützig und pflichtmäßig, mithin ehrenwert ist, Lob und Aufmunterung, aber nicht Hochschätzung verdient; denn der Maxime fehlt der sittliche Gehalt, nämlich solche Handlungen nicht aus Neigung, sondern aus Pflicht zu tun. Gesetzt also, das Gemüt jenes Menschenfreundes wäre vom eigenen Gram umwölkt, der alle Teilnehmung an anderer Schicksal auslöscht, er hätte immer noch Vermögen, andern Notleidenden wohlzutun, aber fremde Not rührte ihn nicht, weil er mit seiner eigenen gnug beschäftigt ist, und nun, da keine Neigung ihn mehr dazu anreizt, risse er sich doch aus dieser tödlichen Unempfindlichkeit heraus, und täte die Handlung ohne alle Neigung, lediglich aus Pflicht, alsdenn hat sie allererst ihren echten moralischen Wert. Noch mehr: wenn die Natur diesem oder jenem überhaupt wenig Sympathie ins Herz gelegt hätte, wenn er (übrigens ein ehrlicher Mann) von Temperament kalt und gleichgültig gegen die Leiden anderer wäre, vielleicht, weil er, selbst gegen seine eigene mit der besondern Gabe der Geduld und aushaltenden \match{Stärke} versehen, dergleichen bei jedem andern auch voraussetzt, oder gar fordert; wenn die Natur einen solchen Mann (welcher wahrlich nicht ihr schlechtestes Produkt sein würde) nicht eigentlich zum Menschenfreunde gebildet hätte, würde er denn nicht noch in sich einen Quell finden, sich selbst einen weit höhern Wert zu geben, als der eines gutartigen Temperaments sein mag? Allerdings! gerade da hebt der Wert des Charakters an, der  moralisch und ohne alle Vergleichung der höchste ist, nämlich daß er wohltue, nicht aus Neigung, sondern aus Pflicht. 
	
	\unnumberedsection{Summe (3)} 
	\subsection*{tg175.2.14} 
	\textbf{Source : }Grundlegung zur Metaphysik der Sitten/Erster Abschnitt: Übergang von der gemeinen sittlichen Vernunfterkenntnis zur philosophischen\\  
	
	\noindent\textbf{Paragraphe : }Seine eigene Glückseligkeit sichern, ist Pflicht (wenigstens indirekt), denn der Mangel der Zufriedenheit mit seinem Zustande, in einem Gedränge von vielen Sorgen und mitten unter unbefriedigten Bedürfnissen, könnte leicht eine große Versuchung zu Übertretung der Pflichten werden. Aber, auch ohne hier auf Pflicht zu sehen, haben alle Menschen schon von selbst die mächtigste und innigste Neigung zur Glückseligkeit, weil sich gerade in dieser Idee alle Neigungen zu einer \match{Summe} vereinigen. Nur ist die Vorschrift der Glückseligkeit mehrenteils so beschaffen, daß sie einigen Neigungen großen Abbruch tut und doch der Mensch sich von der Summe der Befriedigung aller unter dem Namen der Glückseligkeit keinen bestimmten und sichern Begriff machen kann; daher nicht zu verwundern ist, wie eine einzige, in Ansehung dessen, was sie verheißt, und der Zeit, worin ihre Befriedigung erhalten werden kann, bestimmte Neigung eine schwankende Idee überwiegen könne, und der Mensch, z.B. ein Podagrist wählen könne, zu genießen was ihm schmeckt und zu leiden was er kann, weil er, nach seinem Überschlage, hier wenigstens, sich nicht durch vielleicht grundlose Erwartungen eines Glücks, das in der Gesundheit stecken soll, um den Genuß des gegenwärtigen Augenblicks gebracht hat. Aber auch in diesem Falle, wenn die allgemeine Neigung zur Glückseligkeit seinen Willen nicht bestimmte, wenn Gesundheit für ihn wenigstens nicht so notwendig in diesen Überschlag gehörete, so bleibt noch hier, wie in allen andern Fällen, ein Gesetz übrig, nämlich seine Glückseligkeit zu befördern, nicht aus Neigung, sondern aus Pflicht, und da hat sein Verhalten allererst den eigentlichen moralischen Wert. 
	
	\subsection*{tg175.2.5} 
	\textbf{Source : }Grundlegung zur Metaphysik der Sitten/Erster Abschnitt: Übergang von der gemeinen sittlichen Vernunfterkenntnis zur philosophischen\\  
	
	\noindent\textbf{Paragraphe : }Der gute Wille ist nicht durch das, was er bewirkt, oder ausrichtet, nicht durch seine Tauglichkeit zu Erreichung irgend eines vorgesetzten Zweckes, sondern allein durch das Wollen, d.i. an sich, gut, und, für sich selbst betrachtet, ohne Vergleich weit höher zu schätzen, als alles, was durch ihn zu Gunsten irgend einer Neigung, ja, wenn man will, der \match{Summe} aller Neigungen, nur immer zu Stande gebracht werden könnte. Wenn gleich durch eine besondere Ungunst des Schicksals, oder durch kärgliche Ausstattung einer stiefmütterlichen Natur, es diesem Willen gänzlich an Vermögen fehlete, seine Absicht durchzusetzen; wenn bei seiner größten Bestrebung dennoch nichts von ihm ausgerichtet würde, und nur der gute Wille (freilich nicht etwa ein bloßer Wunsch, sondern als die Aufbietung aller Mittel, so weit sie in unserer Gewalt sind) übrig bliebe: so würde er wie ein Juwel doch für sich selbst glänzen, als etwas, das seinen vollen Wert in sich selbst hat. Die Nützlichkeit oder Fruchtlosigkeit kann diesem Werte weder etwas zusetzen, noch abnehmen. Sie würde gleichsam nur die Einfassung sein, um ihn im gemeinen Verkehr besser handhaben zu können, oder die Aufmerksamkeit derer, die noch nicht gnug Kenner sind, auf sich zu ziehen, nicht aber, um ihn Kennern zu empfehlen, und seinen Wert zu bestimmen. 
	
	\subsection*{tg179.2.5} 
	\textbf{Source : }Grundlegung zur Metaphysik der Sitten/Zweiter Abschnitt: Übergang von der populären sittlichen Weltweisheit zur Metaphysik der Sitten/Einteilung aller möglichen Prinzipien der Sittlichkeit aus dem angenommenen Grundbegriffe der Heteronomie\\  
	
	\noindent\textbf{Paragraphe : }Unter den rationalen, oder Vernunftgründen der Sittlichkeit ist doch der ontologische Begriff der Vollkommenheit (so leer, so unbestimmt, mithin unbrauchbar er auch ist, um in dem unermeßlichen Felde möglicher Realität  die für uns schickliche größte \match{Summe} auszufinden, so sehr er auch, um die Realität, von der hier die Rede ist, spezifisch von jeder anderen zu unterscheiden, einen unvermeidlichen Hang hat, sich im Zirkel zu drehen, und die Sittlichkeit, die er erklären soll, ingeheim vorauszusetzen nicht vermeiden kann) dennoch besser als der theologische Begriff, sie von einem göttlichen allervollkommensten Willen abzuleiten, nicht bloß deswegen, weil wir seine Vollkommenheit doch nicht anschauen, sondern sie von unseren Begriffen, unter denen der der Sittlichkeit der vornehmste ist, allein ableiten können, sondern weil, wenn wir dieses nicht tun (wie es denn, wenn es geschähe, ein grober Zirkel im Erklären sein würde), der uns noch übrige Begriff seines Willens aus den Eigenschaften der Ehr- und Herrschbegierde, mit den furchtbaren Vorstellungen der Macht und des Racheifers verbunden, zu einem System der Sitten, welches der Moralität gerade entgegen gesetzt wäre, die Grundlage machen müßte. 
	
	\unnumberedsection{Umfang (1)} 
	\subsection*{tg176.2.12} 
	\textbf{Source : }Grundlegung zur Metaphysik der Sitten/Zweiter Abschnitt: Übergang von der populären sittlichen Weltweisheit zur Metaphysik der Sitten\\  
	
	\noindent\textbf{Paragraphe : }Aus dem Angeführten erhellet: daß alle sittliche Begriffe völlig a priori in der Vernunft ihren Sitz und Ursprung haben, und dieses zwar in der gemeinsten Menschenvernunft eben sowohl, als der im höchsten Maße spekulativen; daß sie von keinem empirischen und darum bloß zufälligen Erkenntnisse  abstrahiert werden können; daß in dieser Reinigkeit ihres Ursprungs eben ihre Würde liege, um uns zu obersten praktischen Prinzipien zu dienen; daß man jedesmal so viel, als man Empirisches hinzu tut, so viel auch ihrem echten Einflusse und dem uneingeschränkten Werte der Handlungen entziehe; daß es nicht allein die größte Notwendigkeit in theoretischer Absicht, wenn es bloß auf Spekulation ankommt, erfodere, sondern auch von der größten praktischen Wichtigkeit sei, ihre Begriffe und Gesetze aus reiner Vernunft zu schöpfen, rein und unvermengt vorzutragen, ja den \match{Umfang} dieses ganzen praktischen oder reinen Vernunfterkenntnisses, d.i. das ganze Vermögen der reinen praktischen Vernunft, zu bestimmen, hierin aber nicht, wie es wohl die spekulative Philosophie erlaubt, ja gar bisweilen notwendig findet, die Prinzipien von der besondern Natur der menschlichen Vernunft abhängig zu machen, sondern darum, weil moralische Gesetze für jedes vernünftige Wesen überhaupt gelten sollen, sie schon aus dem allgemeinen Begriffe eines vernünftigen Wesens überhaupt abzuleiten, und auf solche Weise alle Moral, die zu ihrer Anwendung auf Menschen der Anthropologie bedarf, zuerst unabhängig von dieser als reine Philosophie, d.i. als Metaphysik, vollständig (welches sich in dieser Art ganz abgesonderter Erkenntnisse wohl tun läßt) vorzutragen, wohl bewußt, daß es, ohne im Be sitze derselben zu sein, vergeblich sei, ich will nicht sagen, das Moralische der Pflicht in allem, was pflichtmäßig ist, genau für die spekulative Beurteilung zu bestimmen, sondern so gar im bloß gemeinen und praktischen Gebrauche, vornehmlich der moralischen Unterweisung, unmöglich sei, die Sitten auf ihre echte Prinzipien zu gründen und dadurch reine moralische Gesinnungen zu bewirken und zum höchsten Weltbesten den Gemütern einzupfropfen. 
	
	\unnumberedsection{Unterschied (4)} 
	\subsection*{tg174.2.3} 
	\textbf{Source : }Grundlegung zur Metaphysik der Sitten/Vorrede\\  
	
	\noindent\textbf{Paragraphe : }Alle Vernunfterkenntnis ist entweder material, und betrachtet irgend ein Objekt; oder formal, und beschäftigt sich bloß mit der Form des Verstandes und der Vernunft selbst, und den allgemeinen Regeln des Denkens überhaupt, ohne \match{Unterschied} der Objekte. Die formale Philosophie heißt Logik, die materiale aber, welche es mit bestimmten Gegenständen und den Gesetzen zu tun hat, denen sie unterworfen sind, ist wiederum zwiefach. Denn diese Gesetze sind entweder Gesetze der Natur, oder der Freiheit. Die Wissenschaft von der ersten heißt Physik, die der andern ist Ethik; jene wird auch Naturlehre, diese Sittenlehre genannt. 
	
	\subsection*{tg175.2.11} 
	\textbf{Source : }Grundlegung zur Metaphysik der Sitten/Erster Abschnitt: Übergang von der gemeinen sittlichen Vernunfterkenntnis zur philosophischen\\  
	
	\noindent\textbf{Paragraphe : }Ich übergehe hier alle Handlungen, die schon als pflichtwidrig erkannt werden, ob sie gleich in dieser oder jener Absicht nützlich sein mögen; denn bei denen ist gar nicht einmal die Frage, ob sie aus Pflicht geschehen sein mögen,  da sie dieser sogar widerstreiten. Ich setze auch die Handlungen bei Seite, die würklich pflichtmäßig sind, zu denen aber Menschen unmittelbar keine Neigung haben, sie aber dennoch ausüben, weil sie durch eine andere Neigung dazu getrieben werden. Denn da läßt sich leicht unterscheiden, ob die pflichtmäßige Handlung aus Pflicht oder aus selbstsüchtiger Absicht geschehen sei. Weit schwerer ist dieser \match{Unterschied} zu bemerken, wo die Handlung pflichtmäßig ist und das Subjekt noch überdem unmittelbare Neigung zu ihr hat. Z.B. es ist allerdings pflichtmäßig, daß der Krämer seinen unerfahrnen Käufer nicht überteure, und, wo viel Verkehr ist, tut dieses auch der kluge Kaufmann nicht, sondern hält einen festgesetzten allgemeinen Preis für jedermann, so daß ein Kind eben so gut bei ihm kauft, als jeder anderer. Man wird also ehrlich bedient; allein das ist lange nicht genug, um deswegen zu glauben, der Kaufmann habe aus Pflicht und Grundsätzen der Ehrlichkeit so verfahren; sein Vorteil erforderte es; daß er aber überdem noch eine unmittelbare Neigung zu den Käufern haben sollte, um gleichsam aus Liebe keinem vor dem andern im Preise den Vorzug zu geben, läßt sich hier nicht annehmen. Also war die Handlung weder aus Pflicht, noch aus unmittelbarer Neigung, sondern bloß in eigennütziger Absicht geschehen. 
	
	\subsection*{tg179.2.4} 
	\textbf{Source : }Grundlegung zur Metaphysik der Sitten/Zweiter Abschnitt: Übergang von der populären sittlichen Weltweisheit zur Metaphysik der Sitten/Einteilung aller möglichen Prinzipien der Sittlichkeit aus dem angenommenen Grundbegriffe der Heteronomie\\  
	
	\noindent\textbf{Paragraphe : }
	Empirische Prinzipien taugen überall nicht dazu, um moralische Gesetze darauf zu gründen. Denn die Allgemeinheit, mit der sie für alle vernünftige Wesen ohne \match{Unterschied} gelten sollen, die unbedingte praktische Notwendigkeit, die ihnen dadurch auferlegt wird, fällt weg, wenn der Grund derselben von der besonderen Einrichtung der menschlichen Natur, oder den zufälligen Umständen hergenommen wird, darin sie gesetzt ist. Doch  ist das Prinzip der eigenen Glückseligkeit am meisten verwerflich, nicht bloß deswegen, weil es falsch ist, und die Erfahrung dem Vorgeben, als ob das Wohlbefinden sich jederzeit nach dem Wohlverhalten richte, widerspricht, auch nicht bloß, weil es gar nichts zur Gründung der Sittlichkeit beiträgt, indem es ganz was anderes ist, einen glücklichen, als einen guten Menschen, und diesen klug und auf seinen Vorteil abgewitzt, als ihn tugendhaft zu machen: sondern, weil es der Sittlichkeit Triebfedern unterlegt, die sie eher untergraben und ihre ganze Erhabenheit zernichten, indem sie die Bewegursachen zur Tugend mit denen zum Laster in eine Klasse stellen und nur den Kalkül besser ziehen lehren, den spezifischen Unterschied beider aber ganz und gar auslöschen: dagegen das moralische Gefühl, dieser vermeintliche besondere Sinn
	
	
	16
	(so seicht auch die Berufung auf selbigen ist, indem diejenigen, die nicht denken können, selbst in dem, was bloß auf allgemeine Gesetze ankommt, sich durchs Fühlen auszuhelfen glauben, so wenig auch Gefühle, die dem Grade nach von Natur unendlich von einander unterschieden sind, einen gleichen Maßstab des Guten und Bösen abgeben; auch einer durch sein Gefühl für andere gar nicht gültig urteilen kann), dennoch der Sittlichkeit und ihrer Würde dadurch näher bleibt, daß er der Tugend die Ehre beweist, das Wohlgefallen und die Hochschätzung für sie ihr unmittelbar zuzuschreiben, und ihr nicht gleichsam ins Gesicht sagt, daß es nicht ihre Schönheit, sondern nur der Vorteil sei, der uns an sie knüpfe. 
	
	\subsection*{tg183.2.8} 
	\textbf{Source : }Grundlegung zur Metaphysik der Sitten/Dritter Abschnitt: Übergang von der Metaphysik der Sitten zur Kritik der reinen praktischen Vernunft/Von dem Interesse, welches den Ideen der Sittlichkeit anhängt\\  
	
	\noindent\textbf{Paragraphe : }Es ist eine Bemerkung, welche anzustellen eben kein subtiles Nachdenken erfodert wird, sondern von der man annehmen kann, daß sie wohl der gemeinste Verstand, obzwar, nach seiner Art, durch eine dunkele Unterscheidung der Urteilskraft, die er Gefühl nennt, machen mag: daß alle Vorstellungen, die uns ohne unsere Willkür kommen (wie die der Sinne), uns die Gegenstände nicht anders zu erkennen geben, als sie uns affizieren, wobei, was sie an sich sein mögen, uns unbekannt bleibt, mithin daß, was diese Art Vorstellungen betrifft, wir dadurch, auch bei der angestrengtesten Aufmerksamkeit und Deutlichkeit, die der Verstand nur immer hinzufügen mag, doch bloß zur Erkenntnis der Erscheinungen, niemals der Dinge an sich selbst gelangen können. Sobald dieser \match{Unterschied} (allenfalls bloß durch die bemerkte Verschiedenheit zwischen den Vorstellungen, die uns anders woher gegeben werden, und dabei wir leidend sind, von denen, die wir lediglich aus uns selbst hervorbringen, und dabei wir unsere Tätigkeit beweisen) einmal gemacht ist, so folgt von selbst, daß man hinter den Erscheinungen doch noch etwas anderes, was nicht Erscheinung  ist, nämlich die Dinge an sich, einräumen und annehmen müsse, ob wir gleich uns von selbst bescheiden, daß, da sie uns niemals bekannt werden können, sondern immer nur, wie sie uns affizieren, wir ihnen nicht näher treten, und, was sie an sich sind, niemals wissen können. Dieses muß eine, obzwar rohe, Unterscheidung einer Sinnenwelt von der Verstandeswelt abgeben, davon die erstere, nach Verschiedenheit der Sinnlichkeit in mancherlei Weltbeschauern, auch sehr verschieden sein kann, indessen die zweite, die ihr zum Grunde liegt, immer dieselbe bleibt. So gar sich selbst und zwar nach der Kenntnis, die der Mensch durch innere Empfindung von sich hat, darf er sich nicht anmaßen zu erkennen, wie er an sich selbst sei. Denn da er doch sich selbst nicht gleichsam schafft, und seinen Begriff nicht a priori, sondern empirisch bekömmt, so ist natürlich, daß er auch von sich durch den innern Sinn und folglich nur durch die Erscheinung seiner Natur, und die Art, wie sein Bewußtsein affiziert wird, Kundschaft einziehen könne, indessen er doch notwendiger Weise über diese aus lauter Erscheinungen zusammengesetzte Beschaffenheit seines eigenen Subjekts noch etwas anderes zum Grunde Liegendes, nämlich sein Ich, so wie es an sich selbst beschaffen sein mag, annehmen, und sich also in Absicht auf die bloße Wahrnehmung und Empfänglichkeit der Empfindungen zur Sinnenwelt, in Ansehung dessen aber, was in ihm reine Tätigkeit sein mag (dessen, was gar nicht durch Affizierung der Sinne, sondern unmittelbar zum Bewußtsein gelangt), sich zur intellektuellen Welt zählen muß, die er doch nicht weiter kennt. 
	
	\unnumberedsection{Verbindung (3)} 
	\subsection*{tg174.2.7} 
	\textbf{Source : }Grundlegung zur Metaphysik der Sitten/Vorrede\\  
	
	\noindent\textbf{Paragraphe : }Alle Gewerbe, Handwerke und Künste, haben durch die Verteilung der Arbeiten gewonnen, da nämlich nicht einer alles macht, sondern jeder sich auf gewisse Arbeit, die sich, ihrer Behandlungsweise nach, von andern merklich unterscheidet, einschränkt, um sie in der größten Vollkommenheit und mit mehrerer Leichtigkeit leisten zu können. Wo die Arbeiten so nicht unterschieden und verteilt werden, wo jeder ein Tausendkünstler ist, da liegen die Gewerbe noch in der größten Barbarei. Aber ob dieses zwar für sich ein der Erwägung nicht unwürdiges Objekt wäre, zu fragen: ob die reine Philosophie in allen ihren Teilen nicht ihren besondern Mann erheische, und es um das Ganze des gelehrten Gewerbes nicht besser stehen würde, wenn die, so das Empirische mit dem Rationalen, dem Geschmacke des Publikums gemäß, nach allerlei ihnen selbst unbekannten Verhältnissen gemischt, zu verkaufen gewohnt sind, die sich Selbstdenker, andere aber, die den bloß rationalen Teil zubereiten, Grübler nennen, gewarnt würden, nicht zwei Geschäfte zugleich zu treiben, die in der Art, sie zu behandeln, gar sehr verschieden sind, zu deren jedem vielleicht ein besonderes Talent erfodert wird, und deren \match{Verbindung} in einer Person nur Stümper hervorbringt: so frage ich hier doch nur, ob nicht die Natur der Wissenschaft es erfodere, den empirischen von dem rationalen Teil jederzeit sorgfältig abzusondern, und vor der eigentlichen (empirischen) Physik eine Metaphysik der Natur, vor der praktischen Anthropologie aber eine Metaphysik der Sitten voranzuschicken, die von allem Empirischen sorgfältig gesäubert sein müßte, um zu  wissen, wie viel reine Vernunft in beiden Fällen leisten könne, und aus welchen Quellen sie selbst diese ihre Belehrung a priori schöpfe, es mag übrigens das letztere Geschäfte von allen Sittenlehrern (deren Name Legion heißt), oder nur von einigen, die Beruf dazu fühlen, getrieben werden. 
	
	\subsection*{tg176.2.64} 
	\textbf{Source : }Grundlegung zur Metaphysik der Sitten/Zweiter Abschnitt: Übergang von der populären sittlichen Weltweisheit zur Metaphysik der Sitten\\  
	
	\noindent\textbf{Paragraphe : }Ich verstehe aber unter einem Reiche die systematische \match{Verbindung} verschiedener vernünftiger Wesen durch gemeinschaftliche Gesetze. Weil nun Gesetze die Zwecke ihrer allgemeinen Gültigkeit nach bestimmen, so wird, wenn man von dem persönlichen Unterschiede vernünftiger Wesen, imgleichen allem Inhalte ihrer Privatzwecke abstrahiert, ein Ganzes aller Zwecke (sowohl der vernünftigen Wesen als Zwecke an sich, als auch der eigenen Zwecke, die ein jedes sich selbst setzen mag), in systematischer Verknüpfung, d.i. ein Reich der Zwecke gedacht werden können, welches nach obigen Prinzipien möglich ist. 
	
	\subsection*{tg176.2.65} 
	\textbf{Source : }Grundlegung zur Metaphysik der Sitten/Zweiter Abschnitt: Übergang von der populären sittlichen Weltweisheit zur Metaphysik der Sitten\\  
	
	\noindent\textbf{Paragraphe : }Denn vernünftige Wesen stehen alle unter dem Gesetz, daß jedes derselben sich selbst und alle andere niemals bloß als Mittel, sondern jederzeit zugleich als Zweck an sich selbst behandeln solle. Hiedurch aber entspringt eine systematische \match{Verbindung} vernünftiger Wesen durch gemeinschaftliche objektive Gesetze, d.i. ein Reich, welches, weil diese Gesetze eben die Beziehung dieser Wesen auf einander, als Zwecke und Mittel, zur Absicht haben, ein Reich der Zwecke (freilich nur ein Ideal) heißen kann. 
	
	\unnumberedsection{Verhaltnis (7)} 
	\subsection*{tg176.2.16} 
	\textbf{Source : }Grundlegung zur Metaphysik der Sitten/Zweiter Abschnitt: Übergang von der populären sittlichen Weltweisheit zur Metaphysik der Sitten\\  
	
	\noindent\textbf{Paragraphe : }
	Alle Imperativen werden durch ein Sollen ausgedruckt, und zeigen dadurch das \match{Verhältnis} eines objektiven Gesetzes der Vernunft zu einem Willen an, der seiner subjektiven Beschaffenheit nach dadurch nicht notwendig bestimmt wird (eine Nötigung). Sie sagen, daß etwas zu tun oder zu unterlassen gut sein würde, allein sie sagen es einem Willen, der nicht immer darum etwas tut, weil ihm vorgestellt wird, daß es zu tun gut sei. Praktisch gut ist aber, was vermittelst der Vorstellungen der Vernunft, mithin nicht aus subjektiven Ursachen, sondern objektiv, d.i. aus Gründen, die für jedes vernünftige Wesen, als ein solches, gültig sind, den Willen bestimmt. Es wird vom Angenehmen unterschieden, als demjenigen, was nur vermittelst der Empfindung aus bloß subjektiven Ursachen, die nur für dieses oder jenes seinen Sinn gelten, und nicht als Prinzip der Vernunft, das für jedermann gilt, auf den Willen Einfluß hat.
	
	
	5
	
	
	
	\subsection*{tg176.2.17} 
	\textbf{Source : }Grundlegung zur Metaphysik der Sitten/Zweiter Abschnitt: Übergang von der populären sittlichen Weltweisheit zur Metaphysik der Sitten\\  
	
	\noindent\textbf{Paragraphe : }Ein vollkommen guter Wille würde also eben sowohl unter objektiven Gesetzen (des Guten) stehen, aber nicht dadurch als zu gesetzmäßigen Handlungen genötigt vorgestellt werden können, weil er von selbst, nach seiner subjektiven Beschaffenheit, nur durch die Vorstellung des Guten  bestimmt werden kann. Daher gelten für den göttlichen und überhaupt für einen heiligen Willen keine Imperativen; das Sollen ist hier am unrechten Orte, weil das Wollen schon von selbst mit dem Gesetz notwendig einstimmig ist. Daher sind Imperativen nur Formeln, das \match{Verhältnis} objektiver Gesetze des Wollens überhaupt zu der subjektiven Unvollkommenheit des Willens dieses oder jenes vernünftigen Wesens, z.B. des menschlichen Willens, auszudrücken. 
	
	\subsection*{tg176.2.20} 
	\textbf{Source : }Grundlegung zur Metaphysik der Sitten/Zweiter Abschnitt: Übergang von der populären sittlichen Weltweisheit zur Metaphysik der Sitten\\  
	
	\noindent\textbf{Paragraphe : }Der Imperativ sagt also, welche durch mich mögliche Handlung gut wäre, und stellt die praktische Regel in \match{Verhältnis} auf einen Willen vor, der darum nicht sofort eine Handlung tut, weil sie gut ist, teils weil das Subjekt nicht immer weiß, daß sie gut sei, teils weil, wenn es dieses auch wüßte, die Maximen desselben doch den objektiven Prinzipien einer praktischen Vernunft zuwider sein könnten. 
	
	\subsection*{tg178.2.2} 
	\textbf{Source : }Grundlegung zur Metaphysik der Sitten/Zweiter Abschnitt: Übergang von der populären sittlichen Weltweisheit zur Metaphysik der Sitten/Die Heteronomie des Willens als der Quell aller unechten Prinzipien der Sittlichkeit\\  
	
	\noindent\textbf{Paragraphe : }Wenn der Wille irgend worin anders, als in der Tauglichkeit seiner Maximen zu seiner eigenen allgemeinen Gesetzgebung, mithin, wenn er, indem er über sich selbst hinausgeht, in der Beschaffenheit irgend eines seiner Objekte das Gesetz sucht, das ihn bestimmen soll, so kommt jederzeit Heteronomie heraus. Der Wille gibt alsdenn sich nicht selbst, sondern das Objekt durch sein \match{Verhältnis} zum Willen gibt diesem das Gesetz. Dies Verhältnis, es beruhe nun auf der Neigung, oder auf Vorstellungen der Vernunft, läßt nur hypothetische Imperativen möglich werden: ich soll etwas tun darum, weil ich etwas anderes will. Dagegen sagt der moralische, mithin kategorische Imperativ: ich soll so oder so handeln, ob ich gleich nichts anderes wollte. Z. E. jener sagt: ich soll nicht lügen, wenn ich bei Ehren bleiben will; dieser aber: ich soll nicht lügen, ob es  mir gleich nicht die mindeste Schande zuzöge. Der letztere muß also von allem Gegenstande so fern abstrahieren, daß dieser gar keinen Einfluß auf den Willen habe, damit praktische Vernunft (Wille) nicht fremdes Interesse bloß administriere, sondern bloß ihr eigenes gebietendes Ansehen, als oberste Gesetzgebung, beweise. So soll ich z.B. fremde Glückseligkeit zu befördern suchen, nicht als wenn mir an deren Existenz was gelegen wäre (es sei durch unmittelbare Neigung, oder irgend ein Wohlgefallen indirekt durch Vernunft), sondern bloß deswegen, weil die Maxime, die sie ausschließt, nicht in einem und demselben Wollen, als allgemeinen Gesetz begriffen werden kann. 
	
	\subsection*{tg181.2.4} 
	\textbf{Source : }Grundlegung zur Metaphysik der Sitten/Dritter Abschnitt: Übergang von der Metaphysik der Sitten zur Kritik der reinen praktischen Vernunft/Der Begriff der Freiheit ist der Schlüssel zur Erklärung der Autonomie des Willens\\  
	
	\noindent\textbf{Paragraphe : }Wenn also Freiheit des Willens vorausgesetzt wird, so folgt die Sittlichkeit samt ihrem Prinzip daraus, durch bloße Zergliederung ihres Begriffs. Indessen ist das letztere doch immer ein synthetischer Satz: ein schlechterdings guter Wille ist derjenige, dessen Maxime jederzeit sich selbst, als allgemeines Gesetz betrachtet, in sich enthalten kann; denn durch Zergliederung des Begriffs von einem schlechthin guten Willen kann jene Eigenschaft der Maxime nicht gefunden werden. Solche synthetische Sätze sind aber nur dadurch möglich, daß beide Erkenntnisse durch die Verknüpfung mit einem dritten, darin sie beiderseits anzutreffen sind, unter einander verbunden werden. Der positive Begriff der Freiheit schafft dieses dritte, welches nicht, wie bei den physischen Ursachen, die Natur der Sinnenwelt sein kann (in deren Begriff die Begriffe von etwas als Ursache, in \match{Verhältnis} auf etwas anderes als Wirkung, zusammenkommen). Was dieses dritte sei, worauf uns die Freiheit weiset, und von dem wir a priori eine Idee haben, läßt sich hier sofort noch nicht anzeigen, und die Deduktion des Begriffs der Freiheit aus der reinen praktischen Vernunft, mit ihr auch die Möglichkeit eines kategorischen Imperativs, begreiflich machen, sondern bedarf noch einiger Vorbereitung. 
	
	\subsection*{tg185.2.13} 
	\textbf{Source : }Grundlegung zur Metaphysik der Sitten/Dritter Abschnitt: Übergang von der Metaphysik der Sitten zur Kritik der reinen praktischen Vernunft/Von der äußersten Grenze aller praktischen Philosophie\\  
	
	\noindent\textbf{Paragraphe : }Um das zu wollen, wozu die Vernunft allein dem sinnlich-affizierten vernünftigen Wesen das Sollen vorschreibt, dazu gehört freilich ein Vermögen der Vernunft, ein Gefühl der Lust oder des Wohlgefallens an der Erfüllung der Pflicht einzuflößen, mithin eine Kausalität derselben, die Sinnlichkeit ihren Prinzipien gemäß zu bestimmen. Es ist aber gänzlich unmöglich, einzusehen, d.i. a priori begreiflich zu machen, wie ein bloßer Gedanke, der selbst nichts Sinnliches in sich enthält, eine Empfindung der Lust oder Unlust hervorbringe; denn das ist eine besondere Art von Kausalität, von der, wie von aller Kausalität, wir gar nichts a priori bestimmen können, sondern darum allein die Erfahrung befragen müssen. Da diese aber kein \match{Verhältnis} der Ursache zur Wirkung, als zwischen zwei Gegenständen der Erfahrung, an die Hand geben kann, hier aber reine Vernunft durch bloße Ideen (die gar keinen Gegenstand für Erfahrung abgeben) die Ursache von einer Wirkung, die freilich in der Erfahrung liegt, sein soll, so ist die Erklärung, wie und warum uns die Allgemeinheit der Maxime als Gesetzes, mithin die Sittlichkeit, interessiere, uns Menschen gänzlich unmöglich. So viel ist nur gewiß: daß es nicht darum für uns Gültigkeit hat, weil es interessiert (denn das ist Heteronomie und Abhängigkeit der praktischen Vernunft von Sinnlichkeit, nämlich einem zum Grunde liegenden Gefühl, wobei sie niemals sittlich gesetzgebend sein könnte), sondern daß es interessiert, weil es für uns als Menschen gilt, da es aus unserem Willen als Intelligenz, mithin aus unserem eigentlichen Selbst, entsprungen ist; was aber zur bloßen Erscheinung gehört, wird von der Vernunft notwendig der Beschaffenheit der Sache an sich selbst untergeordnet.
	
	
	\subsection*{tg185.2.7} 
	\textbf{Source : }Grundlegung zur Metaphysik der Sitten/Dritter Abschnitt: Übergang von der Metaphysik der Sitten zur Kritik der reinen praktischen Vernunft/Von der äußersten Grenze aller praktischen Philosophie\\  
	
	\noindent\textbf{Paragraphe : }
	Der Rechtsanspruch aber, selbst der gemeinen Menschenvernunft, auf Freiheit des Willens, gründet sich auf das Bewußtsein und die zugestandene Voraussetzung der Unabhängigkeit der Vernunft, von bloß subjektiv-bestimmten Ursachen, die insgesamt das ausmachen, was bloß zur Empfindung, mithin unter die allgemeine Benennung der Sinnlichkeit, gehört. Der Mensch, der sich auf solche Weise als Intelligenz betrachtet, setzt sich dadurch in eine andere Ordnung der Dinge und in ein \match{Verhältnis} zu bestimmenden Gründen von ganz anderer Art, wenn er sich als Intelligenz mit einem Willen, folglich mit Kausalität begabt, denkt, als wenn er sich wie Phänomen in der Sinnenwelt (welches er wirklich auch ist) wahrnimmt, und seine Kausalität, äußerer Bestimmung nach, Naturgesetzen unterwirft. Nun wird er bald inne, daß beides zugleich stattfinden könne, ja sogar müsse. Denn, daß ein Ding in der Erscheinung (das zur Sinnenwelt gehörig) gewissen Gesetzen unterworfen ist, von welchen eben dasselbe, als Ding oder Wesen an sich selbst, unabhängig ist, enthält nicht den mindesten Widerspruch; daß er sich selbst aber auf diese zwiefache Art vorstellen und denken müsse, beruht, was das erste betrifft, auf dem Bewußtsein seiner selbst als durch Sinne affizierten Gegenstandes, was das zweite anlangt, auf dem Bewußtsein seiner selbst als Intelligenz, d.i. als unabhängig im Vernunftgebrauch von sinnlichen Eindrücken (mithin als zur Verstandeswelt gehörig). 
	
	\unnumberedsection{Widerstand (1)} 
	\subsection*{tg176.2.42} 
	\textbf{Source : }Grundlegung zur Metaphysik der Sitten/Zweiter Abschnitt: Übergang von der populären sittlichen Weltweisheit zur Metaphysik der Sitten\\  
	
	\noindent\textbf{Paragraphe : }Wenn wir nun auf uns selbst bei jeder Übertretung einer Pflicht Acht haben, so finden wir, daß wir wirklich nicht wollen, es solle unsere Maxime ein allgemeines Gesetz werden, denn das ist uns unmöglich, sondern das Gegenteil derselben soll vielmehr allgemein ein Gesetz bleiben; nur nehmen wir uns die Freiheit, für uns, oder (auch nur für diesesmal) zum Vorteil unserer Neigung, davon eine Ausnahme zu machen. Folglich, wenn wir alles aus einem und demselben Gesichtspunkte, nämlich der Vernunft, erwögen, so würden wir einen Widerspruch in unserm eigenen Willen antreffen, nämlich, daß ein gewisses Prinzip objektiv als allgemeines Gesetz notwendig sei und doch subjektiv nicht allgemein gelten, sondern Ausnahmen verstatten sollte. Da wir aber einmal unsere Handlung aus dem Gesichtspunkte eines ganz der Vernunftgemäßen, dann aber auch eben dieselbe Handlung aus dem Gesichtspunkte eines durch Neigung affizierten Willens betrachten, so ist wirklich hier kein Widerspruch, wohl aber ein \match{Widerstand} der Neigung gegen die Vorschrift der Vernunft (antago nismus), wodurch die Allgemeinheit des Prinzips (universalitas) in eine bloße Gemeingültigkeit (generalitas) verwandelt wird, dadurch das praktische Vernunftprinzip mit der Maxime auf dem halben Wege zusammenkommen soll. Ob nun dieses gleich in unserm eigenen unparteiisch angestellten Urteile nicht gerechtfertiget werden kann, so beweiset es doch, daß wir die  Gültigkeit des kategorischen Imperativs wirklich anerkennen, und uns (mit aller Achtung für denselben) nur einige, wie es uns scheint, unerhebliche und uns abgedrungene Ausnahmen erlauben. 
	
	\unnumberedsection{Wirkung (10)} 
	\subsection*{tg175.2.17} 
	\textbf{Source : }Grundlegung zur Metaphysik der Sitten/Erster Abschnitt: Übergang von der gemeinen sittlichen Vernunfterkenntnis zur philosophischen\\  
	
	\noindent\textbf{Paragraphe : }Den dritten Satz, als Folgerung aus beiden vorigen, würde ich so ausdrücken: Pflicht ist die Notwendigkeit einer Handlung aus Achtung fürs Gesetz. Zum Objekte als \match{Wirkung} meiner vorhabenden Handlung kann ich zwar Neigung haben, aber niemals Achtung, eben darum, weil sie bloß eine Wirkung und nicht Tätigkeit eines Willens ist. Eben so kann ich für Neigung überhaupt, sie mag nun meine oder eines andern seine sein, nicht Achtung haben, ich kann sie höchstens im ersten Falle billigen, im zweiten bisweilen selbst lieben, d.i. sie als meinem eigenen  Vorteile günstig ansehen. Nur das, was bloß als Grund, niemals aber als Wirkung mit meinem Willen verknüpft ist, was nicht meiner Neigung dient, sondern sie überwiegt, wenigstens diese von deren Überschlage bei der Wahl ganz ausschließt, mithin das bloße Gesetz für sich, kann ein Gegenstand der Achtung und hiemit ein Gebot sein. Nun soll eine Handlung aus Pflicht den Einfluß der Neigung, und mit ihr jeden Gegenstand des Willens ganz absondern, also bleibt nichts für den Willen übrig, was ihn bestimmen könne, als, objektiv, das Gesetz, und, subjektiv, reine Achtung für dieses praktische Gesetz, mithin die Maxime
	
	
	1
	, einem solchen Gesetze, selbst mit Abbruch aller meiner Neigungen, Folge zu leisten. 
	
	\subsection*{tg175.2.18} 
	\textbf{Source : }Grundlegung zur Metaphysik der Sitten/Erster Abschnitt: Übergang von der gemeinen sittlichen Vernunfterkenntnis zur philosophischen\\  
	
	\noindent\textbf{Paragraphe : }Es liegt also der moralische Wert der Handlung nicht in der Wirkung, die daraus erwartet wird, also auch nicht in irgend einem Prinzip der Handlung, welches seinen Bewegungsgrund von dieser erwarteten \match{Wirkung} zu entlehnen bedarf. Denn alle diese Wirkungen (Annehmlichkeit seines Zustandes, ja gar Beförderung fremder Glückseligkeit) konnten auch durch andere Ursachen zu Stande gebracht werden, und es brauchte also dazu nicht des Willens eines vernünftigen Wesens; worin gleichwohl das höchste und unbedingte Gute allein angetroffen werden kann. Es kann daher nichts anders als die Vorstellung des Gesetzes an sich selbst, die freilich nur im vernünftigen Wesen stattfindet, so fern sie, nicht aber die verhoffte Wirkung, der Bestimmungsgrund des Willens ist, das so vorzügliche Gute, welches wir sittlich nennen, ausmachen, welches in der Person selbst schon gegen wärtig ist, die darnach handelt, nicht aber allererst aus der Wirkung erwartet werden darf.
	
	
	2
	
	
	
	\subsection*{tg175.2.19} 
	\textbf{Source : }Grundlegung zur Metaphysik der Sitten/Erster Abschnitt: Übergang von der gemeinen sittlichen Vernunfterkenntnis zur philosophischen\\  
	
	\noindent\textbf{Paragraphe : }
	Was kann das aber wohl für ein Gesetz sein, dessen Vorstellung, auch ohne auf die daraus erwartete \match{Wirkung} Rücksicht zu nehmen, den Willen bestimmen muß, damit dieser schlechterdings und ohne Einschränkung gut heißen könne? Da ich den Willen aller Antriebe beraubet habe, die ihm aus der Befolgung irgend eines Gesetzes entspringen könnten, so bleibt nichts als die allgemeine Gesetzmäßigkeit der Handlungen überhaupt übrig, welche allein dem Willen zum Prinzip dienen soll, d.i. ich soll niemals anders verfahren, als so, daß ich auch wollen könne, meine Maxime solle ein allgemeines Gesetz werden. Hier ist nun die bloße Gesetzmäßigkeit überhaupt (ohne irgend ein auf gewisse Handlungen bestimmtes Gesetz zum Grunde zu legen) das, was dem Willen zum Prinzip dient, und ihm auch dazu dienen muß, wenn Pflicht nicht überall ein leerer Wahn und chimärischer Begriff sein soll; hiemit stimmt die  gemeine Menschenvernunft in ihrer praktischen Beurteilung auch vollkommen überein, und hat das gedachte Prinzip jederzeit vor Augen. 
	
	\subsection*{tg176.2.48} 
	\textbf{Source : }Grundlegung zur Metaphysik der Sitten/Zweiter Abschnitt: Übergang von der populären sittlichen Weltweisheit zur Metaphysik der Sitten\\  
	
	\noindent\textbf{Paragraphe : }Der Wille wird als ein Vermögen gedacht, der Vorstellung gewisser Gesetze gemäß sich selbst zum Handeln zu bestimmen. Und ein solches Vermögen kann nur in vernünftigen Wesen anzutreffen sein. Nun ist das, was dem Willen zum objektiven Grunde seiner Selbstbestimmung dient, der Zweck, und dieser, wenn er durch bloße Vernunft gegeben wird, muß für alle vernünftige Wesen gleich gelten. Was dagegen bloß den Grund der Möglichkeit der Handlung enthält, deren \match{Wirkung} Zweck ist, heißt das Mittel. Der subjektive Grund des Begehrens ist die Triebfeder, der objektive des Wollens der Bewegungsgrund; daher der Unterschied zwischen subjektiven Zwecken, die auf Triebfedern beruhen, und objektiven, die auf Bewegungsgründe ankommen, welche für jedes vernünftige Wesen gelten. Praktische Prinzipien sind formal, wenn sie von allen subjektiven Zwecken abstrahieren; sie sind aber material, wenn sie diese, mithin gewisse Triebfedern, zum Grunde legen. Die Zwecke, die sich ein vernünftiges Wesen als Wirkungen seiner Handlung nach Belieben vorsetzt (materiale Zwecke), sind insgesamt nur relativ; denn nur bloß ihr Verhältnis auf ein besonders geartetes Begehrungsvermögen des Subjekts gibt ihnen den Wert, der daher keine allgemeine für alle vernünftige Wesen, und auch nicht für jedes Wollen gültige und notwendige Prinzipien, d.i. praktische Gesetze, an die Hand geben kann. Daher sind alle diese relative Zwecke nur der Grund von hypothetischen Imperativen. 
	
	\subsection*{tg176.2.50} 
	\textbf{Source : }Grundlegung zur Metaphysik der Sitten/Zweiter Abschnitt: Übergang von der populären sittlichen Weltweisheit zur Metaphysik der Sitten\\  
	
	\noindent\textbf{Paragraphe : }Nun sage ich: der Mensch, und überhaupt jedes vernünftige Wesen, existiert als Zweck an sich selbst, nicht bloß als Mittel zum beliebigen Gebrauche für diesen oder jenen  Willen, sondern muß in allen seinen, sowohl auf sich selbst, als auch auf andere vernünftige Wesen gerichteten Handlungen jederzeit zugleich als Zweck betrachtet werden. Alle Gegenstände der Neigungen haben nur einen bedingten Wert; denn, wenn die Neigungen und darauf gegründete Bedürfnisse nicht wären, so würde ihr Gegenstand ohne Wert sein. Die Neigungen selber aber, als Quellen der Bedürfnis, haben so wenig einen absoluten Wert, um sie selbst zu wünschen, daß vielmehr, gänzlich davon frei zu sein, der allgemeine Wunsch eines jeden vernünftigen Wesens sein muß. Also ist der Wert aller durch unsere Handlung zu erwerbenden Gegenstände jederzeit bedingt. Die Wesen, deren Dasein zwar nicht auf unserm Willen, sondern der Natur beruht, haben dennoch, wenn sie vernunftlose Wesen sind, nur einen relativen Wert, als Mittel, und heißen daher Sachen, dagegen vernünftige Wesen Personen genannt werden, weil ihre Natur sie schon als Zwecke an sich selbst, d.i. als etwas, das nicht bloß als Mittel gebraucht werden darf, auszeichnet, mithin so fern alle Willkür einschränkt (und ein Gegenstand der Achtung ist). Dies sind also nicht bloß subjektive Zwecke, deren Existenz, als \match{Wirkung} unserer Hand lung, für uns einen Wert hat; sondern objektive Zwecke, d.i. Dinge, deren Dasein an sich selbst Zweck ist, und zwar einen solchen, an dessen Statt kein anderer Zweck gesetzt werden kann, dem sie bloß als Mittel zu Diensten stehen sollten, weil ohne dieses überall gar nichts von absolutem Werte würde angetroffen werden; wenn aber aller Wert bedingt, mithin zufällig wäre, so könnte für die Vernunft überall kein oberstes praktisches Prinzip angetroffen werden. 
	
	\subsection*{tg176.2.56} 
	\textbf{Source : }Grundlegung zur Metaphysik der Sitten/Zweiter Abschnitt: Übergang von der populären sittlichen Weltweisheit zur Metaphysik der Sitten\\  
	
	\noindent\textbf{Paragraphe : }
	Viertens, in Betreff der verdienstlichen Pflicht gegen andere, ist der Naturzweck, den alle Menschen haben, ihre  eigene Glückseligkeit. Nun würde zwar die Menschheit bestehen können, wenn niemand zu des andern Glückseligkeit was beitrüge, dabei aber ihr nichts vorsätzlich entzöge; allein es ist dieses doch nur eine negative und nicht positive Übereinstimmung zur Menschheit, als Zweck an sich selbst, wenn jedermann auch nicht die Zwecke anderer, so viel an ihm ist, zu befördern trachtete. Denn das Subjekt, welches Zweck an sich selbst ist, dessen Zwecke müssen, wenn jene Vorstellung bei mir alle \match{Wirkung} tun soll, auch, so viel möglich, meine Zwecke sein. 
	
	\subsection*{tg179.2.8} 
	\textbf{Source : }Grundlegung zur Metaphysik der Sitten/Zweiter Abschnitt: Übergang von der populären sittlichen Weltweisheit zur Metaphysik der Sitten/Einteilung aller möglichen Prinzipien der Sittlichkeit aus dem angenommenen Grundbegriffe der Heteronomie\\  
	
	\noindent\textbf{Paragraphe : }Allenthalben, wo ein Objekt des Willens zum Grunde gelegt werden muß, um diesem die Regel vorzuschreiben, die ihn bestimme, da ist die Regel nichts als Heteronomie; der Imperativ ist bedingt, nämlich: wenn oder weil man dieses Objekt will, soll man so oder so handeln; mithin kann er niemals moralisch, d.i. kategorisch, gebieten. Er mag nun das Objekt vermittelst der Neigung, wie beim Prinzip der eigenen Glückseligkeit, oder vermittelst der auf Gegenstände unseres möglichen Wollens überhaupt gerichteten Vernunft, im Prinzip der Vollkommenheit, den Willen bestimmen, so bestimmt sich der Wille niemals unmittelbar selbst durch die Vorstellung der Handlung, sondern nur durch die Triebfeder, welche die vorausgesehene \match{Wirkung} der Handlung auf den Willen hat; ich soll etwas tun, darum, weil ich etwas anderes will, und hier muß noch ein anderes Gesetz in meinem Subjekt zum Grunde gelegt werden, nach welchem ich dieses andere notwendig will, welches Gesetz wiederum eines Imperativs bedarf, der diese Maxime einschränke. Denn weil der Antrieb, der die Vorstellung eines durch unsere Kräfte möglichen Objekts nach der Naturbeschaffenheit des Subjekts auf seinen Willen ausüben soll, zur Natur des Subjekts gehöret, es sei der Sinnlichkeit (der Neigung und des Geschmacks), oder des Verstandes und der Vernunft, die nach der besonderen Einrichtung ihrer Natur an einem Objekte sich mit Wohlgefallen üben, so gäbe eigentlich die Natur das Gesetz, welches, als ein solches, nicht allein durch Erfahrung erkannt und bewiesen werden muß, mithin an sich zufällig ist und zur apodiktischen praktischen Regel, dergleichen die moralische sein muß, dadurch untauglich wird, sondern es ist immer nur Heteronomie des Willens, der Wille gibt sich nicht selbst, sondern ein fremder Antrieb gibt ihm, vermittelst  einer auf die Empfänglichkeit desselben gestimmten Natur des Subjekts, das Gesetz. 
	
	\subsection*{tg181.2.3} 
	\textbf{Source : }Grundlegung zur Metaphysik der Sitten/Dritter Abschnitt: Übergang von der Metaphysik der Sitten zur Kritik der reinen praktischen Vernunft/Der Begriff der Freiheit ist der Schlüssel zur Erklärung der Autonomie des Willens\\  
	
	\noindent\textbf{Paragraphe : }Die angeführte Erklärung der Freiheit ist negativ, und daher, um ihr Wesen einzusehen, unfruchtbar; allein es fließt aus ihr ein positiver Begriff derselben, der desto reichhaltiger und fruchtbarer ist. Da der Begriff einer Kausalität den von Gesetzen bei sich führt, nach welchen durch etwas, was wir Ursache nennen, etwas anderes, nämlich die Folge, gesetzt werden muß: so ist die Freiheit, ob sie zwar nicht eine Eigenschaft des Willens nach Naturgesetzen ist, darum doch nicht gar gesetzlos, sondern muß vielmehr eine Kausalität nach unwandelbaren Gesetzen, aber von besonderer Art, sein; denn sonst wäre ein freier Wille ein Unding. Die Naturnotwendigkeit war eine Heteronomie der wirkenden Ursachen; denn jede \match{Wirkung} war nur nach dem Gesetze möglich, daß etwas anderes die wirkende Ursache zur Kausalität bestimmte; was kann denn wohl die Freiheit des Willens sonst sein, als Autonomie, d.i. die Eigenschaft des Willens, sich selbst ein Gesetz zu sein? Der Satz aber: der Wille ist in allen Handlungen sich selbst ein Gesetz, bezeichnet nur das Prinzip, nach keiner anderen Maxime zu handeln, als die sich selbst auch als ein allgemeines Gesetz zum Gegenstande haben kann. Dies ist aber gerade die Formel des kategorischen Imperativs und das Prinzip der Sittlichkeit:  also ist ein freier Wille und ein Wille unter sittlichen Gesetzen einerlei. 
	
	\subsection*{tg183.2.5} 
	\textbf{Source : }Grundlegung zur Metaphysik der Sitten/Dritter Abschnitt: Übergang von der Metaphysik der Sitten zur Kritik der reinen praktischen Vernunft/Von dem Interesse, welches den Ideen der Sittlichkeit anhängt\\  
	
	\noindent\textbf{Paragraphe : }Zwar finden wir wohl, daß wir an einer persönlichen Beschaffenheit ein Interesse nehmen können, die gar kein Interesse des Zustandes bei sich führt, wenn jene uns nur fähig macht, des letzteren teilhaftig zu werden, im Falle die Vernunft die Austeilung desselben bewirken sollte, d.i. daß die bloße Würdigkeit, glücklich zu sein, auch ohne den Bewegungsgrund, dieser Glückseligkeit teilhaftig zu werden, für sich interessieren könne: aber dieses Urteil ist in der Tat nur die \match{Wirkung} von der schon vorausgesetzten Wichtigkeit moralischer Gesetze (wenn wir uns durch die Idee der Freiheit von allem empirischen Interesse trennen), aber, daß wir uns von diesem trennen, d.i. uns als frei im Handeln betrachten, und so uns dennoch für gewissen Gesetzen unterworfen halten sollen, um einen Wert bloß in unserer Person zu finden, der uns allen Verlust dessen, was unserem Zustande einen Wert verschafft, vergüten könne, und wie dieses möglich sei, mithin woher das moralische Gesetz verbinde, können wir auf solche Art noch nicht einsehen. 
	
	\subsection*{tg187.2.4} 
	\textbf{Source : }Grundlegung zur Metaphysik der Sitten/Fußnoten\\  
	
	\noindent\textbf{Paragraphe : }
	
	2 Man könnte mir vorwerfen, als suchte ich hinter dem Worte Achtung nur Zuflucht in einem dunkelen Gefühle, anstatt durch einen Begriff der Vernunft in der Frage deutliche Auskunft zu geben. Allein wenn Achtung gleich ein Gefühl ist, so ist es doch kein durch Einfluß empfangenes, sondern durch einen Vernunftbegriff selbstgewirktes Gefühl und daher von allen Gefühlen der ersteren Art, die sich auf Neigung oder Furcht bringen lassen, spezifisch unterschieden. Was ich unmittelbar als Gesetz für mich erkenne, erkenne ich mit Achtung, welche bloß das Bewußtsein der Unterordnung meines Willens unter einem Gesetze, ohne Vermittelung anderer Einflüsse auf meinen Sinn, bedeutet. Die unmittelbare Bestimmung des Willens durchs Gesetz und das Bewußtsein derselben heißt Achtung, so daß diese als \match{Wirkung} des Gesetzes aufs Subjekt und nicht als Ursache desselben angesehen wird. Eigentlich ist Achtung die Vorstellung von 
	einem Werte, der meiner Selbstliebe Abbruch tut. Also ist es etwas, was weder als Gegenstand der Neigung, noch der Furcht, betrachtet wird, obgleich es mit beiden zugleich etwas Analogisches hat. Der Gegenstand der Achtung ist also lediglich das Gesetz, und zwar dasjenige, das wir uns selbst und doch als an sich notwendig auferlegen. Als Gesetz sind wir ihm unterworfen, ohne die Selbstliebe zu befragen; als uns von uns selbst auferlegt ist es doch eine Folge unsers Willens, und hat in der ersten Rücksicht Analogie mit Furcht, in der zweiten mit Neigung. Alle Achtung für eine Person ist eigentlich nur Achtung fürs Gesetz (der Rechtschaffenheit etc.), wovon jene uns das Beispiel gibt. Weil wir Erweiterung unserer Talente auch als Pflicht ansehen, so stellen wir uns an einer Person von Talenten auch gleichsam das Beispiel eines Gesetzes vor (ihr durch Übung hierin ähnlich zu werden) und das macht unsere Achtung aus. Alles moralische so genannte Interesse besteht lediglich in der Achtung fürs Gesetz. 
	
	\unnumberedsection{Zeit (2)} 
	\subsection*{tg176.2.3} 
	\textbf{Source : }Grundlegung zur Metaphysik der Sitten/Zweiter Abschnitt: Übergang von der populären sittlichen Weltweisheit zur Metaphysik der Sitten\\  
	
	\noindent\textbf{Paragraphe : }Wenn wir unsern bisherigen Begriff der Pflicht aus dem gemeinen Gebrauche unserer praktischen Vernunft gezogen haben, so ist daraus keinesweges zu schließen, als hätten wir ihn als einen Erfahrungsbegriff behandelt. Vielmehr, wenn wir auf die Erfahrung vom Tun und Lassen der Menschen Acht haben, treffen wir häufige, und, wie wir selbst einräumen, gerechte Klagen an, daß man von der Gesinnung, aus reiner Pflicht zu handeln, so gar keine sichere Beispiele anführen könne, daß, wenn gleich manches dem, was Pflicht gebietet, gemäß geschehen mag, dennoch es immer noch zweifelhaft sei, ob es eigentlich aus Pflicht geschehe und also einen moralischen Wert habe. Daher es zu aller \match{Zeit} Philosophen gegeben hat, welche die Wirklichkeit  dieser Gesinnung in den menschlichen Handlungen schlechterdings abgeleugnet, und alles der mehr oder weniger verfeinerten Selbstliebe zugeschrieben haben, ohne doch deswegen die Richtigkeit des Begriffs von Sittlichkeit in Zweifel zu ziehen, vielmehr mit inniglichem Bedauren der Gebrechlichkeit und Unlauterkeit der menschlichen Natur Erwähnung taten, die zwar edel gnug sei, sich eine so achtungswürdige Idee zu ihrer Vorschrift zu machen, aber zugleich zu schwach, um sie zu befolgen, und die Vernunft, die ihr zur Gesetzgebung dienen sollte, nur dazu braucht, um das Interesse der Neigungen, es sei einzeln, oder, wenn es hoch kommt, in ihrer größten Verträglichkeit unter einander, zu besorgen. 
	
	\subsection*{tg176.2.38} 
	\textbf{Source : }Grundlegung zur Metaphysik der Sitten/Zweiter Abschnitt: Übergang von der populären sittlichen Weltweisheit zur Metaphysik der Sitten\\  
	
	\noindent\textbf{Paragraphe : }2) Ein anderer sieht sich durch Not gedrungen, Geld zu borgen. Er weiß wohl, daß er nicht wird bezahlen können, sieht aber auch, daß ihm nichts geliehen werden wird, wenn er nicht festiglich verspricht, es zu einer bestimmten \match{Zeit} zu bezahlen. Er hat Lust, ein solches Versprechen zu tun; noch aber hat er so viel Gewissen, sich zu fragen: ist es nicht unerlaubt und pflichtwidrig, sich auf solche Art aus Not zu helfen? Gesetzt, er beschlösse es doch, so würde seine Maxime der Handlung so lauten: wenn ich mich in Geldnot zu sein glaube, so will ich Geld borgen, und versprechen, es zu bezahlen, ob ich gleich weiß, es werde niemals geschehen. Nun ist dieses Prinzip der Selbstliebe, oder der eigenen Zuträglichkeit, mit meinem ganzen künftigen Wohlbefinden vielleicht wohl zu vereinigen, allein jetzt ist die Frage: ob es recht sei? Ich verwandle also die Zumutung der Selbstliebe in ein allgemeines Gesetz, und richte die Frage so ein: wie es dann stehen würde, wenn meine Maxime ein allgemeines Gesetz würde. Da sehe ich nun sogleich, daß sie niemals als allgemeines Naturgesetz gelten und mit sich selbst zusammenstimmen könne, sondern sich notwendig widersprechen müsse. Denn die Allgemeinheit eines Gesetzes, daß jeder, nachdem er in Not zu sein glaubt, versprechen könne, was ihm einfällt, mit dem Vorsatz, es nicht zu halten, würde das Versprechen und den Zweck, den man damit haben mag, selbst unmöglich machen, indem niemand glauben würde, daß ihm was versprochen sei, sondern über alle solche Äußerung, als eitles Vorgeben, lachen würde. 
	
	\unnumberedchapter{Vetement} 
	\unnumberedsection{Geschmack (1)} 
	\subsection*{tg175.2.12} 
	\textbf{Source : }Grundlegung zur Metaphysik der Sitten/Erster Abschnitt: Übergang von der gemeinen sittlichen Vernunfterkenntnis zur philosophischen\\  
	
	\noindent\textbf{Paragraphe : }Dagegen, sein Leben zu erhalten, ist Pflicht, und überdem hat jedermann dazu noch eine unmittelbare Neigung. Aber um deswillen hat die oft ängstliche Sorgfalt, die der größte Teil der Menschen dafür trägt, doch keinen innern Wert, und die Maxime derselben keinen moralischen Gehalt. Sie bewahren ihr Leben zwar pflichtmäßig, aber nicht aus Pflicht. Dagegen, wenn Widerwärtigkeiten und hoffnungsloser Gram den \match{Geschmack} am Leben gänzlich weggenommen haben; wenn der Unglückliche, stark an Seele, über sein Schicksal mehr entrüstet, als kleinmütig oder niedergeschlagen, den Tod wünscht, und sein Leben doch erhält, ohne es zu lieben, nicht aus Neigung, oder Furcht, sondern aus Pflicht: alsdenn hat seine Maxime einen moralischen Gehalt. 
	
	\unnumberedsection{Juwel (1)} 
	\subsection*{tg175.2.5} 
	\textbf{Source : }Grundlegung zur Metaphysik der Sitten/Erster Abschnitt: Übergang von der gemeinen sittlichen Vernunfterkenntnis zur philosophischen\\  
	
	\noindent\textbf{Paragraphe : }Der gute Wille ist nicht durch das, was er bewirkt, oder ausrichtet, nicht durch seine Tauglichkeit zu Erreichung irgend eines vorgesetzten Zweckes, sondern allein durch das Wollen, d.i. an sich, gut, und, für sich selbst betrachtet, ohne Vergleich weit höher zu schätzen, als alles, was durch ihn zu Gunsten irgend einer Neigung, ja, wenn man will, der Summe aller Neigungen, nur immer zu Stande gebracht werden könnte. Wenn gleich durch eine besondere Ungunst des Schicksals, oder durch kärgliche Ausstattung einer stiefmütterlichen Natur, es diesem Willen gänzlich an Vermögen fehlete, seine Absicht durchzusetzen; wenn bei seiner größten Bestrebung dennoch nichts von ihm ausgerichtet würde, und nur der gute Wille (freilich nicht etwa ein bloßer Wunsch, sondern als die Aufbietung aller Mittel, so weit sie in unserer Gewalt sind) übrig bliebe: so würde er wie ein \match{Juwel} doch für sich selbst glänzen, als etwas, das seinen vollen Wert in sich selbst hat. Die Nützlichkeit oder Fruchtlosigkeit kann diesem Werte weder etwas zusetzen, noch abnehmen. Sie würde gleichsam nur die Einfassung sein, um ihn im gemeinen Verkehr besser handhaben zu können, oder die Aufmerksamkeit derer, die noch nicht gnug Kenner sind, auf sich zu ziehen, nicht aber, um ihn Kennern zu empfehlen, und seinen Wert zu bestimmen. 
	
	\unnumberedsection{Muster (1)} 
	\subsection*{tg176.2.7} 
	\textbf{Source : }Grundlegung zur Metaphysik der Sitten/Zweiter Abschnitt: Übergang von der populären sittlichen Weltweisheit zur Metaphysik der Sitten\\  
	
	\noindent\textbf{Paragraphe : }Man könnte auch der Sittlichkeit nicht übler raten, als wenn man sie von Beispielen entlehnen wollte. Denn jedes Beispiel, was mir davon vorgestellt wird, muß selbst zuvor nach Prinzipien der Moralität beurteilt werden, ob es auch würdig sei, zum ursprünglichen Beispiele, d.i. zum \match{Muster} zu dienen, keinesweges aber kann es den Begriff derselben zu oberst an die Hand geben. Selbst der Heilige des Evangelii muß zuvor mit unserm Ideal der sittlichen Vollkommenheit verglichen werden, ehe man ihn dafür erkennt; auch sagt er von sich selbst: was nennt ihr mich (den ihr sehet) gut, niemand ist gut (das Urbild des Guten) als der einige Gott (den ihr nicht sehet). Woher haben wir aber den Begriff von Gott, als dem höchsten Gut? Lediglich aus der Idee, die die Vernunft a priori von sittlicher Vollkommenheit entwirft, und mit dem Begriffe eines freien Willens unzertrennlich verknüpft. Nachahmung findet im Sittlichen gar nicht statt, und Beispiele dienen nur zur Aufmunterung, d.i. sie setzen die Tunlichkeit dessen, was das Gesetz gebietet, außer Zweifel, sie machen das, was die praktische Regel allgemeiner ausdrückt, anschaulich, können aber niemals  berechtigen, ihr wahres Original, das in der Vernunft liegt, bei Seite zu setzen und sich nach Beispielen zu richten. 
	
	\unnumberedsection{Schmuck (1)} 
	\subsection*{tg187.2.22} 
	\textbf{Source : }Grundlegung zur Metaphysik der Sitten/Fußnoten\\  
	
	\noindent\textbf{Paragraphe : }
	
	11 Die Tugend in ihrer eigentlichen Gestalt erblicken, ist nichts anders, als die Sittlichkeit, von aller Beimischung des Sinnlichen und allem unechten \match{Schmuck} des Lohns, oder der Selbstliebe, entkleidet, darzustellen. Wie sehr sie alsdenn alles übrige, was den Neigungen reizend erscheint, verdunkele, kann jeder vermittelst des mindesten Versuchs seiner nicht ganz für alle Abstraktion verdorbenen Vernunft leicht inne werden. 
	
	\unnumberedsection{Schritt (3)} 
	\subsection*{tg174.2.18} 
	\textbf{Source : }Grundlegung zur Metaphysik der Sitten/Vorrede\\  
	
	\noindent\textbf{Paragraphe : }3. Dritter Abschnitt: Letzter \match{Schritt} von der Metaphysik der Sitten zur Kritik der reinen praktischen Vernunft. 
	
	\subsection*{tg175.2.24} 
	\textbf{Source : }Grundlegung zur Metaphysik der Sitten/Erster Abschnitt: Übergang von der gemeinen sittlichen Vernunfterkenntnis zur philosophischen\\  
	
	\noindent\textbf{Paragraphe : }So wird also die gemeine Menschenvernunft nicht durch irgend ein Bedürfnis der Spekulation (welches ihr, so lange sie sich genügt, bloße gesunde Vernunft zu sein, niemals anwandelt), sondern selbst aus praktischen Gründen  angetrieben, aus ihrem Kreise zu gehen, und einen \match{Schritt} ins Feld einer praktischen Philosophie zu tun, um daselbst, wegen der Quelle ihres Prinzips und richtigen Bestimmung desselben in Gegenhaltung mit den Maximen, die sich auf Bedürfnis und Neigung fußen, Erkundigung und deutliche Anweisung zu bekommen, damit sie aus der Verlegenheit wegen beiderseitiger Ansprüche herauskomme, und nicht Gefahr laufe, durch die Zweideutigkeit, in die sie leicht gerät, um alle echte sittliche Grundsätze gebracht zu werden. Also entspinnt sich eben sowohl in der praktischen gemeinen Vernunft, wenn sie sich kultiviert, unvermerkt eine Dialektik, welche sie nötigt, in der Philosophie Hülfe zu suchen, als es ihr im theoretischen Gebrauche widerfährt, und die erstere wird daher wohl eben so wenig, als die andere, irgendwo sonst, als in einer vollständigen Kritik unserer Vernunft, Ruhe finden. 
	
	\subsection*{tg176.2.47} 
	\textbf{Source : }Grundlegung zur Metaphysik der Sitten/Zweiter Abschnitt: Übergang von der populären sittlichen Weltweisheit zur Metaphysik der Sitten\\  
	
	\noindent\textbf{Paragraphe : }Die Frage ist also diese: ist es ein notwendiges Gesetz für alle vernünftige Wesen, ihre Handlungen jederzeit nach solchen Maximen zu beurteilen, von denen sie selbst wollen können, daß sie zu allgemeinen Gesetzen dienen sollen? Wenn es ein solches ist, so muß es (völlig a priori) schon mit dem Begriffe des Willens eines vernünftigen Wesens überhaupt verbunden sein. Um aber diese Verknüpfung zu entdecken, muß man, so sehr man sich auch sträubt, einen \match{Schritt} hinaus tun, nämlich zur Metaphysik, obgleich in ein Gebiet derselben, welches von dem der spekulativen Philosophie unterschieden ist, nämlich in die Metaphysik der Sitten. In einer praktischen Philosophie, wo es uns nicht darum zu tun ist, Gründe anzunehmen, von dem, was geschieht, sondern Gesetze von dem, was geschehen soll, ob es gleich niemals geschieht, d.i. objektiv-praktische Gesetze: da haben wir nicht nötig, über die Gründe Untersuchung anzustellen, warum etwas gefällt oder mißfällt, wie das Vergnügen der bloßen Empfindung vom Geschmacke, und ob dieser von einem allgemeinen Wohlgefallen der Vernunft unterschieden sei; worauf Gefühl der Lust und Unlust beruhe, und wie hieraus Begierden und Neigungen, aus diesen aber, durch Mitwirkung der Vernunft, Maximen entspringen; denn das gehört alles zu einer empirischen Seelenlehre, welche den zweiten Teil der Naturlehre ausmachen würde, wenn man sie als Philosophie der Natur betrachtet, so fern sie auf empirischen Gesetzen gegründet ist. Hier aber ist vom objektiv-praktischen Gesetze die Rede, mithin von dem Verhältnisse eines Willens zu sich selbst, so fern er sich bloß durch Vernunft bestimmt, da denn alles, was aufs  Empirische Beziehung hat, von selbst wegfällt; weil, wenn die Vernunft für sich allein das Verhalten bestimmt (wovon wir die Möglichkeit jetzt eben untersuchen wollen), sie dieses notwendig a priori tun muß. 
	
	\unnumberedsection{Verarbeitung (1)} 
	\subsection*{tg174.2.13} 
	\textbf{Source : }Grundlegung zur Metaphysik der Sitten/Vorrede\\  
	
	\noindent\textbf{Paragraphe : }Weil aber drittens auch eine Metaphysik der Sitten, ungeachtet des abschreckenden Titels, dennoch eines großen Grades der Popularität und Angemessenheit zum gemeinen Verstande fähig ist, so finde ich für nützlich, diese \match{Verarbeitung} der Grundlage davon abzusondern, um das Subtile, was darin unvermeidlich ist, künftig nicht faßlichern Lehren beifügen zu dürfen. 
	
	\unnumberedchapter{Zoologie} 
	\unnumberedsection{Abstammung (1)} 
	\subsection*{tg176.2.46} 
	\textbf{Source : }Grundlegung zur Metaphysik der Sitten/Zweiter Abschnitt: Übergang von der populären sittlichen Weltweisheit zur Metaphysik der Sitten\\  
	
	\noindent\textbf{Paragraphe : }Alles also, was empirisch ist, ist, als Zutat zum Prinzip der Sittlichkeit, nicht allein dazu ganz untauglich, sondern der Lauterkeit der Sitten selbst höchst nachteilig, an welchen der eigentliche und über allen Preis erhabene Wert eines schlechterdings guten Willens eben darin besteht, daß das Prinzip der Handlung von allen Einflüssen zufälliger Gründe, die nur Erfahrung an die Hand geben kann, frei sei. Wider diese Nachlässigkeit oder gar niedrige Denkungsart, in Aufsuchung des Prinzips unter empirischen Bewegursachen und Gesetzen, kann man auch nicht zu viel und zu oft Warnungen ergehen lassen, indem die menschliche Vernunft in ihrer Ermüdung gern auf diesem Polster ausruht, und in dem Traume süßer Vorspiegelungen (die sie doch statt der Juno eine Wolke umarmen lassen) der Sittlichkeit einen aus Gliedern ganz verschiedener \match{Abstammung} zusammengeflickten Bastard unterschiebt, der allem ähnlich sieht, was man daran sehen will, nur der Tugend  nicht, für den, der sie einmal in ihrer wahren Gestalt erblickt hat.
	
	
	11
	
	
	
	\unnumberedsection{Art (21)} 
	\subsection*{tg174.2.12} 
	\textbf{Source : }Grundlegung zur Metaphysik der Sitten/Vorrede\\  
	
	\noindent\textbf{Paragraphe : }Im Vorsatze nun, eine Metaphysik der Sitten dereinst zu liefern, lasse ich diese Grundlegung vorangehen. Zwar gibt  es eigentlich keine andere Grundlage derselben, als die Kritik einer reinen praktischen Vernunft, so wie zur Metaphysik die schon gelieferte Kritik der reinen spekulativen Vernunft. Allein, teils ist jene nicht von so äußerster Notwendigkeit, als diese, weil die menschliche Vernunft im Moralischen, selbst beim gemeinsten Verstande, leicht zu großer Richtigkeit und Ausführlichkeit gebracht werden kann, da sie hingegen im theoretischen, aber reinen Gebrauch ganz und gar dialektisch ist; teils erfodere ich zur Kritik einer reinen praktischen Vernunft, daß, wenn sie vollendet sein soll, ihre Einheit mit der spekulativen in einem gemeinschaftlichen Prinzip zugleich müsse dargestellt werden können, weil es doch am Ende nur eine und dieselbe Vernunft sein kann, die bloß in der Anwendung unterschieden sein muß. Zu einer solchen Vollständigkeit konnte ich es aber hier noch nicht bringen, ohne Betrachtungen von ganz anderer \match{Art} herbeizuziehen und den Leser zu verwirren. Um deswillen habe ich mich, statt der Benennung einer Kritik der reinen praktischen Vernunft, der von einer Grundlegung zur Metaphysik der Sitten bedient. 
	
	\subsection*{tg174.2.2} 
	\textbf{Source : }Grundlegung zur Metaphysik der Sitten/Vorrede\\  
	
	\noindent\textbf{Paragraphe : }Die alte griechische Philosophie teilte sich in drei Wissenschaften ab: Die Physik, die Ethik, und die Logik. Diese Einteilung ist der Natur der Sache vollkommen angemessen, und man hat an ihr nichts zu verbessern, als etwa nur das Prinzip derselben hinzu zu tun, um sich auf solche \match{Art} teils ihrer Vollständigkeit zu versichern, teils die notwendigen Unterabteilungen richtig bestimmen zu können. 
	
	\subsection*{tg175.2.20} 
	\textbf{Source : }Grundlegung zur Metaphysik der Sitten/Erster Abschnitt: Übergang von der gemeinen sittlichen Vernunfterkenntnis zur philosophischen\\  
	
	\noindent\textbf{Paragraphe : }Die Frage sei z.B.: darf ich, wenn ich im Gedränge bin, nicht ein Versprechen tun, in der Absicht, es nicht zu halten? Ich mache hier leicht den Unterschied, den die Bedeutung der Frage haben kann, ob es klüglich, oder ob es pflichtmäßig sei, ein falsches Versprechen zu tun. Das erstere kann ohne Zweifel öfters stattfinden. Zwar sehe ich wohl, daß es nicht gnug sei, mich vermittelst dieser Ausflucht aus einer gegenwärtigen Verlegenheit zu ziehen, sondern wohl überlegt werden müsse, ob mir aus dieser Lüge nicht hinterher viel größere Ungelegenheit entspringen könne, als die sind, von denen ich mich jetzt befreie, und, da die Folgen bei aller meiner vermeinten Schlauigkeit nicht so leicht vorauszusehen sind, daß nicht ein einmal verlornes Zutrauen mir weit nachteiliger werden könnte, als alles Übel, das ich jetzt zu vermeiden gedenke, ob es nicht klüglicher gehandelt sei, hiebei nach einer allgemeinen Maxime zu verfahren, und es sich zur Gewohnheit zu machen, nichts zu versprechen, als in der Absicht, es zu halten. Allein es leuchtet mir hier bald ein, daß eine solche Maxime doch immer nur die besorglichen Folgen zum Grunde habe. Nun ist es doch etwas ganz anderes, aus Pflicht wahrhaft zu sein, als aus Besorgnis der nachteiligen Folgen; indem, im ersten Falle, der Begriff der Handlung an sich selbst schon ein Gesetz für mich enthält, im zweiten ich mich allererst anderwärtsher umsehen muß, welche Wirkungen für mich wohl damit verbunden sein möchten. Denn, wenn ich von dem Prinzip der Pflicht abweiche, so ist es ganz gewiß böse; werde ich aber meiner Maxime der Klugheit abtrünnig, so kann das mir doch manchmal sehr vorteilhaft sein, wiewohl es freilich sicherer ist, bei ihr zu bleiben. Um indessen mich in Ansehung der Beantwortung dieser Aufgabe, ob ein lügenhaftes Versprechen pflichtmäßig sei, auf die allerkürzeste und doch untrügliche \match{Art} zu belehren, so frage ich mich selbst: würde ich wohl damit zufrieden sein, daß meine Maxime (mich durch ein unwahres Versprechen aus Verlegenheit zu  ziehen) als ein allgemeines Gesetz (sowohl für mich als andere) gelten solle, und würde ich wohl zu mir sagen können: es mag jedermann ein unwahres Versprechen tun, wenn er sich in Verlegenheit befindet, daraus er sich auf andere Art nicht ziehen kann? So werde ich bald inne, daß ich zwar die Lüge, aber ein allgemeines Gesetz zu lügen gar nicht wollen könne; denn nach einem solchen würde es eigentlich gar kein Versprechen geben, weil es vergeblich wäre, meinen Willen in Ansehung meiner künftigen Handlungen andern vorzugeben, die diesem Vorgeben doch nicht glauben, oder, wenn sie es übereilter Weise täten, mich doch mit gleicher Münze bezahlen würden, mithin meine Maxime, so bald sie zum allgemeinen Gesetze gemacht würde, sich selbst zerstören müsse. 
	
	\subsection*{tg176.2.12} 
	\textbf{Source : }Grundlegung zur Metaphysik der Sitten/Zweiter Abschnitt: Übergang von der populären sittlichen Weltweisheit zur Metaphysik der Sitten\\  
	
	\noindent\textbf{Paragraphe : }Aus dem Angeführten erhellet: daß alle sittliche Begriffe völlig a priori in der Vernunft ihren Sitz und Ursprung haben, und dieses zwar in der gemeinsten Menschenvernunft eben sowohl, als der im höchsten Maße spekulativen; daß sie von keinem empirischen und darum bloß zufälligen Erkenntnisse  abstrahiert werden können; daß in dieser Reinigkeit ihres Ursprungs eben ihre Würde liege, um uns zu obersten praktischen Prinzipien zu dienen; daß man jedesmal so viel, als man Empirisches hinzu tut, so viel auch ihrem echten Einflusse und dem uneingeschränkten Werte der Handlungen entziehe; daß es nicht allein die größte Notwendigkeit in theoretischer Absicht, wenn es bloß auf Spekulation ankommt, erfodere, sondern auch von der größten praktischen Wichtigkeit sei, ihre Begriffe und Gesetze aus reiner Vernunft zu schöpfen, rein und unvermengt vorzutragen, ja den Umfang dieses ganzen praktischen oder reinen Vernunfterkenntnisses, d.i. das ganze Vermögen der reinen praktischen Vernunft, zu bestimmen, hierin aber nicht, wie es wohl die spekulative Philosophie erlaubt, ja gar bisweilen notwendig findet, die Prinzipien von der besondern Natur der menschlichen Vernunft abhängig zu machen, sondern darum, weil moralische Gesetze für jedes vernünftige Wesen überhaupt gelten sollen, sie schon aus dem allgemeinen Begriffe eines vernünftigen Wesens überhaupt abzuleiten, und auf solche Weise alle Moral, die zu ihrer Anwendung auf Menschen der Anthropologie bedarf, zuerst unabhängig von dieser als reine Philosophie, d.i. als Metaphysik, vollständig (welches sich in dieser \match{Art} ganz abgesonderter Erkenntnisse wohl tun läßt) vorzutragen, wohl bewußt, daß es, ohne im Be sitze derselben zu sein, vergeblich sei, ich will nicht sagen, das Moralische der Pflicht in allem, was pflichtmäßig ist, genau für die spekulative Beurteilung zu bestimmen, sondern so gar im bloß gemeinen und praktischen Gebrauche, vornehmlich der moralischen Unterweisung, unmöglich sei, die Sitten auf ihre echte Prinzipien zu gründen und dadurch reine moralische Gesinnungen zu bewirken und zum höchsten Weltbesten den Gemütern einzupfropfen. 
	
	\subsection*{tg176.2.13} 
	\textbf{Source : }Grundlegung zur Metaphysik der Sitten/Zweiter Abschnitt: Übergang von der populären sittlichen Weltweisheit zur Metaphysik der Sitten\\  
	
	\noindent\textbf{Paragraphe : }Um aber in dieser Bearbeitung nicht bloß von der gemeinen sittlichen Beurteilung (die hier sehr achtungswürdig ist) zur philosophischen, wie sonst geschehen ist, sondern von einer populären Philosophie, die nicht weiter geht, als  sie durch Tappen vermittelst der Beispiele kommen kann, bis zur Metaphysik (die sich durch nichts Empirisches weiter zurückhalten läßt, und, indem sie den ganzen Inbegriff der Vernunfterkenntnis dieser \match{Art} ausmessen muß, allenfalls bis zu Ideen geht, wo selbst die Beispiele uns verlassen) durch die natürlichen Stufen fortzuschreiten: müssen wir das praktische Vernunftvermögen von seinen allgemeinen Bestimmungsregeln an, bis dahin, wo aus ihm der Begriff der Pflicht entspringt, verfolgen und deutlich darstellen. 
	
	\subsection*{tg176.2.19} 
	\textbf{Source : }Grundlegung zur Metaphysik der Sitten/Zweiter Abschnitt: Übergang von der populären sittlichen Weltweisheit zur Metaphysik der Sitten\\  
	
	\noindent\textbf{Paragraphe : }Weil jedes praktische Gesetz eine mögliche Handlung als gut und darum, für ein durch Vernunft praktisch bestimmbares Subjekt, als notwendig vorstellt, so sind alle Imperativen Formeln der Bestimmung der Handlung, die nach dem Prinzip eines in irgend einer \match{Art} guten Willens notwendig ist. Wenn nun die Handlung bloß wozu anderes, als Mittel, gut sein würde, so ist der Imperativ hypothetisch; wird sie als an sich gut vorgestellt, mithin als notwendig in einem an sich der Vernunftgemäßen Willen, als Prinzip desselben, so ist er kategorisch. 
	
	\subsection*{tg176.2.22} 
	\textbf{Source : }Grundlegung zur Metaphysik der Sitten/Zweiter Abschnitt: Übergang von der populären sittlichen Weltweisheit zur Metaphysik der Sitten\\  
	
	\noindent\textbf{Paragraphe : }Man kann sich das, was nur durch Kräfte irgend eines vernünftigen Wesens möglich ist, auch für irgend einen Willen als mögliche Absicht denken, und daher sind der Prinzipien der Handlung, so fern diese als notwendig vorgestellt wird, um irgend eine dadurch zu bewirkende mögliche Absicht zu erreichen, in der Tat unendlich viel. Alle Wissenschaften haben irgend einen praktischen Teil, der aus Aufgaben besteht, daß irgend ein Zweck für uns möglich sei, und aus Imperativen, wie er erreicht werden könne. Diese können daher überhaupt Imperativen der Geschicklichkeit heißen. Ob der Zweck vernünftig und gut sei, davon ist hier gar nicht die Frage, sondern nur, was man tun müsse, um ihn zu erreichen. Die Vorschriften für den Arzt, um seinen Mann auf gründliche \match{Art} gesund zu machen, und für einen Giftmischer, um ihn sicher zu töten, sind in so fern von gleichem Wert, als eine jede dazu dient, ihre Absicht vollkommen zu bewirken. Weil man in der frühen Jugend nicht weiß, welche Zwecke uns im Leben aufstoßen dürften, so suchen Eltern vornehmlich ihre Kinder recht vielerlei lernen zu lassen, und sorgen für die Geschicklichkeit im Gebrauch der Mittel zu allerlei beliebigen Zwecken, von deren keinem sie bestimmen können, ob er nicht etwa wirklich künftig eine Absicht ihres Zöglings werden könne, wovon es indessen doch möglich ist, daß er sie einmal haben möchte, und diese Sorgfalt ist so groß, daß sie darüber gemeiniglich verabsäumen, ihnen das Urteil über den Wert der Dinge, die sie sich etwa zu Zwecken machen möchten, zu bilden und zu berichtigen. 
	
	\subsection*{tg176.2.26} 
	\textbf{Source : }Grundlegung zur Metaphysik der Sitten/Zweiter Abschnitt: Übergang von der populären sittlichen Weltweisheit zur Metaphysik der Sitten\\  
	
	\noindent\textbf{Paragraphe : }Nun entsteht die Frage: wie sind alle diese Imperative möglich? Diese Frage verlangt nicht zu wissen, wie die Vollziehung der Handlung, welche der Imperativ gebietet, sondern wie bloß die Nötigung des Willens, die der Imperativ in der Aufgabe ausdrückt, gedacht werden könne. Wie ein Imperativ der Geschicklichkeit möglich sei, bedarf wohl keiner besondern Erörterung. Wer den Zweck will, will (so fern die Vernunft auf seine Handlungen entscheidenden Einfluß hat) auch das dazu unentbehrlich notwendige Mittel, das in seiner Gewalt ist. Dieser Satz ist, was das Wollen betrifft, analytisch; denn in dem Wollen eines Objekts, als meiner Wirkung, wird schon meine Kausalität, als handelnder Ursache, d.i. der Gebrauch der Mittel, gedacht, und der Imperativ zieht den Begriff notwendiger Handlungen zu diesem Zwecke schon aus dem Begriff eines Wollens dieses Zwecks heraus (die Mittel selbst zu einer vorgesetzten Absicht  zu bestimmen, dazu gehören allerdings synthetische Sätze, die aber nicht den Grund betreffen, den Actus des Willens, sondern das Objekt wirklich zu machen). Daß, um eine Linie nach einem sichern Prinzip in zwei gleiche Teile zu teilen, ich aus den Enden derselben zwei Kreuzbogen machen müsse, das lehrt die Mathematik freilich nur durch synthetische Sätze; aber daß, wenn ich weiß, durch solche Handlung allein könne die gedachte Wirkung geschehen, ich, wenn ich die Wirkung vollständig will, auch die Handlung wolle, die dazu erfoderlich ist, ist ein analytischer Satz; denn etwas als eine auf gewisse \match{Art} durch mich mögliche Wirkung, und mich, in Ansehung ihrer, auf dieselbe Art handelnd vorstellen, ist ganz einerlei. 
	
	\subsection*{tg176.2.28} 
	\textbf{Source : }Grundlegung zur Metaphysik der Sitten/Zweiter Abschnitt: Übergang von der populären sittlichen Weltweisheit zur Metaphysik der Sitten\\  
	
	\noindent\textbf{Paragraphe : }Dagegen, wie der Imperativ der Sittlichkeit möglich sei, ist ohne Zweifel die einzige einer Auflösung bedürftige Frage, da er gar nicht hypothetisch ist und also die objektiv vorgestellte Notwendigkeit sich auf keine Voraussetzung stützen kann, wie bei den hypothetischen Imperativen. Nur ist immer hiebei nicht aus der Acht zu lassen, daß es durch kein Beispiel, mithin empirisch auszumachen sei, ob es überall irgend einen dergleichen Imperativ gebe, sondern zu besorgen, daß alle, die kategorisch scheinen, doch versteckter Weise hypothetisch sein mögen. Z.B. wenn es heißt: du sollst nichts betrüglich versprechen; und man nimmt an, daß die Notwendigkeit dieser Unterlassung nicht etwa bloße Ratgebung zu Vermeidung irgend eines andern Übels sei, so daß es etwa hieße: du sollst nicht lügenhaft versprechen, damit du nicht, wenn es offenbar wird, dich um den Kredit bringest; sondern eine Handlung dieser \match{Art} müsse für sich selbst als böse betrachtet werden, der Imperativ des Verbots sei also kategorisch: so kann man doch in keinem Beispiel mit Gewißheit dartun, daß der Wille hier ohne andere Triebfeder, bloß durchs Gesetz, bestimmt werde, ob es gleich so scheint; denn es ist immer möglich, daß ingeheim Furcht für Beschämung, vielleicht auch dunkle Besorgnis anderer Gefahren, Einfluß auf den Willen haben möge. Wer kann das Nichtsein einer Ursache durch Erfahrung beweisen, da diese nichts weiter lehrt, als daß wir jene nicht wahrnehmen? Auf solchen Fall aber würde der sogenannte moralische Imperativ, der als ein solcher kategorisch und unbedingt erscheint, in der Tat nur eine pragmatische Vorschrift sein, die uns auf unsern Vorteil aufmerksam macht, und uns bloß lehrt, diesen in Acht zu nehmen. 
	
	\subsection*{tg176.2.30} 
	\textbf{Source : }Grundlegung zur Metaphysik der Sitten/Zweiter Abschnitt: Übergang von der populären sittlichen Weltweisheit zur Metaphysik der Sitten\\  
	
	\noindent\textbf{Paragraphe : }Zweitens ist bei diesem kategorischen Imperativ oder Gesetze der Sittlichkeit der Grund der Schwierigkeit (die Möglichkeit desselben einzusehen) auch sehr groß. Er ist ein synthetisch-praktischer Satz
	
	
	8
	a priori, und da die Möglichkeit der Sätze dieser \match{Art} einzusehen so viel Schwierigkeit im theoretischen Erkenntnisse hat, so läßt sich leicht abnehmen, daß sie im praktischen nicht weniger haben werde. 
	
	\subsection*{tg176.2.38} 
	\textbf{Source : }Grundlegung zur Metaphysik der Sitten/Zweiter Abschnitt: Übergang von der populären sittlichen Weltweisheit zur Metaphysik der Sitten\\  
	
	\noindent\textbf{Paragraphe : }2) Ein anderer sieht sich durch Not gedrungen, Geld zu borgen. Er weiß wohl, daß er nicht wird bezahlen können, sieht aber auch, daß ihm nichts geliehen werden wird, wenn er nicht festiglich verspricht, es zu einer bestimmten Zeit zu bezahlen. Er hat Lust, ein solches Versprechen zu tun; noch aber hat er so viel Gewissen, sich zu fragen: ist es nicht unerlaubt und pflichtwidrig, sich auf solche \match{Art} aus Not zu helfen? Gesetzt, er beschlösse es doch, so würde seine Maxime der Handlung so lauten: wenn ich mich in Geldnot zu sein glaube, so will ich Geld borgen, und versprechen, es zu bezahlen, ob ich gleich weiß, es werde niemals geschehen. Nun ist dieses Prinzip der Selbstliebe, oder der eigenen Zuträglichkeit, mit meinem ganzen künftigen Wohlbefinden vielleicht wohl zu vereinigen, allein jetzt ist die Frage: ob es recht sei? Ich verwandle also die Zumutung der Selbstliebe in ein allgemeines Gesetz, und richte die Frage so ein: wie es dann stehen würde, wenn meine Maxime ein allgemeines Gesetz würde. Da sehe ich nun sogleich, daß sie niemals als allgemeines Naturgesetz gelten und mit sich selbst zusammenstimmen könne, sondern sich notwendig widersprechen müsse. Denn die Allgemeinheit eines Gesetzes, daß jeder, nachdem er in Not zu sein glaubt, versprechen könne, was ihm einfällt, mit dem Vorsatz, es nicht zu halten, würde das Versprechen und den Zweck, den man damit haben mag, selbst unmöglich machen, indem niemand glauben würde, daß ihm was versprochen sei, sondern über alle solche Äußerung, als eitles Vorgeben, lachen würde. 
	
	\subsection*{tg176.2.41} 
	\textbf{Source : }Grundlegung zur Metaphysik der Sitten/Zweiter Abschnitt: Übergang von der populären sittlichen Weltweisheit zur Metaphysik der Sitten\\  
	
	\noindent\textbf{Paragraphe : }Dieses sind nun einige von den vielen wirklichen oder wenigstens von uns dafür gehaltenen Pflichten, deren Abteilung aus dem einigen angeführten Prinzip klar in die Augen fällt. Man muß wollen können, daß eine Maxime unserer Handlung ein allgemeines Gesetz werde: dies ist der Kanon der moralischen Beurteilung derselben überhaupt. Einige Handlungen sind so beschaffen, daß ihre Maxime ohne Widerspruch nicht einmal als allgemeines Naturgesetz 
	gedacht werden kann; weit gefehlt, daß man noch wollen könne, es sollte ein solches werden. Bei andern ist zwar jene innere Unmöglichkeit nicht anzutreffen, aber es ist doch unmöglich, zu wollen, daß ihre Maxime zur Allgemeinheit eines Naturgesetzes erhoben werde, weil ein solcher Wille sich selbst widersprechen würde. Man sieht leicht: daß die erstere der strengen oder engeren (unnachlaßlichen) Pflicht, die zweite nur der weiteren (verdienstlichen) Pflicht widerstreite, und so alle Pflichten, was die \match{Art} der Verbindlichkeit (nicht das Objekt ihrer Handlung) betrifft, durch diese Beispiele in ihrer Abhängigkeit von dem einigen Prinzip vollständig aufgestellt worden. 
	
	\subsection*{tg176.2.72} 
	\textbf{Source : }Grundlegung zur Metaphysik der Sitten/Zweiter Abschnitt: Übergang von der populären sittlichen Weltweisheit zur Metaphysik der Sitten\\  
	
	\noindent\textbf{Paragraphe : }Nun ist Moralität die Bedingung, unter der allein ein vernünftiges Wesen Zweck an sich selbst sein kann; weil nur durch sie es möglich ist, ein gesetzgebend Glied im Reiche der Zwecke zu sein. Also ist Sittlichkeit und die Menschheit, so fern sie derselben fähig ist, dasjenige, was allein Würde hat. Geschicklichkeit und Fleiß im Arbeiten haben einen Marktpreis; Witz, lebhafte Einbildungskraft und Launen einen Affektionspreis; dagegen Treue im Versprechen, Wohlwollen aus Grundsätzen (nicht aus Instinkt) haben einen innern Wert. Die Natur sowohl als Kunst enthalten nichts, was sie, in Ermangelung derselben, an ihre Stelle setzen könnten; denn ihr Wert besteht nicht in den Wirkungen, die daraus entspringen, im Vorteil und Nutzen, den sie schaffen, sondern in den Gesinnungen, d.i. den Maximen des Willens, die sich auf diese \match{Art} in Handlungen zu offenbaren bereit sind, obgleich auch der Erfolg sie nicht begünstigte. Diese Handlungen bedürfen auch keiner Empfehlung von irgend einer subjektiven Disposition oder Geschmack, sie mit unmittelbarer Gunst und Wohlgefallen anzusehen, keines unmittelbaren Hanges oder Gefühles für dieselbe: sie stellen den Willen, der sie ausübt, als Gegenstand einer unmittelbaren Achtung dar, dazu nichts als Vernunft gefodert wird, um sie dem Willen aufzuerlegen, nicht von  ihm zu erschmeicheln, welches letztere bei Pflichten ohnedem ein Widerspruch wäre. Diese Schätzung gibt also den Wert einer solchen Denkungsart als Würde zu erkennen, und setzt sie über allen Preis unendlich weg, mit dem sie gar nicht in Anschlag und Vergleichung gebracht werden kann, ohne sich gleichsam an der Heiligkeit derselben zu vergreifen. 
	
	\subsection*{tg181.2.2} 
	\textbf{Source : }Grundlegung zur Metaphysik der Sitten/Dritter Abschnitt: Übergang von der Metaphysik der Sitten zur Kritik der reinen praktischen Vernunft/Der Begriff der Freiheit ist der Schlüssel zur Erklärung der Autonomie des Willens\\  
	
	\noindent\textbf{Paragraphe : }Der Wille ist eine \match{Art} von Kausalität lebender Wesen, so fern sie vernünftig sind, und Freiheit würde diejenige Eigenschaft dieser Kausalität sein, da sie unabhängig von fremden sie bestimmenden Ursachen wirkend sein kann; so wie Naturnotwendigkeit die Eigenschaft der Kausalität aller vernunftlosen Wesen, durch den Einfluß fremder Ursachen zur Tätigkeit bestimmt zu werden. 
	
	\subsection*{tg183.2.4} 
	\textbf{Source : }Grundlegung zur Metaphysik der Sitten/Dritter Abschnitt: Übergang von der Metaphysik der Sitten zur Kritik der reinen praktischen Vernunft/Von dem Interesse, welches den Ideen der Sittlichkeit anhängt\\  
	
	\noindent\textbf{Paragraphe : }Es scheint also, als setzten wir in der Idee der Freiheit eigentlich das moralische Gesetz, nämlich das Prinzip der Autonomie des Willens selbst, nur voraus, und könnten  seine Realität und objektive Notwendigkeit nicht für sich beweisen, und da hätten wir zwar noch immer etwas ganz Beträchtliches dadurch gewonnen, daß wir wenigstens das echte Prinzip genauer, als wohl sonst geschehen, bestimmt hätten, in Ansehung seiner Gültigkeit aber, und der praktischen Notwendigkeit, sich ihm zu unterwerfen, wären wir um nichts weiter gekommen; denn wir könnten dem, der uns fragte, warum denn die Allgemeingültigkeit unserer Maxime, als eines Gesetzes, die einschränkende Bedingung unserer Handlungen sein müsse, und worauf wir den Wertgründen, den wir dieser \match{Art} zu handeln beilegen, der so groß sein soll, daß es überall kein höheres Interesse geben kann, und wie es zugehe, daß der Mensch dadurch allein seinen persönlichen Wert zu fühlen glaubt, gegen den der, eines angenehmen oder unangenehmen Zustandes, für nichts zu halten sei, keine genugtuende Antwort geben. 
	
	\subsection*{tg183.2.5} 
	\textbf{Source : }Grundlegung zur Metaphysik der Sitten/Dritter Abschnitt: Übergang von der Metaphysik der Sitten zur Kritik der reinen praktischen Vernunft/Von dem Interesse, welches den Ideen der Sittlichkeit anhängt\\  
	
	\noindent\textbf{Paragraphe : }Zwar finden wir wohl, daß wir an einer persönlichen Beschaffenheit ein Interesse nehmen können, die gar kein Interesse des Zustandes bei sich führt, wenn jene uns nur fähig macht, des letzteren teilhaftig zu werden, im Falle die Vernunft die Austeilung desselben bewirken sollte, d.i. daß die bloße Würdigkeit, glücklich zu sein, auch ohne den Bewegungsgrund, dieser Glückseligkeit teilhaftig zu werden, für sich interessieren könne: aber dieses Urteil ist in der Tat nur die Wirkung von der schon vorausgesetzten Wichtigkeit moralischer Gesetze (wenn wir uns durch die Idee der Freiheit von allem empirischen Interesse trennen), aber, daß wir uns von diesem trennen, d.i. uns als frei im Handeln betrachten, und so uns dennoch für gewissen Gesetzen unterworfen halten sollen, um einen Wert bloß in unserer Person zu finden, der uns allen Verlust dessen, was unserem Zustande einen Wert verschafft, vergüten könne, und wie dieses möglich sei, mithin woher das moralische Gesetz verbinde, können wir auf solche \match{Art} noch nicht einsehen. 
	
	\subsection*{tg183.2.6} 
	\textbf{Source : }Grundlegung zur Metaphysik der Sitten/Dritter Abschnitt: Übergang von der Metaphysik der Sitten zur Kritik der reinen praktischen Vernunft/Von dem Interesse, welches den Ideen der Sittlichkeit anhängt\\  
	
	\noindent\textbf{Paragraphe : }Es zeigt sich hier, man muß es frei gestehen, eine \match{Art} von Zirkel, aus dem, wie es scheint, nicht heraus zu kommen ist. Wir nehmen uns in der Ordnung der wirkenden Ursachen  als frei an, um uns in der Ordnung der Zwecke unter sittlichen Gesetzen zu denken, und wir denken uns nachher als diesen Gesetzen unterworfen, weil wir uns die Freiheit des Willens beigelegt haben, denn Freiheit und eigene Gesetzgebung des Willens sind beides Autonomie, mithin Wechselbegriffe, davon aber einer eben um deswillen nicht dazu gebraucht werden kann, um den anderen zu erklären und von ihm Grund anzugeben, sondern höchstens nur, um, in logischer Absicht, verschieden scheinende Vorstellungen von eben demselben Gegenstande auf einen einzigen Begriff (wie verschiedne Brüche gleiches Inhalts auf die kleinsten Ausdrücke) zu bringen. 
	
	\subsection*{tg183.2.8} 
	\textbf{Source : }Grundlegung zur Metaphysik der Sitten/Dritter Abschnitt: Übergang von der Metaphysik der Sitten zur Kritik der reinen praktischen Vernunft/Von dem Interesse, welches den Ideen der Sittlichkeit anhängt\\  
	
	\noindent\textbf{Paragraphe : }Es ist eine Bemerkung, welche anzustellen eben kein subtiles Nachdenken erfodert wird, sondern von der man annehmen kann, daß sie wohl der gemeinste Verstand, obzwar, nach seiner Art, durch eine dunkele Unterscheidung der Urteilskraft, die er Gefühl nennt, machen mag: daß alle Vorstellungen, die uns ohne unsere Willkür kommen (wie die der Sinne), uns die Gegenstände nicht anders zu erkennen geben, als sie uns affizieren, wobei, was sie an sich sein mögen, uns unbekannt bleibt, mithin daß, was diese \match{Art} Vorstellungen betrifft, wir dadurch, auch bei der angestrengtesten Aufmerksamkeit und Deutlichkeit, die der Verstand nur immer hinzufügen mag, doch bloß zur Erkenntnis der Erscheinungen, niemals der Dinge an sich selbst gelangen können. Sobald dieser Unterschied (allenfalls bloß durch die bemerkte Verschiedenheit zwischen den Vorstellungen, die uns anders woher gegeben werden, und dabei wir leidend sind, von denen, die wir lediglich aus uns selbst hervorbringen, und dabei wir unsere Tätigkeit beweisen) einmal gemacht ist, so folgt von selbst, daß man hinter den Erscheinungen doch noch etwas anderes, was nicht Erscheinung  ist, nämlich die Dinge an sich, einräumen und annehmen müsse, ob wir gleich uns von selbst bescheiden, daß, da sie uns niemals bekannt werden können, sondern immer nur, wie sie uns affizieren, wir ihnen nicht näher treten, und, was sie an sich sind, niemals wissen können. Dieses muß eine, obzwar rohe, Unterscheidung einer Sinnenwelt von der Verstandeswelt abgeben, davon die erstere, nach Verschiedenheit der Sinnlichkeit in mancherlei Weltbeschauern, auch sehr verschieden sein kann, indessen die zweite, die ihr zum Grunde liegt, immer dieselbe bleibt. So gar sich selbst und zwar nach der Kenntnis, die der Mensch durch innere Empfindung von sich hat, darf er sich nicht anmaßen zu erkennen, wie er an sich selbst sei. Denn da er doch sich selbst nicht gleichsam schafft, und seinen Begriff nicht a priori, sondern empirisch bekömmt, so ist natürlich, daß er auch von sich durch den innern Sinn und folglich nur durch die Erscheinung seiner Natur, und die Art, wie sein Bewußtsein affiziert wird, Kundschaft einziehen könne, indessen er doch notwendiger Weise über diese aus lauter Erscheinungen zusammengesetzte Beschaffenheit seines eigenen Subjekts noch etwas anderes zum Grunde Liegendes, nämlich sein Ich, so wie es an sich selbst beschaffen sein mag, annehmen, und sich also in Absicht auf die bloße Wahrnehmung und Empfänglichkeit der Empfindungen zur Sinnenwelt, in Ansehung dessen aber, was in ihm reine Tätigkeit sein mag (dessen, was gar nicht durch Affizierung der Sinne, sondern unmittelbar zum Bewußtsein gelangt), sich zur intellektuellen Welt zählen muß, die er doch nicht weiter kennt. 
	
	\subsection*{tg185.2.13} 
	\textbf{Source : }Grundlegung zur Metaphysik der Sitten/Dritter Abschnitt: Übergang von der Metaphysik der Sitten zur Kritik der reinen praktischen Vernunft/Von der äußersten Grenze aller praktischen Philosophie\\  
	
	\noindent\textbf{Paragraphe : }Um das zu wollen, wozu die Vernunft allein dem sinnlich-affizierten vernünftigen Wesen das Sollen vorschreibt, dazu gehört freilich ein Vermögen der Vernunft, ein Gefühl der Lust oder des Wohlgefallens an der Erfüllung der Pflicht einzuflößen, mithin eine Kausalität derselben, die Sinnlichkeit ihren Prinzipien gemäß zu bestimmen. Es ist aber gänzlich unmöglich, einzusehen, d.i. a priori begreiflich zu machen, wie ein bloßer Gedanke, der selbst nichts Sinnliches in sich enthält, eine Empfindung der Lust oder Unlust hervorbringe; denn das ist eine besondere \match{Art} von Kausalität, von der, wie von aller Kausalität, wir gar nichts a priori bestimmen können, sondern darum allein die Erfahrung befragen müssen. Da diese aber kein Verhältnis der Ursache zur Wirkung, als zwischen zwei Gegenständen der Erfahrung, an die Hand geben kann, hier aber reine Vernunft durch bloße Ideen (die gar keinen Gegenstand für Erfahrung abgeben) die Ursache von einer Wirkung, die freilich in der Erfahrung liegt, sein soll, so ist die Erklärung, wie und warum uns die Allgemeinheit der Maxime als Gesetzes, mithin die Sittlichkeit, interessiere, uns Menschen gänzlich unmöglich. So viel ist nur gewiß: daß es nicht darum für uns Gültigkeit hat, weil es interessiert (denn das ist Heteronomie und Abhängigkeit der praktischen Vernunft von Sinnlichkeit, nämlich einem zum Grunde liegenden Gefühl, wobei sie niemals sittlich gesetzgebend sein könnte), sondern daß es interessiert, weil es für uns als Menschen gilt, da es aus unserem Willen als Intelligenz, mithin aus unserem eigentlichen Selbst, entsprungen ist; was aber zur bloßen Erscheinung gehört, wird von der Vernunft notwendig der Beschaffenheit der Sache an sich selbst untergeordnet.
	
	
	\subsection*{tg185.2.4} 
	\textbf{Source : }Grundlegung zur Metaphysik der Sitten/Dritter Abschnitt: Übergang von der Metaphysik der Sitten zur Kritik der reinen praktischen Vernunft/Von der äußersten Grenze aller praktischen Philosophie\\  
	
	\noindent\textbf{Paragraphe : }Indessen muß dieser Scheinwiderspruch wenigstens auf überzeugende \match{Art} vertilgt werden, wenn man gleich, wie Freiheit möglich sei, niemals begreifen könnte. Denn, wenn sogar der Gedanke von der Freiheit sich selbst, oder der  Natur, die eben so notwendig ist, widerspricht, so mußte sie gegen die Naturnotwendigkeit durchaus aufgegeben werden. 
	
	\subsection*{tg187.2.12} 
	\textbf{Source : }Grundlegung zur Metaphysik der Sitten/Fußnoten\\  
	
	\noindent\textbf{Paragraphe : }
	
	6 Das Wort Klugheit wird in zwiefachem Sinn genommen, einmal kann es den Namen Weltklugheit, im zweiten den der Privatklugheit führen. Die erste ist die Geschicklichkeit eines Menschen, auf andere Einfluß zu haben, um sie zu seinen Absichten zu gebrauchen. Die zweite die Einsicht, alle diese Absichten zu seinem eigenen daurenden Vorteil zu vereinigen. Die letztere ist eigentlich diejenige, worauf selbst der Wert der erstern zurückgeführt wird, und wer in der erstern \match{Art} klug ist, nicht aber in der zweiten, von dem könnte man besser sagen: er ist gescheut und verschlagen, im ganzen aber doch unklug. 
	
	\unnumberedsection{Fall (1)} 
	\subsection*{tg176.2.4} 
	\textbf{Source : }Grundlegung zur Metaphysik der Sitten/Zweiter Abschnitt: Übergang von der populären sittlichen Weltweisheit zur Metaphysik der Sitten\\  
	
	\noindent\textbf{Paragraphe : }In der Tat ist es schlechterdings unmöglich, durch Erfahrung einen einzigen \match{Fall} mit völliger Gewißheit auszumachen, da die Maxime einer sonst pflichtmäßigen Handlung lediglich auf moralischen Gründen und auf der Vorstellung seiner Pflicht beruhet habe. Denn es ist zwar bisweilen der Fall, daß wir bei der schärfsten Selbstprüfung gar nichts antreffen, was außer dem moralischen Grunde der Pflicht mächtig genug hätte sein können, uns zu dieser oder jener guten Handlung und so großer Aufopferung zu bewegen; es kann aber daraus gar nicht mit Sicherheit geschlossen werden, daß wirklich gar kein geheimer Antrieb der Selbstliebe, unter der bloßen Vorspiegelung jener Idee, die eigentliche bestimmende Ursache des Willens gewesen sei, dafür wir denn gerne uns mit einem uns fälschlich angemaßten edlern Bewegungsgrunde schmeicheln, in der Tat aber selbst durch die angestrengteste Prüfung hinter die geheimen Triebfedern niemals völlig kommen können, weil, wenn vom moralischen Werte die Rede ist, es nicht auf die Handlungen ankommt, die man sieht, sondern auf jene innere Prinzipien derselben, die man nicht sieht. 
	
	\unnumberedsection{Flugel (1)} 
	\subsection*{tg185.2.16} 
	\textbf{Source : }Grundlegung zur Metaphysik der Sitten/Dritter Abschnitt: Übergang von der Metaphysik der Sitten zur Kritik der reinen praktischen Vernunft/Von der äußersten Grenze aller praktischen Philosophie\\  
	
	\noindent\textbf{Paragraphe : }Hier ist nun die oberste Grenze aller moralischen Nachforschung; welche aber zu bestimmen auch schon darum von großer Wichtigkeit ist, damit die Vernunft nicht einerseits in der Sinnenwelt, auf eine den Sitten schädliche Art, nach der obersten Bewegursache und einem begreiflichen aber empirischen Interesse herumsuche, anderer Seits aber, damit sie auch nicht in dem für sie leeren Raum transzendenter Begriffe, unter dem Namen der intelligibelen Welt, kraftlos ihre \match{Flügel} schwinge, ohne von der Stelle zu kommen, und sich unter Hirngespinsten verliere, übrigens bleibt die Idee einer reinen Verstandeswelt, als eines Ganzen aller Intelligenzen, wozu wir selbst, als vernünftige Wesen (obgleich andererseits zugleich Glieder der Sinnenwelt) gehören, immer eine brauchbare und erlaubte Idee zum Behufe eines vernünftigen Glaubens, wenn gleich alles Wissen an der Grenze derselben ein Ende hat, um durch das  herrliche Ideal eines allgemeinen Reichs der Zwecke an sich selbst (vernünftiger Wesen), zu welchen wir nur alsdann als Glieder gehören können, wenn wir uns nach Maximen der Freiheit, als ob sie Gesetze der Natur wären, sorgfältig verhalten, ein lebhaftes Interesse an dem moralischen Gesetze in uns zu bewirken. 
	
	\unnumberedsection{Geschlecht (1)} 
	\subsection*{tg176.2.40} 
	\textbf{Source : }Grundlegung zur Metaphysik der Sitten/Zweiter Abschnitt: Übergang von der populären sittlichen Weltweisheit zur Metaphysik der Sitten\\  
	
	\noindent\textbf{Paragraphe : }Noch denkt ein vierter, dem es wohl geht, indessen er sieht, daß andere mit großen Mühseligkeiten zu kämpfen haben (denen er auch wohl helfen könnte): was geht's mich an? mag doch ein jeder so glücklich sein, als es der Himmel will, oder er sich selbst machen kann, ich werde ihm nichts entziehen, ja nicht einmal beneiden; nur zu seinem Wohlbefinden, oder seinem Beistande in der Not, habe ich nicht Lust, etwas beizutragen! Nun könnte allerdings, wenn eine solche Denkungsart ein allgemeines Naturgesetz würde, das menschliche \match{Geschlecht} gar wohl bestehen, und ohne Zweifel noch besser, als wenn jedermann Von Teilnehmung und Wohlwollen schwatzt, auch sich beeifert, gelegentlich dergleichen auszuüben, dagegen aber auch, wo er nur kann, betrügt, das Recht der Menschen verkauft, oder ihm sonst Abbruch tut. Aber, obgleich es möglich ist, daß nach jener Maxime ein allgemeines Naturgesetz wohl bestehen könnte: so ist es doch unmöglich, zu wollen, daß ein solches Prinzip als Naturgesetz allenthalben gelte. Denn ein Wille, der dieses beschlösse, würde sich selbst widerstreiten, indem der Fälle sich doch manche eräugnen können, wo er anderer Liebe und Teilnehmung bedarf, und wo er, durch ein solches aus seinem eigenen Willen entsprungenes Naturgesetz, sich selbst alle Hoffnung des Beistandes, den er sich wünscht, rauben würde. 
	
	\unnumberedsection{Heilige (1)} 
	\subsection*{tg176.2.7} 
	\textbf{Source : }Grundlegung zur Metaphysik der Sitten/Zweiter Abschnitt: Übergang von der populären sittlichen Weltweisheit zur Metaphysik der Sitten\\  
	
	\noindent\textbf{Paragraphe : }Man könnte auch der Sittlichkeit nicht übler raten, als wenn man sie von Beispielen entlehnen wollte. Denn jedes Beispiel, was mir davon vorgestellt wird, muß selbst zuvor nach Prinzipien der Moralität beurteilt werden, ob es auch würdig sei, zum ursprünglichen Beispiele, d.i. zum Muster zu dienen, keinesweges aber kann es den Begriff derselben zu oberst an die Hand geben. Selbst der \match{Heilige} des Evangelii muß zuvor mit unserm Ideal der sittlichen Vollkommenheit verglichen werden, ehe man ihn dafür erkennt; auch sagt er von sich selbst: was nennt ihr mich (den ihr sehet) gut, niemand ist gut (das Urbild des Guten) als der einige Gott (den ihr nicht sehet). Woher haben wir aber den Begriff von Gott, als dem höchsten Gut? Lediglich aus der Idee, die die Vernunft a priori von sittlicher Vollkommenheit entwirft, und mit dem Begriffe eines freien Willens unzertrennlich verknüpft. Nachahmung findet im Sittlichen gar nicht statt, und Beispiele dienen nur zur Aufmunterung, d.i. sie setzen die Tunlichkeit dessen, was das Gesetz gebietet, außer Zweifel, sie machen das, was die praktische Regel allgemeiner ausdrückt, anschaulich, können aber niemals  berechtigen, ihr wahres Original, das in der Vernunft liegt, bei Seite zu setzen und sich nach Beispielen zu richten. 
	
	\unnumberedsection{Klaße (1)} 
	\subsection*{tg179.2.4} 
	\textbf{Source : }Grundlegung zur Metaphysik der Sitten/Zweiter Abschnitt: Übergang von der populären sittlichen Weltweisheit zur Metaphysik der Sitten/Einteilung aller möglichen Prinzipien der Sittlichkeit aus dem angenommenen Grundbegriffe der Heteronomie\\  
	
	\noindent\textbf{Paragraphe : }
	Empirische Prinzipien taugen überall nicht dazu, um moralische Gesetze darauf zu gründen. Denn die Allgemeinheit, mit der sie für alle vernünftige Wesen ohne Unterschied gelten sollen, die unbedingte praktische Notwendigkeit, die ihnen dadurch auferlegt wird, fällt weg, wenn der Grund derselben von der besonderen Einrichtung der menschlichen Natur, oder den zufälligen Umständen hergenommen wird, darin sie gesetzt ist. Doch  ist das Prinzip der eigenen Glückseligkeit am meisten verwerflich, nicht bloß deswegen, weil es falsch ist, und die Erfahrung dem Vorgeben, als ob das Wohlbefinden sich jederzeit nach dem Wohlverhalten richte, widerspricht, auch nicht bloß, weil es gar nichts zur Gründung der Sittlichkeit beiträgt, indem es ganz was anderes ist, einen glücklichen, als einen guten Menschen, und diesen klug und auf seinen Vorteil abgewitzt, als ihn tugendhaft zu machen: sondern, weil es der Sittlichkeit Triebfedern unterlegt, die sie eher untergraben und ihre ganze Erhabenheit zernichten, indem sie die Bewegursachen zur Tugend mit denen zum Laster in eine \match{Klasse} stellen und nur den Kalkül besser ziehen lehren, den spezifischen Unterschied beider aber ganz und gar auslöschen: dagegen das moralische Gefühl, dieser vermeintliche besondere Sinn
	
	
	16
	(so seicht auch die Berufung auf selbigen ist, indem diejenigen, die nicht denken können, selbst in dem, was bloß auf allgemeine Gesetze ankommt, sich durchs Fühlen auszuhelfen glauben, so wenig auch Gefühle, die dem Grade nach von Natur unendlich von einander unterschieden sind, einen gleichen Maßstab des Guten und Bösen abgeben; auch einer durch sein Gefühl für andere gar nicht gültig urteilen kann), dennoch der Sittlichkeit und ihrer Würde dadurch näher bleibt, daß er der Tugend die Ehre beweist, das Wohlgefallen und die Hochschätzung für sie ihr unmittelbar zuzuschreiben, und ihr nicht gleichsam ins Gesicht sagt, daß es nicht ihre Schönheit, sondern nur der Vorteil sei, der uns an sie knüpfe. 
	
	\unnumberedsection{Ordnung (4)} 
	\subsection*{tg176.2.25} 
	\textbf{Source : }Grundlegung zur Metaphysik der Sitten/Zweiter Abschnitt: Übergang von der populären sittlichen Weltweisheit zur Metaphysik der Sitten\\  
	
	\noindent\textbf{Paragraphe : }Das Wollen nach diesen dreierlei Prinzipien wird auch durch die Ungleichheit der Nötigung des Willens deutlich unterschieden. Um diese nun auch merklich zu machen, glaube ich, daß man sie in ihrer \match{Ordnung} am angemessensten so benennen würde, wenn man sagte: sie wären entweder Regeln der Geschicklichkeit, oder Ratschläge der  Klugheit, oder Gebote (Gesetze) der Sittlichkeit. Denn nur das Gesetz führt den Begriff einer unbedingten und zwar objektiven und mithin allgemein gültigen Notwendigkeit bei sich, und Gebote sind Gesetze, denen gehorcht, d.i. auch wider Neigung Folge geleistet werden muß. Die Ratgebung enthält zwar Notwendigkeit, die aber bloß unter subjektiver gefälliger Bedingung, ob dieser oder jener Mensch dieses oder jenes zu seiner Glückseligkeit zähle, gelten kann; dagegen der kategorische Imperativ durch keine Bedingung eingeschränkt wird, und als absolut- obgleich praktisch-notwendig ganz eigentlich ein Gebot heißen kann. Man könnte die ersteren Imperative auch technisch (zur Kunst gehörig), die zweiten pragmatisch
	
	
	
	7
	(zur Wohlfahrt), die dritten moralisch (zum freien Verhalten überhaupt, d.i. zu den Sitten gehörig) nennen. 
	
	\subsection*{tg184.2.4} 
	\textbf{Source : }Grundlegung zur Metaphysik der Sitten/Dritter Abschnitt: Übergang von der Metaphysik der Sitten zur Kritik der reinen praktischen Vernunft/Wie ist ein kategorischer Imperativ möglich\\  
	
	\noindent\textbf{Paragraphe : }Der praktische Gebrauch der gemeinen Menschenvernunft bestätigt die Richtigkeit dieser Deduktion. Es ist niemand, selbst der ärgste Bösewicht, wenn er nur sonst Vernunft  zu brauchen gewohnt ist, der nicht, wenn man ihm Beispiele der Redlichkeit in Absichten, der Standhaftigkeit in Befolgung guter Maximen, der Teilnehmung und des allgemeinen Wohlwollens (und noch dazu mit großen Aufopferungen von Vorteilen und Gemächlichkeit verbunden) vorlegt, nicht wünsche, daß er auch so gesinnt sein möchte. Er kann es aber nur wegen seiner Neigungen und Antriebe nicht wohl in sich zu Stande bringen; wobei er dennoch zugleich wünscht, von solchen ihm selbst lästigen Neigungen frei zu sein. Er beweiset hiedurch also, daß er mit einem Willen, der von Antrieben der Sinnlichkeit frei ist, sich in Gedanken in eine ganz andere \match{Ordnung} der Dinge versetze, als die seiner Begierden im Felde der Sinnlichkeit, weil er von jenem Wunsche keine Vergnügung der Begierden, mithin keinen für irgend eine seiner wirklichen oder sonst erdenklichen Neigungen befriedigenden Zustand (denn dadurch würde selbst die Idee, welche ihm den Wunsch ablockt, ihre Vorzüglichkeit einbüßen), sondern nur einen größeren inneren Wert seiner Person erwarten kann. Diese bessere Person glaubt er aber zu sein, wenn er sich in den Standpunkt eines Gliedes der Verstandeswelt versetzt, dazu die Idee der Freiheit, d.i. Unabhängigkeit von bestimmenden Ursachen der Sinnenwelt ihn unwillkürlich nötigt, und in welchem er sich eines guten Willens bewußt ist, der für seinen bösen Willen, als Gliedes der Sinnenwelt, nach seinem eigenen Geständnisse das Gesetz ausmacht, dessen Ansehen er kennt, indem er es übertritt. Das moralische Sollen ist also eigenes notwendiges Wollen als Gliedes einer intelligibelen Welt, und wird nur so fern von ihm als Sollen gedacht, als er sich zugleich wie ein Glied der Sinnenwelt betrachtet. 
	
	\subsection*{tg185.2.7} 
	\textbf{Source : }Grundlegung zur Metaphysik der Sitten/Dritter Abschnitt: Übergang von der Metaphysik der Sitten zur Kritik der reinen praktischen Vernunft/Von der äußersten Grenze aller praktischen Philosophie\\  
	
	\noindent\textbf{Paragraphe : }
	Der Rechtsanspruch aber, selbst der gemeinen Menschenvernunft, auf Freiheit des Willens, gründet sich auf das Bewußtsein und die zugestandene Voraussetzung der Unabhängigkeit der Vernunft, von bloß subjektiv-bestimmten Ursachen, die insgesamt das ausmachen, was bloß zur Empfindung, mithin unter die allgemeine Benennung der Sinnlichkeit, gehört. Der Mensch, der sich auf solche Weise als Intelligenz betrachtet, setzt sich dadurch in eine andere \match{Ordnung} der Dinge und in ein Verhältnis zu bestimmenden Gründen von ganz anderer Art, wenn er sich als Intelligenz mit einem Willen, folglich mit Kausalität begabt, denkt, als wenn er sich wie Phänomen in der Sinnenwelt (welches er wirklich auch ist) wahrnimmt, und seine Kausalität, äußerer Bestimmung nach, Naturgesetzen unterwirft. Nun wird er bald inne, daß beides zugleich stattfinden könne, ja sogar müsse. Denn, daß ein Ding in der Erscheinung (das zur Sinnenwelt gehörig) gewissen Gesetzen unterworfen ist, von welchen eben dasselbe, als Ding oder Wesen an sich selbst, unabhängig ist, enthält nicht den mindesten Widerspruch; daß er sich selbst aber auf diese zwiefache Art vorstellen und denken müsse, beruht, was das erste betrifft, auf dem Bewußtsein seiner selbst als durch Sinne affizierten Gegenstandes, was das zweite anlangt, auf dem Bewußtsein seiner selbst als Intelligenz, d.i. als unabhängig im Vernunftgebrauch von sinnlichen Eindrücken (mithin als zur Verstandeswelt gehörig). 
	
	\subsection*{tg185.2.9} 
	\textbf{Source : }Grundlegung zur Metaphysik der Sitten/Dritter Abschnitt: Übergang von der Metaphysik der Sitten zur Kritik der reinen praktischen Vernunft/Von der äußersten Grenze aller praktischen Philosophie\\  
	
	\noindent\textbf{Paragraphe : }Dadurch, daß die praktische Vernunft sich in eine Verstandeswelt hinein denkt, überschreitet sie gar nicht ihre Grenzen, wohl aber, wenn sie sich hineinschauen, hineinempfinden wollte. Jenes ist nur ein negativer Gedanke, in Ansehung der Sinnenwelt, die der Vernunft in Bestimmung des Willens keine Gesetze gibt, und nur in diesem einzigen Punkte positiv, daß jene Freiheit, als negative Bestimmung, zugleich mit einem (positiven) Vermögen und sogar mit einer Kausalität der Vernunft verbunden sei, welche wir einen Willen nennen, so zu handeln, daß das Prinzip der Handlungen der wesentlichen Beschaffenheit einer Vernunftursache, d.i. der Bedingung der Allgemeingültigkeit der Maxime, als eines Gesetzes, gemäß sei. Würde sie aber noch ein Objekt des Willens, d.i. eine Bewegursache aus der Verstandeswelt herholen, so überschritte sie ihre Grenzen, und maßte sich an, etwas zu kennen, wovon sie nichts weiß. Der Begriff einer Verstandeswelt ist also nur ein Standpunkt, den die Vernunft sich genötigt sieht außer den Erscheinungen zu nehmen, um sich selbst als praktisch zu denken, welches, wenn die Einflüsse der Sinnlichkeit für den Menschen bestimmend wären, nicht möglich sein würde, welches aber doch notwendig ist, wofern ihm nicht das Bewußtsein seiner selbst, als Intelligenz, mithin als vernünftige und durch Vernunft tätige, d.i. frei wirkende Ursache, abgesprochen werden soll. Dieser Gedanke  führt freilich die Idee einer anderen \match{Ordnung} und Gesetzgebung, als die des Naturmechanismus, der die Sinnenwelt trifft, herbei, und macht den Begriff einer intelligibelen Welt (d.i. das Ganze vernünftiger Wesen, als Dinge an sich selbst) notwendig, aber ohne die mindeste Anmaßung, hier weiter, als bloß ihrer formalen Bedingung nach, d.i. der Allgemeinheit der Maxime des Willens, als Gesetze, mithin der Autonomie des letzteren, die allein mit der Freiheit desselben bestehen kann, gemäß zu denken; da hingegen alle Gesetze, die auf ein Objekt bestimmt sind, Heteronomie geben, die nur an Naturgesetzen angetroffen werden und auch nur die Sinnenwelt treffen kann. 
	
\end{document}