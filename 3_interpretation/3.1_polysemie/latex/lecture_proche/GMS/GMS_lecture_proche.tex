\documentclass[a4paper,12pt,twoside]{book}
\usepackage{fontspec}
\usepackage{xunicode}
\usepackage[german]{babel}


\usepackage{xcolor}

\newcommand{\match}[1]{\textcolor{red}{\textbf{#1}}}





\usepackage{fancyhdr}
\usepackage{etoolbox} % For defining robust commands

\pagestyle{fancy}

\fancyhead[LE]{\thepage}
\fancyhead[RE]{\leftmark}

\fancyhead[LO]{\rightmark}
\fancyhead[RO]{\thepage}

% Define a new command for unnumbered chapters
\newcommand{\unnumberedchapter}[1]{
	\chapter*{#1}
	\addcontentsline{toc}{chapter}{#1}
	\markboth{#1}{#1}
}

% Similarly, for sections
\newcommand{\unnumberedsection}[1]{
	\section*{#1}
	\addcontentsline{toc}{section}{#1}
	\markright{#1}
}

\usepackage[hidelinks]{hyperref} %
\usepackage[numbered]{bookmark}%va avec hyperref; marche mieux pour les signets. l'option numbered: les signets dans le pdf sont numérotés

\author{Eglantine Gaglione - M1 HN}
\title{GMS : vocabulaire thématique - mots "non-polysémiques"}

\begin{document}
	
	\maketitle
	
	\tableofcontents
	
	\unnumberedchapter{Agriculture} 
	\unnumberedsection{Fällen (4)} 
	\subsection*{tg174.2.7} 
	\textbf{Source : }Grundlegung zur Metaphysik der Sitten/Vorrede\\  
	
	\textbf{Paragraphe : }Alle Gewerbe, Handwerke und Künste, haben durch die Verteilung der Arbeiten gewonnen, da nämlich nicht einer alles macht, sondern jeder sich auf gewisse Arbeit, die sich, ihrer Behandlungsweise nach, von andern merklich unterscheidet, einschränkt, um sie in der größten Vollkommenheit und mit mehrerer Leichtigkeit leisten zu können. Wo die Arbeiten so nicht unterschieden und verteilt werden, wo jeder ein Tausendkünstler ist, da liegen die Gewerbe noch in der größten Barbarei. Aber ob dieses zwar für sich ein der Erwägung nicht unwürdiges Objekt wäre, zu fragen: ob die reine Philosophie in allen ihren Teilen nicht ihren besondern Mann erheische, und es um das Ganze des gelehrten Gewerbes nicht besser stehen würde, wenn die, so das Empirische mit dem Rationalen, dem Geschmacke des Publikums gemäß, nach allerlei ihnen selbst unbekannten Verhältnissen gemischt, zu verkaufen gewohnt sind, die sich Selbstdenker, andere aber, die den bloß rationalen Teil zubereiten, Grübler nennen, gewarnt würden, nicht zwei Geschäfte zugleich zu treiben, die in der Art, sie zu behandeln, gar sehr verschieden sind, zu deren jedem vielleicht ein besonderes Talent erfodert wird, und deren Verbindung in einer Person nur Stümper hervorbringt: so frage ich hier doch nur, ob nicht die Natur der Wissenschaft es erfodere, den empirischen von dem rationalen Teil jederzeit sorgfältig abzusondern, und vor der eigentlichen (empirischen) Physik eine Metaphysik der Natur, vor der praktischen Anthropologie aber eine Metaphysik der Sitten voranzuschicken, die von allem Empirischen sorgfältig gesäubert sein müßte, um zu  wissen, wie viel reine Vernunft in beiden \match{Fällen} leisten könne, und aus welchen Quellen sie selbst diese ihre Belehrung a priori schöpfe, es mag übrigens das letztere Geschäfte von allen Sittenlehrern (deren Name Legion heißt), oder nur von einigen, die Beruf dazu fühlen, getrieben werden. 
	
	\subsection*{tg174.2.9} 
	\textbf{Source : }Grundlegung zur Metaphysik der Sitten/Vorrede\\  
	
	\textbf{Paragraphe : }Also unterscheiden sich die moralischen Gesetze, samt ihren Prinzipien, unter allem praktischen Erkenntnisse von allem übrigen, darin irgend etwas Empirisches ist, nicht allein wesentlich, sondern alle Moralphilosophie beruht gänzlich auf ihrem reinen Teil, und, auf den Menschen angewandt, entlehnt sie nicht das mindeste von der Kenntnis desselben (Anthropologie), sondern gibt ihm, als vernünftigem Wesen, Gesetze a priori, die freilich noch durch Erfahrung geschärfte Urteilskraft erfodern, um teils zu unterscheiden,  in welchen \match{Fällen} sie ihre Anwendung haben, teils ihnen Eingang in den Willen des Menschen und Nachdruck zur Ausübung zu verschaffen, da diese, als selbst mit so viel Neigungen affiziert, der Idee einer praktischen reinen Vernunft zwar fähig, aber nicht so leicht vermögend ist, sie in seinem Lebenswandel in concreto wirksam zu machen. 
	
	\subsection*{tg175.2.22} 
	\textbf{Source : }Grundlegung zur Metaphysik der Sitten/Erster Abschnitt: Übergang von der gemeinen sittlichen Vernunfterkenntnis zur philosophischen\\  
	
	\textbf{Paragraphe : }So sind wir denn in der moralischen Erkenntnis der gemeinen Menschenvernunft bis zu ihrem Prinzip gelangt, welches sie sich zwar freilich nicht so in einer allgemeinen  Form abgesondert denkt, aber doch jederzeit wirklich vor Augen hat und zum Richtmaße ihrer Beurteilung braucht. Es wäre hier leicht zu zeigen, wie sie, mit diesem Kompasse in der Hand, in allen vorkommenden \match{Fällen} sehr gut Bescheid wisse, zu unterscheiden, was gut, was böse, pflichtmäßig, oder pflichtwidrig sei, wenn man, ohne sie im mindesten etwas Neues zu lehren, sie nur, wie Sokrates tat, auf ihr eigenes Prinzip aufmerksam macht, und daß es also keiner Wissenschaft und Philosophie bedürfe, um zu wissen, was man zu tun habe, um ehrlich und gut, ja sogar, um weise und tugendhaft zu sein. Das ließe sich auch wohl schon zum voraus vermuten, daß die Kenntnis dessen, was zu tun, mithin auch zu wissen jedem Menschen obliegt, auch jedes, selbst des gemeinsten Menschen Sache sein werde. Hier kann man es doch nicht ohne Bewunderung ansehen, wie das praktische Beurteilungsvermögen vor dem theoretischen im gemeinen Menschenverstande so gar viel voraus habe. In dem letzteren, wenn die gemeine Vernunft es wagt, von den Erfahrungsgesetzen und den Wahrnehmungen der Sinne abzugehen, gerät sie in lauter Unbegreiflichkeiten und Widersprüche mit sich selbst, wenigstens in ein Chaos von Ungewißheit, Dunkelheit und Unbestand. Im praktischen aber fängt die Beurteilungskraft denn eben allererst an, sich recht vorteilhaft zu zeigen, wenn der gemeine Verstand alle sinnliche Triebfedern von praktischen Gesetzen ausschließt. Er wird alsdenn so gar subtil, es mag sein, daß er mit seinem Gewissen, oder anderen Ansprüchen in Beziehung auf das, was recht heißen soll, schikanieren, oder auch den Wert der Handlungen zu seiner eigenen Belehrung aufrichtig bestimmen will, und, was das meiste ist, er kann im letzteren Falle sich eben so gut Hoffnung machen, es recht zu treffen, als es sich immer ein Philosoph versprechen mag, ja ist beinahe noch sicherer hierin, als selbst der letztere, weil dieser doch kein anderes Prinzip als jener haben, sein Urteil aber, durch eine Menge fremder, nicht zur Sache gehöriger Erwägungen, leicht verwirren und von der geraden Richtung abweichend machen kann. Wäre es demnach nicht ratsamer,  es in moralischen Dingen bei dem gemeinen Vernunfturteil bewenden zu lassen, und höchstens nur Philosophie anzubringen, um das System der Sitten desto vollständiger und faßlicher, imgleichen die Regeln derselben zum Gebrauche (noch mehr aber zum Disputieren) bequemer darzustellen, nicht aber, um selbst in praktischer Absicht den gemeinen Menschenverstand von seiner glücklichen Einfalt abzubringen, und ihn durch Philosophie auf einen neuen Weg der Untersuchung und Belehrung zu bringen? 
	
	\subsection*{tg176.2.27} 
	\textbf{Source : }Grundlegung zur Metaphysik der Sitten/Zweiter Abschnitt: Übergang von der populären sittlichen Weltweisheit zur Metaphysik der Sitten\\  
	
	\textbf{Paragraphe : }Die Imperativen der Klugheit würden, wenn es nur so leicht wäre, einen bestimmten Begriff von Glückseligkeit zu geben, mit denen der Geschicklichkeit ganz und gar übereinkommen und eben sowohl analytisch sein. Denn es würde eben sowohl hier, als dort, heißen: wer den Zweck will, will auch (der Vernunft gemäß notwendig) die einzigen Mittel, die dazu in seiner Gewalt sind. Allein es ist ein Unglück, daß der Begriff der Glückseligkeit ein so unbestimmter Begriff ist, daß, obgleich jeder Mensch zu dieser zu gelangen wünscht, er doch niemals bestimmt und mit sich selbst einstimmig sagen kann, was er eigentlich wünsche und wolle. Die Ursache davon ist: daß alle Elemente, die zum Begriff der Glückseligkeit gehören, insgesamt empirisch sind, d.i. aus der Erfahrung müssen entlehnt werden, daß gleichwohl zur Idee der Glückseligkeit ein absolutes Ganze, ein Maximum des Wohlbefindens, in meinem gegenwärtigen und jedem zukünftigen Zustande erforderlich ist. Nun ist's unmöglich, daß das einsehendste und zugleich allervermögendste, aber doch endliche Wesen sich einen bestimmten Begriff von dem mache, was er hier eigentlich wolle. Will er Reichtum, wie viel Sorge, Neid und Nachstellung könnte er sich dadurch nicht auf den Hals ziehen. Will er viel Erkenntnis und Einsicht, vielleicht könnte das ein nur um desto schärferes Auge werden, um die Übel, die sich für ihn jetzt noch verbergen und doch nicht vermieden werden können, ihm nur  um desto schrecklicher zu zeigen, oder seinen Begierden, die ihm schon genug zu schaffen machen, noch mehr Bedürfnisse aufzubürden. Will er ein langes Leben, wer steht ihm dafür, daß es nicht ein langes Elend sein würde? Will er wenigstens Gesundheit, wie oft hat noch Ungemächlichkeit des Körpers von Ausschweifung abgehalten, darein unbeschränkte Gesundheit würde haben fallen lassen, u.s.w. Kurz, er ist nicht vermögend, nach irgend einem Grundsatze, mit völliger Gewißheit zu bestimmen, was ihn wahrhaftig glücklich machen werde, darum, weil hiezu Allwissenheit erforderlich sein würde. Man kann also nicht nach bestimmten Prinzipien handeln, um glücklich zu sein, sondern nur nach empirischen Ratschlägen, z.B. der Diät, der Sparsamkeit, der Höflichkeit, der Zurückhaltung u.s.w., von welchen die Erfahrung lehrt, daß sie das Wohlbefinden im Durchschnitt am meisten befördern. Hieraus folgt, daß die Imperativen der Klugheit, genau zu reden, gar nicht gebieten, d.i. Handlungen objektiv als praktisch-notwendig darstellen können, daß sie eher für Anratungen (consilia) als Gebote (praecepta) der Vernunft zu halten sind, daß die Aufgabe: sicher und allgemein zu bestimmen, welche Handlung die Glückseligkeit eines vernünftigen Wesens befördern werde, völlig unauflöslich, mithin kein Imperativ in Ansehung derselben möglich sei, der im strengen Verstande geböte, das zu tun, was glücklich macht, weil Glückseligkeit nicht ein Ideal der Vernunft, sondern der Einbildungskraft ist, was bloß auf empirischen Gründen beruht, von denen man vergeblich erwartet, daß sie eine Handlung bestimmen sollten, dadurch die Totalität einer in der Tat unendlichen Reihe von Folgen erreicht würde. Dieser Imperativ der Klugheit würde indessen, wenn man annimmt, die Mittel zur Glückseligkeit ließen sich sicher angeben, ein analytisch-praktischer Satz sein; denn er ist von dem Imperativ der Geschicklichkeit nur darin unterschieden, daß bei diesem der Zweck bloß möglich, bei jenem aber gegeben ist; da beide aber bloß die Mittel zu demjenigen gebieten, von dem man voraussetzt, daß man es als Zweck wollte: so ist der  Imperativ, der das Wollen der Mittel für den, der den Zweck will, gebietet, in beiden \match{Fällen} analytisch. Es ist also in Ansehung der Möglichkeit eines solchen Imperativs auch keine Schwierigkeit. 
	
	\unnumberedsection{Vermehrung (1)} 
	\subsection*{tg176.2.80} 
	\textbf{Source : }Grundlegung zur Metaphysik der Sitten/Zweiter Abschnitt: Übergang von der populären sittlichen Weltweisheit zur Metaphysik der Sitten\\  
	
	\textbf{Paragraphe : }Nun folgt hieraus unstreitig: daß jedes vernünftige Wesen, als Zweck an sich selbst, sich in Ansehung aller Gesetze, denen es nur immer unterworfen sein mag, zugleich als allgemein gesetzgebend müsse ansehen können, weil eben diese Schicklichkeit seiner Maximen zur allgemeinen Gesetzgebung es als Zweck an sich selbst auszeichnet, imgleichen, daß dieses seine Würde (Prärogativ) vor allen bloßen Naturwesen es mit sich bringe, seine Maximen jederzeit aus dem Gesichtspunkte seiner selbst, zugleich aber auch jedes andern vernünftigen als gesetzgebenden Wesens (die darum auch Personen heißen), nehmen zu müssen. Nun ist auf solche Weise eine Welt vernünftiger Wesen (mundus intelligibilis) als ein Reich der Zwecke möglich, und zwar durch die eigene Gesetzgebung aller Personen als Glieder. Demnach muß ein jedes vernünftige Wesen so handeln, als ob es durch seine Maximen jederzeit ein gesetzgebendes Glied im allgemeinen Reiche der Zwecke wäre. Das formale Prinzip dieser Maximen ist: handle so, als ob deine Maxime zugleich zum allgemeinen Gesetze (aller vernünftigen Wesen) dienen sollte. Ein Reich der Zwecke ist also nur möglich nach der Analogie mit einem Reiche der Natur, jenes aber nur nach Maximen, d.i. sich selbst auferlegten Regeln, diese nur nach Gesetzen äußerlich genötigter wirkenden Ursachen. Demunerachtet gibt man doch auch dem Naturganzen, ob es schon als Maschine angesehen wird, dennoch, so fern es auf vernünftige Wesen, als seine Zwecke, Beziehung hat, aus diesem Grunde den Namen eines Reichs der Natur. Ein solches Reich der Zwecke würde nun durch Maximen, deren Regel der kategorische Imperativ aller vernünftigen Wesen vorschreibt, wirklich zu Stande kommen, wenn sie allgemein befolgt würden. Allein, obgleich das vernünftige Wesen darauf nicht rechnen kann, daß, wenn es auch gleich  diese Maxime selbst pünktlich befolgte, darum jedes andere eben derselben treu sein würde, imgleichen, daß das Reich der Natur und die zweckmäßige Anordnung desselben, mit ihm, als einem schicklichen Gliede, zu einem durch ihn selbst möglichen Reiche der Zwecke zusammenstimmen, d.i. seine Erwartung der Glückseligkeit begünstigen werde: so bleibt doch jenes Gesetz: handle nach Maximen eines allgemein gesetzgebenden Gliedes zu einem bloß möglichen Reiche der Zwecke, in seiner vollen Kraft, weil es kategorisch gebietend ist. Und hierin liegt eben das Paradoxon; daß bloß die Würde der Menschheit, als vernünftiger Natur, ohne irgend einen andern dadurch zu erreichenden Zweck, oder Vorteil, mithin die Achtung für eine bloße Idee, dennoch zur unnachlaßlichen Vorschrift des Willens dienen sollte, und daß gerade in dieser Unabhängigkeit der Maxime von allen solchen Triebfedern die Erhabenheit derselben bestehe, und die Würdigkeit eines jeden vernünftigen Subjekts, ein gesetzgebendes Glied im Reiche der Zwecke zu sein; denn sonst würde es nur als dem Naturgesetze seiner Bedürfnis unterworfen vorgestellt werden müssen. Obgleich auch das Naturreich sowohl, als das Reich der Zwecke, als unter einem Oberhaupte vereinigt gedacht würde, und dadurch das letztere nicht mehr bloße Idee bliebe, sondern wahre Realität erhielte, so würde hiedurch zwar jener der Zuwachs einer starken Triebfeder, niemals aber \match{Vermehrung} ihres innern Werts zu statten kommen; denn, diesem ungeachtet, müßte doch selbst dieser alleinige unumschränkte Gesetzgeber immer so vorgestellt werden, wie er den Wert der vernünftigen Wesen, nur nach ihrem uneigennützigen, bloß aus jener Idee ihnen selbst vorgeschriebenen Verhalten, beurteilte. Das Wesen der Dinge ändert sich durch ihre äußere Verhältnisse nicht, und was, ohne an das letztere zu denken, den absoluten Wert des Menschen allein ausmacht, darnach muß er auch, von wem es auch sei, selbst vom höchsten Wesen, beurteilt werden. Moralität ist also das Verhältnis der Handlungen zur Autonomie des Willens, das ist, zur möglichen allgemeinen Gesetzgebung durch die  Maximen desselben. Die Handlung, die mit der Autonomie des Willens zusammen bestehen kann, ist erlaubt; die nicht damit stimmt, ist unerlaubt. Der Wille, dessen Maximen notwendig mit den Gesetzen der Autonomie zusammenstimmen, ist ein heiliger, schlechterdings guter Wille. Die Abhängigkeit eines nicht schlechterdings guten Willens vom Prinzip der Autonomie (die moralische Nötigung) ist Verbindlichkeit. Diese kann also auf ein heiliges Wesen nicht gezogen werden. Die objektive Notwendigkeit einer Handlung aus Verbindlichkeit heißt Pflicht. 
	
	\unnumberedchapter{Monde} 
	\unnumberedsection{Dunkelheit (1)} 
	\subsection*{tg175.2.22} 
	\textbf{Source : }Grundlegung zur Metaphysik der Sitten/Erster Abschnitt: Übergang von der gemeinen sittlichen Vernunfterkenntnis zur philosophischen\\  
	
	\textbf{Paragraphe : }So sind wir denn in der moralischen Erkenntnis der gemeinen Menschenvernunft bis zu ihrem Prinzip gelangt, welches sie sich zwar freilich nicht so in einer allgemeinen  Form abgesondert denkt, aber doch jederzeit wirklich vor Augen hat und zum Richtmaße ihrer Beurteilung braucht. Es wäre hier leicht zu zeigen, wie sie, mit diesem Kompasse in der Hand, in allen vorkommenden Fällen sehr gut Bescheid wisse, zu unterscheiden, was gut, was böse, pflichtmäßig, oder pflichtwidrig sei, wenn man, ohne sie im mindesten etwas Neues zu lehren, sie nur, wie Sokrates tat, auf ihr eigenes Prinzip aufmerksam macht, und daß es also keiner Wissenschaft und Philosophie bedürfe, um zu wissen, was man zu tun habe, um ehrlich und gut, ja sogar, um weise und tugendhaft zu sein. Das ließe sich auch wohl schon zum voraus vermuten, daß die Kenntnis dessen, was zu tun, mithin auch zu wissen jedem Menschen obliegt, auch jedes, selbst des gemeinsten Menschen Sache sein werde. Hier kann man es doch nicht ohne Bewunderung ansehen, wie das praktische Beurteilungsvermögen vor dem theoretischen im gemeinen Menschenverstande so gar viel voraus habe. In dem letzteren, wenn die gemeine Vernunft es wagt, von den Erfahrungsgesetzen und den Wahrnehmungen der Sinne abzugehen, gerät sie in lauter Unbegreiflichkeiten und Widersprüche mit sich selbst, wenigstens in ein Chaos von Ungewißheit, \match{Dunkelheit} und Unbestand. Im praktischen aber fängt die Beurteilungskraft denn eben allererst an, sich recht vorteilhaft zu zeigen, wenn der gemeine Verstand alle sinnliche Triebfedern von praktischen Gesetzen ausschließt. Er wird alsdenn so gar subtil, es mag sein, daß er mit seinem Gewissen, oder anderen Ansprüchen in Beziehung auf das, was recht heißen soll, schikanieren, oder auch den Wert der Handlungen zu seiner eigenen Belehrung aufrichtig bestimmen will, und, was das meiste ist, er kann im letzteren Falle sich eben so gut Hoffnung machen, es recht zu treffen, als es sich immer ein Philosoph versprechen mag, ja ist beinahe noch sicherer hierin, als selbst der letztere, weil dieser doch kein anderes Prinzip als jener haben, sein Urteil aber, durch eine Menge fremder, nicht zur Sache gehöriger Erwägungen, leicht verwirren und von der geraden Richtung abweichend machen kann. Wäre es demnach nicht ratsamer,  es in moralischen Dingen bei dem gemeinen Vernunfturteil bewenden zu lassen, und höchstens nur Philosophie anzubringen, um das System der Sitten desto vollständiger und faßlicher, imgleichen die Regeln derselben zum Gebrauche (noch mehr aber zum Disputieren) bequemer darzustellen, nicht aber, um selbst in praktischer Absicht den gemeinen Menschenverstand von seiner glücklichen Einfalt abzubringen, und ihn durch Philosophie auf einen neuen Weg der Untersuchung und Belehrung zu bringen? 
	
	\unnumberedsection{Gebiet (1)} 
	\subsection*{tg176.2.47} 
	\textbf{Source : }Grundlegung zur Metaphysik der Sitten/Zweiter Abschnitt: Übergang von der populären sittlichen Weltweisheit zur Metaphysik der Sitten\\  
	
	\textbf{Paragraphe : }Die Frage ist also diese: ist es ein notwendiges Gesetz für alle vernünftige Wesen, ihre Handlungen jederzeit nach solchen Maximen zu beurteilen, von denen sie selbst wollen können, daß sie zu allgemeinen Gesetzen dienen sollen? Wenn es ein solches ist, so muß es (völlig a priori) schon mit dem Begriffe des Willens eines vernünftigen Wesens überhaupt verbunden sein. Um aber diese Verknüpfung zu entdecken, muß man, so sehr man sich auch sträubt, einen Schritt hinaus tun, nämlich zur Metaphysik, obgleich in ein \match{Gebiet} derselben, welches von dem der spekulativen Philosophie unterschieden ist, nämlich in die Metaphysik der Sitten. In einer praktischen Philosophie, wo es uns nicht darum zu tun ist, Gründe anzunehmen, von dem, was geschieht, sondern Gesetze von dem, was geschehen soll, ob es gleich niemals geschieht, d.i. objektiv-praktische Gesetze: da haben wir nicht nötig, über die Gründe Untersuchung anzustellen, warum etwas gefällt oder mißfällt, wie das Vergnügen der bloßen Empfindung vom Geschmacke, und ob dieser von einem allgemeinen Wohlgefallen der Vernunft unterschieden sei; worauf Gefühl der Lust und Unlust beruhe, und wie hieraus Begierden und Neigungen, aus diesen aber, durch Mitwirkung der Vernunft, Maximen entspringen; denn das gehört alles zu einer empirischen Seelenlehre, welche den zweiten Teil der Naturlehre ausmachen würde, wenn man sie als Philosophie der Natur betrachtet, so fern sie auf empirischen Gesetzen gegründet ist. Hier aber ist vom objektiv-praktischen Gesetze die Rede, mithin von dem Verhältnisse eines Willens zu sich selbst, so fern er sich bloß durch Vernunft bestimmt, da denn alles, was aufs  Empirische Beziehung hat, von selbst wegfällt; weil, wenn die Vernunft für sich allein das Verhalten bestimmt (wovon wir die Möglichkeit jetzt eben untersuchen wollen), sie dieses notwendig a priori tun muß. 
	
	\unnumberedsection{Wolke (1)} 
	\subsection*{tg176.2.46} 
	\textbf{Source : }Grundlegung zur Metaphysik der Sitten/Zweiter Abschnitt: Übergang von der populären sittlichen Weltweisheit zur Metaphysik der Sitten\\  
	
	\textbf{Paragraphe : }Alles also, was empirisch ist, ist, als Zutat zum Prinzip der Sittlichkeit, nicht allein dazu ganz untauglich, sondern der Lauterkeit der Sitten selbst höchst nachteilig, an welchen der eigentliche und über allen Preis erhabene Wert eines schlechterdings guten Willens eben darin besteht, daß das Prinzip der Handlung von allen Einflüssen zufälliger Gründe, die nur Erfahrung an die Hand geben kann, frei sei. Wider diese Nachlässigkeit oder gar niedrige Denkungsart, in Aufsuchung des Prinzips unter empirischen Bewegursachen und Gesetzen, kann man auch nicht zu viel und zu oft Warnungen ergehen lassen, indem die menschliche Vernunft in ihrer Ermüdung gern auf diesem Polster ausruht, und in dem Traume süßer Vorspiegelungen (die sie doch statt der Juno eine \match{Wolke} umarmen lassen) der Sittlichkeit einen aus Gliedern ganz verschiedener Abstammung zusammengeflickten Bastard unterschiebt, der allem ähnlich sieht, was man daran sehen will, nur der Tugend  nicht, für den, der sie einmal in ihrer wahren Gestalt erblickt hat.
	
	
	11
	
	
	
	\unnumberedchapter{Sciencesexactes} 
	\unnumberedsection{Aufschub (1)} 
	\subsection*{tg179.2.7} 
	\textbf{Source : }Grundlegung zur Metaphysik der Sitten/Zweiter Abschnitt: Übergang von der populären sittlichen Weltweisheit zur Metaphysik der Sitten/Einteilung aller möglichen Prinzipien der Sittlichkeit aus dem angenommenen Grundbegriffe der Heteronomie\\  
	
	\textbf{Paragraphe : }Übrigens glaube ich einer weitläuftigen Widerlegung aller dieser Lehrbegriffe überhoben sein zu können. Sie ist so leicht, sie ist von denen selbst, deren Amt es erfodert, sich doch für eine dieser Theorien zu erklären (weil Zuhörer den \match{Aufschub} des Urteils nicht wohl leiden mögen), selbst vermutlich so wohl eingesehen, daß dadurch nur überflüssige Arbeit geschehen würde. Was uns aber hier mehr interessiert, ist, zu wissen: daß diese Prinzipien überall nichts als Heteronomie des Willens zum ersten Grunde der Sittlichkeit  aufstellen, und eben darum notwendig ihres Zwecks verfehlen müssen. 
	
	\unnumberedsection{Grad (2)} 
	\subsection*{tg175.2.8} 
	\textbf{Source : }Grundlegung zur Metaphysik der Sitten/Erster Abschnitt: Übergang von der gemeinen sittlichen Vernunfterkenntnis zur philosophischen\\  
	
	\textbf{Paragraphe : }In der Tat finden wir auch, daß, je mehr eine kultivierte Vernunft sich mit der Absicht auf den Genuß des Lebens  und der Glückseligkeit abgibt, desto weiter der Mensch von der wahren Zufriedenheit abkomme, woraus bei vielen, und zwar den Versuchtesten im Gebrauche derselben, wenn sie nur aufrichtig genug sind, es zu gestehen, ein gewisser \match{Grad} von Misologie, d.i. Haß der Vernunft entspringt, weil sie nach dem Überschlage alles Vorteils, den sie, ich will nicht sagen von der Erfindung aller Künste des gemeinen Luxus, sondern so gar von den Wissenschaften (die ihnen am Ende auch ein Luxus des Verstandes zu sein scheinen) ziehen, dennoch finden, daß sie sich in der Tat nur mehr Mühseligkeit auf den Hals gezogen, als an Glückseligkeit gewonnen haben, und darüber endlich den gemeinern Schlag der Menschen, welcher der Leitung des bloßen Naturinstinkts näher ist, und der seiner Vernunft nicht viel Einfluß auf sein Tun und Lassen verstattet, eher beneiden, als geringschätzen. Und so weit muß man gestehen, daß das Urteil derer, die die ruhmredige Hochpreisungen der Vorteile, die uns die Vernunft in Ansehung der Glückseligkeit und Zufriedenheit des Lebens verschaffen sollte, sehr mäßigen und sogar unter Null herabsetzen, keinesweges grämisch, oder gegen die Güte der Weltregierung undankbar sei, sondern daß diesen Urteilen ingeheim die Idee von einer andern und viel würdigern Absicht ihrer Existenz zum Grunde liege, zu welcher, und nicht der Glückseligkeit, die Vernunft ganz eigentlich bestimmt sei, und welcher darum, als oberster Bedingung, die Privatabsicht des Menschen größtenteils nachstehen muß. 
	
	\subsection*{tg183.2.9} 
	\textbf{Source : }Grundlegung zur Metaphysik der Sitten/Dritter Abschnitt: Übergang von der Metaphysik der Sitten zur Kritik der reinen praktischen Vernunft/Von dem Interesse, welches den Ideen der Sittlichkeit anhängt\\  
	
	\textbf{Paragraphe : }Dergleichen Schluß muß der nachdenkende Mensch von allen Dingen, die ihm vorkommen mögen, fällen; vermutlich ist er auch im gemeinsten Verstande anzutreffen, der, wie bekannt, sehr geneigt ist, hinter den Gegenständen der Sinne noch immer etwas Unsichtbares, für sich selbst Tätiges, zu erwarten, es aber wiederum dadurch verdirbt, daß er dieses Unsichtbare sich bald wiederum versinnlicht, d.i. zum Gegenstande der Anschauung machen will, und dadurch also nicht um einen \match{Grad} klüger wird. 
	
	\unnumberedsection{Mathematik (1)} 
	\subsection*{tg187.2.6} 
	\textbf{Source : }Grundlegung zur Metaphysik der Sitten/Fußnoten\\  
	
	\textbf{Paragraphe : }
	
	3 Man kann, wenn man will, (so wie die reine \match{Mathematik} von der angewandten, die reine Logik von der angewandten unterschieden wird, also) die reine Philosophie der Sitten (Metaphysik) von der angewandten (nämlich auf die menschliche Natur) unterscheiden. Durch diese Benennung wird man auch so fort erinnert, daß die sittlichen Prinzipien nicht auf die Eigenheiten der menschlichen Natur gegründet, sondern für sich a priori bestehend sein müssen, aus solchen aber, wie für jede vernünftige Natur, also auch für die menschliche, praktische Regeln müssen abgeleitet werden können. 
	
	\unnumberedsection{Physik (1)} 
	\subsection*{tg174.2.6} 
	\textbf{Source : }Grundlegung zur Metaphysik der Sitten/Vorrede\\  
	
	\textbf{Paragraphe : }Auf solche Weise entspringt die Idee einer zwiefachen Metaphysik, einer Metaphysik der Natur und einer Metaphysik der Sitten. Die \match{Physik} wird also ihren empirischen, aber auch einen rationalen Teil haben; die Ethik gleichfalls; wiewohl hier der empirische Teil besonders praktische Anthropologie, der rationale aber eigentlich Moral heißen könnte. 
	
	\unnumberedsection{Schätzung (4)} 
	\subsection*{tg175.2.10} 
	\textbf{Source : }Grundlegung zur Metaphysik der Sitten/Erster Abschnitt: Übergang von der gemeinen sittlichen Vernunfterkenntnis zur philosophischen\\  
	
	\textbf{Paragraphe : }Um aber den Begriff eines an sich selbst hochzuschätzenden und ohne weitere Absicht guten Willens, so wie er schon dem natürlichen gesunden Verstande beiwohnet und nicht so wohl gelehret als vielmehr nur aufgeklärt zu werden bedarf, diesen Begriff, der in der \match{Schätzung} des ganzen Werts unserer Handlungen immer obenan steht und die Bedingung alles übrigen ausmacht, zu entwickeln: wollen wir den Begriff der Pflicht vor uns nehmen, der den eines guten Willens, obzwar unter gewissen subjektiven Einschränkungen und Hindernissen, enthält, die aber doch, weit gefehlt, daß sie ihn verstecken und unkenntlich machen sollten, ihn vielmehr durch Abstechung heben und desto heller hervorscheinen lassen. 
	
	\subsection*{tg175.2.21} 
	\textbf{Source : }Grundlegung zur Metaphysik der Sitten/Erster Abschnitt: Übergang von der gemeinen sittlichen Vernunfterkenntnis zur philosophischen\\  
	
	\textbf{Paragraphe : }Was ich also zu tun habe, damit mein Wollen sittlich gut sei, darzu brauche ich gar keine weit ausholende Scharfsinnigkeit. Unerfahren in Ansehung des Weltlaufs, unfähig, auf alle sich eräugnende Vorfälle desselben gefaßt zu sein, frage ich mich nur: Kannst du auch wollen, daß deine Maxime ein allgemeines Gesetz werde? wo nicht, so ist sie verwerflich, und das zwar nicht um eines dir, oder auch anderen, daraus bevorstehenden Nachteils willen, sondern weil sie nicht als Prinzip in eine mögliche allgemeine Gesetzgebung passen kann, für diese aber zwingt mir die Vernunft unmittelbare Achtung ab, von der ich zwar jetzt noch nicht einsehe, worauf sie sich gründe (welches der Philosoph untersuchen mag), wenigstens aber doch so viel verstehe: daß es eine \match{Schätzung} des Wertes sei, welcher allen Wert dessen, was durch Neigung angepriesen wird, weit überwiegt, und daß die Notwendigkeit meiner Handlungen aus reiner Achtung fürs praktische Gesetz dasjenige sei, was die Pflicht ausmacht, der jeder andere Bewegungsgrund weichen muß, weil sie die Bedingung eines an sich guten Willens ist, dessen Wert über alles geht. 
	
	\subsection*{tg175.2.6} 
	\textbf{Source : }Grundlegung zur Metaphysik der Sitten/Erster Abschnitt: Übergang von der gemeinen sittlichen Vernunfterkenntnis zur philosophischen\\  
	
	\textbf{Paragraphe : }Es liegt gleichwohl in dieser Idee von dem absoluten Werte des bloßen Willens, ohne einigen Nutzen bei \match{Schätzung} desselben in Anschlag zu bringen, etwas so Befremdliches, daß, unerachtet aller Einstimmung selbst der gemeinen  Vernunft mit derselben, dennoch ein Verdacht entspringen muß, daß vielleicht bloß hochfliegende Phantasterei ingeheim zum Grunde liege, und die Natur in ihrer Absicht, warum sie unserm Willen Vernunft zur Regiererin beigelegt habe, falsch verstanden sein möge. Daher wollen wir diese Idee aus diesem Gesichtspunkte auf die Prüfung stellen. 
	
	\subsection*{tg176.2.73} 
	\textbf{Source : }Grundlegung zur Metaphysik der Sitten/Zweiter Abschnitt: Übergang von der populären sittlichen Weltweisheit zur Metaphysik der Sitten\\  
	
	\textbf{Paragraphe : }Und was ist es denn nun, was die sittlich gute Gesinnung oder die Tugend berechtigt, so hohe Ansprüche zu machen? Es ist nichts Geringeres als der Anteil, den sie dem vernünftigen Wesen an der allgemeinen Gesetzgebung verschafft, und es hiedurch zum Gliede in einem möglichen Reiche der Zwecke tauglich macht, wozu es durch seine eigene Natur schon bestimmt war, als Zweck an sich selbst und eben darum als gesetzgebend im Reiche der Zwecke, in Ansehung aller Naturgesetze als frei, nur denjenigen allein gehorchend, die es selbst gibt und nach welchen seine Maximen zu einer allgemeinen Gesetzgebung (der er sich zugleich selbst unterwirft) gehören können. Denn es hat nichts einen Wert, als den, welchen ihm das Gesetz bestimmt. Die Gesetzgebung selbst aber, die allen Wert bestimmt, muß eben darum eine Würde, d.i. unbedingten, unvergleichbaren Wert haben, für welchen das Wort Achtung allein den geziemenden Ausdruck der \match{Schätzung} abgibt, die ein vernünftiges Wesen über sie anzustellen hat. Autonomie ist also der Grund der Würde der menschlichen und jeder vernünftigen Natur. 
	
	\unnumberedsection{Seiten (1)} 
	\subsection*{tg183.2.11} 
	\textbf{Source : }Grundlegung zur Metaphysik der Sitten/Dritter Abschnitt: Übergang von der Metaphysik der Sitten zur Kritik der reinen praktischen Vernunft/Von dem Interesse, welches den Ideen der Sittlichkeit anhängt\\  
	
	\textbf{Paragraphe : }Um deswillen muß ein vernünftiges Wesen sich selbst, als Intelligenz (also nicht von \match{Seiten} seiner untern Kräfte), nicht als zur Sinnen-, sondern zur Verstandeswelt gehörig, ansehen; mithin hat es zwei Standpunkte, daraus es sich selbst betrachten, und Gesetze des Gebrauchs seiner Kräfte, folglich aller seiner Handlungen, erkennen kann, einmal, so fern es zur Sinnenwelt gehört, unter Naturgesetzen (Heteronomie), zweitens, als zur intelligibelen Welt gehörig, unter Gesetzen, die, von der Natur unabhängig, nicht empirisch, sondern bloß in der Vernunft gegründet sein. 
	
	\unnumberedsection{Summe (3)} 
	\subsection*{tg175.2.14} 
	\textbf{Source : }Grundlegung zur Metaphysik der Sitten/Erster Abschnitt: Übergang von der gemeinen sittlichen Vernunfterkenntnis zur philosophischen\\  
	
	\textbf{Paragraphe : }Seine eigene Glückseligkeit sichern, ist Pflicht (wenigstens indirekt), denn der Mangel der Zufriedenheit mit seinem Zustande, in einem Gedränge von vielen Sorgen und mitten unter unbefriedigten Bedürfnissen, könnte leicht eine große Versuchung zu Übertretung der Pflichten werden. Aber, auch ohne hier auf Pflicht zu sehen, haben alle Menschen schon von selbst die mächtigste und innigste Neigung zur Glückseligkeit, weil sich gerade in dieser Idee alle Neigungen zu einer \match{Summe} vereinigen. Nur ist die Vorschrift der Glückseligkeit mehrenteils so beschaffen, daß sie einigen Neigungen großen Abbruch tut und doch der Mensch sich von der Summe der Befriedigung aller unter dem Namen der Glückseligkeit keinen bestimmten und sichern Begriff machen kann; daher nicht zu verwundern ist, wie eine einzige, in Ansehung dessen, was sie verheißt, und der Zeit, worin ihre Befriedigung erhalten werden kann, bestimmte Neigung eine schwankende Idee überwiegen könne, und der Mensch, z.B. ein Podagrist wählen könne, zu genießen was ihm schmeckt und zu leiden was er kann, weil er, nach seinem Überschlage, hier wenigstens, sich nicht durch vielleicht grundlose Erwartungen eines Glücks, das in der Gesundheit stecken soll, um den Genuß des gegenwärtigen Augenblicks gebracht hat. Aber auch in diesem Falle, wenn die allgemeine Neigung zur Glückseligkeit seinen Willen nicht bestimmte, wenn Gesundheit für ihn wenigstens nicht so notwendig in diesen Überschlag gehörete, so bleibt noch hier, wie in allen andern Fällen, ein Gesetz übrig, nämlich seine Glückseligkeit zu befördern, nicht aus Neigung, sondern aus Pflicht, und da hat sein Verhalten allererst den eigentlichen moralischen Wert. 
	
	\subsection*{tg175.2.5} 
	\textbf{Source : }Grundlegung zur Metaphysik der Sitten/Erster Abschnitt: Übergang von der gemeinen sittlichen Vernunfterkenntnis zur philosophischen\\  
	
	\textbf{Paragraphe : }Der gute Wille ist nicht durch das, was er bewirkt, oder ausrichtet, nicht durch seine Tauglichkeit zu Erreichung irgend eines vorgesetzten Zweckes, sondern allein durch das Wollen, d.i. an sich, gut, und, für sich selbst betrachtet, ohne Vergleich weit höher zu schätzen, als alles, was durch ihn zu Gunsten irgend einer Neigung, ja, wenn man will, der \match{Summe} aller Neigungen, nur immer zu Stande gebracht werden könnte. Wenn gleich durch eine besondere Ungunst des Schicksals, oder durch kärgliche Ausstattung einer stiefmütterlichen Natur, es diesem Willen gänzlich an Vermögen fehlete, seine Absicht durchzusetzen; wenn bei seiner größten Bestrebung dennoch nichts von ihm ausgerichtet würde, und nur der gute Wille (freilich nicht etwa ein bloßer Wunsch, sondern als die Aufbietung aller Mittel, so weit sie in unserer Gewalt sind) übrig bliebe: so würde er wie ein Juwel doch für sich selbst glänzen, als etwas, das seinen vollen Wert in sich selbst hat. Die Nützlichkeit oder Fruchtlosigkeit kann diesem Werte weder etwas zusetzen, noch abnehmen. Sie würde gleichsam nur die Einfassung sein, um ihn im gemeinen Verkehr besser handhaben zu können, oder die Aufmerksamkeit derer, die noch nicht gnug Kenner sind, auf sich zu ziehen, nicht aber, um ihn Kennern zu empfehlen, und seinen Wert zu bestimmen. 
	
	\subsection*{tg179.2.5} 
	\textbf{Source : }Grundlegung zur Metaphysik der Sitten/Zweiter Abschnitt: Übergang von der populären sittlichen Weltweisheit zur Metaphysik der Sitten/Einteilung aller möglichen Prinzipien der Sittlichkeit aus dem angenommenen Grundbegriffe der Heteronomie\\  
	
	\textbf{Paragraphe : }Unter den rationalen, oder Vernunftgründen der Sittlichkeit ist doch der ontologische Begriff der Vollkommenheit (so leer, so unbestimmt, mithin unbrauchbar er auch ist, um in dem unermeßlichen Felde möglicher Realität  die für uns schickliche größte \match{Summe} auszufinden, so sehr er auch, um die Realität, von der hier die Rede ist, spezifisch von jeder anderen zu unterscheiden, einen unvermeidlichen Hang hat, sich im Zirkel zu drehen, und die Sittlichkeit, die er erklären soll, ingeheim vorauszusetzen nicht vermeiden kann) dennoch besser als der theologische Begriff, sie von einem göttlichen allervollkommensten Willen abzuleiten, nicht bloß deswegen, weil wir seine Vollkommenheit doch nicht anschauen, sondern sie von unseren Begriffen, unter denen der der Sittlichkeit der vornehmste ist, allein ableiten können, sondern weil, wenn wir dieses nicht tun (wie es denn, wenn es geschähe, ein grober Zirkel im Erklären sein würde), der uns noch übrige Begriff seines Willens aus den Eigenschaften der Ehr- und Herrschbegierde, mit den furchtbaren Vorstellungen der Macht und des Racheifers verbunden, zu einem System der Sitten, welches der Moralität gerade entgegen gesetzt wäre, die Grundlage machen müßte. 
	
	\unnumberedsection{Umfang (1)} 
	\subsection*{tg176.2.12} 
	\textbf{Source : }Grundlegung zur Metaphysik der Sitten/Zweiter Abschnitt: Übergang von der populären sittlichen Weltweisheit zur Metaphysik der Sitten\\  
	
	\textbf{Paragraphe : }Aus dem Angeführten erhellet: daß alle sittliche Begriffe völlig a priori in der Vernunft ihren Sitz und Ursprung haben, und dieses zwar in der gemeinsten Menschenvernunft eben sowohl, als der im höchsten Maße spekulativen; daß sie von keinem empirischen und darum bloß zufälligen Erkenntnisse  abstrahiert werden können; daß in dieser Reinigkeit ihres Ursprungs eben ihre Würde liege, um uns zu obersten praktischen Prinzipien zu dienen; daß man jedesmal so viel, als man Empirisches hinzu tut, so viel auch ihrem echten Einflusse und dem uneingeschränkten Werte der Handlungen entziehe; daß es nicht allein die größte Notwendigkeit in theoretischer Absicht, wenn es bloß auf Spekulation ankommt, erfodere, sondern auch von der größten praktischen Wichtigkeit sei, ihre Begriffe und Gesetze aus reiner Vernunft zu schöpfen, rein und unvermengt vorzutragen, ja den \match{Umfang} dieses ganzen praktischen oder reinen Vernunfterkenntnisses, d.i. das ganze Vermögen der reinen praktischen Vernunft, zu bestimmen, hierin aber nicht, wie es wohl die spekulative Philosophie erlaubt, ja gar bisweilen notwendig findet, die Prinzipien von der besondern Natur der menschlichen Vernunft abhängig zu machen, sondern darum, weil moralische Gesetze für jedes vernünftige Wesen überhaupt gelten sollen, sie schon aus dem allgemeinen Begriffe eines vernünftigen Wesens überhaupt abzuleiten, und auf solche Weise alle Moral, die zu ihrer Anwendung auf Menschen der Anthropologie bedarf, zuerst unabhängig von dieser als reine Philosophie, d.i. als Metaphysik, vollständig (welches sich in dieser Art ganz abgesonderter Erkenntnisse wohl tun läßt) vorzutragen, wohl bewußt, daß es, ohne im Be sitze derselben zu sein, vergeblich sei, ich will nicht sagen, das Moralische der Pflicht in allem, was pflichtmäßig ist, genau für die spekulative Beurteilung zu bestimmen, sondern so gar im bloß gemeinen und praktischen Gebrauche, vornehmlich der moralischen Unterweisung, unmöglich sei, die Sitten auf ihre echte Prinzipien zu gründen und dadurch reine moralische Gesinnungen zu bewirken und zum höchsten Weltbesten den Gemütern einzupfropfen. 
	
	\unnumberedsection{Verbindung (3)} 
	\subsection*{tg174.2.7} 
	\textbf{Source : }Grundlegung zur Metaphysik der Sitten/Vorrede\\  
	
	\textbf{Paragraphe : }Alle Gewerbe, Handwerke und Künste, haben durch die Verteilung der Arbeiten gewonnen, da nämlich nicht einer alles macht, sondern jeder sich auf gewisse Arbeit, die sich, ihrer Behandlungsweise nach, von andern merklich unterscheidet, einschränkt, um sie in der größten Vollkommenheit und mit mehrerer Leichtigkeit leisten zu können. Wo die Arbeiten so nicht unterschieden und verteilt werden, wo jeder ein Tausendkünstler ist, da liegen die Gewerbe noch in der größten Barbarei. Aber ob dieses zwar für sich ein der Erwägung nicht unwürdiges Objekt wäre, zu fragen: ob die reine Philosophie in allen ihren Teilen nicht ihren besondern Mann erheische, und es um das Ganze des gelehrten Gewerbes nicht besser stehen würde, wenn die, so das Empirische mit dem Rationalen, dem Geschmacke des Publikums gemäß, nach allerlei ihnen selbst unbekannten Verhältnissen gemischt, zu verkaufen gewohnt sind, die sich Selbstdenker, andere aber, die den bloß rationalen Teil zubereiten, Grübler nennen, gewarnt würden, nicht zwei Geschäfte zugleich zu treiben, die in der Art, sie zu behandeln, gar sehr verschieden sind, zu deren jedem vielleicht ein besonderes Talent erfodert wird, und deren \match{Verbindung} in einer Person nur Stümper hervorbringt: so frage ich hier doch nur, ob nicht die Natur der Wissenschaft es erfodere, den empirischen von dem rationalen Teil jederzeit sorgfältig abzusondern, und vor der eigentlichen (empirischen) Physik eine Metaphysik der Natur, vor der praktischen Anthropologie aber eine Metaphysik der Sitten voranzuschicken, die von allem Empirischen sorgfältig gesäubert sein müßte, um zu  wissen, wie viel reine Vernunft in beiden Fällen leisten könne, und aus welchen Quellen sie selbst diese ihre Belehrung a priori schöpfe, es mag übrigens das letztere Geschäfte von allen Sittenlehrern (deren Name Legion heißt), oder nur von einigen, die Beruf dazu fühlen, getrieben werden. 
	
	\subsection*{tg176.2.64} 
	\textbf{Source : }Grundlegung zur Metaphysik der Sitten/Zweiter Abschnitt: Übergang von der populären sittlichen Weltweisheit zur Metaphysik der Sitten\\  
	
	\textbf{Paragraphe : }Ich verstehe aber unter einem Reiche die systematische \match{Verbindung} verschiedener vernünftiger Wesen durch gemeinschaftliche Gesetze. Weil nun Gesetze die Zwecke ihrer allgemeinen Gültigkeit nach bestimmen, so wird, wenn man von dem persönlichen Unterschiede vernünftiger Wesen, imgleichen allem Inhalte ihrer Privatzwecke abstrahiert, ein Ganzes aller Zwecke (sowohl der vernünftigen Wesen als Zwecke an sich, als auch der eigenen Zwecke, die ein jedes sich selbst setzen mag), in systematischer Verknüpfung, d.i. ein Reich der Zwecke gedacht werden können, welches nach obigen Prinzipien möglich ist. 
	
	\subsection*{tg176.2.65} 
	\textbf{Source : }Grundlegung zur Metaphysik der Sitten/Zweiter Abschnitt: Übergang von der populären sittlichen Weltweisheit zur Metaphysik der Sitten\\  
	
	\textbf{Paragraphe : }Denn vernünftige Wesen stehen alle unter dem Gesetz, daß jedes derselben sich selbst und alle andere niemals bloß als Mittel, sondern jederzeit zugleich als Zweck an sich selbst behandeln solle. Hiedurch aber entspringt eine systematische \match{Verbindung} vernünftiger Wesen durch gemeinschaftliche objektive Gesetze, d.i. ein Reich, welches, weil diese Gesetze eben die Beziehung dieser Wesen auf einander, als Zwecke und Mittel, zur Absicht haben, ein Reich der Zwecke (freilich nur ein Ideal) heißen kann. 
	
	\unnumberedsection{Verhältnis (7)} 
	\subsection*{tg176.2.16} 
	\textbf{Source : }Grundlegung zur Metaphysik der Sitten/Zweiter Abschnitt: Übergang von der populären sittlichen Weltweisheit zur Metaphysik der Sitten\\  
	
	\textbf{Paragraphe : }
	Alle Imperativen werden durch ein Sollen ausgedruckt, und zeigen dadurch das \match{Verhältnis} eines objektiven Gesetzes der Vernunft zu einem Willen an, der seiner subjektiven Beschaffenheit nach dadurch nicht notwendig bestimmt wird (eine Nötigung). Sie sagen, daß etwas zu tun oder zu unterlassen gut sein würde, allein sie sagen es einem Willen, der nicht immer darum etwas tut, weil ihm vorgestellt wird, daß es zu tun gut sei. Praktisch gut ist aber, was vermittelst der Vorstellungen der Vernunft, mithin nicht aus subjektiven Ursachen, sondern objektiv, d.i. aus Gründen, die für jedes vernünftige Wesen, als ein solches, gültig sind, den Willen bestimmt. Es wird vom Angenehmen unterschieden, als demjenigen, was nur vermittelst der Empfindung aus bloß subjektiven Ursachen, die nur für dieses oder jenes seinen Sinn gelten, und nicht als Prinzip der Vernunft, das für jedermann gilt, auf den Willen Einfluß hat.
	
	
	5
	
	
	
	\subsection*{tg176.2.17} 
	\textbf{Source : }Grundlegung zur Metaphysik der Sitten/Zweiter Abschnitt: Übergang von der populären sittlichen Weltweisheit zur Metaphysik der Sitten\\  
	
	\textbf{Paragraphe : }Ein vollkommen guter Wille würde also eben sowohl unter objektiven Gesetzen (des Guten) stehen, aber nicht dadurch als zu gesetzmäßigen Handlungen genötigt vorgestellt werden können, weil er von selbst, nach seiner subjektiven Beschaffenheit, nur durch die Vorstellung des Guten  bestimmt werden kann. Daher gelten für den göttlichen und überhaupt für einen heiligen Willen keine Imperativen; das Sollen ist hier am unrechten Orte, weil das Wollen schon von selbst mit dem Gesetz notwendig einstimmig ist. Daher sind Imperativen nur Formeln, das \match{Verhältnis} objektiver Gesetze des Wollens überhaupt zu der subjektiven Unvollkommenheit des Willens dieses oder jenes vernünftigen Wesens, z.B. des menschlichen Willens, auszudrücken. 
	
	\subsection*{tg176.2.20} 
	\textbf{Source : }Grundlegung zur Metaphysik der Sitten/Zweiter Abschnitt: Übergang von der populären sittlichen Weltweisheit zur Metaphysik der Sitten\\  
	
	\textbf{Paragraphe : }Der Imperativ sagt also, welche durch mich mögliche Handlung gut wäre, und stellt die praktische Regel in \match{Verhältnis} auf einen Willen vor, der darum nicht sofort eine Handlung tut, weil sie gut ist, teils weil das Subjekt nicht immer weiß, daß sie gut sei, teils weil, wenn es dieses auch wüßte, die Maximen desselben doch den objektiven Prinzipien einer praktischen Vernunft zuwider sein könnten. 
	
	\subsection*{tg178.2.2} 
	\textbf{Source : }Grundlegung zur Metaphysik der Sitten/Zweiter Abschnitt: Übergang von der populären sittlichen Weltweisheit zur Metaphysik der Sitten/Die Heteronomie des Willens als der Quell aller unechten Prinzipien der Sittlichkeit\\  
	
	\textbf{Paragraphe : }Wenn der Wille irgend worin anders, als in der Tauglichkeit seiner Maximen zu seiner eigenen allgemeinen Gesetzgebung, mithin, wenn er, indem er über sich selbst hinausgeht, in der Beschaffenheit irgend eines seiner Objekte das Gesetz sucht, das ihn bestimmen soll, so kommt jederzeit Heteronomie heraus. Der Wille gibt alsdenn sich nicht selbst, sondern das Objekt durch sein \match{Verhältnis} zum Willen gibt diesem das Gesetz. Dies Verhältnis, es beruhe nun auf der Neigung, oder auf Vorstellungen der Vernunft, läßt nur hypothetische Imperativen möglich werden: ich soll etwas tun darum, weil ich etwas anderes will. Dagegen sagt der moralische, mithin kategorische Imperativ: ich soll so oder so handeln, ob ich gleich nichts anderes wollte. Z. E. jener sagt: ich soll nicht lügen, wenn ich bei Ehren bleiben will; dieser aber: ich soll nicht lügen, ob es  mir gleich nicht die mindeste Schande zuzöge. Der letztere muß also von allem Gegenstande so fern abstrahieren, daß dieser gar keinen Einfluß auf den Willen habe, damit praktische Vernunft (Wille) nicht fremdes Interesse bloß administriere, sondern bloß ihr eigenes gebietendes Ansehen, als oberste Gesetzgebung, beweise. So soll ich z.B. fremde Glückseligkeit zu befördern suchen, nicht als wenn mir an deren Existenz was gelegen wäre (es sei durch unmittelbare Neigung, oder irgend ein Wohlgefallen indirekt durch Vernunft), sondern bloß deswegen, weil die Maxime, die sie ausschließt, nicht in einem und demselben Wollen, als allgemeinen Gesetz begriffen werden kann. 
	
	\subsection*{tg181.2.4} 
	\textbf{Source : }Grundlegung zur Metaphysik der Sitten/Dritter Abschnitt: Übergang von der Metaphysik der Sitten zur Kritik der reinen praktischen Vernunft/Der Begriff der Freiheit ist der Schlüssel zur Erklärung der Autonomie des Willens\\  
	
	\textbf{Paragraphe : }Wenn also Freiheit des Willens vorausgesetzt wird, so folgt die Sittlichkeit samt ihrem Prinzip daraus, durch bloße Zergliederung ihres Begriffs. Indessen ist das letztere doch immer ein synthetischer Satz: ein schlechterdings guter Wille ist derjenige, dessen Maxime jederzeit sich selbst, als allgemeines Gesetz betrachtet, in sich enthalten kann; denn durch Zergliederung des Begriffs von einem schlechthin guten Willen kann jene Eigenschaft der Maxime nicht gefunden werden. Solche synthetische Sätze sind aber nur dadurch möglich, daß beide Erkenntnisse durch die Verknüpfung mit einem dritten, darin sie beiderseits anzutreffen sind, unter einander verbunden werden. Der positive Begriff der Freiheit schafft dieses dritte, welches nicht, wie bei den physischen Ursachen, die Natur der Sinnenwelt sein kann (in deren Begriff die Begriffe von etwas als Ursache, in \match{Verhältnis} auf etwas anderes als Wirkung, zusammenkommen). Was dieses dritte sei, worauf uns die Freiheit weiset, und von dem wir a priori eine Idee haben, läßt sich hier sofort noch nicht anzeigen, und die Deduktion des Begriffs der Freiheit aus der reinen praktischen Vernunft, mit ihr auch die Möglichkeit eines kategorischen Imperativs, begreiflich machen, sondern bedarf noch einiger Vorbereitung. 
	
	\subsection*{tg185.2.13} 
	\textbf{Source : }Grundlegung zur Metaphysik der Sitten/Dritter Abschnitt: Übergang von der Metaphysik der Sitten zur Kritik der reinen praktischen Vernunft/Von der äußersten Grenze aller praktischen Philosophie\\  
	
	\textbf{Paragraphe : }Um das zu wollen, wozu die Vernunft allein dem sinnlich-affizierten vernünftigen Wesen das Sollen vorschreibt, dazu gehört freilich ein Vermögen der Vernunft, ein Gefühl der Lust oder des Wohlgefallens an der Erfüllung der Pflicht einzuflößen, mithin eine Kausalität derselben, die Sinnlichkeit ihren Prinzipien gemäß zu bestimmen. Es ist aber gänzlich unmöglich, einzusehen, d.i. a priori begreiflich zu machen, wie ein bloßer Gedanke, der selbst nichts Sinnliches in sich enthält, eine Empfindung der Lust oder Unlust hervorbringe; denn das ist eine besondere Art von Kausalität, von der, wie von aller Kausalität, wir gar nichts a priori bestimmen können, sondern darum allein die Erfahrung befragen müssen. Da diese aber kein \match{Verhältnis} der Ursache zur Wirkung, als zwischen zwei Gegenständen der Erfahrung, an die Hand geben kann, hier aber reine Vernunft durch bloße Ideen (die gar keinen Gegenstand für Erfahrung abgeben) die Ursache von einer Wirkung, die freilich in der Erfahrung liegt, sein soll, so ist die Erklärung, wie und warum uns die Allgemeinheit der Maxime als Gesetzes, mithin die Sittlichkeit, interessiere, uns Menschen gänzlich unmöglich. So viel ist nur gewiß: daß es nicht darum für uns Gültigkeit hat, weil es interessiert (denn das ist Heteronomie und Abhängigkeit der praktischen Vernunft von Sinnlichkeit, nämlich einem zum Grunde liegenden Gefühl, wobei sie niemals sittlich gesetzgebend sein könnte), sondern daß es interessiert, weil es für uns als Menschen gilt, da es aus unserem Willen als Intelligenz, mithin aus unserem eigentlichen Selbst, entsprungen ist; was aber zur bloßen Erscheinung gehört, wird von der Vernunft notwendig der Beschaffenheit der Sache an sich selbst untergeordnet.
	
	
	\subsection*{tg185.2.7} 
	\textbf{Source : }Grundlegung zur Metaphysik der Sitten/Dritter Abschnitt: Übergang von der Metaphysik der Sitten zur Kritik der reinen praktischen Vernunft/Von der äußersten Grenze aller praktischen Philosophie\\  
	
	\textbf{Paragraphe : }
	Der Rechtsanspruch aber, selbst der gemeinen Menschenvernunft, auf Freiheit des Willens, gründet sich auf das Bewußtsein und die zugestandene Voraussetzung der Unabhängigkeit der Vernunft, von bloß subjektiv-bestimmten Ursachen, die insgesamt das ausmachen, was bloß zur Empfindung, mithin unter die allgemeine Benennung der Sinnlichkeit, gehört. Der Mensch, der sich auf solche Weise als Intelligenz betrachtet, setzt sich dadurch in eine andere Ordnung der Dinge und in ein \match{Verhältnis} zu bestimmenden Gründen von ganz anderer Art, wenn er sich als Intelligenz mit einem Willen, folglich mit Kausalität begabt, denkt, als wenn er sich wie Phänomen in der Sinnenwelt (welches er wirklich auch ist) wahrnimmt, und seine Kausalität, äußerer Bestimmung nach, Naturgesetzen unterwirft. Nun wird er bald inne, daß beides zugleich stattfinden könne, ja sogar müsse. Denn, daß ein Ding in der Erscheinung (das zur Sinnenwelt gehörig) gewissen Gesetzen unterworfen ist, von welchen eben dasselbe, als Ding oder Wesen an sich selbst, unabhängig ist, enthält nicht den mindesten Widerspruch; daß er sich selbst aber auf diese zwiefache Art vorstellen und denken müsse, beruht, was das erste betrifft, auf dem Bewußtsein seiner selbst als durch Sinne affizierten Gegenstandes, was das zweite anlangt, auf dem Bewußtsein seiner selbst als Intelligenz, d.i. als unabhängig im Vernunftgebrauch von sinnlichen Eindrücken (mithin als zur Verstandeswelt gehörig). 
	
	\unnumberedchapter{Vetement} 
	\unnumberedsection{Juwel (1)} 
	\subsection*{tg175.2.5} 
	\textbf{Source : }Grundlegung zur Metaphysik der Sitten/Erster Abschnitt: Übergang von der gemeinen sittlichen Vernunfterkenntnis zur philosophischen\\  
	
	\textbf{Paragraphe : }Der gute Wille ist nicht durch das, was er bewirkt, oder ausrichtet, nicht durch seine Tauglichkeit zu Erreichung irgend eines vorgesetzten Zweckes, sondern allein durch das Wollen, d.i. an sich, gut, und, für sich selbst betrachtet, ohne Vergleich weit höher zu schätzen, als alles, was durch ihn zu Gunsten irgend einer Neigung, ja, wenn man will, der Summe aller Neigungen, nur immer zu Stande gebracht werden könnte. Wenn gleich durch eine besondere Ungunst des Schicksals, oder durch kärgliche Ausstattung einer stiefmütterlichen Natur, es diesem Willen gänzlich an Vermögen fehlete, seine Absicht durchzusetzen; wenn bei seiner größten Bestrebung dennoch nichts von ihm ausgerichtet würde, und nur der gute Wille (freilich nicht etwa ein bloßer Wunsch, sondern als die Aufbietung aller Mittel, so weit sie in unserer Gewalt sind) übrig bliebe: so würde er wie ein \match{Juwel} doch für sich selbst glänzen, als etwas, das seinen vollen Wert in sich selbst hat. Die Nützlichkeit oder Fruchtlosigkeit kann diesem Werte weder etwas zusetzen, noch abnehmen. Sie würde gleichsam nur die Einfassung sein, um ihn im gemeinen Verkehr besser handhaben zu können, oder die Aufmerksamkeit derer, die noch nicht gnug Kenner sind, auf sich zu ziehen, nicht aber, um ihn Kennern zu empfehlen, und seinen Wert zu bestimmen. 
	
	\unnumberedsection{Verarbeitung (1)} 
	\subsection*{tg174.2.13} 
	\textbf{Source : }Grundlegung zur Metaphysik der Sitten/Vorrede\\  
	
	\textbf{Paragraphe : }Weil aber drittens auch eine Metaphysik der Sitten, ungeachtet des abschreckenden Titels, dennoch eines großen Grades der Popularität und Angemessenheit zum gemeinen Verstande fähig ist, so finde ich für nützlich, diese \match{Verarbeitung} der Grundlage davon abzusondern, um das Subtile, was darin unvermeidlich ist, künftig nicht faßlichern Lehren beifügen zu dürfen. 
	
\end{document}