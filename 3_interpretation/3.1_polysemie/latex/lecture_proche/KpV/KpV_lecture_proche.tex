\documentclass[a4paper,12pt,twoside]{book}
\usepackage{fontspec}
\usepackage{xunicode}
\usepackage[german]{babel}


\usepackage{xcolor}

\newcommand{\match}[1]{\textcolor{red}{\textbf{#1}}}





\usepackage{fancyhdr}
\usepackage{etoolbox} % For defining robust commands

\pagestyle{fancy}

\fancyhead[LE]{\thepage}
\fancyhead[RE]{\leftmark}

\fancyhead[LO]{\rightmark}
\fancyhead[RO]{\thepage}

% Define a new command for unnumbered chapters
\newcommand{\unnumberedchapter}[1]{
	\chapter*{#1}
	\addcontentsline{toc}{chapter}{#1}
	\markboth{#1}{#1}
}

% Similarly, for sections
\newcommand{\unnumberedsection}[1]{
	\section*{#1}
	\addcontentsline{toc}{section}{#1}
	\markright{#1}
}

\usepackage[hidelinks]{hyperref} %
\usepackage[numbered]{bookmark}%va avec hyperref; marche mieux pour les signets. l'option numbered: les signets dans le pdf sont numérotés

\author{Eglantine Gaglione - M1 HN}
\title{KpV : vocabulaire thématique - mots "non-polysémiques"}

\begin{document}
	
	\maketitle
	
	\tableofcontents
	
	\unnumberedchapter{Agriculture} 
	\unnumberedsection{Boden (3)} 
	\subsection*{tg215.2.6} 
	\textbf{Source : }Kritik der praktischen Vernunft/Erster Teil. Elementarlehre der reinen praktischen Vernunft/Erstes Buch. Die Analytik der reinen praktischen Vernunft/Drittes Hauptstück. Von den Triebfedern der reinen praktischen Vernunft/Kritische Beleuchtung der Analytik der reinen praktischen Vernunft\\  
	
	\textbf{Paragraphe : }Die Unterscheidung der Glückseligkeit sichre von der Sittenlehre, in derer ersteren empirische Prinzipien das ganze Fundament, von der zweiten aber auch nicht den mindesten Beisatz derselben ausmachen, ist nun in der Analytik der reinen praktischen Vernunft die erste und wichtigste ihr obliegende Beschäftigung, in der sie so pünktlich, ja, wenn es auch hieße, peinlich, verfahren muß, als je der Geometer in seinem Geschäfte. Es kommt aber dem Philosophen, der hier (wie jederzeit im Vernunfterkenntnisse durch bloße Begriffe, ohne Konstruktion derselben) mit größerer Schwierigkeit zu kämpfen hat, weil er keine Anschauung  (reinem Noumen) zum Grunde legen kann, doch auch zu statten: daß er, beinahe wie der Chemist, zu aller Zeit ein Experiment mit jedes Menschen praktischer Vernunft anstellen kann, um den moralischen (reinen) Bestimmungsgrund vom empirischen zu unterscheiden; wenn er nämlich zu dem empirisch-affizierten Willen (z.B. desjenigen, der gerne lügen möchte, weil er sich dadurch was erwerben kann) das moralische Gesetz (als Bestimmungsgrund) zusetzt. Es ist, als ob der Scheidekünstler der Solution der Kalkerde in Salzgeist Alkali zusetzt; der Salzgeist verläßt so fort den Kalk, vereinigt sich mit dem Alkali, und jener wird zu \match{Boden} gestürzt. Eben so haltet dem, der sonst ein ehrlicher Mann ist (oder sich doch diesmal nur in Gedanken in die Stelle eines ehrlichen Mannes versetzt), das moralische Gesetz vor, an dem er die Nichtswürdigkeit eines Lügners erkennt, so fort verläßt seine praktische Vernunft (im Urteil über das, was von ihm geschehen sollte) den Vorteil, vereinigt sich mit dem, was ihm die Achtung für seine eigene Person erhält (der Wahrhaftigkeit), und der Vorteil wird nun von jedermann, nachdem er von allem Anhängsel der Vernunft (welche nur gänzlich auf der Seite der Pflicht ist) abgesondert und gewaschen worden, gewogen, um mit der Vernunft noch wohl in anderen Fällen in Verbindung zu treten, nur nicht, wo er dem moralischen Gesetze, welches die Vernunft niemals verläßt, sondern sich innigst damit vereinigt, zuwider sein könnte. 
	
	\subsection*{tg221.2.4} 
	\textbf{Source : }Kritik der praktischen Vernunft/Erster Teil. Elementarlehre der reinen praktischen Vernunft/Zweites Buch. Dialektik der reinen praktischen Vernunft/Zweites Hauptstück. Von der Dialektik der reinen Vernunft in Bestimmung des Begriffs vom höchsten Gut/III. Von dem Primat der reinen praktischen Vernunft in ihrer Verbindung mit der spekulativen\\  
	
	\textbf{Paragraphe : }
	In der Tat, so fern praktische Vernunft als pathologisch bedingt, d.i. das Interesse der Neigungen unter dem sinnlichen Prinzip der Glückseligkeit bloß verwaltend, zum Grunde gelegt würde, so ließe sich diese Zumutung an die spekulative Vernunft gar nicht tun. Mahomets Paradies, oder der Theosophen und Mystiker schmelzende Vereinigung mit der Gottheit, so wie jedem sein Sinn steht, würden der Vernunft ihre Ungeheuer aufdringen, und es wäre eben so gut, gar keine zu haben, als sie auf solche Weise allen Träumereien preiszugeben. Allein wenn reine Vernunft für sich praktisch sein kann und es wirklich ist, wie das Bewußtsein des moralischen Gesetzes es ausweiset, so ist es doch immer nur eine und dieselbe Vernunft, die, es sei in theoretischer oder praktischer Absicht, nach Prinzipien a priori urteilt, und da ist es klar, daß, wenn ihr Vermögen in der ersteren gleich nicht zulangt, gewisse Sätze behauptend festzusetzen, indessen daß sie ihr auch eben nicht widersprechen, eben diese Sätze, so bald sie unabtrennlich zum praktischen Interesse der reinen Vernunft gehören, zwar als ein ihr fremdes Angebot, das nicht auf ihrem \match{Boden} erwachsen, aber doch hinreichend beglaubigt ist, annehmen, und sie, mit allem, was sie als spekulative Vernunft in ihrer Macht hat, zu vergleichen und zu verknüpfen suchen müsse; doch sich bescheidend, daß dieses nicht ihre Einsichten, aber doch Erweiterungen ihres Gebrauchs in irgend einer anderen, nämlich praktischen, Absicht sind, welches ihrem Interesse, das in der Einschränkung des spekulativen Frevels besteht, ganz und gar nicht zuwider ist. 
	
	\subsection*{tg225.2.11} 
	\textbf{Source : }Kritik der praktischen Vernunft/Erster Teil. Elementarlehre der reinen praktischen Vernunft/Zweites Buch. Dialektik der reinen praktischen Vernunft/Zweites Hauptstück. Von der Dialektik der reinen Vernunft in Bestimmung des Begriffs vom höchsten Gut/VII. Wie eine Erweiterung der reinen Vernunft, in praktischer Absicht, ohne damit ihr Erkenntnis, als spekulativ, zugleich zu erweitern, zu denken möglich sei\\  
	
	\textbf{Paragraphe : }Wenn man in der Geschichte der griechischen Philosophie über den Anaxagoras hinaus keine deutliche Spuren einer reinen Vernunfttheologie antrifft, so ist der Grund nicht darin gelegen, daß es den älteren Philosophen an Verstande und Einsicht fehlte, um durch den Weg der Spekulation, wenigstens mit Beihülfe einer ganz vernünftigen Hypothese, sich dahin zu erheben; was konnte leichter, was natürlicher sein, als der sich von selbst jedermann darbietende Gedanke, statt unbestimmter Grade der Vollkommenheit verschiedener Weltursachen, eine einzige vernünftige anzunehmen, die alle Vollkommenheit hat? Aber die Übel in der Welt schienen ihnen viel zu wichtige Einwürfe zu sein, um zu einer solchen Hypothese sich für berechtigt zu halten. Mithin zeigten sie darin eben Verstand und Einsicht, daß sie sich jene nicht erlaubten, und vielmehr in den Naturursachen herumsuchten, ob sie unter ihnen nicht die zu Urwesen erfoderliche Beschaffenheit und Vermögen antreffen möchten. Aber nachdem dieses scharfsinnige Volk so weit in Nachforschungen fortgerückt war, selbst sittliche Gegenstände, darüber andere Völker niemals mehr als geschwatzt haben, philosophisch zu behandeln: da fanden sie allererst ein neues Bedürfnis, nämlich ein praktisches, welches nicht ermangelte, ihnen den Begriff des Urwesens bestimmt anzugeben, wobei die spekulative Vernunft das Zusehen hatte, höchstens noch das Verdienst, einen Begriff, der nicht auf ihrem \match{Boden} erwachsen war, auszuschmücken, und mit einem Gefolge von Bestätigungen aus der Naturbetrachtung, die nun allererst hervortraten, wohl nicht das Ansehen desselben (welches schon gegründet war), sondern vielmehr nur das Gepränge mit vermeinter theoretischer Vernunfteinsicht zu befördern. 
	
	\unnumberedsection{Fällen (2)} 
	\subsection*{tg204.2.12} 
	\textbf{Source : }Kritik der praktischen Vernunft/Erster Teil. Elementarlehre der reinen praktischen Vernunft/Erstes Buch. Die Analytik der reinen praktischen Vernunft/Erstes Hauptstück. Von den Grundsätzen der reinen praktischen Vernunft/3. Lehrsatz II\\  
	
	\textbf{Paragraphe : }Glücklich zu sein, ist notwendig das Verlangen jedes vernünftigen aber endlichen Wesens, und also ein unvermeidlicher Bestimmungsgrund seines Begehrungsvermögens. Denn die Zufriedenheit mit seinem ganzen Dasein ist nicht etwa ein ursprünglicher Besitz, und eine Seligkeit, welche ein Bewußtsein seiner unabhängigen Selbstgenugsamkeit voraussetzen würde, sondern ein durch seine endliche Natur selbst ihm aufgedrungenes Problem, weil es bedürftig ist, und dieses Bedürfnis betrifft die Materie seines Begehrungsvermögens, d.i. etwas, was sich auf ein subjektiv zum Grunde liegendes Gefühl der Lust oder Unlust bezieht, dadurch das, was es zur Zufriedenheit mit seinem Zustande bedarf, bestimmt wird. Aber eben darum, weil dieser materiale Bestimmungsgrund von dem Subjekte bloß empirisch erkannt werden kann, ist es unmöglich, diese Aufgabe als ein Gesetz zu betrachten, weil dieses als objektiv in allen \match{Fällen} und für alle vernünftige Wesen eben denselben Bestimmungsgrund des Willens enthalten müßte. Denn obgleich der Begriff der Glückseligkeit der praktischen Beziehung der Objekte aufs Begehrungsvermögen allerwärts zum Grunde liegt, so ist er doch nur der allgemeine Titel der subjektiven Bestimmungsgründe, und bestimmt nichts spezifisch, darum es doch in dieser praktischen Aufgabe allein zu tun ist, und ohne welche Bestimmung sie gar nicht aufgelöset werden kann. Worin nämlich jeder seine Glückseligkeit zu setzen habe, kommt auf jedes sein besonderes Gefühl der Lust und Unlust an, und selbst in einem und demselben Subjekt auf die Verschiedenheit der Bedürfnis, nach den Abänderungen dieses Gefühls, und ein subjektiv notwendiges Gesetz (als Naturgesetz) ist also objektiv ein gar sehr zufälliges praktisches Prinzip, das in verschiedenen Subjekten sehr verschieden sein kann und  muß, mithin niemals ein Gesetz abgeben kann, weil es, bei der Begierde nach Glückseligkeit, nicht auf die Form der Gesetzmäßigkeit, sondern lediglich auf die Materie ankommt, nämlich ob und wie viel Vergnügen ich in der Befolgung des Gesetzes zu erwarten habe. Prinzipien der Selbstliebe können zwar allgemeine Regeln der Geschicklichkeit (Mittel zu Absichten auszufinden) enthalten, alsdenn sind es aber bloß theoretische Prinzipien
	
	
	7
	, (z.B. wie derjenige, der gerne Brot essen möchte, sich eine Mühle auszudenken habe). Aber praktische Vorschriften, die sich auf sie gründen, können niemals allgemein sein, denn der Bestimmungsgrund des Begehrungsvermögens ist auf das Gefühl der Lust und Unlust, das niemals als allgemein, auf dieselben Gegenstände gerichtet, angenommen werden kann, gegründet. 
	
	\subsection*{tg225.2.10} 
	\textbf{Source : }Kritik der praktischen Vernunft/Erster Teil. Elementarlehre der reinen praktischen Vernunft/Zweites Buch. Dialektik der reinen praktischen Vernunft/Zweites Hauptstück. Von der Dialektik der reinen Vernunft in Bestimmung des Begriffs vom höchsten Gut/VII. Wie eine Erweiterung der reinen Vernunft, in praktischer Absicht, ohne damit ihr Erkenntnis, als spekulativ, zugleich zu erweitern, zu denken möglich sei\\  
	
	\textbf{Paragraphe : }Ich versuche nun, diesen Begriff an das Objekt der praktischen Vernunft zu halten, und da finde ich, daß der moralische Grundsatz ihn nur als möglich, unter Voraussetzung eines Welturhebers von höchster Vollkommenheit, zulasse. Er muß allwissend sein, um mein Verhalten bis zum Innersten meiner Gesinnung in allen möglichen \match{Fällen} und in alle Zukunft zu er kennen; allmächtig, um ihm die angemessenen Folgen zu erteilen; eben so allgegenwärtig, ewig, usw. Mithin bestimmt das moralische Gesetz durch den Begriff des höchsten Guts, als Gegenstandes einer reinen praktischen Vernunft, den Begriff des Urwesens als höchsten Wesens, welches der physische (und höher fortgesetzt der metaphysische), mithin der ganze spekulative  Gang der Vernunft nicht bewirken konnte. Also ist der Begriff von Gott ein ursprünglich nicht zur Physik, d.i. für die spekulative Vernunft, sondern zur Moral gehöriger Begriff, und eben das kann man auch von den übrigen Vernunftbegriffen sagen, von denen wir, als Postulaten derselben in ihrem praktischen Gebrauche, oben gehandelt haben. 
	
	\unnumberedsection{Gut (22)} 
	\subsection*{tg209.2.11} 
	\textbf{Source : }Kritik der praktischen Vernunft/Erster Teil. Elementarlehre der reinen praktischen Vernunft/Erstes Buch. Die Analytik der reinen praktischen Vernunft/Erstes Hauptstück. Von den Grundsätzen der reinen praktischen Vernunft/8. Lehrsatz IV\\  
	
	\textbf{Paragraphe : }Wenn ein dir sonst beliebter Umgangsfreund sich bei dir wegen eines falschen abgelegten Zeugnisses dadurch zu rechtfertigen vermeinete, daß er zuerst die, seinem Vorgeben nach, heilige Pflicht der eigenen Glückseligkeit vorschützte, alsdenn die Vorteile herzählte, die er sich alle dadurch erworben, die Klugheit namhaft machte, die er beobachtet, um wider alle Entdeckung sicher zu sein, selbst wider die von Seiten deiner selbst, dem er das Geheimnis darum allein offenbaret, damit er es zu aller Zeit ableugnen könne; dann aber im ganzen Ernst vorgäbe, er habe eine wahre Menschenpflicht ausgeübt: so würdest du ihm entweder gerade ins Gesicht lachen, oder mit Abscheu davon zurückbeben, ob du gleich, wenn jemand bloß auf eigene Vorteile seine Grundsätze gesteuert hat, wider diese Maßregeln nicht das mindeste einzuwenden hättest. Oder setzet, es empfehle euch jemand einen Mann zum Haushalter, dem ihr alle eure Angelegenheiten blindlings anvertrauen könnet, und, um euch Zutrauen einzuflößen, rühmete er ihn als einen klugen Menschen, der sich auf seinen eigenen Vorteil meisterhaft verstehe, auch als einen rastlos wirksamen, der keine Gelegenheit dazu ungenutzt vorbeigehen ließe, endlich, damit auch ja nicht Besorgnisse wegen eines pöbelhaften Eigennutzes desselben im Wege stünden, rühmete er, wie er recht fein zu leben verstünde, nicht im Geldsammeln oder brutaler Üppigkeit, sondern in der Erweiterung seiner Kenntnisse, einem wohlgewählten belehrenden Umgange, selbst im Wohltun der Dürftigen, sein Vergnügen suchte, übrigens aber wegen der Mittel (die doch ihren Wert oder Unwert nur vom Zwecke entlehnen) nicht bedenklich wäre, und fremdes Geld und \match{Gut} ihm hiezu, so bald er nur wisse,  daß er es unentdeckt und ungehindert tun könne, so gut wie sein eigenes wäre: so würdet ihr entweder glauben, der Empfehlende habe euch zum besten, oder er habe den Verstand verloren. – So deutlich und scharf sind die Grenzen der Sittlichkeit und der Selbstliebe abgeschnitten, daß selbst das gemeinste Auge den Unterschied, ob etwas zu der einen oder der andern gehöre, gar nicht verfehlen kann. Folgende wenige Bemerkungen können zwar bei einer so offenbaren Wahrheit überflüssig scheinen, allein sie dienen doch wenigstens dazu, dem Urteile der gemeinen Menschenvernunft etwas mehr Deutlichkeit zu verschaffen. 
	
	\subsection*{tg210.2.5} 
	\textbf{Source : }Kritik der praktischen Vernunft/Erster Teil. Elementarlehre der reinen praktischen Vernunft/Erstes Buch. Die Analytik der reinen praktischen Vernunft/Erstes Hauptstück. Von den Grundsätzen der reinen praktischen Vernunft/I. Von der Deduktion der Grundsätze der reinen praktischen Vernunft\\  
	
	\textbf{Paragraphe : }Dieses Gesetz soll der Sinnenwelt, als einer sinnlichen Natur (was die vernünftigen Wesen betrifft), die Form einer Verstandeswelt, d.i. einer übersinnlichen Natur verschaffen, ohne doch jener ihrem Mechanism Abbruch zu tun. Nun ist Natur im allgemeinsten Verstande die Existenz der Dinge unter Gesetzen. Die sinnliche Natur vernünftiger Wesen überhaupt ist die Existenz derselben unter empirisch bedingten Gesetzen, mithin für die Vernunft Heteronomie. Die übersinnliche Natur eben derselben Wesen ist dagegen ihre Existenz nach Gesetzen, die von aller empirischen Bedingung unabhängig sind, mithin zur Autonomie der reinen Vernunft gehören. Und, da die Gesetze, nach welchen  das Dasein der Dinge vom Erkenntnis abhängt, praktisch sind: so ist die übersinnliche Natur, so weit wir uns einen Begriff von ihr machen können, nichts anders, als eine Natur unter der Autonomie der reinen praktischen Vernunft. Das Gesetz dieser Autonomie aber ist das moralische Gesetz; welches also das Grundgesetz einer übersinnlichen Natur und einer reinen Verstandeswelt ist, deren Gegenbild in der Sinnenwelt, aber doch zugleich ohne Abbruch der Gesetze derselben, existieren soll. Man könnte jene die urbildliche (natura archetypa), die wir bloß in der Vernunft erkennen, diese aber, weil sie die mögliche Wirkung der Idee der ersteren, als Bestimmungsgrundes des Willens, enthält, die nachgebildete (natura ectypa) nennen. Denn in der Tat versetzt uns das moralische Gesetz, der Idee nach, in eine Natur, in welcher reine Vernunft, wenn sie mit dem ihr angemessenen physischen Vermögen begleitet wäre, das höchste \match{Gut} hervorbringen würde, und bestimmt unseren Willen, die Form der Sinnenwelt, als einem Ganzen vernünftiger Wesen, zu erteilen. 
	
	\subsection*{tg212.2.14} 
	\textbf{Source : }Kritik der praktischen Vernunft/Erster Teil. Elementarlehre der reinen praktischen Vernunft/Erstes Buch. Die Analytik der reinen praktischen Vernunft\\  
	
	\textbf{Paragraphe : }Diese Anmerkung, welche bloß die Methode der obersten moralischen Untersuchungen betrifft, ist von Wichtigkeit. Sie erklärt auf einmal den veranlassenden Grund aller Verirrungen der Philosophen in Ansehung des obersten Prinzips der Moral. Denn sie suchten einen Gegenstand des Willens auf, um ihn zur Materie und dem Grunde eines Gesetzes zu machen (welches alsdenn nicht unmittelbar, sondern vermittelst jenes an das Gefühl der Lust oder Unlust gebrachten Gegenstandes, der Bestimmungsgrund des Willens sein sollte, anstatt daß sie zuerst nach einem Gesetze hätten forschen sollen, das a priori und unmittelbar den Willen, und diesem gemäß allererst den Gegenstand bestimmete). Nun mochten sie diesen Gegenstand der Lust, der den obersten Begriff des Guten abgeben sollte, in der Glückseligkeit, in der Vollkommenheit, im moralischen Gesetze, oder im Willen Gottes setzen, so war ihr Grundsatz allemal Heteronomie, sie mußten unvermeidlich auf empirische Bedingungen zu einem moralischen Gesetze stoßen: weil sie ihren Gegenstand, als unmittelbaren Bestimmungsgrund des Willens, nur nach seinem unmittelbaren Verhalten zum Gefühl, welches allemal empirisch ist, gut oder böse nennen konnten. Nur ein formales Gesetz, d.i. ein solches, welches der Vernunft nichts weiter als die Form ihrer allgemeinen Gesetzgebung zur obersten Bedingung der Maximen vorschreibt, kann a priori ein Bestimmungsgrund der praktischen Vernunft sein. Die Alten verrieten indessen diesen Fehler dadurch unverhohlen, daß sie ihre moralische Untersuchung gänzlich auf die Bestimmung des Begriffs vom höchsten Gut, mithin eines Gegenstandes setzten, welchen sie nachher zum Bestimmungsgrunde des Willens im moralischen  Gesetze zu machten gedachten: ein Objekt, welches weit hinterher, wenn das moralische Gesetz allererst für sich bewährt und als unmittelbarer Bestimmungsgrund des Willens gerechtfertigt ist, dem nunmehr seiner Form nach a priori bestimmten Willen als Gegenstand vorgestellt werden kann, welches wir in der Dialektik der reinen praktischen Vernunft uns unterfangen wollen. Die Neueren, bei denen die Frage über das höchste \match{Gut} außer Gebrauch gekommen, zum wenigsten nur Nebensache geworden zu sein scheint, verstecken obigen Fehler (wie in vielen andern Fällen) hinter unbestimmten Worten, indessen, daß man ihn gleichwohl aus ihren Systemen hervorblicken sieht, da er alsdenn allenthalben Heteronomie der praktischen Vernunft verrät, daraus nimmermehr ein a priori allgemein gebietendes moralisches Gesetz entspringen kann. 
	
	\subsection*{tg217.2.5} 
	\textbf{Source : }Kritik der praktischen Vernunft/Erster Teil. Elementarlehre der reinen praktischen Vernunft/Zweites Buch. Dialektik der reinen praktischen Vernunft/Erstes Hauptstück. Von einer Dialektik der reinen praktischen Vernunft überhaupt\\  
	
	\textbf{Paragraphe : }Diese Idee praktisch –, d.i. für die Maxime unseres vernünftigen Verhaltens, hinreichend zu bestimmen, ist die 
	Weisheitslehre, und diese wiederum, als Wissenschaft, ist Philosophie, in der Bedeutung, wie die Alten das Wort verstanden, bei denen sie eine Anweisung zu dem Begriffe war, worin das höchste \match{Gut} zu setzen, und zum Verhalten, durch welches es zu erwerben sei. Es wäre gut, wenn wir dieses Wort bei seiner alten Bedeutung ließen, als eine Lehre vom höchsten Gut, so fern die Vernunft bestrebt ist, es darin zur Wissenschaft zu bringen. Denn einesteils würde die angehängte einschränkende Bedingung dem griechischen Ausdrucke (welcher Liebe zur Weisheit bedeutet) angemessen und doch zugleich hinreichend sein, die Liebe zur Wissenschaft, mithin aller spekulativen Erkenntnis der Vernunft, so fern sie ihr, sowohl zu jenem Begriffe, als auch dem praktischen Bestimmungsgrunde dienlich ist, unter dem Namen der Philosophie, mit zu befassen, und doch den Hauptzweck, um dessentwillen sie allein Weisheitslehre genannt werden kann, nicht aus den Augen verlieren lassen. Anderen Teils würde es auch nicht übel sein, den Eigendünkel desjenigen, der es wagte, sich des Titels eines Philosophen selbst anzumaßen, abzuschrecken, wenn man ihm schon durch die Definition den Maßstab der Selbstschätzung vorhielte, der seine Ansprüche sehr herabstimmen wird; denn ein Weisheitslehrer zu sein, möchte wohl etwas mehr, als einen Schüler bedeuten, der noch immer nicht weit genug gekommen ist, um sich selbst, vielweniger um andere, mit sicherer Erwartung eines so hohen Zwecks, zu leiten; es würde einen Meister in Kenntnis der Weisheit bedeuten, welches mehr sagen will, als ein bescheidener Mann sich selber anmaßen wird, und Philosophie würde, so wie die Weisheit, selbst noch immer ein Ideal bleiben, welches objektiv in der Vernunft allein vollständig vorgestellt wird, subjektiv aber, für die Person, nur das Ziel seiner unaufhörlichen Bestrebung ist, und in dessen Besitz, unter dem angemaßten Namen eines Philosophen, zu sein, nur der vorzugeben berechtigt ist, der auch die unfehlbare Wirkung derselben (in Beherrschung seiner selbst, und dem ungezweifelten Interesse, das er vorzüglich am allgemeinen Guten nimmt) an seiner Person, als Beispiele, aufstellen kann, welches die Alten auch foderten, um jenen Ehrennamen verdienen zu können. 
	
	\subsection*{tg217.2.7} 
	\textbf{Source : }Kritik der praktischen Vernunft/Erster Teil. Elementarlehre der reinen praktischen Vernunft/Zweites Buch. Dialektik der reinen praktischen Vernunft/Erstes Hauptstück. Von einer Dialektik der reinen praktischen Vernunft überhaupt\\  
	
	\textbf{Paragraphe : }Das moralische Gesetz ist der alleinige Bestimmungsgrund des reinen Willens. Da dieses aber bloß formal ist (nämlich, allein die Form der Maxime, als allgemein gesetzgebend, fodert), so abstrahiert es, als Bestimmungsgrund, von aller Materie, mithin von allem Objekte, des Wollens. Mithin mag das höchste \match{Gut} immer der ganze Gegenstand einer reinen praktischen Vernunft, d.i. eines reinen Willens sein, so ist es darum doch nicht für den Bestimmungsgrund desselben zu halten, und das moralische Gesetz muß allein als der Grund angesehen werden, jenes, und dessen Bewirkung oder Beförderung, sich zum Objekte zu machen. Diese Erinnerung ist in einem so delikaten Falle, als die Bestimmung sittlicher Prinzipien ist, wo auch die kleinste Mißdeutung Gesinnungen verfälscht, von Erheblichkeit. Denn man wird aus der Analytik ersehen haben, daß, wenn man vor dem moralischen Gesetze irgend ein Objekt, unter dem Namen eines Guten, als Bestimmungsgrund des Willens annimmt, und von ihm denn das oberste praktische Prinzip ableitet, dieses alsdenn jederzeit Heteronomie herbeibringen und das moralische Prinzip verdrängen würde. 
	
	\subsection*{tg217.2.8} 
	\textbf{Source : }Kritik der praktischen Vernunft/Erster Teil. Elementarlehre der reinen praktischen Vernunft/Zweites Buch. Dialektik der reinen praktischen Vernunft/Erstes Hauptstück. Von einer Dialektik der reinen praktischen Vernunft überhaupt\\  
	
	\textbf{Paragraphe : }Es versteht sich aber von selbst, daß, wenn im Begriffe des höchsten Guts das moralische Gesetz, als oberste Bedingung, schon mit eingeschlossen ist, alsdenn das höchste \match{Gut} nicht bloß Objekt, sondern auch sein Begriff, und die Vorstellung der durch unsere praktische Vernunft möglichen Existenz desselben zugleich der Bestimmungsgrund des reinen Willens sei; weil alsdenn in der Tat das in diesem Begriffe schon eingeschlossene und mitgedachte moralische Gesetz und kein anderer Gegenstand, nach dem Prinzip der Autonomie, den Willen bestimmt. Diese Ordnung der Begriffe von der Willensbestimmung darf nicht aus den Augen gelassen werden; weil man sonst sich selbst mißversteht und sich zu widersprechen glaubt, wo doch alles in der vollkommensten Harmonie neben einander steht. 
	
	\subsection*{tg218.2.2} 
	\textbf{Source : }Kritik der praktischen Vernunft/Erster Teil. Elementarlehre der reinen praktischen Vernunft/Zweites Buch. Dialektik der reinen praktischen Vernunft\\  
	
	\textbf{Paragraphe : }Der Begriff des Höchsten enthält schon eine Zweideutigkeit, die, wenn man darauf nicht Acht hat, unnötige Streitigkeiten veranlassen kann. Das Höchste kann das Oberste (supremum) oder auch das Vollendete (consummatum) bedeuten. Das erstere ist diejenige Bedingung, die selbst unbedingt, d.i. keiner andern untergeordnet ist (originarium); das zweite dasjenige Ganze, das kein Teil eines noch größeren Ganzen von derselben Art ist (perfectissimum). Daß Tugend (als die Würdigkeit glücklich zu sein) die oberste Bedingung alles dessen, was uns nur wünschenswert scheinen mag, mithin auch aller unserer Bewerbung um Glückseligkeit, mithin das oberste \match{Gut} sei, ist in der Analytik bewiesen worden. Darum ist sie aber noch nicht das ganze und vollendete Gut, als Gegenstand des Begehrungsvermögens vernünftiger endlicher Wesen; denn, um das zu sein, wird auch Glückseligkeit dazu erfodert, und zwar nicht bloß in den parteiischen Augen der Person, die sich selbst zum Zwecke macht, sondern selbst im Urteile einer unparteiischen Vernunft, die jene überhaupt in der Welt als Zweck an sich betrachtet. Denn der Glückseligkeit bedürftig, ihrer auch würdig, dennoch aber derselben nicht teilhaftig zu sein, kann mit dem vollkommenen Wollen eines vernünftigen Wesens, welches zugleich alle Gewalt hätte, wenn wir uns auch nur ein solches zum Versuche denken, gar nicht zusammen bestehen. So fern nun Tugend  und Glückseligkeit zusammen den Besitz des höchsten Guts in einer Person, hiebei aber auch Glückseligkeit, ganz genau in Proportion der Sittlichkeit (als Wert der Person und deren Würdigkeit glücklich zu sein) ausgeteilt, das höchste Gut einer möglichen Welt ausmachen: so bedeutet dieses das Ganze, das vollendete Gute, worin doch Tugend immer, als Bedingung, das oberste Gut ist, weil es weiter keine Bedingung über sich hat, Glückseligkeit immer etwas, was dem, der sie besitzt, zwar angenehm, aber nicht für sich allein schlechterdings und in aller Rücksicht gut ist, sondern jederzeit das moralische gesetzmäßige Verhalten als Bedingung voraussetzt. 
	
	\subsection*{tg218.2.7} 
	\textbf{Source : }Kritik der praktischen Vernunft/Erster Teil. Elementarlehre der reinen praktischen Vernunft/Zweites Buch. Dialektik der reinen praktischen Vernunft\\  
	
	\textbf{Paragraphe : }Nun ist aber aus der Analytik klar, daß die Maximen der Tugend und die der eigenen Glückseligkeit in Ansehung ihres obersten praktischen Prinzips ganz ungleichartig sind, und, weit gefehlt, einhellig zu sein, ob sie gleich zu einem höchsten Guten gehören, um das letztere möglich zu machen, einander in demselben Subjekte gar sehr einschränken und Abbruch tun. Also bleibt die Frage: wie ist das höchste \match{Gut} praktisch möglich, noch immer, unerachtet aller bisherigen Koalitionsversuche, eine unaufgelösete Aufgabe. Das aber, was sie zu einer schwer zu lösenden Aufgabe macht, ist in der Analytik gegeben, nämlich daß Glückseligkeit und Sittlichkeit zwei spezifisch ganz verschiedene Elemente des höchsten Guts sind, und ihre Verbindung also nicht analytisch erkannt werden könne (daß etwa der, so seine Glückseligkeit sucht, in diesem seinem Verhalten sich durch bloße Auflösung seiner Begriffe tugendhaft, oder der, so der Tugend folgt, sich im Bewußtsein eines solchen Verhaltens schon ipso facto glücklich finden werde), sondern eine Synthesis der Begriffe sei. Weil aber diese Verbindung als a priori, mithin praktisch notwendig, folglich nicht aus der Erfahrung abgeleitet, erkannt wird, und die Möglichkeit des höchsten Guts also auf keinen empirischen Prinzipien beruht, so wird die Deduktion dieses Begriffs transzendental sein müssen. Es ist a priori (moralisch) notwendig, das höchste Gut durch Freiheit des Willens hervorzubringen; es muß also auch die Bedingung der Möglichkeit desselben lediglich auf Erkenntnisgründen a priori beruhen. 
	
	\subsection*{tg219.2.2} 
	\textbf{Source : }Kritik der praktischen Vernunft/Erster Teil. Elementarlehre der reinen praktischen Vernunft/Zweites Buch. Dialektik der reinen praktischen Vernunft/Zweites Hauptstück. Von der Dialektik der reinen Vernunft in Bestimmung des Begriffs vom höchsten Gut/I. Die Antinomie der praktischen Vernunft\\  
	
	\textbf{Paragraphe : }In dem höchsten für uns praktischen, d.i. durch unsern Willen wirklich zu machenden, Gute werden Tugend und Glückseligkeit als notwendig verbunden gedacht, so, daß das eine durch reine praktische Vernunft nicht angenommen werden kann, ohne daß das andere auch zu ihm gehöre. Nun ist diese Verbindung (wie eine jede überhaupt) entweder analytisch, oder synthetisch. Da diese gegebene aber nicht analytisch sein kann, wie nur eben vorher gezeigt worden, so muß sie synthetisch, und zwar als Verknüpfung der Ursache mit der Wirkung gedacht werden; weil sie ein praktisches Gut, d.i. was durch Handlung möglich ist, betrifft. Es muß also entweder die Begierde nach Glückseligkeit die Bewegursache zu Maximen der Tugend, oder die Maxime der Tugend muß die wirkende Ursache der Glückseligkeit sein. Das erste ist schlechterdings unmöglich: weil (wie in der Analytik bewiesen worden) Maximen, die den Bestimmungsgrund des Willens in dem Verlangen nach seiner Glückseligkeit setzen, gar nicht moralisch sind, und keine Tugend gründen können. Das zweite ist aber auch unmöglich, weil alle praktische Verknüpfung der Ursachen und der Wirkungen in der Welt, als Erfolg der Willensbestimmung sich nicht nach moralischen Gesinnungen des Willens, sondern der Kenntnis der Naturgesetze und dem physischen Vermögen, sie zu seinen Absichten zu gebrauchen, richtet, folglich keine notwendige und zum höchsten \match{Gut} zureichende Verknüpfung der Glückseligkeit mit der Tugend in der Welt, durch die pünktlichste Beobachtung der moralischen Gesetze, erwartet werden kann. Da nun die Beförderung des höchsten Guts, welches diese Verknüpfung in seinem Begriffe enthält, ein a priori notwendiges Objekt unseres Willens ist, und mit dem moralischen Gesetze unzertrennlich zusammenhängt, so muß die Unmöglichkeit des ersteren auch die Falschheit des zweiten beweisen. Ist also das höchste Gut nach praktischen Regeln unmöglich, so muß auch das moralische Gesetz, welches gebietet, dasselbe  zu befördern, phantastisch und auf leere eingebildete Zwecke gestellt, mithin an sich falsch sein. 
	
	\subsection*{tg220.2.9} 
	\textbf{Source : }Kritik der praktischen Vernunft/Erster Teil. Elementarlehre der reinen praktischen Vernunft/Zweites Buch. Dialektik der reinen praktischen Vernunft/Zweites Hauptstück. Von der Dialektik der reinen Vernunft in Bestimmung des Begriffs vom höchsten Gut/II. Kritische Aufhebung der Antinomie der praktischen Vernunft\\  
	
	\textbf{Paragraphe : }Aus dieser Auflösung der Antinomie der praktischen reinen Vernunft folgt, daß sich in praktischen Grundsätzen eine natürliche und notwendige Verbindung zwischen dem Bewußtsein der Sittlichkeit, und der Erwartung einer ihr proportionierten Glückseligkeit, als Folge derselben, wenigstens als möglich denken (darum aber freilich noch eben nicht erkennen und einsehen) lasse; dagegen, daß Grundsätze der Bewerbung um Glückseligkeit unmöglich Sittlichkeit hervorbringen können: daß also das oberste Gut(als die erste Bedingung des höchsten Guts) Sittlichkeit, Glückseligkeit dagegen zwar das zweite Element desselben ausmache, doch so, daß diese nur die moralisch-bedingte, aber doch notwendige Folge der ersteren sei. In dieser Unterordnung allein ist das höchste \match{Gut} das ganze Objekt der reinen praktischen Vernunft, die es sich notwendig als möglich vorstellen muß, weil es ein Gebot derselben ist, zu dessen Hervorbringung alles Mögliche beizutragen. Weil aber die Möglichkeit einer solchen Verbindung des Bedingten mit seiner Bedingung gänzlich zum übersinnlichen Verhältnisse der Dinge gehört, und nach Gesetzen der Sinnenwelt gar nicht gegeben werden kann, obzwar die praktische Folge dieser Idee, nämlich die Handlungen, die darauf abzielen. das höchste Gut wirklichzumachen, zur Sinnenwelt gehören: so werden wir die Gründe jener Möglichkeit erstlich in Ansehung dessen, was unmittelbar in unserer Gewalt ist, und dann zweitens in dem, was uns Vernunft, als Ergänzung unseres Unvermögens, zur Möglichkeit des höchsten Guts (nach praktischen Prinzipien notwendig) darbietet und nicht in unserer Gewalt ist, darzustellen suchen. 
	
	\subsection*{tg223.2.10} 
	\textbf{Source : }Kritik der praktischen Vernunft/Erster Teil. Elementarlehre der reinen praktischen Vernunft/Zweites Buch. Dialektik der reinen praktischen Vernunft/Zweites Hauptstück. Von der Dialektik der reinen Vernunft in Bestimmung des Begriffs vom höchsten Gut/V. Das Dasein Gottes, als ein Postulat der reinen praktischen Vernunft\\  
	
	\textbf{Paragraphe : }Auch kann man hieraus ersehen: daß, wenn man nach dem letzten Zwecke Gottes in Schöpfung der Welt frägt, man nicht die Glückseligkeit der vernünftigen Wesen in ihr, sondern das höchste \match{Gut} nennen müsse, welches jenem Wunsche dieser Wesen noch eine Bedingung, nämlich die, der Glückseligkeit würdig zu sein, d.i. die Sittlichkeit eben derselben vernünftigen Wesen, hinzufügt, die allein den Maßstab enthält, nach welchem sie allein der ersteren, durch die Hand eines weisen Urhebers, teilhaftig zu werden hoffen können. Denn, da Weisheit, theoretisch betrachtet, die Erkenntnis des höchsten Guts, und, praktisch, die Angemessenheit des Willens zum höchsten Gute bedeutet, so kann man einer höchsten selbständigen Weisheit nicht einen Zweck beilegen, der bloß auf Gütigkeit gegründet wäre. Denn dieser ihre Wirkung (in Ansehung der Glückseligkeit der vernünftigen Wesen)  kann man nur unter den einschränkenden Bedingungen der Übereinstimmung mit der Heiligkeit
	
	
	
	14
	seines Willens, als dem höchsten ursprünglichen Gute angemessen, denken. Daher diejenige, welche den Zweck der Schöpfung in die Ehre Gottes (vorausgesetzt, daß man diese nicht anthropomorphistisch, als Neigung, gepriesen zu werden, denkt) setzten, wohl den besten Ausdruck getroffen haben. Denn nichts ehrt Gott mehr, als das, was das Schätzbarste in der Welt ist, die Achtung für sein Gebot, die Beobachtung der heiligen Pflicht, die uns sein Gesetz auferlegt, wenn seine herrliche Anstalt dazu kommt, eine solche schöne Ordnung mit angemessener Glückseligkeit zu krönen. Wenn ihn das letztere (auf menschliche Art zu reden) liebenswürdig macht, so ist er durch das erstere ein Gegenstand der Anbetung (Adoration). Selbst Menschen können sich durch Wohltun zwar Liebe, aber dadurch allein niemals Achtung erwerben, so daß die größte Wohltätigkeit ihnen nur dadurch Ehre macht, daß sie nach Würdigkeit ausgeübt wird. 
	
	\subsection*{tg223.2.3} 
	\textbf{Source : }Kritik der praktischen Vernunft/Erster Teil. Elementarlehre der reinen praktischen Vernunft/Zweites Buch. Dialektik der reinen praktischen Vernunft/Zweites Hauptstück. Von der Dialektik der reinen Vernunft in Bestimmung des Begriffs vom höchsten Gut/V. Das Dasein Gottes, als ein Postulat der reinen praktischen Vernunft\\  
	
	\textbf{Paragraphe : }
	Glückseligkeit ist der Zustand eines vernünftigen Wesens in der Welt, dem es, im Ganzen seiner Existenz, alles nach Wunsch und Willen geht, und beruhet also auf der Übereinstimmung der Natur zu sei nem ganzen Zwecke, imgleichen zum wesentlichen Bestimmungsgrunde seines Willens. Nun gebietet das moralische Gesetz, als ein Gesetz der Freiheit, durch Bestimmungsgründe, die von der Natur und der Übereinstimmung derselben zu unserem Begehrungsvermögen (als Triebfedern) ganz unabhängig sein sollen; das handelnde vernünftige Wesen in der Welt aber ist doch nicht zugleich Ursache der Welt und der Natur selbst. Also ist in dem moralischen Gesetze nicht der mindeste Grund zu einem notwendigen Zusammenhang zwischen Sittlichkeit und der ihr proportionierten Glückseligkeit eines zur Welt als Teil gehörigen, und daher von ihr abhängigen, Wesens, welches eben darum durch seinen Willen nicht Ursache dieser Natur sein, und sie, was seine Glückseligkeit betrifft, mit seinen praktischen Grundsätzen aus eigenen Kräften nicht durchgängig einstimmig machen kann. Gleichwohl wird in der praktischen Aufgabe der reinen Vernunft, d.i. der notwendigen Bearbeitung zum höchsten Gute, ein solcher Zusammenhang als notwendig postuliert: wir sollen das höchste \match{Gut} (welches also doch möglich sein muß) zu befördern suchen. Also wird auch das Dasein einer von der Natur unterschiedenen Ursache der gesamten Natur, welche den Grund dieses Zusammenhanges, nämlich der genauen Übereinstimmung der Glückseligkeit mit der Sittlichkeit, enthalte, postuliert. Diese oberste Ursache aber soll den Grund der Übereinstimmung der Natur nicht bloß mit einem Gesetze des Willens der vernünftigen Wesen, sondern  mit der Vorstellung dieses Gesetzes, so fern diese es sich zum obersten Bestimmungsgrunde des Willens setzen, also nicht bloß mit den Sitten der Form nach, sondern auch ihrer Sittlichkeit, als dem Bewegungsgrunde derselben, d.i. mit ihrer moralischen Gesinnung enthalten. Also ist das höchste Gut in der Welt nur möglich, so fern eine oberste der Natur angenommen wird, die eine der moralischen Gesinnung gemäße Kausalität hat. Nun ist ein Wesen, das der Handlungen nach der Vorstellung von Gesetzen fähig ist, eine Intelligenz (vernünftig Wesen) und die Kausalität eines solchen Wesens nach dieser Vorstellung der Gesetze ein Wille desselben. Also ist die oberste Ursache der Natur, so fern sie zum höchsten Gute vorausgesetzt werden muß, ein Wesen, das durch Verstand und Willen die Ursache (folglich der Urheber) der Natur ist, d.i. Gott. Folglich ist das Postulat der Möglichkeit des höchsten abgeleiteten Guts (der besten Welt) zugleich das Postulat der Wirklichkeit eines höchsten ursprünglichen Guts, nämlich der Existenz Gottes. Nun war es Pflicht für uns, das höchste Gut zu befördern, mithin nicht allein Befugnis, sondern auch mit der Pflicht als Bedürfnis verbundene Notwendigkeit, die Möglichkeit dieses höchsten Guts vorauszusetzen, welches, da es nur unter der Bedingung des Daseins Gottes stattfindet, die Voraussetzung desselben mit der Pflicht unzertrennlich verbindet, d.i. es ist moralisch notwendig, das Dasein Gottes anzunehmen. 
	
	\subsection*{tg223.2.5} 
	\textbf{Source : }Kritik der praktischen Vernunft/Erster Teil. Elementarlehre der reinen praktischen Vernunft/Zweites Buch. Dialektik der reinen praktischen Vernunft/Zweites Hauptstück. Von der Dialektik der reinen Vernunft in Bestimmung des Begriffs vom höchsten Gut/V. Das Dasein Gottes, als ein Postulat der reinen praktischen Vernunft\\  
	
	\textbf{Paragraphe : }Aus dieser Deduktion wird es nunmehr begreiflich, warum die griechischen Schulen zur Auflösung ihres Problems von der praktischen Möglichkeit des höchsten Guts niemals gelangen konnten; weil sie nur immer die Regel des Gebrauchs, den der Wille des Menschen von seiner Freiheit macht, zum einzigen und für sich allein zureichenden Grunde derselben machten, ohne, ihrem Bedünken nach, das Dasein Gottes dazu zu bedürfen. Zwar taten sie daran recht, daß sie das Prinzip der Sitten unabhängig von diesem Postulat, für sich selbst, aus dem Verhältnis der Vernunft allein zum Willen, festsetzten, und es mithin zur obersten praktischen Bedingung des höchsten Guts machten; es war aber darum nicht die ganze Bedingung der Möglichkeit desselben. Die Epikureer hatten nun zwar ein ganz falsches Prinzip der Sitten zum obersten angenommen, nämlich das der Glückseligkeit, und eine Maxime der beliebigen Wahl, nach jedes seiner Neigung, für ein Gesetz untergeschoben; aber darin verfuhren sie doch konsequent genug, daß sie ihr höchstes \match{Gut} eben so, nämlich der Niedrigkeit ihres Grundsatzes proportionierlich, abwürdigten, und keine größere Glückseligkeit erwarteten, als die sich durch menschliche Klugheit (wozu auch Enthaltsamkeit und Mäßigung der Neigungen gehört) erwerben läßt, die, wie man weiß, kümmerlich genug, und nach Umständen sehr verschiedentlich, ausfallen muß; die Ausnahmen, welche ihre  Maximen unaufhörlich einräumen mußten, und die sie zu Gesetzen untauglich machen, nicht einmal gerechnet. Die Stoiker hatten dagegen ihr oberstes praktisches Prinzip, nämlich die Tugend, als Bedingung des höchsten Guts ganz richtig gewählt, aber, indem sie den Grad derselben, der für das reine Gesetz derselben erforderlich ist, als in diesem Leben völlig erreichbar vorstelleten, nicht allein das moralische Vermögen des Menschen, unter dem Namen eines Weisen, über alle Schranken seiner Natur hoch gespannt, und etwas, das aller Menschenkenntnis widerspricht, angenommen, sondern auch vornehmlich das zweite zum höchsten Gut gehörige Bestandstück, nämlich die Glückseligkeit, gar nicht für einen besonderen Gegenstand des menschlichen Begehrungsvermögens wollen gelten lassen, sondern ihren Weisen, gleich einer Gottheit, im Bewußtsein der Vortrefflichkeit seiner Person, von der Natur (in Absicht auf seine Zufriedenheit) ganz unabhängig gemacht, indem sie ihn zwar Übeln des Lebens aussetzten, aber nicht unterwarfen (zugleich auch als frei vom Bösen darstelleten), und so wirklich das zweite Element des höchsten Guts, eigene Glückseligkeit wegließen, indem sie es bloß im Handeln und der Zufriedenheit mit seinem persönlichen Werte setzten, und also im Bewußtsein der sittlichen Denkungsart mit einschlossen, worin sie aber durch die Stimme ihrer eigenen Natur hinreichend hätten widerlegt werden können. 
	
	\subsection*{tg223.2.6} 
	\textbf{Source : }Kritik der praktischen Vernunft/Erster Teil. Elementarlehre der reinen praktischen Vernunft/Zweites Buch. Dialektik der reinen praktischen Vernunft/Zweites Hauptstück. Von der Dialektik der reinen Vernunft in Bestimmung des Begriffs vom höchsten Gut/V. Das Dasein Gottes, als ein Postulat der reinen praktischen Vernunft\\  
	
	\textbf{Paragraphe : }Die Lehre des Christentums
	
	
	13
	, wenn man sie auch noch nicht als Religionslehre betrachtet, gibt in diesem Stücke  einen Begriff des höchsten Guts (des Reichs Gottes), der allein der strengsten Foderung der praktischen Vernunft ein Gnüge tut. Das moralische Gesetz ist heilig (unnachsichtlich) und fodert Heiligkeit der Sitten, obgleich alle moralische Vollkommenheit, zu welcher der Mensch gelangen kann, immer nur Tugend ist, d.i. gesetzmäßige Gesinnung aus Achtung fürs Gesetz, folglich Bewußtsein eines kontinuierlichen Hanges zur Übertretung, wenigstens Unlauterkeit, d.i. Beimischung vieler unechter (nicht moralischer) Bewegungsgründe zur Befolgung des Gesetzes, folglich eine mit Demut verbundene Selbstschätzung, und also in Ansehung der Heiligkeit, welche das christliche Gesetz fodert, nichts als Fortschritt ins Unendliche dem Geschöpfe  übrig läßt, eben daher aber auch dasselbe zur Hoffnung seiner ins Unendliche gehenden Fortdauer berechtigt. Der Wert einer dem moralischen Gesetze völlig angemessenen Gesinnung ist unendlich; weil alle mögliche Glückseligkeit, im Urteile eines weisen und alles vermögenden Austeilers derselben, keine andere Einschränkung hat, als den Mangel der Angemessenheit vernünftiger Wesen an ihrer Pflicht. Aber das moralische Gesetz für sich verheißt doch keine Glückseligkeit; denn diese ist, nach Begriffen von einer Naturordnung überhaupt, mit der Befolgung desselben nicht notwendig verbunden. Die christliche Sittenlehre ergänzt nun diesen Mangel (des zweiten unentbehrlichen Bestandstücks des höchsten Guts) durch die Darstellung der Welt, darin vernünftige Wesen sich dem sittlichen Gesetze von ganzer Seele weihen, als eines Reichs Gottes, in welchem Natur und Sitten in eine, jeder von beiden für sich selbst fremde, Harmonie, durch einen heiligen Urheber kommen, der das abgeleitete höchste \match{Gut} möglich macht. Die Heiligkeit der Sitten wird ihnen in diesem Leben schon zur Richtschnur angewiesen, das dieser proportionierte Wohl aber, die Seligkeit, nur als in einer Ewigkeit erreichbar vorgestellt; weil jene immer das Urbild ihres Verhaltens in jedem Stande sein muß, und das Fortschreiten zu ihr schon in diesem Leben möglich und notwendig ist, diese aber in dieser Welt, unter dem Namen der Glückseligkeit, gar nicht erreicht werden kann (so viel auf unser Vermögen ankommt), und daher lediglich zum Gegenstande der Hoffnung gemacht wird. Diesem ungeachtet ist das christliche Prinzip der Moral selbst doch nicht theologisch (mithin Heteronomie), sondern Autonomie der reinen praktischen Vernunft für sich selbst, weil sie die Erkenntnis Gottes und seines Willens nicht zum Grunde dieser Gesetze, sondern nur der Gelangung zum höchsten Gute, unter der Bedingung der Befolgung derselben macht, und selbst die eigentliche Triebfeder zu Befolgung der ersteren nicht in den gewünschten Folgen derselben, sondern in der Vorstellung der Pflicht allein setzt, als in deren treuer Beobachtung die Würdigkeit des Erwerbs der letztern allein besteht. 
	
	\subsection*{tg223.2.7} 
	\textbf{Source : }Kritik der praktischen Vernunft/Erster Teil. Elementarlehre der reinen praktischen Vernunft/Zweites Buch. Dialektik der reinen praktischen Vernunft/Zweites Hauptstück. Von der Dialektik der reinen Vernunft in Bestimmung des Begriffs vom höchsten Gut/V. Das Dasein Gottes, als ein Postulat der reinen praktischen Vernunft\\  
	
	\textbf{Paragraphe : }
	Auf solche Weise führt das moralische Gesetz durch den Begriff des höchsten Guts, als das Objekt und den Endzweck der reinen praktischen Vernunft, zur Religion, d.i. zur Erkenntnis aller Pflichten als göttlicher Gebote, nicht als Sanktionen, d.i. willkürliche für sich selbst zufällige Verordnungen, eines fremden Willens, sondern als wesentlicher Gesetze eines jeden freien Willens für sich selbst, die aber dennoch als Gebote des höchsten Wesens angesehen werden müssen, weil wir nur von einem moralisch-vollkommenen (heiligen und gütigen), zugleich auch allgewaltigen Willen das höchste Gut, welches zum Gegenstande unserer Bestrebung zu setzen uns das moralische Gesetz zur Pflicht macht, und also durch Übereinstimmung mit diesem Willen dazu zu gelangen hoffen können. Auch hier bleibt daher alles uneigennützig und bloß auf Pflicht gegründet; ohne daß Furcht oder Hoffnung als Triebfedern zum Grunde gelegt werden dürften, die, wenn sie zu Prinzipien werden, den ganzen moralischen Wert der Handlungen vernichten. Das moralische Gesetz gebietet, das höchste mögliche \match{Gut} in einer Welt mir zum letzten Gegenstande alles Verhaltens zu machen. Dieses aber kann ich nicht zu bewirken hoffen, als nur durch die Übereinstimmung meines Willens mit dem eines heiligen und gütigen Welturhebers, und, obgleich in dem Begriffe des höchsten Guts, als dem eines Ganzen, worin die größte Glückseligkeit mit dem größten Maße sittlicher (in Geschöpfen möglicher) Vollkommenheit, als in der genausten Proportion verbunden vorgestellt wird, meine eigene Glückseligkeit mit enthalten ist: so ist doch nicht sie, sondern das moralische Gesetz (welches vielmehr mein unbegrenztes Verlangen darnach auf Bedingungen strenge einschränkt) der Bestimmungsgrund des Willens, der zur Beförderung des höchsten Guts angewiesen wird. 
	
	\subsection*{tg223.2.9} 
	\textbf{Source : }Kritik der praktischen Vernunft/Erster Teil. Elementarlehre der reinen praktischen Vernunft/Zweites Buch. Dialektik der reinen praktischen Vernunft/Zweites Hauptstück. Von der Dialektik der reinen Vernunft in Bestimmung des Begriffs vom höchsten Gut/V. Das Dasein Gottes, als ein Postulat der reinen praktischen Vernunft\\  
	
	\textbf{Paragraphe : }
	
	Würdig ist jemand des Besitzes einer Sache, oder eines Zustandes, wenn, daß er in diesem Besitze sei, mit dem höchsten Gute zusammenstimmt. Man kann jetzt leicht einsehen, daß alle Würdigkeit auf das sittliche Verhalten ankomme, weil dieses im Begriffe des höchsten Guts die Bedingung des übrigen (was zum Zustande gehört), nämlich des Anteils an Glückseligkeit ausmacht. Nun folgt hieraus: daß man die Moral an sich niemals als Glückseligkeitslehre behandeln müsse, d.i. als eine Anweisung, der Glückseligkeit teilhaftig zu werden; denn sie hat es lediglich mit der Vernunftbedingung (conditio sine qua non) der letzteren, nicht mit einem Erwerbmittel derselben zu tun. Wenn sie aber (die bloß Pflichten auferlegt, nicht eigennützigen Wünschen Maßregeln an die Hand gibt) vollständig vorgetragen worden: alsdenn allererst kann, nachdem der sich auf ein Gesetz gründende moralische Wunsch, das höchste \match{Gut} zu befördern (das Reich Gottes zu uns zu bringen), der vorher keiner eigennützigen Seele aufsteigen konnte, erweckt, und ihm zum Behuf der Schritt zur Religion geschehen ist, diese Sittenlehre auch Glückseligkeitslehre genannt werden, weil die Hoffnung dazu nur mit der Religion allererst anhebt. 
	
	\subsection*{tg224.2.3} 
	\textbf{Source : }Kritik der praktischen Vernunft/Erster Teil. Elementarlehre der reinen praktischen Vernunft/Zweites Buch. Dialektik der reinen praktischen Vernunft/Zweites Hauptstück. Von der Dialektik der reinen Vernunft in Bestimmung des Begriffs vom höchsten Gut/VI. Über die Postulate der reinen praktischen Vernunft überhaupt\\  
	
	\textbf{Paragraphe : }Diese Postulate sind die der Unsterblichkeit, der Freiheit, positiv betrachtet (als der Kausalität eines Wesens, so fern es zur intelligibelen Welt gehört), und des Daseins Gottes. Das erste fließt aus der praktisch notwendigen Bedingung der Angemessenheit der Dauer zur Vollständigkeit der Erfüllung des moralischen Gesetzes; das zweite aus der notwendigen Voraussetzung der Unabhängigkeit von der Sinnenwelt und des Vermögens der Bestimmung seines Willens, nach dem Gesetze einer intelligibelen Welt, d.i. der Freiheit; das dritte aus der Notwendigkeit der Bedingung zu einer solchen intelligibelen Welt, um das höchste \match{Gut} zu sein, durch die Voraussetzung des höchsten selbständigen Guts, d.i. des Daseins Gottes. 
	
	\subsection*{tg226.2.2} 
	\textbf{Source : }Kritik der praktischen Vernunft/Erster Teil. Elementarlehre der reinen praktischen Vernunft/Zweites Buch. Dialektik der reinen praktischen Vernunft/Zweites Hauptstück. Von der Dialektik der reinen Vernunft in Bestimmung des Begriffs vom höchsten Gut/VIII. Vom Fürwahrhalten aus einem Bedürfnisse der reinen Vernunft\\  
	
	\textbf{Paragraphe : }Ein Bedürfnis der reinen Vernunft in ihrem spekulativen Gebrauche führt nur auf Hypothesen, das der reinen praktischen Vernunft aber zu Postulaten; denn im ersteren Falle steige ich vom Abgeleiteten so hoch hinauf in der Reihe der Gründe, wie ich will, und bedarf eines Ungrundes, nicht um jenem Abgeleiteten (z.B. der Kausalverbindung der Dinge und Veränderungen in der Welt) objektive Realität zu geben, sondern nur um meine forschende Vernunft in Ansehung desselben vollständig zu befriedigen. So sehe ich Ordnung und Zweckmäßigkeit in der Natur vor mir, und bedarf nicht, um mich von deren Wirklichkeit zu versichern, zur Spekulation zu schreiten, sondern nur, um sie zu erklären, eine Gottheit, als deren Ursache, voraus zu setzen; da denn, weil von einer Wirkung der Schluß auf eine bestimmte, vornehmlich so genau und so vollständig bestimmte Ursache, als wir an Gott zu denken haben, immer unsicher und mißlich ist, eine solche Voraussetzung nicht weitergebracht werden kann, als zu dem Grade der, für uns Menschen, allervernünftigsten Meinung.
	
	
	16
	Dagegen ist ein Bedürfnis der reinen praktischen Vernunft, auf einer Pflicht gegründet, etwas (das höchste Gut) zum Gegenstande meines Willens zu machen, um es nach allen meinen Kräften zu befördern; wobei ich aber die Möglichkeit desselben, mithin auch die Bedingungen dazu, nämlich Gott, Freiheit und Unsterblichkeit voraussetzen muß, weil  ich diese durch meine spekulative Vernunft nicht beweisen, obgleich auch nicht widerlegen kann. Diese Pflicht gründet sich auf einem, freilich von diesen letzteren Voraussetzungen ganz unabhängigen, für sich selbst apodiktisch gewissen, nämlich dem moralischen, Gesetze, und ist, so fern, keiner anderweitigen Unterstützung durch theoretische Meinung von der innern Beschaffenheit der Dinge, der geheimen Abzweckung der Weltordnung, oder eines ihr vorstehenden Regierers, bedürftig, um uns auf das vollkommenste zu unbedingt-gesetzmäßigen Handlungen zu verbinden. Aber der subjektive Effekt dieses Gesetzes, nämlich die ihm angemessene und durch dasselbe auch notwendige Gesinnung, das praktisch mögliche höchste \match{Gut} zu befördern, setzt doch wenigstens voraus, daß das letztere möglich sei, widrigenfalls es praktisch-unmöglich wäre, dem Objekte eines Begriffes nachzustreben, welcher im Grunde leer und ohne Objekt wäre. Nun betreffen obige Postulate nur die physische oder metaphysische, mit einem Worte, in der Natur der Dinge liegende Bedingungen der Möglichkeit des höchsten Guts, aber nicht zum Behuf einer beliebigen spekulativen Absicht, sondern eines praktisch notwendigen Zwecks des reinen Vernunftwillens, der hier nicht wählt, sondern einem unnachlaßlichen Vernunftgebote gehorcht, welches seinen Grund, objektiv, in der Beschaffenheit der Dinge hat, so wie sie durch reine Vernunft allgemein beurteilt werden müssen, und gründet sich nicht etwa auf Neigung, die zum Behuf dessen, was wir aus bloß subjektiven Gründen wünschen, so fort die Mittel dazu als möglich, oder den Gegenstand wohl gar als wirklich, anzunehmen keinesweges berechtigt ist. Also ist dieses ein Bedürfnis in schlechterdings notwendiger Absicht, und rechtfertigt seine Voraussetzung nicht bloß als erlaubte Hypothese, sondern als Postulat in praktischer Absicht; und, zugestanden, daß das reine moralische Gesetz jedermann, als Gebot (nicht als Klugheitsregel), unnachlaßlich verbinde, darf der Rechtschaffene wohl sagen: ich will, daß ein Gott, daß mein Dasein in dieser Welt, auch außer der Naturverknüpfung, noch ein Dasein in einer reinen Verstandeswelt,  endlich auch daß meine Dauer endlos sei, ich beharre darauf und lasse mir diesen Glauben nicht nehmen; denn dieses ist das einzige, wo mein Interesse, weil ich von demselben nichts nachlassen darf, mein Urteil unvermeidlich bestimmt, ohne auf Vernünfteleien zu achten, so wenig ich auch darauf zu antworten oder ihnen scheinbarere entgegen zu stellen im Stande sein möchte.
	
	
	17
	
	
	
	\subsection*{tg226.2.6} 
	\textbf{Source : }Kritik der praktischen Vernunft/Erster Teil. Elementarlehre der reinen praktischen Vernunft/Zweites Buch. Dialektik der reinen praktischen Vernunft/Zweites Hauptstück. Von der Dialektik der reinen Vernunft in Bestimmung des Begriffs vom höchsten Gut/VIII. Vom Fürwahrhalten aus einem Bedürfnisse der reinen Vernunft\\  
	
	\textbf{Paragraphe : }Um bei dem Gebrauche eines noch so ungewohnten Begriffs als der eines reinen praktischen Vernunftglaubens ist, Mißdeutungen zu verhüten, sei mir erlaubt, noch eine Anmerkung hinzuzufügen. – Es sollte fast scheinen, als ob dieser Vernunftglaube hier selbst als Gebot angekündigt werde, nämlich das höchste \match{Gut} für möglich anzunehmen. Ein Glaube aber, der geboten wird, ist ein Unding. Man erinnere sich aber der obigen Auseinandersetzung dessen, was im Begriffe des höchsten Guts anzunehmen verlangt wird, und wird man inne werden, daß diese Möglichkeit anzunehmen gar nicht geboten werden dürfe, und keine praktische Gesinnungen fodere, sie einzuräumen, sondern daß spekulative Vernunft sie ohne Gesuch zugeben müsse; denn daß eine, dem moralischen Gesetze angemessene, Würdigkeit der vernünftigen Wesen in der Welt, glücklich zu sein, mit einem dieser proportionierten Besitze dieser Glückseligkeit in Verbindung, an sich unmöglich sei, kann doch niemand behaupten wollen. Nun gibt uns in Ansehung des ersten Stücks des höchsten Guts, nämlich was die Sittlichkeit betrifft, das moralische Gesetz bloß ein Gebot, und, die Möglichkeit jenes Bestandstücks zu bezweifeln, wäre eben so viel, als das moralische Gesetz selbst in Zweifel ziehen. Was aber das zweite Stück jenes Objekts, nämlich die jener Würdigkeit durchgängig angemessene Glückseligkeit, betrifft, so ist zwar die Möglichkeit derselben überhaupt einzuräumen gar nicht eines Gebots bedürftig, denn die theoretische Vernunft hat selbst nichts dawider: nur die Art, wie wir uns eine solche Harmonie der Naturgesetze mit denen der Freiheit denken sollen, hat etwas an sich, in Ansehung dessen uns eine Wahl zukommt, weil theoretische Vernunft hierüber nichts mit apodiktischer Gewißheit entscheidet, und, in Ansehung dieser, kann es ein moralisches Interesse geben, das den Ausschlag gibt. 
	
	\subsection*{tg226.2.8} 
	\textbf{Source : }Kritik der praktischen Vernunft/Erster Teil. Elementarlehre der reinen praktischen Vernunft/Zweites Buch. Dialektik der reinen praktischen Vernunft/Zweites Hauptstück. Von der Dialektik der reinen Vernunft in Bestimmung des Begriffs vom höchsten Gut/VIII. Vom Fürwahrhalten aus einem Bedürfnisse der reinen Vernunft\\  
	
	\textbf{Paragraphe : }Allein jetzt kommt ein Entscheidungsgrund von anderer Art ins Spiel, um im Schwanken der spekulativen Vernunft den Ausschlag zu geben. Das Gebot, das höchste \match{Gut} zu befördern, ist objektiv (in der praktischen Vernunft), die Möglichkeit desselben überhaupt gleichfalls objektiv (in der theoretischen Vernunft, die nichts dawider hat) gegründet. Allein die Art, wie wir uns diese Möglichkeit vorstellen sollen, ob nach allgemeinen Naturgesetzen, ohne einen der Natur vorstehenden weisen Urheber, oder nur unter dessen Voraussetzung, das kann die Vernunft objektiv nicht entscheiden. Hier tritt nun eine subjektive Bedingung der Vernunft ein: die einzige ihr theoretisch mögliche, zugleich der Moralität (die unter einem objektiven Gesetze der Vernunft steht) allein zuträgliche Art, sich die genaue Zusammenstimmung des Reichs der Natur mit dem Reiche der Sitten, als Bedingung der Möglichkeit des höchsten Guts, zu denken. Da nun die Beförderung desselben, und also die Voraussetzung seiner Möglichkeit, objektiv (aber nur der praktischen Vernunft zu Folge) notwendig ist, zugleich aber die Art, auf welche Weise wir es uns als möglich denken wollen, in unserer Wahl steht, in welcher aber ein freies Interesse der reinen praktischen Vernunft für die Annehmung eines weisen Welturhebers entscheidet: so ist das Prinzip, was unser Urteil hierin bestimmt, zwar subjektiv, als Bedürfnis, aber auch zugleich als Beförderungsmittel dessen, was objektiv (praktisch) notwendig ist, der Grund einer Maxime des Fürwahrhaltens in moralischer Absicht, d.i. ein reiner praktischer Vernunftglaube. Dieser ist also nicht geboten, sondern, als freiwillige, zur moralischen (gebotenen) Absicht zuträgliche, überdem noch mit dem theoretischen Bedürfnisse der Vernunft einstimmige Bestimmung unseres Urteils, jene Existenz anzunehmen  und dem Vernunftgebrauch ferner zum Grunde zu legen, selbst aus der moralischen Gesinnung entsprungen; kann also öfters selbst bei Wohlgesinneten bisweilen in Schwanken niemals aber in Unglauben geraten. 
	
	\subsection*{tg228.2.10} 
	\textbf{Source : }Kritik der praktischen Vernunft/Zweiter Teil. Methodenlehre der reinen praktischen Vernunft\\  
	
	\textbf{Paragraphe : }Alle Gefühle, vornehmlich die, so ungewohnte Anstrengung bewirken sollen, müssen in dem Augenblicke, da sie in ihrer Heftigkeit sind, und ehe sie verbrausen, ihre Wirkung tun, sonst tun sie nichts; indem das Herz natürlicherweise zu seiner natürlichen gemäßigten Lebensbewegung zurückkehrt, und sonach in die Mattigkeit verfällt, die ihm vorher eigen war; weil zwar etwas, was es reizte, nichts aber, das es stärkte, an dasselbe gebracht war. Grundsätze müssen auf Begriffe errichtet werden, auf alle andere Grundlage können nur Anwandelungen zu Stande kommen, die der Person keinen moralischen Wert, ja nicht einmal eine Zuversicht auf sich selbst verschaffen können, ohne die das Bewußtsein seiner moralischen Gesinnung und eines solchen Charakters, das höchste \match{Gut} im Menschen, gar nicht stattfinden  kann. Diese Begriffe nun, wenn sie subjektiv praktisch werden sollen, müssen nicht bei den objektiven Gesetzen der Sittlichkeit stehen bleiben, um sie zu bewundern, und in Beziehung auf die Menschheit hochzuschätzen, sondern ihre Vorstellung in Relation auf den Menschen und auf sein Individuum betrachten; da denn jenes Gesetz in einer zwar höchst achtungswürdigen, aber nicht so gefälligen Gestalt erscheint, als ob es zu dem Elemente gehöre, daran er natürlicher Weise gewohnt ist, sondern wie es ihn nötiget, dieses oft, nicht ohne Selbstverleugnung, zu verlassen, und sich in ein höheres zu begeben, darin er sich, mit unaufhörlicher Besorgnis des Rückfalls, nur mit Mühe erhalten kann. Mit einem Worte, das moralische Gesetz verlangt Befolgung aus Pflicht, nicht aus Vorliebe, die man gar nicht voraussetzen kann und soll. 
	
	\subsection*{tg230.2.34} 
	\textbf{Source : }Kritik der praktischen Vernunft/Fußnoten\\  
	
	\textbf{Paragraphe : }
	
	17 Im deutschen Museum, Febr. 1787, findet sich eine Abhandlung von einem sehr feinen und hellen Kopfe, dem sel. Wizenmann, dessen früher Tod zu bedauren ist, darin er die Befugnis, aus einem Bedürfnisse auf die objektive Realität des Gegenstandes desselben zu schließen, bestreitet, und seinen Gegenstand durch das Beispiel eines Verliebten erläutert, der, indem er sich in eine Idee von Schönheit, welche bloß sein Hirngespinst ist, vernarrt hätte, schließen wollte, daß ein solches Objekt wirklich wo existiere. Ich gebe ihm hierin vollkommen recht, in allen Fällen, wo das Bedürfnis auf Neigung gegründet ist, die nicht einmal notwendig für den, der damit angefochten ist, die Existenz ihres Objekts postulieren kann, vielweniger eine für jedermann gültige Foderung enthält, und daher ein bloß subjektiver Grund der Wünsche ist. Hier aber ist es ein Vernunftbedürfnis, aus einem objektiven Bestimmungsgrunde des Willens, nämlich dem moralischen Gesetze entspringend, welches jedes vernünftige Wesen notwendig verbindet, also zur Voraussetzung der ihm angemessenen Bedingungen in der Natur a priori berechtigt, und die letztern von dem vollständigen praktischen Gebrauche der Vernunft unzertrennlich macht. Es ist Pflicht, das höchste \match{Gut} nach unserem größten Vermögen wirklichzumachen; daher muß es doch auch möglich sein; mithin ist es für jedes vernünftige Wesen in der Welt auch unvermeidlich, dasjenige vorauszusetzen, was zu dessen objektiver Möglichkeit notwendig ist. Die Voraussetzung ist so notwendig, als das moralische Gesetz, in Beziehung auf welches sie auch nur gültig ist. 
	
	\unnumberedsection{Kern (1)} 
	\subsection*{tg214.2.20} 
	\textbf{Source : }Kritik der praktischen Vernunft/Erster Teil. Elementarlehre der reinen praktischen Vernunft/Erstes Buch. Die Analytik der reinen praktischen Vernunft\\  
	
	\textbf{Paragraphe : }Hiemit stimmt aber die Möglichkeit eines solchen Gebots, als: Liebe Gott über alles und deinen Nächsten als dich selbst,
	
	
	11
	ganz wohl zusammen. Denn es fodert doch, als Gebot, Achtung für ein Gesetz, das Liebe befiehlt, und überläßt es nicht der beliebigen Wahl, sich diese zum Prinzip zu machen. Aber Liebe zu Gott als Neigung (pathologische Liebe) ist unmöglich; denn er ist kein Gegenstand der Sinne. Eben dieselbe gegen Menschen ist zwar möglich, kann aber nicht geboten werden; denn es steht in keines Menschen Vermögen, jemanden bloß auf Befehl zu lieben. Also ist es bloß die praktische Liebe, die in jenem \match{Kern} aller Gesetze verstanden wird. Gott lieben, heißt in dieser Bedeutung, seine Gebote gerne tun; den Nächsten lieben, heißt, alle Pflicht gegen ihn gerne ausüben. Das Gebot aber, das dieses zur Regel macht, kann auch nicht diese Gesinnung in pflichtmäßigen Handlungen zu haben, sondern bloß darnach zu streben gebieten. Denn ein Gebot, daß man etwas gerne tun soll, ist in sich widersprechend, weil, wenn wir, was uns zu tun obliege, schon von selbst wissen, wenn wir uns überdem auch bewußt wären, es gerne zu tun, ein Gebot darüber ganz unnötig, und, tun wir es zwar, aber eben nicht gerne, sondern nur aus Achtung fürs Gesetz, ein Gebot, welches diese Achtung eben zur Triebfeder der Maxime macht, gerade der gebotenen Gesinnung zuwider wirken würde. Jenes Gesetz aller Gesetze stellt also, wie alle moralische Vorschrift des Evangelii, die sittliche  Gesinnung in ihrer ganzen Vollkommenheit dar, so wie sie als ein Ideal der Heiligkeit von keinem Geschöpfe erreichbar, dennoch das Urbild ist, welchem wir uns zu näheren, und, in einem ununterbrochenen, aber unendlichen Progressus, gleich zu werden streben sollen. Könnte nämlich ein vernünftig Geschöpf jemals dahin kommen, alle moralische Gesetze völlig gerne zu tun, so würde das so viel bedeuten, als, es fände sich in ihm auch nicht einmal die Möglichkeit einer Begierde, die ihn zur Abweichung von ihnen reizte; denn die Überwindung einer solchen kostet dem Subjekt immer Aufopferung, bedarf also Selbstzwang, d.i. innere Nötigung zu dem, was man nicht ganz gern tut. Zu dieser Stufe der moralischen Gesinnung aber kann es ein Geschöpf niemals bringen. Denn da es ein Geschöpf, mithin in Ansehung dessen, was er zur gänzlichen Zufriedenheit mit seinem Zustande fodert, immer abhängig ist, so kann es niemals von Begierden und Neigungen ganz frei sein, die, weil sie auf physischen Ursachen beruhen, mit dem moralischen Gesetze, das ganz andere Quellen hat, nicht von selbst stimmen, mithin es jederzeit notwendig machen, in Rücksicht auf dieselbe, die Gesinnung seiner Maximen auf moralische Nötigung, nicht auf bereitwillige Ergebenheit, sondern auf Achtung, welche die Befolgung des Gesetzes, obgleich sie ungerne geschähe, fodert, nicht auf Liebe, die keine innere Weigerung des Willens gegen das Gesetz besorgt, zu gründen, gleichwohl aber diese letztere, nämlich die bloße Liebe zum Gesetze (da es alsdenn aufhören würde, Gebot zu sein, und Moralität, die nun subjektiv in Heiligkeit überginge, aufhören würde, Tugend zu sein) sich zum beständigen, obgleich unerreichbaren Ziele seiner Bestrebung zu machen. Denn an dem, was wir hochschätzen, aber doch (wegen des Bewußtseins unserer Schwächen) scheuen, verwandelt sich, durch die mehrere Leichtigkeit, ihm Gnüge zu tun, die ehrfurchtsvolle Scheu in Zuneigung, und Achtung in Liebe, wenigstens würde es die Vollendung einer dem Gesetze gewidmeten Gesinnung sein, wenn es jemals einem Geschöpfe möglich wäre, sie zu erreichen. 
	
	\unnumberedsection{Lage (1)} 
	\subsection*{tg197.2.12} 
	\textbf{Source : }Kritik der praktischen Vernunft/Vorrede\\  
	
	\textbf{Paragraphe : }Wenn es um die Bestimmung eines besonderen Vermögens der menschlichen Seele, nach seinen Quellen, Inhalte und Grenzen zu tun ist, so kann man zwar, nach der Natur  des menschlichen Erkenntnisses, nicht anders als von den Teilen derselben, ihrer genauen und (so viel als nach der jetzigen \match{Lage} unserer schon erworbenen Elemente derselben möglich ist) vollständigen Darstellung anfangen. Aber es ist noch eine zweite Aufmerksamkeit, die mehr philosophisch und architektonisch ist; nämlich, die Idee des Ganzen richtig zu fassen, und aus derselben alle jene Teile in ihrer wechselseitigen Beziehung auf einander, vermittelst der Ableitung derselben von dem Begriffe jenes Ganzen, in einem reinen Vernunftvermögen ins Auge zu fassen. Diese Prüfung und Gewährleistung ist nur durch die innigste Bekanntschaft mit dem System möglich, und die, welche in Ansehung der ersteren Nachforschung verdrossen gewesen, also diese Bekanntschaft zu erwerben nicht der Mühe wert geachtet haben, gelangen nicht zur zweiten Stufe, nämlich der Übersicht, welche eine synthetische Wiederkehr zu demjenigen ist, was vorher analytisch gegeben worden, und es ist kein Wunder, wenn sie allerwärts Inkonsequenzen finden, obgleich die Lücken, die diese vermuten lassen, nicht im System selbst, sondern bloß in ihrem eigenen unzusammenhängenden Gedankengange anzutreffen sind. 
	
	\unnumberedsection{Trieb (1)} 
	\subsection*{tg230.2.32} 
	\textbf{Source : }Kritik der praktischen Vernunft/Fußnoten\\  
	
	\textbf{Paragraphe : }
	
	16 Aber selbst auch hier würden wir nicht ein Bedürfnis der Vernunft vorschützen können, läge nicht ein problematischer, aber doch unvermeidlicher Begriff der Vernunft vor Augen, nämlich der eines schlechterdings notwendigen Wesens. Dieser Begriff will nun bestimmt sein, und das ist, wenn der \match{Trieb} zur Erweiterung dazu kommt, der objektive Grund eines Bedürfnisses der spekulativen Vernunft, nämlich den Begriff eines notwendigen Wesens, welches andern zum Urgrunde dienen soll, näher zu bestimmen, und dieses letzte also wodurch kenntlich zu machen. Ohne solche vorausgehende notwendige Probleme gibt es keine Bedürfnisse, wenigstens nicht der reinen Vernunft; die übrigen sind Bedürfnisse der Neigung. 
	
	\unnumberedchapter{Alimentation} 
	\unnumberedsection{Brot (1)} 
	\subsection*{tg204.2.12} 
	\textbf{Source : }Kritik der praktischen Vernunft/Erster Teil. Elementarlehre der reinen praktischen Vernunft/Erstes Buch. Die Analytik der reinen praktischen Vernunft/Erstes Hauptstück. Von den Grundsätzen der reinen praktischen Vernunft/3. Lehrsatz II\\  
	
	\textbf{Paragraphe : }Glücklich zu sein, ist notwendig das Verlangen jedes vernünftigen aber endlichen Wesens, und also ein unvermeidlicher Bestimmungsgrund seines Begehrungsvermögens. Denn die Zufriedenheit mit seinem ganzen Dasein ist nicht etwa ein ursprünglicher Besitz, und eine Seligkeit, welche ein Bewußtsein seiner unabhängigen Selbstgenugsamkeit voraussetzen würde, sondern ein durch seine endliche Natur selbst ihm aufgedrungenes Problem, weil es bedürftig ist, und dieses Bedürfnis betrifft die Materie seines Begehrungsvermögens, d.i. etwas, was sich auf ein subjektiv zum Grunde liegendes Gefühl der Lust oder Unlust bezieht, dadurch das, was es zur Zufriedenheit mit seinem Zustande bedarf, bestimmt wird. Aber eben darum, weil dieser materiale Bestimmungsgrund von dem Subjekte bloß empirisch erkannt werden kann, ist es unmöglich, diese Aufgabe als ein Gesetz zu betrachten, weil dieses als objektiv in allen Fällen und für alle vernünftige Wesen eben denselben Bestimmungsgrund des Willens enthalten müßte. Denn obgleich der Begriff der Glückseligkeit der praktischen Beziehung der Objekte aufs Begehrungsvermögen allerwärts zum Grunde liegt, so ist er doch nur der allgemeine Titel der subjektiven Bestimmungsgründe, und bestimmt nichts spezifisch, darum es doch in dieser praktischen Aufgabe allein zu tun ist, und ohne welche Bestimmung sie gar nicht aufgelöset werden kann. Worin nämlich jeder seine Glückseligkeit zu setzen habe, kommt auf jedes sein besonderes Gefühl der Lust und Unlust an, und selbst in einem und demselben Subjekt auf die Verschiedenheit der Bedürfnis, nach den Abänderungen dieses Gefühls, und ein subjektiv notwendiges Gesetz (als Naturgesetz) ist also objektiv ein gar sehr zufälliges praktisches Prinzip, das in verschiedenen Subjekten sehr verschieden sein kann und  muß, mithin niemals ein Gesetz abgeben kann, weil es, bei der Begierde nach Glückseligkeit, nicht auf die Form der Gesetzmäßigkeit, sondern lediglich auf die Materie ankommt, nämlich ob und wie viel Vergnügen ich in der Befolgung des Gesetzes zu erwarten habe. Prinzipien der Selbstliebe können zwar allgemeine Regeln der Geschicklichkeit (Mittel zu Absichten auszufinden) enthalten, alsdenn sind es aber bloß theoretische Prinzipien
	
	
	7
	, (z.B. wie derjenige, der gerne \match{Brot} essen möchte, sich eine Mühle auszudenken habe). Aber praktische Vorschriften, die sich auf sie gründen, können niemals allgemein sein, denn der Bestimmungsgrund des Begehrungsvermögens ist auf das Gefühl der Lust und Unlust, das niemals als allgemein, auf dieselben Gegenstände gerichtet, angenommen werden kann, gegründet. 
	
	\unnumberedsection{Gang (5)} 
	\subsection*{tg210.2.13} 
	\textbf{Source : }Kritik der praktischen Vernunft/Erster Teil. Elementarlehre der reinen praktischen Vernunft/Erstes Buch. Die Analytik der reinen praktischen Vernunft/Erstes Hauptstück. Von den Grundsätzen der reinen praktischen Vernunft/I. Von der Deduktion der Grundsätze der reinen praktischen Vernunft\\  
	
	\textbf{Paragraphe : }Die Exposition des obersten Grundsatzes der praktischen Vernunft ist nun geschehen, d.i. erstlich, was er enthalte, daß er gänzlich a priori und unabhängig von empirischen Prinzipien für sich bestehe, und dann, worin er sich von allen anderen praktischen Grundsätzen unterscheide, gezeigt worden. Mit der Deduktion, d.i. der Rechtfertigung seiner objektiven und allgemeinen Gültigkeit und der Einsicht der Möglichkeit eines solchen synthetischen Satzes a priori, darf man nicht so gut fortzukommen hoffen, als es mit den Grundsätzen des reinen theoretischen Verstandes anging. Denn diese bezogen sich auf Gegenstände möglicher Erfahrung, nämlich auf Erscheinungen, und man konnte beweisen, daß nur dadurch, daß diese Erscheinungen nach Maßgabe jener Gesetze unter die Kategorien gebracht werden, diese Erscheinungen als Gegenstände der Erfahrung erkannt werden können, folglich alle mögliche Erfahrung diesen Gesetzen an gemessen sein müsse. Einen solchen  \match{Gang} kann ich aber mit der Deduktion des moralischen Gesetzes nicht nehmen. Denn es betrifft nicht das Erkenntnis von der Beschaffenheit der Gegenstände, die der Vernunft irgend wodurch anderwärts gegeben werden mögen, sondern ein Erkenntnis, so fern es der Grund von der Existenz der Gegenstände selbst werden kann und die Vernunft durch dieselbe Kausalität in einem vernünftigen Wesen hat, d.i. reine Vernunft, die als ein unmittelbar den Willen bestimmendes Vermögen angesehen werden kann. 
	
	\subsection*{tg215.2.23} 
	\textbf{Source : }Kritik der praktischen Vernunft/Erster Teil. Elementarlehre der reinen praktischen Vernunft/Erstes Buch. Die Analytik der reinen praktischen Vernunft/Drittes Hauptstück. Von den Triebfedern der reinen praktischen Vernunft/Kritische Beleuchtung der Analytik der reinen praktischen Vernunft\\  
	
	\textbf{Paragraphe : }Nur auf eines sei es mir erlaubt bei dieser Gelegenheit noch aufmerksam zu machen, nämlich daß jeder Schritt, den man mit der reinen Vernunft tut, sogar im praktischen Felde, wo man auf subtile Spekulation gar nicht Rücksicht nimmt, dennoch sich so genau und zwar von selbst an alle Momente der Kritik der theoretischen Vernunft anschließe, als ob jeder mit überlegter Vorsicht, bloß um dieser Bestätigung zu verschaffen, ausgedacht wäre. Eine solche auf keinerlei Weise gesuchte, sondern (wie man sich selbst davon überzeugen kann, wenn man nur die moralischen Nachforschungen bis zu ihren Prinzipien fortsetzen will) sich von selbst findende, genaue Eintreffung der wichtigsten Sätze der praktischen Vernunft, mit denen oft zu subtil und unnötig scheinenden Bemerkungen der Kritik der spekulativen, überrascht und setzt in Verwunderung, und bestärkt die schon von andern erkannte und gepriesene Maxime, in jeder wissenschaftlichen Untersuchung mit aller möglichen Genauigkeit und Offenheit seinen \match{Gang} ungestört fortzusetzen,  ohne sich an das zu kehren, wowider sie außer ihrem Felde etwa verstoßen möchte, sondern sie für sich allein, so viel man kann, wahr und vollständig zu vollführen. Öftere Beobachtung hat mich überzeugt, daß, wenn man diese Geschäfte zu Ende gebracht hat, das, was in der Hälfte desselben, in Betracht anderer Lehren außerhalb, mir bisweilen sehr bedenklich schien, wenn ich diese Bedenklichkeit nur so lange aus den Augen ließ, und bloß auf mein Geschäft Acht hatte, bis es vollendet sei, endlich auf unerwartete Weise mit demjenigen vollkommen zusammenstimmte, was sich ohne die mindeste Rücksicht auf jene Lehren, ohne Parteilichkeit und Vorliebe für dieselbe, von selbst gefunden hatte. Schriftsteller würden sich manche Irrtümer, manche verlorne Mühe (weil sie auf Blendwerk gestellt war) ersparen, wenn sie sich nur entschließen könnten, mit etwas mehr Offenheit zu Werke zu gehen. 
	
	\subsection*{tg225.2.10} 
	\textbf{Source : }Kritik der praktischen Vernunft/Erster Teil. Elementarlehre der reinen praktischen Vernunft/Zweites Buch. Dialektik der reinen praktischen Vernunft/Zweites Hauptstück. Von der Dialektik der reinen Vernunft in Bestimmung des Begriffs vom höchsten Gut/VII. Wie eine Erweiterung der reinen Vernunft, in praktischer Absicht, ohne damit ihr Erkenntnis, als spekulativ, zugleich zu erweitern, zu denken möglich sei\\  
	
	\textbf{Paragraphe : }Ich versuche nun, diesen Begriff an das Objekt der praktischen Vernunft zu halten, und da finde ich, daß der moralische Grundsatz ihn nur als möglich, unter Voraussetzung eines Welturhebers von höchster Vollkommenheit, zulasse. Er muß allwissend sein, um mein Verhalten bis zum Innersten meiner Gesinnung in allen möglichen Fällen und in alle Zukunft zu er kennen; allmächtig, um ihm die angemessenen Folgen zu erteilen; eben so allgegenwärtig, ewig, usw. Mithin bestimmt das moralische Gesetz durch den Begriff des höchsten Guts, als Gegenstandes einer reinen praktischen Vernunft, den Begriff des Urwesens als höchsten Wesens, welches der physische (und höher fortgesetzt der metaphysische), mithin der ganze spekulative  \match{Gang} der Vernunft nicht bewirken konnte. Also ist der Begriff von Gott ein ursprünglich nicht zur Physik, d.i. für die spekulative Vernunft, sondern zur Moral gehöriger Begriff, und eben das kann man auch von den übrigen Vernunftbegriffen sagen, von denen wir, als Postulaten derselben in ihrem praktischen Gebrauche, oben gehandelt haben. 
	
	\subsection*{tg228.2.6} 
	\textbf{Source : }Kritik der praktischen Vernunft/Zweiter Teil. Methodenlehre der reinen praktischen Vernunft\\  
	
	\textbf{Paragraphe : }Wenn man auf den \match{Gang} der Gespräche in gemischten Gesellschaften, die nicht bloß aus Gelehrten und Vernünftlern, sondern auch aus Leuten von Geschäften oder Frauenzimmer bestehen, Acht hat, so bemerkt man, daß, außer dem Erzählen und Scherzen, noch eine Unterhaltung, nämlich das Räsonieren, darin Platz findet; weil das erstere, wenn es Neuigkeit, und, mit ihr, Interesse bei sich führen soll, bald erschöpft, das zweite aber leicht schal wird. Unter allem Räsonieren ist aber keines, was mehr den Beitritt der Personen, die sonst bei allem Vernünfteln bald lange Weile haben, erregt, und eine gewisse Lebhaftigkeit in die Gesellschaft bringt, als das über den sittlichen Wert dieser oder jener Handlung, dadurch der Charakter irgend einer Person ausgemacht werden soll. Diejenige, welchen sonst alles Subtile und Grüblerische in theoretischen Fragen trocken und verdrießlich ist, treten bald bei, wenn es darauf ankommt, den moralischen Gehalt einer erzählten guten oder bösen Handlung auszumachen, und sind so genau, so grüblerisch, so subtil, alles, was die Reinigkeit der Absicht, und mithin  den Grad der Tugend in derselben vermindern, oder auch nur verdächtig machen könnte, auszusinnen, als man bei keinem Objekte der Spekulation sonst von ihnen erwartet. Man kann in diesen Beurteilungen oft den Charakter der über andere urteilenden Personen selbst hervorschimmern sehen, deren einige vorzüglich geneigt scheinen, indem sie ihr Richteramt, vornehmlich über Verstorbene, ausüben, das Gute, was von dieser oder jener Tat derselben erzählt wird, wider alle kränkende Einwürfe der Unlauterkeit und zuletzt den ganzen sittlichen Wert der Person wider den Vorwurf der Verstellung und geheimen Bösartigkeit zu verteidigen, andere dagegen mehr auf Anklagen und Beschuldigungen sinnen, diesen Wert anzufechten. Doch kann man den letzteren nicht immer die Absicht beimessen, Tugend aus allen Beispielen der Menschen gänzlich wegvernünfteln zu wollen, um sie dadurch zum leeren Namen zu machen, sondern es ist oft nur wohlgemeinte Strenge in Bestimmung des echten sittlichen Gehalts, nach einem unnachsichtlichen Gesetze, mit welchem und nicht mit Beispielen verglichen der Eigendünkel im Moralischen sehr sinkt, und Demut nicht etwa bloß gelehrt, sondern bei scharfer Selbstprüfung von jedem gefühlt wird. Dennoch kann man den Verteidigern der Reinigkeit der Absicht in gegebenen Beispielen es mehrenteils ansehen, daß sie ihr da, wo sie die Vermutung der Rechtschaffenheit für sich hat, auch den mindesten Fleck gerne abwischen möchten, aus dem Bewegungsgrunde, damit nicht, wenn allen Beispielen ihre Wahrhaftigkeit gestritten und aller menschlichen Tugend die Lauterkeit weggeleugnet würde, diese nicht endlich gar für ein bloßes Hirngespinst gehalten, und so alle Bestrebung zu derselben als eitles Geziere und trüglicher Eigendünkel geringschätzig gemacht werde. 
	
	\subsection*{tg229.2.3} 
	\textbf{Source : }Kritik der praktischen Vernunft/Beschluß\\  
	
	\textbf{Paragraphe : }Allein, Bewunderung und Achtung können zwar zur Nachforschung reizen, aber den Mangel derselben nicht ersetzen.  Was ist nun zu tun, um diese, auf nutzbare und der Erhabenheit des Gegenstandes angemessene Art, anzustellen? Beispiele mögen hiebei zur Warnung, aber auch zur Nachahmung dienen. Die Weltbetrachtung fing von dem herrlichsten Anblicke an, den menschliche Sinne nur immer vorlegen, und unser Verstand, in ihrem weiten Umfange zu verfolgen, nur immer vertragen kann, und endigte – mit der Sterndeutung. Die Moral fing mit der edelsten Eigenschaft in der menschlichen Natur an, deren Entwickelung und Kultur auf unendlichen Nutzen hinaussieht, und endigte – mit der Schwärmerei, oder dem Aber glauben. So geht es allen noch rohen Versuchen, in denen der vornehmste Teil des Geschäftes auf den Gebrauch der Vernunft ankommt, der nicht, so wie der Gebrauch der Füße, sich von selbst, vermittelst der öftern Ausübung, findet, vornehmlich wenn er Eigenschaften betrifft, die sich nicht so unmittelbar in der gemeinen Erfahrung darstellen lassen. Nachdem aber, wiewohl spät, die Maxime in Schwang gekommen war, alle Schritte vorher wohl zu überlegen, die die Vernunft zu tun vorhat, und sie nicht anders, als im Gleise einer vorher wohl überdachten Methode, ihren \match{Gang} machen zu lassen, so bekam die Beurteilung des Weltgebäudes eine ganz andere Richtung, und, mit dieser, zugleich einen, ohne Vergleichung, glücklichern Ausgang. Der Fall eines Steins, die Bewegung einer Schleuder, in ihre Elemente und dabei sich äußernde Kräfte aufgelöst, und mathematisch bearbeitet, brachte zuletzt diejenige klare und für alle Zukunft unveränderliche Einsicht in den Weltbau hervor, die, bei fortgehender Beobachtung, hoffen kann, sich immer nur zu erweitern, niemals aber, zurückgehen zu müssen, fürchten darf. 
	
	\unnumberedsection{Kopf (1)} 
	\subsection*{tg214.2.10} 
	\textbf{Source : }Kritik der praktischen Vernunft/Erster Teil. Elementarlehre der reinen praktischen Vernunft/Erstes Buch. Die Analytik der reinen praktischen Vernunft\\  
	
	\textbf{Paragraphe : }
	Achtung geht jederzeit nur auf Personen, niemals auf Sachen. Die letztere können Neigung, und, wenn es Tiere sind (z.B. Pferde, Hunde etc.), so gar Liebe, oder auch Furcht, wie das Meer, ein Vulkan, ein Raubtier, niemals aber Achtung in uns erwecken. Etwas, was diesem Gefühl schon näher tritt, ist Bewunderung, und diese, als Affekt, das Erstaunen, kann auch auf Sachen gehen, z.B. himmelhohe Berge, die Größe, Menge und Weite der Weltkörper, die Stärke und Geschwindigkeit mancher Tiere, u.s.w. Aber alles dieses ist nicht Achtung. Ein Mensch kann mir auch ein Gegenstand der Liebe, der Furcht, oder der Bewunderung, so gar bis zum Erstaunen und doch darum kein Gegenstand der Achtung sein. Seine scherzhafte Laune, sein Mut und Stärke, seine Macht, durch seinen Rang, den er unter anderen hat, können mir dergleichen Empfindungen einflößen, es fehlt aber immer noch an innerer Achtung gegen ihn. Fontenelle sagt: vor einem Vornehmen bücke ich mich, aber mein Geist bückt sich nicht. Ich kann hinzu setzen: vor einem niedrigen, bürgerlich-gemeinen Mann, an dem ich eine Rechtschaffenheit des Charakters in einem gewissen Maße, als ich mir von mir selbst nicht bewußt bin, wahrnehme, bückt sich mein Geist, ich mag wollen oder nicht, und den \match{Kopf} noch so hoch tragen, um ihn meinen Vorrang nicht übersehen zu lassen. 
	
	\unnumberedsection{Körper (1)} 
	\subsection*{tg215.2.11} 
	\textbf{Source : }Kritik der praktischen Vernunft/Erster Teil. Elementarlehre der reinen praktischen Vernunft/Erstes Buch. Die Analytik der reinen praktischen Vernunft/Drittes Hauptstück. Von den Triebfedern der reinen praktischen Vernunft/Kritische Beleuchtung der Analytik der reinen praktischen Vernunft\\  
	
	\textbf{Paragraphe : }Wenn ich von einem Menschen, der einen Diebstahl verübt, sage: diese Tat sei nach dem Naturgesetze der Kausalität aus den Bestimmungsgründen der vorhergehenden Zeit ein notwendiger Erfolg, so war es unmöglich, daß sie hat unterbleiben können; wie kann denn die Beurteilung nach dem moralischen Gesetze hierin eine Änderung machen, und voraussetzen, daß sie doch habe unterlassen werden können, weil das Gesetz sagt, sie hätte unterlassen werden sollen, d.i. wie kann derjenige, in demselben Zeitpunkte, in Absicht auf dieselbe Handlung, ganz frei heißen, in welchem, und in  derselben Absicht, er doch unter einer unvermeidlichen Naturnotwendigkeit steht? Eine Ausflucht darin suchen, daß man bloß die Art der Bestimmungsgründe seiner Kausalität nach dem Naturgesetze einem komparativen Begriffe von Freiheit anpaßt (nach welchem das bisweilen freie Wirkung heißt, davon der bestimmende Naturgrund innerlich im wirkenden Wesen liegt, z.B. das, was ein geworfener \match{Körper} verrichtet, wenn er in freier Bewegung ist, da man das Wort Freiheit braucht, weil er, während daß er im Fluge ist, nicht von außen wodurch getrieben wird, oder wie wir die Bewegung einer Uhr auch eine freie Bewegung nennen, weil sie ihren Zeiger selbst treibt, der also nicht äußerlich geschoben werden darf, eben so die Handlungen des Menschen, ob sie gleich, durch ihre Bestimmungsgründe, die in der Zeit vorhergehen, notwendig sind, dennoch frei nennen, weil es doch innere durch unsere eigene Kräfte hervorgebrachte Vorstellungen, dadurch nach veranlassenden Umständen erzeugte Begierden und mithin nach unserem eigenen Belieben bewirkte Handlungen sind), ist ein elender Behelf, womit sich noch immer einige hinhalten lassen, und so jenes schwere Problem mit einer kleinen Wortklauberei aufgelöset zu haben meinen, an dessen Auflösung Jahrtausende vergeblich gearbeitet haben, die daher wohl schwerlich so ganz auf der Oberfläche gefunden werden dürfte. Es kommt nämlich bei der Frage nach derjenigen Freiheit, die allen moralischen Gesetzen und der ihnen gemäßen Zurechnung zum Grunde gelegt werden muß, darauf gar nicht an, ob die nach einem Naturgesetze bestimmte Kausalität durch Bestimmungsgründe, die im Subjekte, oder außer ihm liegen, und im ersteren Fall, ob sie durch Instinkt oder mit Vernunft gedachte Bestimmungsgründe notwendig sei; wenn diese bestimmende Vorstellungen, nach dem Geständnisse eben dieser Männer selbst, den Grund ihrer Existenz doch in der Zeit und zwar dem vorigen Zustande haben, dieser aber wieder in einem vorhergehenden etc., so mögen sie, diese Bestimmungen, immer innerlich sein, sie mögen psychologische und nicht mechanische Kausalität haben, d.i. durch Vorstellungen, und nicht durch körperliche Bewegung,  Handlung hervorbringen, so sind es immer Bestimmungsgründe der Kausalität eines Wesens, so fern sein Dasein in der Zeit bestimmbar ist, mithin unter notwendig machenden Bedingungen der vergangenen Zeit, die also, wenn das Subjekt handeln soll, nicht mehr in seiner Gewalt sind, die also zwar psychologische Freiheit(wenn man ja dieses Wort von einer bloß inneren Verkettung der Vorstellungen der Seele brauchen will), aber doch Naturnotwendigkeit bei sich führen, mithin keine transzendentale Freiheit übrig lassen, welche als Unabhängigkeit von allem Empirischen und also von der Natur überhaupt gedacht werden muß, sie mag nun Gegenstand des inneren Sinnes, bloß in der Zeit, oder auch äußeren Sinne, im Raume und der Zeit zugleich betrachtet werden, ohne welche Freiheit (in der letzteren eigentlichen Bedeutung), die allein a priori praktisch ist, kein moralisch Gesetz, keine Zurechnung nach demselben, möglich ist. Eben um deswillen kann man auch alle Notwendigkeit der Begebenheiten in der Zeit, nach dem Naturgesetze der Kausalität, den Mechanismus der Natur nennen, ob man gleich darunter nicht versteht, daß Dinge, die ihm unterworfen sind, wirkliche materielle Maschinen sein müßten. Hier wird nur auf die Notwendigkeit der Verknüpfung der Begebenheiten in einer Zeitreihe, so wie sie sich nach dem Naturgesetze entwickelt, gesehen, man mag nun das Subjekt, in welchem dieser Ablauf geschieht, automaton materiale, da das Maschinenwesen durch Materie, oder mit Leibnizen spirituale, da es durch Vorstellungen betrieben wird, nennen, und wenn die Freiheit unseres Willens keine andere als die letztere (etwa die psychologische und komparative, nicht transzendentale, d.i. absolute zugleich) wäre, so würde sie im Grunde nichts besser, als die Freiheit eines Bratenwenders sein, der auch, wenn er einmal aufgezogen worden, von selbst seine Bewegungen verrichtet. 
	
	\unnumberedchapter{Botanique} 
	\unnumberedsection{Familie (1)} 
	\subsection*{tg228.2.8} 
	\textbf{Source : }Kritik der praktischen Vernunft/Zweiter Teil. Methodenlehre der reinen praktischen Vernunft\\  
	
	\textbf{Paragraphe : }
	Wenn man aber trägt: was denn eigentlich die reine Sittlichkeit ist, an der, als dem Probemetall, man jeder Handlung moralischen Gehalt prüfen müsse, so muß ich gestehen, daß nur Philosophen die Entscheidung dieser Frage zweifelhaft machen können; denn in der gemeinen Menschenvernunft ist sie, zwar nicht durch abgezogene allgemeine Formeln, aber doch durch den gewöhnlichen Gebrauch, gleichsam als der Unterschied zwischen der rechten und linken Hand, längst entschieden. Wir wollen also vorerst das Prüfungsmerkmal der reinen Tugend an einem Beispiele zeigen, und, indem wir uns vorstellen, daß es etwa einem zehnjährigen Knaben zur Beurteilung vorgelegt worden, sehen, ob er auch von selber, ohne durch den Lehrer dazu angewiesen zu sein, notwendig so urteilen müßte. Man erzähle die Geschichte eines redlichen Mannes, den man bewegen will, den Verleumdern einer unschuldigen, übrigens nicht vermögenden Person (wie etwa Anna von Boleyn auf Anklage Heinrichs VIII. von England) beizutreten. Man bietet Gewinne, d.i. große Geschenke oder hohen Rang an, er schlägt sie aus. Dieses wird bloßen Beifall und Billigung in der Seele des Zuhörers wirken, weil es Gewinn ist. Nun fängt man es mit Androhung des Verlusts an. Es sind unter diesen Verleumdern seine besten Freunde, die ihm jetzt ihre Freundschaft aufsagen, nahe Verwandte, die ihn (der ohne Vermögen ist) zu enterben drohen, Mächtige, die ihn in jedem Orte und Zustande verfolgen und kränken können, ein Landesfürst, der ihn mit dem Verlust der Freiheit, ja des Lebens selbst bedroht. Um ihn aber, damit das Maß des Leidens voll sei, auch den Schmerz fühlen zu lassen, den nur  das sittlich gute Herz recht inniglich fühlen kann, mag man seine mit äußerster Not und Dürftigkeit bedrohete \match{Familie} ihn um Nachgiebigkeit anflehend, ihn selbst, obzwar rechtschaffen, doch eben nicht von festen unempfindlichen Organen des Gefühls, für Mitleid sowohl als eigener Not, in einem Augenblick, darin er wünscht, den Tag nie erlebt zu haben, der ihn einem so unaussprechlichen Schmerz aussetzte, dennoch seinem Vorsatze der Redlichkeit, ohne zu wanken oder nur zu zweifeln, treu bleibend, vorstellen: so wird mein jugendlicher Zuhörer stufenweise, von der bloßen Billigung zur Bewunderung, von da zum Erstaunen, endlich bis zur größten Verehrung, und einem lebhaften Wunsche, selbst ein solcher Mann sein zu können (obzwar freilich nicht in seinem Zustande), erhoben werden; und gleichwohl ist hier die Tugend nur darum so viel wert, weil sie so viel kostet, nicht weil sie etwas einbringt. Die ganze Bewunderung und selbst Bestrebung zur Ähnlichkeit mit diesem Charakter beruht hier gänzlich auf der Reinigkeit des sittlichen Grundsatzes, welche nur dadurch recht in die Augen fallend vorgestellet werden kann, daß man alles, was Menschen nur zur Glückseligkeit zählen mögen, von den Triebfedern der Handlung wegnimmt. Also muß die Sittlichkeit auf das menschliche Herz desto mehr Kraft haben, je reiner sie dargestellt wird. Woraus denn folgt, daß, wenn das Gesetz der Sitten und das Bild der Heiligkeit und Tugend auf unsere Seele überall einigen Einfluß ausüben soll, sie diesen nur so fern ausüben könne, als sie rein, unvermengt von Absichten auf sein Wohlbefinden, als Triebfeder ans Herz gelegt wird, darum weil sie sich im Leiden am herrlichsten zeigt. Dasjenige aber, dessen Wegräumung die Wirkung einer bewegenden Kraft verstärkt, muß ein Hindernis gewesen sein. Folglich ist alle Beimischung der Triebfedern, die von eigener Glückseligkeit hergenommen werden, ein Hindernis, dem moralischen Gesetze Einfluß aufs menschliche Herz zu verschaffen. – Ich behaupte ferner, daß selbst in jener bewunderten Handlung, wenn der Bewegungsgrund, daraus sie geschah, die Hochschätzung seiner Pflicht war, alsdenn eben diese Achtung fürs Gesetz, nicht etwa ein Anspruch  auf die innere Meinung von Großmut und edler verdienstlicher Denkungsart, gerade auf das Gemüt des Zuschauers die größte Kraft habe, folglich Pflicht, nicht Verdienst, den nicht allein bestimmtesten, sondern, wenn sie im rechten Lichte ihrer Unverletzlichkeit vorgestellt wird, auch den eindringendsten Einfluß aufs Gemüt haben müsse. 
	
	\unnumberedsection{Frucht (1)} 
	\subsection*{tg214.2.12} 
	\textbf{Source : }Kritik der praktischen Vernunft/Erster Teil. Elementarlehre der reinen praktischen Vernunft/Erstes Buch. Die Analytik der reinen praktischen Vernunft\\  
	
	\textbf{Paragraphe : }Die Achtung ist so wenig ein Gefühl der Lust, daß man sich ihr in Ansehung eines Menschen nur ungern überläßt. Man sucht etwas ausfindig zu machen, was uns die Last derselben erleichtern könne, irgend einen Tadel, um uns wegen der Demütigung, die uns durch ein solches Beispiel widerfährt, schadlos zu halten. Selbst Verstorbene sind, vornehmlich wenn ihr Beispiel unnachahmlich scheint, vor dieser Kritik nicht immer gesichert. So gar das moralische Gesetz selbst, in seiner feierlichen Majestät, ist diesem Bestreben, sich der Achtung dagegen zu erwehren, ausgesetzt. Meint man wohl, daß es einer anderen Ursache zuzuschreiben sei, weswegen man es gern zu unserer vertraulichen Neigung herabwürdigen möchte, und sich aus anderen Ursachen alles so bemühe, um es zur beliebten Vorschrift unseres eigenen wohlverstandenen Vorteils zu machen, als daß man der abschreckenden Achtung, die uns unsere eigene Unwürdigkeit so strenge vorhält, loswerden möge? Gleichwohl ist darin doch auch wiederum so wenig Unlust: daß, wenn man einmal den Eigendünkel abgelegt, und jener Achtung praktischen Einfluß verstattet hat, man sich wiederum an der Herrlichkeit dieses Gesetzes nicht satt  sehen kann, und die Seele sich in dem Maße selbst zu erheben glaubt, als sie das heilige Gesetz über sich und ihre gebrechliche Natur erhaben sieht. Zwar können große Talente und eine ihnen proportionierte Tätigkeit auch Achtung, oder ein mit derselben analogisches Gefühl, bewirken, es ist auch ganz anständig, es ihnen zu widmen, und da scheint es, als ob Bewunderung mit jener Empfindung einerlei sei. Allein, wenn man näher zusieht, so wird man bemerken, daß, da es immer ungewiß bleibt, wie viel das angeborne Talent und wie viel Kultur durch eigenen Fleiß an der Geschicklichkeit Teil habe, so stellt uns die Vernunft die letztere mutmaßlich als \match{Frucht} der Kultur, mithin als Verdienst vor, welches unseren Eigendünkel merklich herabstimmt, und uns darüber entweder Vorwürfe macht, oder uns die Befolgung eines solchen Beispiels, in der Art, wie es uns angemessen ist, auferlegt. Sie ist also nicht bloße Bewunderung, diese Achtung, die wir einer solchen Person (eigentlich dem Gesetze, was uns sein Beispiel vorhält) beweisen; welches sich auch dadurch bestätigt, daß der gemeine Haufe der Liebhaber, wenn er das Schlechte des Charakters eines solchen Mannes (wie etwa Voltaire) sonst woher erkundigt zu haben glaubt, alle Achtung gegen ihn aufgibt, der wahre Gelehrte aber sie noch immer wenigstens im Gesichtspunkte seiner Talente fühlt, weil er selbst in einem Geschäfte und Berufe verwickelt ist, welches die Nachahmung desselben ihm gewissermaßen zum Gesetze macht. 
	
	\unnumberedsection{Große (2)} 
	\subsection*{tg228.2.7} 
	\textbf{Source : }Kritik der praktischen Vernunft/Zweiter Teil. Methodenlehre der reinen praktischen Vernunft\\  
	
	\textbf{Paragraphe : }Ich weiß nicht, warum die Erzieher der Jugend von diesem Hange der Vernunft, in aufgeworfenen praktischen Fragen selbst die subtilste Prüfung mit Vergnügen einzuschlagen, nicht schon längst Gebrauch gemacht haben, und, nachdem sie einen bloß moralischen Katechism zum Grunde legten, sie nicht die Biographien alter und neuer Zeiten in  der Absicht durchsuchten, um Belege zu den vorgelegten Pflichten bei der Hand zu haben, an denen sie, vornehmlich durch die Vergleichung ähnlicher Handlungen unter verschiedenen Umständen, die Beurteilung ihrer Zöglinge in Tätigkeit setzten, um den mindern oder größeren moralischen Gehalt derselben zu bemerken, als worin sie selbst die frühe Jugend, die zu aller Spekulation sonst noch unreif ist, bald sehr scharfsichtig, und dabei, weil sie den Fortschritt ihrer Urteilskraft fühlt, nicht wenig interessiert finden werden, was aber das Vornehmste ist, mit Sicherheit hoffen können, daß die öftere Übung, das Wohlverhalten in seiner ganzen Reinigkeit zu kennen und ihm Beifall zu geben, dagegen selbst die kleinste Abweichung von ihr mit Bedauern oder Verachtung zu bemerken, ob es zwar bis dahin nur ein Spiel der Urteilskraft, in welchem Kinder mit einander wetteifern können, getrieben wird, dennoch einen dauerhaften Eindruck der Hochschätzung auf der einen und des Abscheues auf der andern Seite zurücklassen werde, welche, durch bloße Gewohnheit, solche Handlungen als beifalls-oder tadelswürdig öfters anzusehen, zur Rechtschaffenheit im künftigen Lebenswandel eine gute Grundlage ausmachen würden. Nur wünsche ich sie mit Beispielen sogenannter edler (überverdienstlicher) Handlungen, mit welchen unsere empfindsame Schriften so viel um sich werfen, zu verschonen, und alles bloß auf Pflicht und den Wert, den ein Mensch sich in seinen eigenen Augen durch das Bewußtsein, sie nicht übertreten zu haben, geben kann und muß, auszusetzen, weil, was auf leere Wünsche und Sehnsuchten nach unersteiglicher Vollkommenheit hinausläuft, lauter Romanhelden hervorbringt, die, indem sie sich auf ihr Gefühl für das überschwenglich-\match{Große} viel zu Gute tun, sich dafür von der Beobachtung der gemeinen und gangbaren Schuldigkeit, die alsdenn ihnen nur unbedeutend klein scheint, frei sprechen.
	
	
	18
	
	
	
	\subsection*{tg229.2.2} 
	\textbf{Source : }Kritik der praktischen Vernunft/Beschluß\\  
	
	\textbf{Paragraphe : }Zwei Dinge erfüllen das Gemüt mit immer neuer und zunehmenden Bewunderung und Ehrfurcht, je öfter und anhaltender sich das Nachdenken damit beschäftigt: Der bestirnte Himmel über mir, und das moralische Gesetz in mir. Beide darf ich nicht als in Dunkelheiten verhüllt, oder im Überschwenglichen, außer meinem Gesichtskreise, suchen und bloß vermuten; ich sehe sie vor mir und verknüpfe sie unmittelbar mit dem Bewußtsein meiner Existenz. Das erste fängt von dem Platze an, den ich in der äußern Sinnenwelt einnehme, und erweitert die Verknüpfung, darin ich stehe, ins unabsehlich-\match{Große} mit Welten über Welten und Systemen von Systemen, überdem noch in grenzenlose Zeiten ihrer periodischen Bewegung, deren Anfang und Fortdauer. Das zweite fängt von meinem unsichtbaren Selbst, meiner Persönlichkeit, an, und stellt mich in einer Welt dar, die wahre Unendlichkeit hat, aber nur dem Verstande spürbar ist, und mit welcher (dadurch aber auch zugleich mit allen jenen sichtbaren Welten) ich mich nicht, wie dort, in bloß zufälliger, sondern allgemeiner und notwendiger Verknüpfung erkenne. Der erstere Anblick einer zahllosen Weltenmenge vernichtet gleichsam meine Wichtigkeit, als eines tierischen Geschöpfs, das die Materie, daraus es ward, dem Planeten (einem bloßen Punkt im Weltall) wieder zurückgeben muß, nachdem es eine kurze Zeit (man weiß nicht wie) mit Lebenskraft versehen gewesen. Der zweite erhebt dagegen meinen Wert, als einer Intelligenz, unendlich, durch meine Persönlichkeit, in welcher das moralische Gesetz mir ein von der Tierheit und selbst von der ganzen Sinnenwelt unabhängiges Leben offenbart, wenigstens so viel sich aus der zweckmäßigen Bestimmung meines Daseins durch dieses Gesetz, welche nicht auf Bedingungen und Grenzen dieses Lebens eingeschränkt ist, sondern ins Unendliche geht, abnehmen läßt. 
	
	\unnumberedsection{Kern (1)} 
	\subsection*{tg214.2.20} 
	\textbf{Source : }Kritik der praktischen Vernunft/Erster Teil. Elementarlehre der reinen praktischen Vernunft/Erstes Buch. Die Analytik der reinen praktischen Vernunft\\  
	
	\textbf{Paragraphe : }Hiemit stimmt aber die Möglichkeit eines solchen Gebots, als: Liebe Gott über alles und deinen Nächsten als dich selbst,
	
	
	11
	ganz wohl zusammen. Denn es fodert doch, als Gebot, Achtung für ein Gesetz, das Liebe befiehlt, und überläßt es nicht der beliebigen Wahl, sich diese zum Prinzip zu machen. Aber Liebe zu Gott als Neigung (pathologische Liebe) ist unmöglich; denn er ist kein Gegenstand der Sinne. Eben dieselbe gegen Menschen ist zwar möglich, kann aber nicht geboten werden; denn es steht in keines Menschen Vermögen, jemanden bloß auf Befehl zu lieben. Also ist es bloß die praktische Liebe, die in jenem \match{Kern} aller Gesetze verstanden wird. Gott lieben, heißt in dieser Bedeutung, seine Gebote gerne tun; den Nächsten lieben, heißt, alle Pflicht gegen ihn gerne ausüben. Das Gebot aber, das dieses zur Regel macht, kann auch nicht diese Gesinnung in pflichtmäßigen Handlungen zu haben, sondern bloß darnach zu streben gebieten. Denn ein Gebot, daß man etwas gerne tun soll, ist in sich widersprechend, weil, wenn wir, was uns zu tun obliege, schon von selbst wissen, wenn wir uns überdem auch bewußt wären, es gerne zu tun, ein Gebot darüber ganz unnötig, und, tun wir es zwar, aber eben nicht gerne, sondern nur aus Achtung fürs Gesetz, ein Gebot, welches diese Achtung eben zur Triebfeder der Maxime macht, gerade der gebotenen Gesinnung zuwider wirken würde. Jenes Gesetz aller Gesetze stellt also, wie alle moralische Vorschrift des Evangelii, die sittliche  Gesinnung in ihrer ganzen Vollkommenheit dar, so wie sie als ein Ideal der Heiligkeit von keinem Geschöpfe erreichbar, dennoch das Urbild ist, welchem wir uns zu näheren, und, in einem ununterbrochenen, aber unendlichen Progressus, gleich zu werden streben sollen. Könnte nämlich ein vernünftig Geschöpf jemals dahin kommen, alle moralische Gesetze völlig gerne zu tun, so würde das so viel bedeuten, als, es fände sich in ihm auch nicht einmal die Möglichkeit einer Begierde, die ihn zur Abweichung von ihnen reizte; denn die Überwindung einer solchen kostet dem Subjekt immer Aufopferung, bedarf also Selbstzwang, d.i. innere Nötigung zu dem, was man nicht ganz gern tut. Zu dieser Stufe der moralischen Gesinnung aber kann es ein Geschöpf niemals bringen. Denn da es ein Geschöpf, mithin in Ansehung dessen, was er zur gänzlichen Zufriedenheit mit seinem Zustande fodert, immer abhängig ist, so kann es niemals von Begierden und Neigungen ganz frei sein, die, weil sie auf physischen Ursachen beruhen, mit dem moralischen Gesetze, das ganz andere Quellen hat, nicht von selbst stimmen, mithin es jederzeit notwendig machen, in Rücksicht auf dieselbe, die Gesinnung seiner Maximen auf moralische Nötigung, nicht auf bereitwillige Ergebenheit, sondern auf Achtung, welche die Befolgung des Gesetzes, obgleich sie ungerne geschähe, fodert, nicht auf Liebe, die keine innere Weigerung des Willens gegen das Gesetz besorgt, zu gründen, gleichwohl aber diese letztere, nämlich die bloße Liebe zum Gesetze (da es alsdenn aufhören würde, Gebot zu sein, und Moralität, die nun subjektiv in Heiligkeit überginge, aufhören würde, Tugend zu sein) sich zum beständigen, obgleich unerreichbaren Ziele seiner Bestrebung zu machen. Denn an dem, was wir hochschätzen, aber doch (wegen des Bewußtseins unserer Schwächen) scheuen, verwandelt sich, durch die mehrere Leichtigkeit, ihm Gnüge zu tun, die ehrfurchtsvolle Scheu in Zuneigung, und Achtung in Liebe, wenigstens würde es die Vollendung einer dem Gesetze gewidmeten Gesinnung sein, wenn es jemals einem Geschöpfe möglich wäre, sie zu erreichen. 
	
	\unnumberedsection{Stachel (1)} 
	\subsection*{tg227.2.3} 
	\textbf{Source : }Kritik der praktischen Vernunft/Erster Teil. Elementarlehre der reinen praktischen Vernunft/Zweites Buch. Dialektik der reinen praktischen Vernunft/Zweites Hauptstück. Von der Dialektik der reinen Vernunft in Bestimmung des Begriffs vom höchsten Gut/IX. Von der der praktischen Bestimmung des Menschen weislich angemessenen Proportion seiner Erkenntnisvermögen\\  
	
	\textbf{Paragraphe : }Gesetzt nun, sie wäre hierin unserem Wunsche willfährig gewesen, und hätte uns diejenige Einsichtsfähigkeit, oder Erleuchtung erteilt, die wir gerne besitzen möchten, oder in deren Besitz einige wohl gar wähnen sich wirklich zu befinden, was würde allem Ansehn nach wohl die Folge hievon sein? Wofern nicht zugleich unsere ganze Natur umgeändert wäre, so würden die Neigungen, die doch allemal das erste Wort haben, zuerst ihre Befriedigung, und, mit vernünftiger Überlegung verbunden, ihre größtmögliche und daurende Befriedigung, unter dem Namen der Glückseligkeit, verlangen; das moralische Gesetz würde nachher sprechen, um jene in ihren geziemenden Schranken zu halten, und sogar sie alle insgesamt einem höheren, auf keine Neigung Rücksicht nehmenden, Zwecke zu unterwerfen.  Aber, statt des Streits, den jetzt die moralische Gesinnung mit den Neigungen zu führen hat, in welchem, nach einigen Niederlagen, doch allmählich moralische Stärke der Seele zu erwerben ist, würden Gott und Ewigkeit, mit ihrer furchtbaren Majestät, uns unablässig vor Augen liegen (denn, was wir vollkommen beweisen können, gilt, in Ansehung der Gewißheit, uns so viel, als wovon wir uns durch den Augenschein versichern). Die Übertretung des Gesetzes würde freilich vermieden, das Gebotene getan werden; weil aber die Gesinnung, aus welcher Handlungen geschehen sollen, durch kein Gebot mit eingeflößt werden kann, der \match{Stachel} der Tätigkeit hier aber sogleich bei Hand, und äußerlich ist, die Vernunft also sich nicht allererst empor arbeiten darf, um Kraft zum Widerstande gegen Neigungen durch lebendige Vorstellung der Würde des Gesetzes zu sammeln, so würden die mehresten gesetzmäßigen Handlungen aus Furcht, nur wenige aus Hoffnung und gar keine aus Pflicht geschehen, ein moralischer Wert der Handlungen aber, worauf doch allein der Wert der Person und selbst der der Welt, in den Augen der höchsten Weisheit, ankommt, würde gar nicht existieren. Das Verhalten der Menschen, so lange ihre Natur, wie sie jetzt ist, bliebe, würde also in einen bloßen Mechanismus verwandelt werden, wo, wie im Marionettenspiel, alles gut gestikulieren, aber in den Figuren doch kein Leben anzutreffen sein würde. Nun, da es mit uns ganz anders beschaffen ist, da wir, mit aller Anstrengung unserer Vernunft, nur eine sehr dunkele und zweideutige Aussicht in die Zukunft haben, der Weltregierer uns sein Dasein und seine Herrlichkeit nur mutmaßen, nicht erblicken, oder klar beweisen läßt, dagegen das moralische Gesetz in uns, ohne uns etwas mit Sicherheit zu verheißen, oder zu drohen, von uns uneigennützige Achtung fodert, übrigens aber, wenn diese Achtung tätig und herrschend geworden, allererst alsdenn und nur dadurch, Aussichten ins Reich des Übersinnlichen, aber auch nur mit schwachen Blicken erlaubt: so kann wahrhafte sittliche, dem Gesetze unmittelbar geweihete Gesinnung stattfinden und das vernünftige Geschöpf des Anteils am höchsten Gute würdig  werden, das dem moralischen Werte seiner Person und nicht bloß seinen Handlungen angemessen ist. Also möchte es auch hier wohl damit seine Richtigkeit haben, was uns das Studium der Natur und des Menschen sonst hinreichend lehrt, daß die unerforschliche Weisheit, durch die wir existieren, nicht minder verehrungswürdig ist, in dem, was sie uns versagte, als in dem, was sie uns zu teil werden ließ. 
	
	\unnumberedsection{Stein (1)} 
	\subsection*{tg197.2.9} 
	\textbf{Source : }Kritik der praktischen Vernunft/Vorrede\\  
	
	\textbf{Paragraphe : }So viel zur Rechtfertigung, warum in diesem Werke die Begriffe und Grundsätze der reinen spekulativen Vernunft, welche doch ihre besondere Kritik schon erlitten haben, hier hin und wieder nochmals der Prüfung unterworfen werden, welches dem systematischen Gange einer zu errichtenden Wissenschaft sonst nicht wohl geziemet (da abgeurteilte Sachen billig nur angeführt und nicht wiederum in Anregung gebracht werden müssen), doch hier erlaubt, ja nötig war; weil die Vernunft mit jenen Begriffen im Übergange zu einem ganz anderen Gebrauche betrachtet wird, als den sie dort von ihnen machte. Ein solcher Übergang macht aber eine Vergleichung des älteren mit dem neuern Gebrauche notwendig, um das neue Gleis von dem vorigen wohl zu unterscheiden und zugleich den Zusammenhang derselben bemerken zu lassen. Man wird also Betrachtungen dieser Art, unter andern diejenige, welche nochmals auf den Begriff der Freiheit, aber im praktischen Gebrauche der reinen Vernunft, gerichtet worden, nicht wie Einschiebsel betrachten,  die etwa nur dazu dienen sollen, um Lücken des kritischen Systems der spekulativen Vernunft auszufüllen (denn dieses ist in seiner Absicht vollständig), und, wie es bei einem übereilten Baue herzugehen pflegt, hintennach noch Stützen und Strebepfeiler anzubringen, sondern als wahre Glieder, die den Zusammenhang des Systems bemerklich machen, und Begriffe, die dort nur problematisch vorgestellt werden konnten, jetzt in ihrer realen Darstellung einsehen zu lassen. Diese Erinnerung geht vornehmlich den Begriff der Freiheit an, von dem man mit Befremdung bemerken muß, daß noch so viele ihn ganz wohl einzusehen und die Möglichkeit derselben erklären zu können sich rühmen, indem sie ihn bloß in psychologischer Beziehung betrachten, indessen daß, wenn sie ihn vorher in transzendentaler genau erwogen hätten, sie so wohl seine Unentbehrlichkeit, als problematischen Begriffs, in vollständigem Gebrauche der spekulativen Vernunft, als auch die völlige Unbegreiflichkeit desselben hätten erkennen, und, wenn sie nachher mit ihm zum praktischen Gebrauche gingen, gerade auf die nämliche Bestimmung des letzteren in Ansehung seiner Grundsätze von selbst hätten kommen müssen, zu welcher sie sich sonst so ungern verstehen wollen. Der Begriff der Freiheit ist der \match{Stein} des Anstoßes für alle Empiristen, aber auch der Schlüssel zu den erhabensten praktischen Grundsätzen für kritische Moralisten, die dadurch einsehen, daß sie notwendig rational verfahren müssen. Um deswillen ersuche ich den Leser, das, was zum Schlusse der Analytik über diesen Begriff gesagt wird, nicht mit flüchtigem Auge zu übersehen. 
	
	\unnumberedsection{Wurzel (2)} 
	\subsection*{tg213.2.7} 
	\textbf{Source : }Kritik der praktischen Vernunft/Erster Teil. Elementarlehre der reinen praktischen Vernunft/Erstes Buch. Die Analytik der reinen praktischen Vernunft/Zweites Hauptstück. Von dem Begriffe eines Gegenstandes der reinen praktischen Vernunft/Von der Typik der reinen praktischen Urteilskraft\\  
	
	\textbf{Paragraphe : }Übrigens, da von allem Intelligibelen schlechterdings nichts als (vermittelst des moralischen Gesetzes) die Freiheit, und auch diese nur, so fern sie eine von jenem unzertrennliche Voraussetzung ist, und ferner alle intelligibele Gegenstände, auf welche uns die Vernunft, nach Anleitung  jenes Gesetzes, etwa noch führen möchte, wiederum für uns keine Realität weiter haben, als zum Behuf desselben Gesetzes und des Gebrauches der reinen praktischen Vernunft, diese aber zum Typus der Urteilskraft die Natur (der reinen Verstandesform derselben nach) zu gebrauchen berechtigt und auch benötigt ist: so dient die gegenwärtige Anmerkung dazu, um zu verhüten, daß, was bloß zur Typik der Begriffe gehört, nicht zu den Begriffen selbst gezählt werde. Diese also, als Typik der Urteilskraft, bewahrt für dem Empirism der praktischen Vernunft, der die praktischen Begriffe, des Guten und Bösen, bloß in Erfahrungsfolgen (der sogenannten Glückseligkeit) setzt, obzwar diese und die unendlichen nützlichen Folgen eines durch Selbstliebe bestimmten Willens, wenn dieser sich selbst zugleich zum allgemeinen Naturgesetze machte, allerdings zum ganz angemessenen Typus für das Sittlichgute dienen kann, aber mit diesem doch nicht einerlei ist. Eben dieselbe Typik bewahrt auch vor dem Mystizism der praktischen Vernunft, welche das, was nur zum Symbol dienete, zum Schema macht, d.i. wirkliche, und doch nicht sinnliche, Anschauungen (eines unsichtbaren Reichs Gottes) der Anwendung der moralischen Begriffe unterlegt und ins Überschwengliche hinausschweift. Dem Gebrauche der moralischen Begriffe ist bloß der Rationalism der Urteilskraft angemessen, der von der sinnlichen Natur nichts weiter nimmt, als was auch reine Vernunft für sich denken kann, d.i. die Gesetzmäßigkeit, und in die übersinnliche nichts hineinträgt, als was umgekehrt sich durch Handlungen in der Sinnenwelt nach der formalen Regel eines Naturgesetzes überhaupt wirklich darstellen läßt. Indessen ist die Verwahrung vor dem Empirism der praktischen Vernunft viel wichtiger und anratungswürdiger, womit der Mystizism sich doch noch mit der Reinigkeit und Erhabenheit des moralischen Gesetzes zusammen verträgt und außerdem es nicht eben natürlich und der gemeinen Denkungsart angemessen ist, seine Einbildungskraft bis zu übersinnlichen Anschauungen anzuspannen, mithin auf dieser Seite die Gefahr nicht so allgemein  ist; dahingegen der Empirism die Sittlichkeit in Gesinnungen (worin doch, und nicht bloß in Handlungen, der hohe Wert besteht, den sich die Menschheit durch sie verschaffen kann und soll) mit der \match{Wurzel} ausrottet, und ihr ganz etwas anderes, nämlich ein empirisches Interesse, womit die Neigungen überhaupt unter sich Verkehr treiben, statt der Pflicht unterschiebt, überdem auch, eben darum, mit allen Neigungen, die (sie mögen einen Zuschnitt bekommen, welchen sie wollen), wenn sie zur Würde eines obersten praktischen Prinzips erhoben werden, die Menschheit degradieren, und da sie gleichwohl der Sinnesart aller so günstig sind, aus der Ursache weit gefährlicher ist, als alle Schwärmerei, die niemals einen daurenden Zustand vieler Menschen ausmachen kann. 
	
	\subsection*{tg214.2.24} 
	\textbf{Source : }Kritik der praktischen Vernunft/Erster Teil. Elementarlehre der reinen praktischen Vernunft/Erstes Buch. Die Analytik der reinen praktischen Vernunft\\  
	
	\textbf{Paragraphe : }
	Pflicht! du erhabener großer Name, der du nichts Beliebtes, was Einschmeichelung bei sich führt, in dir fassest, sondern Unterwerfung verlangst, doch auch nichts drohest, was natürliche Abneigung im Gemüte erregte und schreckte, um den Willen zu bewegen, sondern bloß ein Gesetz aufstellst, welches von selbst im Gemüte Eingang findet, und doch sich selbst wider Willen Verehrung (wenn gleich nicht immer Befolgung) erwirbt, vor dem alle Neigungen verstummen, wenn sie gleich in Geheim ihm entgegen wirken, welches ist der deiner würdige Ursprung, und wo findet man die \match{Wurzel} deiner edlen Abkunft, welche alle Verwandtschaft mit Neigungen stolz ausschlägt, und von welcher Wurzel abzustammen die unnachlaßliche Bedingung desjenigen Werts ist, den sich Menschen allein selbst geben können? 
	
	\unnumberedchapter{Monde} 
	\unnumberedsection{Abgrund (1)} 
	\subsection*{tg197.2.3} 
	\textbf{Source : }Kritik der praktischen Vernunft/Vorrede\\  
	
	\textbf{Paragraphe : }Mit diesem Vermögen steht auch die transzendentale Freiheit nunmehro fest, und zwar in derjenigen absoluten Bedeutung genommen, worin die spekulative Vernunft beim Gebrauche des Begriffs der Kausalität sie bedurfte, um sich wider die Antinomie zu retten, darin sie unvermeidlich gerät, wenn sie in der Reihe der Kausalverbindung sich das Unbedingte denken will, welchen Begriff sie aber nur problematisch, als nicht unmöglich zu denken, aufstellen konnte, ohne ihm seine objektive Realität zu sichern, sondern allein, um nicht durch vorgebliche Unmöglichkeit dessen, was sie doch wenigstens als denkbar gelten lassen muß, in ihrem Wesen angefochten und in einen \match{Abgrund} des Skeptizismus gestürzt zu werden. 
	
	\unnumberedsection{Boden (2)} 
	\subsection*{tg215.2.6} 
	\textbf{Source : }Kritik der praktischen Vernunft/Erster Teil. Elementarlehre der reinen praktischen Vernunft/Erstes Buch. Die Analytik der reinen praktischen Vernunft/Drittes Hauptstück. Von den Triebfedern der reinen praktischen Vernunft/Kritische Beleuchtung der Analytik der reinen praktischen Vernunft\\  
	
	\textbf{Paragraphe : }Die Unterscheidung der Glückseligkeit sichre von der Sittenlehre, in derer ersteren empirische Prinzipien das ganze Fundament, von der zweiten aber auch nicht den mindesten Beisatz derselben ausmachen, ist nun in der Analytik der reinen praktischen Vernunft die erste und wichtigste ihr obliegende Beschäftigung, in der sie so pünktlich, ja, wenn es auch hieße, peinlich, verfahren muß, als je der Geometer in seinem Geschäfte. Es kommt aber dem Philosophen, der hier (wie jederzeit im Vernunfterkenntnisse durch bloße Begriffe, ohne Konstruktion derselben) mit größerer Schwierigkeit zu kämpfen hat, weil er keine Anschauung  (reinem Noumen) zum Grunde legen kann, doch auch zu statten: daß er, beinahe wie der Chemist, zu aller Zeit ein Experiment mit jedes Menschen praktischer Vernunft anstellen kann, um den moralischen (reinen) Bestimmungsgrund vom empirischen zu unterscheiden; wenn er nämlich zu dem empirisch-affizierten Willen (z.B. desjenigen, der gerne lügen möchte, weil er sich dadurch was erwerben kann) das moralische Gesetz (als Bestimmungsgrund) zusetzt. Es ist, als ob der Scheidekünstler der Solution der Kalkerde in Salzgeist Alkali zusetzt; der Salzgeist verläßt so fort den Kalk, vereinigt sich mit dem Alkali, und jener wird zu \match{Boden} gestürzt. Eben so haltet dem, der sonst ein ehrlicher Mann ist (oder sich doch diesmal nur in Gedanken in die Stelle eines ehrlichen Mannes versetzt), das moralische Gesetz vor, an dem er die Nichtswürdigkeit eines Lügners erkennt, so fort verläßt seine praktische Vernunft (im Urteil über das, was von ihm geschehen sollte) den Vorteil, vereinigt sich mit dem, was ihm die Achtung für seine eigene Person erhält (der Wahrhaftigkeit), und der Vorteil wird nun von jedermann, nachdem er von allem Anhängsel der Vernunft (welche nur gänzlich auf der Seite der Pflicht ist) abgesondert und gewaschen worden, gewogen, um mit der Vernunft noch wohl in anderen Fällen in Verbindung zu treten, nur nicht, wo er dem moralischen Gesetze, welches die Vernunft niemals verläßt, sondern sich innigst damit vereinigt, zuwider sein könnte. 
	
	\subsection*{tg221.2.4} 
	\textbf{Source : }Kritik der praktischen Vernunft/Erster Teil. Elementarlehre der reinen praktischen Vernunft/Zweites Buch. Dialektik der reinen praktischen Vernunft/Zweites Hauptstück. Von der Dialektik der reinen Vernunft in Bestimmung des Begriffs vom höchsten Gut/III. Von dem Primat der reinen praktischen Vernunft in ihrer Verbindung mit der spekulativen\\  
	
	\textbf{Paragraphe : }
	In der Tat, so fern praktische Vernunft als pathologisch bedingt, d.i. das Interesse der Neigungen unter dem sinnlichen Prinzip der Glückseligkeit bloß verwaltend, zum Grunde gelegt würde, so ließe sich diese Zumutung an die spekulative Vernunft gar nicht tun. Mahomets Paradies, oder der Theosophen und Mystiker schmelzende Vereinigung mit der Gottheit, so wie jedem sein Sinn steht, würden der Vernunft ihre Ungeheuer aufdringen, und es wäre eben so gut, gar keine zu haben, als sie auf solche Weise allen Träumereien preiszugeben. Allein wenn reine Vernunft für sich praktisch sein kann und es wirklich ist, wie das Bewußtsein des moralischen Gesetzes es ausweiset, so ist es doch immer nur eine und dieselbe Vernunft, die, es sei in theoretischer oder praktischer Absicht, nach Prinzipien a priori urteilt, und da ist es klar, daß, wenn ihr Vermögen in der ersteren gleich nicht zulangt, gewisse Sätze behauptend festzusetzen, indessen daß sie ihr auch eben nicht widersprechen, eben diese Sätze, so bald sie unabtrennlich zum praktischen Interesse der reinen Vernunft gehören, zwar als ein ihr fremdes Angebot, das nicht auf ihrem \match{Boden} erwachsen, aber doch hinreichend beglaubigt ist, annehmen, und sie, mit allem, was sie als spekulative Vernunft in ihrer Macht hat, zu vergleichen und zu verknüpfen suchen müsse; doch sich bescheidend, daß dieses nicht ihre Einsichten, aber doch Erweiterungen ihres Gebrauchs in irgend einer anderen, nämlich praktischen, Absicht sind, welches ihrem Interesse, das in der Einschränkung des spekulativen Frevels besteht, ganz und gar nicht zuwider ist. 
	
	\unnumberedsection{Gebiet (1)} 
	\subsection*{tg198.2.3} 
	\textbf{Source : }Kritik der praktischen Vernunft/Einleitung: Von der Idee einer Kritik der praktischen Vernunft\\  
	
	\textbf{Paragraphe : }Der theoretische Gebrauch der Vernunft beschäftigte sich mit Gegenständen des bloßen Erkenntnisvermögens, und eine Kritik derselben, in Absicht auf diesen Gebrauch, betraf eigentlich nur das reine Erkenntnisvermögen, weil dieses Verdacht erregte, der sich auch hernach bestätigte, daß es sich leichtlich über seine Grenzen, unter unerreichbare Gegenstände, oder gar einander widerstreitende Begriffe, verlöre. Mit dem praktischen Gebrauche der Vernunft verhält es sich schon anders. In diesem beschäftigt sich die Vernunft mit Bestimmungsgründen des Willens, welcher ein Vermögen ist, den Vorstellungen entsprechende Gegenstände entweder hervorzubringen, oder doch sich selbst zu Bewirkung derselben (das physische Vermögen mag nun hinreichend sein, oder nicht), d.i. seine Kausalität zu bestimmen. Denn da kann wenigstens die Vernunft zur Willensbestimmung zulangen, und hat so fern immer objektive Realität, als es nur auf das Wollen ankommt. Hier ist also die erste Frage: ob reine Vernunft zur Bestimmung des Willens für sich allein zulange, oder ob sie nur als empirisch- bedingte ein Bestimmungsgrund derselben sein könne. Nun tritt hier ein durch die Kritik der reinen Vernunft gerechtfertigter, obzwar keiner empirischen Darstellung fähiger Begriff der Kausalität, nämlich der der Freiheit, ein, und wenn wir anjetzt Gründe ausfindig machen können, zu beweisen, daß diese Eigenschaft dem menschlichen Willen (und so auch dem Willen aller vernünftigen Wesen) in der Tat zukomme, so wird dadurch nicht allein dargetan, daß reine Vernunft praktisch sein könne, sondern daß sie allein, und nicht die empirisch-beschränkte, unbedingterweise praktisch sei. Folglich werden wir nicht eine Kritik der reinen praktischen, sondern nur der praktischen Vernunft überhaupt zu bearbeiten haben. Denn reine Vernunft, wenn allererst dargetan worden, daß es eine solche gebe, bedarf keiner Kritik. Sie ist es, welche selbst die Richtschnur zur Kritik alles ihres Gebrauchs enthält. Die Kritik der praktischen Vernunft überhaupt  hat also die Obliegenheit, die empirisch bedingte Vernunft von der Anmaßung abzuhalten, ausschließungsweise den Bestimmungsgrund des Willens allein abgeben zu wollen. Der Gebrauch der reinen Vernunft, wenn, daß es eine solche gebe, ausgemacht ist, ist allein immanent; der empirisch-bedingte, der sich die Alleinherrschaft anmaßt, ist dagegen transzendent, und äußert sich in Zumutungen und Geboten, die ganz über ihr \match{Gebiet} hinausgehen, welches gerade das umgekehrte Verhältnis von dem ist, was von der reinen Vernunft im spekulativen Gebrauche gesagt werden konnte. 
	
	\unnumberedsection{Gold (1)} 
	\subsection*{tg204.2.9} 
	\textbf{Source : }Kritik der praktischen Vernunft/Erster Teil. Elementarlehre der reinen praktischen Vernunft/Erstes Buch. Die Analytik der reinen praktischen Vernunft/Erstes Hauptstück. Von den Grundsätzen der reinen praktischen Vernunft/3. Lehrsatz II\\  
	
	\textbf{Paragraphe : }Man muß sich wundern, wie sonst scharfsinnige Männer einen Unterschied zwischen dem unteren und oberen Begehrungsvermögen darin zu finden glauben können, ob die Vorstellungen, die mit dem Gefühl der Lust verbunden sind, in den Sinnen, oder dem Verstande ihren Ursprung haben. Denn es kommt, wenn man nach den Bestimmungsgründen des Begehrens fragt und sie in einer von irgend etwas erwarteten Annehmlichkeit setzt, gar nicht darauf an, wo die Vorstellung dieses vergnügenden Gegenstandes herkomme, sondern nur, wie sehr sie vergnügt. Wenn eine Vorstellung, sie mag immerhin im Verstande ihren Sitz und Ursprung haben, die Willkür nur dadurch bestimmen  kann, daß sie ein Gefühl einer Lust im Subjekte voraussetzet, so ist, daß sie ein Bestimmungsgrund der Willkür sei, gänzlich von der Beschaffenheit des inneren Sinnes abhängig, daß dieser nämlich dadurch mit Annehmlichkeit affiziert werden kann. Die Vorstellungen der Gegenstände mögen noch so ungleichartig, sie mögen Verstandes-, selbst Vernunftvorstellungen im Gegensatze der Vorstellungen der Sinne sein, so ist doch das Gefühl der Lust, wodurch jene doch eigentlich nur den Bestimmungsgrund des Willens ausmachen (die Annehmlichkeit, das Vergnügen, das man davon erwartet, welches die Tätigkeit zur Hervorbringung des Objekts antreibt), nicht allein so fern von einerlei Art, daß es jederzeit bloß empirisch erkannt werden kann, sondern auch so fern, als er eine und dieselbe Lebenskraft, die sich im Begehrungsvermögen äußert, affiziert, und in dieser Beziehung von jedem anderen Bestimmungsgrunde in nichts, als dem Grade, verschieden sein kann. Wie würde man sonsten zwischen zwei der Vorstellungsart nach gänzlich verschiedenen Bestimmungsgründen eine Vergleichung der Größe nach anstellen können, um den, der am meisten das Begehrungsvermögen affiziert, vorzuziehen? Eben derselbe Mensch kann ein ihm lehrreiches Buch, das ihm nur einmal zu Händen kommt, ungelesen zurückgeben, um die Jagd nicht zu versäumen, in der Mitte einer schönen Rede weggehen, um zur Mahlzeit nicht zu spät zu kommen, eine Unterhaltung durch vernünftige Gespräche, die er sonst sehr schätzt, verlassen, um sich an den Spieltisch zu setzen, so gar einen Armen, dem wohlzutun ihm sonst Freude ist, abweisen, weil er jetzt eben nicht mehr Geld in der Tasche hat, als er braucht, um den Eintritt in die Komödie zu bezahlen. Beruht die Willensbestimmung auf dem Gefühle der Annehmlichkeit oder Unannehmlichkeit, die er aus irgend einer Ursache erwartet, so ist es ihm gänzlich einerlei, durch welche Vorstellungsart er affiziert werde. Nur wie stark, wie lange, wie leicht erworben und oft wiederholt diese Annehmlichkeit sei, daran liegt es ihm, um sich zur Wahl zu entschließen. So wie demjenigen, der \match{Gold} zur Ausgabe  braucht, gänzlich einerlei ist, ob die Materie desselben, das Gold, aus dem Gebirge gegraben, oder aus dem Sande gewaschen ist, wenn es nur allenthalben für denselben Wert angenommen wird, so fragt kein Mensch, wenn es ihm bloß an der Annehmlichkeit des Lebens gelegen ist, ob Verstandes- oder Sinnesvorstellungen, sondern nur, wie viel und großes Vergnügen sie ihm auf die längste Zeit verschaffen. Nur diejenigen, welche der reinen Vernunft das Vermögen, ohne Voraussetzung irgend eines Gefühls den Willen zu bestimmen, gerne abstreiten möchten, können sich so weit von ihrer eigenen Erklärung verirren, das, was sie selbst vorher auf ein und eben dasselbe Prinzip gebracht haben, dennoch hernach für ganz ungleichartig zu erklären. So findet sich z.B., daß man auch an bloßer Kraftanwendung, an dem Bewußtsein seiner Seelenstärke in Überwindung der Hindernisse, die sich unserem Vorsatze entgegensetzen, an der Kultur der Geistestalente, u.s.w., Vergnügen finden könne, und wir nennen das mit Recht feinere Freuden und Ergötzungen, weil sie mehr, wie andere, in unserer Gewalt sind, sich nicht abnutzen, das Gefühl zu noch mehrerem Genuß derselben vielmehr stärken, und, indem sie ergötzen, zugleich kultivieren. Allein sie darum für eine andere Art, den Willen zu bestimmen, als bloß durch den Sinn, auszugeben, da sie doch einmal, zur Möglichkeit jener Vergnügen, ein darauf in uns angelegtes Gefühle als erste Bedingung dieses Wohlgefallens, voraussetzen, ist gerade so, als wenn Unwissende, die gerne in der Metaphysik pfuschern möchten, sich die Materie so fein, so überfein, daß sie selbst darüber schwindlig werden möchten, denken, und dann glauben, auf diese Art sich ein geistiges und doch ausgedehntes Wesen erdacht zu haben. Wenn wir es, mit dem Epikur, bei der Tugend aufs bloße Vergnügen aussetzen, das sie verspricht, um den Willen zu bestimmen: so können wir ihn hernach nicht tadeln, daß er dieses mit denen der gröbsten Sinne für ganz gleichartig hält; denn man hat gar nicht Grund, ihm aufzubürden, daß er die Vorstellungen, wodurch dieses Gefühl in uns erregt würde, bloß den körperlichen Sinnen beigemessen hätte. Er hat von vielen derselben  den Quell, so viel man erraten kann, eben sowohl in dem Gebrauch des höheren Erkenntnisvermögens gesucht; aber das hinderte ihn nicht und konnte ihn auch nicht hindern, nach genanntem Prinzip das Vergnügen selbst, das uns jene allenfalls intellektuelle Vorstellungen gewähren, und wodurch sie allein Bestimmungsgründe des Willens sein können, gänzlich für gleichartig zu halten. Konsequent zu sein, ist die größte Obliegenheit eines Philosophen, und wird doch am seltensten angetroffen. Die alten griechischen Schulen geben uns davon mehr Beispiele, als wir in unserem synkretistischen Zeitalter antreffen, wo ein gewisses Koalitionssystem widersprechender Grundsätze voll Unredlichkeit und Seichtigkeit erkünstelt wird, weil es sich einem Publikum besser empfiehlt, das zufrieden ist, von allem etwas, und im ganzen nichts zu wissen, und dabei in allen Sätteln gerecht zu sein. Das Prinzip der eigenen Glückseligkeit, so viel Verstand und Vernunft bei ihm auch gebraucht werden mag, würde doch für den Willen keine andere Bestimmungsgründe, als die dem unteren Begehrungsvermögen angemessen sind, in sich fassen, und es gibt also entweder gar kein Begehrungsvermögen oder reine Vernunft muß für sich allein praktisch sein, d.i. ohne Voraussetzung irgend eines Gefühls, mithin ohne Vorstellungen des Angenehmen oder Unangenehmen, als der Materie des Begehrungsvermögens, die jederzeit eine empirische Bedingung der Prinzipien ist, durch die bloße Form der praktischen Regel den Willen bestimmen können. Alsdenn allein ist Vernunft nur, so fern sie für sich selbst den Willen bestimmt (nicht im Dienste der Neigungen ist), ein wahres oberes Begehrungsvermögen, dem das pathologisch bestimmbare untergeordnet ist, und wirklich, ja spezifisch von diesem unterschieden, so daß sogar die mindeste Beimischung von den Antrieben der letzteren ihrer Stärke und Vorzuge Abbruch tut, so wie das mindeste Empirische, als Bedingung in einer mathematischen Demonstration, ihre Würde und Nachdruck herabsetzt und vernichtet. Die Vernunft bestimmt in einem praktischen Gesetze unmittelbar  den Willen, nicht vermittelst eines dazwischen kommenden Gefühls der Lust und Unlust, selbst nicht an diesem Gesetze, und nur, daß sie als reine Vernunft praktisch sein kann, macht es ihr möglich, gesetzgebend zu sein. 
	
	\unnumberedsection{Kern (1)} 
	\subsection*{tg214.2.20} 
	\textbf{Source : }Kritik der praktischen Vernunft/Erster Teil. Elementarlehre der reinen praktischen Vernunft/Erstes Buch. Die Analytik der reinen praktischen Vernunft\\  
	
	\textbf{Paragraphe : }Hiemit stimmt aber die Möglichkeit eines solchen Gebots, als: Liebe Gott über alles und deinen Nächsten als dich selbst,
	
	
	11
	ganz wohl zusammen. Denn es fodert doch, als Gebot, Achtung für ein Gesetz, das Liebe befiehlt, und überläßt es nicht der beliebigen Wahl, sich diese zum Prinzip zu machen. Aber Liebe zu Gott als Neigung (pathologische Liebe) ist unmöglich; denn er ist kein Gegenstand der Sinne. Eben dieselbe gegen Menschen ist zwar möglich, kann aber nicht geboten werden; denn es steht in keines Menschen Vermögen, jemanden bloß auf Befehl zu lieben. Also ist es bloß die praktische Liebe, die in jenem \match{Kern} aller Gesetze verstanden wird. Gott lieben, heißt in dieser Bedeutung, seine Gebote gerne tun; den Nächsten lieben, heißt, alle Pflicht gegen ihn gerne ausüben. Das Gebot aber, das dieses zur Regel macht, kann auch nicht diese Gesinnung in pflichtmäßigen Handlungen zu haben, sondern bloß darnach zu streben gebieten. Denn ein Gebot, daß man etwas gerne tun soll, ist in sich widersprechend, weil, wenn wir, was uns zu tun obliege, schon von selbst wissen, wenn wir uns überdem auch bewußt wären, es gerne zu tun, ein Gebot darüber ganz unnötig, und, tun wir es zwar, aber eben nicht gerne, sondern nur aus Achtung fürs Gesetz, ein Gebot, welches diese Achtung eben zur Triebfeder der Maxime macht, gerade der gebotenen Gesinnung zuwider wirken würde. Jenes Gesetz aller Gesetze stellt also, wie alle moralische Vorschrift des Evangelii, die sittliche  Gesinnung in ihrer ganzen Vollkommenheit dar, so wie sie als ein Ideal der Heiligkeit von keinem Geschöpfe erreichbar, dennoch das Urbild ist, welchem wir uns zu näheren, und, in einem ununterbrochenen, aber unendlichen Progressus, gleich zu werden streben sollen. Könnte nämlich ein vernünftig Geschöpf jemals dahin kommen, alle moralische Gesetze völlig gerne zu tun, so würde das so viel bedeuten, als, es fände sich in ihm auch nicht einmal die Möglichkeit einer Begierde, die ihn zur Abweichung von ihnen reizte; denn die Überwindung einer solchen kostet dem Subjekt immer Aufopferung, bedarf also Selbstzwang, d.i. innere Nötigung zu dem, was man nicht ganz gern tut. Zu dieser Stufe der moralischen Gesinnung aber kann es ein Geschöpf niemals bringen. Denn da es ein Geschöpf, mithin in Ansehung dessen, was er zur gänzlichen Zufriedenheit mit seinem Zustande fodert, immer abhängig ist, so kann es niemals von Begierden und Neigungen ganz frei sein, die, weil sie auf physischen Ursachen beruhen, mit dem moralischen Gesetze, das ganz andere Quellen hat, nicht von selbst stimmen, mithin es jederzeit notwendig machen, in Rücksicht auf dieselbe, die Gesinnung seiner Maximen auf moralische Nötigung, nicht auf bereitwillige Ergebenheit, sondern auf Achtung, welche die Befolgung des Gesetzes, obgleich sie ungerne geschähe, fodert, nicht auf Liebe, die keine innere Weigerung des Willens gegen das Gesetz besorgt, zu gründen, gleichwohl aber diese letztere, nämlich die bloße Liebe zum Gesetze (da es alsdenn aufhören würde, Gebot zu sein, und Moralität, die nun subjektiv in Heiligkeit überginge, aufhören würde, Tugend zu sein) sich zum beständigen, obgleich unerreichbaren Ziele seiner Bestrebung zu machen. Denn an dem, was wir hochschätzen, aber doch (wegen des Bewußtseins unserer Schwächen) scheuen, verwandelt sich, durch die mehrere Leichtigkeit, ihm Gnüge zu tun, die ehrfurchtsvolle Scheu in Zuneigung, und Achtung in Liebe, wenigstens würde es die Vollendung einer dem Gesetze gewidmeten Gesinnung sein, wenn es jemals einem Geschöpfe möglich wäre, sie zu erreichen. 
	
	\unnumberedsection{Lage (1)} 
	\subsection*{tg197.2.12} 
	\textbf{Source : }Kritik der praktischen Vernunft/Vorrede\\  
	
	\textbf{Paragraphe : }Wenn es um die Bestimmung eines besonderen Vermögens der menschlichen Seele, nach seinen Quellen, Inhalte und Grenzen zu tun ist, so kann man zwar, nach der Natur  des menschlichen Erkenntnisses, nicht anders als von den Teilen derselben, ihrer genauen und (so viel als nach der jetzigen \match{Lage} unserer schon erworbenen Elemente derselben möglich ist) vollständigen Darstellung anfangen. Aber es ist noch eine zweite Aufmerksamkeit, die mehr philosophisch und architektonisch ist; nämlich, die Idee des Ganzen richtig zu fassen, und aus derselben alle jene Teile in ihrer wechselseitigen Beziehung auf einander, vermittelst der Ableitung derselben von dem Begriffe jenes Ganzen, in einem reinen Vernunftvermögen ins Auge zu fassen. Diese Prüfung und Gewährleistung ist nur durch die innigste Bekanntschaft mit dem System möglich, und die, welche in Ansehung der ersteren Nachforschung verdrossen gewesen, also diese Bekanntschaft zu erwerben nicht der Mühe wert geachtet haben, gelangen nicht zur zweiten Stufe, nämlich der Übersicht, welche eine synthetische Wiederkehr zu demjenigen ist, was vorher analytisch gegeben worden, und es ist kein Wunder, wenn sie allerwärts Inkonsequenzen finden, obgleich die Lücken, die diese vermuten lassen, nicht im System selbst, sondern bloß in ihrem eigenen unzusammenhängenden Gedankengange anzutreffen sind. 
	
	\unnumberedsection{Lauf (1)} 
	\subsection*{tg230.2.8} 
	\textbf{Source : }Kritik der praktischen Vernunft/Fußnoten\\  
	
	\textbf{Paragraphe : }
	
	4 Man könnte mir noch den Einwurf machen, warum ich nicht auch den Begriff des Begehrungsvermögens, oder des Gefühls der Lust vorher erklärt habe; obgleich dieser Vorwurf unbillig sein würde, weil man diese Erklärung, als in der Psychologie gegeben, billig sollte voraussetzen können. Es könnte aber freilich die Definition daselbst so eingerichtet sein, daß das Gefühl der Lust der Bestimmung des Begehrungsvermögens zum Grunde gelegt würde (wie es auch wirklich gemeinhin so zu geschehen pflegt), dadurch aber das oberste Prinzip der praktischen Philosophie notwendig empirisch ausfallen müßte, welches doch allererst auszumachen ist, und in dieser Kritik gänzlich widerlegt wird. Daher will ich diese Erklärung hier so geben, wie sie sein muß, um diesen streitigen Punkt, wie billig, im Anfange unentschieden zu lassen. – Leben ist das Vermögen eines Wesens, nach Gesetzen des Begehrungsvermögens zu handeln. Das Begehrungsvermögen ist das Vermögen desselben, durch seine Vorstellungen Ursache von der Wirklichkeit der Gegenstände dieser Vorstellungen zu sein. Lust ist die Vorstellung der Übereinstimmung des Gegenstandes oder der Handlung mit den subjektiven Bedingungen des Lebens, d.i. mit dem Vermögen der Kausalität einer Vorstellung in Ansehung der Wirklichkeit ihres Objekts (oder der Bestimmung der Kräfte des Subjekts zur Handlung, es hervorzubringen). Mehr brauche ich nicht zum Behuf der Kritik von Begriffen, die aus der Psychologie entlehnt werden, das übrige leistet die Kritik selbst. Man wird leicht gewahr, daß die Frage, ob die Lust dem Begehrungsvermögen jederzeit zum Grunde gelegt werden müsse, oder ob sie auch unter gewissen Bedingungen nur auf die Bestimmung desselben folge, durch diese Erklärung unentschieden bleibt; denn sie ist aus lauter Merkmalen des reinen Verstandes, d.i. Kategorien zusammengesetzt, die nichts Empirisches enthalten. Eine solche Behutsamkeit ist in der ganzen Philosophie sehr empfehlungswürdig, und wird dennoch oft verabsäumt, nämlich, seinen Urteilen vor der vollständigen Zergliederung des Begriffs, die oft nur sehr spät erreicht wird, durch gewagte Definition nicht vorzugreifen. Man wird auch durch den ganzen \match{Lauf} der Kritik (der theoretischen sowohl als praktischen Vernunft) bemerken, daß sich in demselben mannigfaltige Veranlassung vorfinde, manche Mängel im alten dogmatischen Gange der Philosophie zu ergänzen, und Fehler abzuändern, die nicht eher bemerkt werden, als wenn man von Begriffen einen Gebrauch der Vernunft macht, der aufs Ganze derselben geht. 
	
	\unnumberedsection{Licht (2)} 
	\subsection*{tg197.2.18} 
	\textbf{Source : }Kritik der praktischen Vernunft/Vorrede\\  
	
	\textbf{Paragraphe : }Doch, da es in diesem philosophischen und kritischen Zeitalter schwerlich mit jenem Empirism Ernst sein kann, und er vermutlich nur zur Übung der Urteilskraft, und, um durch den Kontrast die Notwendigkeit rationaler Prinzipien a priori in ein helleres \match{Licht} zu setzen, aufgestellet wird: so kann man es denen doch Dank wissen, die sich mit dieser sonst eben nicht belehrenden Arbeit bemühen wollen. 
	
	\subsection*{tg197.2.8} 
	\textbf{Source : }Kritik der praktischen Vernunft/Vorrede\\  
	
	\textbf{Paragraphe : }
	Hiedurch verstehe ich auch, warum die erheblichsten Einwürfe wider die Kritik, die mir bisher noch vorgekommen sind, sich gerade um diese zwei Angel drehen: nämlich, einerseits, im theoretischen Erkenntnis geleugnete und im praktischen behauptete objektive Realität der auf Noumenen angewandten Kategorien, andererseits die paradoxe Foderung, sich als Subjekt der Freiheit zum Noumen, zugleich aber auch in Absicht auf die Natur zum Phänomen in seinem eigenen empirischen Bewußtsein zu machen. Denn, so lange man sich noch keine bestimmte Begriffe von Sittlichkeit und Freiheit machte, konnte man nicht erraten, was man einerseits der vorgeblichen Erscheinung als Noumen zum Grunde legen wolle, und andererseits, ob es überall auch möglich sei, sich noch von ihm einen Begriff zu machen, wenn man vorher alle Begriffe des reinen Verstandes im theoretischen Gebrauche schon ausschließungsweise den bloßen Erscheinungen gewidmet hätte. Nur eine ausführliche Kritik der praktischen Vernunft kann alle diese Mißdeutung heben, und die konsequente Denkungsart, welche eben ihren größten Vorzug ausmacht, in ein helles \match{Licht} setzen. 
	
	\unnumberedsection{Oberfläche (1)} 
	\subsection*{tg215.2.11} 
	\textbf{Source : }Kritik der praktischen Vernunft/Erster Teil. Elementarlehre der reinen praktischen Vernunft/Erstes Buch. Die Analytik der reinen praktischen Vernunft/Drittes Hauptstück. Von den Triebfedern der reinen praktischen Vernunft/Kritische Beleuchtung der Analytik der reinen praktischen Vernunft\\  
	
	\textbf{Paragraphe : }Wenn ich von einem Menschen, der einen Diebstahl verübt, sage: diese Tat sei nach dem Naturgesetze der Kausalität aus den Bestimmungsgründen der vorhergehenden Zeit ein notwendiger Erfolg, so war es unmöglich, daß sie hat unterbleiben können; wie kann denn die Beurteilung nach dem moralischen Gesetze hierin eine Änderung machen, und voraussetzen, daß sie doch habe unterlassen werden können, weil das Gesetz sagt, sie hätte unterlassen werden sollen, d.i. wie kann derjenige, in demselben Zeitpunkte, in Absicht auf dieselbe Handlung, ganz frei heißen, in welchem, und in  derselben Absicht, er doch unter einer unvermeidlichen Naturnotwendigkeit steht? Eine Ausflucht darin suchen, daß man bloß die Art der Bestimmungsgründe seiner Kausalität nach dem Naturgesetze einem komparativen Begriffe von Freiheit anpaßt (nach welchem das bisweilen freie Wirkung heißt, davon der bestimmende Naturgrund innerlich im wirkenden Wesen liegt, z.B. das, was ein geworfener Körper verrichtet, wenn er in freier Bewegung ist, da man das Wort Freiheit braucht, weil er, während daß er im Fluge ist, nicht von außen wodurch getrieben wird, oder wie wir die Bewegung einer Uhr auch eine freie Bewegung nennen, weil sie ihren Zeiger selbst treibt, der also nicht äußerlich geschoben werden darf, eben so die Handlungen des Menschen, ob sie gleich, durch ihre Bestimmungsgründe, die in der Zeit vorhergehen, notwendig sind, dennoch frei nennen, weil es doch innere durch unsere eigene Kräfte hervorgebrachte Vorstellungen, dadurch nach veranlassenden Umständen erzeugte Begierden und mithin nach unserem eigenen Belieben bewirkte Handlungen sind), ist ein elender Behelf, womit sich noch immer einige hinhalten lassen, und so jenes schwere Problem mit einer kleinen Wortklauberei aufgelöset zu haben meinen, an dessen Auflösung Jahrtausende vergeblich gearbeitet haben, die daher wohl schwerlich so ganz auf der \match{Oberfläche} gefunden werden dürfte. Es kommt nämlich bei der Frage nach derjenigen Freiheit, die allen moralischen Gesetzen und der ihnen gemäßen Zurechnung zum Grunde gelegt werden muß, darauf gar nicht an, ob die nach einem Naturgesetze bestimmte Kausalität durch Bestimmungsgründe, die im Subjekte, oder außer ihm liegen, und im ersteren Fall, ob sie durch Instinkt oder mit Vernunft gedachte Bestimmungsgründe notwendig sei; wenn diese bestimmende Vorstellungen, nach dem Geständnisse eben dieser Männer selbst, den Grund ihrer Existenz doch in der Zeit und zwar dem vorigen Zustande haben, dieser aber wieder in einem vorhergehenden etc., so mögen sie, diese Bestimmungen, immer innerlich sein, sie mögen psychologische und nicht mechanische Kausalität haben, d.i. durch Vorstellungen, und nicht durch körperliche Bewegung,  Handlung hervorbringen, so sind es immer Bestimmungsgründe der Kausalität eines Wesens, so fern sein Dasein in der Zeit bestimmbar ist, mithin unter notwendig machenden Bedingungen der vergangenen Zeit, die also, wenn das Subjekt handeln soll, nicht mehr in seiner Gewalt sind, die also zwar psychologische Freiheit(wenn man ja dieses Wort von einer bloß inneren Verkettung der Vorstellungen der Seele brauchen will), aber doch Naturnotwendigkeit bei sich führen, mithin keine transzendentale Freiheit übrig lassen, welche als Unabhängigkeit von allem Empirischen und also von der Natur überhaupt gedacht werden muß, sie mag nun Gegenstand des inneren Sinnes, bloß in der Zeit, oder auch äußeren Sinne, im Raume und der Zeit zugleich betrachtet werden, ohne welche Freiheit (in der letzteren eigentlichen Bedeutung), die allein a priori praktisch ist, kein moralisch Gesetz, keine Zurechnung nach demselben, möglich ist. Eben um deswillen kann man auch alle Notwendigkeit der Begebenheiten in der Zeit, nach dem Naturgesetze der Kausalität, den Mechanismus der Natur nennen, ob man gleich darunter nicht versteht, daß Dinge, die ihm unterworfen sind, wirkliche materielle Maschinen sein müßten. Hier wird nur auf die Notwendigkeit der Verknüpfung der Begebenheiten in einer Zeitreihe, so wie sie sich nach dem Naturgesetze entwickelt, gesehen, man mag nun das Subjekt, in welchem dieser Ablauf geschieht, automaton materiale, da das Maschinenwesen durch Materie, oder mit Leibnizen spirituale, da es durch Vorstellungen betrieben wird, nennen, und wenn die Freiheit unseres Willens keine andere als die letztere (etwa die psychologische und komparative, nicht transzendentale, d.i. absolute zugleich) wäre, so würde sie im Grunde nichts besser, als die Freiheit eines Bratenwenders sein, der auch, wenn er einmal aufgezogen worden, von selbst seine Bewegungen verrichtet. 
	
	\unnumberedsection{Stein (1)} 
	\subsection*{tg197.2.9} 
	\textbf{Source : }Kritik der praktischen Vernunft/Vorrede\\  
	
	\textbf{Paragraphe : }So viel zur Rechtfertigung, warum in diesem Werke die Begriffe und Grundsätze der reinen spekulativen Vernunft, welche doch ihre besondere Kritik schon erlitten haben, hier hin und wieder nochmals der Prüfung unterworfen werden, welches dem systematischen Gange einer zu errichtenden Wissenschaft sonst nicht wohl geziemet (da abgeurteilte Sachen billig nur angeführt und nicht wiederum in Anregung gebracht werden müssen), doch hier erlaubt, ja nötig war; weil die Vernunft mit jenen Begriffen im Übergange zu einem ganz anderen Gebrauche betrachtet wird, als den sie dort von ihnen machte. Ein solcher Übergang macht aber eine Vergleichung des älteren mit dem neuern Gebrauche notwendig, um das neue Gleis von dem vorigen wohl zu unterscheiden und zugleich den Zusammenhang derselben bemerken zu lassen. Man wird also Betrachtungen dieser Art, unter andern diejenige, welche nochmals auf den Begriff der Freiheit, aber im praktischen Gebrauche der reinen Vernunft, gerichtet worden, nicht wie Einschiebsel betrachten,  die etwa nur dazu dienen sollen, um Lücken des kritischen Systems der spekulativen Vernunft auszufüllen (denn dieses ist in seiner Absicht vollständig), und, wie es bei einem übereilten Baue herzugehen pflegt, hintennach noch Stützen und Strebepfeiler anzubringen, sondern als wahre Glieder, die den Zusammenhang des Systems bemerklich machen, und Begriffe, die dort nur problematisch vorgestellt werden konnten, jetzt in ihrer realen Darstellung einsehen zu lassen. Diese Erinnerung geht vornehmlich den Begriff der Freiheit an, von dem man mit Befremdung bemerken muß, daß noch so viele ihn ganz wohl einzusehen und die Möglichkeit derselben erklären zu können sich rühmen, indem sie ihn bloß in psychologischer Beziehung betrachten, indessen daß, wenn sie ihn vorher in transzendentaler genau erwogen hätten, sie so wohl seine Unentbehrlichkeit, als problematischen Begriffs, in vollständigem Gebrauche der spekulativen Vernunft, als auch die völlige Unbegreiflichkeit desselben hätten erkennen, und, wenn sie nachher mit ihm zum praktischen Gebrauche gingen, gerade auf die nämliche Bestimmung des letzteren in Ansehung seiner Grundsätze von selbst hätten kommen müssen, zu welcher sie sich sonst so ungern verstehen wollen. Der Begriff der Freiheit ist der \match{Stein} des Anstoßes für alle Empiristen, aber auch der Schlüssel zu den erhabensten praktischen Grundsätzen für kritische Moralisten, die dadurch einsehen, daß sie notwendig rational verfahren müssen. Um deswillen ersuche ich den Leser, das, was zum Schlusse der Analytik über diesen Begriff gesagt wird, nicht mit flüchtigem Auge zu übersehen. 
	
	\unnumberedsection{Strom (1)} 
	\subsection*{tg215.2.13} 
	\textbf{Source : }Kritik der praktischen Vernunft/Erster Teil. Elementarlehre der reinen praktischen Vernunft/Erstes Buch. Die Analytik der reinen praktischen Vernunft/Drittes Hauptstück. Von den Triebfedern der reinen praktischen Vernunft/Kritische Beleuchtung der Analytik der reinen praktischen Vernunft\\  
	
	\textbf{Paragraphe : }Hiemit stimmen auch die Richteraussprüche desjenigen wundersamen Vermögens in uns, welches wir Gewissen nennen, vollkommen überein. Ein Mensch mag künsteln, soviel als er will, um ein gesetzwidriges Betragen, dessen er sich  erinnert, sich als unvorsätzliches Versehen, als bloße Unbehutsamkeit, die man niemals gänzlich vermeiden kann, folglich als etwas, worin er vom \match{Strom} der Naturnotwendigkeit fortgerissen wäre, vorzumalen und sich darüber für schuldfrei zu erklären, so findet er doch, daß der Advokat, der zu seinem Vorteil spricht, den Ankläger in ihm keinesweges zum Verstummen bringen könne, wenn er sich bewußt ist, daß er zu der Zeit, als er das Unrecht verübte, nur bei Sinnen, d.i. im Gebrauche seiner Freiheit war, und gleichwohl erklärt er sich sein Vergehen, aus gewisser übeln, durch allmähliche Vernachlässigung der Achtsamkeit auf sich selbst zugezogener Gewohnheit, bis auf den Grad, daß er es als eine natürliche Folge derselben ansehen kann, ohne daß dieses ihn gleichwohl wider den Selbsttadel und den Verweis sichern kann, den er sich selbst macht. Darauf gründet sich denn auch die Reue über eine längst begangene Tat bei jeder Erinnerung derselben; eine schmerzhafte, durch moralische Gesinnung gewirkte Empfindung, die so fern praktisch leer ist, als sie nicht dazu dienen kann, das Geschehene ungeschehen zu machen, und sogar ungereimt sein würde (wie Priestley, als ein echter, konsequent verfahrender Fatalist, sie auch dafür erklärt, und in Ansehung welcher Offenherzigkeit er mehr Beifall verdient, als diejenige, welche, indem sie den Mechanism des Willens in der Tat, die Freiheit desselben aber mit Worten behaupten, noch immer dafür gehalten sein wollen, daß sie jene, ohne doch die Möglichkeit einer solchen Zurechnung begreiflich zu machen, in ihrem synkretistischen System mit einschließen), aber, als Schmerz, doch ganz rechtmäßig ist, weil die Vernunft, wenn es auf das Gesetz unserer intelligibelen Existenz (das moralische) ankommt, keinen Zeitunterschied anerkennt, und nur frägt, ob die Begebenheit mir als Tat angehöre, alsdenn aber immer dieselbe Empfindung damit moralisch verknüpft, sie mag jetzt geschehen, oder vorlängst geschehen sein. Denn das Sinnenleben hat in Ansehung des intelligibelen Bewußtseins seines Daseins (der Freiheit) absolute Einheit eines Phänomens, welches, so fern es bloß Erscheinungen von der Gesinnung, die das moralische Gesetz angeht  (von dem Charakter), enthält, nicht nach der Naturnotwendigkeit, die ihm als Erscheinung zukommt, sondern nach der absoluten Spontaneität der Freiheit beurteilt werden muß. Man kann also einräumen, daß, wenn es für uns möglich wäre, in eines Menschen Denkungsart, so wie sie sich durch innere sowohl als äußere Handlungen zeigt, so tiefe Einsicht zu haben, daß jede, auch die mindeste Triebfeder dazu uns bekannt würde, imgleichen alle auf diese wirkende äußere Veranlassungen, man eines Menschen Verhalten auf die Zukunft mit Gewißheit, so wie eine Mond- oder Sonnenfinsternis, ausrechnen könnte, und dennoch dabei behaupten, daß der Mensch frei sei. Wenn wir nämlich noch eines andern Blicks (der uns aber freilich gar nicht verliehen ist, sondern an dessen Statt wir nur den Vernunftbegriff haben), nämlich einer intellektuellen Anschauung desselben Subjekts fähig wären, so würden wir doch inne werden, daß diese ganze Kette von Erscheinungen in Ansehung dessen, was nur immer das moralische Gesetz angehen kann, von der Spontaneität des Subjekts, als Dinges an sich selbst, abhängt, von deren Bestimmung sich gar keine physische Erklärung geben läßt. In Ermangelung dieser Anschauung versichert uns das moralische Gesetz diesen Unterschied der Beziehung unserer Handlungen, als Erscheinungen, auf das Sinnenwesen unseres Subjekts, von derjenigen, dadurch dieses Sinnenwesen selbst auf das intelligibele Substrat in uns bezogen wird. – In dieser Rücksicht, die unserer Vernunft natürlich, obgleich unerklärlich ist, lassen sich auch Beurteilungen rechtfertigen, die, mit aller Gewissenhaftigkeit gefället, dennoch dem ersten Anscheine nach aller Billigkeit ganz zu widerstreiten scheinen. Es gibt Fälle, wo Menschen von Kindheit auf, selbst unter einer Erziehung, die, mit der ihrigen zugleich, andern ersprießlich war, dennoch so frühe Bosheit zeigen, und so bis in ihre Mannesjahre zu steigen fortfahren, daß man sie für geborne Bösewichter, und gänzlich, was die Denkungsart betrifft, für unbesserlich hält, gleichwohl aber sie wegen ihres Tuns und Lassens eben so richtet, ihnen ihre Verbrechen eben so als Schuld verweiset, ja sie (die Kinder) selbst diese Verweise so ganz gegründet  finden, als ob sie, ungeachtet der ihnen beigemessenen hoffnungslosen Naturbeschaffenheit ihres Gemüts, eben so verantwortlich blieben, als jeder andere Mensch. Dieses würde nicht geschehen können, wenn wir nicht voraussetzten, daß alles, was aus seiner Willkür entspringt (wie ohne Zweifel jede vorsätzlich verübte Handlung), eine freie Kausalität zum Grunde habe, welche von der frühen Jugend an ihren Charakter in ihren Erscheinungen (den Handlungen) ausdrückt, die wegen der Gleichförmigkeit des Verhaltens einen Naturzusammenhang kenntlich machen, der aber nicht die arge Beschaffenheit des Willens notwendig macht, sondern vielmehr die Folge der freiwillig angenommenen bösen und unwandelbaren Grundsätze ist, welche ihn nur noch um desto verwerflicher und strafwürdiger machen. 
	
	\unnumberedchapter{Sciencesexactes} 
	\unnumberedsection{Arbeit (1)} 
	\subsection*{tg197.2.10} 
	\textbf{Source : }Kritik der praktischen Vernunft/Vorrede\\  
	
	\textbf{Paragraphe : }Ob ein solches System, als hier von der reinen praktischen Vernunft aus der Kritik der letzteren entwickelt wird, viel oder wenig Mühe gemacht habe, um vornehmlich den rechten Gesichtspunkt, aus dem das Ganze derselben richtig vorgezeichnet werden kann, nicht zu verfehlen, muß ich den Kennern einer dergleichen \match{Arbeit} zu beurteilen überlassen. Es setzt zwar die Grundlegung zur Metaphysik der Sitten voraus, aber nur in so fern, als diese mit dem Prinzip der Pflicht vorläufige Bekanntschaft macht und eine bestimmte  Formel derselben angibt und rechtfertigt;
	
	
	3
	sonst besteht es durch sich selbst. Daß die Einteilung aller praktischen Wissenschaften zur Vollständigkeit nicht mit beigefügt worden, wie es die Kritik der spekulativen Vernunft leistete, dazu ist auch gültiger Grund in der Beschaffenheit dieses praktischen Vernunftvermögens anzutreffen. Denn die besondere Bestimmung der Pflichten, als Menschenpflichten, um sie einzuteilen, ist nur möglich, wenn vorher das Subjekt dieser Bestimmung (der Mensch), nach der Beschaffenheit, mit der er wirklich ist, obzwar nur so viel als in Beziehung auf Pflicht überhaupt nötig ist, erkannt worden; diese aber gehört nicht in eine Kritik der praktischen Vernunft überhaupt, die nur die Prinzipien ihrer Möglichkeit, ihres Umfanges und Grenzen vollständig ohne besondere Beziehung auf die menschliche Natur angeben soll. Die Einteilung gehört also hier zum System der Wissenschaft, nicht zum System der Kritik. 
	
	\unnumberedsection{Bewegung (1)} 
	\subsection*{tg229.2.3} 
	\textbf{Source : }Kritik der praktischen Vernunft/Beschluß\\  
	
	\textbf{Paragraphe : }Allein, Bewunderung und Achtung können zwar zur Nachforschung reizen, aber den Mangel derselben nicht ersetzen.  Was ist nun zu tun, um diese, auf nutzbare und der Erhabenheit des Gegenstandes angemessene Art, anzustellen? Beispiele mögen hiebei zur Warnung, aber auch zur Nachahmung dienen. Die Weltbetrachtung fing von dem herrlichsten Anblicke an, den menschliche Sinne nur immer vorlegen, und unser Verstand, in ihrem weiten Umfange zu verfolgen, nur immer vertragen kann, und endigte – mit der Sterndeutung. Die Moral fing mit der edelsten Eigenschaft in der menschlichen Natur an, deren Entwickelung und Kultur auf unendlichen Nutzen hinaussieht, und endigte – mit der Schwärmerei, oder dem Aber glauben. So geht es allen noch rohen Versuchen, in denen der vornehmste Teil des Geschäftes auf den Gebrauch der Vernunft ankommt, der nicht, so wie der Gebrauch der Füße, sich von selbst, vermittelst der öftern Ausübung, findet, vornehmlich wenn er Eigenschaften betrifft, die sich nicht so unmittelbar in der gemeinen Erfahrung darstellen lassen. Nachdem aber, wiewohl spät, die Maxime in Schwang gekommen war, alle Schritte vorher wohl zu überlegen, die die Vernunft zu tun vorhat, und sie nicht anders, als im Gleise einer vorher wohl überdachten Methode, ihren Gang machen zu lassen, so bekam die Beurteilung des Weltgebäudes eine ganz andere Richtung, und, mit dieser, zugleich einen, ohne Vergleichung, glücklichern Ausgang. Der Fall eines Steins, die \match{Bewegung} einer Schleuder, in ihre Elemente und dabei sich äußernde Kräfte aufgelöst, und mathematisch bearbeitet, brachte zuletzt diejenige klare und für alle Zukunft unveränderliche Einsicht in den Weltbau hervor, die, bei fortgehender Beobachtung, hoffen kann, sich immer nur zu erweitern, niemals aber, zurückgehen zu müssen, fürchten darf. 
	
	\unnumberedsection{Dauer (1)} 
	\subsection*{tg224.2.3} 
	\textbf{Source : }Kritik der praktischen Vernunft/Erster Teil. Elementarlehre der reinen praktischen Vernunft/Zweites Buch. Dialektik der reinen praktischen Vernunft/Zweites Hauptstück. Von der Dialektik der reinen Vernunft in Bestimmung des Begriffs vom höchsten Gut/VI. Über die Postulate der reinen praktischen Vernunft überhaupt\\  
	
	\textbf{Paragraphe : }Diese Postulate sind die der Unsterblichkeit, der Freiheit, positiv betrachtet (als der Kausalität eines Wesens, so fern es zur intelligibelen Welt gehört), und des Daseins Gottes. Das erste fließt aus der praktisch notwendigen Bedingung der Angemessenheit der \match{Dauer} zur Vollständigkeit der Erfüllung des moralischen Gesetzes; das zweite aus der notwendigen Voraussetzung der Unabhängigkeit von der Sinnenwelt und des Vermögens der Bestimmung seines Willens, nach dem Gesetze einer intelligibelen Welt, d.i. der Freiheit; das dritte aus der Notwendigkeit der Bedingung zu einer solchen intelligibelen Welt, um das höchste Gut zu sein, durch die Voraussetzung des höchsten selbständigen Guts, d.i. des Daseins Gottes. 
	
	\unnumberedsection{Definition (2)} 
	\subsection*{tg217.2.5} 
	\textbf{Source : }Kritik der praktischen Vernunft/Erster Teil. Elementarlehre der reinen praktischen Vernunft/Zweites Buch. Dialektik der reinen praktischen Vernunft/Erstes Hauptstück. Von einer Dialektik der reinen praktischen Vernunft überhaupt\\  
	
	\textbf{Paragraphe : }Diese Idee praktisch –, d.i. für die Maxime unseres vernünftigen Verhaltens, hinreichend zu bestimmen, ist die 
	Weisheitslehre, und diese wiederum, als Wissenschaft, ist Philosophie, in der Bedeutung, wie die Alten das Wort verstanden, bei denen sie eine Anweisung zu dem Begriffe war, worin das höchste Gut zu setzen, und zum Verhalten, durch welches es zu erwerben sei. Es wäre gut, wenn wir dieses Wort bei seiner alten Bedeutung ließen, als eine Lehre vom höchsten Gut, so fern die Vernunft bestrebt ist, es darin zur Wissenschaft zu bringen. Denn einesteils würde die angehängte einschränkende Bedingung dem griechischen Ausdrucke (welcher Liebe zur Weisheit bedeutet) angemessen und doch zugleich hinreichend sein, die Liebe zur Wissenschaft, mithin aller spekulativen Erkenntnis der Vernunft, so fern sie ihr, sowohl zu jenem Begriffe, als auch dem praktischen Bestimmungsgrunde dienlich ist, unter dem Namen der Philosophie, mit zu befassen, und doch den Hauptzweck, um dessentwillen sie allein Weisheitslehre genannt werden kann, nicht aus den Augen verlieren lassen. Anderen Teils würde es auch nicht übel sein, den Eigendünkel desjenigen, der es wagte, sich des Titels eines Philosophen selbst anzumaßen, abzuschrecken, wenn man ihm schon durch die \match{Definition} den Maßstab der Selbstschätzung vorhielte, der seine Ansprüche sehr herabstimmen wird; denn ein Weisheitslehrer zu sein, möchte wohl etwas mehr, als einen Schüler bedeuten, der noch immer nicht weit genug gekommen ist, um sich selbst, vielweniger um andere, mit sicherer Erwartung eines so hohen Zwecks, zu leiten; es würde einen Meister in Kenntnis der Weisheit bedeuten, welches mehr sagen will, als ein bescheidener Mann sich selber anmaßen wird, und Philosophie würde, so wie die Weisheit, selbst noch immer ein Ideal bleiben, welches objektiv in der Vernunft allein vollständig vorgestellt wird, subjektiv aber, für die Person, nur das Ziel seiner unaufhörlichen Bestrebung ist, und in dessen Besitz, unter dem angemaßten Namen eines Philosophen, zu sein, nur der vorzugeben berechtigt ist, der auch die unfehlbare Wirkung derselben (in Beherrschung seiner selbst, und dem ungezweifelten Interesse, das er vorzüglich am allgemeinen Guten nimmt) an seiner Person, als Beispiele, aufstellen kann, welches die Alten auch foderten, um jenen Ehrennamen verdienen zu können. 
	
	\subsection*{tg230.2.8} 
	\textbf{Source : }Kritik der praktischen Vernunft/Fußnoten\\  
	
	\textbf{Paragraphe : }
	
	4 Man könnte mir noch den Einwurf machen, warum ich nicht auch den Begriff des Begehrungsvermögens, oder des Gefühls der Lust vorher erklärt habe; obgleich dieser Vorwurf unbillig sein würde, weil man diese Erklärung, als in der Psychologie gegeben, billig sollte voraussetzen können. Es könnte aber freilich die \match{Definition} daselbst so eingerichtet sein, daß das Gefühl der Lust der Bestimmung des Begehrungsvermögens zum Grunde gelegt würde (wie es auch wirklich gemeinhin so zu geschehen pflegt), dadurch aber das oberste Prinzip der praktischen Philosophie notwendig empirisch ausfallen müßte, welches doch allererst auszumachen ist, und in dieser Kritik gänzlich widerlegt wird. Daher will ich diese Erklärung hier so geben, wie sie sein muß, um diesen streitigen Punkt, wie billig, im Anfange unentschieden zu lassen. – Leben ist das Vermögen eines Wesens, nach Gesetzen des Begehrungsvermögens zu handeln. Das Begehrungsvermögen ist das Vermögen desselben, durch seine Vorstellungen Ursache von der Wirklichkeit der Gegenstände dieser Vorstellungen zu sein. Lust ist die Vorstellung der Übereinstimmung des Gegenstandes oder der Handlung mit den subjektiven Bedingungen des Lebens, d.i. mit dem Vermögen der Kausalität einer Vorstellung in Ansehung der Wirklichkeit ihres Objekts (oder der Bestimmung der Kräfte des Subjekts zur Handlung, es hervorzubringen). Mehr brauche ich nicht zum Behuf der Kritik von Begriffen, die aus der Psychologie entlehnt werden, das übrige leistet die Kritik selbst. Man wird leicht gewahr, daß die Frage, ob die Lust dem Begehrungsvermögen jederzeit zum Grunde gelegt werden müsse, oder ob sie auch unter gewissen Bedingungen nur auf die Bestimmung desselben folge, durch diese Erklärung unentschieden bleibt; denn sie ist aus lauter Merkmalen des reinen Verstandes, d.i. Kategorien zusammengesetzt, die nichts Empirisches enthalten. Eine solche Behutsamkeit ist in der ganzen Philosophie sehr empfehlungswürdig, und wird dennoch oft verabsäumt, nämlich, seinen Urteilen vor der vollständigen Zergliederung des Begriffs, die oft nur sehr spät erreicht wird, durch gewagte Definition nicht vorzugreifen. Man wird auch durch den ganzen Lauf der Kritik (der theoretischen sowohl als praktischen Vernunft) bemerken, daß sich in demselben mannigfaltige Veranlassung vorfinde, manche Mängel im alten dogmatischen Gange der Philosophie zu ergänzen, und Fehler abzuändern, die nicht eher bemerkt werden, als wenn man von Begriffen einen Gebrauch der Vernunft macht, der aufs Ganze derselben geht. 
	
	\unnumberedsection{Entdeckung (2)} 
	\subsection*{tg197.2.15} 
	\textbf{Source : }Kritik der praktischen Vernunft/Vorrede\\  
	
	\textbf{Paragraphe : }Was Schlimmeres könnte aber diesen Bemühungen wohl nicht begegnen, als wenn jemand die unerwartete \match{Entdeckung} machte, daß es überall gar kein Erkenntnis a priori gebe, noch geben könne. Allein es hat hiemit keine Not. Es wäre eben so viel, als ob jemand durch Vernunft beweisen wollte, daß es keine Vernunft gebe. Denn wir sagen nur, daß wir etwas durch Vernunft erkennen, wenn wir uns bewußt sind, daß wir es auch hätten wissen können, wenn es uns auch nicht so in der Erfahrung vorgekommen wäre; mithin ist Vernunfterkenntnis und Erkenntnis a priori einerlei. Aus einem Erfahrungssatze Notwendigkeit (ex pumice aquam) auspressen wollen, mit dieser auch wahre Allgemeinheit (ohne welche kein Vernunftschluß, mithin auch nicht der Schluß aus der Analogie, welche eine wenigstens präsumierte Allgemeinheit und objektive Notwendigkeit ist, und diese also doch immer voraussetzt) einem Urteile verschaffen wollen, ist gerader Widerspruch. Subjektive Notwendigkeit, d.i. Gewohnheit, statt der objektiven, die nur in Urteilen a priori stattfindet, unterschieben, heißt der Vernunft das Vermögen absprechen, über den Gegenstand zu urteilen, d.i. ihn, und was ihm zukomme, zu erkennen, und z. B. von dem, was öfters und immer auf einen gewissen vorhergehenden Zustand folgte, nicht sagen, daß man aus diesem auf jenes schließen könne (denn das würde objektive Notwendigkeit und Begriff von einer Verbindung a priori bedeuten), sondern nur ähnliche Fälle (mit den Tieren auf ähnliche Art) erwarten dürfe, d.i. den Begriff der Ursache im Grunde als  falsch und bloßen Gedankenbetrug verwerfen. Diesem Mangel der objektiven und daraus folgenden allgemeinen Gültigkeit dadurch abhelfen wollen, daß man doch keinen Grund sähe, andern vernünftigen Wesen eine andere Vorstellungsart beizulegen, wenn das einen gültigen Schluß abgäbe, so würde uns unsere Unwissenheit mehr Dienste zu Erweiterung unserer Erkenntnis leisten, als alles Nachdenken. Denn bloß deswegen, weil wir andere vernünftige Wesen außer dem Menschen nicht kennen, würden wir ein Recht haben, sie als so beschaffen anzunehmen, wie wir uns erkennen, d.i. wir würden sie wirklich kennen. Ich erwähne hier nicht einmal, daß nicht die Allgemeinheit des Fürwahrhaltens die objektive Gültigkeit eines Urteils (d.i. die Gültigkeit desselben als Erkenntnisses) beweise, sondern, wenn jene auch zufälliger Weise zuträfe, dieses doch noch nicht einen Beweis der Übereinstimmung mit dem Objekt abgeben könne; vielmehr die objektive Gültigkeit allein den Grund einer notwendigen allgemeinen Einstimmung ausmache. 
	
	\subsection*{tg209.2.11} 
	\textbf{Source : }Kritik der praktischen Vernunft/Erster Teil. Elementarlehre der reinen praktischen Vernunft/Erstes Buch. Die Analytik der reinen praktischen Vernunft/Erstes Hauptstück. Von den Grundsätzen der reinen praktischen Vernunft/8. Lehrsatz IV\\  
	
	\textbf{Paragraphe : }Wenn ein dir sonst beliebter Umgangsfreund sich bei dir wegen eines falschen abgelegten Zeugnisses dadurch zu rechtfertigen vermeinete, daß er zuerst die, seinem Vorgeben nach, heilige Pflicht der eigenen Glückseligkeit vorschützte, alsdenn die Vorteile herzählte, die er sich alle dadurch erworben, die Klugheit namhaft machte, die er beobachtet, um wider alle \match{Entdeckung} sicher zu sein, selbst wider die von Seiten deiner selbst, dem er das Geheimnis darum allein offenbaret, damit er es zu aller Zeit ableugnen könne; dann aber im ganzen Ernst vorgäbe, er habe eine wahre Menschenpflicht ausgeübt: so würdest du ihm entweder gerade ins Gesicht lachen, oder mit Abscheu davon zurückbeben, ob du gleich, wenn jemand bloß auf eigene Vorteile seine Grundsätze gesteuert hat, wider diese Maßregeln nicht das mindeste einzuwenden hättest. Oder setzet, es empfehle euch jemand einen Mann zum Haushalter, dem ihr alle eure Angelegenheiten blindlings anvertrauen könnet, und, um euch Zutrauen einzuflößen, rühmete er ihn als einen klugen Menschen, der sich auf seinen eigenen Vorteil meisterhaft verstehe, auch als einen rastlos wirksamen, der keine Gelegenheit dazu ungenutzt vorbeigehen ließe, endlich, damit auch ja nicht Besorgnisse wegen eines pöbelhaften Eigennutzes desselben im Wege stünden, rühmete er, wie er recht fein zu leben verstünde, nicht im Geldsammeln oder brutaler Üppigkeit, sondern in der Erweiterung seiner Kenntnisse, einem wohlgewählten belehrenden Umgange, selbst im Wohltun der Dürftigen, sein Vergnügen suchte, übrigens aber wegen der Mittel (die doch ihren Wert oder Unwert nur vom Zwecke entlehnen) nicht bedenklich wäre, und fremdes Geld und Gut ihm hiezu, so bald er nur wisse,  daß er es unentdeckt und ungehindert tun könne, so gut wie sein eigenes wäre: so würdet ihr entweder glauben, der Empfehlende habe euch zum besten, oder er habe den Verstand verloren. – So deutlich und scharf sind die Grenzen der Sittlichkeit und der Selbstliebe abgeschnitten, daß selbst das gemeinste Auge den Unterschied, ob etwas zu der einen oder der andern gehöre, gar nicht verfehlen kann. Folgende wenige Bemerkungen können zwar bei einer so offenbaren Wahrheit überflüssig scheinen, allein sie dienen doch wenigstens dazu, dem Urteile der gemeinen Menschenvernunft etwas mehr Deutlichkeit zu verschaffen. 
	
	\unnumberedsection{Ewigkeit (1)} 
	\subsection*{tg223.2.2} 
	\textbf{Source : }Kritik der praktischen Vernunft/Erster Teil. Elementarlehre der reinen praktischen Vernunft/Zweites Buch. Dialektik der reinen praktischen Vernunft/Zweites Hauptstück. Von der Dialektik der reinen Vernunft in Bestimmung des Begriffs vom höchsten Gut/V. Das Dasein Gottes, als ein Postulat der reinen praktischen Vernunft\\  
	
	\textbf{Paragraphe : }Das moralische Gesetz führete in der vorhergehenden Zergliederung zur praktischen Aufgabe, welche, ohne allen Beitritt sinnlicher Triebfedern, bloß durch reine Vernunft vorgeschrieben wird, nämlich der notwendigen Vollständigkeit des ersten und vornehmsten Teils des höchsten Guts, der Sittlichkeit, und, da diese nur in einer \match{Ewigkeit} völlig aufgelöset werden kann, zum Postulat der Unsterblichkeit. Eben dieses Gesetz muß auch zur Möglichkeit des zweiten Elements des höchsten Guts, nämlich der jener Sittlichkeit angemessenen Glückseligkeit, eben so uneigennützig,  wie vorher, aus bloßer unparteiischer Vernunft, nämlich auf die Voraussetzung des Daseins einer dieser Wirkung adäquaten Ursache führen, d.i. die Existenz Gottes, als zur Möglichkeit des höchsten Guts (welches Objekt unseres Willens mit der moralischen Gesetzgebung der reinen Vernunft notwendig verbunden ist) notwendig gehörig, postulieren. Wir wollen diesen Zusammenhang überzeugend darstellen. 
	
	\unnumberedsection{Fortdauer (1)} 
	\subsection*{tg223.2.6} 
	\textbf{Source : }Kritik der praktischen Vernunft/Erster Teil. Elementarlehre der reinen praktischen Vernunft/Zweites Buch. Dialektik der reinen praktischen Vernunft/Zweites Hauptstück. Von der Dialektik der reinen Vernunft in Bestimmung des Begriffs vom höchsten Gut/V. Das Dasein Gottes, als ein Postulat der reinen praktischen Vernunft\\  
	
	\textbf{Paragraphe : }Die Lehre des Christentums
	
	
	13
	, wenn man sie auch noch nicht als Religionslehre betrachtet, gibt in diesem Stücke  einen Begriff des höchsten Guts (des Reichs Gottes), der allein der strengsten Foderung der praktischen Vernunft ein Gnüge tut. Das moralische Gesetz ist heilig (unnachsichtlich) und fodert Heiligkeit der Sitten, obgleich alle moralische Vollkommenheit, zu welcher der Mensch gelangen kann, immer nur Tugend ist, d.i. gesetzmäßige Gesinnung aus Achtung fürs Gesetz, folglich Bewußtsein eines kontinuierlichen Hanges zur Übertretung, wenigstens Unlauterkeit, d.i. Beimischung vieler unechter (nicht moralischer) Bewegungsgründe zur Befolgung des Gesetzes, folglich eine mit Demut verbundene Selbstschätzung, und also in Ansehung der Heiligkeit, welche das christliche Gesetz fodert, nichts als Fortschritt ins Unendliche dem Geschöpfe  übrig läßt, eben daher aber auch dasselbe zur Hoffnung seiner ins Unendliche gehenden \match{Fortdauer} berechtigt. Der Wert einer dem moralischen Gesetze völlig angemessenen Gesinnung ist unendlich; weil alle mögliche Glückseligkeit, im Urteile eines weisen und alles vermögenden Austeilers derselben, keine andere Einschränkung hat, als den Mangel der Angemessenheit vernünftiger Wesen an ihrer Pflicht. Aber das moralische Gesetz für sich verheißt doch keine Glückseligkeit; denn diese ist, nach Begriffen von einer Naturordnung überhaupt, mit der Befolgung desselben nicht notwendig verbunden. Die christliche Sittenlehre ergänzt nun diesen Mangel (des zweiten unentbehrlichen Bestandstücks des höchsten Guts) durch die Darstellung der Welt, darin vernünftige Wesen sich dem sittlichen Gesetze von ganzer Seele weihen, als eines Reichs Gottes, in welchem Natur und Sitten in eine, jeder von beiden für sich selbst fremde, Harmonie, durch einen heiligen Urheber kommen, der das abgeleitete höchste Gut möglich macht. Die Heiligkeit der Sitten wird ihnen in diesem Leben schon zur Richtschnur angewiesen, das dieser proportionierte Wohl aber, die Seligkeit, nur als in einer Ewigkeit erreichbar vorgestellt; weil jene immer das Urbild ihres Verhaltens in jedem Stande sein muß, und das Fortschreiten zu ihr schon in diesem Leben möglich und notwendig ist, diese aber in dieser Welt, unter dem Namen der Glückseligkeit, gar nicht erreicht werden kann (so viel auf unser Vermögen ankommt), und daher lediglich zum Gegenstande der Hoffnung gemacht wird. Diesem ungeachtet ist das christliche Prinzip der Moral selbst doch nicht theologisch (mithin Heteronomie), sondern Autonomie der reinen praktischen Vernunft für sich selbst, weil sie die Erkenntnis Gottes und seines Willens nicht zum Grunde dieser Gesetze, sondern nur der Gelangung zum höchsten Gute, unter der Bedingung der Befolgung derselben macht, und selbst die eigentliche Triebfeder zu Befolgung der ersteren nicht in den gewünschten Folgen derselben, sondern in der Vorstellung der Pflicht allein setzt, als in deren treuer Beobachtung die Würdigkeit des Erwerbs der letztern allein besteht. 
	
	\unnumberedsection{Genauigkeit (2)} 
	\subsection*{tg214.2.17} 
	\textbf{Source : }Kritik der praktischen Vernunft/Erster Teil. Elementarlehre der reinen praktischen Vernunft/Erstes Buch. Die Analytik der reinen praktischen Vernunft\\  
	
	\textbf{Paragraphe : }Es ist von der größten Wichtigkeit in allen moralischen Beurteilungen, auf das subjektive Prinzip aller Maximen mit der äußersten \match{Genauigkeit} Acht zu haben, damit alle Moralität der Handlungen in der Notwendigkeit derselben aus Pflicht und aus Achtung fürs Gesetz, nicht aus Liebe und Zuneigung zu dem, was die Handlungen hervorbringen sollen, gesetzt werde. Für Menschen und alle erschaffene vernünftige Wesen ist die moralische Notwendigkeit Nötigung, d.i. Verbindlichkeit, und jede darauf gegründete Handlung als Pflicht, nicht aber als eine uns von selbst schon beliebte, oder beliebt werden könnende Verfahrungsart vorzustellen. Gleich als ob wir es dahin jemals bringen könnten, daß ohne Achtung fürs Gesetz, welche mit Furcht oder wenigstens Besorgnis vor Übertretung verbunden ist, wir, wie die über alle Abhängigkeit erhabene Gottheit, von selbst, gleich sam durch eine uns zur Natur gewordene, niemals zu verrückende Übereinstimmung des Willens mit dem reinen Sittengesetze (welches also, da wir niemals versucht  werden können, ihm untreu zu werden, wohl endlich gar aufhören könnte, für uns Gebot zu sein), jemals in den Besitz einer Heiligkeit des Willens kommen könnten. 
	
	\subsection*{tg215.2.23} 
	\textbf{Source : }Kritik der praktischen Vernunft/Erster Teil. Elementarlehre der reinen praktischen Vernunft/Erstes Buch. Die Analytik der reinen praktischen Vernunft/Drittes Hauptstück. Von den Triebfedern der reinen praktischen Vernunft/Kritische Beleuchtung der Analytik der reinen praktischen Vernunft\\  
	
	\textbf{Paragraphe : }Nur auf eines sei es mir erlaubt bei dieser Gelegenheit noch aufmerksam zu machen, nämlich daß jeder Schritt, den man mit der reinen Vernunft tut, sogar im praktischen Felde, wo man auf subtile Spekulation gar nicht Rücksicht nimmt, dennoch sich so genau und zwar von selbst an alle Momente der Kritik der theoretischen Vernunft anschließe, als ob jeder mit überlegter Vorsicht, bloß um dieser Bestätigung zu verschaffen, ausgedacht wäre. Eine solche auf keinerlei Weise gesuchte, sondern (wie man sich selbst davon überzeugen kann, wenn man nur die moralischen Nachforschungen bis zu ihren Prinzipien fortsetzen will) sich von selbst findende, genaue Eintreffung der wichtigsten Sätze der praktischen Vernunft, mit denen oft zu subtil und unnötig scheinenden Bemerkungen der Kritik der spekulativen, überrascht und setzt in Verwunderung, und bestärkt die schon von andern erkannte und gepriesene Maxime, in jeder wissenschaftlichen Untersuchung mit aller möglichen \match{Genauigkeit} und Offenheit seinen Gang ungestört fortzusetzen,  ohne sich an das zu kehren, wowider sie außer ihrem Felde etwa verstoßen möchte, sondern sie für sich allein, so viel man kann, wahr und vollständig zu vollführen. Öftere Beobachtung hat mich überzeugt, daß, wenn man diese Geschäfte zu Ende gebracht hat, das, was in der Hälfte desselben, in Betracht anderer Lehren außerhalb, mir bisweilen sehr bedenklich schien, wenn ich diese Bedenklichkeit nur so lange aus den Augen ließ, und bloß auf mein Geschäft Acht hatte, bis es vollendet sei, endlich auf unerwartete Weise mit demjenigen vollkommen zusammenstimmte, was sich ohne die mindeste Rücksicht auf jene Lehren, ohne Parteilichkeit und Vorliebe für dieselbe, von selbst gefunden hatte. Schriftsteller würden sich manche Irrtümer, manche verlorne Mühe (weil sie auf Blendwerk gestellt war) ersparen, wenn sie sich nur entschließen könnten, mit etwas mehr Offenheit zu Werke zu gehen. 
	
	\unnumberedsection{Geometrie (1)} 
	\subsection*{tg208.2.8} 
	\textbf{Source : }Kritik der praktischen Vernunft/Erster Teil. Elementarlehre der reinen praktischen Vernunft/Erstes Buch. Die Analytik der reinen praktischen Vernunft/Erstes Hauptstück. Von den Grundsätzen der reinen praktischen Vernunft/7. Grundgesetz der reinen praktischen Vernunft\\  
	
	\textbf{Paragraphe : }Die reine \match{Geometrie} hat Postulate als praktische Sätze, die aber nichts weiter enthalten, als die Voraussetzung, daß man etwas tun könne, wenn etwa gefodert würde, man solle es tun, und diese sind die einzigen Sätze derselben, die ein Dasein betreffen. Es sind also praktische Regeln unter einer problematischen Bedingung des Willens. Hier aber sagt die Regel: man solle schlechthin auf gewisse Weise verfahren. Die praktische Regel ist also unbedingt, mithin, als kategorisch praktischer Satz, a priori vorgestellt, wodurch der Wille schlechterdings und unmittelbar (durch die praktische Regel selbst, die also hier Gesetz ist) objektiv bestimmt wird. Denn reine, an sich praktische Vernunft ist hier unmittelbar gesetzgebend. Der Wille wird als unabhängig von empirischen Bedingungen, mithin, als reiner Wille, durch die bloße Form des Gesetzes als bestimmt gedacht, und dieser Bestimmungsgrund als die oberste Bedingung aller Maximen angesehen. Die Sache ist befremdlich genug, und hat ihres gleichen in der ganzen übrigen praktischen Erkenntnis nicht. Denn der Gedanke a priori von einer möglichen allgemeinen Gesetzgebung, der also bloß problematisch ist, wird, ohne von der Erfahrung oder irgend einem äußeren Willen etwas zu entlehnen, als Gesetz unbedingt geboten. Es ist aber auch nicht eine Vorschrift, nach welcher eine Handlung geschehen soll, dadurch eine begehrte Wirkung möglich ist (denn da wäre die Regel immer physisch bedingt), sondern eine Regel, die bloß den Willen, in Ansehung der Form seiner Maximen, a priori bestimmt, und da ist ein Gesetz, welches bloß zum Behuf der subjektiven Form der Grundsätze dient, als Bestimmungsgrund durch die objektive Form eines Gesetzes überhaupt, wenigstens zu denken, nicht unmöglich. Man kann das Bewußtsein dieses Grundgesetzes ein Faktum der Vernunft nennen, weil man es nicht aus vorhergehenden Datis der Vernunft, z.B. dem Bewußtsein der Freiheit (denn dieses ist uns nicht vorher gegeben), herausvernünfteln kann, sondern weil es sich für sich selbst uns aufdringt als  synthetischer Satz a priori, der auf keiner, weder reinen noch empirischen Anschauung gegründet ist, ob er gleich analytisch sein würde, wenn man die Freiheit des Willens voraussetzte, wozu aber, als positivem Begriffe, eine intellektuelle Anschauung erfodert werden würde, die man hier gar nicht annehmen darf. Doch muß man, um dieses Gesetz ohne Mißdeutung als gegeben anzusehen, wohl bemerken: daß es kein empirisches, sondern das einzige Faktum der reinen Vernunft sei, die sich dadurch als ursprünglich gesetzgebend (sic volo, sic iubeo) ankündigt. 
	
	\unnumberedsection{Gewicht (1)} 
	\subsection*{tg214.2.8} 
	\textbf{Source : }Kritik der praktischen Vernunft/Erster Teil. Elementarlehre der reinen praktischen Vernunft/Erstes Buch. Die Analytik der reinen praktischen Vernunft\\  
	
	\textbf{Paragraphe : }
	Das moralische Gesetz also, so wie es formaler Bestimmungsgrund der Handlung ist, durch praktische reine Vernunft, so wie es zwar auch materialer, aber nur objektiver Bestimmungsgrund der Gegenstände der Handlung, unter dem Namen des Guten und Bösen, ist, so ist es auch subjektiver Bestimmungsgrund, d.i. Triebfeder, zu dieser Handlung, indem es auf die Sittlichkeit des Subjekts Einfluß hat, und ein Gefühl bewirkt, welches dem Einflusse des Gesetzes auf den Willen beförderlich ist. Hier geht kein Gefühl im Subjekt vorher, das auf Moralität gestimmt wäre. Denn das ist unmöglich, weil alles Gefühl sinnlich ist; die Triebfeder der sittlichen Gesinnung aber muß von aller sinnlichen Bedingung frei sein. Vielmehr ist das sinnliche Gefühl, was allen unseren Neigungen zum Grunde liegt, zwar die Bedingung derjenigen Empfindung, die wir Achtung nennen, aber die Ursache der Bestimmung desselben liegt in der reinen praktischen Vernunft, und diese Empfindung kann daher, ihres Ursprunges wegen, nicht pathologisch, sondern muß praktisch-gewirkt heißen; indem dadurch, daß die Vorstellung des moralischen Gesetzes der Selbstliebe den Einfluß, und dem Eigendünkel den Wahn benimmt, das Hindernis der reinen praktischen Vernunft vermindert, und die Vorstellung des Vorzuges ihres objektiven Gesetzes vor den Antrieben der Sinnlichkeit, mithin das \match{Gewicht} des ersteren relativ (in Ansehung eines durch die letztere affizierten Willens) durch die Wegschaffung des Gegengewichts, im Urteile der Vernunft, hervorgebracht wird. Und so ist die Achtung fürs Gesetz nicht Triebfeder zur Sittlichkeit, sondern sie ist die Sittlichkeit selbst, subjektiv als Triebfeder betrachtet, indem die reine praktische Vernunft dadurch, daß sie der Selbstliebe, im Gegensatze mit ihr, alle Ansprüche abschlägt, dem Gesetze, das jetzt allein Einfluß hat, Ansehen verschafft. Hiebei ist nun zu bemerken: daß, so wie die Achtung eine Wirkung aufs Gefühl, mithin auf die Sinnlichkeit eines vernünftigen Wesens ist, es diese Sinnlichkeit, mithin auch die Endlichkeit solcher Wesen, denen das moralische Gesetz Achtung auferlegt, voraussetze, und  daß einem höchsten, oder auch einem von aller Sinnlichkeit freien Wesen, welchem diese also auch kein Hindernis der praktischen Vernunft sein kann, Achtung fürs Gesetz nicht beigelegt werden könne. 
	
	\unnumberedsection{Gleichheit (1)} 
	\subsection*{tg202.2.5} 
	\textbf{Source : }Kritik der praktischen Vernunft/Erster Teil. Elementarlehre der reinen praktischen Vernunft/Erstes Buch. Die Analytik der reinen praktischen Vernunft/Erstes Hauptstück. Von den Grundsätzen der reinen praktischen Vernunft/1. Erklärung\\  
	
	\textbf{Paragraphe : }Wenn man annimmt, daß reine Vernunft einen praktisch, d.i. zur Willensbestimmung hinreichenden Grund in sich enthalten könne, so gibt es praktische Gesetze; wo aber nicht, so werden alle praktische Grundsätze bloße Maximen sein. In einem pathologisch-affizierten Willen eines vernünftigen Wesens kann ein Widerstreit der Maximen, wider die von ihm selbst erkannte praktische Gesetze, angetroffen werden. Z.B. es kann sich jemand zur Maxime machen, keine Beleidigung ungerächet zu erdulden, und doch zugleich einsehen, daß dieses kein praktisches Gesetz, sondern nur seine Maxime sei, dagegen, als Regel für den Willen eines jeden vernünftigen Wesens, in einer und derselben Maxime, mit sich selbst nicht zusammen stimmen könne. In der Naturerkenntnis sind die Prinzipien dessen, was geschieht (z.B. das Prinzip der \match{Gleichheit} der Wirkung und Gegenwirkung in der Mitteilung der Bewegung), zugleich Gesetze der Natur; denn der Gebrauch der Vernunft ist dort theoretisch und durch die Beschaffenheit des Objekts bestimmt. In der praktischen Erkenntnis, d.i. derjenigen, welche es bloß mit Bestimmungsgründen des Willens zu tun  hat, sind Grundsätze, die man sich macht, darum noch nicht Gesetze, darunter man unvermeidlich stehe, weil die Vernunft im Praktischen es mit dem Subjekte zu tun hat, nämlich dem Begehrungsvermögen, nach dessen besonderer Beschaffenheit sich die Regel vielfältig richten kann. – Die praktische Regel ist jederzeit ein Produkt der Vernunft, weil sie Handlung, als Mittel zur Wirkung, als Absicht vorschreibt. Diese Regel ist aber für ein Wesen, bei dem Vernunft nicht ganz allein Bestimmungsgrund des Willens ist, ein Imperativ, d.i. eine Regel, die durch ein Sollen, welches die objektive Nötigung der Handlung ausdrückt, bezeichnet wird, und bedeutet, daß, wenn die Vernunft den Willen gänzlich bestimmete, die Handlung unausbleiblich nach dieser Regel geschehen würde. Die Imperativen gelten also objektiv, und sind von Maximen, als subjektiven Grundsätzen, gänzlich unterschieden. Jene bestimmen aber entweder die Bedingungen der Kausalität des vernünftigen Wesens, als wirkender Ursache, bloß in Ansehung der Wirkung und Zulänglichkeit zu derselben, oder sie bestimmen nur den Willen, er mag zur Wirkung hinreichend sein oder nicht. Die erstere würden hypothetische Imperativen sein, und bloße Vorschriften der Geschicklichkeit enthalten; die zweiten würden dagegen kategorisch und allein praktische Gesetze sein. Maximen sind also zwar Grundsätze, aber nicht Imperativen. Die Imperativen selber aber, wenn sie bedingt sind, d.i. nicht den Willen schlechthin als Willen, sondern nur in Ansehung einer begehrten Wirkung bestimmen, d.i. hypothetische Imperativen sind, sind zwar praktische Vorschriften, aber keine Gesetze. Die letztern müssen den Willen als Willen, noch ehe ich frage, ob ich gar das zu einer begehrten Wirkung erforderliche Vermögen habe, oder, was mir, um diese hervorzubringen, zu tun sei, hinreichend bestimmen, mithin kategorisch sein, sonst sind es keine Gesetze; weil ihnen die Notwendigkeit fehlt, welche, wenn sie praktisch sein soll, von pathologischen, mithin dem Willen zufällig anklebenden Bedingungen unabhängig sein muß. Saget jemanden, z.B., daß er in der Jugend arbeiten und sparen müsse, um im Alter nicht  zu darben: so ist dieses eine richtige und zugleich wichtige praktische Vorschrift des Willens. Man sieht aber leicht, daß der Wille hier auf etwas anderes verwiesen werde, wovon man voraussetzt, daß er es begehre, und dieses Begehren muß man ihm, dem Täter selbst, überlassen, ob er noch andere Hülfsquellen, außer seinem selbst erworbenen Vermögen, vorhersehe, oder ob er gar nicht hoffe, alt zu werden, oder sich denkt im Falle der Not dereinst schlecht behelfen zu können. Die Vernunft, aus der allein alle Regel, die Notwendigkeit enthalten soll, entspringen kann, legt in diese ihre Vorschrift zwar auch Notwendigkeit (denn ohne das wäre sie kein Imperativ), aber diese ist nur subjektiv bedingt, und man kann sie nicht in allen Subjekten in gleichem Grade voraussetzen. Zu ihrer Gesetzgebung aber wird erfodert, daß sie bloß sich selbst vorauszusetzen bedürfe, weil die Regel nur alsdenn objektiv und allgemein gültig ist, wenn sie ohne zufällige, subjektive Bedingungen gilt, die ein vernünftig Wesen von dem anderen unterscheiden. Nun sagt jemanden: er solle niemals lügenhaft versprechen, so ist dies eine Regel, die bloß seinen Willen betrifft; die Absichten, die der Mensch haben mag, mögen durch denselben erreicht werden können, oder nicht; das bloße Wollen ist das, was durch jene Regel völlig a priori bestimmt werden soll. Findet sich nun, daß diese Regel praktisch richtig sei, so ist sie ein Gesetz, weil sie ein kategorischer Imperativ ist. Also beziehen sich praktische Gesetze allein auf den Willen, unangesehen dessen, was durch die Kausalität desselben ausgerichtet wird, und man kann von der letztern (als zur Sinnenwelt gehörig) abstrahieren, um sie rein zu haben. 
	
	\unnumberedsection{Grad (1)} 
	\subsection*{tg228.2.6} 
	\textbf{Source : }Kritik der praktischen Vernunft/Zweiter Teil. Methodenlehre der reinen praktischen Vernunft\\  
	
	\textbf{Paragraphe : }Wenn man auf den Gang der Gespräche in gemischten Gesellschaften, die nicht bloß aus Gelehrten und Vernünftlern, sondern auch aus Leuten von Geschäften oder Frauenzimmer bestehen, Acht hat, so bemerkt man, daß, außer dem Erzählen und Scherzen, noch eine Unterhaltung, nämlich das Räsonieren, darin Platz findet; weil das erstere, wenn es Neuigkeit, und, mit ihr, Interesse bei sich führen soll, bald erschöpft, das zweite aber leicht schal wird. Unter allem Räsonieren ist aber keines, was mehr den Beitritt der Personen, die sonst bei allem Vernünfteln bald lange Weile haben, erregt, und eine gewisse Lebhaftigkeit in die Gesellschaft bringt, als das über den sittlichen Wert dieser oder jener Handlung, dadurch der Charakter irgend einer Person ausgemacht werden soll. Diejenige, welchen sonst alles Subtile und Grüblerische in theoretischen Fragen trocken und verdrießlich ist, treten bald bei, wenn es darauf ankommt, den moralischen Gehalt einer erzählten guten oder bösen Handlung auszumachen, und sind so genau, so grüblerisch, so subtil, alles, was die Reinigkeit der Absicht, und mithin  den \match{Grad} der Tugend in derselben vermindern, oder auch nur verdächtig machen könnte, auszusinnen, als man bei keinem Objekte der Spekulation sonst von ihnen erwartet. Man kann in diesen Beurteilungen oft den Charakter der über andere urteilenden Personen selbst hervorschimmern sehen, deren einige vorzüglich geneigt scheinen, indem sie ihr Richteramt, vornehmlich über Verstorbene, ausüben, das Gute, was von dieser oder jener Tat derselben erzählt wird, wider alle kränkende Einwürfe der Unlauterkeit und zuletzt den ganzen sittlichen Wert der Person wider den Vorwurf der Verstellung und geheimen Bösartigkeit zu verteidigen, andere dagegen mehr auf Anklagen und Beschuldigungen sinnen, diesen Wert anzufechten. Doch kann man den letzteren nicht immer die Absicht beimessen, Tugend aus allen Beispielen der Menschen gänzlich wegvernünfteln zu wollen, um sie dadurch zum leeren Namen zu machen, sondern es ist oft nur wohlgemeinte Strenge in Bestimmung des echten sittlichen Gehalts, nach einem unnachsichtlichen Gesetze, mit welchem und nicht mit Beispielen verglichen der Eigendünkel im Moralischen sehr sinkt, und Demut nicht etwa bloß gelehrt, sondern bei scharfer Selbstprüfung von jedem gefühlt wird. Dennoch kann man den Verteidigern der Reinigkeit der Absicht in gegebenen Beispielen es mehrenteils ansehen, daß sie ihr da, wo sie die Vermutung der Rechtschaffenheit für sich hat, auch den mindesten Fleck gerne abwischen möchten, aus dem Bewegungsgrunde, damit nicht, wenn allen Beispielen ihre Wahrhaftigkeit gestritten und aller menschlichen Tugend die Lauterkeit weggeleugnet würde, diese nicht endlich gar für ein bloßes Hirngespinst gehalten, und so alle Bestrebung zu derselben als eitles Geziere und trüglicher Eigendünkel geringschätzig gemacht werde. 
	
	\unnumberedsection{Größe (1)} 
	\subsection*{tg215.2.22} 
	\textbf{Source : }Kritik der praktischen Vernunft/Erster Teil. Elementarlehre der reinen praktischen Vernunft/Erstes Buch. Die Analytik der reinen praktischen Vernunft/Drittes Hauptstück. Von den Triebfedern der reinen praktischen Vernunft/Kritische Beleuchtung der Analytik der reinen praktischen Vernunft\\  
	
	\textbf{Paragraphe : }Da es eigentlich der Begriff der Freiheit ist, der, unter allen Ideen der reinen spekulativen Vernunft, allein so große Erweiterung im Felde des Übersinnlichen, wenn gleich nur in Ansehung des praktischen Erkenntnisses verschafft, so frage ich mich: woher denn ihm ausschließungsweise eine so große Fruchtbarkeit zu Teil geworden sei, indessen die übrigen zwar die leere Stelle für reine mögliche Verstandeswesen bezeichnen, den Begriff von ihnen aber durch nichts bestimmen können. Ich begreife bald, daß, da ich nichts ohne Kategorie denken kann, diese auch in der Idee der Vernunft von der Freiheit, mit der ich mich beschäftige, zuerst müsse aufgesucht werden, welche hier die Kategorie der Kausalität ist, und daß ich, wenn gleich dem Vernunftbegriffe der Freiheit, als überschwenglichem Begriffe, keine korrespondierende Anschauung untergelegt werden kann, dennoch dem Verstandesbegriffe (der Kausalität), für dessen Synthesis jener das Unbedingte fodert, zuvor eine sinnliche Anschauung gegeben werden müsse, dadurch ihm zuerst die objektive Realität gesichert wird. Nun sind alle Kategorien in zwei Klassen, die mathematische, welche bloß auf die Einheit der Synthesis in der Vorstellung der Objekte, und die dynamische, welche auf die in der Vorstellung der Existenz der Objekte gehen, eingeteilt. Die erstere (die der \match{Größe} und der Qualität) enthalten jederzeit eine Synthesis des Gleichartigen, in welcher das Unbedingte, zu dem in der sinnlichen Anschauung gegebenen Bedingten in Raum und Zeit, da es selbst wiederum zum Raume und der Zeit gehören, und also  immer wieder unbedingt sein mußte, gar nicht kann gefunden werden; daher auch in der Dialektik der reinen theoretischen Vernunft die einander entgegengesetzte Arten, das Unbedingte und die Totalität der Bedingungen für sie zu finden, beide falsch waren. Die Kategorien der zweiten Klasse (die der Kausalität und der Notwendigkeit eines Dinges) erforderten diese Gleichartigkeit (des Bedingten und der Bedingung in der Synthesis) gar nicht, weil hier nicht die Anschauung, wie sie aus einem Mannigfaltigen in ihr zusammengesetzt, sondern nur, wie die Existenz des ihr korrespondierenden bedingten Gegenstandes zu der Existenz der Bedingung (im Verstande als damit verknüpft) hinzukomme, vorgestellt werden solle, und da war es erlaubt, zu dem durchgängig Bedingten in der Sinnenwelt (so wohl in Ansehung der Kausalität als des zufälligen Daseins der Dinge selbst) das Unbedingte, obzwar übrigens unbestimmt, in der intelligibelen Welt zu setzen, und die Synthesis transzendent zu machen; daher denn auch in der Dialektik der r. spek. V. sich fand, daß beide, dem Scheine nach, einander entgegengesetzte Arten, das Unbedingte zum Bedingten zu finden, z.B. in der Synthesis der Kausalität zum Bedingten, in der Reihe der Ursachen und Wirkungen der Sinnenwelt, die Kausalität, die weiter nicht sinnlich bedingt ist, zu denken, sich in der Tat nicht widerspreche, und daß dieselbe Handlung, die, als zur Sinnenwelt gehörig, jederzeit sinnlich bedingt, d.i. mechanisch-notwendig ist, doch zugleich auch, als zur Kausalität des handelnden Wesens, so fern es zur intelligibelen Welt gehörig ist, eine sinnlich unbedingte Kausalität zum Grunde haben, mithin als frei gedacht werden könne. Nun kam es bloß darauf an, daß dieses Können in ein Sein verwandelt würde, d.i., daß man in einem wirklichen Falle, gleichsam durch ein Faktum, beweisen könne: daß gewisse Handlungen eine solche Kausalität (die intellektuelle, sinnlich unbedingte) voraussetzen, sie mögen nun wirklich, oder auch nur geboten, d.i. objektiv praktisch notwendig sein. An wirklich in der Erfahrung  gegebenen Handlungen, als Begebenheiten der Sinnenwelt, konnten wir diese Verknüpfung nicht anzutreffen hoffen, weil die Kausalität durch Freiheit immer außer der Sinnenwelt im Intelligibelen gesucht werden muß. Andere Dinge, außer den Sinnenwesen, sind uns aber zur Wahrnehmung und Beobachtung nicht gegeben. Also blieb nichts übrig, als daß etwa ein unwidersprechlicher und zwar objektiver Grundsatz der Kausalität, welcher alle sinnliche Bedingung von ihrer Bestimmung ausschließt, d.i. ein Grundsatz, in welchem die Vernunft sich nicht weiter auf etwas anderes als Bestimmungsgrund in Ansehung der Kausalität beruft, sondern den sie durch jenen Grundsatz schon selbst enthält, und wo sie also, als reine Vernunft, selbst praktisch ist, gefunden werde. Dieser Grundsatz aber bedarf keines Suchens und keiner Erfindung; er ist längst in aller Menschen Vernunft gewesen und ihrem Wesen einverleibt, und ist der Grundsatz der Sittlichkeit. Also ist jene unbedingte Kausalität und das Vermögen derselben, die Freiheit, mit dieser aber ein Wesen (ich selber), welches zur Sinnenwelt gehört, doch zugleich als zur intelligibelen gehörig nicht bloß unbestimmt und problematisch gedacht (welches schon die spekulative Vernunft als tunlich ausmitteln konnte), sondern sogar in Ansehung des Gesetzes ihrer Kausalität bestimmt und assertorisch erkannt, und so uns die Wirklichkeit der intelligibelen Welt, und zwar in praktischer Rücksicht bestimmt, gegeben worden, und diese Bestimmung, die in theoretischer Absicht transzendent (überschwenglich)sein würde, ist in praktischer immanent. Dergleichen Schritt aber konnten wir in Ansehung der zweiten dynamischen Idee, nämlich der eines notwendigen Wesens nicht tun. Wir konnten zu ihm aus der Sinnenwelt, ohne Vermittelung der ersteren dyn. Idee, nicht hinauf kommen. Denn, wollten wir es versuchen, so müßten wir den Sprung gewagt haben, alles das, was uns gegeben ist, zu verlassen, und uns zu dem hinzuschwingen, wovon uns auch nichts gegeben ist, wodurch wir die Verknüpfung eines solchen intelligibelen Wesens mit der Sinnenwelt vermitteln könnten (weil das notwendige Wesen als außer uns gegeben  erkannt werden sollte); welches dagegen in Ansehung unseres eignen Subjekts, so fern es sich durchs moralische Gesetz einerseits als intelligibeles Wesen (vermöge der Freiheit) bestimmt, andererseits als nach dieser Bestimmung in der Sinnenwelt tätig, selbst erkennt, wie jetzt der Augenschein dartut, ganz wohl möglich ist. Der einzige Begriff der Freiheit verstattet es, daß wir nicht außer uns hinausgehen dürfen, um das Unbedingte und Intelligibele zu dem Bedingten und Sinnlichen zu finden. Denn es ist unsere Vernunft selber, die sich durchs höchste und unbedingte praktische Gesetz, und das Wesen, das sich dieses Gesetzes bewußt ist (unsere eigene Person), als zur reinen Verstandeswelt gehörig, und zwar sogar mit Bestimmung der Art, wie es als ein solches tätig sein könne, erkennt. So läßt sich begreifen, warum in dem ganzen Vernunftvermögen nur das Praktische dasjenige sein könne, welches uns über die Sinnenwelt hinaushilft, und Erkenntnisse von einer übersinnlichen Ordnung und Verknüpfung verschaffe, die aber eben darum freilich nur so weit, als es gerade für die reine praktische Absicht nötig ist, ausgedehnt werden können. 
	
	\unnumberedsection{Identität (3)} 
	\subsection*{tg218.2.4} 
	\textbf{Source : }Kritik der praktischen Vernunft/Erster Teil. Elementarlehre der reinen praktischen Vernunft/Zweites Buch. Dialektik der reinen praktischen Vernunft\\  
	
	\textbf{Paragraphe : }Von den alten griechischen Schulen waren eigentlich nur zwei, die in Bestimmung des Begriffs vom höchsten Gute so fern zwar einerlei Methode befolgten, daß sie Tugend und Glückseligkeit nicht als zwei verschiedene Elemente des höchsten Guts gelten ließen, mithin die Einheit des Prinzips nach der Regel der \match{Identität} suchten; aber darin schieden sie sich wiederum, daß sie unter beiden den Grundbegriff verschiedentlich wählten. Der Epikureer sagte: sich seiner auf Glückseligkeit führenden Maxime bewußt sein, das ist Tugend; der Stoiker: sich seiner Tugend bewußt sein, ist Glückseligkeit. Dem erstern war Klugheit so viel als  Sittlichkeit; dem zweiten, der eine höhere Benennung für die Tugend wählete, war Sittlichkeit allein wahre Weisheit. 
	
	\subsection*{tg218.2.5} 
	\textbf{Source : }Kritik der praktischen Vernunft/Erster Teil. Elementarlehre der reinen praktischen Vernunft/Zweites Buch. Dialektik der reinen praktischen Vernunft\\  
	
	\textbf{Paragraphe : }Man muß bedauren, daß die Scharfsinnigkeit dieser Männer (die man doch zugleich darüber bewundern muß, daß sie in so frühen Zeiten schon alle erdenkliche Wege philosophischer Eroberungen versuchten) unglücklich angewandt war, zwischen äußerst ungleichartigen Begriffen, dem der Glückseligkeit und dem der Tugend, \match{Identität} zu ergrübeln. Allein es war dem dialektischen Geiste ihrer Zeiten angemessen, was auch jetzt bisweilen subtile Köpfe verleitet, wesentliche und nie zu vereinigende Unterschiede in Prinzipien dadurch aufzuheben, daß man sie in Wortstreit zu verwandeln sucht, und so, dem Scheine nach, Einheit des Begriffs bloß unter verschiedenen Benennungen erkünstelt, und dieses trifft gemeiniglich solche Fälle, wo die Vereinigung ungleichartiger Gründe so tief oder hoch liegt, oder eine so gänzliche Umänderung der sonst im philosophischen System angenommenen Lehren erfodern würde, daß man Scheu trägt, sich in den realen Unterschied tief einzulassen, und ihn lieber als Uneinigkeit in bloßen Formalien behandelt. 
	
	\subsection*{tg218.2.6} 
	\textbf{Source : }Kritik der praktischen Vernunft/Erster Teil. Elementarlehre der reinen praktischen Vernunft/Zweites Buch. Dialektik der reinen praktischen Vernunft\\  
	
	\textbf{Paragraphe : }Indem beide Schulen Einerleiheit der praktischen Prinzipien der Tugend und Glückseligkeit zu ergrübeln suchten, so waren sie darum nicht unter sich einhellig, wie sie diese \match{Identität} herauszwingen wollten, sondern schieden sich in unendliche Weiten von einander, indem die eine ihr Prinzip auf der ästhetischen, die andere auf der logischen Seite, jene im Bewußtsein der sinnlichen Bedürfnis, die andere in der Unabhängigkeit der praktischen Vernunft von allen sinnlichen Bestimmungsgründen setzte. Der Begriff der Tugend lag, nach dem Epikureer, schon in der Maxime, seine eigene Glückseligkeit zu befördern; das Gefühl der Glückseligkeit war dagegen nach dem Stoiker schon im Bewußtsein seiner Tugend enthalten. Was aber in einem andern Begriffe enthalten ist, ist zwar mit einem Teile des Enthaltenden, aber nicht mit dem Ganzen einerlei, und zween Ganze können überdem spezifisch von einander unterschieden sein, ob sie zwar aus eben demselben Stoffe bestehen, wenn nämlich die Teile in beiden auf ganz verschiedene Art zu einem  Ganzen verbunden werden. Der Stoiker behauptete, Tugend sei das ganze höchste Gut, und Glückseligkeit nur das Bewußtsein des Besitzes derselben, als zum Zustand des Subjekts gehörig. Der Epikureer behauptete, Glückseligkeit sei das ganze höchste Gut, und Tugend nur die Form der Maxime, sich um sie zu bewerben, nämlich im vernünftigen Gebrauche der Mittel zu derselben. 
	
	\unnumberedsection{Konstruktion (1)} 
	\subsection*{tg215.2.6} 
	\textbf{Source : }Kritik der praktischen Vernunft/Erster Teil. Elementarlehre der reinen praktischen Vernunft/Erstes Buch. Die Analytik der reinen praktischen Vernunft/Drittes Hauptstück. Von den Triebfedern der reinen praktischen Vernunft/Kritische Beleuchtung der Analytik der reinen praktischen Vernunft\\  
	
	\textbf{Paragraphe : }Die Unterscheidung der Glückseligkeit sichre von der Sittenlehre, in derer ersteren empirische Prinzipien das ganze Fundament, von der zweiten aber auch nicht den mindesten Beisatz derselben ausmachen, ist nun in der Analytik der reinen praktischen Vernunft die erste und wichtigste ihr obliegende Beschäftigung, in der sie so pünktlich, ja, wenn es auch hieße, peinlich, verfahren muß, als je der Geometer in seinem Geschäfte. Es kommt aber dem Philosophen, der hier (wie jederzeit im Vernunfterkenntnisse durch bloße Begriffe, ohne \match{Konstruktion} derselben) mit größerer Schwierigkeit zu kämpfen hat, weil er keine Anschauung  (reinem Noumen) zum Grunde legen kann, doch auch zu statten: daß er, beinahe wie der Chemist, zu aller Zeit ein Experiment mit jedes Menschen praktischer Vernunft anstellen kann, um den moralischen (reinen) Bestimmungsgrund vom empirischen zu unterscheiden; wenn er nämlich zu dem empirisch-affizierten Willen (z.B. desjenigen, der gerne lügen möchte, weil er sich dadurch was erwerben kann) das moralische Gesetz (als Bestimmungsgrund) zusetzt. Es ist, als ob der Scheidekünstler der Solution der Kalkerde in Salzgeist Alkali zusetzt; der Salzgeist verläßt so fort den Kalk, vereinigt sich mit dem Alkali, und jener wird zu Boden gestürzt. Eben so haltet dem, der sonst ein ehrlicher Mann ist (oder sich doch diesmal nur in Gedanken in die Stelle eines ehrlichen Mannes versetzt), das moralische Gesetz vor, an dem er die Nichtswürdigkeit eines Lügners erkennt, so fort verläßt seine praktische Vernunft (im Urteil über das, was von ihm geschehen sollte) den Vorteil, vereinigt sich mit dem, was ihm die Achtung für seine eigene Person erhält (der Wahrhaftigkeit), und der Vorteil wird nun von jedermann, nachdem er von allem Anhängsel der Vernunft (welche nur gänzlich auf der Seite der Pflicht ist) abgesondert und gewaschen worden, gewogen, um mit der Vernunft noch wohl in anderen Fällen in Verbindung zu treten, nur nicht, wo er dem moralischen Gesetze, welches die Vernunft niemals verläßt, sondern sich innigst damit vereinigt, zuwider sein könnte. 
	
	\unnumberedsection{Kraft (2)} 
	\subsection*{tg214.2.28} 
	\textbf{Source : }Kritik der praktischen Vernunft/Erster Teil. Elementarlehre der reinen praktischen Vernunft/Erstes Buch. Die Analytik der reinen praktischen Vernunft\\  
	
	\textbf{Paragraphe : }So ist die echte Triebfeder der reinen praktischen Vernunft beschaffen; sie ist keine andere, als das reine moralische Gesetz selber, so fern es uns die Erhabenheit unserer eigenen übersinnlichen Existenz spüren läßt, und subjektiv,  in Menschen, die sich zugleich ihres sinnlichen Daseins und der damit verbundenen Abhängigkeit von ihrer so fern sehr pathologisch affizierten Natur bewußt sind, Achtung für ihre höhere Bestimmung wirkt. Nun lassen sich mit dieser Triebfeder gar wohl so viele Reize und Annehmlichkeiten des Lebens verbinden, daß auch um dieser willen allein schon die klügste Wahl eines vernünftigen und über das größte Wohl des Lebens nachdenkenden Epikureers sich für das sittliche Wohlverhalten erklären würde, und es kann auch ratsam sein, diese Aussicht auf einen fröhlichen Genuß des Lebens mit jener obersten und schon für sich allein hinlänglich-bestimmenden Bewegursache zu verbinden; aber nur um den Anlockungen, die das Laster auf der Gegenseite vorzuspiegeln nicht ermangelt, das Gegengewicht zu halten, nicht um hierin die eigentliche bewegende Kraft, auch nicht dem mindesten Teile nach, zu setzen, wenn von Pflicht die Rede ist. Denn das würde so viel sein, als die moralische Gesinnung in ihrer Quelle verunreinigen wollen. Die Ehrwürdigkeit der Pflicht hat nichts mit Lebensgenuß zu schaffen; sie hat ihr eigentümliches Gesetz, auch ihr eigentümliches Gericht, und wenn man auch beide noch so sehr zusammenschütteln wollte, um sie vermischt, gleichsam als Arzeneimittel, der kranken Seele zuzureichen, so scheiden sie sich doch alsbald von selbst, und, tun sie es nicht, so wirkt das erste gar nicht, wenn aber auch das physische Leben hiebei einige \match{Kraft} gewönne, so würde doch das moralische ohne Rettung dahin schwinden. 
	
	\subsection*{tg228.2.11} 
	\textbf{Source : }Kritik der praktischen Vernunft/Zweiter Teil. Methodenlehre der reinen praktischen Vernunft\\  
	
	\textbf{Paragraphe : }Laßt uns nun im Beispiele sehen, ob in der Vorstellung einer Handlung als edler und großmütiger Handlung mehr subjektiv bewegende \match{Kraft} einer Triebfeder liege, als, wenn diese bloß als Pflicht in Verhältnis auf das ernste moralische Gesetz vorgestellt wird. Die Handlung, da jemand, mit der größten Gefahr des Lebens, Leute aus dem Schiffbruche zu retten sucht, wenn er zuletzt dabei selbst sein Leben einbüßt, wird zwar einerseits zur Pflicht, andererseits aber und größtenteils auch für verdienstliche Handlung angerechnet, aber unsere Hochschätzung derselben wird gar sehr durch den Begriff von Pflicht gegen sich selbst, welche hier etwas Abbruch zu leiden scheint, geschwächt. Entscheidender ist die großmütige Aufopferung seines Lebens zur Erhaltung des Vaterlandes, und doch, ob es auch so vollkommen Pflicht sei, sich von selbst und unbefohlen dieser Absicht zu weihen, darüber bleibt einiger Skrupel übrig, und die Handlung hat nicht die ganze Kraft eines Musters und Antriebes zur Nachahmung in sich. Ist es aber unerläßliche Pflicht, deren Übertretung das moralische Gesetz an sich und ohne Rücksicht auf Menschenwohl verletzt, und dessen Heiligkeit gleichsam mit Füßen tritt (dergleichen Pflichten man Pflichten gegen Gott zu nennen pflegt, weil wir uns in ihm das Ideal der Heiligkeit in Substanz denken), so widmen  wir der Befolgung desselben, mit Aufopferung alles dessen, was für die innigste aller unserer Neigungen nur immer einen Wert haben mag, die allervollkommenste Hochachtung, und wir finden unsere Seele durch ein solches Beispiel gestärkt und erhoben, wenn wir an demselben uns überzeugen können, daß die menschliche Natur zu einer so großen Erhebung über alles, was Natur nur immer an Triebfedern zum Gegenteil aufbringen mag, fähig sei. Juvenal stellt ein solches Beispiel in einer Steigerung vor, die den Leser die Kraft der Triebfeder, die im reinen Gesetze der Pflicht, als Pflicht, steckt, lebhaft empfinden läßt: 
	
	\unnumberedsection{Licht (1)} 
	\subsection*{tg197.2.8} 
	\textbf{Source : }Kritik der praktischen Vernunft/Vorrede\\  
	
	\textbf{Paragraphe : }
	Hiedurch verstehe ich auch, warum die erheblichsten Einwürfe wider die Kritik, die mir bisher noch vorgekommen sind, sich gerade um diese zwei Angel drehen: nämlich, einerseits, im theoretischen Erkenntnis geleugnete und im praktischen behauptete objektive Realität der auf Noumenen angewandten Kategorien, andererseits die paradoxe Foderung, sich als Subjekt der Freiheit zum Noumen, zugleich aber auch in Absicht auf die Natur zum Phänomen in seinem eigenen empirischen Bewußtsein zu machen. Denn, so lange man sich noch keine bestimmte Begriffe von Sittlichkeit und Freiheit machte, konnte man nicht erraten, was man einerseits der vorgeblichen Erscheinung als Noumen zum Grunde legen wolle, und andererseits, ob es überall auch möglich sei, sich noch von ihm einen Begriff zu machen, wenn man vorher alle Begriffe des reinen Verstandes im theoretischen Gebrauche schon ausschließungsweise den bloßen Erscheinungen gewidmet hätte. Nur eine ausführliche Kritik der praktischen Vernunft kann alle diese Mißdeutung heben, und die konsequente Denkungsart, welche eben ihren größten Vorzug ausmacht, in ein helles \match{Licht} setzen. 
	
	\unnumberedsection{Mathematik (7)} 
	\subsection*{tg197.2.16} 
	\textbf{Source : }Kritik der praktischen Vernunft/Vorrede\\  
	
	\textbf{Paragraphe : }
	Hume würde sich bei diesem System des allgemeinen Empirisms in Grundsätzen auch sehr wohl befinden; denn er verlangte, wie bekannt, nichts mehr, als daß, statt aller objektiven Bedeutung der Notwendigkeit im Begriffe der Ursache, eine bloß subjektive, nämlich Gewohnheit, angenommen werde, um der Vernunft alles Urteil über Gott, Freiheit und Unsterblichkeit abzusprechen; und er verstand sich gewiß sehr gut darauf, um, wenn man ihm nur die Prinzipien zugestand, Schlüsse mit aller logischen Bündigkeit daraus zu folgern. Aber so allgemein hat selbst Hume den Empirism nicht gemacht, um auch die \match{Mathematik} darin einzuschließen. Er hielt ihre Sätze für analytisch, und, wenn das seine Richtigkeit hätte, würden sie in der Tat auch apodiktisch sein, gleichwohl aber daraus kein Schluß auf ein Vermögen der Vernunft, auch in der Philosophie apodiktische Urteile, nämlich solche, die synthetisch wären (wie der Satz der Kausalität), zu fällen, gezogen werden können. Nähme man aber den Empirism der Prinzipien allgemein an, so wäre auch Mathematik damit eingeflochten. 
	
	\subsection*{tg197.2.17} 
	\textbf{Source : }Kritik der praktischen Vernunft/Vorrede\\  
	
	\textbf{Paragraphe : }
	Wenn nun diese mit der Vernunft, die bloß empirische Grundsätze zuläßt, in Widerstreit gerät, wie dieses in der Antinomie, da \match{Mathematik} die unendliche Teilbarkeit des Raumes unwidersprechlich beweiset, der Empirism aber sie nicht verstatten kann, unvermeidlich ist: so ist die größte mögliche Evidenz der Demonstration, mit den vorgeblichen Schlüssen aus Erfahrungsprinzipien, in offenbarem Widerspruch, und nun muß man, wie der Blinde des Cheselden fragen: was betrügt mich, das Gesicht oder Gefühl? (Denn der Empirism gründet sich auf einer gefühlten, der Rationalism aber auf einer eingesehenen Notwendigkeit.) Und so offenbaret sich der allgemeine Empirism als den echten Skeptizism, den man dem Hume fälschlich in so unbeschränkter Bedeutung beilegte,
	
	
	6
	da er wenigstens einen sicheren Probierstein der Erfahrung an der Mathematik übrig ließ, statt daß jener schlechterdings keinen Probierstein derselben (der immer nur in Prinzipien a priori angetroffen werden kann) verstattet, obzwar diese doch nicht aus bloßen Gefühlen, sondern auch aus Urteilen besteht. 
	
	\subsection*{tg211.2.4} 
	\textbf{Source : }Kritik der praktischen Vernunft/Erster Teil. Elementarlehre der reinen praktischen Vernunft/Erstes Buch. Die Analytik der reinen praktischen Vernunft/Erstes Hauptstück. Von den Grundsätzen der reinen praktischen Vernunft/II. Von dem Befugnisse der reinen Vernunft, im praktischen Gebrauche, zu einer Erweiterung, die ihr im spekulativen für sich nicht möglich ist\\  
	
	\textbf{Paragraphe : }Die \match{Mathematik} war so lange noch gut weggekommen, weil Hume dafür hielt, daß ihre Sätze alle analytisch wären, d.i. von einer Bestimmung zur andern, um der Identität willen, mithin nach dem Satze des Widerspruchs fortschritten (welches aber falsch ist, indem sie vielmehr alle synthetisch sind, und, obgleich z.B. die Geometrie es nicht mit der Existenz der Dinge, sondern nur ihrer Bestimmung a priori in einer möglichen Anschauung zu tun hat, dennoch eben so gut, wie durch Kausalbegriffe, von einer Bestimmung A zu einer ganz verschiedenen B, als dennoch mit jener notwendig verknüpft, übergeht). Aber endlich muß jene wegen ihrer apodiktischen Gewißheit so hochgepriesene Wissenschaft doch dem Empirismus in Grundsätzen, aus demselben Grunde, warum Hume, an der Stelle der objektiven Notwendigkeit in dem Begriffe der Ursache, die Gewohnheit setzte, auch unterliegen, und sich, unangesehen alles ihres Stolzes, gefallen lassen, ihre kühne, a priori Beistimmung gebietende Ansprüche herabzustimmen, und den Beifall für die Allgemeingültigkeit ihrer Sätze von der Gunst der Beobachter erwarten, die als Zeugen es doch nicht weigern würden zu gestehen, daß sie das, was der Geometer als Grundsätze vorträgt, jederzeit auch so wahrgenommen hätten,  folglich, ob es gleich eben nicht notwendig wäre, doch fernerhin, es so erwarten zu dürfen, erlauben würden. Auf diese Weise führt Humens Empirism in Grundsätzen auch unvermeidlich auf den Skeptizism, selbst in Ansehung der Mathematik, folglich in allem wissenschaftlichen theoretischen Gebrauche der Vernunft (denn dieser gehört entweder zur Philosophie, oder zur Mathematik). Ob der gemeine Vernunftgebrauch (bei einem so schrecklichen Umsturz, als man den Häuptern der Erkenntnis begegnen sieht) besser durchkommen, und nicht vielmehr, noch unwiederbringlicher, in eben diese Zerstörung alles Wissens werde verwickelt werden, mithin ein allgemeiner Skeptizism nicht aus denselben Grundsätzen folgen müsse (der freilich aber nur die Gelehrten treffen würde), das will jeden seihst beurteilen lassen. 
	
	\subsection*{tg215.2.7} 
	\textbf{Source : }Kritik der praktischen Vernunft/Erster Teil. Elementarlehre der reinen praktischen Vernunft/Erstes Buch. Die Analytik der reinen praktischen Vernunft/Drittes Hauptstück. Von den Triebfedern der reinen praktischen Vernunft/Kritische Beleuchtung der Analytik der reinen praktischen Vernunft\\  
	
	\textbf{Paragraphe : }Aber diese Unterscheidung des Glückseligkeitsprinzips von dem der Sittlichkeit ist darum nicht so fort Entgegensetzung beider, und die reine praktische Vernunft will nicht, man solle die Ansprüche auf Glückseligkeit aufgeben, sondern nur, so bald von Pflicht die Rede ist, darauf gar nicht Rücksicht nehmen. Es kann sogar in gewissem Betracht Pflicht sein, für seine Glückseligkeit zu sorgen; teils weil sie (wozu Geschicklichkeit, Gesundheit, Reichtum gehört) Mittel zu Erfüllung seiner Pflicht enthält, teils weil der Mangel derselben (z.B. Armut) Versuchungen enthält, seine Pflicht zu übertreten. Nur, seine Glückseligkeit zu befördern,  kann unmittelbar niemals Pflicht, noch weniger ein Prinzip aller Pflicht sein. Da nun alle Bestimmungsgründe des Willens, außer dem einigen reinen praktischen Vernunftgesetze (dem moralischen), insgesamt empirisch sind, als solche also zum Glückseligkeitsprinzip gehören, so müssen sie insgesamt vom obersten sittlichen Grundsatze abgesondert, und ihm nie als Bedingung einverleibt werden, weil dieses eben so sehr allen sittlichen Wert, als empirische Beimischung zu geometrischen Grundsätzen alle mathematische Evidenz, das Vortrefflichste, was (nach Platos Urteile) die \match{Mathematik} an sich hat, und das selbst allem Nutzen derselben vorgeht, aufheben würde. 
	
	\subsection*{tg229.2.4} 
	\textbf{Source : }Kritik der praktischen Vernunft/Beschluß\\  
	
	\textbf{Paragraphe : }Diesen Weg nun in Behandlung der moralischen Anlagen unserer Natur gleichfalls einzuschlagen, kann uns jenes Beispiel anrätig sein, und Hoffnung zu ähnlichem guten Erfolg geben. Wir haben doch die Beispiele der moralisch-urteilenden Vernunft bei Hand. Diese nun in ihre Elementarbegriffe zu zergliedern, in Ermangelung der \match{Mathematik} aber ein der Chemie ähnliches Verfahren, der Scheidung des Empirischen vom Rationalen, das sich in ihnen vorfinden  möchte, in wiederholten Versuchen am gemeinen Menschenverstande vorzunehmen, kann uns beides rein, und, was jedes für sich allein leisten könne, mit Gewißheit kennbar machen, und so, teils der Verirrung einer noch rohen ungeübten Beurteilung, teils (welches weit nötiger ist) den Genieschwüngen vorbeugen, durch welche, wie es von Adepten des Steins der Weisen zu geschehen pflegt, ohne alle methodische Nachforschung und Kenntnis der Natur, geträumte Schätze versprochen und wahre verschleudert werden. Mit einem Worte: Wissenschaft (kritisch gesucht und methodisch eingeleitet) ist die enge Pforte, die zur Weisheitslehre führt, wenn unter dieser nicht bloß verstanden wird, was man tun, sondern was Lehrern zur Richtschnur dienen soll, um den Weg zur Weisheit, den jedermann gehen soll, gut und kenntlich zu bahnen, und andere vor Irrwegen zu sicheren; eine Wissenschaft, deren Aufbewahrerin jederzeit die Philosophie bleiben muß, an deren subtiler Untersuchung das Publikum keinen Anteil, wohl aber an den Lehren zu nehmen hat, die ihm, nach einer solchen Bearbeitung, allererst recht hell einleuchten können. 
	
	\subsection*{tg230.2.14} 
	\textbf{Source : }Kritik der praktischen Vernunft/Fußnoten\\  
	
	\textbf{Paragraphe : }
	
	7 Sätze, welche in der \match{Mathematik} oder Naturlehre praktisch genannt werden, sollten eigentlich technisch heißen. Denn um die Willensbestimmung ist es diesen Lehren gar nicht zu tun; sie zeigen nur das Mannigfaltige der möglichen Handlung an, welches eine gewisse Wirkung hervorzubringen hinreichend ist, und sind also eben so theoretisch, als alle Sätze, welche die Verknüpfung der Ursache mit einer Wirkung aussagen. Wem nun die letztere beliebt, der muß sich auch gefallen lassen, die erstere zu sein. 
	
	\subsection*{tg230.2.30} 
	\textbf{Source : }Kritik der praktischen Vernunft/Fußnoten\\  
	
	\textbf{Paragraphe : }
	
	15
	Gelehrsamkeit ist eigentlich nur der Inbegriff historischer Wissenschaften. Folglich kann nur der Lehrer der geoffenbarten Theologie ein Gottesgelehrter heißen. Wollte man aber auch den, der im Besitze von Vernunftwissenschaften (\match{Mathematik} und Philosophie) ist, einen Gelehrten nennen, obgleich dieses schon der Wortbedeutung (als die jederzeit nur dasjenige, was man durchaus gelehret werden muß, und was man also nicht von selbst, durch Vernunft, erfinden kann, zur Gelehrsamkeit zählt) widerstreiten würde: so möchte wohl der Philosoph mit seiner Erkenntnis Gottes, als positiver Wissenschaft, eine zu schlechte Figur machen, um sich deshalb einen Gelehrten nennen zu lassen. 
	
	\unnumberedsection{Maß (1)} 
	\subsection*{tg227.2.2} 
	\textbf{Source : }Kritik der praktischen Vernunft/Erster Teil. Elementarlehre der reinen praktischen Vernunft/Zweites Buch. Dialektik der reinen praktischen Vernunft/Zweites Hauptstück. Von der Dialektik der reinen Vernunft in Bestimmung des Begriffs vom höchsten Gut/IX. Von der der praktischen Bestimmung des Menschen weislich angemessenen Proportion seiner Erkenntnisvermögen\\  
	
	\textbf{Paragraphe : }Wenn die menschliche Natur zum höchsten Gute zu streben bestimmt ist, so muß auch das \match{Maß} ihrer Erkenntnisvermögen, vornehmlich ihr Verhältnis unter einander, als zu diesem Zwecke schicklich, angenommen werden. Nun beweiset aber die Kritik der reinen spekulativen Vernunft die größte Unzulänglichkeit derselben, um die wichtigsten Aufgaben, die ihr vorgelegt werden, dem Zwecke angemessen aufzulösen, ob sie zwar die natürlichen und nicht zu übersehenden Winke eben derselben Vernunft, imgleichen die großen Schritte, die sie tun kann, nicht verkennt, um sich diesem großen Ziele, das ihr ausgesteckt ist, zu näheren, aber doch, ohne es jemals für sich selbst, sogar mit Beihülfe der größten Naturkenntnis, zu erreichen. Also scheint die Natur hier uns nur stiefmütterlich mit einem zu unserem Zwecke benötigten Vermögen versorgt zu haben. 
	
	\unnumberedsection{Maßstab (2)} 
	\subsection*{tg214.2.11} 
	\textbf{Source : }Kritik der praktischen Vernunft/Erster Teil. Elementarlehre der reinen praktischen Vernunft/Erstes Buch. Die Analytik der reinen praktischen Vernunft\\  
	
	\textbf{Paragraphe : }
	Warum das? Sein Beispiel hält mir ein Gesetz vor, das meinen Eigendünkel niederschlägt, wenn ich es mit meinem Verhalten vergleiche, und dessen Befolgung, mithin die Tunlichkeit desselben, ich durch die Tat bewiesen vor mir sehe. Nun mag ich mir sogar eines gleichen Grades der Rechtschaffenheit bewußt sein, und die Achtung bleibt doch. Denn, da beim Menschen immer alles Gute mangelhaft ist, so schlägt das Gesetz, durch ein Beispiel anschaulich gemacht, doch immer meinen Stolz nieder, wozu der Mann, den ich vor mir sehe, dessen Unlauterkeit, die ihm immer noch anhängen mag, mir nicht so, wie mir die meinige, bekannt ist, der mir also in reinerem Lichte erscheint, einen \match{Maßstab} abgibt. Achtung ist ein Tribut, den wir dem Verdienste nicht verweigern können, wir mögen wollen oder nicht; wir mögen allenfalls äußerlich damit zurückhalten, so können wir doch nicht verhüten, sie innerlich zu empfinden. 
	
	\subsection*{tg223.2.10} 
	\textbf{Source : }Kritik der praktischen Vernunft/Erster Teil. Elementarlehre der reinen praktischen Vernunft/Zweites Buch. Dialektik der reinen praktischen Vernunft/Zweites Hauptstück. Von der Dialektik der reinen Vernunft in Bestimmung des Begriffs vom höchsten Gut/V. Das Dasein Gottes, als ein Postulat der reinen praktischen Vernunft\\  
	
	\textbf{Paragraphe : }Auch kann man hieraus ersehen: daß, wenn man nach dem letzten Zwecke Gottes in Schöpfung der Welt frägt, man nicht die Glückseligkeit der vernünftigen Wesen in ihr, sondern das höchste Gut nennen müsse, welches jenem Wunsche dieser Wesen noch eine Bedingung, nämlich die, der Glückseligkeit würdig zu sein, d.i. die Sittlichkeit eben derselben vernünftigen Wesen, hinzufügt, die allein den \match{Maßstab} enthält, nach welchem sie allein der ersteren, durch die Hand eines weisen Urhebers, teilhaftig zu werden hoffen können. Denn, da Weisheit, theoretisch betrachtet, die Erkenntnis des höchsten Guts, und, praktisch, die Angemessenheit des Willens zum höchsten Gute bedeutet, so kann man einer höchsten selbständigen Weisheit nicht einen Zweck beilegen, der bloß auf Gütigkeit gegründet wäre. Denn dieser ihre Wirkung (in Ansehung der Glückseligkeit der vernünftigen Wesen)  kann man nur unter den einschränkenden Bedingungen der Übereinstimmung mit der Heiligkeit
	
	
	
	14
	seines Willens, als dem höchsten ursprünglichen Gute angemessen, denken. Daher diejenige, welche den Zweck der Schöpfung in die Ehre Gottes (vorausgesetzt, daß man diese nicht anthropomorphistisch, als Neigung, gepriesen zu werden, denkt) setzten, wohl den besten Ausdruck getroffen haben. Denn nichts ehrt Gott mehr, als das, was das Schätzbarste in der Welt ist, die Achtung für sein Gebot, die Beobachtung der heiligen Pflicht, die uns sein Gesetz auferlegt, wenn seine herrliche Anstalt dazu kommt, eine solche schöne Ordnung mit angemessener Glückseligkeit zu krönen. Wenn ihn das letztere (auf menschliche Art zu reden) liebenswürdig macht, so ist er durch das erstere ein Gegenstand der Anbetung (Adoration). Selbst Menschen können sich durch Wohltun zwar Liebe, aber dadurch allein niemals Achtung erwerben, so daß die größte Wohltätigkeit ihnen nur dadurch Ehre macht, daß sie nach Würdigkeit ausgeübt wird. 
	
	\unnumberedsection{Mikroskop (1)} 
	\subsection*{tg228.2.22} 
	\textbf{Source : }Kritik der praktischen Vernunft/Zweiter Teil. Methodenlehre der reinen praktischen Vernunft\\  
	
	\textbf{Paragraphe : }Die Methode nimmt also folgenden Gang. Zuerst ist es nur darum zu tun, die Beurteilung nach moralischen Gesetzen zu einer natürlichen, alle unsere eigene sowohl als die Beobachtung fremder freier Handlungen begleitenden Beschäftigung und gleichsam zur Gewohnheit zu machen, und sie zu schärfen, indem man vorerst frägt, ob die Handlung objektiv dem moralischen Gesetze, und welchem, gemäß sei; wobei man denn die Aufmerksamkeit auf dasjenige Gesetz, welches bloß einen Grund zur Verbindlichkeit an die Hand gibt, von dem unterscheidet, welches in der Tat verbindend ist (leges obligandi a legibus obligantibus), (wie z.B. das Gesetz desjenigen, was das Bedürfnis der Menschen im Gegensatze dessen, was das Recht derselben von mir fordert, wovon das letztere wesentliche, das erstere aber nur außerwesentliche Pflichten vorschreibt) und so verschiedene Pflichten, die in einer Handlung zusammenkommen, unterscheiden lehrt. Der andere Punkt, worauf die Aufmerksamkeit gerichtet werden muß, ist die Frage: ob die Handlung auch (subjektiv) um des moralischen Gesetzes willen geschehen, und also sie nicht allein sittliche Richtigkeit, als Tat, sondern auch sittlichen Wert, als Gesinnung, ihrer Maxime nach habe. Nun ist kein Zweifel, daß diese Übung, und das Bewußtsein einer daraus entspringenden Kultur unserer bloß über das Praktische urteilenden Vernunft, ein gewisses Interesse, selbst am Gesetze derselben, mithin an sittlich guten Handlungen nach und nach hervorbringen müsse. Denn wir gewinnen endlich das lieb, dessen Betrachtung uns den erweiterten Gebrauch unserer Erkenntniskräfte empfinden läßt, welchen vornehmlich dasjenige befördert, worin wir moralische Richtigkeit antreffen; weil sich die Vernunft in einer solchen Ordnung der Dinge mit ihrem Vermögen, a priori nach Prinzipien zu bestimmen was geschehen soll, allein gut finden kann. Gewinnt doch ein Naturbeobachter Gegenstände, die seinen Sinnen anfangs anstößig sind, endlich lieb, wenn er die große Zweckmäßigkeit ihrer Organisation daran entdeckt,  und so seine Vernunft an ihrer Betrachtung weidet, und Leibniz brachte ein Insekt, welches er durchs \match{Mikroskop} sorgfältig betrachtet hatte, schonend wiederum auf sein Blatt zurück, weil er sich durch seinen Anblick belehrt gefunden, und von ihm gleichsam eine Wohltat genossen hatte. 
	
	\unnumberedsection{Moment (1)} 
	\subsection*{tg210.2.11} 
	\textbf{Source : }Kritik der praktischen Vernunft/Erster Teil. Elementarlehre der reinen praktischen Vernunft/Erstes Buch. Die Analytik der reinen praktischen Vernunft/Erstes Hauptstück. Von den Grundsätzen der reinen praktischen Vernunft/I. Von der Deduktion der Grundsätze der reinen praktischen Vernunft\\  
	
	\textbf{Paragraphe : }Die zweite, als zur Kritik der praktischen Vernunft gehörig, fodert keine Erklärung, wie die Objekte des Begehrungsvermögens möglich sind, denn das bleibt, als Aufgabe der theoretischen Naturkenntnis, der Kritik der spekulativen Vernunft überlassen, sondern nur, wie Vernunft die Maxime des Willens bestimmen könne, ob es nur vermittelst empirischer Vorstellung, als Bestimmungsgründe, geschehe, oder ob auch reine Vernunft praktisch und ein Gesetz einer möglichen, gar nicht empirisch erkennbaren, Naturordnung sein würde. Die Möglichkeit einer solchen übersinnlichen Natur, deren Begriff zugleich der Grund der Wirklichkeit derselben durch unseren freien Willen sein könne, bedarf keiner Anschauung a priori (einer intelligibelen Welt), die in diesem Falle, als übersinnlich, für uns auch unmöglich sein müßte. Denn es kommt nur auf den Bestimmungsgrund des Wollens in den Maximen desselben an, ob jener empirisch, oder ein Begriff der reinen Vernunft (von der Gesetzmäßigkeit derselben überhaupt) sei, und wie er letzteres sein könne. Ob die Kausalität des Willens zur Wirklichkeit der Objekte zulange, oder nicht, bleibt den theoretischen Prinzipien der Vernunft zu beurteilen überlassen, als Untersuchung der Möglichkeit der Objekte des Wollens, deren Anschauung also in der praktischen Aufgabe gar kein \match{Moment} derselben ausmacht. Nur auf die Willensbestimmung und den Bestimmungsgrund der Maxime desselben, als eines freien Willens, kommt es hier an, nicht auf den Erfolg. Denn,  wenn der Wille nur für die reine Vernunft gesetzmäßig ist, so mag es mit dem Vermögen desselben in der Ausführung stehen, wie es wolle, es mag nach diesen Maximen der Gesetzgebung einer möglichen Natur eine solche wirklich daraus entspringen, oder nicht, darum bekümmert sich die Kritik, die da untersucht, ob und wie reine Vernunft praktisch, d.i. unmittelbar willenbestimmend, sein könne, gar nicht. 
	
	\unnumberedsection{Nummer (1)} 
	\subsection*{tg212.2.51} 
	\textbf{Source : }Kritik der praktischen Vernunft/Erster Teil. Elementarlehre der reinen praktischen Vernunft/Erstes Buch. Die Analytik der reinen praktischen Vernunft\\  
	
	\textbf{Paragraphe : }
	Ich füge hier nichts weiter zur Erläuterung gegenwärtiger Tafel bei, weil sie für sich verständlich genug ist. Dergleichen nach Prinzipien abgefaßte Einteilung ist aller Wissenschaft, ihrer Gründlichkeit sowohl als Verständlichkeit halber, sehr zuträglich. So weiß man, z.B., aus obiger Tafel und der ersten \match{Nummer} derselben sogleich, wovon man in praktischen Erwägungen anfangen müsse: von den Maximen, die jeder auf seine Neigung gründet, den Vorschriften, die für eine Gattung vernünftiger Wesen, so fern sie in gewissen Neigungen übereinkommen, gelten, und endlich dem Gesetze, welches für alle, unangesehen ihrer Neigungen, gilt, u.s.w. Auf diese Weise übersieht man den ganzen Plan, von dem, was man zu leisten hat, so gar jede Frage der praktischen Philosophie, die zu beantworten, und zugleich die Ordnung, die zu befolgen ist. 
	
	\unnumberedsection{Physik (1)} 
	\subsection*{tg225.2.9} 
	\textbf{Source : }Kritik der praktischen Vernunft/Erster Teil. Elementarlehre der reinen praktischen Vernunft/Zweites Buch. Dialektik der reinen praktischen Vernunft/Zweites Hauptstück. Von der Dialektik der reinen Vernunft in Bestimmung des Begriffs vom höchsten Gut/VII. Wie eine Erweiterung der reinen Vernunft, in praktischer Absicht, ohne damit ihr Erkenntnis, als spekulativ, zugleich zu erweitern, zu denken möglich sei\\  
	
	\textbf{Paragraphe : }Nach diesen Erinnerungen ist nun auch die Beantwortung der wichtigen Frage leicht zu finden: Ob der Begriff von Gott ein zur \match{Physik} (mithin auch zur Metaphysik, als die nur die reinen Prinzipien a priori der ersteren in allgemeiner  Bedeutung enthält) oder ein zur Moral gehöriger Begriff sei. Natureinrichtungen, oder deren Veränderung zu erklären, wenn man da zu Gott, als dem Urheber aller Dinge, seine Zuflucht nimmt, ist wenigstens keine physische Erklärung, und überall ein Geständnis, man sei mit seiner Philosophie zu Ende; weil man genötigt ist, etwas, wovon man sonst für sich keinen Begriff hat, anzunehmen, um sich von der Möglichkeit dessen, was man vor Augen sieht, einen Begriff machen zu können. Durch Metaphysik aber von der Kenntnis dieser Welt zum Begriffe von Gott und dem Beweise seiner Existenz durch sichere Schlüsse zu gelangen, ist darum unmöglich, weil wir diese Welt als das vollkommenste mögliche Ganze, mithin, zu diesem Behuf, alle mögliche Welten (um sie mit dieser vergleichen zu können) erkennen, mithin allwissend sein müßten, um zu sagen, daß sie nur durch einen Gott (wie wir uns diesen Begriff denken müssen) möglich war. Vollends aber die Existenz dieses Wesens aus bloßen Begriffen zu erkennen, ist schlechterdings unmöglich, weil ein jeder Existentialsatz, d.i. der, so von einem Wesen, von dem ich mir einen Begriff mache, sagt, daß es existiere, ein synthetischer Satz ist, d.i. ein solcher, dadurch ich über jenen Begriff hinausgehe und mehr von ihm sage, als im Begriffe gedacht war: nämlich daß diesem Begriffe im Verstande noch ein Gegenstand außer dem Verstande korrespondierend gesetzt sei, welches offenbar unmöglich ist durch irgend einen Schluß herauszubringen. Also bleibt nur ein einziges Verfahren für die Vernunft übrig, zu diesem Erkenntnisse zu gelangen, da sie nämlich, als reine Vernunft, von dem obersten Prinzip ihres reinen praktischen Gebrauchs ausgehend (indem dieser ohnedem bloß auf die Existenz von etwas, als Folge der Vernunft, gerichtet ist), ihr Objekt bestimmt. Und da zeigt sich, nicht allein in ihrer unvermeidlichen Aufgabe, nämlich der notwendigen Richtung des Willens auf das höchste Gut, die Notwendigkeit, ein solches Urwesen, in Beziehung auf die Möglichkeit dieses Guten in der Welt, anzunehmen, sondern, was das Merkwürdigste ist, etwas, was dem Fortgange der Vernunft auf dem Naturwege ganz  mangelte, nämlich ein genau bestimmter Begriff dieses Urwesens. Da wir diese Welt nur zu einem kleinen Teile kennen, noch weniger sie mit allen möglichen Welten vergleichen können, so können wir von ihrer Ordnung, Zweckmäßigkeit und Größe wohl auf einen weisen, gütigen, mächtigen etc. Urheber derselben schließen, aber nicht auf seine Allwissenheit, Allgütigkeit, Allmacht, u.s.w. Man kann auch gar wohl einräumen: daß man diesen unvermeidlichen Mangel durch eine erlaubte ganz vernünftige Hypothese zu ergänzen wohl befugt sei; daß nämlich, wenn in so viel Stücken, als sich unserer näheren Kenntnis darbieten, Weisheit, Gütigkeit etc. hervorleuchtet, in allen übrigen es eben so sein werde, und es also vernünftig sei, dem Welturheber alle mögliche Vollkommenheit beizulegen; aber das sind keine Schlüsse, wodurch wir uns auf unsere Einsicht etwas dünken, sondern nur Befugnisse, die man uns nachsehen kann, und doch noch einer anderweitigen Empfehlung bedürfen, um davon Gebrauch zu machen. Der Begriff von Gott bleibt also auf dem empirischen Wege (der Physik) immer ein nicht genau bestimmter Begriff von der Vollkommenheit des ersten Wesens, um ihn dem Begriffe einer Gottheit für angemessen zu halten (mit der Metaphysik aber in ihrem transzendentalen Teile ist gar nichts auszurichten). 
	
	\unnumberedsection{Postulat (2)} 
	\subsection*{tg222.2.3} 
	\textbf{Source : }Kritik der praktischen Vernunft/Erster Teil. Elementarlehre der reinen praktischen Vernunft/Zweites Buch. Dialektik der reinen praktischen Vernunft/Zweites Hauptstück. Von der Dialektik der reinen Vernunft in Bestimmung des Begriffs vom höchsten Gut/IV. Die Unsterblichkeit der Seele, als ein Postulat der reinen praktischen Vernunft\\  
	
	\textbf{Paragraphe : }Dieser unendliche Progressus ist aber nur unter Voraussetzung einer ins Unendliche fortdaurenden Existenz und Persönlichkeit desselben vernünftigen Wesens (welche man die Unsterblichkeit der Seele nennt) möglich. Also ist das höchste Gut, praktisch, nur unter der Voraussetzung der Unsterblichkeit der Seele möglich; mithin diese, als unzertrennlich mit dem moralischen Gesetz verbunden, ein \match{Postulat} der reinen praktischen Vernunft (worunter ich  einen theoretischen, als solchen aber nicht erweislichen Satz verstehe, so fern er einem a priori unbedingt geltenden praktischen Gesetze unzertrennlich anhängt). 
	
	\subsection*{tg223.2.3} 
	\textbf{Source : }Kritik der praktischen Vernunft/Erster Teil. Elementarlehre der reinen praktischen Vernunft/Zweites Buch. Dialektik der reinen praktischen Vernunft/Zweites Hauptstück. Von der Dialektik der reinen Vernunft in Bestimmung des Begriffs vom höchsten Gut/V. Das Dasein Gottes, als ein Postulat der reinen praktischen Vernunft\\  
	
	\textbf{Paragraphe : }
	Glückseligkeit ist der Zustand eines vernünftigen Wesens in der Welt, dem es, im Ganzen seiner Existenz, alles nach Wunsch und Willen geht, und beruhet also auf der Übereinstimmung der Natur zu sei nem ganzen Zwecke, imgleichen zum wesentlichen Bestimmungsgrunde seines Willens. Nun gebietet das moralische Gesetz, als ein Gesetz der Freiheit, durch Bestimmungsgründe, die von der Natur und der Übereinstimmung derselben zu unserem Begehrungsvermögen (als Triebfedern) ganz unabhängig sein sollen; das handelnde vernünftige Wesen in der Welt aber ist doch nicht zugleich Ursache der Welt und der Natur selbst. Also ist in dem moralischen Gesetze nicht der mindeste Grund zu einem notwendigen Zusammenhang zwischen Sittlichkeit und der ihr proportionierten Glückseligkeit eines zur Welt als Teil gehörigen, und daher von ihr abhängigen, Wesens, welches eben darum durch seinen Willen nicht Ursache dieser Natur sein, und sie, was seine Glückseligkeit betrifft, mit seinen praktischen Grundsätzen aus eigenen Kräften nicht durchgängig einstimmig machen kann. Gleichwohl wird in der praktischen Aufgabe der reinen Vernunft, d.i. der notwendigen Bearbeitung zum höchsten Gute, ein solcher Zusammenhang als notwendig postuliert: wir sollen das höchste Gut (welches also doch möglich sein muß) zu befördern suchen. Also wird auch das Dasein einer von der Natur unterschiedenen Ursache der gesamten Natur, welche den Grund dieses Zusammenhanges, nämlich der genauen Übereinstimmung der Glückseligkeit mit der Sittlichkeit, enthalte, postuliert. Diese oberste Ursache aber soll den Grund der Übereinstimmung der Natur nicht bloß mit einem Gesetze des Willens der vernünftigen Wesen, sondern  mit der Vorstellung dieses Gesetzes, so fern diese es sich zum obersten Bestimmungsgrunde des Willens setzen, also nicht bloß mit den Sitten der Form nach, sondern auch ihrer Sittlichkeit, als dem Bewegungsgrunde derselben, d.i. mit ihrer moralischen Gesinnung enthalten. Also ist das höchste Gut in der Welt nur möglich, so fern eine oberste der Natur angenommen wird, die eine der moralischen Gesinnung gemäße Kausalität hat. Nun ist ein Wesen, das der Handlungen nach der Vorstellung von Gesetzen fähig ist, eine Intelligenz (vernünftig Wesen) und die Kausalität eines solchen Wesens nach dieser Vorstellung der Gesetze ein Wille desselben. Also ist die oberste Ursache der Natur, so fern sie zum höchsten Gute vorausgesetzt werden muß, ein Wesen, das durch Verstand und Willen die Ursache (folglich der Urheber) der Natur ist, d.i. Gott. Folglich ist das \match{Postulat} der Möglichkeit des höchsten abgeleiteten Guts (der besten Welt) zugleich das Postulat der Wirklichkeit eines höchsten ursprünglichen Guts, nämlich der Existenz Gottes. Nun war es Pflicht für uns, das höchste Gut zu befördern, mithin nicht allein Befugnis, sondern auch mit der Pflicht als Bedürfnis verbundene Notwendigkeit, die Möglichkeit dieses höchsten Guts vorauszusetzen, welches, da es nur unter der Bedingung des Daseins Gottes stattfindet, die Voraussetzung desselben mit der Pflicht unzertrennlich verbindet, d.i. es ist moralisch notwendig, das Dasein Gottes anzunehmen. 
	
	\unnumberedsection{Probe (2)} 
	\subsection*{tg213.2.5} 
	\textbf{Source : }Kritik der praktischen Vernunft/Erster Teil. Elementarlehre der reinen praktischen Vernunft/Erstes Buch. Die Analytik der reinen praktischen Vernunft/Zweites Hauptstück. Von dem Begriffe eines Gegenstandes der reinen praktischen Vernunft/Von der Typik der reinen praktischen Urteilskraft\\  
	
	\textbf{Paragraphe : }Die Regel der Urteilskraft unter Gesetzen der reinen praktischen Vernunft ist diese: Frage dich selbst, ob die Handlung, die du vorhast, wenn sie nach einem Gesetze der Natur, von der du selbst ein Teil wärest, geschehen sollte, sie du wohl, als durch deinen Willen möglich, ansehen könntest. Nach dieser Regel beurteilt in der Tat jedermann Handlungen, ob sie sittlich-gut oder böse sind. So sagt man: Wie, wenn ein jeder, wo er seinen Vorteil zu schaffen glaubt, sich erlaubte, zu betrügen, oder befugt hielte, sich das Leben abzukürzen, so bald ihn ein völliger Überdruß desselben befällt, oder anderer Not mit völliger Gleichgültigkeit ansähe,  und du gehörtest mit zu einer solchen Ordnung der Dinge, würdest du darin wohl mit Einstimmung deines Willens sein? Nun weiß ein jeder wohl: daß, wenn er sich in Geheim Betrug erlaubt, darum eben nicht jedermann es auch tue, oder wenn er unbemerkt lieblos ist, nicht sofort jedermann auch gegen ihn es sein würde; daher ist diese Vergleichung der Maxime seiner Handlungen mit einem allgemeinen Naturgesetze auch nicht der Bestimmungsgrund seines Willens. Aber das letztere ist doch ein Typus der Beurteilung der ersteren nach sittlichen Prinzipien. Wenn die Maxime der Handlung nicht so beschaffen ist, daß sie an der Form eines Naturgesetzes überhaupt die \match{Probe} hält, so ist sie sittlich-unmöglich. So urteilt selbst der gemeinste Verstand; denn das Naturgesetz liegt allen seinen gewöhnlichsten, selbst den Erfahrungsurteilen immer zum Grunde. Er hat es also jederzeit bei der Hand, nur daß er in Fällen, wo die Kausalität aus Freiheit beurteilt werden soll, jenes Naturgesetz bloß zum Typus eines Gesetzes der Freiheit macht, weil er, ohne etwas, was er zum Beispiele im Erfahrungsfalle machen könnte, bei Hand zu haben, dem Gesetze einer reinen praktischen Vernunft nicht den Gebrauch in der Anwendung verschaffen könnte. 
	
	\subsection*{tg215.2.5} 
	\textbf{Source : }Kritik der praktischen Vernunft/Erster Teil. Elementarlehre der reinen praktischen Vernunft/Erstes Buch. Die Analytik der reinen praktischen Vernunft/Drittes Hauptstück. Von den Triebfedern der reinen praktischen Vernunft/Kritische Beleuchtung der Analytik der reinen praktischen Vernunft\\  
	
	\textbf{Paragraphe : }Betrachten wir nun aller auch den Inhalt der Erkenntnis, die wir von einer reinen praktischen Vernunft, und durch dieselbe, haben können, so wie ihn die Analytik derselben darlegt, so finden sich, bei einer merkwürdigen Analogie zwischen ihr und der theoretischen, nicht weniger merkwürdige Unterschiede. In Ansehung der theoretischen könnte das Vermögen eines reinen Vernunfterkenntnisses a priori durch Beispiele aus Wissenschaften (bei denen man, da sie ihre Prinzipien auf so mancherlei Art durch methodischen Gebrauch auf die \match{Probe} stellen, nicht so leicht, wie im gemeinen Erkenntnisse, geheime Beimischung empirischer Erkenntnisgründe zu besorgen hat) ganz leicht und evident bewiesen werden. Aber daß reine Vernunft, ohne Beimischung irgend eines empirischen Bestimmungsgrundes, für sich allein auch praktisch sei, das mußte man aus dem gemeinsten praktischen Vernunftgebrauche dartun können, indem man den obersten praktischen Grundsatz, als einen solchen, den jede natürliche Menschenvernunft, als völlig a priori, von keinen sinnlichen Datis abhängend, für das oberste Gesetz seines Willens erkennt, beglaubigte. Man mußte ihn zuerst, der Reinigkeit seines Ursprungs nach, selbst im Urteile dieser gemeinen Vernunft bewähren und rechtfertigen, ehe ihn noch die Wissenschaft in die Hände nehmen konnte, um Gebrauch von ihm zu machen, gleichsam als ein Faktum, das vor allem Vernünfteln über seine Möglichkeit und allen Folgerungen, die daraus zu ziehen sein möchten, vorhergeht. Aber dieser Umstand läßt sich auch aus dem kurz vorher Angeführten gar wohl erklären; weil praktische reine Vernunft notwendig von Grundsätzen anfangen muß, die also aller Wissenschaft, als erste Data, zum Grunde gelegt werden müssen, und nicht allererst aus ihr entspringen können. Diese Rechtfertigung der moralischen Prinzipien, als Grundsätze einer  reinen Vernunft, konnte aber auch darum gar wohl, und mit gnugsamer Sicherheit, durch bloße Berufung auf das Urteil des gemeinen Menschenverstandes geführet werden, weil sich alles Empirische, was sich als Bestimmungsgrund des Willens in unsere Maximen einschleichen möchte, durch das Gefühl des Vergnügens oder Schmerzens, das ihm so fern, als es Begierde erregt, notwendig anhängt, sofort kenntlich macht, diesem aber jene reine praktische Vernunft geradezu widersteht, es in ihr Prinzip, als Bedingung, aufzunehmen. Die Ungleichartigkeit der Bestimmungsgründe (der empirischen und rationalen) wird durch diese Widerstrebung einer praktisch-gesetzgebenden Vernunft, wider alle sich einmengende Neigung, durch eine eigentümliche Art von Empfindung, welche aber nicht vor der Gesetzgebung der praktischen Vernunft vorhergeht, sondern vielmehr durch dieselbe allein und zwar als ein Zwang gewirkt wird, nämlich durch das Gefühl einer Achtung, dergleichen kein Mensch für Neigungen hat, sie mögen sein, welcher Art sie wollen, wohl aber fürs Gesetz, so kenntlich gemacht und so gehoben und hervorstechend, daß keiner, auch der gemeinste Menschenverstand, in einem vorgelegten Beispiele nicht den Augenblick inne werden sollte, daß durch empirische Gründe des Wollens ihm zwar ihren Anreizen zu folgen geraten, niemals aber einem anderen, als lediglich dem reinen praktischen Vernunftgesetze, zu gehorchen zugemutet werden könne. 
	
	\unnumberedsection{Proportion (4)} 
	\subsection*{tg212.2.10} 
	\textbf{Source : }Kritik der praktischen Vernunft/Erster Teil. Elementarlehre der reinen praktischen Vernunft/Erstes Buch. Die Analytik der reinen praktischen Vernunft\\  
	
	\textbf{Paragraphe : }Was wir gut nennen sollen, muß in jedes vernünftigen Menschen Urteil ein Gegenstand des Begehrungsvermögens sein, und das Böse in den Augen von jedermann ein Gegenstand des Abscheues; mithin bedarf es, außer dem Sinne, zu dieser Beurteilung noch Vernunft. So ist es mit der Wahrhaftigkeit im Gegensatz mit der Lüge, so mit der Gerechtigkeit im Gegensatz der Gewalttätigkeit etc. bewandt. Wir können aber etwas ein Übel nennen, welches doch jedermann zugleich für gut, bisweilen mittelbar, bisweilen gar für unmittelbar erklären muß. Der eine chirurgische Operation an sich verrichten läßt, fühlt sie ohne Zweifel als ein Übel; aber durch Vernunft erklärt er, und jedermann, sie für gut. Wenn aber jemand, der friedliebende Leute gerne neckt und beunruhigt, endlich einmal anläuft und mit einer tüchtigen Tracht Schläge abgefertigt wird: so ist dieses allerdings ein Übel, aber jedermann gibt dazu seinen Beifall und hält es an sich für gut, wenn auch nichts weiter daraus entspränge; ja selbst der, der sie empfängt, muß in seiner Vernunft erkennen, daß ihm Recht geschehe, weil er die \match{Proportion} zwischen dem Wohlbefinden und Wohlverhalten, welche die Vernunft ihm unvermeidlich vorhält, hier genau in Ausübung gebracht sieht. 
	
	\subsection*{tg218.2.2} 
	\textbf{Source : }Kritik der praktischen Vernunft/Erster Teil. Elementarlehre der reinen praktischen Vernunft/Zweites Buch. Dialektik der reinen praktischen Vernunft\\  
	
	\textbf{Paragraphe : }Der Begriff des Höchsten enthält schon eine Zweideutigkeit, die, wenn man darauf nicht Acht hat, unnötige Streitigkeiten veranlassen kann. Das Höchste kann das Oberste (supremum) oder auch das Vollendete (consummatum) bedeuten. Das erstere ist diejenige Bedingung, die selbst unbedingt, d.i. keiner andern untergeordnet ist (originarium); das zweite dasjenige Ganze, das kein Teil eines noch größeren Ganzen von derselben Art ist (perfectissimum). Daß Tugend (als die Würdigkeit glücklich zu sein) die oberste Bedingung alles dessen, was uns nur wünschenswert scheinen mag, mithin auch aller unserer Bewerbung um Glückseligkeit, mithin das oberste Gut sei, ist in der Analytik bewiesen worden. Darum ist sie aber noch nicht das ganze und vollendete Gut, als Gegenstand des Begehrungsvermögens vernünftiger endlicher Wesen; denn, um das zu sein, wird auch Glückseligkeit dazu erfodert, und zwar nicht bloß in den parteiischen Augen der Person, die sich selbst zum Zwecke macht, sondern selbst im Urteile einer unparteiischen Vernunft, die jene überhaupt in der Welt als Zweck an sich betrachtet. Denn der Glückseligkeit bedürftig, ihrer auch würdig, dennoch aber derselben nicht teilhaftig zu sein, kann mit dem vollkommenen Wollen eines vernünftigen Wesens, welches zugleich alle Gewalt hätte, wenn wir uns auch nur ein solches zum Versuche denken, gar nicht zusammen bestehen. So fern nun Tugend  und Glückseligkeit zusammen den Besitz des höchsten Guts in einer Person, hiebei aber auch Glückseligkeit, ganz genau in \match{Proportion} der Sittlichkeit (als Wert der Person und deren Würdigkeit glücklich zu sein) ausgeteilt, das höchste Gut einer möglichen Welt ausmachen: so bedeutet dieses das Ganze, das vollendete Gute, worin doch Tugend immer, als Bedingung, das oberste Gut ist, weil es weiter keine Bedingung über sich hat, Glückseligkeit immer etwas, was dem, der sie besitzt, zwar angenehm, aber nicht für sich allein schlechterdings und in aller Rücksicht gut ist, sondern jederzeit das moralische gesetzmäßige Verhalten als Bedingung voraussetzt. 
	
	\subsection*{tg220.2.5} 
	\textbf{Source : }Kritik der praktischen Vernunft/Erster Teil. Elementarlehre der reinen praktischen Vernunft/Zweites Buch. Dialektik der reinen praktischen Vernunft/Zweites Hauptstück. Von der Dialektik der reinen Vernunft in Bestimmung des Begriffs vom höchsten Gut/II. Kritische Aufhebung der Antinomie der praktischen Vernunft\\  
	
	\textbf{Paragraphe : }Wenn wir uns genötigt sehen, die Möglichkeit des höchsten Guts, dieses durch die Vernunft allen vernünftigen Wesen ausgesteckten Ziels aller ihrer moralischen Wünsche, in solcher Weite, nämlich in der Verknüpfung mit einer intelligibelen Welt, zu suchen, so muß es befremden, daß gleichwohl die Philosophen, alter so wohl, als neuer Zeiten, die Glückseligkeit mit der Tugend in ganz geziemender \match{Proportion} schon in diesem Leben (in der Sinnenwelt) haben finden, oder sich ihrer bewußt zu sein haben überreden können. Denn Epikur sowohl, als die Stoiker, erhoben die Glückseligkeit, die aus dem Bewußtsein der Tugend im Leben entspringe, über alles, und der erstere war in seinen praktischen Vorschriften nicht so niedrig gesinnt, als man aus den Prinzipien seiner Theorie, die er zum Erklären, nicht zum Handeln brauchte, schließen möchte, oder, wie sie viele, durch den Ausdruck Wollust, für Zufriedenheit, verleitet, ausdeuteten, sondern rechnete die uneigennützigste Ausübung des Guten mit zu den Genußarten der innigsten Freude, und die Gnügsamkeit und Bändigung der Neigungen, so wie sie immer der strengste Moralphilosoph fodern mag, gehörte mit zu seinem Plane eines Vergnügens (er verstand  darunter das stets fröhliche Herz); wobei er von den Stoikern vornehmlich nur darin abwich, daß er in diesem Vergnügen den Bewegungsgrund setzte, welches die letztern, und zwar mit Recht, verweigerten. Denn einesteils fiel der tugendhafte Epikur, so wie noch jetzt viele moralisch wohlgesinnte, obgleich über ihre Prinzipien nicht tief genug nachdenkende Männer, in den Fehler, die tugendhafte Gesinnung in denen Personen schon vorauszusetzen, für die er die Triebfeder zur Tugend zuerst angeben wollte (und in der Tat kann der Rechtschaffene sich nicht glücklich finden, wenn er sich nicht zuvor seiner Rechtschaffenheit bewußt ist; weil, bei jener Gesinnung, die Verweise, die er bei Übertretungen sich selbst zu machen durch seine eigene Denkungsart genötigt sein würde, und die moralische Selbstverdammung ihn alles Genusses der Annehmlichkeit, die sonst sein Zustand enthalten mag, berauben würden). Allein die Frage ist: wodurch wird eine solche Gesinnung und Denkungsart, den Wert seines Daseins zu schätzen, zuerst möglich; da vor derselben noch gar kein Gefühl für einen moralischen Wert überhaupt im Subjekte angetroffen werden würde. Der Mensch wird, wenn er tugendhaft ist, freilich, ohne sich in jeder Handlung seiner Rechtschaffenheit bewußt zu sein, des Lebens nicht froh werden, so günstig ihm auch das Glück im physischen Zustande desselben sein mag; aber um ihn allererst tugendhaft zu machen, mithin ehe er noch den moralischen Wert seiner Existenz so hoch anschlägt, kann man ihm da wohl die Seelenruhe anpreisen, die aus dem Bewußtsein einer Rechtschaffenheit entspringen werde, für die er doch keinen Sinn hat? 
	
	\subsection*{tg223.2.7} 
	\textbf{Source : }Kritik der praktischen Vernunft/Erster Teil. Elementarlehre der reinen praktischen Vernunft/Zweites Buch. Dialektik der reinen praktischen Vernunft/Zweites Hauptstück. Von der Dialektik der reinen Vernunft in Bestimmung des Begriffs vom höchsten Gut/V. Das Dasein Gottes, als ein Postulat der reinen praktischen Vernunft\\  
	
	\textbf{Paragraphe : }
	Auf solche Weise führt das moralische Gesetz durch den Begriff des höchsten Guts, als das Objekt und den Endzweck der reinen praktischen Vernunft, zur Religion, d.i. zur Erkenntnis aller Pflichten als göttlicher Gebote, nicht als Sanktionen, d.i. willkürliche für sich selbst zufällige Verordnungen, eines fremden Willens, sondern als wesentlicher Gesetze eines jeden freien Willens für sich selbst, die aber dennoch als Gebote des höchsten Wesens angesehen werden müssen, weil wir nur von einem moralisch-vollkommenen (heiligen und gütigen), zugleich auch allgewaltigen Willen das höchste Gut, welches zum Gegenstande unserer Bestrebung zu setzen uns das moralische Gesetz zur Pflicht macht, und also durch Übereinstimmung mit diesem Willen dazu zu gelangen hoffen können. Auch hier bleibt daher alles uneigennützig und bloß auf Pflicht gegründet; ohne daß Furcht oder Hoffnung als Triebfedern zum Grunde gelegt werden dürften, die, wenn sie zu Prinzipien werden, den ganzen moralischen Wert der Handlungen vernichten. Das moralische Gesetz gebietet, das höchste mögliche Gut in einer Welt mir zum letzten Gegenstande alles Verhaltens zu machen. Dieses aber kann ich nicht zu bewirken hoffen, als nur durch die Übereinstimmung meines Willens mit dem eines heiligen und gütigen Welturhebers, und, obgleich in dem Begriffe des höchsten Guts, als dem eines Ganzen, worin die größte Glückseligkeit mit dem größten Maße sittlicher (in Geschöpfen möglicher) Vollkommenheit, als in der genausten \match{Proportion} verbunden vorgestellt wird, meine eigene Glückseligkeit mit enthalten ist: so ist doch nicht sie, sondern das moralische Gesetz (welches vielmehr mein unbegrenztes Verlangen darnach auf Bedingungen strenge einschränkt) der Bestimmungsgrund des Willens, der zur Beförderung des höchsten Guts angewiesen wird. 
	
	\unnumberedsection{Schätzung (1)} 
	\subsection*{tg214.2.13} 
	\textbf{Source : }Kritik der praktischen Vernunft/Erster Teil. Elementarlehre der reinen praktischen Vernunft/Erstes Buch. Die Analytik der reinen praktischen Vernunft\\  
	
	\textbf{Paragraphe : }Achtung fürs moralische Gesetz ist also die einzige und zugleich unbezweifelte moralische Triebfeder, so wie dieses Gefühl auch auf kein Objekt anders, als lediglich aus diesem Grunde gerichtet ist. Zuerst bestimmt das moralische Gesetz objektiv und unmittelbar den Willen im Urteile der Vernunft; Freiheit, deren Kausalität bloß durchs Gesetz bestimmbar ist, besteht aber eben darin, daß sie alle Neigungen, mithin die \match{Schätzung} der Person selbst auf die Bedingung der Befolgung ihres reinen Gesetzes einschränkt. Diese Einschränkung tut nun eine Wirkung aufs Gefühl, und bringt Empfindung der Unlust hervor, die aus dem moralischen Gesetze a priori erkannt werden kann. Da sie aber  bloß so fern eine negative Wirkung ist, die, als aus dem Einflusse einer reinen praktischen Vernunft entsprungen, vornehmlich der Tätigkeit des Subjekts, so fern Neigungen die Bestimmungsgründe desselben sind, mithin der Meinung seines persönlichen Werts Abbruch tut (der ohne Einstimmung mit dem moralischen Gesetze auf nichts herabgesetzt wird), so ist die Wirkung dieses Gesetzes aufs Gefühl bloß Demütigung, welches wir also zwar a priori einsehen, aber an ihr nicht die Kraft des reinen praktischen Gesetzes als Triebfeder, sondern nur den Widerstand gegen Triebfedern der Sinnlichkeit erkennen können. Weil aber dasselbe Gesetz doch objektiv, d.i. in der Vorstellung der reinen Vernunft, ein unmittelbarer Bestimmungsgrund des Willens ist, folglich diese Demütigung nur relativ auf die Reinigkeit des Gesetzes stattfindet, so ist die Herabsetzung der Ansprüche der moralischen Selbstschätzung, d.i. die Demütigung auf der sinnlichen Seite, eine Erhebung der moralischen, d.i. der praktischen Schätzung des Gesetzes selbst, auf der intellektuellen, mit einem Worte Achtung fürs Gesetz, also auch ein, seiner intellektuellen Ursache nach, positives Gefühl, das a priori erkannt wird. Denn eine jede Verminderung der Hindernisse einer Tätigkeit ist Beförderung dieser Tätigkeit selbst. Die Anerkennung des moralischen Gesetzes aber ist das Bewußtsein einer Tätigkeit der praktischen Vernunft aus objektiven Gründen, die bloß darum nicht ihre Wirkung in Handlungen äußert, weil subjektive Ursachen (pathologische) sie hindern. Also muß die Achtung fürs moralische Gesetz auch als positive aber indirekte Wirkung desselben aufs Gefühl, so fern jenes den hindernden Einfluß der Neigungen durch Demütigung des Eigendünkels schwächt, mithin als subjektiver Grund der Tätigkeit, d.i. als Triebfeder zu Befolgung desselben, und als Grund zu Maximen eines ihm gemäßen Lebenswandels angesehen werden. Aus dem Begriffe einer Triebfeder entspringt der eines Interesse; welches niemals einem Wesen, als was Vernunft hat, beigelegt wird, und eine Triebfeder des Willens bedeutet, so fern sie durch Vernunft vorgestellt wird. Da das Gesetz selbst in einem moralisch-guten Willen die Triebfeder  sein muß, so ist das moralische Interesse ein reines sinnenfreies Interesse der bloßen praktischen Vernunft. Auf dem Begriffe eines Interesse gründet sich auch der einer Maxime. Diese ist also nur alsdenn moralisch echt, wenn sie auf dem bloßen Interesse, das man an der Befolgung des Gesetzes nimmt, beruht. Alle drei Begriffe aber, der einer Triebfeder, eines Interesse und einer Maxime, können nur auf endliche Wesen angewandt werden. Denn sie setzen insgesamt eine Eingeschränktheit der Natur eines Wesens voraus, da die subjektive Beschaffenheit seiner Willkür mit dem objektiven Gesetze einer praktischen Vernunft nicht von selbst übereinstimmt; ein Bedürfnis, irgend wodurch zur Tätigkeit angetrieben zu werden, weil ein inneres Hindernis derselben entgegensteht. Auf den göttlichen Willen können sie also nicht angewandt werden. 
	
	\unnumberedsection{Seiten (1)} 
	\subsection*{tg212.2.11} 
	\textbf{Source : }Kritik der praktischen Vernunft/Erster Teil. Elementarlehre der reinen praktischen Vernunft/Erstes Buch. Die Analytik der reinen praktischen Vernunft\\  
	
	\textbf{Paragraphe : }Es kommt allerdings auf unser Wohl und Weh in der Beurteilung unserer praktischen Vernunft gar sehr viel, und, was unsere Natur als sinnlicher Wesen betrifft, alles auf unsere Glückseligkeit an, wenn diese, wie Vernunft es  vorzüglich fodert, nicht nach der vorübergehenden Empfindung, sondern nach dem Einflusse, den diese Zufälligkeit auf unsere ganze Existenz und die Zufriedenheit mit derselben hat, beurteilt wird; aber alles überhaupt kommt darauf doch nicht an. Der Mensch ist ein bedürftiges Wesen, so fern er zur Sinnenwelt gehört und so fern hat seine Vernunft allerdings einen nicht abzulehnenden Auftrag, von \match{Seiten} der Sinnlichkeit, sich um das Interesse derselben zu bekümmern und sich praktische Maximen, auch in Absicht auf die Glückseligkeit dieses, und, wo möglich, auch eines zukünftigen Lebens, zu machen. Aber er ist doch nicht so ganz Tier, um gegen alles, was Vernunft für sich selbst sagt, gleichgültig zu sein, und diese bloß zum Werkzeuge der Befriedigung seines Bedürfnisses, als Sinnenwesens, zu gebrauchen. Denn im Werte über die bloße Tierheit erhebt ihn das gar nicht, daß er Vernunft hat, wenn sie ihm nur zum Behuf desjenigen dienen soll, was bei Tieren der Instinkt verrichtet; sie wäre als denn nur eine besondere Manier, deren sich die Natur bedient hätte, um den Menschen zu demselben Zwecke, dazu sie Tiere bestimmt hat, auszurüsten, ohne ihn zu einem höheren Zwecke zu bestimmen. Er bedarf also freilich, nach dieser einmal mit ihm getroffenen Naturanstalt, Vernunft, um sein Wohl und Weh jederzeit in Betrachtung zu ziehen, aber er hat sie überdem noch zu einem höheren Behuf, nämlich auch das, was an sich gut oder böse ist, und worüber reine, sinnlich gar nicht interessierte Vernunft nur allein urteilen kann, nicht allein mit in Überlegung zu nehmen, sondern diese Beurteilung von jener gänzlich zu unterscheiden, und sie zur obersten Bedingung des letzteren zu machen. 
	
	\unnumberedsection{Strom (1)} 
	\subsection*{tg215.2.13} 
	\textbf{Source : }Kritik der praktischen Vernunft/Erster Teil. Elementarlehre der reinen praktischen Vernunft/Erstes Buch. Die Analytik der reinen praktischen Vernunft/Drittes Hauptstück. Von den Triebfedern der reinen praktischen Vernunft/Kritische Beleuchtung der Analytik der reinen praktischen Vernunft\\  
	
	\textbf{Paragraphe : }Hiemit stimmen auch die Richteraussprüche desjenigen wundersamen Vermögens in uns, welches wir Gewissen nennen, vollkommen überein. Ein Mensch mag künsteln, soviel als er will, um ein gesetzwidriges Betragen, dessen er sich  erinnert, sich als unvorsätzliches Versehen, als bloße Unbehutsamkeit, die man niemals gänzlich vermeiden kann, folglich als etwas, worin er vom \match{Strom} der Naturnotwendigkeit fortgerissen wäre, vorzumalen und sich darüber für schuldfrei zu erklären, so findet er doch, daß der Advokat, der zu seinem Vorteil spricht, den Ankläger in ihm keinesweges zum Verstummen bringen könne, wenn er sich bewußt ist, daß er zu der Zeit, als er das Unrecht verübte, nur bei Sinnen, d.i. im Gebrauche seiner Freiheit war, und gleichwohl erklärt er sich sein Vergehen, aus gewisser übeln, durch allmähliche Vernachlässigung der Achtsamkeit auf sich selbst zugezogener Gewohnheit, bis auf den Grad, daß er es als eine natürliche Folge derselben ansehen kann, ohne daß dieses ihn gleichwohl wider den Selbsttadel und den Verweis sichern kann, den er sich selbst macht. Darauf gründet sich denn auch die Reue über eine längst begangene Tat bei jeder Erinnerung derselben; eine schmerzhafte, durch moralische Gesinnung gewirkte Empfindung, die so fern praktisch leer ist, als sie nicht dazu dienen kann, das Geschehene ungeschehen zu machen, und sogar ungereimt sein würde (wie Priestley, als ein echter, konsequent verfahrender Fatalist, sie auch dafür erklärt, und in Ansehung welcher Offenherzigkeit er mehr Beifall verdient, als diejenige, welche, indem sie den Mechanism des Willens in der Tat, die Freiheit desselben aber mit Worten behaupten, noch immer dafür gehalten sein wollen, daß sie jene, ohne doch die Möglichkeit einer solchen Zurechnung begreiflich zu machen, in ihrem synkretistischen System mit einschließen), aber, als Schmerz, doch ganz rechtmäßig ist, weil die Vernunft, wenn es auf das Gesetz unserer intelligibelen Existenz (das moralische) ankommt, keinen Zeitunterschied anerkennt, und nur frägt, ob die Begebenheit mir als Tat angehöre, alsdenn aber immer dieselbe Empfindung damit moralisch verknüpft, sie mag jetzt geschehen, oder vorlängst geschehen sein. Denn das Sinnenleben hat in Ansehung des intelligibelen Bewußtseins seines Daseins (der Freiheit) absolute Einheit eines Phänomens, welches, so fern es bloß Erscheinungen von der Gesinnung, die das moralische Gesetz angeht  (von dem Charakter), enthält, nicht nach der Naturnotwendigkeit, die ihm als Erscheinung zukommt, sondern nach der absoluten Spontaneität der Freiheit beurteilt werden muß. Man kann also einräumen, daß, wenn es für uns möglich wäre, in eines Menschen Denkungsart, so wie sie sich durch innere sowohl als äußere Handlungen zeigt, so tiefe Einsicht zu haben, daß jede, auch die mindeste Triebfeder dazu uns bekannt würde, imgleichen alle auf diese wirkende äußere Veranlassungen, man eines Menschen Verhalten auf die Zukunft mit Gewißheit, so wie eine Mond- oder Sonnenfinsternis, ausrechnen könnte, und dennoch dabei behaupten, daß der Mensch frei sei. Wenn wir nämlich noch eines andern Blicks (der uns aber freilich gar nicht verliehen ist, sondern an dessen Statt wir nur den Vernunftbegriff haben), nämlich einer intellektuellen Anschauung desselben Subjekts fähig wären, so würden wir doch inne werden, daß diese ganze Kette von Erscheinungen in Ansehung dessen, was nur immer das moralische Gesetz angehen kann, von der Spontaneität des Subjekts, als Dinges an sich selbst, abhängt, von deren Bestimmung sich gar keine physische Erklärung geben läßt. In Ermangelung dieser Anschauung versichert uns das moralische Gesetz diesen Unterschied der Beziehung unserer Handlungen, als Erscheinungen, auf das Sinnenwesen unseres Subjekts, von derjenigen, dadurch dieses Sinnenwesen selbst auf das intelligibele Substrat in uns bezogen wird. – In dieser Rücksicht, die unserer Vernunft natürlich, obgleich unerklärlich ist, lassen sich auch Beurteilungen rechtfertigen, die, mit aller Gewissenhaftigkeit gefället, dennoch dem ersten Anscheine nach aller Billigkeit ganz zu widerstreiten scheinen. Es gibt Fälle, wo Menschen von Kindheit auf, selbst unter einer Erziehung, die, mit der ihrigen zugleich, andern ersprießlich war, dennoch so frühe Bosheit zeigen, und so bis in ihre Mannesjahre zu steigen fortfahren, daß man sie für geborne Bösewichter, und gänzlich, was die Denkungsart betrifft, für unbesserlich hält, gleichwohl aber sie wegen ihres Tuns und Lassens eben so richtet, ihnen ihre Verbrechen eben so als Schuld verweiset, ja sie (die Kinder) selbst diese Verweise so ganz gegründet  finden, als ob sie, ungeachtet der ihnen beigemessenen hoffnungslosen Naturbeschaffenheit ihres Gemüts, eben so verantwortlich blieben, als jeder andere Mensch. Dieses würde nicht geschehen können, wenn wir nicht voraussetzten, daß alles, was aus seiner Willkür entspringt (wie ohne Zweifel jede vorsätzlich verübte Handlung), eine freie Kausalität zum Grunde habe, welche von der frühen Jugend an ihren Charakter in ihren Erscheinungen (den Handlungen) ausdrückt, die wegen der Gleichförmigkeit des Verhaltens einen Naturzusammenhang kenntlich machen, der aber nicht die arge Beschaffenheit des Willens notwendig macht, sondern vielmehr die Folge der freiwillig angenommenen bösen und unwandelbaren Grundsätze ist, welche ihn nur noch um desto verwerflicher und strafwürdiger machen. 
	
	\unnumberedsection{Substanz (1)} 
	\subsection*{tg215.2.15} 
	\textbf{Source : }Kritik der praktischen Vernunft/Erster Teil. Elementarlehre der reinen praktischen Vernunft/Erstes Buch. Die Analytik der reinen praktischen Vernunft/Drittes Hauptstück. Von den Triebfedern der reinen praktischen Vernunft/Kritische Beleuchtung der Analytik der reinen praktischen Vernunft\\  
	
	\textbf{Paragraphe : }Wenn man uns nämlich auch einräumt, daß das intelligibele Subjekt in Ansehung einer gegebenen Handlung noch frei sein kann, obgleich es als Subjekt, das auch zur Sinnenwelt gehörig, in Ansehung derselben mechanisch bedingt ist, so scheint es doch, man müsse, so bald man annimmt, Gott, als allgemeines Urwesen, sei die Ursache auch der Existenz der \match{Substanz} (ein Satz, der niemals aufgegeben  werden darf, ohne den Begriff von Gott als Wesen aller Wesen, und hiemit seine Allgenugsamkeit, auf die alles in der Theologie ankommt, zugleich mit aufzugeben), auch einräumen: Die Handlungen des Menschen haben in demjenigen ihren bestimmenden Grund, was gänzlich außer ihrer Gewalt ist, nämlich in der Kausalität eines von ihm unterschiedenen höchsten Wesens, von welchem das Dasein des erstern, und die ganze Bestimmung seiner Kausalität ganz und gar abhängt. In der Tat: wären die Handlungen des Menschen, so wie sie zu seinen Bestimmungen in der Zeit gehören, nicht bloße Bestimmungen desselben als Erscheinung, sondern als Dinges an sich selbst, so würde die Freiheit nicht zu retten sein. Der Mensch wäre Marionette, oder ein Vaucansonsches Automat, gezimmert und aufgezogen von dem obersten Meister aller Kunstwerke, und das Selbstbewußtsein würde es zwar zu einem denkenden Automate machen, in welchem aber das Bewußtsein seiner Spontaneität, wenn sie für Freiheit gehalten wird, bloße Täuschung wäre, indem sie nur komparativ so genannt zu werden verdient, weil die nächsten bestimmenden Ursachen seiner Bewegung, und eine lange Reihe derselben zu ihren bestimmenden Ursachen hinauf, zwar innerlich sind, die letzte und höchste aber doch gänzlich in einer fremden Hand angetroffen wird. Daher sehe ich nicht ab, wie diejenige, welche noch immer dabei beharren, Zeit und Raum für zum Dasein der Dinge an sich selbst gehörige Bestimmungen anzusehen, hier die Fatalität der Handlungen vermeiden wollen, oder, wenn sie so geradezu (wie der sonst scharfsinnige Mendelssohn tat) beide nur als zur Existenz endlicher und abgeleiteter Wesen, aber nicht zu der des unendlichen Urwesens notwendig gehörige Bedingungen einräumen, sich rechtfertigen wollen, woher sie diese Befugnis nehmen, einen solchen Unterschied zu machen, sogar wie sie auch nur dem Widerspruche ausweichen wollen, den sie begehen, wenn sie das Dasein in der Zeit als den endlichen Dingen an sich notwendig anhängende Bestimmung ansehen, da Gott die Ursache dieses Daseins ist, er aber doch nicht die Ursache der Zeit (oder des Raums) selbst sein kann (weil diese als  notwendige Bedingung a priori dem Dasein der Dinge vorausgesetzt sein muß), seine Kausalität folglich in Ansehung der Existenz dieser Dinge, selbst der Zeit nach, bedingt sein muß, wobei nun alle die Widersprüche gegen die Begriffe seiner Unendlichkeit und Unabhängigkeit unvermeidlich eintreten müssen. Hingegen ist es uns ganz leicht, die Bestimmung der göttlichen Existenz, als unabhängig von allen Zeitbedingungen, zum Unterschiede von der eines Wesens der Sinnenwelt, als die Existenz eines Wesens an sich selbst, von der eines Dinges in der Erscheinung zu unterscheiden. Daher, wenn man jene Idealität der Zeit und des Raums nicht annimmt, nur allein der Spinozism übrig bleibt, in welchem Raum und Zeit wesentliche Bestimmungen des Urwesens selbst sind, die von ihm abhängige Dinge aber (also auch wir selbst) nicht Substanzen, sondern bloß ihm inhärierende Akzidenzen sind; weil, wenn diese Dinge bloß, als seine Wirkungen, in der Zeit existieren, welche die Bedingung ihrer Existenz an sich wäre, auch die Handlungen dieser Wesen bloß seine Handlungen sein müßten, die er irgendwo und irgendwann ausübte. Daher schließt der Spinozism, unerachtet der Ungereimtheit seiner Grundidee, doch weit bündiger, als es nach der Schöpfungstheorie geschehen kann, wenn die für Substanzen angenommene und an sich in der Zeit existierende Wesen Wirkungen einer obersten Ursache, und doch nicht zugleich zu ihm und seiner Handlung, sondern für sich als Substanzen angesehen werden. 
	
	\unnumberedsection{Umfang (1)} 
	\subsection*{tg215.2.3} 
	\textbf{Source : }Kritik der praktischen Vernunft/Erster Teil. Elementarlehre der reinen praktischen Vernunft/Erstes Buch. Die Analytik der reinen praktischen Vernunft/Drittes Hauptstück. Von den Triebfedern der reinen praktischen Vernunft/Kritische Beleuchtung der Analytik der reinen praktischen Vernunft\\  
	
	\textbf{Paragraphe : }Die Analytik der reinen theoretischen Vernunft hatte es mit dem Erkenntnisse der Gegenstände, die dem Verstande gegeben werden mögen, zu tun, und mußte also von der Anschauung, mithin (weil diese jederzeit sinnlich ist) von der Sinnlichkeit anfangen, von da aber allererst zu Begriffen (der Gegenstände dieser Anschauung) fortschreiten, und durfte, nur nach beider Voranschickung, mit Grundsätzen endigen. Dagegen, weil praktische Vernunft es nicht mit Gegenständen, sie zu erkennen, sondern mit ihrem eigenen Vermögen, jene (der Erkenntnis derselben gemäß) wirklich zu machen, d.i. es mit einem Willen zu tun hat, welcher eine Kausalität ist, so fern Vernunft den Bestimmungsgrund derselben enthält, da sie folglich kein Objekt der Anschauung, sondern (weil der Begriff der Kausalität jederzeit die Beziehung auf ein Gesetz enthält, welches die Existenz des Mannigfaltigen im Verhältnisse zu einander bestimmt), als praktische Vernunft, nur ein Gesetz derselben anzugeben hat: so muß eine Kritik der Analytik derselben, so fern sie eine praktische Vernunft sein soll (welches die eigentliche Aufgabe ist), von der Möglichkeit praktischer Grundsätze a priori anfangen. Von da konnte sie allein zu Begriffen der Gegenstände einer praktischen Vernunft, nämlich denen des schlechthin-Guten und Bösen fortgehen, um sie jenen Grundsätzen gemäß allererst zu geben (denn diese sind vor jenen Prinzipien als Gutes und Böses durch gar kein Erkenntnisvermögen zu geben möglich), und nur alsdenn konnte allererst das letzte Hauptstück, nämlich das von dem Verhältnisse der reinen praktischen Vernunft zur Sinnlichkeit und ihrem notwendigen, a priori zu erkennenden Einflusse auf dieselbe, d.i. vom moralischen Gefühle, den Teil beschließen. So teilete  denn die Analytik der praktischen reinen Vernunft ganz analogisch mit der theoretischen den ganzen \match{Umfang} aller Bedingungen ihres Gebrauchs, aber in umgekehrter Ordnung. Die Analytik der theoretischen reinen Vernunft wurde in transzendentale Ästhetik und transzendentale Logik eingeteilt, die der praktischen umgekehrt in Logik und Ästhetik der reinen praktischen Vernunft (wenn es mir erlaubt ist, diese sonst gar nicht angemessene Benennungen, bloß der Analogie wegen, hier zu gebrauchen), die Logik wiederum dort in die Analytik der Begriffe und die der Grundsätze, hier in die der Grundsätze und Begriffe. Die Ästhetik hatte dort noch zwei Teile, wegen der doppelten Art einer sinnlichen Anschauung; hier wird die Sinnlichkeit gar nicht als Anschauungsfähigkeit, sondern bloß als Gefühl (das ein subjektiver Grund des Begehrens sein kann) betrachtet, und in Ansehung dessen verstattet die reine praktische Vernunft keine weitere Einteilung. 
	
	\unnumberedsection{Verbindung (8)} 
	\subsection*{tg211.2.11} 
	\textbf{Source : }Kritik der praktischen Vernunft/Erster Teil. Elementarlehre der reinen praktischen Vernunft/Erstes Buch. Die Analytik der reinen praktischen Vernunft/Erstes Hauptstück. Von den Grundsätzen der reinen praktischen Vernunft/II. Von dem Befugnisse der reinen Vernunft, im praktischen Gebrauche, zu einer Erweiterung, die ihr im spekulativen für sich nicht möglich ist\\  
	
	\textbf{Paragraphe : }Aber diese einmal eingeleitete objektive Realität eines reinen Verstandesbegriffs im Felde des Übersinnlichen gibt nunmehr allen übrigen Kategorien, obgleich immer nur, so fern sie mit dem Bestimmungsgrunde des reinen Willens (dem moralischen Gesetze) in notwendiger \match{Verbindung} stehen, auch objektive, nur keine andere als bloß praktisch-anwendbare Realität, indessen sie auf theoretische Erkenntnisse dieser Gegenstände, als Einsicht der Natur derselben durch reine Vernunft, nicht den mindesten Einfluß hat, um dieselbe zu erweitern. Wie wir denn auch in der Folge finden werden, daß sie immer nur auf Wesen als Intelligenzen, und an diesen auch nur auf das Verhältnis der Vernunft zum Willen, mithin immer nur aufs Praktische Beziehung haben und weiter hinaus sich kein Erkenntnis derselben anmaßen, was aber mit ihnen in Verbindung noch sonst für Eigenschaften, die zur theoretischen Vorstellungsart solcher übersinnlichen Dinge gehören, herbeigezogen werden möchten, diese insgesamt alsdenn gar nicht zum Wissen, sondern nur zur Befugnis (in praktischer Absicht aber gar zur Notwendigkeit), sie anzunehmen und vorauszusetzen, gezählt werden, selbst da, wo man übersinnliche Wesen (als Gott) nach einer Analogie, d.i. dem reinen Vernunftverhältnisse, dessen wir in Ansehung der sinnlichen uns praktisch bedienen, und so der reinen theoretischen Vernunft durch die Anwendung aufs Übersinnliche, aber nur in praktischer Absicht, zum Schwärmen ins Überschwengliche nicht den mindesten Vorschub gibt. 
	
	\subsection*{tg211.2.3} 
	\textbf{Source : }Kritik der praktischen Vernunft/Erster Teil. Elementarlehre der reinen praktischen Vernunft/Erstes Buch. Die Analytik der reinen praktischen Vernunft/Erstes Hauptstück. Von den Grundsätzen der reinen praktischen Vernunft/II. Von dem Befugnisse der reinen Vernunft, im praktischen Gebrauche, zu einer Erweiterung, die ihr im spekulativen für sich nicht möglich ist\\  
	
	\textbf{Paragraphe : }
	
	David Hume, von dem man sagen kann, daß er alle Anfechtung der Rechte einer reinen Vernunft, welche eine gänzliche Untersuchung derselben notwendig machten, eigentlich anfing, schloß so. Der Begriff der Ursache ist ein Begriff, der die Notwendigkeit der Verknüpfung der Existenz des Verschiedenen, und zwar, so fern es verschieden ist, enthält, so: daß, wenn A gesetzt wird, ich erkenne, daß etwas davon ganz Verschiedenes, B, notwendig auch existieren müsse. Notwendigkeit kann aber nur einer Verknüpfung beigelegt werden, so fern sie a priori erkannt wird; denn die Erfahrung würde von einer \match{Verbindung} nur zu erkennen geben, daß sie sei, aber nicht, daß sie so notwendigerweise sei. Nun ist es, sagt er, unmöglich, die Verbindung, die zwischen einem Dinge und einem anderen (oder einer Bestimmung und einer anderen, ganz von ihr verschiedenen), wenn sie nicht in der Wahrnehmung gegeben werden, a priori und als notwendig zu erkennen. Also ist der Begriff einer Ursache selbst lügenhaft und betrügerisch, und ist, am gelindesten davon zu reden, eine so fern noch zu entschuldigende Täuschung, da die Gewohnheit (eine subjektive Notwendigkeit), gewisse Dinge, oder ihre Bestimmungen, öfters neben, oder nach einander ihrer Existenz nach, als sich beigesellet, wahrzunehmen, unvermerkt für eine objektive Notwendigkeit, in den Gegenständen selbst eine solche Verknüpfung zu setzen, genommen, und so der Begriff einer Ursache erschlichen und nicht rechtmäßig erworben ist, ja auch niemals erworben oder beglaubigt werden kann, weil er eine an sich nichtige, chimärische, vor keiner Vernunft haltbare Verknüpfung fodert, der gar kein Objekt jemals korrespondieren kann. – So ward nun zuerst in Ansehung alles Erkenntnisses, das die Existenz der Dinge betrifft (die Mathematik blieb also davon noch ausgenommen), der Empirismus als die einzige Quelle der Prinzipien eingeführt, mit ihm aber zugleich der härteste Skeptizism selbst in Ansehung der ganzen Naturwissenschaft (als Philosophie). Denn wir können, nach solchen Grundsätzen, niemals aus gegebenen Bestimmungen der Dinge ihrer Existenz nach auf eine Folge schließen (denn dazu würde der Begriff einer Ursache, der die Notwendigkeit einer solchen Verknüpfung enthält, erfodert werden), sondern nur, nach der Regel der Einbildungskraft, ähnliche Fälle, wie sonst, erwarten, welche Erwartung aber niemals sicher ist, sie mag auch noch so oft eingetroffen sein. Ja bei keiner Begebenheit könnte man sagen: es müsse etwas vor ihr vorhergegangen sein, worauf sie notwendig folgte, d.i. sie müsse eine Ursache haben, und also, wenn man auch noch so öftere Fälle kennete, wo dergleichen vorherging, so daß eine Regel davon abgezogen werden konnte, so könnte man darum es nicht als immer und notwendig sich auf die Art zutragend annehmen, und so müsse man dem blinden Zufalle, bei welchem aller Vernunftgebrauch aufhört, auch sein Recht lassen, welches denn den Skeptizism, in Ansehung der von Wirkungen zu Ursachen aufsteigenden Schlüsse, fest gründet und unwiderleglich macht. 
	
	\subsection*{tg211.2.6} 
	\textbf{Source : }Kritik der praktischen Vernunft/Erster Teil. Elementarlehre der reinen praktischen Vernunft/Erstes Buch. Die Analytik der reinen praktischen Vernunft/Erstes Hauptstück. Von den Grundsätzen der reinen praktischen Vernunft/II. Von dem Befugnisse der reinen Vernunft, im praktischen Gebrauche, zu einer Erweiterung, die ihr im spekulativen für sich nicht möglich ist\\  
	
	\textbf{Paragraphe : }Aus meinen Untersuchungen aber ergab es sich, daß die Gegenstände, mit denen wir es in der Erfahrung zu tun haben, keinesweges Dinge an sich selbst, sondern bloß Erscheinungen sind, und daß, obgleich bei Dingen an sich selbst gar nicht abzusehen ist, ja unmöglich ist einzusehen, wie, wenn A gesetzt wird, es widersprechend sein solle. B, welches von A ganz verschieden ist, nicht zu setzen (die Notwendigkeit der Verknüpfung zwischen A als Ursache und B als Wirkung), es sich doch ganz wohl denken lasse, daß sie als Erscheinungen in einer Erfahrung auf gewisse Weise (z.B. in Ansehung der Zeitverhältnisse) notwendig verbunden sein müssen und nicht getrennt werden können, ohne derjenigen \match{Verbindung} zu widersprechen, vermittelst deren diese Erfahrung möglich ist, in welcher sie Gegenstände und uns allein erkennbar sind. Und so fand es sich auch in der Tat: so, daß ich den Begriff der Ursache nicht allein nach seiner objektiven Realität in Ansehung der Gegenstände der Erfahrung beweisen, sondern ihn auch, als Begriff a priori, wegen der Notwendigkeit der Verknüpfung, die er bei sich führt, deduzieren, d.i. seine Möglichkeit aus reinem Verstande, ohne empirische Quellen, dartun, und so, nach Wegschaffung des Empirismus seines Ursprungs, die unvermeidliche Folge desselben, nämlich den Skeptizism, zuerst in Ansehung der Naturwissenschaft, dann auch, wegen des ganz vollkommen aus denselben Gründen Folgenden in Ansehung der Mathematik, beider Wissenschaften, die auf Gegenstände möglicher Erfahrung bezogen werden, und hiemit den totalen Zweifel an allem, was theoretische Vernunft einzusehen behauptet, aus dem Grunde heben konnte. 
	
	\subsection*{tg214.2.14} 
	\textbf{Source : }Kritik der praktischen Vernunft/Erster Teil. Elementarlehre der reinen praktischen Vernunft/Erstes Buch. Die Analytik der reinen praktischen Vernunft\\  
	
	\textbf{Paragraphe : }Es liegt so etwas Besonderes in der grenzenlosen Hochschätzung des reinen, von allem Vorteil entblößten, moralischen Gesetzes, so wie es praktische Vernunft uns zur Befolgung vorstellt, deren Stimme auch den kühnsten Frevler zittern macht, und ihn nötigt, sich vor seinem Anblicke zu verbergen: daß man sich nicht wundern darf, diesen Einfluß einer bloß intellektuellen Idee aufs Gefühl für spekulative Vernunft unergründlich zu finden, und sich damit begnügen zu müssen, daß man a priori doch noch so viel einsehen kann: ein solches Gefühl sei unzertrennlich mit der Vorstellung des moralischen Gesetzes in jedem endlichen vernünftigen Wesen verbunden. Wäre dieses Gefühl der Achtung pathologisch und also ein auf dem inneren Sinne gegründetes Gefühl der Lust, so würde es vergeblich sein, eine \match{Verbindung} derselben mit irgend einer Idee a priori zu entdecken. Nun aber ist ein Gefühl, was bloß aufs Praktische geht, und zwar der Vorstellung eines Gesetzes lediglich seiner Form nach, nicht irgend eines Objekts desselben wegen, anhängt, mithin weder zum Vergnügen, noch zum Schmerze gerechnet werden kann, und dennoch ein Interesse an der Befolgung desselben hervorbringt, welches wir das moralische nennen; wie denn auch die Fähigkeit, ein solches  Interesse am Gesetze zu nehmen (oder die Achtung fürs moralische Gesetz selbst) eigentlich das moralische Gefühl ist. 
	
	\subsection*{tg218.2.7} 
	\textbf{Source : }Kritik der praktischen Vernunft/Erster Teil. Elementarlehre der reinen praktischen Vernunft/Zweites Buch. Dialektik der reinen praktischen Vernunft\\  
	
	\textbf{Paragraphe : }Nun ist aber aus der Analytik klar, daß die Maximen der Tugend und die der eigenen Glückseligkeit in Ansehung ihres obersten praktischen Prinzips ganz ungleichartig sind, und, weit gefehlt, einhellig zu sein, ob sie gleich zu einem höchsten Guten gehören, um das letztere möglich zu machen, einander in demselben Subjekte gar sehr einschränken und Abbruch tun. Also bleibt die Frage: wie ist das höchste Gut praktisch möglich, noch immer, unerachtet aller bisherigen Koalitionsversuche, eine unaufgelösete Aufgabe. Das aber, was sie zu einer schwer zu lösenden Aufgabe macht, ist in der Analytik gegeben, nämlich daß Glückseligkeit und Sittlichkeit zwei spezifisch ganz verschiedene Elemente des höchsten Guts sind, und ihre \match{Verbindung} also nicht analytisch erkannt werden könne (daß etwa der, so seine Glückseligkeit sucht, in diesem seinem Verhalten sich durch bloße Auflösung seiner Begriffe tugendhaft, oder der, so der Tugend folgt, sich im Bewußtsein eines solchen Verhaltens schon ipso facto glücklich finden werde), sondern eine Synthesis der Begriffe sei. Weil aber diese Verbindung als a priori, mithin praktisch notwendig, folglich nicht aus der Erfahrung abgeleitet, erkannt wird, und die Möglichkeit des höchsten Guts also auf keinen empirischen Prinzipien beruht, so wird die Deduktion dieses Begriffs transzendental sein müssen. Es ist a priori (moralisch) notwendig, das höchste Gut durch Freiheit des Willens hervorzubringen; es muß also auch die Bedingung der Möglichkeit desselben lediglich auf Erkenntnisgründen a priori beruhen. 
	
	\subsection*{tg219.2.2} 
	\textbf{Source : }Kritik der praktischen Vernunft/Erster Teil. Elementarlehre der reinen praktischen Vernunft/Zweites Buch. Dialektik der reinen praktischen Vernunft/Zweites Hauptstück. Von der Dialektik der reinen Vernunft in Bestimmung des Begriffs vom höchsten Gut/I. Die Antinomie der praktischen Vernunft\\  
	
	\textbf{Paragraphe : }In dem höchsten für uns praktischen, d.i. durch unsern Willen wirklich zu machenden, Gute werden Tugend und Glückseligkeit als notwendig verbunden gedacht, so, daß das eine durch reine praktische Vernunft nicht angenommen werden kann, ohne daß das andere auch zu ihm gehöre. Nun ist diese \match{Verbindung} (wie eine jede überhaupt) entweder analytisch, oder synthetisch. Da diese gegebene aber nicht analytisch sein kann, wie nur eben vorher gezeigt worden, so muß sie synthetisch, und zwar als Verknüpfung der Ursache mit der Wirkung gedacht werden; weil sie ein praktisches Gut, d.i. was durch Handlung möglich ist, betrifft. Es muß also entweder die Begierde nach Glückseligkeit die Bewegursache zu Maximen der Tugend, oder die Maxime der Tugend muß die wirkende Ursache der Glückseligkeit sein. Das erste ist schlechterdings unmöglich: weil (wie in der Analytik bewiesen worden) Maximen, die den Bestimmungsgrund des Willens in dem Verlangen nach seiner Glückseligkeit setzen, gar nicht moralisch sind, und keine Tugend gründen können. Das zweite ist aber auch unmöglich, weil alle praktische Verknüpfung der Ursachen und der Wirkungen in der Welt, als Erfolg der Willensbestimmung sich nicht nach moralischen Gesinnungen des Willens, sondern der Kenntnis der Naturgesetze und dem physischen Vermögen, sie zu seinen Absichten zu gebrauchen, richtet, folglich keine notwendige und zum höchsten Gut zureichende Verknüpfung der Glückseligkeit mit der Tugend in der Welt, durch die pünktlichste Beobachtung der moralischen Gesetze, erwartet werden kann. Da nun die Beförderung des höchsten Guts, welches diese Verknüpfung in seinem Begriffe enthält, ein a priori notwendiges Objekt unseres Willens ist, und mit dem moralischen Gesetze unzertrennlich zusammenhängt, so muß die Unmöglichkeit des ersteren auch die Falschheit des zweiten beweisen. Ist also das höchste Gut nach praktischen Regeln unmöglich, so muß auch das moralische Gesetz, welches gebietet, dasselbe  zu befördern, phantastisch und auf leere eingebildete Zwecke gestellt, mithin an sich falsch sein. 
	
	\subsection*{tg221.2.2} 
	\textbf{Source : }Kritik der praktischen Vernunft/Erster Teil. Elementarlehre der reinen praktischen Vernunft/Zweites Buch. Dialektik der reinen praktischen Vernunft/Zweites Hauptstück. Von der Dialektik der reinen Vernunft in Bestimmung des Begriffs vom höchsten Gut/III. Von dem Primat der reinen praktischen Vernunft in ihrer Verbindung mit der spekulativen\\  
	
	\textbf{Paragraphe : }Unter dem Primate zwischen zweien oder mehreren durch Vernunft verbundenen Dingen verstehe ich den Vorzug des einen, der erste Bestimmungsgrund der \match{Verbindung} mit allen übrigen zu sein. In engerer, praktischen Bedeutung bedeutet es den Vorzug des Interesse des einen, so fern ihm (welches keinem andern nachgesetzt werden kann) das Interesse der andern untergeordnet ist. Einem jeden Vermögen des Gemüts kann man ein Interesse beilegen, d.i. ein Prinzip, welches die Bedingung enthält, unter welcher allein die Ausübung desselben befördert wird. Die Vernunft, als das Vermögen der Prinzipien, bestimmt das Interesse aller Gemütskräfte, das ihrige aber sich selbst. Das Interesse  ihres spekulativen Gebrauchs besteht in der Erkenntnis des Objekts bis zu den höchsten Prinzipien a priori, das des praktischen Gebrauchs in der Bestimmung des Willens, in Ansehung des letzten und vollständigen Zwecks. Das, was zur Möglichkeit eines Vernunftgebrauchs überhaupt erfoderlich ist, nämlich daß die Prinzipien und Behauptungen derselben einander nicht widersprechen müssen, macht keinen Teil ihres Interesse aus, sondern ist die Bedingung, überhaupt Vernunft zu haben; nur die Erweiterung, nicht die bloße Zusammenstimmung mit sich selbst, wird zum Interesse derselben gezählt. 
	
	\subsection*{tg221.2.5} 
	\textbf{Source : }Kritik der praktischen Vernunft/Erster Teil. Elementarlehre der reinen praktischen Vernunft/Zweites Buch. Dialektik der reinen praktischen Vernunft/Zweites Hauptstück. Von der Dialektik der reinen Vernunft in Bestimmung des Begriffs vom höchsten Gut/III. Von dem Primat der reinen praktischen Vernunft in ihrer Verbindung mit der spekulativen\\  
	
	\textbf{Paragraphe : }In der \match{Verbindung} also der reinen spekulativen mit der reinen praktischen Vernunft zu einem Erkenntnisse führt die letztere das Primat, vorausgesetzt nämlich, daß diese Verbindung nicht etwa zufällig und beliebig, sondern a priori auf der Vernunft selbst gegründet, mithin notwendig sei. Denn es würde ohne diese Unterordnung ein Widerstreit der Vernunft mit ihr selbst entstehen; weil, wenn sie einander bloß beigeordnet (koordiniert) wären, die erstere für sich ihre Grenze enge verschließen und nichts von der  letzteren in ihr Gebiet aufnehmen, diese aber ihre Grenzen dennoch über alles ausdehnen, und, wo es ihr Bedürfnis erheischt, jene innerhalb der ihrigen mit zu befassen suchen würde. Der spekulativen Vernunft aber untergeordnet zu sein, und also die Ordnung umzukehren, kann man der reinen praktischen gar nicht zumuten, weil alles Interesse zuletzt praktisch ist, und selbst das der spekulativen Vernunft nur bedingt und im praktischen Gebrauche allein vollständig ist. 
	
	\unnumberedsection{Vereinigung (2)} 
	\subsection*{tg221.2.4} 
	\textbf{Source : }Kritik der praktischen Vernunft/Erster Teil. Elementarlehre der reinen praktischen Vernunft/Zweites Buch. Dialektik der reinen praktischen Vernunft/Zweites Hauptstück. Von der Dialektik der reinen Vernunft in Bestimmung des Begriffs vom höchsten Gut/III. Von dem Primat der reinen praktischen Vernunft in ihrer Verbindung mit der spekulativen\\  
	
	\textbf{Paragraphe : }
	In der Tat, so fern praktische Vernunft als pathologisch bedingt, d.i. das Interesse der Neigungen unter dem sinnlichen Prinzip der Glückseligkeit bloß verwaltend, zum Grunde gelegt würde, so ließe sich diese Zumutung an die spekulative Vernunft gar nicht tun. Mahomets Paradies, oder der Theosophen und Mystiker schmelzende \match{Vereinigung} mit der Gottheit, so wie jedem sein Sinn steht, würden der Vernunft ihre Ungeheuer aufdringen, und es wäre eben so gut, gar keine zu haben, als sie auf solche Weise allen Träumereien preiszugeben. Allein wenn reine Vernunft für sich praktisch sein kann und es wirklich ist, wie das Bewußtsein des moralischen Gesetzes es ausweiset, so ist es doch immer nur eine und dieselbe Vernunft, die, es sei in theoretischer oder praktischer Absicht, nach Prinzipien a priori urteilt, und da ist es klar, daß, wenn ihr Vermögen in der ersteren gleich nicht zulangt, gewisse Sätze behauptend festzusetzen, indessen daß sie ihr auch eben nicht widersprechen, eben diese Sätze, so bald sie unabtrennlich zum praktischen Interesse der reinen Vernunft gehören, zwar als ein ihr fremdes Angebot, das nicht auf ihrem Boden erwachsen, aber doch hinreichend beglaubigt ist, annehmen, und sie, mit allem, was sie als spekulative Vernunft in ihrer Macht hat, zu vergleichen und zu verknüpfen suchen müsse; doch sich bescheidend, daß dieses nicht ihre Einsichten, aber doch Erweiterungen ihres Gebrauchs in irgend einer anderen, nämlich praktischen, Absicht sind, welches ihrem Interesse, das in der Einschränkung des spekulativen Frevels besteht, ganz und gar nicht zuwider ist. 
	
	\subsection*{tg230.2.4} 
	\textbf{Source : }Kritik der praktischen Vernunft/Fußnoten\\  
	
	\textbf{Paragraphe : }
	
	2 Die \match{Vereinigung} der Kausalität, als Freiheit, mit ihr, als Naturmechanism, davon die erste durchs Sittengesetz, die zweite durchs Naturgesetz, und zwar in einem und demselben Subjekte, dem Menschen, fest steht, ist unmöglich, ohne diesen in Beziehung auf das erstere als Wesen an sich selbst, auf das zweite aber als Erscheinung, jenes im reinen, dieses im empirischen Bewußtsein, vorzustellen. Ohne dieses ist 
	der Widerspruch der Vernunft mit sich selbst unvermeidlich. 
	
	\unnumberedsection{Verfahren (1)} 
	\subsection*{tg213.2.4} 
	\textbf{Source : }Kritik der praktischen Vernunft/Erster Teil. Elementarlehre der reinen praktischen Vernunft/Erstes Buch. Die Analytik der reinen praktischen Vernunft/Zweites Hauptstück. Von dem Begriffe eines Gegenstandes der reinen praktischen Vernunft/Von der Typik der reinen praktischen Urteilskraft\\  
	
	\textbf{Paragraphe : }Dem Naturgesetze, als Gesetze, welchem die Gegenstände sinnlicher Anschauung, als solche, unterworfen sind, muß ein Schema, d.i. ein allgemeines \match{Verfahren} der Einbildungskraft (den reinen Verstandesbegriff, den das Gesetz bestimmt, den Sinnen a priori darzustellen), korrespondieren. Aber dem Gesetze der Freiheit (als einer gar nicht sinnlich bedingten Kausalität), mithin auch dem Begriffe des unbedingt-Guten, kann keine Anschauung, mithin kein Schema zum Behuf seiner Anwendung in concreto untergelegt werden. Folglich hat das Sittengesetz kein anderes, die Anwendung desselben auf Gegenstände der Natur vermittelndes Erkenntnisvermögen, als den Verstand (nicht die Einbildungskraft), welcher einer Idee der Vernunft nicht ein Schema der Sinnlichkeit, sondern ein Gesetz, aber doch ein solches, das an Gegenständen der Sinne in concreto dargestellt werden kann, mithin ein Naturgesetz, aber nur seiner Form nach, als Gesetz zum Behuf der Urteilskraft unterlegen kann, und dieses können wir daher den Typus des Sittengesetzes nennen. 
	
	\unnumberedsection{Verhältnis (2)} 
	\subsection*{tg208.2.14} 
	\textbf{Source : }Kritik der praktischen Vernunft/Erster Teil. Elementarlehre der reinen praktischen Vernunft/Erstes Buch. Die Analytik der reinen praktischen Vernunft/Erstes Hauptstück. Von den Grundsätzen der reinen praktischen Vernunft/7. Grundgesetz der reinen praktischen Vernunft\\  
	
	\textbf{Paragraphe : }Das vorher genannte Faktum ist unleugbar. Man darf nur das Urteil zergliedern, welches die Menschen über die Gesetzmäßigkeit ihrer Handlungen fällen: so wird man jederzeit finden, daß, was auch die Neigung dazwischen sprechen mag, ihre Vernunft dennoch, unbestechlich und durch sich selbst gezwungen, die Maxime des Willens bei einer Handlung jederzeit an den reinen Willen halte, d.i. an sich selbst, indem sie sich als a priori praktisch betrachtet. Dieses Prinzip der Sittlichkeit nun, eben um der Allgemeinheit der Gesetzgebung willen, die es zum formalen obersten Bestimmungsgrunde des Willens, unangesehen aller subjektiven Verschiedenheiten desselben, macht, erklärt die Vernunft zugleich zu einem Gesetze für alle vernünftige Wesen, so fern sie überhaupt einen Willen, d.i. ein Vermögen haben, ihre Kausalität durch die Vorstellung von Regeln zu bestimmen, mithin so fern sie der Handlungen nach Grundsätzen, folglich auch nach praktischen Prinzipien a priori (denn diese haben allein diejenige Notwendigkeit, welche die Vernunft zum Grundsatze fodert), fähig sein. Es  schränkt sich also nicht bloß auf Menschen ein, sondern geht auf alle endliche Wesen, die Vernunft und Willen haben, ja schließt sogar das unendliche Wesen, als oberste Intelligenz, mit ein. Im ersteren Falle aber hat das Gesetz die Form eines Imperativs, weil man an jenem zwar, als vernünftigem Wesen, einen reinen, aber, als mit Bedürfnissen und sinnlichen Bewegursachen affiziertem Wesen, keinen heiligen Willen, d.i. einen solchen, der keiner dem moralischen Gesetze widerstreitenden Maximen fähig wäre, voraussetzen kann. Das moralische Gesetz ist daher bei jenen ein Imperativ, der kategorisch gebietet, weil das Gesetz unbedingt ist; das \match{Verhältnis} eines solchen Willens zu diesem Gesetze ist Abhängigkeit, unter dem Namen der Verbindlichkeit, welche eine Nötigung, obzwar durch bloße Vernunft und dessen objektives Gesetz, zu einer Handlung bedeutet, die darum Pflicht heißt, weil eine pathologisch affizierte (obgleich dadurch nicht bestimmte, mithin auch immer freie) Willkür einen Wunsch bei sich führt, der aus subjektiven Ursachen entspringt, daher auch dem reinen objektiven Bestimmungsgrunde oft entgegen sein kann, und also eines Widerstandes der praktischen Vernunft, der ein innerer, aber intellektueller, Zwang genannt werden kann, als moralischer Nötigung bedarf. In der allergnugsamsten Intelligenz wird die Willkür, als keiner Maxime fähig, die nicht zugleich objektiv Gesetz sein konnte, mit Recht vorgestellt, und der Begriff der Heiligkeit, der ihr um deswillen zukommt, setzt sie zwar nicht über alle praktische, aber doch über alle praktisch-einschränkende Gesetze, mithin Verbindlichkeit und Pflicht weg. Diese Heiligkeit des Willens ist gleichwohl eine praktische Idee, welche notwendig zum Urbilde dienen muß, welchem sich ins Unendliche zu nähern das einzige ist, was allen endlichen vernünftigen Wesen zusteht, und welche das reine Sittengesetz, das darum selbst heilig heißt, ihnen beständig und richtig vor Augen hält, von welchem ins Unendliche gehenden Progressus seiner Maximen und Unwandelbarkeit derselben zum beständigen Fortschreiten sicher zu sein, d.i. Tugend, das  Höchste ist, was endliche praktische Vernunft bewirken kann, die selbst wiederum wenigstens als natürlich erworbenes Vermögen nie vollendet sein kann, weil die Sicherheit in solchem Falle niemals apodiktische Gewißheit wird, und als Überredung sehr gefährlich ist. 
	
	\subsection*{tg225.2.8} 
	\textbf{Source : }Kritik der praktischen Vernunft/Erster Teil. Elementarlehre der reinen praktischen Vernunft/Zweites Buch. Dialektik der reinen praktischen Vernunft/Zweites Hauptstück. Von der Dialektik der reinen Vernunft in Bestimmung des Begriffs vom höchsten Gut/VII. Wie eine Erweiterung der reinen Vernunft, in praktischer Absicht, ohne damit ihr Erkenntnis, als spekulativ, zugleich zu erweitern, zu denken möglich sei\\  
	
	\textbf{Paragraphe : }Dieses letztere ist so augenscheinlich, und kann so klar durch die Tat bewiesen werden, daß man getrost alle vermeinte natürliche Gottesgelehrte (ein wunderlicher Name)
	
	
	15
	auffodern kann, auch nur eine diesen ihren Gegenstand (über die bloß ontologischen Prädikate hinaus) bestimmende Eigenschaft, etwa des Verstandes, oder des Willens, zu nennen, an der man nicht unwidersprechlich dartun könnte, daß, wenn man alles Anthropomorphistische davon absondert, uns nur das bloße Wort übrig bleibe, ohne damit den mindesten Begriff verbinden zu können, dadurch eine Erweiterung der theoretischen Erkenntnis gehofft werden dürfte. In Ansehung des Praktischen aber bleibt uns von den Eigenschaften eines Verstandes und Willens doch noch der Begriff eines Verhältnisses übrig, welchem das praktische Gesetz (das gerade dieses \match{Verhältnis} des Verstandes zum Willen a priori bestimmt) objektive Realität verschafft. Ist dieses nun einmal geschehen, so wird dem Begriffe des Objekts eines moralisch bestimmten Willens (dem des höchsten Guts) und mit ihm den Bedingungen seiner Möglichkeit, den Ideen von Gott, Freiheit und Unsterblichkeit, auch Realität, aber immer nur in Beziehung auf die Ausübung des moralischen Gesetzes (zu keinem spekulativen Behuf), gegeben. 
	
	\unnumberedsection{Wurzel (1)} 
	\subsection*{tg214.2.24} 
	\textbf{Source : }Kritik der praktischen Vernunft/Erster Teil. Elementarlehre der reinen praktischen Vernunft/Erstes Buch. Die Analytik der reinen praktischen Vernunft\\  
	
	\textbf{Paragraphe : }
	Pflicht! du erhabener großer Name, der du nichts Beliebtes, was Einschmeichelung bei sich führt, in dir fassest, sondern Unterwerfung verlangst, doch auch nichts drohest, was natürliche Abneigung im Gemüte erregte und schreckte, um den Willen zu bewegen, sondern bloß ein Gesetz aufstellst, welches von selbst im Gemüte Eingang findet, und doch sich selbst wider Willen Verehrung (wenn gleich nicht immer Befolgung) erwirbt, vor dem alle Neigungen verstummen, wenn sie gleich in Geheim ihm entgegen wirken, welches ist der deiner würdige Ursprung, und wo findet man die \match{Wurzel} deiner edlen Abkunft, welche alle Verwandtschaft mit Neigungen stolz ausschlägt, und von welcher Wurzel abzustammen die unnachlaßliche Bedingung desjenigen Werts ist, den sich Menschen allein selbst geben können? 
	
	\unnumberedsection{Zeitalter (1)} 
	\subsection*{tg197.2.18} 
	\textbf{Source : }Kritik der praktischen Vernunft/Vorrede\\  
	
	\textbf{Paragraphe : }Doch, da es in diesem philosophischen und kritischen \match{Zeitalter} schwerlich mit jenem Empirism Ernst sein kann, und er vermutlich nur zur Übung der Urteilskraft, und, um durch den Kontrast die Notwendigkeit rationaler Prinzipien a priori in ein helleres Licht zu setzen, aufgestellet wird: so kann man es denen doch Dank wissen, die sich mit dieser sonst eben nicht belehrenden Arbeit bemühen wollen. 
	
	\unnumberedsection{Zwei (2)} 
	\subsection*{tg218.2.3} 
	\textbf{Source : }Kritik der praktischen Vernunft/Erster Teil. Elementarlehre der reinen praktischen Vernunft/Zweites Buch. Dialektik der reinen praktischen Vernunft\\  
	
	\textbf{Paragraphe : }\match{Zwei} in einem Begriffe notwendig verbundene Bestimmungen müssen als Grund und Folge verknüpft sein, und zwar entweder so, daß diese Einheit als analytisch (logische Verknüpfung) oder als synthetisch (reale Verbindung), jene nach dem Gesetze der Identität, diese der Kausalität betrachtet wird. Die Verknüpfung der Tugend mit der Glückseligkeit kann also entweder so verstanden werden, daß die Bestrebung tugendhaft zu sein und die vernünftige Bewerbung um Glückseligkeit nicht zwei verschiedene, sondern ganz identische Handlungen wären, da denn der ersteren keine andere Maxime, als zu der letztern zum Grunde gelegt zu werden brauchte: oder jene Verknüpfung wird darauf ausgesetzt, daß Tugend die Glückseligkeit als etwas von dem Bewußtsein der ersteren Unterschiedenes, wie die Ursache eine Wirkung, hervorbringe. 
	
	\subsection*{tg229.2.2} 
	\textbf{Source : }Kritik der praktischen Vernunft/Beschluß\\  
	
	\textbf{Paragraphe : }\match{Zwei} Dinge erfüllen das Gemüt mit immer neuer und zunehmenden Bewunderung und Ehrfurcht, je öfter und anhaltender sich das Nachdenken damit beschäftigt: Der bestirnte Himmel über mir, und das moralische Gesetz in mir. Beide darf ich nicht als in Dunkelheiten verhüllt, oder im Überschwenglichen, außer meinem Gesichtskreise, suchen und bloß vermuten; ich sehe sie vor mir und verknüpfe sie unmittelbar mit dem Bewußtsein meiner Existenz. Das erste fängt von dem Platze an, den ich in der äußern Sinnenwelt einnehme, und erweitert die Verknüpfung, darin ich stehe, ins unabsehlich-Große mit Welten über Welten und Systemen von Systemen, überdem noch in grenzenlose Zeiten ihrer periodischen Bewegung, deren Anfang und Fortdauer. Das zweite fängt von meinem unsichtbaren Selbst, meiner Persönlichkeit, an, und stellt mich in einer Welt dar, die wahre Unendlichkeit hat, aber nur dem Verstande spürbar ist, und mit welcher (dadurch aber auch zugleich mit allen jenen sichtbaren Welten) ich mich nicht, wie dort, in bloß zufälliger, sondern allgemeiner und notwendiger Verknüpfung erkenne. Der erstere Anblick einer zahllosen Weltenmenge vernichtet gleichsam meine Wichtigkeit, als eines tierischen Geschöpfs, das die Materie, daraus es ward, dem Planeten (einem bloßen Punkt im Weltall) wieder zurückgeben muß, nachdem es eine kurze Zeit (man weiß nicht wie) mit Lebenskraft versehen gewesen. Der zweite erhebt dagegen meinen Wert, als einer Intelligenz, unendlich, durch meine Persönlichkeit, in welcher das moralische Gesetz mir ein von der Tierheit und selbst von der ganzen Sinnenwelt unabhängiges Leben offenbart, wenigstens so viel sich aus der zweckmäßigen Bestimmung meines Daseins durch dieses Gesetz, welche nicht auf Bedingungen und Grenzen dieses Lebens eingeschränkt ist, sondern ins Unendliche geht, abnehmen läßt. 
	
	\unnumberedchapter{Vetement} 
	\unnumberedsection{Anhänger (1)} 
	\subsection*{tg230.2.26} 
	\textbf{Source : }Kritik der praktischen Vernunft/Fußnoten\\  
	
	\textbf{Paragraphe : }
	
	13 Man hält gemeiniglich dafür, die christliche Vorschrift der Sitten habe in Ansehung ihrer Reinigkeit vor dem moralischen Begriffe der Stoiker nichts voraus; allein der Unterschied beider ist doch sehr sichtbar. Das stoische System machte das Bewußtsein der Seelenstärke zum Angel, um den sich alle sittliche Gesinnungen wenden sollten, und, ob die \match{Anhänger} dessen zwar von Pflichten redeten, auch sie ganz wohl bestimmeten, so setzen sie doch die Triebfeder und den eigentlichen Bestimmungsgrund des Willens in einer Erhebung der Denkungsart über die niedrige und nur durch Seelenschwäche machthabende Triebfedern der Sinne. Tugend war also bei ihnen ein gewisser Heroism des über tierische Natur des Menschen sich erhebenden Weisen, der ihm selbst genug ist, andern zwar Pflichten vorträgt, selbst aber über sie erhoben und keiner Versuchung zu Übertretung des sittlichen Gesetzes unterworfen ist. Dieses alles aber konnten sie nicht tun, wenn sie sich dieses Gesetz in der Reinigkeit und Strenge, als es die Vorschrift des Evangelii tut, vorgestellt hätten. Wenn ich unter einer Idee eine Vollkommenheit verstehe, der nichts in der Erfahrung adäquat gegeben werden kann, so sind die moralischen Ideen darum nichts Überschwengliches, d.i. dergleichen, wovon wir auch nicht einmal den Begriff hinreichend bestimmen könnten, oder von dem es ungewiß ist, ob ihm überall ein Gegenstand korrespondiere, wie die Ideen der spekulativen Vernunft, sondern dienen, als Urbilder der praktischen Vollkommenheit, zur unentbehrlichen Richtschnur des sittlichen Verhaltens, und zugleich zum Maßstabe der Vergleichung. Wenn ich nun die christliche Moral von ihrer philosophischen Seite betrachte, so würde sie, mit den Ideen der griechischen Schulen verglichen, so erscheinen: Die Ideen der Kyniker, der Epikureer, der Stoiker und des Christen sind: die Natureinfalt, die Klugheit, die Weisheit und die Heiligkeit. In Ansehung des Weges, dazu zu gelangen, unterschieden sich die griechischen Philosophen so von einander, daß die Kyniker dazu den gemeinen Menschenverstand, die andern nur den Weg der Wissenschaft, beide also doch bloßen Gebrauch der natürlichen Kräfte dazu hinreichend fanden. Die christliche Moral, weil sie ihre Vorschrift (wie es auch sein muß) so rein und unnachsichtlich einrichtet, benimmt dem Menschen das Zutrauen, wenigstens hier im Leben, ihr völlig adäquat zu sein, richtet es aber doch auch dadurch wiederum auf, daß, wenn wir so gut handeln, als in unserem Vermögen ist, wir hoffen können, daß, was nicht in unserm Vermögen ist, uns anderweitig werde zu statten kommen, wir mögen nun wissen, auf welche Art, oder nicht. Aristoteles und Plato unterschieden sich nur in Ansehung des Ursprungs unserer sittlichen Begriffe. 
	
	\unnumberedsection{Fleck (1)} 
	\subsection*{tg228.2.6} 
	\textbf{Source : }Kritik der praktischen Vernunft/Zweiter Teil. Methodenlehre der reinen praktischen Vernunft\\  
	
	\textbf{Paragraphe : }Wenn man auf den Gang der Gespräche in gemischten Gesellschaften, die nicht bloß aus Gelehrten und Vernünftlern, sondern auch aus Leuten von Geschäften oder Frauenzimmer bestehen, Acht hat, so bemerkt man, daß, außer dem Erzählen und Scherzen, noch eine Unterhaltung, nämlich das Räsonieren, darin Platz findet; weil das erstere, wenn es Neuigkeit, und, mit ihr, Interesse bei sich führen soll, bald erschöpft, das zweite aber leicht schal wird. Unter allem Räsonieren ist aber keines, was mehr den Beitritt der Personen, die sonst bei allem Vernünfteln bald lange Weile haben, erregt, und eine gewisse Lebhaftigkeit in die Gesellschaft bringt, als das über den sittlichen Wert dieser oder jener Handlung, dadurch der Charakter irgend einer Person ausgemacht werden soll. Diejenige, welchen sonst alles Subtile und Grüblerische in theoretischen Fragen trocken und verdrießlich ist, treten bald bei, wenn es darauf ankommt, den moralischen Gehalt einer erzählten guten oder bösen Handlung auszumachen, und sind so genau, so grüblerisch, so subtil, alles, was die Reinigkeit der Absicht, und mithin  den Grad der Tugend in derselben vermindern, oder auch nur verdächtig machen könnte, auszusinnen, als man bei keinem Objekte der Spekulation sonst von ihnen erwartet. Man kann in diesen Beurteilungen oft den Charakter der über andere urteilenden Personen selbst hervorschimmern sehen, deren einige vorzüglich geneigt scheinen, indem sie ihr Richteramt, vornehmlich über Verstorbene, ausüben, das Gute, was von dieser oder jener Tat derselben erzählt wird, wider alle kränkende Einwürfe der Unlauterkeit und zuletzt den ganzen sittlichen Wert der Person wider den Vorwurf der Verstellung und geheimen Bösartigkeit zu verteidigen, andere dagegen mehr auf Anklagen und Beschuldigungen sinnen, diesen Wert anzufechten. Doch kann man den letzteren nicht immer die Absicht beimessen, Tugend aus allen Beispielen der Menschen gänzlich wegvernünfteln zu wollen, um sie dadurch zum leeren Namen zu machen, sondern es ist oft nur wohlgemeinte Strenge in Bestimmung des echten sittlichen Gehalts, nach einem unnachsichtlichen Gesetze, mit welchem und nicht mit Beispielen verglichen der Eigendünkel im Moralischen sehr sinkt, und Demut nicht etwa bloß gelehrt, sondern bei scharfer Selbstprüfung von jedem gefühlt wird. Dennoch kann man den Verteidigern der Reinigkeit der Absicht in gegebenen Beispielen es mehrenteils ansehen, daß sie ihr da, wo sie die Vermutung der Rechtschaffenheit für sich hat, auch den mindesten \match{Fleck} gerne abwischen möchten, aus dem Bewegungsgrunde, damit nicht, wenn allen Beispielen ihre Wahrhaftigkeit gestritten und aller menschlichen Tugend die Lauterkeit weggeleugnet würde, diese nicht endlich gar für ein bloßes Hirngespinst gehalten, und so alle Bestrebung zu derselben als eitles Geziere und trüglicher Eigendünkel geringschätzig gemacht werde. 
	
	\unnumberedsection{Größe (4)} 
	\subsection*{tg204.2.9} 
	\textbf{Source : }Kritik der praktischen Vernunft/Erster Teil. Elementarlehre der reinen praktischen Vernunft/Erstes Buch. Die Analytik der reinen praktischen Vernunft/Erstes Hauptstück. Von den Grundsätzen der reinen praktischen Vernunft/3. Lehrsatz II\\  
	
	\textbf{Paragraphe : }Man muß sich wundern, wie sonst scharfsinnige Männer einen Unterschied zwischen dem unteren und oberen Begehrungsvermögen darin zu finden glauben können, ob die Vorstellungen, die mit dem Gefühl der Lust verbunden sind, in den Sinnen, oder dem Verstande ihren Ursprung haben. Denn es kommt, wenn man nach den Bestimmungsgründen des Begehrens fragt und sie in einer von irgend etwas erwarteten Annehmlichkeit setzt, gar nicht darauf an, wo die Vorstellung dieses vergnügenden Gegenstandes herkomme, sondern nur, wie sehr sie vergnügt. Wenn eine Vorstellung, sie mag immerhin im Verstande ihren Sitz und Ursprung haben, die Willkür nur dadurch bestimmen  kann, daß sie ein Gefühl einer Lust im Subjekte voraussetzet, so ist, daß sie ein Bestimmungsgrund der Willkür sei, gänzlich von der Beschaffenheit des inneren Sinnes abhängig, daß dieser nämlich dadurch mit Annehmlichkeit affiziert werden kann. Die Vorstellungen der Gegenstände mögen noch so ungleichartig, sie mögen Verstandes-, selbst Vernunftvorstellungen im Gegensatze der Vorstellungen der Sinne sein, so ist doch das Gefühl der Lust, wodurch jene doch eigentlich nur den Bestimmungsgrund des Willens ausmachen (die Annehmlichkeit, das Vergnügen, das man davon erwartet, welches die Tätigkeit zur Hervorbringung des Objekts antreibt), nicht allein so fern von einerlei Art, daß es jederzeit bloß empirisch erkannt werden kann, sondern auch so fern, als er eine und dieselbe Lebenskraft, die sich im Begehrungsvermögen äußert, affiziert, und in dieser Beziehung von jedem anderen Bestimmungsgrunde in nichts, als dem Grade, verschieden sein kann. Wie würde man sonsten zwischen zwei der Vorstellungsart nach gänzlich verschiedenen Bestimmungsgründen eine Vergleichung der \match{Größe} nach anstellen können, um den, der am meisten das Begehrungsvermögen affiziert, vorzuziehen? Eben derselbe Mensch kann ein ihm lehrreiches Buch, das ihm nur einmal zu Händen kommt, ungelesen zurückgeben, um die Jagd nicht zu versäumen, in der Mitte einer schönen Rede weggehen, um zur Mahlzeit nicht zu spät zu kommen, eine Unterhaltung durch vernünftige Gespräche, die er sonst sehr schätzt, verlassen, um sich an den Spieltisch zu setzen, so gar einen Armen, dem wohlzutun ihm sonst Freude ist, abweisen, weil er jetzt eben nicht mehr Geld in der Tasche hat, als er braucht, um den Eintritt in die Komödie zu bezahlen. Beruht die Willensbestimmung auf dem Gefühle der Annehmlichkeit oder Unannehmlichkeit, die er aus irgend einer Ursache erwartet, so ist es ihm gänzlich einerlei, durch welche Vorstellungsart er affiziert werde. Nur wie stark, wie lange, wie leicht erworben und oft wiederholt diese Annehmlichkeit sei, daran liegt es ihm, um sich zur Wahl zu entschließen. So wie demjenigen, der Gold zur Ausgabe  braucht, gänzlich einerlei ist, ob die Materie desselben, das Gold, aus dem Gebirge gegraben, oder aus dem Sande gewaschen ist, wenn es nur allenthalben für denselben Wert angenommen wird, so fragt kein Mensch, wenn es ihm bloß an der Annehmlichkeit des Lebens gelegen ist, ob Verstandes- oder Sinnesvorstellungen, sondern nur, wie viel und großes Vergnügen sie ihm auf die längste Zeit verschaffen. Nur diejenigen, welche der reinen Vernunft das Vermögen, ohne Voraussetzung irgend eines Gefühls den Willen zu bestimmen, gerne abstreiten möchten, können sich so weit von ihrer eigenen Erklärung verirren, das, was sie selbst vorher auf ein und eben dasselbe Prinzip gebracht haben, dennoch hernach für ganz ungleichartig zu erklären. So findet sich z.B., daß man auch an bloßer Kraftanwendung, an dem Bewußtsein seiner Seelenstärke in Überwindung der Hindernisse, die sich unserem Vorsatze entgegensetzen, an der Kultur der Geistestalente, u.s.w., Vergnügen finden könne, und wir nennen das mit Recht feinere Freuden und Ergötzungen, weil sie mehr, wie andere, in unserer Gewalt sind, sich nicht abnutzen, das Gefühl zu noch mehrerem Genuß derselben vielmehr stärken, und, indem sie ergötzen, zugleich kultivieren. Allein sie darum für eine andere Art, den Willen zu bestimmen, als bloß durch den Sinn, auszugeben, da sie doch einmal, zur Möglichkeit jener Vergnügen, ein darauf in uns angelegtes Gefühle als erste Bedingung dieses Wohlgefallens, voraussetzen, ist gerade so, als wenn Unwissende, die gerne in der Metaphysik pfuschern möchten, sich die Materie so fein, so überfein, daß sie selbst darüber schwindlig werden möchten, denken, und dann glauben, auf diese Art sich ein geistiges und doch ausgedehntes Wesen erdacht zu haben. Wenn wir es, mit dem Epikur, bei der Tugend aufs bloße Vergnügen aussetzen, das sie verspricht, um den Willen zu bestimmen: so können wir ihn hernach nicht tadeln, daß er dieses mit denen der gröbsten Sinne für ganz gleichartig hält; denn man hat gar nicht Grund, ihm aufzubürden, daß er die Vorstellungen, wodurch dieses Gefühl in uns erregt würde, bloß den körperlichen Sinnen beigemessen hätte. Er hat von vielen derselben  den Quell, so viel man erraten kann, eben sowohl in dem Gebrauch des höheren Erkenntnisvermögens gesucht; aber das hinderte ihn nicht und konnte ihn auch nicht hindern, nach genanntem Prinzip das Vergnügen selbst, das uns jene allenfalls intellektuelle Vorstellungen gewähren, und wodurch sie allein Bestimmungsgründe des Willens sein können, gänzlich für gleichartig zu halten. Konsequent zu sein, ist die größte Obliegenheit eines Philosophen, und wird doch am seltensten angetroffen. Die alten griechischen Schulen geben uns davon mehr Beispiele, als wir in unserem synkretistischen Zeitalter antreffen, wo ein gewisses Koalitionssystem widersprechender Grundsätze voll Unredlichkeit und Seichtigkeit erkünstelt wird, weil es sich einem Publikum besser empfiehlt, das zufrieden ist, von allem etwas, und im ganzen nichts zu wissen, und dabei in allen Sätteln gerecht zu sein. Das Prinzip der eigenen Glückseligkeit, so viel Verstand und Vernunft bei ihm auch gebraucht werden mag, würde doch für den Willen keine andere Bestimmungsgründe, als die dem unteren Begehrungsvermögen angemessen sind, in sich fassen, und es gibt also entweder gar kein Begehrungsvermögen oder reine Vernunft muß für sich allein praktisch sein, d.i. ohne Voraussetzung irgend eines Gefühls, mithin ohne Vorstellungen des Angenehmen oder Unangenehmen, als der Materie des Begehrungsvermögens, die jederzeit eine empirische Bedingung der Prinzipien ist, durch die bloße Form der praktischen Regel den Willen bestimmen können. Alsdenn allein ist Vernunft nur, so fern sie für sich selbst den Willen bestimmt (nicht im Dienste der Neigungen ist), ein wahres oberes Begehrungsvermögen, dem das pathologisch bestimmbare untergeordnet ist, und wirklich, ja spezifisch von diesem unterschieden, so daß sogar die mindeste Beimischung von den Antrieben der letzteren ihrer Stärke und Vorzuge Abbruch tut, so wie das mindeste Empirische, als Bedingung in einer mathematischen Demonstration, ihre Würde und Nachdruck herabsetzt und vernichtet. Die Vernunft bestimmt in einem praktischen Gesetze unmittelbar  den Willen, nicht vermittelst eines dazwischen kommenden Gefühls der Lust und Unlust, selbst nicht an diesem Gesetze, und nur, daß sie als reine Vernunft praktisch sein kann, macht es ihr möglich, gesetzgebend zu sein. 
	
	\subsection*{tg215.2.22} 
	\textbf{Source : }Kritik der praktischen Vernunft/Erster Teil. Elementarlehre der reinen praktischen Vernunft/Erstes Buch. Die Analytik der reinen praktischen Vernunft/Drittes Hauptstück. Von den Triebfedern der reinen praktischen Vernunft/Kritische Beleuchtung der Analytik der reinen praktischen Vernunft\\  
	
	\textbf{Paragraphe : }Da es eigentlich der Begriff der Freiheit ist, der, unter allen Ideen der reinen spekulativen Vernunft, allein so große Erweiterung im Felde des Übersinnlichen, wenn gleich nur in Ansehung des praktischen Erkenntnisses verschafft, so frage ich mich: woher denn ihm ausschließungsweise eine so große Fruchtbarkeit zu Teil geworden sei, indessen die übrigen zwar die leere Stelle für reine mögliche Verstandeswesen bezeichnen, den Begriff von ihnen aber durch nichts bestimmen können. Ich begreife bald, daß, da ich nichts ohne Kategorie denken kann, diese auch in der Idee der Vernunft von der Freiheit, mit der ich mich beschäftige, zuerst müsse aufgesucht werden, welche hier die Kategorie der Kausalität ist, und daß ich, wenn gleich dem Vernunftbegriffe der Freiheit, als überschwenglichem Begriffe, keine korrespondierende Anschauung untergelegt werden kann, dennoch dem Verstandesbegriffe (der Kausalität), für dessen Synthesis jener das Unbedingte fodert, zuvor eine sinnliche Anschauung gegeben werden müsse, dadurch ihm zuerst die objektive Realität gesichert wird. Nun sind alle Kategorien in zwei Klassen, die mathematische, welche bloß auf die Einheit der Synthesis in der Vorstellung der Objekte, und die dynamische, welche auf die in der Vorstellung der Existenz der Objekte gehen, eingeteilt. Die erstere (die der \match{Größe} und der Qualität) enthalten jederzeit eine Synthesis des Gleichartigen, in welcher das Unbedingte, zu dem in der sinnlichen Anschauung gegebenen Bedingten in Raum und Zeit, da es selbst wiederum zum Raume und der Zeit gehören, und also  immer wieder unbedingt sein mußte, gar nicht kann gefunden werden; daher auch in der Dialektik der reinen theoretischen Vernunft die einander entgegengesetzte Arten, das Unbedingte und die Totalität der Bedingungen für sie zu finden, beide falsch waren. Die Kategorien der zweiten Klasse (die der Kausalität und der Notwendigkeit eines Dinges) erforderten diese Gleichartigkeit (des Bedingten und der Bedingung in der Synthesis) gar nicht, weil hier nicht die Anschauung, wie sie aus einem Mannigfaltigen in ihr zusammengesetzt, sondern nur, wie die Existenz des ihr korrespondierenden bedingten Gegenstandes zu der Existenz der Bedingung (im Verstande als damit verknüpft) hinzukomme, vorgestellt werden solle, und da war es erlaubt, zu dem durchgängig Bedingten in der Sinnenwelt (so wohl in Ansehung der Kausalität als des zufälligen Daseins der Dinge selbst) das Unbedingte, obzwar übrigens unbestimmt, in der intelligibelen Welt zu setzen, und die Synthesis transzendent zu machen; daher denn auch in der Dialektik der r. spek. V. sich fand, daß beide, dem Scheine nach, einander entgegengesetzte Arten, das Unbedingte zum Bedingten zu finden, z.B. in der Synthesis der Kausalität zum Bedingten, in der Reihe der Ursachen und Wirkungen der Sinnenwelt, die Kausalität, die weiter nicht sinnlich bedingt ist, zu denken, sich in der Tat nicht widerspreche, und daß dieselbe Handlung, die, als zur Sinnenwelt gehörig, jederzeit sinnlich bedingt, d.i. mechanisch-notwendig ist, doch zugleich auch, als zur Kausalität des handelnden Wesens, so fern es zur intelligibelen Welt gehörig ist, eine sinnlich unbedingte Kausalität zum Grunde haben, mithin als frei gedacht werden könne. Nun kam es bloß darauf an, daß dieses Können in ein Sein verwandelt würde, d.i., daß man in einem wirklichen Falle, gleichsam durch ein Faktum, beweisen könne: daß gewisse Handlungen eine solche Kausalität (die intellektuelle, sinnlich unbedingte) voraussetzen, sie mögen nun wirklich, oder auch nur geboten, d.i. objektiv praktisch notwendig sein. An wirklich in der Erfahrung  gegebenen Handlungen, als Begebenheiten der Sinnenwelt, konnten wir diese Verknüpfung nicht anzutreffen hoffen, weil die Kausalität durch Freiheit immer außer der Sinnenwelt im Intelligibelen gesucht werden muß. Andere Dinge, außer den Sinnenwesen, sind uns aber zur Wahrnehmung und Beobachtung nicht gegeben. Also blieb nichts übrig, als daß etwa ein unwidersprechlicher und zwar objektiver Grundsatz der Kausalität, welcher alle sinnliche Bedingung von ihrer Bestimmung ausschließt, d.i. ein Grundsatz, in welchem die Vernunft sich nicht weiter auf etwas anderes als Bestimmungsgrund in Ansehung der Kausalität beruft, sondern den sie durch jenen Grundsatz schon selbst enthält, und wo sie also, als reine Vernunft, selbst praktisch ist, gefunden werde. Dieser Grundsatz aber bedarf keines Suchens und keiner Erfindung; er ist längst in aller Menschen Vernunft gewesen und ihrem Wesen einverleibt, und ist der Grundsatz der Sittlichkeit. Also ist jene unbedingte Kausalität und das Vermögen derselben, die Freiheit, mit dieser aber ein Wesen (ich selber), welches zur Sinnenwelt gehört, doch zugleich als zur intelligibelen gehörig nicht bloß unbestimmt und problematisch gedacht (welches schon die spekulative Vernunft als tunlich ausmitteln konnte), sondern sogar in Ansehung des Gesetzes ihrer Kausalität bestimmt und assertorisch erkannt, und so uns die Wirklichkeit der intelligibelen Welt, und zwar in praktischer Rücksicht bestimmt, gegeben worden, und diese Bestimmung, die in theoretischer Absicht transzendent (überschwenglich)sein würde, ist in praktischer immanent. Dergleichen Schritt aber konnten wir in Ansehung der zweiten dynamischen Idee, nämlich der eines notwendigen Wesens nicht tun. Wir konnten zu ihm aus der Sinnenwelt, ohne Vermittelung der ersteren dyn. Idee, nicht hinauf kommen. Denn, wollten wir es versuchen, so müßten wir den Sprung gewagt haben, alles das, was uns gegeben ist, zu verlassen, und uns zu dem hinzuschwingen, wovon uns auch nichts gegeben ist, wodurch wir die Verknüpfung eines solchen intelligibelen Wesens mit der Sinnenwelt vermitteln könnten (weil das notwendige Wesen als außer uns gegeben  erkannt werden sollte); welches dagegen in Ansehung unseres eignen Subjekts, so fern es sich durchs moralische Gesetz einerseits als intelligibeles Wesen (vermöge der Freiheit) bestimmt, andererseits als nach dieser Bestimmung in der Sinnenwelt tätig, selbst erkennt, wie jetzt der Augenschein dartut, ganz wohl möglich ist. Der einzige Begriff der Freiheit verstattet es, daß wir nicht außer uns hinausgehen dürfen, um das Unbedingte und Intelligibele zu dem Bedingten und Sinnlichen zu finden. Denn es ist unsere Vernunft selber, die sich durchs höchste und unbedingte praktische Gesetz, und das Wesen, das sich dieses Gesetzes bewußt ist (unsere eigene Person), als zur reinen Verstandeswelt gehörig, und zwar sogar mit Bestimmung der Art, wie es als ein solches tätig sein könne, erkennt. So läßt sich begreifen, warum in dem ganzen Vernunftvermögen nur das Praktische dasjenige sein könne, welches uns über die Sinnenwelt hinaushilft, und Erkenntnisse von einer übersinnlichen Ordnung und Verknüpfung verschaffe, die aber eben darum freilich nur so weit, als es gerade für die reine praktische Absicht nötig ist, ausgedehnt werden können. 
	
	\subsection*{tg225.2.7} 
	\textbf{Source : }Kritik der praktischen Vernunft/Erster Teil. Elementarlehre der reinen praktischen Vernunft/Zweites Buch. Dialektik der reinen praktischen Vernunft/Zweites Hauptstück. Von der Dialektik der reinen Vernunft in Bestimmung des Begriffs vom höchsten Gut/VII. Wie eine Erweiterung der reinen Vernunft, in praktischer Absicht, ohne damit ihr Erkenntnis, als spekulativ, zugleich zu erweitern, zu denken möglich sei\\  
	
	\textbf{Paragraphe : }
	Wenn, nächstdem, diese Ideen von Gott, einer intelligibelen Welt (dem Reiche Gottes) und der Unsterblichkeit durch Prädikate bestimmt werten, die von unserer eigenen Natur hergenommen sind, so darf man diese Bestimmung weder als Versinnlichung jener reinen Vernunftideen (Anthropomorphismen), noch als überschwengliches Erkenntnis übersinnlicher Gegenstände ansehen; denn diese Prädikate sind keine andere als Verstand und Wille, und zwar so im Verhältnisse gegen einander betrachtet, als sie im moralischen Gesetze gedacht werden müssen, also nur, so weit von ihnen ein reiner praktischer Gebrauch gemacht wird. Von allem übrigen, was diesen Begriffen psychologisch anhängt, d.i. so fern wir diese unsere Vermögen in ihrer Ausübung empirisch beobachten (z.B., daß der Verstand des Menschen diskursiv ist, seine Vorstellungen also Gedanken, nicht Anschauungen sind, daß diese in der Zeit auf einander folgen, daß sein Wille immer mit einer Abhängigkeit der Zufriedenheit von der Existenz seines Gegenstandes behaftet ist, u.s.w., welches im höchsten Wesen so nicht sein kann), wird alsdenn abstrahiert, und so bleibt von den Begriffen, durch die wir uns ein reines Verstandeswesen denken, nichts mehr übrig, als gerade zur Möglichkeit erfoderlich ist, sich ein moralisch Gesetz zu denken, mithin zwar ein Erkenntnis Gottes, aber nur in praktischer Beziehung, wodurch, wenn wir den Versuch machen, es zu einem theoretischen zu erweitern, wir einen Verstand desselben bekommen, der nicht denkt, sondern anschaut, einen Willen, der auf Gegenstände gerichtet ist, von deren Existenz seine Zufriedenheit nicht im mindesten abhängt (ich will nicht einmal der transzendentalen Prädikate erwähnen, als z.B. eine \match{Größe} der Existenz, d.i. Dauer, die aber nicht in der Zeit, als dem einzigen uns möglichen Mittel, uns Dasein als Größe vorzustellen, stattfindet), lauter Eigenschaften, von denen wir uns gar keinen Begriff, zum Erkenntnisse des Gegenstandes tauglich, machen können, und dadurch belehrt werden, daß sie niemals zu einer Theorie von übersinnlichen Wesen gebraucht werden können, und also, auf dieser Seite, ein spekulatives Erkenntnis zu  gründen gar nicht vermögen, sondern ihren Gebrauch lediglich auf die Ausübung des moralischen Gesetzes einschränken. 
	
	\subsection*{tg225.2.9} 
	\textbf{Source : }Kritik der praktischen Vernunft/Erster Teil. Elementarlehre der reinen praktischen Vernunft/Zweites Buch. Dialektik der reinen praktischen Vernunft/Zweites Hauptstück. Von der Dialektik der reinen Vernunft in Bestimmung des Begriffs vom höchsten Gut/VII. Wie eine Erweiterung der reinen Vernunft, in praktischer Absicht, ohne damit ihr Erkenntnis, als spekulativ, zugleich zu erweitern, zu denken möglich sei\\  
	
	\textbf{Paragraphe : }Nach diesen Erinnerungen ist nun auch die Beantwortung der wichtigen Frage leicht zu finden: Ob der Begriff von Gott ein zur Physik (mithin auch zur Metaphysik, als die nur die reinen Prinzipien a priori der ersteren in allgemeiner  Bedeutung enthält) oder ein zur Moral gehöriger Begriff sei. Natureinrichtungen, oder deren Veränderung zu erklären, wenn man da zu Gott, als dem Urheber aller Dinge, seine Zuflucht nimmt, ist wenigstens keine physische Erklärung, und überall ein Geständnis, man sei mit seiner Philosophie zu Ende; weil man genötigt ist, etwas, wovon man sonst für sich keinen Begriff hat, anzunehmen, um sich von der Möglichkeit dessen, was man vor Augen sieht, einen Begriff machen zu können. Durch Metaphysik aber von der Kenntnis dieser Welt zum Begriffe von Gott und dem Beweise seiner Existenz durch sichere Schlüsse zu gelangen, ist darum unmöglich, weil wir diese Welt als das vollkommenste mögliche Ganze, mithin, zu diesem Behuf, alle mögliche Welten (um sie mit dieser vergleichen zu können) erkennen, mithin allwissend sein müßten, um zu sagen, daß sie nur durch einen Gott (wie wir uns diesen Begriff denken müssen) möglich war. Vollends aber die Existenz dieses Wesens aus bloßen Begriffen zu erkennen, ist schlechterdings unmöglich, weil ein jeder Existentialsatz, d.i. der, so von einem Wesen, von dem ich mir einen Begriff mache, sagt, daß es existiere, ein synthetischer Satz ist, d.i. ein solcher, dadurch ich über jenen Begriff hinausgehe und mehr von ihm sage, als im Begriffe gedacht war: nämlich daß diesem Begriffe im Verstande noch ein Gegenstand außer dem Verstande korrespondierend gesetzt sei, welches offenbar unmöglich ist durch irgend einen Schluß herauszubringen. Also bleibt nur ein einziges Verfahren für die Vernunft übrig, zu diesem Erkenntnisse zu gelangen, da sie nämlich, als reine Vernunft, von dem obersten Prinzip ihres reinen praktischen Gebrauchs ausgehend (indem dieser ohnedem bloß auf die Existenz von etwas, als Folge der Vernunft, gerichtet ist), ihr Objekt bestimmt. Und da zeigt sich, nicht allein in ihrer unvermeidlichen Aufgabe, nämlich der notwendigen Richtung des Willens auf das höchste Gut, die Notwendigkeit, ein solches Urwesen, in Beziehung auf die Möglichkeit dieses Guten in der Welt, anzunehmen, sondern, was das Merkwürdigste ist, etwas, was dem Fortgange der Vernunft auf dem Naturwege ganz  mangelte, nämlich ein genau bestimmter Begriff dieses Urwesens. Da wir diese Welt nur zu einem kleinen Teile kennen, noch weniger sie mit allen möglichen Welten vergleichen können, so können wir von ihrer Ordnung, Zweckmäßigkeit und \match{Größe} wohl auf einen weisen, gütigen, mächtigen etc. Urheber derselben schließen, aber nicht auf seine Allwissenheit, Allgütigkeit, Allmacht, u.s.w. Man kann auch gar wohl einräumen: daß man diesen unvermeidlichen Mangel durch eine erlaubte ganz vernünftige Hypothese zu ergänzen wohl befugt sei; daß nämlich, wenn in so viel Stücken, als sich unserer näheren Kenntnis darbieten, Weisheit, Gütigkeit etc. hervorleuchtet, in allen übrigen es eben so sein werde, und es also vernünftig sei, dem Welturheber alle mögliche Vollkommenheit beizulegen; aber das sind keine Schlüsse, wodurch wir uns auf unsere Einsicht etwas dünken, sondern nur Befugnisse, die man uns nachsehen kann, und doch noch einer anderweitigen Empfehlung bedürfen, um davon Gebrauch zu machen. Der Begriff von Gott bleibt also auf dem empirischen Wege (der Physik) immer ein nicht genau bestimmter Begriff von der Vollkommenheit des ersten Wesens, um ihn dem Begriffe einer Gottheit für angemessen zu halten (mit der Metaphysik aber in ihrem transzendentalen Teile ist gar nichts auszurichten). 
	
	\unnumberedsection{Maß (2)} 
	\subsection*{tg227.2.2} 
	\textbf{Source : }Kritik der praktischen Vernunft/Erster Teil. Elementarlehre der reinen praktischen Vernunft/Zweites Buch. Dialektik der reinen praktischen Vernunft/Zweites Hauptstück. Von der Dialektik der reinen Vernunft in Bestimmung des Begriffs vom höchsten Gut/IX. Von der der praktischen Bestimmung des Menschen weislich angemessenen Proportion seiner Erkenntnisvermögen\\  
	
	\textbf{Paragraphe : }Wenn die menschliche Natur zum höchsten Gute zu streben bestimmt ist, so muß auch das \match{Maß} ihrer Erkenntnisvermögen, vornehmlich ihr Verhältnis unter einander, als zu diesem Zwecke schicklich, angenommen werden. Nun beweiset aber die Kritik der reinen spekulativen Vernunft die größte Unzulänglichkeit derselben, um die wichtigsten Aufgaben, die ihr vorgelegt werden, dem Zwecke angemessen aufzulösen, ob sie zwar die natürlichen und nicht zu übersehenden Winke eben derselben Vernunft, imgleichen die großen Schritte, die sie tun kann, nicht verkennt, um sich diesem großen Ziele, das ihr ausgesteckt ist, zu näheren, aber doch, ohne es jemals für sich selbst, sogar mit Beihülfe der größten Naturkenntnis, zu erreichen. Also scheint die Natur hier uns nur stiefmütterlich mit einem zu unserem Zwecke benötigten Vermögen versorgt zu haben. 
	
	\subsection*{tg228.2.8} 
	\textbf{Source : }Kritik der praktischen Vernunft/Zweiter Teil. Methodenlehre der reinen praktischen Vernunft\\  
	
	\textbf{Paragraphe : }
	Wenn man aber trägt: was denn eigentlich die reine Sittlichkeit ist, an der, als dem Probemetall, man jeder Handlung moralischen Gehalt prüfen müsse, so muß ich gestehen, daß nur Philosophen die Entscheidung dieser Frage zweifelhaft machen können; denn in der gemeinen Menschenvernunft ist sie, zwar nicht durch abgezogene allgemeine Formeln, aber doch durch den gewöhnlichen Gebrauch, gleichsam als der Unterschied zwischen der rechten und linken Hand, längst entschieden. Wir wollen also vorerst das Prüfungsmerkmal der reinen Tugend an einem Beispiele zeigen, und, indem wir uns vorstellen, daß es etwa einem zehnjährigen Knaben zur Beurteilung vorgelegt worden, sehen, ob er auch von selber, ohne durch den Lehrer dazu angewiesen zu sein, notwendig so urteilen müßte. Man erzähle die Geschichte eines redlichen Mannes, den man bewegen will, den Verleumdern einer unschuldigen, übrigens nicht vermögenden Person (wie etwa Anna von Boleyn auf Anklage Heinrichs VIII. von England) beizutreten. Man bietet Gewinne, d.i. große Geschenke oder hohen Rang an, er schlägt sie aus. Dieses wird bloßen Beifall und Billigung in der Seele des Zuhörers wirken, weil es Gewinn ist. Nun fängt man es mit Androhung des Verlusts an. Es sind unter diesen Verleumdern seine besten Freunde, die ihm jetzt ihre Freundschaft aufsagen, nahe Verwandte, die ihn (der ohne Vermögen ist) zu enterben drohen, Mächtige, die ihn in jedem Orte und Zustande verfolgen und kränken können, ein Landesfürst, der ihn mit dem Verlust der Freiheit, ja des Lebens selbst bedroht. Um ihn aber, damit das \match{Maß} des Leidens voll sei, auch den Schmerz fühlen zu lassen, den nur  das sittlich gute Herz recht inniglich fühlen kann, mag man seine mit äußerster Not und Dürftigkeit bedrohete Familie ihn um Nachgiebigkeit anflehend, ihn selbst, obzwar rechtschaffen, doch eben nicht von festen unempfindlichen Organen des Gefühls, für Mitleid sowohl als eigener Not, in einem Augenblick, darin er wünscht, den Tag nie erlebt zu haben, der ihn einem so unaussprechlichen Schmerz aussetzte, dennoch seinem Vorsatze der Redlichkeit, ohne zu wanken oder nur zu zweifeln, treu bleibend, vorstellen: so wird mein jugendlicher Zuhörer stufenweise, von der bloßen Billigung zur Bewunderung, von da zum Erstaunen, endlich bis zur größten Verehrung, und einem lebhaften Wunsche, selbst ein solcher Mann sein zu können (obzwar freilich nicht in seinem Zustande), erhoben werden; und gleichwohl ist hier die Tugend nur darum so viel wert, weil sie so viel kostet, nicht weil sie etwas einbringt. Die ganze Bewunderung und selbst Bestrebung zur Ähnlichkeit mit diesem Charakter beruht hier gänzlich auf der Reinigkeit des sittlichen Grundsatzes, welche nur dadurch recht in die Augen fallend vorgestellet werden kann, daß man alles, was Menschen nur zur Glückseligkeit zählen mögen, von den Triebfedern der Handlung wegnimmt. Also muß die Sittlichkeit auf das menschliche Herz desto mehr Kraft haben, je reiner sie dargestellt wird. Woraus denn folgt, daß, wenn das Gesetz der Sitten und das Bild der Heiligkeit und Tugend auf unsere Seele überall einigen Einfluß ausüben soll, sie diesen nur so fern ausüben könne, als sie rein, unvermengt von Absichten auf sein Wohlbefinden, als Triebfeder ans Herz gelegt wird, darum weil sie sich im Leiden am herrlichsten zeigt. Dasjenige aber, dessen Wegräumung die Wirkung einer bewegenden Kraft verstärkt, muß ein Hindernis gewesen sein. Folglich ist alle Beimischung der Triebfedern, die von eigener Glückseligkeit hergenommen werden, ein Hindernis, dem moralischen Gesetze Einfluß aufs menschliche Herz zu verschaffen. – Ich behaupte ferner, daß selbst in jener bewunderten Handlung, wenn der Bewegungsgrund, daraus sie geschah, die Hochschätzung seiner Pflicht war, alsdenn eben diese Achtung fürs Gesetz, nicht etwa ein Anspruch  auf die innere Meinung von Großmut und edler verdienstlicher Denkungsart, gerade auf das Gemüt des Zuschauers die größte Kraft habe, folglich Pflicht, nicht Verdienst, den nicht allein bestimmtesten, sondern, wenn sie im rechten Lichte ihrer Unverletzlichkeit vorgestellt wird, auch den eindringendsten Einfluß aufs Gemüt haben müsse. 
	
	\unnumberedsection{Änderung (2)} 
	\subsection*{tg215.2.11} 
	\textbf{Source : }Kritik der praktischen Vernunft/Erster Teil. Elementarlehre der reinen praktischen Vernunft/Erstes Buch. Die Analytik der reinen praktischen Vernunft/Drittes Hauptstück. Von den Triebfedern der reinen praktischen Vernunft/Kritische Beleuchtung der Analytik der reinen praktischen Vernunft\\  
	
	\textbf{Paragraphe : }Wenn ich von einem Menschen, der einen Diebstahl verübt, sage: diese Tat sei nach dem Naturgesetze der Kausalität aus den Bestimmungsgründen der vorhergehenden Zeit ein notwendiger Erfolg, so war es unmöglich, daß sie hat unterbleiben können; wie kann denn die Beurteilung nach dem moralischen Gesetze hierin eine \match{Änderung} machen, und voraussetzen, daß sie doch habe unterlassen werden können, weil das Gesetz sagt, sie hätte unterlassen werden sollen, d.i. wie kann derjenige, in demselben Zeitpunkte, in Absicht auf dieselbe Handlung, ganz frei heißen, in welchem, und in  derselben Absicht, er doch unter einer unvermeidlichen Naturnotwendigkeit steht? Eine Ausflucht darin suchen, daß man bloß die Art der Bestimmungsgründe seiner Kausalität nach dem Naturgesetze einem komparativen Begriffe von Freiheit anpaßt (nach welchem das bisweilen freie Wirkung heißt, davon der bestimmende Naturgrund innerlich im wirkenden Wesen liegt, z.B. das, was ein geworfener Körper verrichtet, wenn er in freier Bewegung ist, da man das Wort Freiheit braucht, weil er, während daß er im Fluge ist, nicht von außen wodurch getrieben wird, oder wie wir die Bewegung einer Uhr auch eine freie Bewegung nennen, weil sie ihren Zeiger selbst treibt, der also nicht äußerlich geschoben werden darf, eben so die Handlungen des Menschen, ob sie gleich, durch ihre Bestimmungsgründe, die in der Zeit vorhergehen, notwendig sind, dennoch frei nennen, weil es doch innere durch unsere eigene Kräfte hervorgebrachte Vorstellungen, dadurch nach veranlassenden Umständen erzeugte Begierden und mithin nach unserem eigenen Belieben bewirkte Handlungen sind), ist ein elender Behelf, womit sich noch immer einige hinhalten lassen, und so jenes schwere Problem mit einer kleinen Wortklauberei aufgelöset zu haben meinen, an dessen Auflösung Jahrtausende vergeblich gearbeitet haben, die daher wohl schwerlich so ganz auf der Oberfläche gefunden werden dürfte. Es kommt nämlich bei der Frage nach derjenigen Freiheit, die allen moralischen Gesetzen und der ihnen gemäßen Zurechnung zum Grunde gelegt werden muß, darauf gar nicht an, ob die nach einem Naturgesetze bestimmte Kausalität durch Bestimmungsgründe, die im Subjekte, oder außer ihm liegen, und im ersteren Fall, ob sie durch Instinkt oder mit Vernunft gedachte Bestimmungsgründe notwendig sei; wenn diese bestimmende Vorstellungen, nach dem Geständnisse eben dieser Männer selbst, den Grund ihrer Existenz doch in der Zeit und zwar dem vorigen Zustande haben, dieser aber wieder in einem vorhergehenden etc., so mögen sie, diese Bestimmungen, immer innerlich sein, sie mögen psychologische und nicht mechanische Kausalität haben, d.i. durch Vorstellungen, und nicht durch körperliche Bewegung,  Handlung hervorbringen, so sind es immer Bestimmungsgründe der Kausalität eines Wesens, so fern sein Dasein in der Zeit bestimmbar ist, mithin unter notwendig machenden Bedingungen der vergangenen Zeit, die also, wenn das Subjekt handeln soll, nicht mehr in seiner Gewalt sind, die also zwar psychologische Freiheit(wenn man ja dieses Wort von einer bloß inneren Verkettung der Vorstellungen der Seele brauchen will), aber doch Naturnotwendigkeit bei sich führen, mithin keine transzendentale Freiheit übrig lassen, welche als Unabhängigkeit von allem Empirischen und also von der Natur überhaupt gedacht werden muß, sie mag nun Gegenstand des inneren Sinnes, bloß in der Zeit, oder auch äußeren Sinne, im Raume und der Zeit zugleich betrachtet werden, ohne welche Freiheit (in der letzteren eigentlichen Bedeutung), die allein a priori praktisch ist, kein moralisch Gesetz, keine Zurechnung nach demselben, möglich ist. Eben um deswillen kann man auch alle Notwendigkeit der Begebenheiten in der Zeit, nach dem Naturgesetze der Kausalität, den Mechanismus der Natur nennen, ob man gleich darunter nicht versteht, daß Dinge, die ihm unterworfen sind, wirkliche materielle Maschinen sein müßten. Hier wird nur auf die Notwendigkeit der Verknüpfung der Begebenheiten in einer Zeitreihe, so wie sie sich nach dem Naturgesetze entwickelt, gesehen, man mag nun das Subjekt, in welchem dieser Ablauf geschieht, automaton materiale, da das Maschinenwesen durch Materie, oder mit Leibnizen spirituale, da es durch Vorstellungen betrieben wird, nennen, und wenn die Freiheit unseres Willens keine andere als die letztere (etwa die psychologische und komparative, nicht transzendentale, d.i. absolute zugleich) wäre, so würde sie im Grunde nichts besser, als die Freiheit eines Bratenwenders sein, der auch, wenn er einmal aufgezogen worden, von selbst seine Bewegungen verrichtet. 
	
	\subsection*{tg215.2.16} 
	\textbf{Source : }Kritik der praktischen Vernunft/Erster Teil. Elementarlehre der reinen praktischen Vernunft/Erstes Buch. Die Analytik der reinen praktischen Vernunft/Drittes Hauptstück. Von den Triebfedern der reinen praktischen Vernunft/Kritische Beleuchtung der Analytik der reinen praktischen Vernunft\\  
	
	\textbf{Paragraphe : }Die Auflösung obgedachter Schwierigkeit geschieht, kurz und einleuchtend, auf folgende Art: Wenn die Existenz in der Zeit eine bloße sinnliche Vorstellungsart der denkenden Wesen in der Welt ist, folglich sie, als Dinge an sich selbst, nicht angeht: so ist die Schöpfung dieser Wesen eine Schöpfung der Dinge an sich selbst; weil der Begriff einer Schöpfung nicht zu der sinnlichen Vorstellungsart der Existenz und zur Kausalität gehört, sondern nur auf Noumenen bezogen werden kann. Folglich, wenn ich von Wesen in der  Sinnenwelt sage: sie sind erschaffen; so betrachte ich sie so fern als Noumenen. So, wie es also ein Widerspruch wäre, zu sagen, Gott sei ein Schöpfer von Erscheinungen, so ist es auch ein Widerspruch, zu sagen, er sei, als Schöpfer, Ursache der Handlungen in der Sinnenwelt, mithin als Erscheinungen, wenn er gleich Ursache des Daseins der handelnden Wesen (als Noumenen) ist. Ist es nun möglich (wenn wir nur das Dasein in der Zeit für etwas, was bloß von Erscheinungen, nicht von Dingen an sich selbst gilt, annehmen), die Freiheit, unbeschadet dem Naturmechanism der Handlungen als Erscheinungen, zu behaupten, so kann, daß die handelnden Wesen Geschöpfe sind, nicht die mindeste \match{Änderung} hierin machen, weil die Schöpfung ihre intelligibele, aber nicht sensibele Existenz betrifft, und also nicht als Bestimmungsgrund der Erscheinungen angesehen werden kann; welches aber ganz anders ausfallen würde, wenn die Weltwesen als Dinge an sich selbst in der Zeit existierten, da der Schöpfer der Substanz zugleich der Urheber des ganzen Maschinenwesens an dieser Substanz sein würde. 
	
	\unnumberedchapter{Zoologie} 
	\unnumberedsection{Familie (1)} 
	\subsection*{tg228.2.8} 
	\textbf{Source : }Kritik der praktischen Vernunft/Zweiter Teil. Methodenlehre der reinen praktischen Vernunft\\  
	
	\textbf{Paragraphe : }
	Wenn man aber trägt: was denn eigentlich die reine Sittlichkeit ist, an der, als dem Probemetall, man jeder Handlung moralischen Gehalt prüfen müsse, so muß ich gestehen, daß nur Philosophen die Entscheidung dieser Frage zweifelhaft machen können; denn in der gemeinen Menschenvernunft ist sie, zwar nicht durch abgezogene allgemeine Formeln, aber doch durch den gewöhnlichen Gebrauch, gleichsam als der Unterschied zwischen der rechten und linken Hand, längst entschieden. Wir wollen also vorerst das Prüfungsmerkmal der reinen Tugend an einem Beispiele zeigen, und, indem wir uns vorstellen, daß es etwa einem zehnjährigen Knaben zur Beurteilung vorgelegt worden, sehen, ob er auch von selber, ohne durch den Lehrer dazu angewiesen zu sein, notwendig so urteilen müßte. Man erzähle die Geschichte eines redlichen Mannes, den man bewegen will, den Verleumdern einer unschuldigen, übrigens nicht vermögenden Person (wie etwa Anna von Boleyn auf Anklage Heinrichs VIII. von England) beizutreten. Man bietet Gewinne, d.i. große Geschenke oder hohen Rang an, er schlägt sie aus. Dieses wird bloßen Beifall und Billigung in der Seele des Zuhörers wirken, weil es Gewinn ist. Nun fängt man es mit Androhung des Verlusts an. Es sind unter diesen Verleumdern seine besten Freunde, die ihm jetzt ihre Freundschaft aufsagen, nahe Verwandte, die ihn (der ohne Vermögen ist) zu enterben drohen, Mächtige, die ihn in jedem Orte und Zustande verfolgen und kränken können, ein Landesfürst, der ihn mit dem Verlust der Freiheit, ja des Lebens selbst bedroht. Um ihn aber, damit das Maß des Leidens voll sei, auch den Schmerz fühlen zu lassen, den nur  das sittlich gute Herz recht inniglich fühlen kann, mag man seine mit äußerster Not und Dürftigkeit bedrohete \match{Familie} ihn um Nachgiebigkeit anflehend, ihn selbst, obzwar rechtschaffen, doch eben nicht von festen unempfindlichen Organen des Gefühls, für Mitleid sowohl als eigener Not, in einem Augenblick, darin er wünscht, den Tag nie erlebt zu haben, der ihn einem so unaussprechlichen Schmerz aussetzte, dennoch seinem Vorsatze der Redlichkeit, ohne zu wanken oder nur zu zweifeln, treu bleibend, vorstellen: so wird mein jugendlicher Zuhörer stufenweise, von der bloßen Billigung zur Bewunderung, von da zum Erstaunen, endlich bis zur größten Verehrung, und einem lebhaften Wunsche, selbst ein solcher Mann sein zu können (obzwar freilich nicht in seinem Zustande), erhoben werden; und gleichwohl ist hier die Tugend nur darum so viel wert, weil sie so viel kostet, nicht weil sie etwas einbringt. Die ganze Bewunderung und selbst Bestrebung zur Ähnlichkeit mit diesem Charakter beruht hier gänzlich auf der Reinigkeit des sittlichen Grundsatzes, welche nur dadurch recht in die Augen fallend vorgestellet werden kann, daß man alles, was Menschen nur zur Glückseligkeit zählen mögen, von den Triebfedern der Handlung wegnimmt. Also muß die Sittlichkeit auf das menschliche Herz desto mehr Kraft haben, je reiner sie dargestellt wird. Woraus denn folgt, daß, wenn das Gesetz der Sitten und das Bild der Heiligkeit und Tugend auf unsere Seele überall einigen Einfluß ausüben soll, sie diesen nur so fern ausüben könne, als sie rein, unvermengt von Absichten auf sein Wohlbefinden, als Triebfeder ans Herz gelegt wird, darum weil sie sich im Leiden am herrlichsten zeigt. Dasjenige aber, dessen Wegräumung die Wirkung einer bewegenden Kraft verstärkt, muß ein Hindernis gewesen sein. Folglich ist alle Beimischung der Triebfedern, die von eigener Glückseligkeit hergenommen werden, ein Hindernis, dem moralischen Gesetze Einfluß aufs menschliche Herz zu verschaffen. – Ich behaupte ferner, daß selbst in jener bewunderten Handlung, wenn der Bewegungsgrund, daraus sie geschah, die Hochschätzung seiner Pflicht war, alsdenn eben diese Achtung fürs Gesetz, nicht etwa ein Anspruch  auf die innere Meinung von Großmut und edler verdienstlicher Denkungsart, gerade auf das Gemüt des Zuschauers die größte Kraft habe, folglich Pflicht, nicht Verdienst, den nicht allein bestimmtesten, sondern, wenn sie im rechten Lichte ihrer Unverletzlichkeit vorgestellt wird, auch den eindringendsten Einfluß aufs Gemüt haben müsse. 
	
	\unnumberedsection{Fruchtbarkeit (1)} 
	\subsection*{tg215.2.22} 
	\textbf{Source : }Kritik der praktischen Vernunft/Erster Teil. Elementarlehre der reinen praktischen Vernunft/Erstes Buch. Die Analytik der reinen praktischen Vernunft/Drittes Hauptstück. Von den Triebfedern der reinen praktischen Vernunft/Kritische Beleuchtung der Analytik der reinen praktischen Vernunft\\  
	
	\textbf{Paragraphe : }Da es eigentlich der Begriff der Freiheit ist, der, unter allen Ideen der reinen spekulativen Vernunft, allein so große Erweiterung im Felde des Übersinnlichen, wenn gleich nur in Ansehung des praktischen Erkenntnisses verschafft, so frage ich mich: woher denn ihm ausschließungsweise eine so große \match{Fruchtbarkeit} zu Teil geworden sei, indessen die übrigen zwar die leere Stelle für reine mögliche Verstandeswesen bezeichnen, den Begriff von ihnen aber durch nichts bestimmen können. Ich begreife bald, daß, da ich nichts ohne Kategorie denken kann, diese auch in der Idee der Vernunft von der Freiheit, mit der ich mich beschäftige, zuerst müsse aufgesucht werden, welche hier die Kategorie der Kausalität ist, und daß ich, wenn gleich dem Vernunftbegriffe der Freiheit, als überschwenglichem Begriffe, keine korrespondierende Anschauung untergelegt werden kann, dennoch dem Verstandesbegriffe (der Kausalität), für dessen Synthesis jener das Unbedingte fodert, zuvor eine sinnliche Anschauung gegeben werden müsse, dadurch ihm zuerst die objektive Realität gesichert wird. Nun sind alle Kategorien in zwei Klassen, die mathematische, welche bloß auf die Einheit der Synthesis in der Vorstellung der Objekte, und die dynamische, welche auf die in der Vorstellung der Existenz der Objekte gehen, eingeteilt. Die erstere (die der Größe und der Qualität) enthalten jederzeit eine Synthesis des Gleichartigen, in welcher das Unbedingte, zu dem in der sinnlichen Anschauung gegebenen Bedingten in Raum und Zeit, da es selbst wiederum zum Raume und der Zeit gehören, und also  immer wieder unbedingt sein mußte, gar nicht kann gefunden werden; daher auch in der Dialektik der reinen theoretischen Vernunft die einander entgegengesetzte Arten, das Unbedingte und die Totalität der Bedingungen für sie zu finden, beide falsch waren. Die Kategorien der zweiten Klasse (die der Kausalität und der Notwendigkeit eines Dinges) erforderten diese Gleichartigkeit (des Bedingten und der Bedingung in der Synthesis) gar nicht, weil hier nicht die Anschauung, wie sie aus einem Mannigfaltigen in ihr zusammengesetzt, sondern nur, wie die Existenz des ihr korrespondierenden bedingten Gegenstandes zu der Existenz der Bedingung (im Verstande als damit verknüpft) hinzukomme, vorgestellt werden solle, und da war es erlaubt, zu dem durchgängig Bedingten in der Sinnenwelt (so wohl in Ansehung der Kausalität als des zufälligen Daseins der Dinge selbst) das Unbedingte, obzwar übrigens unbestimmt, in der intelligibelen Welt zu setzen, und die Synthesis transzendent zu machen; daher denn auch in der Dialektik der r. spek. V. sich fand, daß beide, dem Scheine nach, einander entgegengesetzte Arten, das Unbedingte zum Bedingten zu finden, z.B. in der Synthesis der Kausalität zum Bedingten, in der Reihe der Ursachen und Wirkungen der Sinnenwelt, die Kausalität, die weiter nicht sinnlich bedingt ist, zu denken, sich in der Tat nicht widerspreche, und daß dieselbe Handlung, die, als zur Sinnenwelt gehörig, jederzeit sinnlich bedingt, d.i. mechanisch-notwendig ist, doch zugleich auch, als zur Kausalität des handelnden Wesens, so fern es zur intelligibelen Welt gehörig ist, eine sinnlich unbedingte Kausalität zum Grunde haben, mithin als frei gedacht werden könne. Nun kam es bloß darauf an, daß dieses Können in ein Sein verwandelt würde, d.i., daß man in einem wirklichen Falle, gleichsam durch ein Faktum, beweisen könne: daß gewisse Handlungen eine solche Kausalität (die intellektuelle, sinnlich unbedingte) voraussetzen, sie mögen nun wirklich, oder auch nur geboten, d.i. objektiv praktisch notwendig sein. An wirklich in der Erfahrung  gegebenen Handlungen, als Begebenheiten der Sinnenwelt, konnten wir diese Verknüpfung nicht anzutreffen hoffen, weil die Kausalität durch Freiheit immer außer der Sinnenwelt im Intelligibelen gesucht werden muß. Andere Dinge, außer den Sinnenwesen, sind uns aber zur Wahrnehmung und Beobachtung nicht gegeben. Also blieb nichts übrig, als daß etwa ein unwidersprechlicher und zwar objektiver Grundsatz der Kausalität, welcher alle sinnliche Bedingung von ihrer Bestimmung ausschließt, d.i. ein Grundsatz, in welchem die Vernunft sich nicht weiter auf etwas anderes als Bestimmungsgrund in Ansehung der Kausalität beruft, sondern den sie durch jenen Grundsatz schon selbst enthält, und wo sie also, als reine Vernunft, selbst praktisch ist, gefunden werde. Dieser Grundsatz aber bedarf keines Suchens und keiner Erfindung; er ist längst in aller Menschen Vernunft gewesen und ihrem Wesen einverleibt, und ist der Grundsatz der Sittlichkeit. Also ist jene unbedingte Kausalität und das Vermögen derselben, die Freiheit, mit dieser aber ein Wesen (ich selber), welches zur Sinnenwelt gehört, doch zugleich als zur intelligibelen gehörig nicht bloß unbestimmt und problematisch gedacht (welches schon die spekulative Vernunft als tunlich ausmitteln konnte), sondern sogar in Ansehung des Gesetzes ihrer Kausalität bestimmt und assertorisch erkannt, und so uns die Wirklichkeit der intelligibelen Welt, und zwar in praktischer Rücksicht bestimmt, gegeben worden, und diese Bestimmung, die in theoretischer Absicht transzendent (überschwenglich)sein würde, ist in praktischer immanent. Dergleichen Schritt aber konnten wir in Ansehung der zweiten dynamischen Idee, nämlich der eines notwendigen Wesens nicht tun. Wir konnten zu ihm aus der Sinnenwelt, ohne Vermittelung der ersteren dyn. Idee, nicht hinauf kommen. Denn, wollten wir es versuchen, so müßten wir den Sprung gewagt haben, alles das, was uns gegeben ist, zu verlassen, und uns zu dem hinzuschwingen, wovon uns auch nichts gegeben ist, wodurch wir die Verknüpfung eines solchen intelligibelen Wesens mit der Sinnenwelt vermitteln könnten (weil das notwendige Wesen als außer uns gegeben  erkannt werden sollte); welches dagegen in Ansehung unseres eignen Subjekts, so fern es sich durchs moralische Gesetz einerseits als intelligibeles Wesen (vermöge der Freiheit) bestimmt, andererseits als nach dieser Bestimmung in der Sinnenwelt tätig, selbst erkennt, wie jetzt der Augenschein dartut, ganz wohl möglich ist. Der einzige Begriff der Freiheit verstattet es, daß wir nicht außer uns hinausgehen dürfen, um das Unbedingte und Intelligibele zu dem Bedingten und Sinnlichen zu finden. Denn es ist unsere Vernunft selber, die sich durchs höchste und unbedingte praktische Gesetz, und das Wesen, das sich dieses Gesetzes bewußt ist (unsere eigene Person), als zur reinen Verstandeswelt gehörig, und zwar sogar mit Bestimmung der Art, wie es als ein solches tätig sein könne, erkennt. So läßt sich begreifen, warum in dem ganzen Vernunftvermögen nur das Praktische dasjenige sein könne, welches uns über die Sinnenwelt hinaushilft, und Erkenntnisse von einer übersinnlichen Ordnung und Verknüpfung verschaffe, die aber eben darum freilich nur so weit, als es gerade für die reine praktische Absicht nötig ist, ausgedehnt werden können. 
	
	\unnumberedsection{Gattung (1)} 
	\subsection*{tg212.2.51} 
	\textbf{Source : }Kritik der praktischen Vernunft/Erster Teil. Elementarlehre der reinen praktischen Vernunft/Erstes Buch. Die Analytik der reinen praktischen Vernunft\\  
	
	\textbf{Paragraphe : }
	Ich füge hier nichts weiter zur Erläuterung gegenwärtiger Tafel bei, weil sie für sich verständlich genug ist. Dergleichen nach Prinzipien abgefaßte Einteilung ist aller Wissenschaft, ihrer Gründlichkeit sowohl als Verständlichkeit halber, sehr zuträglich. So weiß man, z.B., aus obiger Tafel und der ersten Nummer derselben sogleich, wovon man in praktischen Erwägungen anfangen müsse: von den Maximen, die jeder auf seine Neigung gründet, den Vorschriften, die für eine \match{Gattung} vernünftiger Wesen, so fern sie in gewissen Neigungen übereinkommen, gelten, und endlich dem Gesetze, welches für alle, unangesehen ihrer Neigungen, gilt, u.s.w. Auf diese Weise übersieht man den ganzen Plan, von dem, was man zu leisten hat, so gar jede Frage der praktischen Philosophie, die zu beantworten, und zugleich die Ordnung, die zu befolgen ist. 
	
	\unnumberedsection{Große (2)} 
	\subsection*{tg228.2.7} 
	\textbf{Source : }Kritik der praktischen Vernunft/Zweiter Teil. Methodenlehre der reinen praktischen Vernunft\\  
	
	\textbf{Paragraphe : }Ich weiß nicht, warum die Erzieher der Jugend von diesem Hange der Vernunft, in aufgeworfenen praktischen Fragen selbst die subtilste Prüfung mit Vergnügen einzuschlagen, nicht schon längst Gebrauch gemacht haben, und, nachdem sie einen bloß moralischen Katechism zum Grunde legten, sie nicht die Biographien alter und neuer Zeiten in  der Absicht durchsuchten, um Belege zu den vorgelegten Pflichten bei der Hand zu haben, an denen sie, vornehmlich durch die Vergleichung ähnlicher Handlungen unter verschiedenen Umständen, die Beurteilung ihrer Zöglinge in Tätigkeit setzten, um den mindern oder größeren moralischen Gehalt derselben zu bemerken, als worin sie selbst die frühe Jugend, die zu aller Spekulation sonst noch unreif ist, bald sehr scharfsichtig, und dabei, weil sie den Fortschritt ihrer Urteilskraft fühlt, nicht wenig interessiert finden werden, was aber das Vornehmste ist, mit Sicherheit hoffen können, daß die öftere Übung, das Wohlverhalten in seiner ganzen Reinigkeit zu kennen und ihm Beifall zu geben, dagegen selbst die kleinste Abweichung von ihr mit Bedauern oder Verachtung zu bemerken, ob es zwar bis dahin nur ein Spiel der Urteilskraft, in welchem Kinder mit einander wetteifern können, getrieben wird, dennoch einen dauerhaften Eindruck der Hochschätzung auf der einen und des Abscheues auf der andern Seite zurücklassen werde, welche, durch bloße Gewohnheit, solche Handlungen als beifalls-oder tadelswürdig öfters anzusehen, zur Rechtschaffenheit im künftigen Lebenswandel eine gute Grundlage ausmachen würden. Nur wünsche ich sie mit Beispielen sogenannter edler (überverdienstlicher) Handlungen, mit welchen unsere empfindsame Schriften so viel um sich werfen, zu verschonen, und alles bloß auf Pflicht und den Wert, den ein Mensch sich in seinen eigenen Augen durch das Bewußtsein, sie nicht übertreten zu haben, geben kann und muß, auszusetzen, weil, was auf leere Wünsche und Sehnsuchten nach unersteiglicher Vollkommenheit hinausläuft, lauter Romanhelden hervorbringt, die, indem sie sich auf ihr Gefühl für das überschwenglich-\match{Große} viel zu Gute tun, sich dafür von der Beobachtung der gemeinen und gangbaren Schuldigkeit, die alsdenn ihnen nur unbedeutend klein scheint, frei sprechen.
	
	
	18
	
	
	
	\subsection*{tg229.2.2} 
	\textbf{Source : }Kritik der praktischen Vernunft/Beschluß\\  
	
	\textbf{Paragraphe : }Zwei Dinge erfüllen das Gemüt mit immer neuer und zunehmenden Bewunderung und Ehrfurcht, je öfter und anhaltender sich das Nachdenken damit beschäftigt: Der bestirnte Himmel über mir, und das moralische Gesetz in mir. Beide darf ich nicht als in Dunkelheiten verhüllt, oder im Überschwenglichen, außer meinem Gesichtskreise, suchen und bloß vermuten; ich sehe sie vor mir und verknüpfe sie unmittelbar mit dem Bewußtsein meiner Existenz. Das erste fängt von dem Platze an, den ich in der äußern Sinnenwelt einnehme, und erweitert die Verknüpfung, darin ich stehe, ins unabsehlich-\match{Große} mit Welten über Welten und Systemen von Systemen, überdem noch in grenzenlose Zeiten ihrer periodischen Bewegung, deren Anfang und Fortdauer. Das zweite fängt von meinem unsichtbaren Selbst, meiner Persönlichkeit, an, und stellt mich in einer Welt dar, die wahre Unendlichkeit hat, aber nur dem Verstande spürbar ist, und mit welcher (dadurch aber auch zugleich mit allen jenen sichtbaren Welten) ich mich nicht, wie dort, in bloß zufälliger, sondern allgemeiner und notwendiger Verknüpfung erkenne. Der erstere Anblick einer zahllosen Weltenmenge vernichtet gleichsam meine Wichtigkeit, als eines tierischen Geschöpfs, das die Materie, daraus es ward, dem Planeten (einem bloßen Punkt im Weltall) wieder zurückgeben muß, nachdem es eine kurze Zeit (man weiß nicht wie) mit Lebenskraft versehen gewesen. Der zweite erhebt dagegen meinen Wert, als einer Intelligenz, unendlich, durch meine Persönlichkeit, in welcher das moralische Gesetz mir ein von der Tierheit und selbst von der ganzen Sinnenwelt unabhängiges Leben offenbart, wenigstens so viel sich aus der zweckmäßigen Bestimmung meines Daseins durch dieses Gesetz, welche nicht auf Bedingungen und Grenzen dieses Lebens eingeschränkt ist, sondern ins Unendliche geht, abnehmen läßt. 
	
	\unnumberedsection{Lauf (1)} 
	\subsection*{tg230.2.8} 
	\textbf{Source : }Kritik der praktischen Vernunft/Fußnoten\\  
	
	\textbf{Paragraphe : }
	
	4 Man könnte mir noch den Einwurf machen, warum ich nicht auch den Begriff des Begehrungsvermögens, oder des Gefühls der Lust vorher erklärt habe; obgleich dieser Vorwurf unbillig sein würde, weil man diese Erklärung, als in der Psychologie gegeben, billig sollte voraussetzen können. Es könnte aber freilich die Definition daselbst so eingerichtet sein, daß das Gefühl der Lust der Bestimmung des Begehrungsvermögens zum Grunde gelegt würde (wie es auch wirklich gemeinhin so zu geschehen pflegt), dadurch aber das oberste Prinzip der praktischen Philosophie notwendig empirisch ausfallen müßte, welches doch allererst auszumachen ist, und in dieser Kritik gänzlich widerlegt wird. Daher will ich diese Erklärung hier so geben, wie sie sein muß, um diesen streitigen Punkt, wie billig, im Anfange unentschieden zu lassen. – Leben ist das Vermögen eines Wesens, nach Gesetzen des Begehrungsvermögens zu handeln. Das Begehrungsvermögen ist das Vermögen desselben, durch seine Vorstellungen Ursache von der Wirklichkeit der Gegenstände dieser Vorstellungen zu sein. Lust ist die Vorstellung der Übereinstimmung des Gegenstandes oder der Handlung mit den subjektiven Bedingungen des Lebens, d.i. mit dem Vermögen der Kausalität einer Vorstellung in Ansehung der Wirklichkeit ihres Objekts (oder der Bestimmung der Kräfte des Subjekts zur Handlung, es hervorzubringen). Mehr brauche ich nicht zum Behuf der Kritik von Begriffen, die aus der Psychologie entlehnt werden, das übrige leistet die Kritik selbst. Man wird leicht gewahr, daß die Frage, ob die Lust dem Begehrungsvermögen jederzeit zum Grunde gelegt werden müsse, oder ob sie auch unter gewissen Bedingungen nur auf die Bestimmung desselben folge, durch diese Erklärung unentschieden bleibt; denn sie ist aus lauter Merkmalen des reinen Verstandes, d.i. Kategorien zusammengesetzt, die nichts Empirisches enthalten. Eine solche Behutsamkeit ist in der ganzen Philosophie sehr empfehlungswürdig, und wird dennoch oft verabsäumt, nämlich, seinen Urteilen vor der vollständigen Zergliederung des Begriffs, die oft nur sehr spät erreicht wird, durch gewagte Definition nicht vorzugreifen. Man wird auch durch den ganzen \match{Lauf} der Kritik (der theoretischen sowohl als praktischen Vernunft) bemerken, daß sich in demselben mannigfaltige Veranlassung vorfinde, manche Mängel im alten dogmatischen Gange der Philosophie zu ergänzen, und Fehler abzuändern, die nicht eher bemerkt werden, als wenn man von Begriffen einen Gebrauch der Vernunft macht, der aufs Ganze derselben geht. 
	
	\unnumberedsection{Stachel (1)} 
	\subsection*{tg227.2.3} 
	\textbf{Source : }Kritik der praktischen Vernunft/Erster Teil. Elementarlehre der reinen praktischen Vernunft/Zweites Buch. Dialektik der reinen praktischen Vernunft/Zweites Hauptstück. Von der Dialektik der reinen Vernunft in Bestimmung des Begriffs vom höchsten Gut/IX. Von der der praktischen Bestimmung des Menschen weislich angemessenen Proportion seiner Erkenntnisvermögen\\  
	
	\textbf{Paragraphe : }Gesetzt nun, sie wäre hierin unserem Wunsche willfährig gewesen, und hätte uns diejenige Einsichtsfähigkeit, oder Erleuchtung erteilt, die wir gerne besitzen möchten, oder in deren Besitz einige wohl gar wähnen sich wirklich zu befinden, was würde allem Ansehn nach wohl die Folge hievon sein? Wofern nicht zugleich unsere ganze Natur umgeändert wäre, so würden die Neigungen, die doch allemal das erste Wort haben, zuerst ihre Befriedigung, und, mit vernünftiger Überlegung verbunden, ihre größtmögliche und daurende Befriedigung, unter dem Namen der Glückseligkeit, verlangen; das moralische Gesetz würde nachher sprechen, um jene in ihren geziemenden Schranken zu halten, und sogar sie alle insgesamt einem höheren, auf keine Neigung Rücksicht nehmenden, Zwecke zu unterwerfen.  Aber, statt des Streits, den jetzt die moralische Gesinnung mit den Neigungen zu führen hat, in welchem, nach einigen Niederlagen, doch allmählich moralische Stärke der Seele zu erwerben ist, würden Gott und Ewigkeit, mit ihrer furchtbaren Majestät, uns unablässig vor Augen liegen (denn, was wir vollkommen beweisen können, gilt, in Ansehung der Gewißheit, uns so viel, als wovon wir uns durch den Augenschein versichern). Die Übertretung des Gesetzes würde freilich vermieden, das Gebotene getan werden; weil aber die Gesinnung, aus welcher Handlungen geschehen sollen, durch kein Gebot mit eingeflößt werden kann, der \match{Stachel} der Tätigkeit hier aber sogleich bei Hand, und äußerlich ist, die Vernunft also sich nicht allererst empor arbeiten darf, um Kraft zum Widerstande gegen Neigungen durch lebendige Vorstellung der Würde des Gesetzes zu sammeln, so würden die mehresten gesetzmäßigen Handlungen aus Furcht, nur wenige aus Hoffnung und gar keine aus Pflicht geschehen, ein moralischer Wert der Handlungen aber, worauf doch allein der Wert der Person und selbst der der Welt, in den Augen der höchsten Weisheit, ankommt, würde gar nicht existieren. Das Verhalten der Menschen, so lange ihre Natur, wie sie jetzt ist, bliebe, würde also in einen bloßen Mechanismus verwandelt werden, wo, wie im Marionettenspiel, alles gut gestikulieren, aber in den Figuren doch kein Leben anzutreffen sein würde. Nun, da es mit uns ganz anders beschaffen ist, da wir, mit aller Anstrengung unserer Vernunft, nur eine sehr dunkele und zweideutige Aussicht in die Zukunft haben, der Weltregierer uns sein Dasein und seine Herrlichkeit nur mutmaßen, nicht erblicken, oder klar beweisen läßt, dagegen das moralische Gesetz in uns, ohne uns etwas mit Sicherheit zu verheißen, oder zu drohen, von uns uneigennützige Achtung fodert, übrigens aber, wenn diese Achtung tätig und herrschend geworden, allererst alsdenn und nur dadurch, Aussichten ins Reich des Übersinnlichen, aber auch nur mit schwachen Blicken erlaubt: so kann wahrhafte sittliche, dem Gesetze unmittelbar geweihete Gesinnung stattfinden und das vernünftige Geschöpf des Anteils am höchsten Gute würdig  werden, das dem moralischen Werte seiner Person und nicht bloß seinen Handlungen angemessen ist. Also möchte es auch hier wohl damit seine Richtigkeit haben, was uns das Studium der Natur und des Menschen sonst hinreichend lehrt, daß die unerforschliche Weisheit, durch die wir existieren, nicht minder verehrungswürdig ist, in dem, was sie uns versagte, als in dem, was sie uns zu teil werden ließ. 
	
	\unnumberedsection{Stimme (3)} 
	\subsection*{tg209.2.10} 
	\textbf{Source : }Kritik der praktischen Vernunft/Erster Teil. Elementarlehre der reinen praktischen Vernunft/Erstes Buch. Die Analytik der reinen praktischen Vernunft/Erstes Hauptstück. Von den Grundsätzen der reinen praktischen Vernunft/8. Lehrsatz IV\\  
	
	\textbf{Paragraphe : }Das gerade Widerspiel des Prinzips der Sittlichkeit ist: wenn das der eigenen Glückseligkeit zum Bestimmungsgrunde des Willens gemacht wird, wozu, wie ich oben gezeigt habe, alles überhaupt gezählt werden muß, was den Bestimmungsgrund, der zum Gesetze dienen soll, irgend worin anders, als in der gesetzgebenden Form der Maxime setzt. Dieser Widerstreit ist aber nicht bloß logisch, wie der zwischen empirisch- bedingten Regeln, die man doch zu notwendigen Erkenntnisprinzipien erheben wollte, sondern praktisch, und würde, wäre nicht die \match{Stimme} der Vernunft  in Beziehung auf den Willen so deutlich, so unüberschreibar, selbst für den gemeinsten Menschen so vernehmlich, die Sittlichkeit gänzlich zu Grunde richten; so aber kann sie sich nur noch in den kopfverwirrenden Spekulationen der Schulen erhalten, die dreist genug sein, sich gegen jene himmlische Stimme taub zu machen, um eine Theorie, die kein Kopfbrechen kostet, aufrecht zu erhalten. 
	
	\subsection*{tg214.2.14} 
	\textbf{Source : }Kritik der praktischen Vernunft/Erster Teil. Elementarlehre der reinen praktischen Vernunft/Erstes Buch. Die Analytik der reinen praktischen Vernunft\\  
	
	\textbf{Paragraphe : }Es liegt so etwas Besonderes in der grenzenlosen Hochschätzung des reinen, von allem Vorteil entblößten, moralischen Gesetzes, so wie es praktische Vernunft uns zur Befolgung vorstellt, deren \match{Stimme} auch den kühnsten Frevler zittern macht, und ihn nötigt, sich vor seinem Anblicke zu verbergen: daß man sich nicht wundern darf, diesen Einfluß einer bloß intellektuellen Idee aufs Gefühl für spekulative Vernunft unergründlich zu finden, und sich damit begnügen zu müssen, daß man a priori doch noch so viel einsehen kann: ein solches Gefühl sei unzertrennlich mit der Vorstellung des moralischen Gesetzes in jedem endlichen vernünftigen Wesen verbunden. Wäre dieses Gefühl der Achtung pathologisch und also ein auf dem inneren Sinne gegründetes Gefühl der Lust, so würde es vergeblich sein, eine Verbindung derselben mit irgend einer Idee a priori zu entdecken. Nun aber ist ein Gefühl, was bloß aufs Praktische geht, und zwar der Vorstellung eines Gesetzes lediglich seiner Form nach, nicht irgend eines Objekts desselben wegen, anhängt, mithin weder zum Vergnügen, noch zum Schmerze gerechnet werden kann, und dennoch ein Interesse an der Befolgung desselben hervorbringt, welches wir das moralische nennen; wie denn auch die Fähigkeit, ein solches  Interesse am Gesetze zu nehmen (oder die Achtung fürs moralische Gesetz selbst) eigentlich das moralische Gefühl ist. 
	
	\subsection*{tg223.2.5} 
	\textbf{Source : }Kritik der praktischen Vernunft/Erster Teil. Elementarlehre der reinen praktischen Vernunft/Zweites Buch. Dialektik der reinen praktischen Vernunft/Zweites Hauptstück. Von der Dialektik der reinen Vernunft in Bestimmung des Begriffs vom höchsten Gut/V. Das Dasein Gottes, als ein Postulat der reinen praktischen Vernunft\\  
	
	\textbf{Paragraphe : }Aus dieser Deduktion wird es nunmehr begreiflich, warum die griechischen Schulen zur Auflösung ihres Problems von der praktischen Möglichkeit des höchsten Guts niemals gelangen konnten; weil sie nur immer die Regel des Gebrauchs, den der Wille des Menschen von seiner Freiheit macht, zum einzigen und für sich allein zureichenden Grunde derselben machten, ohne, ihrem Bedünken nach, das Dasein Gottes dazu zu bedürfen. Zwar taten sie daran recht, daß sie das Prinzip der Sitten unabhängig von diesem Postulat, für sich selbst, aus dem Verhältnis der Vernunft allein zum Willen, festsetzten, und es mithin zur obersten praktischen Bedingung des höchsten Guts machten; es war aber darum nicht die ganze Bedingung der Möglichkeit desselben. Die Epikureer hatten nun zwar ein ganz falsches Prinzip der Sitten zum obersten angenommen, nämlich das der Glückseligkeit, und eine Maxime der beliebigen Wahl, nach jedes seiner Neigung, für ein Gesetz untergeschoben; aber darin verfuhren sie doch konsequent genug, daß sie ihr höchstes Gut eben so, nämlich der Niedrigkeit ihres Grundsatzes proportionierlich, abwürdigten, und keine größere Glückseligkeit erwarteten, als die sich durch menschliche Klugheit (wozu auch Enthaltsamkeit und Mäßigung der Neigungen gehört) erwerben läßt, die, wie man weiß, kümmerlich genug, und nach Umständen sehr verschiedentlich, ausfallen muß; die Ausnahmen, welche ihre  Maximen unaufhörlich einräumen mußten, und die sie zu Gesetzen untauglich machen, nicht einmal gerechnet. Die Stoiker hatten dagegen ihr oberstes praktisches Prinzip, nämlich die Tugend, als Bedingung des höchsten Guts ganz richtig gewählt, aber, indem sie den Grad derselben, der für das reine Gesetz derselben erforderlich ist, als in diesem Leben völlig erreichbar vorstelleten, nicht allein das moralische Vermögen des Menschen, unter dem Namen eines Weisen, über alle Schranken seiner Natur hoch gespannt, und etwas, das aller Menschenkenntnis widerspricht, angenommen, sondern auch vornehmlich das zweite zum höchsten Gut gehörige Bestandstück, nämlich die Glückseligkeit, gar nicht für einen besonderen Gegenstand des menschlichen Begehrungsvermögens wollen gelten lassen, sondern ihren Weisen, gleich einer Gottheit, im Bewußtsein der Vortrefflichkeit seiner Person, von der Natur (in Absicht auf seine Zufriedenheit) ganz unabhängig gemacht, indem sie ihn zwar Übeln des Lebens aussetzten, aber nicht unterwarfen (zugleich auch als frei vom Bösen darstelleten), und so wirklich das zweite Element des höchsten Guts, eigene Glückseligkeit wegließen, indem sie es bloß im Handeln und der Zufriedenheit mit seinem persönlichen Werte setzten, und also im Bewußtsein der sittlichen Denkungsart mit einschlossen, worin sie aber durch die \match{Stimme} ihrer eigenen Natur hinreichend hätten widerlegt werden können. 
	
\end{document}